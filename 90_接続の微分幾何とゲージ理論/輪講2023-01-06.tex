\documentclass[report]{jlreq}
\usepackage{../global}
\usepackage{./local}
\begin{document}

\tableofcontents
\markboth{\contentsname}{}

% ============================================================
%
% ============================================================
\newpage
\setcounter{chapter}{1}
\chapter{接続}

% ------------------------------------------------------------
%
% ------------------------------------------------------------
\setcounter{section}{9}
\section{テンソル場の共変微分}

\subsection{共変微分の座標表示}

\begin{example}
    \TODO{$\xi \in \Gamma(TM)$}
\end{example}

\begin{example}
    \TODO{$\xi \in \Gamma(T^*M)$}
\end{example}

\begin{example}
    \TODO{$\xi \in \Gamma(TM \otimes T^*M \otimes T^*M)$}
\end{example}

\begin{example}
    \TODO{$\xi \in \Gamma((TM)^{\otimes r} \otimes (T^*M)^{\otimes s})$}
\end{example}

\begin{example}
    $E, TM$の両方に接続が与えられているとき
    \TODO{$\xi \in \Gamma(E \otimes (TM)^{\otimes r} \otimes (T^*M)^{\otimes s})$}
\end{example}

\subsection{共変外微分の座標表示}

\begin{example}[$E$に値をもつ$p$形式]
    \TODO{$\xi \in A^p(E)$}
\end{example}

\begin{example}[曲率]
    \TODO{}
\end{example}

\begin{example}[$E$がテンソル場の場合]
    \TODO{}
\end{example}

% ------------------------------------------------------------
%
% ------------------------------------------------------------
\section{アフィン接続の測地線}

\begin{definition}[正規座標系]
    \TODO{}
\end{definition}


\newpage
\phantomsection
\addcontentsline{toc}{chapter}{参考文献}
\renewcommand{\bibname}{参考文献}
\markboth{\bibname}{}
\begin{thebibliography}{9}
    \bibitem{leesmo}
        John. M. Lee.
        \textit{Introduction to Smooth Manifolds}.
        Springer,
        2012
    \bibitem{leerie}
        John. M. Lee.
        \textit{Introduction to Riemannian Manifolds}.
        Springer,
        2018
    \bibitem{kob} 小林 昭七. "接続の微分幾何とゲージ理論". 裳華房, 2004
    \bibitem{tu}
        Loring W. Tu.
        \textit{Differential Geometry}.
        Springer,
        2017
    \bibitem{rotman} Joseph J. Rotman \textit{An Introduction to Homological Algebra}. Springer, 2008
    \bibitem{kol}
        Ivan Kolář, Jan Slovák, Peter W. Michor.
        \textit{Natural Operations in Differential Geometry}.
        Springer Berlin, Heidelberg,
        1993
\end{thebibliography}

\end{document}