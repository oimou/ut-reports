\documentclass[report]{jlreq}
\usepackage{../../global}
\usepackage{./local}
\subfiletrue
\begin{document}

この部では、偏微分方程式の基礎概念と1階偏微分方程式を扱う。

% ============================================================
%
% ============================================================
\chapter{偏微分方程式の基礎}

偏微分方程式の基礎概念を述べる。

% ------------------------------------------------------------
%
% ------------------------------------------------------------
\section{解の存在と一意性}



% ------------------------------------------------------------
%
% ------------------------------------------------------------
\section{Sobolev Spaces and Weak Solutions}

\begin{definition}[Sobolev Spaces]
    \TODO{}
\end{definition}

% ============================================================
%
% ============================================================
\chapter{1階偏微分方程式}

1階偏微分方程式の一般論を述べた後、簡単な例に触れる。

% ------------------------------------------------------------
%
% ------------------------------------------------------------
\section{1階偏微分方程式}

\begin{definition}[1階偏微分方程式]
    $U \opensubset \R^n$とし、$F \colon \R^n \times \R \times \wb{U} \to \R$を写像とする。
    未知関数$u \colon \wb{U} \to \R$に対し、式
    \begin{equation}
        F(Du, u, x) = 0 \quad (x \in U)
    \end{equation}
    を\term{1階偏微分方程式}[first-order partial differential equation]{1階偏微分方程式}[1かいへんびぶんほうていしき]という。
\end{definition}

\end{document}