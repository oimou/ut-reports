\documentclass[report]{jlreq}
\usepackage{../../global}
\usepackage{./local}
\subfiletrue
\begin{document}

fuga

% ============================================================
%
% ============================================================
\chapter{トーラス上の Fourier 変換}

% ------------------------------------------------------------
%
% ------------------------------------------------------------
\section{Fourier Coefficients and Summation Methods}

\begin{definition}[Fourier 係数]
    $f \in L^1(\bbT)$とする。
    各$n \ge 0$に対し
    \begin{equation}
        \what{f}(n) \coloneqq \frac{1}{2\pi}
            \int_0^{2\pi} f(t) e^{-int} dt
    \end{equation}
    を$f$の\term{第$n$ Fourier 係数}{Fourier 係数}[Fourierけいすう]という。
\end{definition}

\begin{definition}[Fourier 部分和]
    $f \in L^1(\bbT)$とする。
    各$N \ge 0$に対し
    \begin{equation}
        S_N(f)(t) \coloneqq \sum_{n=-N}^N \what{f}(n) e^{int}
    \end{equation}
    を$f$の\term{第$N$ Fourier 部分和}{Fourier 部分和}[Fourierぶぶんわ]という。
\end{definition}

次の定理は、
Fourier 変換/逆変換に対する反転公式の、
Fourier 係数/級数に対する類似物である。

\begin{theorem}
    $f \in L^1(\bbT)$とする。
    このとき、$\sum_{n = -\infty}^{\infty} |\what{f}(n)| < \infty$ならば、
    \begin{equation}
        f(t) = \sum_{n = -\infty}^{\infty} \what{f}(n) e^{int}
            \quad \text{a.e. } t \in \bbT
    \end{equation}
    が成り立つ。
\end{theorem}

\begin{proof}
    省略
\end{proof}

% ------------------------------------------------------------
%
% ------------------------------------------------------------
\section{%
    \texorpdfstring{%
        Fourier Transform of $L^2$ functions%
    }{%
        Fourier Transform of L^2 functions%
    }%
}

% ------------------------------------------------------------
%
% ------------------------------------------------------------
\section{Trigonometric Series}



% ============================================================
%
% ============================================================
\chapter{%
    \texorpdfstring{%
        $\R$と$\R^d$上の Fourier 変換%
    }{%
        R と R^d 上の Fourier 変換%
    }%
}

% ------------------------------------------------------------
%
% ------------------------------------------------------------
\section{Fourier 変換の基本性質}

\begin{definition}[Fourier 変換]
    $f \in L^1(\R)$とする。$\R$上の関数$\what{f}$を
    \begin{equation}
        \what{f}(\xi) \coloneqq \int_\R f(x) e^{-i\xi x} dx \quad (\xi \in \R)
    \end{equation}
    で定義し、これを$f$の\term{Fourier 変換}[Fourier transform]{Fourier 変換}[Fourierへんかん]という。
    $\what{f}$を$f^\wedge$とも書く。
    $\what{\R} \coloneqq \R$とおき、
    $\R, \what{\R}$の座標をそれぞれ$x, \xi$で表すことが多い。
\end{definition}

\begin{proposition}
    $f \in L^1(\R)$に対し、$\what{f}$は$\what{\R}$上一様連続である。
\end{proposition}

\begin{proof}
    \begin{alignat}{1}
        |\what{f}(\xi + \eta) - \what{f}(\xi)|
            &= \left| \int_\R f(x) (e^{-i (\xi + \eta) x} - e^{-i \xi x}) dx \right| \\
            &\le \int_\R |f(x)| |e^{-i \eta x} - 1| dx \label[equation]{eq:tri-ineq} \\
            &\le 2 \|f\|_1 \\
            &< \infty
    \end{alignat}
    より、\cref{eq:tri-ineq}は$\eta \in \wR$に関し可積分であり、
    優収束定理より$\eta \to 0$で$0$に収束する。
\end{proof}

\begin{proposition}
    次の図式は可換である:
    \begin{equation}
        \begin{tikzcd}[column sep=huge, row sep=large]
            L^1(\R) \times L^1(\R)
                \ar{r}{\calF \times \calF}
                \ar{d}[swap]{\text{たたみ込み}}
                & C_0(\R) \times C_0(\R) \ar{d}{各点での積} \\
            L^1(\R) \ar{r}[swap]{\calF}
                & C_0(\R)
        \end{tikzcd}
    \end{equation}
    ただし、$C_0(\R)$は$\R$上の連続関数であって$|x| \to \infty$で$0$に収束するもの全体の集合である。
\end{proposition}

\begin{proof}
    省略
\end{proof}

% ------------------------------------------------------------
%
% ------------------------------------------------------------
\section{The Dirichlet and Fej\'{e}r Kernels}

$\bbT$の場合と同様に Dirichlet 核と Fej\'{e}r 核を定義する。

\begin{definition}[Dirichlet 核]
    $\R$上の関数の族
    \begin{equation}
        D_\lambda(x) \coloneqq \frac{1}{2\pi} \int_{-\lambda}^\lambda e^{i\xi x} d\xi
            \quad (\lambda > 0)
    \end{equation}
    を\term{Dirichlet 核}[Dirichlet kernel]{Dirichlet 核}[Dirichletかく]という。
\end{definition}

\begin{proposition}
    \begin{equation}
        D_\lambda(x) = \frac{\sin \lambda x}{\pi x} \quad (x \neq 0)
    \end{equation}
\end{proposition}

\begin{proof}
    省略
\end{proof}

\begin{definition}[Fej\'{e}r 核]
    $\R$上の関数の族
    \begin{equation}
        K_\lambda(x) \coloneqq \frac{1}{2\pi}
            \int_{-\lambda}^\lambda \left(1 - \frac{|\xi|}{\lambda}\right) e^{i\xi x} d\xi
            \quad (\lambda > 0)
    \end{equation}
    を\term{Fej\'{e}r 核}[Fej\'{e}r kernel]{Fej\'{e}r 核}[Fej\'{e}rかく]という。
\end{definition}

\begin{proposition}
    \begin{equation}
        K_\lambda(x) = \frac{\lambda}{2\pi} \left(\frac{\sin(\lambda x / 2)}{\lambda x / 2}\right)^2
            \quad (x \neq 0)
    \end{equation}
\end{proposition}

\begin{proof}
    省略
\end{proof}

\begin{definition}[総和核]
    次をみたす$L^1(\R)$の元の族$(k_\lambda)_{\lambda > 0}$を
    $\R$上の\term{総和核}[summability kernel]{総和核}[そうわかく]という:
    \begin{enumerate}
        \item $\forall \lambda > 0$に対し$\int_\R k_\lambda(x) dx = 1$
        \item $\exists C > 0$が存在して、$\forall \lambda > 0$に対し$\|k_\lambda\|_1 \ge C$
        \item $\forall \delta > 0$に対し、$\lambda \to \infty$で$\int_{|x| \ge \delta} |k_\lambda(x)| dx \to 0$
    \end{enumerate}
\end{definition}

\begin{example}
    ~
    \begin{itemize}
        \item Dirichlet 核は$\R$上の総和核ではない。
        \item Fej\'{e}r 核は$\R$上の総和核である。
    \end{itemize}
\end{example}

総和核のたたみ込みにより、関数の近似が得られる。

\begin{theorem}
    $\{ k_\lambda(x) \}_{\lambda > 0}$を$\R$上の総和核とする。
    このとき、
    \begin{enumerate}
        \item $\R$上の関数$f$が有界かつ一様連続ならば、
            $\lambda \to \infty$で$k_\lambda * f$は$f$に$\R$上一様収束する。
        \item $1 \le p < \infty$と$f \in L^p(\R)$に対し、
            $\lambda \to \infty$で$k_\lambda * f$は$f$に$L^p$収束する。
    \end{enumerate}
\end{theorem}

\begin{proof}
    省略
\end{proof}

\begin{theorem}[Riemann-Lebesgue の補題]
    任意の$f \in L^1(\R)$に対し、
    $|\xi| \to \infty$で$\what{f}(\xi) \to 0$
\end{theorem}

\begin{proof}
    省略
\end{proof}

Fej\'{e}r 核を用いて次の定理が示せる。
Fej\'{e}r 核を用いた証明は$\bbT$の場合と似ている。

\begin{theorem}[反転公式]
    $f \in L^1(\R)$とし、$\what{f} \in L^1(\R)$と仮定する。
    このとき、
    \begin{equation}
        f(x) = \frac{1}{2\pi} \int_{\wR} \what{f}(\xi) e^{i\xi x} d\xi
            \quad \text{a.e. } x \in \R
    \end{equation}
    が成り立つ。
\end{theorem}

\begin{proof}
    省略
\end{proof}

% ------------------------------------------------------------
%
% ------------------------------------------------------------
\section{%
    \texorpdfstring{%
        Relations to Fourier Transform on $\bbT$%
    }{%
        Relations to Fourier Transform on T%
    }%
}

Fourier 係数と Fourier 変換の関連をみる。

\begin{theorem}[各点収束性]
    \TODO{}
\end{theorem}

\begin{proof}
    省略
\end{proof}

% ------------------------------------------------------------
%
% ------------------------------------------------------------
\section{Periodization}

\begin{definition}[周期化]
    $f \in L^1(\R)$とする。
    $\bbT$上の関数
    \begin{equation}
        F(t) \coloneqq \sum_{k \in \Z} f(t + 2\pi k)
    \end{equation}
    を$f$の\term{周期化}[periodization]{周期化}[しゅうきか]という。
\end{definition}

\begin{theorem}[Poisson の和公式]
    \TODO{}
\end{theorem}

\begin{proof}
    省略
\end{proof}

\begin{example}[Fej\'{e}r 核]
    \TODO{}
\end{example}

\begin{example}[Dirichlet 核]
    \TODO{}
\end{example}

% ------------------------------------------------------------
%
% ------------------------------------------------------------
\section{Schwartz Functions}

$\R^d$での話題に移る。

\begin{definition}[Fourier 変換]
    $f \in L^d(\R)$とする。$\R$上の関数$\what{f}$を
    \begin{equation}
        \what{f}(\xi) \coloneqq \int_{\R^d} f(x) e^{-i\xi \cdot x} dx \quad (\xi \in \R^d)
    \end{equation}
    で定義し、これを$f$の\term{Fourier 変換}[Fourier transform]{Fourier 変換}[Fourierへんかん]という。
\end{definition}

\begin{definition}[急減少]
    $\R^d$上の関数$f$が
    \term{急減少}{急減少}[きゅうげんしょう]であるとは、
    任意の$k \in \N$に対し
    $|x| \to \infty$で$f(x) = o(|x|^{-k})$となることをいう。
\end{definition}

\begin{definition}[Schwartz 急減少関数]
    $f \in \smooth(\R^d)$とする。
    $f$が\term{Schwartz 急減少関数}[Schwartz function]{Schwartz 急減少関数}[Schwartzきゅうげんしょうかんすう]
    であるとは、
    任意の多重指数$\alpha$に対し$\partial^\alpha f$が急減少であることをいう。
    $\calS = \calS(\R^d)$で$\R^d$上の Schwartz 急減少関数全体の集合を表す。
\end{definition}

\begin{proposition}
    次の図式は可換である:
    \begin{equation}
        \begin{tikzcd}[row sep=large]
            \calS \ar{r}{\calF} \ar{d}[swap]{\partial_x^\alpha} 
                & \calS \ar{d}{i^{|\alpha|}\xi^\alpha \times} \\
            \calS \ar{r}[swap]{\calF}
                & \calS
        \end{tikzcd}
        \quad
        \begin{tikzcd}[row sep=large]
            \calS \ar{r}{\calF} \ar{d}[swap]{x^\alpha \times} 
                & \calS \ar{d}{i^{|\alpha|}\partial_\xi^\alpha} \\
            \calS \ar{r}[swap]{\calF}
                & \calS
        \end{tikzcd}
    \end{equation}
\end{proposition}

\begin{proof}
    省略
\end{proof}

\begin{theorem}[反転公式]
    $f \in L^d(\R)$とし、$\what{f} \in L^d(\R)$と仮定する。
    このとき、
    \begin{equation}
        f(x) = \frac{1}{(2\pi)^d} \int_{\wR} \what{f}(\xi) e^{i\xi x} d\xi
            \quad \text{a.e. } x \in \R^d
    \end{equation}
    が成り立つ。
\end{theorem}

\begin{proof}
    省略
\end{proof}

\begin{theorem}[Planchrel の定理]
    \TODO{}
\end{theorem}

\begin{proof}
    省略
\end{proof}

% ------------------------------------------------------------
%
% ------------------------------------------------------------
\section{Some Partial Differential Equations}

% ============================================================
%
% ============================================================
\chapter{Distributions}

% ------------------------------------------------------------
%
% ------------------------------------------------------------
\section{Distributions}

Distributions の定義と具体例を与える。

\begin{definition}[テスト関数]
    $\calD \coloneqq \calD(\R^d) \coloneqq C_\smooth(\R^d)$とおく。
    $\calD$の元を\term{テスト関数}[test function]{テスト関数}[てすとかんすう]という。
\end{definition}

\begin{definition}[$\calD$の位相]
    \TODO{}
\end{definition}

\begin{definition}[超関数]
    $\calD$の位相的双対空間を$\calD'$と書き、
    $\calD'$の元を\term{超関数}[distribution]{超関数}[ちょうかんすう]という。
    $T \in \calD', \varphi \in \calD$に対し$T(\varphi)$を$\langle T, \varphi \rangle$とも書く。
\end{definition}

\begin{example}[局所可積分関数]
    \TODO{}
\end{example}

\begin{proposition}[変分法の基本補題]
    \TODO{}
\end{proposition}

\begin{proof}
    省略
\end{proof}

\begin{example}[Dirac のデルタ]
    \TODO{}
\end{example}

\begin{example}[波動方程式]
    \TODO{}
\end{example}

\begin{example}[Poisson 方程式]
    \TODO{}
\end{example}

% ------------------------------------------------------------
%
% ------------------------------------------------------------
\section{Differentiation of Distributions}

\begin{definition}[超関数の微分]
    $T \in \calD'$と多重指数$\alpha$に対し、
    $\partial^\alpha T \in \calD'$を
    \begin{equation}
        \langle \partial^\alpha T, \varphi \rangle
            \coloneqq (-1)^{|\alpha|} \langle T, \partial^\alpha \varphi \rangle
            \quad (\varphi \in \calD)
    \end{equation}
    と定義する(ことができる)。
\end{definition}

\begin{example}[Heaviside 関数]
    \TODO{}
\end{example}

\begin{theorem}[2次元 Laplace 方程式の基本解]
    $\R^2$上の局所可積分関数
    \begin{equation}
        E(x) \coloneqq \begin{cases}
            \frac{1}{2\pi} \log |x| &\quad (x \neq 0) \\
            0 &\quad (x = 0)
        \end{cases}
    \end{equation}
    は Laplacian $\Delta$の基本解である。
\end{theorem}

\begin{proof}
    $\Delta T_E = \delta$すなわち
    $\langle \Delta T_E, \varphi \rangle = \varphi(0)\; (\varphi \in \calD)$を示せばよい。
    積分の形で書けば
    \begin{equation}
        \int_{\R^2} E(x, y) \Delta\varphi(x, y)\, dx dy = \varphi(0, 0)
    \end{equation}
    である。ここで$\eps > 0$とし、
    \begin{equation}
        \Omega \coloneqq \{ (x, y) \colon x^2 + y^2 \ge \eps^2 \}
    \end{equation}
    とおく。
    $\Omega$上の1次微分形式$\omega$を
    \begin{equation}
        \omega \coloneqq \varphi_x E\, dy
    \end{equation}
    で定める。
    $\omega$は compactly supported だから、Stokes の定理より
    \begin{equation}
        \int_\Omega d\omega = \int_{\partial\Omega} \omega
    \end{equation}
    が成り立つ。
    \TODO{}
\end{proof}

% ------------------------------------------------------------
%
% ------------------------------------------------------------
\section{Convolutions}

% ------------------------------------------------------------
%
% ------------------------------------------------------------
\section{Distributions with Compact Support}

% ------------------------------------------------------------
%
% ------------------------------------------------------------
\section{Tempered Distributions}

Tempered distributions の Fourier 変換の基本性質を確かめる。




\end{document}