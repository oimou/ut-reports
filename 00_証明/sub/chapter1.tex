\documentclass[report]{jlreq}
\usepackage{../../global}
\usepackage{./local}
\subfiletrue
%\makeindex
\begin{document}

% ============================================================
%
% ============================================================
\chapter{証明の書き方}

% ------------------------------------------------------------
%
% ------------------------------------------------------------
\section{論理のフレーズ}

\begin{description}
    \item[〜を示す]
        今から示したいことを述べるときに使う。
    \item[〜と定める]
        何かを定義するときに使う。
    \item[〜とおく]
        何かを定義するときに使う。
    \item[任意の]
        任意のものを表すときに使う。
    \item[〜とする]
        \begin{enumerate}
            \item 任意のものを表すときに使う。「$n \in \Z$とする」
            \item 何かを仮定するときに使う。「$f$は連続であるとする」
        \end{enumerate}
    \item[ある〜]
        存在を表すときに使う。
    \item[〜をひとつ選ぶ]
        存在するもののうちひとつを選ぶときに使う。
    \item[固定する]
    \item[〜だったとすると]
        背理法の仮定を述べるときに使う。
    \item[〜とあわせて]
        既存の事実と組み合わせて新しい事実を導くときに使う。
    \item[〜である]
        事実を述べるときに使う。
    \item[〜となる]
        直前に述べたことから従うことを述べるときに使う。
    \item[〜が従う]
        直前に述べたことから従うことを述べるときに使う。
    \item[〜が成り立つ]
        直前に述べたことから従うことを述べるときに使う。
    \item[一般性を失うことなく]
    \item[〜としてよい]
\end{description}

% ------------------------------------------------------------
%
% ------------------------------------------------------------
\section{つなぎ言葉}

つなぎ言葉は、証明の流れを説明するために使う。

\subsection{議論の段階}

\begin{description}
    \item[まず]
        段階的な議論の最初の段階を始めるときに使う。
    \item[次に]
        段階的な議論の次の段階を始めるときに使う。
    \item[最後に]
        段階的な議論の最後の段階を始めるときに使う。

    \item[対偶を示すために]
        対偶を示す議論を始めるときに結論の否定を提示しておくために使う。
        「対偶を示すために$x \neq 0$と仮定する」
    \item[背理法のために]
        背理法の議論を始めるときに結論の否定を提示しておくために使う。
        「背理法のために$x \neq 0$と仮定する」
\end{description}

\subsection{話の接続}

\begin{description}
    \item[さらに]
    \item[また]

    \item[一方]
        \begin{enumerate}
            \item 直前の話と並行的なことを述べるときに使う。
            \item 直前の話と対照的なことを述べるときに使う。
        \end{enumerate}
    \item[ところが]
        直前の話と対照的なことを述べるときに使う。

    \item[すると]
        直前の行為から従うことを述べるときに使う。
        「標準射影を$\pi$とおく。すると$\pi$は図式を可換にする」
    \item[このとき]
        直前の話に関する付加情報を述べるときに使う。
        「条件を満たす整数$n$が存在する。このとき$n$は$0$でない」
    \item[よって]
        直前の事実から従うことを述べるときに使う。
        「$4$は合成数である。よって$4$は素数でない」
    \item[したがって]
        直前の事実から従うことを述べるときに使う。
        「$4$は合成数である。したがって$4$は素数でない」
\end{description}

\subsection{情報の付与}

\begin{description}
    \item[ただし]
    \item[実際]
\end{description}

\subsection{話題の展開}

\begin{description}
    \item[さて]
        話題を大きく変えるときに使う。
    \item[ここで]
        話題を少し変えるときに使う。
    \item[そこで]
        直前の文をもとに議論を進めるときに使う。
\end{description}

\subsection{文中の表現}

\begin{description}
    \item[ゆえに]
    \item[従う]
    \item[得る]
    \item[ならば]
    \item[ひいては]
\end{description}

% ------------------------------------------------------------
%
% ------------------------------------------------------------
\section{フレーズ}

\begin{description}
    \item[〜より]
    \item[いま]
    \item[とくに]
    \item[より強く]
    \item[すなわち]
    \item[それぞれ]
    \item[resp.]
    \item[アナロジー]
    \item[不合理]
    \item[まわり]
    \item[注意する]
        どのように注意するのか曖昧になりやすいので、使用には注意を要する。
\end{description}

\subsection{証明の締めくくり}

\begin{description}
    \item[示せた]
        議論が完了したときに使う。
        「示された」「いえた」という場合もある。
    \item[完成した]
        長い議論が完了したときに使う。
        「証明が完成した」「帰納法が完成した」
    \item[これが示したいことであった]
\end{description}




\end{document}