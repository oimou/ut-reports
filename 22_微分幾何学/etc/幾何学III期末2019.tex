\documentclass[report, notitlepage]{jlreq}
\usepackage{docmute}
\usepackage{../../global}
\usepackage{../sub/local}
\def\assetspath{./}
%\makeindex
%\makeglossaries

\begin{document}

\begin{problem}[問1]
    $M$を境界のない多様体とし、
    $\omega \in \Omega^1(M), \; \alpha \in \Omega^2(M)$とする。
    $\omega$は$0$をとらない ($M$上至るところ$0$でない) とし、
    また$\omega \wedge \alpha = 0$とする。
    \begin{enumerate}
        \item $M = \R^n$とする。
            このとき、$\beta \in \Omega^1(\R^n)$が存在して
            $\alpha = \omega \wedge \beta$が成り立つことを示せ。
        \item $M$が一般の場合に、
            $\beta \in \Omega^1(\R^n)$が存在して
            $\alpha = \omega \wedge \beta$が成り立つことを示せ。
    \end{enumerate}
\end{problem}

\begin{answer}
    \uline{(1), (2)} \quad
    $p \in M$とし、
    $p$のまわりのチャート$U \opensubset M$とその上の
    局所座標$x^1, \dots, x^n$に関し座標表示を
    $\omega = \sum_i \omega_i dx^i, \;
        \alpha = \sum_{i < j} \alpha_{ij} dx^i \wedge dx^j$
    とおく。
    いま$\omega \neq 0$だからある添字$i_0$が存在して
    $\omega_{i_0} \neq 0$である。
    そこで必要ならば$U$を小さくとりなおして
    $\omega_{i_0}$は$U$上つねに非零であるとしてよい。
    いま$0 = \omega \wedge \alpha$より
    \begin{alignat}{1}
        0 = \omega \wedge \alpha
            &= \left(
                \sum_{i} \omega_i dx^i
            \right)
            \wedge \left(
                \sum_{i < j} \alpha_{ij} dx^i \wedge dx^j
            \right) \\
            &= \sum_{i_1 < i_2 < i_3}
                \underbrace{\left(
                    \omega_{i_1} \alpha_{i_2 i_3}
                    - \omega_{i_2} \alpha_{i_1 i_3}
                    + \omega_{i_3} \alpha_{i_1 i_2}
                \right)}_{(*)}
                dx^{i_1} \wedge dx^{i_2} \wedge dx^{i_3}
    \end{alignat}
    が成り立ち、$(*)$の部分は$U$上つねに0である。
    ここで$U$上の1形式$\beta$を
    $\beta = \sum_{i} \beta_i dx^i$とおくと
    \begin{equation}
        \omega \wedge \beta
            = \det \begin{bmatrix}
                \omega_i & \beta_i \\
                \omega_j & \beta_j
            \end{bmatrix}
            dx^i \wedge dx^j
    \end{equation}
    だから、
    $U$上で$\omega \wedge \beta = \alpha$が成り立つ条件は
    \begin{equation}
        \det \begin{bmatrix}
            \omega_i & \beta_i \\
            \omega_j & \beta_j
        \end{bmatrix}
        = \alpha_{ij}
        \quad
        (i < j)
        \label[equation]{eq:1-1}
    \end{equation}
    である。
    考察として、添字に$i_0$を含まない$\alpha_{ij}$は
    関係式(*)により添字に$i_0$を含む$\alpha_{ij}$により決定される。
    そこでまず$i < j$のいずれかが$i_0$である場合に限って
    \cref{eq:1-1}を$\beta_1, \dots, \beta_n$に関し解くことを考える。
    これは実際解けて、まず$\beta_{i_0} = 0$とおき、
    各$i < i_0$に対し$\beta_i = - \alpha_{i i_0} / \omega_{i_0}$とおき、
    各$j > i_0$に対し$\beta_j = \alpha_{i_0 j} / \omega_{i_0}$とおけばよい。
    こうして得られた$\beta$は
    添字に$i_0$を含まない$\alpha_{ij}$についても
    \cref{eq:1-1}をみたす。
    実際、$i_0 < i < j$なら(*)より
    \begin{alignat}{1}
        \alpha_{ij}
            &= \frac{1}{\omega_{i_0}}
                (\omega_i \alpha_{i_0 j} - \omega_j \alpha_{i_0 i}) \\
            &= \frac{1}{\omega_{i_0}}
                \left(
                    \omega_i (\omega_{i_0} \beta_j - \omega_j \beta_{i_0})
                    - \omega_j (\omega_{i_0} \beta_i - \omega_i \beta_{i_0})
                \right) \\
            &= \omega_i \beta_j - \omega_j \beta_i \\
            &= \det \begin{bmatrix}
                \omega_i & \beta_i \\
                \omega_j & \beta_j
            \end{bmatrix}
    \end{alignat}
    である。
    $i < i_0 < j$や$i < j < i_0$の場合も同様に成り立つ。
    したがって$\beta$は
    $U$上で$\omega \wedge \beta = \alpha$をみたす。

    各$p \in M$に対し
    以上のように$U_p = U$を選んで$U_p$上で$\beta_p = \beta$を構成すると、
    いま$M$は第2可算だから開被覆$\{ U_p \}_{p \in M}$に
    従属する1の分割$\{ \rho_p \}_{p \in M}$が存在する。
    そこで$M$上の1形式$\beta$を
    $\beta = \sum_p \rho_p \beta_p$で定義すれば
    \begin{equation}
        \omega \wedge \beta
            = \omega \wedge \sum_p \rho_p \beta_p
            = \sum_p \rho_p \omega \wedge \beta_p
            = \sum_p \rho_p \alpha|_{U_p}
            = \alpha
    \end{equation}
    を得る。
\end{answer}

\begin{problem}[問2]
    $(x, y)$を$\R^2$の標準的な座標とする。
    $M = \R^2 \setminus \{ (0, 0) \}$とし、
    $\R^2$の向きから定まる向きを入れる。
    また、$M$上の Riemann 計量$g$を
    \begin{equation}
        g_{(x, y)} = \frac{dx \otimes dx + dy \otimes dy}{x^2 + y^2}
    \end{equation}
    により定め、
    $g$により定まる$M$の体積形式を$\mu$とする。
    最後に、
    $\mathrm{CO}_2 = \{
        A \in M_2(\R)
        \mid
        \exists r > 0, \; B \in \mathrm{O}_2, \; A = rB
    \}$と書く。
    \begin{enumerate}
        \item $g$により定まる$M$の体積形式を求めよ。
        \item $A \in \mathrm{CO}_2$とし、
            $\varphi \colon M \to M$を
            $\varphi(x, y) = A \begin{bmatrix}
                x \\
                y
            \end{bmatrix}$により定める (値は列ベクトルだが行ベクトルとみなす)。
            $\varphi^* g = g$が成り立つことを示せ。
            また、$\varphi^* \mu = \mu$が成り立つことを示せ。
        \item $0 < r \le R$とし、
            $D = \{ (x, y) \in M \mid r^2 \le x^2 + y^2 \le R^2 \}$とおく。
            $\int_D \mu$を求めよ。
    \end{enumerate}
\end{problem}

\begin{answer}
    \uline{(1)} \quad
    $\mu = \sqrt{|\det g|} dx \wedge dy = \frac{1}{x^2 + y^2} dx \wedge dy$.

    \uline{(2)} \quad
    $A = (a_{ij})_{i, j} = rB, \; 
        r > 0, \;
        B = (b_{ij})_{i, j} \in \mathrm{O}_2$
    とおく。
    $\varphi$の定義より形式的に
    $\varphi_*
        \left(
            \deldel{x}, \deldel{y}
        \right)
        =
        \left(
            \deldel{x}, \deldel{y}
        \right)
        A$
    が成り立つから
    \begin{alignat}{1}
        (\varphi^* g)_{(x, y)} \left( \deldel{x}, \deldel{x} \right)
            &= g_{\varphi(x, y)}
                \left( \varphi_* \deldel{x}, \varphi_* \deldel{x} \right) \\
            &= g_{\varphi(x, y)}
                \left(
                    a_{11} \deldel{x} + a_{21} \deldel{y}, \;
                    a_{11} \deldel{x} + a_{21} \deldel{y}
                \right) \\
            &= a_{11}^2 g_{\varphi(x, y)}
                \left( \deldel{x}, \deldel{x} \right)
                + a_{21}^2 g_{\varphi(x, y)}
                \left( \deldel{y}, \deldel{y} \right) \\
            &= \frac{a_{11}^2 + a_{21}^2}{r^2 (x^2 + y^2)} \\
            &= \frac{1}{x^2 + y^2}
    \end{alignat}
    同様にして
    \begin{equation}
        (\varphi^* g)_{(x, y)} \left( \deldel{y}, \deldel{y} \right)
            = \frac{1}{x^2 + y^2},
        \quad
        (\varphi^* g)_{(x, y)} \left( \deldel{x}, \deldel{y} \right)
            = 0,
        \quad
        (\varphi^* g)_{(x, y)} \left( \deldel{y}, \deldel{x} \right)
            = 0
    \end{equation}
    となり$\varphi^* g = g$が成り立つ。

    $\varphi^* \mu = \mu$を示す。
    $\varphi = (\varphi^1, \varphi^2)$とおけば
    \begin{alignat}{1}
        \varphi^* (dx \wedge dy)
            &= d(\varphi^* x) \wedge d(\varphi^* y) \\
            &= d\varphi^1 \wedge d\varphi^2 \\
            &= (a_{11} dx + a_{21} dy) \wedge (a_{12} dx + a_{22} dy) \\
            &= (a_{11} a_{22} - a_{12} a_{21}) dx \wedge dy \\
            &= r^2 \, dx \wedge dy
    \end{alignat}
    だから
    \begin{equation}
        (\varphi^* \mu)_{(x, y)}
            = \frac{1}{r^2 (x^2 + y^2)} \varphi^* (dx \wedge dy)
            = \frac{1}{x^2 + y^2}
            = \mu_{(x, y)}
    \end{equation}
    を得る。

    \uline{(3)} \quad
    $D$には$\R^2$の標準的な向きから定まる向きが入っているものと考える。
    \begin{alignat}{1}
        \int_D \mu
            &= \int_D \frac{1}{x^2 + y^2} dx \wedge dy \\
            &= \int_D \frac{1}{x^2 + y^2} dx dy \\
            &= \int_r^R \int_0^{2\pi} \frac{1}{\rho} d\rho d\theta \\
            &= 2 \pi (\log R - \log r)
    \end{alignat}
    である。
\end{answer}

\begin{problem}[問3]
\end{problem}

\begin{answer}
    \uline{(1)} \quad
    直接計算により
    \begin{alignat}{1}
        d\left(
            \frac{x}{(x^2 + y^2 + z^2)^{3/2}} dy \wedge dz
        \right)
            &= \frac{x^2 + y^2 + z^2 - 3x^2}{(x^2 + y^2 + z^2)^{5/2}}
                dx \wedge dy \wedge dz \\
        d\left(
            \frac{y}{(x^2 + y^2 + z^2)^{3/2}} dz \wedge dx
        \right)
            &= \frac{x^2 + y^2 + z^2 - 3y^2}{(x^2 + y^2 + z^2)^{5/2}}
                dx \wedge dy \wedge dz \\
        d\left(
            \frac{z}{(x^2 + y^2 + z^2)^{3/2}} dx \wedge dy
        \right)
            &= \frac{x^2 + y^2 + z^2 - 3z^2}{(x^2 + y^2 + z^2)^{5/2}}
                dx \wedge dy \wedge dz
    \end{alignat}
    だから$d\omega = 0$、したがって$\omega$は閉形式である。

    \uline{(2)} \quad
    単位閉球$D^3 \subset \R^3$に$\R^3$から定まる自然な向きを入れ、
    $S^2 = \del D^3$には境界としての向きを入れる。
    包含写像$S^2 \to M$を$\iota$とおく。
    $d\omega = 0$が$M$上で成り立つから
    原点での値を$0$として$d\omega$を$\R^3$上、
    とくに$D^3$上に拡張できる。
    Stokes の定理より$\int_{S^2} \iota^* \omega = \int_{D^3} d\omega = 0$
    が成り立つ。

    一方$\int_{S^2} \iota^* \omega$を直接計算すると
    値が$0$にならないことを示す。
    $S^2$上で$\iota^* (x^2 + y^2 + z^2) = 1$だから
    \begin{equation}
        \iota^* \omega
            = x \, dy \wedge dz + y \, dz \wedge dx + z \, dx \wedge dy
    \end{equation}
    である。ただし右辺の$x, y, z$は
    $\R^3$の座標$x, y, z$を$\iota$で引き戻したものを
    記号の濫用で同じ文字を使っている。
    そこで$\R^3$上の2-形式$\mu$を
    \begin{equation}
        \mu = x \, dy \wedge dz + y \, dz \wedge dx + z \, dx \wedge dy
    \end{equation}
    で定めると$\iota^* \omega = \iota^* \mu$が成り立つ。
    $d\mu = 3 \, dx \wedge dy \wedge dz$だから
    \begin{alignat}{1}
        \int_{S^2} \iota^* \omega
            &= \int_{\del D^3} \iota^* \omega \\
            &= \int_{\del D^3} \iota^* \mu \\
            &= \int_{D^3} d\mu
                \quad (\text{Stokes の定理}) \\
            &= \int_{D^3} 3 \, dx \wedge dy \wedge dz \\
            &= 4 \pi
    \end{alignat}
    を得る。
    これは$\int_{S^2} \iota^* \omega = 0$に矛盾。
    したがって$\omega$は完全形式でない。

    %$A \coloneqq [0, \pi] \times [0, 2\pi] \subset \R^2$とおき
    %\begin{equation}
    %    F \colon A \to S^2,
    %    \quad
    %    (s, t) \mapsto (\sin s \cos t, \sin s \sin t, \cos s)
    %\end{equation}
    %と定める。
    %$A$の$\R^2$における内部$\mathring{A}$への$F$の制限は
    %向きを保つ微分同相写像であり、
    %$\Cl F(\mathring{A}) = S^2$が成り立つ。
    %また、$F$の Jacobi 行列は
    %\begin{equation}
    %    \begin{bmatrix}
    %        \cos s \cos t & -\sin s \sin t \\
    %        \cos s \sin t & \sin s \cos t \\
    %        -\sin s & 0
    %    \end{bmatrix}
    %\end{equation}
    %だから
    %\begin{alignat}{1}
    %    dy \wedge dz
    %        &= \sin^2 s \cos t \, ds \wedge dt \\
    %    dz \wedge dx
    %        &= \sin^2 s \sin t \, ds \wedge dt \\
    %    dx \wedge dy
    %        &= \sin s \cos t \, ds \wedge dt
    %\end{alignat}
    %である。
    %よって
    %\begin{alignat}{1}
    %    \int_{S^2} \omega
    %        &= \int_A \left(
    %            \sin s \cos t \cdot \sin^2 s \cos t
    %            + \sin s \sin t \cdot \sin^2 s \sin t
    %            + \cos s \cdot \sin s \cos s
    %        \right) \, ds \wedge dt \\
    %        &= \int_{s = 0}^\pi \int_{t = 0}^{2\pi}
    %            \sin s \cos^2 s \, ds dt \\
    %        &= \frac{4}{3} \pi
    %\end{alignat}
    %を得る。
    %これは$\int_{S^2} \omega = 0$に矛盾。
    %したがって$\omega$は完全形式でない。

    \uline{(3)} \quad
    \TODO{書き直す}
    $A \coloneqq [0, \pi] \times [0, 2\pi] \subset \R^2$とおき
    \begin{equation}
        G \colon A \to S,
        \quad
        (s, t) \mapsto (
            10 \sin s \cos t + 1,
            10 \sin s \sin t + 2,
            10 \cos s + 3
        )
    \end{equation}
    と定めると、
    $A$の$\R^2$における内部$\mathring{A}$への$G$の制限は
    向きを保つ微分同相写像であり、
    $\Cl G(\mathring{A}) = S$が成り立つ。
    よって、(2) での計算結果も用いれば
    \begin{alignat}{1}
        \int_S \omega
            &= \int_A \big(
                (10 \sin s \cos t + 1) \cdot 100 \sin^2 s \cos t \\
            &\qquad + (10 \sin s \sin t + 2) \cdot 100 \sin^2 s \sin t \\
            &\qquad + (10 \cos s + 3) \cdot 100 \sin s \cos s
                \big) \, ds \wedge dt \\
            &= 4000 \pi
                + 100 \int_A \left(
                    \sin^2 s \cos t
                    + 2 \sin^2 s \sin t
                    + 3 \sin s \cos s
                \right) \, ds \wedge dt \\
            &= 4000 \pi
                + 600 \pi \underbrace{\int_0^{\pi} \sin s \cos s \, ds}_{= 0} \\
            &= 4000 \pi
    \end{alignat}
    を得る。
\end{answer}

\begin{problem}[問4]
\end{problem}

\begin{answer}
    \uline{(1)} \quad
    $(a, b, c) \in \R^3$とする。
    $z$が関数$\R^3 \to \R$であることに注意して
    初期値問題
    \begin{alignat}{1}
        1 &= dz(X_{(a, b, c)})
            = \dd{t}\bigg|_{t = 0} z \circ \varphi_t(a, b, c) \\
        z \circ \varphi_0(a, b, c)
            &= c
    \end{alignat}
    を考えると
    $z \circ \varphi_t(a, b, c) = c + t \; (t \in \R)$
    は解のひとつであり、
    解の一意性よりこれが唯一の解である。
    よって$z \circ \varphi_{-c}(a, b, c) = c - c = 0$だから
    $\varphi_{-c}(a, b, c) \in \R^2$である。

    \uline{(2)} \quad
    (2-a) \Rightarrow (2-b): \quad
    \begin{alignat}{1}
        \pi \circ \varphi_t(x, y, z)
            &= \varphi_{- z \circ \varphi_t(x, y, z)}(\varphi_t(x, y, z)) \\
            &= \varphi_{- t - z}(\varphi_t(x, y, z)) \\
            &= \varphi_{-z} \circ \varphi_{-t} \circ \varphi_t(x, y, z) \\
            &= \varphi_{-z} (x, y, z) \\
            &= \pi(x, y, z)
    \end{alignat}
    だから
    $\varphi_t^* \omega
        = \varphi_t^* \pi^* \alpha
        = (\pi \circ \varphi_t)^* \alpha
        = \pi^* \alpha
        = \omega$
    である。

    (2-b) \Rightarrow (2-a): \quad

    \TODO{わからない}

    \uline{(3)} \quad
    (2-b) との同値を示す。

    (2-b) \Rightarrow (3): \quad
    $p = (x, y, z) \in \R^3$とする。
    Lie 微分の定義より
    \begin{equation}
        (L_X \omega)_p
            = \dd{t}\bigg|_{t = 0}
                (\varphi_t^* \omega)_p
            = \lim_{t \to 0}
                \frac{(\varphi_t^* \omega)_p - \omega_p}{t}
            = 0
    \end{equation}
    である。

    (3) \Rightarrow (2-b): \quad
    $p = (x, y, z) \in \R^3$とする。
    一般に次が成り立つ。
    \begin{equation}
        \dd{t}\bigg|_{t = t_0} (\varphi_t^* \omega)_p
            = (\varphi_{t_0}^* (L_X \omega))_p
            \quad
            (\forall t_0 \in \R)
    \end{equation}
    実際、変数変換$s = t - t_0$により
    \begin{alignat}{1}
        \dd{t}\bigg|_{t = t_0} (\varphi_t^* \omega)_p
            &= \dd{s}\bigg|_{s = 0}
                (\varphi_{s + t_0}^* \omega)_p \\
            &= \dd{s}\bigg|_{s = 0}
                (
                    \varphi_{t_0}^*
                    \varphi_{s}^*
                    \omega
                )_p \\
            &= \dd{s}\bigg|_{s = 0}
                (d(\varphi_{t_0})_p)^*
                (
                    \varphi_{s}^*
                    \omega
                )_{\varphi_{t_0} (p)} \\
            &= \dd{s}\bigg|_{s = 0}
                (d(\varphi_{t_0})_p)^*
                (d(\varphi_{s})_{\varphi_{t_0} (p)})^*
                \omega_{\varphi_s \varphi_{t_0} (p)} \\
            &= (d(\varphi_{t_0})_p)^*
                \left(
                    \dd{s}\bigg|_{s = 0}
                    (d(\varphi_{s})_{\varphi_{t_0} (p)})^*
                    \omega_{\varphi_s \varphi_{t_0} (p)}
                \right)
                \quad
                (\text{連続性}) \\
            &= (d(\varphi_{t_0})_p)^*
                \left(
                    \dd{s}\bigg|_{s = 0}
                    (\varphi_s^* \omega)_{\varphi_{t_0} (p)}
                \right) \\
            &= (d(\varphi_{t_0})_p)^*
                (L_X \omega)_{\varphi_{t_0} (p)} \\
            &= (\varphi_{t_0}^* (L_X \omega))_p
    \end{alignat}
    だからである。
    よって$L_X \omega = 0$なら
    $\dd{t}\bigg|_{t = t_0} (\varphi_t^* \omega)_p = 0$
    だから
    $(\varphi_t^* \omega)_p$は$t$によらず一定で、
    したがって$(\varphi_t^* \omega)_p = \omega_p$を得る。

    \uline{(4)} \quad
    (3) との同値を示す。

    (4) \Rightarrow (3): \quad
    Cartan のホモトピー公式
    $L_X \omega = d \iota_X \omega + \iota_X d \omega$
    より明らか。

    (3) \Rightarrow (4): \quad

    \TODO{わからない}
\end{answer}

\begin{problem}[問5]
\end{problem}

\begin{answer}
    \uline{(1)} \quad
    $E$が向きづけ可能なら
    $\bigwedge^1 E = E$が自明束だから
    大域自明化が存在する。

    \uline{(2-i, ii)} \quad
    $[0, 1] \times \R$上の同値関係$\sim$を
    $(0, y) \sim (1, -y)$により生成されるものとして定め、
    $E \coloneqq ([0, 1] \times \R) / \sim$とおき、
    商写像を$p$とおく。
    ベクトル束
    \begin{alignat}{1}
        \pi \colon E \to S^1, \quad p(x, y) \mapsto e^{2\pi ix}
    \end{alignat}
    は向きづけ不可能であることを示す。
    $E$が向きづけ可能であったとすると
    $E$はベクトル束として$S^1 \times \R$と同型だから
    とくに同相写像$\varphi \colon E \to S^1 \times \R$が存在する。
    $K \coloneqq S^1 \times \{ 0 \} \subset S^1 \times \R$、
    $L \coloneqq \varphi^{-1}(K)$とおく。
    $K$はコンパクトだから$L$もコンパクトである。
    このとき$p^{-1}(L)$は$[0, 1] \times \R$の有界閉集合である。
    よってある$a > 0$が存在して
    $[0, 1] \times [-a, a] \supset p^{-1}(L)$、
    したがって$L' \coloneqq p([0, 1] \times [-a, a]) \supset L$となる。
    $L'$はコンパクトだから
    $K' \coloneqq \varphi(L')$もコンパクトである。
    したがって
    ある$b > 0$が存在して
    $S^1 \times (-b, b) \supset K' \supset K$をみたす。
    よって$(S^1 \times \R) \setminus K'$は連結でない。
    一方$E \setminus L'$は連結である。
    これで矛盾がいえた。
\end{answer}

\begin{problem}[問6]
\end{problem}

\begin{answer}
    \uline{(1)} \quad
    同型$H_\dR^0(S^1) \cong \R$は
    $c \in \R$と定数関数$S^1 \to \R, z \mapsto c$を対応させることで定める。
    \begin{equation}
        \begin{tikzcd}
            \R
                \ar{r}
                & H_\dR^0(S^1)
                    \ar{r}{f_n^*}
                & H_\dR^0(S^1)
                    \ar{r}
                & \R \\
            1
                \ar[mapsto]{r}
                & {[1]}
                    \ar[mapsto]{r}
                & f_n^*{[1]} = [1 \circ f_n] = [1]
                    \ar[mapsto]{r}
                & 1
        \end{tikzcd}
    \end{equation}
    より$f_n^*$は恒等写像である。

    \uline{(2)} \quad
    $S^1$には単位閉円板$D^2$の境界としての向きが入っているとする。
    $S^1$の極座標を$\theta$とおくと$d\theta$は$S^1$の体積形式である。
    $d\theta$に関する$S^1$の体積は$\int_{S^1} d\theta = 2\pi$だから
    基本コホモロジー類は$[S^1] = \frac{1}{2\pi} [d\theta]$である。
    同型$H_\dR^1(S^1) \cong \R$は
    $1 \in \R$と$[S^1]$を対応させることで定める。
    \begin{equation}
        \begin{tikzcd}
            \R
                \ar{r}
                & H_\dR^1(S^1)
                    \ar{r}{f_n^*}
                & H_\dR^1(S^1)
                    \ar{r}
                & \R \\
            1
                \ar[mapsto]{r}
                & {[S^1]}
                    \ar[mapsto]{r}
                & f_n^*{[S^1]}
                    = \frac{1}{2\pi}[d(\theta \circ f_n)]
                    = n [S^1]
                    \ar[mapsto]{r}
                & n
        \end{tikzcd}
    \end{equation}
    より$f_n^*$は$n$倍写像である。
\end{answer}


\end{document}