\documentclass[report, notitlepage]{jlreq}
\usepackage{docmute}
\usepackage{global}
\usepackage{./sub/local}
\def\assetspath{./}
\makeindex
\makeglossaries

\title{可微分多様体と微分幾何学の基礎}
\author{Yahata}
\date{}

\begin{document}

\maketitle
\begin{abstract}
    多様体とは、曲線や曲面といった図形の概念を一般化したものであり、
    微分幾何学に限らず様々な幾何学の土俵となっている。
    微分幾何学は、多様体の上で微積分を用いて展開される幾何学であり、
    一般相対性理論をはじめとして物理学に多くの応用がある。

    本稿では可微分多様体および微分幾何学の基礎事項を整理する。
    第1部では可微分多様体の基礎と de Rham コホモロジーについて述べる。
    内容は\cite{Lee12}を参考にしている。
    第2部ではベクトル束と主ファイバー束の接続について述べる。
    扱う内容は\cite{小林04}をベースとし、
    定義などは\cite{Lee18}や\cite{Tu17}を参考にしている。
    第3部では擬 Riemann 多様体と計量について述べる。
    内容は\cite{Lee18}を参考にしている。
\end{abstract}

\setcounter{tocdepth}{1}
\tableofcontents
\markboth{\contentsname}{}

% ============================================================
%
% ============================================================
\part{可微分多様体の基礎}
\documentclass[report]{jlreq}
\usepackage{global}
\usepackage{./local}
\subfiletrue
%\makeindex
\begin{document}


% ============================================================
%
% ============================================================
\chapter{群}

群について述べる。

% ------------------------------------------------------------
%
% ------------------------------------------------------------
\section{群}

\begin{definition}[モノイド]
    $M$を集合、
    $e \in M$、
    $\cdot \colon M \times M \to M$を写像とし、
    各$x, y \in M$に対し$\cdot (x, y)$を
    $x \cdot y$や$xy$と書くことにする。
    組$(M, \cdot, e)$が
    \term{モノイド}[monoid]{モノイド}
    であるとは、次が成り立つことをいう:
    \begin{description}
        \item[(M1) 結合律]
            各$x, y, z \in M$に対して
            $(x \cdot y) \cdot z = x \cdot (y \cdot z)$
            が成り立つ。
        \item[(M2) 単位元]
            各$x \in M$に対して
            $x \cdot e = x = e \cdot x$
            が成り立つ。
    \end{description}
    組$(M, \cdot, e)$のことを
    記号の濫用で単に$(M, \cdot)$や$M$と書くことがある。
    さらに
    \begin{itemize}
        \item $e$を$M$の
            \term{単位元}[unit]{単位元}[たんいげん]という。
    \end{itemize}
\end{definition}

\begin{definition}[群]
    モノイド$(G, \cdot, e)$が
    \term{群}[group]{群}[ぐん]であるとは、
    次が成り立つことをいう:
    \begin{description}
        \item[(G1) 逆元]
            各$x \in G$に対して
            ある$y \in G$が存在して
            $x \cdot y = e = y \cdot x$
            が成り立つ。
    \end{description}
    さらに
    \begin{itemize}
        \item $y$を$x$の
            \term{逆元}[inverse]{逆元}[ぎゃくげん]といい、
            $x^{-1}$と書く。
    \end{itemize}
\end{definition}

\begin{definition}[アーベル群]
    群$(G, +, 0)$が
    \term{アーベル群}[abelian group]{アーベル群}[あーべるぐん]であるとは、
    次が成り立つことをいう:
    \begin{description}
        \item[(A1) 可換性]
            各$x, y \in G$に対して
            $x + y = y + x$
            が成り立つ。
    \end{description}
\end{definition}

\begin{definition}[群準同型]
    \TODO{}
\end{definition}



% ------------------------------------------------------------
%
% ------------------------------------------------------------
\section{部分群}

\begin{proposition}[部分群の特徴付け]
    \TODO{}
\end{proposition}

\begin{proof}
    \TODO{}
\end{proof}

\begin{definition}[生成された部分群]
    $G$を群、$S \subset G$とする。
    このとき、集合
    \begin{equation}
        \langle S \rangle
            \coloneqq \{
                g_1^{\eps_1} \cdots g_n^{\eps_n}
                \mid
                n \in \Z_{\ge 1}, \;
                g_i \in S, \;
                \eps_i \in \{ \pm 1 \}
            \}
    \end{equation}
    は定義から明らかに$G$の部分群となる。
    $\langle S \rangle$を
    \term{$S$により生成された$G$の部分群}[subgroup of $G$ generated by $S$]
        {生成された部分群}[せいせいされたぶぶんぐん]
    といい、
    $S$を$\langle S \rangle$の
    \term{生成系}[generating set]{生成系}[せいせいけい]
    という。

    $G$が有限集合$S$により生成されるとき、
    $G = \langle S \rangle$は
    \term{有限生成}[finitely generated]{有限生成}[ゆうげんせいせい]
    であるといい、
    さらに$S$が1点集合$S = \{ x \}$のとき
    波括弧を省略して$\langle x \rangle$と書き、
    $G = \langle x \rangle$はa
    \term{巡回群}[cyclic group]{巡回群}[じゅんかいぐん]
    であるという。
\end{definition}

\begin{proposition}[生成された部分群の特徴付け]
    $G$を群、$S \subset G$とする。
    このとき
    \begin{equation}
        \langle S \rangle
            = \bigcap_{\substack{
                G' \subset G \colon \text{部分群} \\
                G' \supset S
            }} G'
    \end{equation}
    が成り立つ。
\end{proposition}

\begin{proof}
    \TODO{}
\end{proof}



% ------------------------------------------------------------
%
% ------------------------------------------------------------
\section{群作用}

群の作用について述べる。

\begin{definition}[作用]
    $G$を群、$X$を集合とする。
    写像
    \begin{equation}
        G \times X \to X,
        \quad
        (g, x) \mapsto gx
    \end{equation}
    が与えられていて
    \begin{enumerate}
        \item 各$g_1, g_2 \in G, \; x \in X$に対して
            $(g_1 g_2) x = g_1 (g_2 x)$が成り立つ。
        \item 各$x \in X$に対して$e_G x = x$が成り立つ。
    \end{enumerate}
    をみたすとき、
    $G$は$X$に左から\term{作用}[act]{作用}[さよう]するという。
    $G$が左から作用している集合を
    \term{左$G$-集合}[left $G$-set]{$G$-集合}[Gしゅうごう]
    という。
    右からの作用も同様に定まる。
\end{definition}

\begin{definition}[軌道]
    $G$を群、$X$を左$G$-集合とする。
    $X$上の同値関係を
    \begin{equation}
        \text{$x$と$y$が同値}
        \quad \logeq \quad
        \exists g \in G \quad \text{s.t.} \quad gx = y
    \end{equation}
    で定めることができ、
    この同値関係に関する同値類を
    \term{軌道}[orbit]{軌道}[きどう]
    という。
\end{definition}

\begin{definition}[固定部分群]
    \idxsym{stabilizer}{$\Stab_G(x)$}{$x$の固定部分群}
    $G$を群、$X$を左$G$-集合とする。
    各$x \in X$に対し、$G$の部分群
    \begin{equation}
        \Stab_G(x) \coloneqq \{ g \in G \colon xg = x \}
    \end{equation}
    を$x$の
    \term{固定部分群}[stabilizer]{固定部分群}[こていぶぶんぐん]
    という。
\end{definition}

\begin{definition}[忠実作用]
    $G$を群、$X$を左$G$-集合とする。
    $G$の$X$への作用が
    \term{忠実}[faithful]{忠実}[ちゅうじつ]
    あるいは
    \term{効果的}[effective]{効果的}[こうかてき]
    であるとは、次が成り立つことをいう:
    \begin{itemize}
        \item すべての$x \in X$を
            固定する$g \in G$は単位元のみである。
    \end{itemize}
    定義から明らかに、作用が忠実であることは
    作用の定める表現$G \to \Aut(X)$が単射であることと同値である。
\end{definition}

\begin{definition}[自由作用]
    $G$を群、$X$を左$G$-集合とする。
    $G$の$X$への作用が
    \term{自由}[free]{自由}[じゆう]
    であるとは、
    単位元以外の$g \in G$はすべての$x \in X$を動かすように作用すること、すなわち
    \begin{equation}
        \forall g \in G \; (g \neq 1 \Rightarrow (\forall x \in X \; (xg \neq x)))
    \end{equation}
    が成り立つことをいう。
    これはすべての$x \in X$に対し
    $\Stab_G(x)$が自明群であることと同値である。
\end{definition}

\begin{definition}[推移的作用]
    $G$を群、$X$を左$G$-集合とする。
    各$x \in X$に対し$xG \coloneqq \{ xg \in X \colon g \in G \}$と書く。
    $G$の$X$への作用が
    \term{推移的}[transitive]{推移的}[すいいてき]
    であるとは、
    \begin{equation}
        X = xG \quad (\forall x \in X)
    \end{equation}
    が成り立つことをいう。これは次と同値である:
    \begin{itemize}
        \item $\forall x_0 \in X$を固定すると、
            $\forall y \in X$に対し$\exists g \in G$がとれて$y = x_0 g$が成り立つ。
    \end{itemize}
\end{definition}

\subsection{$G$-torsor}

\begin{definition}[$G$-torsor]
    $G$を群、$X$を非空な左$G$-集合とする。
    \term{shear map}{shear map}
    と呼ばれる写像
    \begin{equation}
        G \times X \to X \times X,
        \quad
        (g, x) \mapsto (gx, x)
    \end{equation}
    が全単射であるとき、
    $X$を\term{$G$-torsor}{$G$-torsor}[G-torsor]
    という。
\end{definition}

\begin{proposition}[$G$-torsor の特徴付け]
    $G$を群、$X$を左$G$-集合とする。
    このとき、次は同値である:
    \begin{enumerate}
        \item $X$は$G$-torsorである。
        \item $G$の$X$への作用は推移的かつ自由である。
        \item $G$の$X$への作用は推移的であり、さらに
            固定部分群が自明群であるような$x \in X$が存在する。
        \item $X$と$G$は左$G$-集合として同型である。
    \end{enumerate}
\end{proposition}

\begin{proof}
    \TODO{}
\end{proof}

\begin{theorem}[類等式]
    \TODO{}
\end{theorem}

\begin{proof}
    \TODO{}
\end{proof}

\begin{theorem}[Lagrange]
    \TODO{}
\end{theorem}

\begin{proof}
    \TODO{}
\end{proof}



% ------------------------------------------------------------
%
% ------------------------------------------------------------
\section{商群}



% ------------------------------------------------------------
%
% ------------------------------------------------------------
\section{準同型定理}

\begin{theorem}[準同型定理]
    \TODO{}
\end{theorem}

\begin{proof}
    \TODO{}
\end{proof}

\begin{theorem}[部分群の対応原理]
    \TODO{}
\end{theorem}

\begin{proof}
    \TODO{}
\end{proof}


% ------------------------------------------------------------
%
% ------------------------------------------------------------
\section{Sylow の定理}

\begin{theorem}[Sylow]
    \TODO{}
\end{theorem}

\begin{proof}
    \TODO{}
\end{proof}



% ------------------------------------------------------------
%
% ------------------------------------------------------------
\section{群の表現}
\label[section]{section:group-action}

\TODO{群の作用とはどう違う?}

\begin{definition}[群の表現]
    $G$を群、$\calC$を圏とする。
    $G$は、射を群の元とし単一の対象$*$からなる圏とみなせる。
    $\calC$における$G$の
    \term{表現}[representation]{表現}[ひょうげん]
    とは、圏$G$から$\calC$への関手のことである。
    $T \colon G \to \calC$を表現とするとき、
    各射$T(g)$は$\calC$の対象$X \coloneqq T(*)$上の自己同型射を与えるから、
    群準同型$G \to \Aut(X)$が定まる。
    この群準同型も\term{表現}[representation]{表現}[ひょうげん]と呼ぶ。
\end{definition}

\begin{remark}
    群の作用は
    集合の圏における群の表現
    (これを\term{置換表現}[permutation representation]{置換表現}[ちかんひょうげん]という)
    に他ならない。
\end{remark}

\begin{example}
    ~
    \begin{itemize}
        \item 有限群の表現
        \item 位相群の表現
        \item Lie 群の表現
        \item \TODO{}
    \end{itemize}
\end{example}

% ------------------------------------------------------------
%
% ------------------------------------------------------------
\section{自由群}

% ------------------------------------------------------------
%
% ------------------------------------------------------------
\section{自由積と融合積}

% ------------------------------------------------------------
%
% ------------------------------------------------------------
\section{アーベル化}

\begin{theorem}[アーベル化の普遍性]
    \TODO{}
\end{theorem}

\begin{proof}
    \TODO{}
\end{proof}

% ------------------------------------------------------------
%
% ------------------------------------------------------------
\section{可解群}




% ============================================================
%
% ============================================================
\chapter{基本的な群}

% ------------------------------------------------------------
%
% ------------------------------------------------------------
\section{対称群}

% ------------------------------------------------------------
%
% ------------------------------------------------------------
\section{2面体群}

% ------------------------------------------------------------
%
% ------------------------------------------------------------
\section{4元数群}

% ------------------------------------------------------------
%
% ------------------------------------------------------------
\section{一般線型群}




\end{document}
\documentclass[report]{jlreq}
\usepackage{global}
\usepackage{./local}
\subfiletrue
\def\assetspath{../}
%\makeindex
\begin{document}

第1部では、
情報幾何の議論に必要となる確率論と数理統計の前提知識を整理する。
ただしここではごく基礎的な事項のみを述べ、
それ以外の話題は第2部で情報幾何的な視点から扱うものとする。
そもそも確率・統計を幾何学的に定式化しようとする際には、
座標変換に関する不変性などを定義に繰り入れる必要がある。
このとき定義が見かけ上煩雑になることで、
確率・統計的な意味を一時的に見失ってしまうおそれがある。
それを防ぐために、確率・統計的な意味を一旦明らかにしておくのもこの部の目的である。

% ============================================================
%
% ============================================================
\chapter{確率論}

確率論の基礎事項を整理する。

% ------------------------------------------------------------
%
% ------------------------------------------------------------
\section{確率空間}

\begin{definition}[確率空間]
    測度空間$(\Omega, \calF, P)$であって
    \begin{enumerate}
        \item 各$E \in \calF$に対し$P(E) \ge 0$
        \item $P(\Omega) = 1$
    \end{enumerate}
    をみたすものを
    \term{確率空間}[probability space]{確率空間}[かくりつくうかん]
    といい、
    $P$を$(\Omega, \calF)$上の
    \term{確率測度}[probability measure]{確率測度}[かくりつそくど]
    あるいは
    \term{確率分布}[probability distribution]{確率分布}[かくりつぶんぷ]
    という。
\end{definition}

\begin{definition}[確率変数]
    $(\Omega, \calF, P)$を確率空間、
    $(\calX, \calA)$を可測空間とする。
    可測関数$X \colon (\Omega, \calF) \to (\calX, \calA)$を
    $(\calX, \calA)$に値をもつ
    \term{確率変数}[random variable; r.v.]{確率変数}[かくりつへんすう]
    という。
    とくに$\calX = \R, \; \calA = \calB(\R) \; (\text{Borel 集合族})$のとき、
    $X$を単に\term{確率変数}{確率変数}[かくりつへんすう]という。
\end{definition}

\begin{definition}[確率変数の確率分布]
    $X \colon (\Omega, \calF) \to (\calX, \calA)$を確率変数とする。
    このとき、写像
    \begin{equation}
        P^X \colon \calA \to [0, +\infty],
            \quad
            E \mapsto P(X^{-1}(E))
            \quad
            (E \in \calA)
    \end{equation}
    は$(\calX, \calA)$上の測度となる (このあと示す)。
    これを
    \term{$X$の確率分布}[probability distribution of $X$]
        {確率分布!確率変数の---}[かくりつぶんぷ]
    という。

    $X$の確率分布が$(\calX, \calA)$上のある確率分布$\nu$に等しいとき、
    $X$は
    \term{$\nu$に従う}{確率分布に従う}[かくりつぶんぷにしたがう]
    という。
\end{definition}

\begin{proof}
    \TODO{}
\end{proof}

確率論には「\term{0-1法則}[zero-one law]{0-1法則}[0-1ほうそく]」
と呼ばれるいくつかの定理がある。
次に述べる Borel-Cantelli の補題はそのひとつである。

\begin{theorem}[Borel-Cantelli の補題]
    \TODO{}
\end{theorem}

\begin{proof}
    \TODO{}
\end{proof}

\begin{theorem}[Jensen の不等式]
    $(\Omega, \calF, P)$を確率空間、
    $A \subset \R^n$を凸集合、
    $\Psi \colon A \to \R$を凸関数とする。
    このとき、$\Psi(f)$が$P$-可積分となるような
    任意の$P$-可積分関数$f$に対し
    \begin{equation}
        \Psi\myparen{
            \int_\Omega f(x) \, P(dx)
        }
            \le \int_\Omega \Psi(f(x)) \, P(dx)
    \end{equation}
    が成り立つ。
\end{theorem}

\begin{proof}
    \TODO{cf. \cite[p.153]{Bog07}}
\end{proof}

% ------------------------------------------------------------
%
% ------------------------------------------------------------
\section{離散確率分布}

\begin{definition}[離散確率分布]
    $\calX$を高々可算集合、
    $\calA \coloneqq 2^{\calX}$とするとき、
    可測空間$(\calX, \calA)$を
    \term{離散確率空間}[discrete probability space]
        {離散確率空間}[りさんかくりつくうかん]
    といい、$(\calX, \calA)$上の確率分布を
    \term{離散確率分布}[discrete probability distribution]
        {離散確率分布}[りさんかくりつぶんぷ]
    という。
\end{definition}

\begin{definition}[確率質量関数]
    $(\calX, \calA)$を離散確率空間、
    $\mu$を$(\calX, \calA)$上の確率分布とする。
    このとき、$(\calX, \calA)$上の数え上げ測度に関する
    $\mu$の Radon-Nikodym 微分\footnote{
        数え上げ測度に関する$\mu$の Radon-Nikodym 微分はつねに存在する。
        なぜなら、$(\calX, \calA)$上のいかなる確率測度も
        $(\calX, \calA)$上の数え上げ測度に関して絶対連続だからである。
    }、すなわち
    \begin{equation}
        \mu(E)
            = \sum_{x \in E} p(x)
            \quad (E \in \calA)
    \end{equation}
    なる関数$p \colon \calX \to [0, +\infty]$を
    $\mu$の\term{確率質量関数}[probability mass function; PMF]
        {確率質量関数}[かくりつしつりょうかんすう]
    という。
\end{definition}

\begin{definition}[確率母関数]
    \TODO{}
\end{definition}

% ------------------------------------------------------------
%
% ------------------------------------------------------------
\section{連続確率分布}

$\R$上の確率分布は確率論や統計学において重要である。

\begin{definition}[連続確率分布]
    $\calB$を$\R$の Borel 集合族とするとき、
    可測空間$(\R, \calB)$を
    \term{連続確率空間}[continuous probability space]
        {連続確率空間}[れんぞくかくりつくうかん]
    といい、$(\R, \calB)$上の確率分布を
    \term{連続確率分布}[continuous probability distribution]
        {連続確率分布}[れんぞくかくりつぶんぷ]
    という。
\end{definition}

\begin{definition}[絶対連続分布]
    \TODO{}
\end{definition}

\begin{definition}[確率密度関数]
    $(\R, \calB)$を連続確率空間、
    $\mu$を
    $(\R, \calB)$上の Lebesgue 測度に関し絶対連続な確率分布とする。
    このとき、
    $\mu$の Radon-Nikodym 微分、すなわち
    \begin{equation}
        \mu(E)
            = \int_E p(x) \, dx
            \quad (E \in \calB)
    \end{equation}
    なる関数$p \colon \R \to [0, +\infty]$を
    $\mu$の\term{確率密度関数}[probability density function; PDF]
        {確率密度関数}[かくりつみつどかんすう]
    という。
\end{definition}

\begin{definition}[モーメント母関数]
    \TODO{}
\end{definition}

モーメント母関数と異なり、特性関数は常に存在する。

\begin{definition}[特性関数]
    \TODO{}
\end{definition}

% ------------------------------------------------------------
%
% ------------------------------------------------------------
\section{多変量分布}

\TODO{}

% ------------------------------------------------------------
%
% ------------------------------------------------------------
\section{確率変数の収束}

\begin{theorem}[大数の法則]
    \TODO{}
\end{theorem}

\begin{proof}
    \TODO{}
\end{proof}

\begin{theorem}[中心極限定理]
    \TODO{}
\end{theorem}

\begin{proof}
    \TODO{}
\end{proof}



% ============================================================
%
% ============================================================
\chapter{推定}

この章では、統計的推論の基本的な問題のひとつである推定について述べる。

% ------------------------------------------------------------
%
% ------------------------------------------------------------
\section{十分統計量}

標本 (=確率変数) \TODO{標本とは確率変数のことなのか?$\Omega$の元ではないのか?}の実現値に対し
その特徴を要約した値を割り当てる関数を統計量という。
事象は統計量を通して「観測」される。
もちろん標本それ自体も統計量である。

\begin{definition}[統計量]
    \TODO{}
\end{definition}

統計量を用いて母集団 (=確率分布) のパラメータを推定することを考える。

\begin{definition}[点推定]
    標本$X$から母集団のパラメータ$\theta$を特定することを
    \term{推定}[estimate]{推定}[すいてい]
    といい、とくに一意に推定することを
    \term{点推定}[point estimation]{点推定}[てんすいてい]
    という。
    点推定において、標本$X$の実現値$x$に対し
    $\theta$の\term{推定値}[estimate]{推定値}[すいていち]
    $\what{\theta}(x)$を割り当てる関数$\what{\theta}$を
    $\theta$の\term{推定量}[estimator]{推定量}[すいていりょう]
    という。
\end{definition}

\begin{remark}
    今後、点推定のことを単に推定ということにする。
\end{remark}

推定のために"十分"な情報を含んだ統計量を十分統計量という。

\begin{definition}[十分統計量]
    \TODO{}
\end{definition}

\begin{example}[Bernoulli 分布の例]
    \TODO{}
\end{example}

次の定理はある統計量が十分統計量であるための必要十分条件を与え、
具体的な判定に役立つ。

\begin{theorem}[Fisher-Neyman の分解定理]
    \TODO{}
\end{theorem}

\begin{proof}
    \TODO{}
\end{proof}

\begin{example}[正規分布の例]
    \TODO{}
\end{example}

% ------------------------------------------------------------
%
% ------------------------------------------------------------
\section{指数型分布族}

指数型分布族について述べる。

\begin{definition}[指数型分布族]
    $(\calX, \calA)$を可測空間、
    $\mu$を$(\calX, \calA)$上の$\sigma$-有限測度、
    $\calP = (P_\theta)_{\theta \in \Theta}$を
    $(\calX, \calA)$上の確率分布族とする。
    ここで、各$P_\theta$が$\mu$に関し絶対連続で、
    Radon-Nikodym 微分が
    \begin{equation}
        \frac{dP_\theta}{d\mu}(x)
            = g(x) \exp\myparen{
                \sum_{i = 1}^m a_i(\theta) T_i(x)
                - \psi(\theta)
            }
            = g(x) \exp(a(\theta) \cdot T(x) - \psi(\theta))
            \quad
            \text{$\mu$-a.e. $x \in \calX$}
    \end{equation}
    の形に表せるとき、
    $\calP$を\term{指数型分布族}[exponential family]
        {指数型分布族}[しすうがたぶんぷぞく]
    という。
    ただし、$T_i, g$は$(\calX, \calA)$上の可測関数、
    $a_i, \psi$は$\Theta$上の実数値関数である。
    とくに
    \begin{equation}
        \frac{dP_\theta}{d\mu}(x)
            = g(x) \exp\myparen{
                \sum_{i = 1}^m \theta_i T_i(x)
                - \psi(\theta)
            }
            = g(x) \exp(\theta \cdot T(x) - \psi(\theta))
            \quad
            \text{$\mu$-a.e. $x \in \calX$}
    \end{equation}
    の形を
    \term{正準形}[canonical form]{正準形}[せいじゅんけい]
    という\footnote{
        指数型分布族は常に正準形で書くことができる。
        実際、$\theta$の代わりに$a(\theta)$をパラメータとすればよい。
    }。
\end{definition}

以下に指数型分布族に関する具体例を挙げる。

\begin{example}[正規分布]
    \TODO{}
\end{example}

\begin{proposition}
    指数型分布族の定義の$T = (T_1, \dots, T_m)$は
    $\theta$の十分統計量である。
\end{proposition}

\begin{proof}
    Fisher-Neyman の分解定理より従う。
\end{proof}

% ------------------------------------------------------------
%
% ------------------------------------------------------------
\section{混合型分布族}

\begin{definition}[混合型分布族]
    \TODO{}
\end{definition}

% ------------------------------------------------------------
%
% ------------------------------------------------------------
\section{点推定の手法}

\begin{definition}[最尤推定]
    \TODO{}
\end{definition}

\begin{definition}[モーメント法]
    \TODO{}
\end{definition}

\begin{definition}[最小2乗法]
    \TODO{}
\end{definition}

% ------------------------------------------------------------
%
% ------------------------------------------------------------
\section{推定量の性質}

\begin{definition}[不偏性]
    \TODO{}
\end{definition}

\begin{definition}[一様最小分散不偏推定量; UMVUE]
    \TODO{}
\end{definition}

% ------------------------------------------------------------
%
% ------------------------------------------------------------
\section{推定量の評価}

\begin{definition}[スコア関数]
    \TODO{}
\end{definition}

Fisher 情報量を定義する。

\begin{definition}[Fisher 情報量]
    \TODO{}
\end{definition}

不偏推定量の分散の下界を Fisher 情報量の言葉で与えるのが Cramer-Rao 不等式である。
\TODO{Cramer-Rao 不等式は Fisher "計量" に対しどのような意味を持つ?}

\begin{proposition}[Cramer-Rao 不等式]
    \TODO{}
\end{proposition}

\begin{proof}
    \TODO{}
\end{proof}

\begin{definition}[有効性]
    \TODO{}
\end{definition}

% ------------------------------------------------------------
%
% ------------------------------------------------------------
\section{漸近理論}

漸近理論について述べる。
一般に最尤推定量は有効性をみたすとは限らないが、
漸近的には有効性をみたすことが知られている\footnote{
    ただし Neyman-Scott 問題においては
    最尤推定量は漸近有効とは限らない。
}。

\TODO{}



% ============================================================
%
% ============================================================
\chapter{仮説検定}

この章では、統計的推論の基本的な問題のひとつである仮説検定について述べる。

% ------------------------------------------------------------
%
% ------------------------------------------------------------
\section{仮説検定}

\TODO{}



\end{document}

\part{接続}
\documentclass[report]{jlreq}
\usepackage{../../global}
\usepackage{./local}
\subfiletrue
%\makeindex
\begin{document}

\TODO{接続とは一体何なのか?}

この部では接続について論じる。
接続とは、ベクトル場を方向微分して
新たなベクトル場を作る手続きのようなものである。
接束の接続はアファイン接続と呼ばれ、とくに重要である。


% ============================================================
%
% ============================================================
\newpage
\chapter{ベクトル値微分形式}

% ------------------------------------------------------------
%
% ------------------------------------------------------------
\section{ベクトル値微分形式}
\label[section]{sec:vector-valued-forms}

微分形式の概念をベクトル束に値をもつように一般化する。
これは後に主ファイバー束の接続を定義するために用いる。

\begin{definition}[ベクトル束に値をもつ微分形式]
    $M$を多様体、$E \to M$をベクトル束とし、$p \in \Z_{\ge 0}$とする。
    ベクトル束$\bigwedge^p T^*M \otimes E$の切断を
    \term{$E$に値をもつ$p$-形式}
    {ベクトル束に値をもつ微分形式}[べくとるそくにあたいをもつびぶんけいしき]
    あるいは
    \term{$E$-値$p$-形式}[$E$-valued $p$-form]
    {ベクトル束に値をもつ微分形式}[べくとるそくにあたいをもつびぶんけいしき]
    という。
    $E$-値$p$-形式全体のなす集合を
    \begin{equation}
        A^p(E) \coloneqq \Gamma\Bigl(
            \Bigl(\bigwedge^p T^*M\Bigr) \otimes E
        \Bigr)
    \end{equation}
    と書く。
    $E$-値$p$-形式は
    $\theta \otimes \xi \; (\theta \in A^p(M), \; \xi \in A^0(E))$の形
    の元の和に (一意ではないが) 書ける。
\end{definition}

\begin{remark}
    \TODO{どういうこと?}
    ベクトル空間の同型
    \begin{equation}
        \Hom(\Lambda^k T_xM, V)
            \cong (\Lambda^k T_xM)^* \otimes V
            \cong (\Lambda^k T_x^*M) \otimes V
    \end{equation}
    に注意すれば、$V$に値をもつ$k$-形式の値は、
    確かに$\Lambda^k T_xM \to V$の$\R$線型写像とみなせることがわかる。
\end{remark}

\begin{remark}
    テキストでは$\theta$と$\xi$の順序が逆になったりしているが、
    ここでは$\theta \otimes \xi$の順序に統一する。
\end{remark}

ベクトル値形式は
従来の意味での微分形式ではなく、
したがって外積は定義されていないが、
通常の外積から自然に定義が拡張される。

\begin{definition}[ベクトル値形式の外積]
    $M$を多様体、$E \to M$をベクトル束、
    $p, q \in \Z_{\ge 0}$とする。
    $\wedge \colon A^p(M) \times A^q(M) \to A^{p + q}(M)$を
    通常の外積とし、
    その一般化として
    $\wedge \colon A^p(M) \times A^q(E) \to A^{p + q}(E)$を
    \begin{alignat}{1}
        (\omega, \xi)
            = \left(
                \omega,
                \sum_{i} \alpha_i \otimes \xi_i
            \right)
            \mapsto
            \omega \wedge \xi
            &\coloneqq
            \sum_{i} \omega \wedge \alpha_i \otimes \xi_i \\
        &\qquad \quad
            (\alpha_i \in A^q(M), \; \xi_i \in A^0(E))
    \end{alignat}
    と定める。
    これは明らかに$\xi$の表し方によらず well-defined に定まる。
\end{definition}

\begin{definition}[ベクトル値形式の内積]
    $M$を多様体、
    $E \to M, \; F \to M$をベクトル束、
    $g \colon A^0(E) \times A^0(F) \to A^0(M)$を
    $\smooth(M)$-双線型写像とする。
    $g$の一般化として、同じ記号で写像
    $g \colon A^p(E) \times A^q(F) \to A^{p + q}(M)$を
    \begin{alignat}{1}
        (\omega, \xi)
            = \left(
                \sum_{i} \alpha_i \otimes \omega_i,
                \sum_{j} \beta_j \otimes \xi_j
            \right)
            &\mapsto
            g(\omega, \xi)
            \coloneqq
            \sum_{i, j}
            g(\omega_i, \xi_j)
            \alpha_i \wedge \beta_j \\
        &
            (
                \alpha_i \in A^p(M), \; \beta_j \in A^q(M), \;
                \omega_i \in A^0(E), \; \xi_j \in A^0(F)
            )
    \end{alignat}
    と定める。
    これは$\omega, \xi$の表し方によらず well-defined に定まり (証明略)、
    また$\smooth(M)$-双線型写像である。
\end{definition}

\begin{remark}
    上の定義の双線型写像$g \colon A^0(E) \times A^0(F) \to A^0(M)$の例としては、
    \begin{itemize}
        \item 双対の定める内積
            $\langle , \rangle \colon A^0(E^*) \times A^0(E) \to A^0(M)$
        \item 計量
            $g \colon A^0(E) \times A^0(E) \to A^0(M)$
    \end{itemize}
    などがある。
\end{remark}



% ============================================================
%
% ============================================================
\newpage
\chapter{主ファイバー束}

主ファイバー束は、多様体$M$上局所自明な群の族である。
ここで主ファイバー束という概念を持ち出す理由はベクトル束を調べるためであるが、
実際ベクトル束と主ファイバー束の間には良い関係がある。
というのも、ベクトル束はフレーム束と呼ばれる主ファイバー束と対応し、
逆に主ファイバー束はその構造群の表現を通してベクトル束と対応する。
したがって、あるベクトル束について調べたいときに
代わりに主ファイバー束を考えることで議論の見通しがよくなることがある。
そこで、この章では主ファイバー束とベクトル束の基本的な関係を調べることにする。

% ------------------------------------------------------------
%
% ------------------------------------------------------------
\section{ファイバー束}

ファイバー束を定義する。

\TODO{ファイバー束は構造群付きを基本として、
    修飾しない場合は自明な構造群を持つものと定義したい}

\begin{definition}[ファイバー束]
    $M, F$を多様体とする。
    多様体$E$が
    \term{ファイバー束}[fiber bundle]{ファイバー束}[ふぁいばーそく]
    であるとは、
    $E$が次をみたすことである:
    \begin{enumerate}
        \item 全射な{\smooth}写像$\pi \colon E \to M$が与えられている。
        \item \TODO{局所自明性}
    \end{enumerate}
\end{definition}

\TODO{主ファイバー束をファイバー束の特別な場合として定義したい}

\begin{definition}[主ファイバー束]
    $M$を多様体、
    $G$を Lie 群とする。
    多様体$P$が
    $G$を\term{構造群}[structure group]{構造群}[こうぞうぐん]とする$M$上の
    \term{主ファイバー束}[principal fiber bundle]{主ファイバー束}[しゅふぁいばーそく]、
    あるいは\term{主$G$束}[principal $G$-bundle]{主$G$束}[しゅGそく]であるとは、
    $P$が次をみたすことである:
    \begin{enumerate}
        \item 全射な{\smooth}写像$p \colon P \to M$が与えられている。
        \item $G$は$P$に右から{\smooth}に作用しており、
            さらに次をみたす:
            \begin{enumerate}[label=(\arabic{enumi}-\alph*)]
                \item 作用はファイバーを保つ。
                \item 作用はファイバー上単純推移的\footnote{
                        作用が\term{単純推移的}[simply transitive]
                        {単純推移的}[たんじゅんすいいてき]
                        であるとは、自由かつ推移的であることをいう。
                    }である。
            \end{enumerate}
        \item $M$のある開被覆$\{ U_\alpha \}_{\alpha \in A}$が存在して、
            各$U_\alpha$上に
            次をみたす写像
            $\sigma_\alpha \colon U_\alpha \to p^{-1}(U_\alpha)$
            が存在する:
            \begin{enumerate}[label=(\arabic{enumi}-\alph*)]
                \item $\sigma_\alpha$は{\smooth}であって
                    $p \circ \sigma_\alpha = \id_{U_\alpha}$をみたす。
                    すなわち$\sigma_\alpha$は$U_\alpha$上の
                    $P$の切断である。
                \item (局所自明性) 写像
                    \begin{equation}
                        \varphi_\alpha \colon p^{-1}(U_\alpha) \to U_\alpha \times G,
                        \quad
                        \underbrace{\sigma_\alpha(x) . s}_{
                            \mathclap{\text{群作用を「$.$」で書く。}}
                        } \mapsto (x, s)
                    \end{equation}
                    が diffeo である
                    (写像として well-defined に定まることはすぐ後で確かめる)\footnote{
                        このように定めた写像$\varphi_\alpha$が diffeo かどうか
                        (とくに{\smooth}かどうか) は
                        他の条件からはおそらく導かれない気がするので (\TODO{本当に?})、
                        独立な条件として与えておくことにする。
                        \TODO{cf. \url{https://math.stackexchange.com/questions/2930299/trivialization-from-a-smooth-frame}}
                    }。
            \end{enumerate}
    \end{enumerate}
    ここで
    \begin{itemize}
        \item $\varphi_\alpha$を$U_\alpha$上の$P$の
            \term{局所自明化}[local trivialization]{局所自明化}[きょくしょじめいか]
            という。
    \end{itemize}
\end{definition}

\begin{lemma}[$G$-torsor の特徴付け]
    $G$を群、
    $X$を空でない集合とし、
    $G$は$X$に右から作用しているとする。
    このとき次は同値である:
    \begin{enumerate}
        \item $G$の作用が単純推移的である。
        \item 写像
            \begin{equation}
                \theta \colon X \times G \to X \times X,
                \quad
                (x, g) \mapsto (x.g, x)
            \end{equation}
            が全単射である\footnote{
                写像$\theta$を shear map といい、
                shear map が全単射のとき$X$を$G$-torsor という。
            }。
    \end{enumerate}
    したがって、とくに上の定義の$\varphi_\alpha$が確かに写像として定まる。
\end{lemma}

\begin{proof}
    \begin{alignat}{1}
        \theta \colon \text{ 全射}
            &\iff \forall x, y \in X \; \exists g \in G \; [x.g = y] \\
            &\iff \text{$G$の作用が推移的} \\
        \theta \colon \text{ 単射}
            &\iff \forall x \in X \;
                \forall g, g' \in G \;
                [x.g = x.g' \implies g = g'] \\
            &\iff \forall x \in X \;
                \forall g, g' \in G \;
                [x = x.g'g^{-1} \implies g'g^{-1} = 1] \\
            &\iff \forall x \in X \;
                \forall g \in G \;
                [x = x.g \implies g = 1] \\
            &\iff \text{$G$の作用が自由}
    \end{alignat}
\end{proof}

\begin{definition}[変換関数]
    $M$を多様体、
    $p \colon P \to M$を主$G$束とすると、
    主$G$束の定義より、$M$の open cover $\{U_\alpha\}_{\alpha \in A}$であって
    各$U_\alpha$上に切断
    $\sigma_\alpha \colon U_\alpha \to p^{-1}(U_\alpha)$
    を持つものがとれる。
    各$\alpha, \beta \in A, \; U_\alpha \cap U_\beta \neq \emptyset$
    に対し、
    写像$\psi_{\alpha\beta} \colon U_\alpha \cap U_\beta \to G$を
    $x \in U_\alpha \cap U_\beta$を
    $\sigma_\beta(x) = \sigma_\alpha(x) . s$なる$s \in G$
    に写す写像、すなわち
    \begin{equation}
        x \overset{\sigma_\beta}{\mapsto} \sigma_\beta(x) = \sigma_\alpha(x) . s
            \overset{
                \substack{\sigma_\alpha \text{ より定まる} \\ \text{局所自明化}}
            }{\mapsto} (x, s)
            \overset{\mathrm{pr}_2}{\mapsto} s
    \end{equation}
    で定めると、これは{\smooth}である。
    {\smooth}写像の族$\{ \psi_{\alpha\beta} \}$を、
    切断の族$\{ \sigma_\alpha \}$から定まる
    $P$の\term{変換関数}[transition function]{変換関数}[へんかんかんすう]という。
\end{definition}

% ------------------------------------------------------------
%
% ------------------------------------------------------------
\section{ベクトル束と主ファイバー束の同伴}

\subsection{ベクトル束から主ファイバー束へ}

多様体上のランク$r$ベクトル束が与えられると、
フレーム束とよばれる主$\GL(r, \R)$束を構成できる。

\TODO{フレーム束はフレーム多様体をファイバーとする主$\GL(r, \R)$束?}

\TODO{フレーム束の主ファイバー束構造は全単射により誘導する?}

\begin{definition}[フレーム束]
    $M$を$n$次元多様体、
    $E \to M$をランク$r$ベクトル束とする。
    $M$の atlas
    $\{ (U_\alpha, \psi_\alpha) \}_{\alpha \in A}$であって、
    各$\alpha$に対して$U_\alpha$上の$E$の局所自明化$\rho_\alpha$が存在するものがとれる。
    \begin{innerproof}
        各$x \in M$に対し、
        多様体の定義とベクトル束の定義より、
        $x$の$M$における開近傍$V_x, W_x$であって
        $V_x$を定義域とするチャートが存在し、
        かつ$W_x$上の$E$の局所自明化が存在するようなものがとれる。
        そこで$U_x \coloneqq V_x \cap W_x$とおけば
        $\{ U_x \}_{x \in M}$が求める atlas となる。
    \end{innerproof}
    $E$の局所自明化の族$\{ \rho_\alpha \}$により定まる
    $E$の変換関数を$\{ \rho_{\alpha\beta} \}$とおく。
    $E$の\term{フレーム束}[frame bundle]{フレーム束}[ふれーむそく]
    とよばれる主$\GL(r, \R)$束
    $p \colon P \to M$を次のように構成する:
    \begin{enumerate}
        \item 各$x \in M$に対し、集合$P_x$を
            \begin{equation}
                P_x \coloneqq \{
                    u \colon \R^r \to E_x
                    \mid
                    \text{$u$は線型同型}
                \}
            \end{equation}
            で定める。
            $P_x$は$E_x$の基底全体の集合とみなせる。
        \item $P_x$らの disjoint union を
            \begin{equation}
                P \coloneqq \coprod_{x \in M} P_x
            \end{equation}
            とおく。
        \item 射影$p \colon P \to M$を
            \begin{equation}
                p((x, u)) \coloneqq x 
            \end{equation}
            で定義する。
        \item $\GL(r, \R)$の$P$への右作用$\beta$を
            次のように定める:
            \begin{equation}
                \beta \colon P \times \GL(r, \R) \to P,
                \quad
                ((x, u), s) \mapsto (x, u \circ s)
            \end{equation}
        \item 各$\alpha \in A$に対し、
            $U_\alpha$上の$E$の局所自明化$\rho_\alpha$をひとつ選び、
            それにより定まる$E$のフレームを
            $e_1^{(\alpha)}, \dots, e_r^{(\alpha)}$とおく。
            写像$\sigma_\alpha \colon U_\alpha \to p^{-1}(U_\alpha)$を
            次のように定める:
            \begin{itemize}
                \item 各$x \in U_\alpha$に対し、
                    $E_x$の基底$e_1^{(\alpha)}(x), \dots, e_r^{(\alpha)}(x)$により
                    定まる線型同型$\R^r \to E_x$を
                    一時的な記号で$\sigma_\alpha(x)_2$と書く。
                \item $\sigma_\alpha(x) \coloneqq (x, \sigma_\alpha(x)_2)$と定める。
                    記号の濫用で$\sigma_\alpha(x)_2$も$\sigma_\alpha(x)$と書く。
            \end{itemize}
        \item 写像$\varphi_\alpha$を
            \begin{equation}
                \varphi_\alpha
                    \colon p^{-1}(U_\alpha) \to U_\alpha \times \GL(r, \R),
                    \quad
                    (x, \sigma_\alpha(x) \circ s) \mapsto (x, s)
            \end{equation}
            と定める。
            ただし、$(x, \sigma_\alpha(x) \circ s)$から
            $s$が一意に定まることは
            $s = \sigma_\alpha(x)^{-1} \circ \sigma_\alpha(x) \circ s$
            と表せることよりわかる。
            また、$\varphi_\alpha$は明らかに可逆である。
        \item 写像族$\{ \varphi_\alpha \}$を用いて
            $P$に多様体構造が入る (このあとすぐ示す)。
        \item $p \colon P \to M$は、
            $\{ \sigma_\alpha \colon U_\alpha \to p^{-1}(U_\alpha) \}$
            を切断の族、
            これにより定まる変換関数を$\{ \rho_{\alpha\beta} \}$として
            $M$上の主$\GL(r, \R)$束となる (このあとすぐ示す)。
    \end{enumerate}
    $P$は$E$に\term{同伴する}[associated]{同伴する}[どうはんする]
    主ファイバー束と呼ばれる。
\end{definition}

\begin{proof}
    $\GL(r, \R) = \R^{r^2}$と同一視する。
    まず$P$に多様体構造が入ることを示す。
    $M$の atlas $\{ (U_\alpha, \psi_\alpha) \}$は、
    小さい範囲に制限した chart、すなわち
    \begin{equation}
        (U'_\alpha, \psi_\alpha|_{U'_\alpha})
        \quad
        (\alpha \in A, \; U'_\alpha \opensubset U_\alpha)
    \end{equation}
    をすべて含むとしてよい。
    写像族$\{ \Phi_\alpha \colon p^{-1}(U_\alpha) \to \R^{n + r^2} \}$を
    \begin{equation}
        \begin{tikzcd}
            p^{-1}(U_\alpha)
                \ar{r}{\varphi_\alpha}
                \ar[bend right=30, end anchor=south west]{rr}[swap]{\Phi_\alpha}
                & U_\alpha \times \GL(r, \R)
                \ar{r}{\psi_\alpha \times \id}
                & \psi_\alpha(U_\alpha) \times \R^{r^2}
                \subset \R^{n + r^2}
        \end{tikzcd}
    \end{equation}
    を可換にするものとして定める。
    $P$に$\{ \Phi_\alpha \}$を atlas とする多様体構造が入ることを示すため、
    Smooth Manifold Chart Lemma (\cref{lemma:smooth-manifold-chart-lemma})
    の条件を確認する。
    $\varphi_\alpha$が可逆であることと
    $\psi_\alpha$が$M$の chart であることから、
    $\Phi_\alpha$は$\R^{n + r^2}$の開部分集合
    $\psi_\alpha(U_\alpha) \times \R^{r^2}$への全単射である。
    よって (i) が満たされる。

    各$\alpha, \beta \in A$に対し
    $\psi_\alpha, \psi_\beta$が$M$の chart であることから
    \begin{align}
        \Phi_\alpha(p^{-1}(U_\alpha) \cap p^{-1}(U_\beta))
            = \psi_\alpha(U_\alpha \cap U_\beta) \times \R^{r^2} \\
        \Phi_\beta(p^{-1}(U_\alpha) \cap p^{-1}(U_\beta))
            = \psi_\beta(U_\alpha \cap U_\beta) \times \R^{r^2}
    \end{align}
    はいずれも$\R^{n + r^2}$の開部分集合である。
    よって (ii) が満たされる。

    各$\alpha, \beta \in A$に対し
    合成写像$\varphi_\beta \circ \varphi_\alpha^{-1}$は
    \begin{equation}
        \begin{tikzcd}
            (U_\alpha \cap U_\beta) \times \GL(r, \R)
                \ar{r}{\varphi_\alpha^{-1}}
                & \pi^{-1}(U_\alpha \cap U_\beta)
                \ar{r}{\varphi_\beta}
                & (U_\alpha \cap U_\beta) \times \GL(r, \R) \\[-1em]
            (x, s)
                \ar[mapsto]{r}
                & (x, \sigma_\alpha(x) \circ s)
                \ar[mapsto]{r}
                & (x, \sigma_\beta(x)^{-1} \circ \sigma_\alpha(x) \circ s)
        \end{tikzcd}
    \end{equation}
    という対応を与えるが、
    ここで$\sigma_\beta(x)^{-1} \circ (\sigma_\alpha(x)) \circ s$は
    $(x, s)$に関し{\smooth}である。
    \begin{innerproof}
        $s$を右から合成する演算は
        Lie 群$\GL(r, \R)$における積なので{\smooth}である。
        そこで$\sigma_\beta(x)^{-1} \circ \sigma_\alpha(x)$について考える。
        いま各$x \in U_\alpha \cap U_\beta$に対し
        \begin{equation}
            \begin{tikzcd}
                \R^r \ar{rr}{\sigma_\beta(x)^{-1} \circ \sigma_\alpha(x)}
                    \ar{dr}[swap]{\sigma_\beta(x)}
                    & & \R^r \ar{dl}{\sigma_\alpha(x)} \\
                & E_x
            \end{tikzcd}
        \end{equation}
        は可換であるが、
        $\sigma_\alpha, \sigma_\beta$は定め方から
        $E$の局所自明化の$E_x$への制限$\rho_\alpha(x), \rho_\beta(x)$の逆写像である。
        よって写像
        \begin{equation}
            U_\alpha \cap U_\beta \to \GL(r, \R),
            \quad
            x \mapsto \sigma_\beta(x)^{-1} \circ \sigma_\alpha(x)
        \end{equation}
        は$E$の変換関数$\rho_{\beta\alpha}$に他ならず、
        したがってこれは{\smooth}である。
        よって、$\sigma_\beta(x)^{-1} \circ (\sigma_\alpha(x)) \circ s$は
        $(x, s)$に関し{\smooth}である。
    \end{innerproof}
    したがって
    \begin{equation}
        \Phi_\beta \circ \Phi_\alpha^{-1}
            = (\psi_\beta \times \id) \circ \varphi_\beta
                \circ \varphi_\alpha^{-1}
                \circ (\psi_\alpha \times \id)^{-1}
    \end{equation}
    は$\Phi_\alpha(p^{-1}(U_\alpha) \cap p^{-1}(U_\beta))$上{\smooth}である。
    よって (iii) が満たされる。

    $\{ (U_\alpha, \psi_\alpha) \}$は
    小さい範囲に制限した chart をすべて含むことから
    明らかに (iv) が満たされる。

    以上で Smooth Manifold Chart Lemma の条件が確認できた。
    したがって$P$は
    $\{ (p^{-1}(U_\alpha), \Phi_\alpha) \}$を atlas として多様体となる。

    つぎに、$P$は
    $\{ \sigma_\alpha \colon U_\alpha \to p^{-1}(U_\alpha) \}$
    を切断の族として
    $M$上の主$\GL(r, \R)$束となることを示す。
    そのためには次を示せばよい:
    \begin{enumerate}
        \item $p$が{\smooth}であること
        \item 作用$\beta$がファイバーを保つこと
        \item 作用$\beta$がファイバー上単純推移的であること
        \item 作用$\beta$が{\smooth}であること
        \item $\sigma_\alpha$が$U_\alpha$上の$P$の切断となること
        \item 主ファイバー束の定義の局所自明性が満たされること
        \item $\{ \sigma_\alpha \}$により定まる$P$の変換関数が
            $\{ \rho_{\alpha\beta} \}$であること
    \end{enumerate}
    ここで、$\varphi_\alpha$らは diffeo である。実際、図式
    \begin{equation}
        \begin{tikzcd}
            p^{-1}(U_\alpha)
                \ar{r}{\varphi_\alpha}
                \ar[bend right=30, end anchor=south west]{rr}[swap]{\Phi_\alpha}
                & U_\alpha \times \GL(r, \R)
                \ar{r}{\psi_\alpha \times \id}
                & \psi_\alpha(U_\alpha) \times \R^{r^2}
                \subset \R^{n + r^2}
        \end{tikzcd}
    \end{equation}
    が可換であることと$\psi_\alpha \times \id, \; \Phi_\alpha$が
    diffeo であることから従う。

    $p$が{\smooth}であることは
    各点の近傍での{\smooth}性を示せばよいが、これは
    各$(x, u) \in P$に対し$p^{-1}(U_\alpha)$が開近傍となるような
    $\alpha \in A$がとれて
    \begin{equation}
        \begin{tikzcd}
            p^{-1}(U_\alpha)
                \ar{rd}[swap]{p}
                \ar{r}{\Phi_\alpha}
                & p^{-1}(U_\alpha) \times \R^{n + r^2}
                \ar{d}{\mathrm{pr}_1} \\
            & P
        \end{tikzcd}
    \end{equation}
    が可換となることから従う。

    $\GL(r, \R)$の$P$への作用
    \begin{equation}
        \beta((x, u), s)
            = (x, u \circ s)
    \end{equation}
    がファイバーを保つことは定義から明らか。

    $\beta$がファイバー$P_x = p^{-1}(x) \; (x \in M)$上単純推移的であることは、
    shear map
    \begin{equation}
        P_x \times \GL(r, \R) \to P_x \times P_x,
        \quad
        ((x, u), s) \mapsto ((x, u \circ s), (x, u))
    \end{equation}
    が逆写像
    \begin{equation}
        P_x \times P_x \to P_x \times \GL(r, \R),
        \quad
        ((x, t), (x, u)) \mapsto ((x, u), u^{-1} \circ t)
    \end{equation}
    を持つことから従う。

    $\beta$が{\smooth}であることを示す。
    $(x, u) \in P$の近傍$U_\alpha$上で
    \begin{equation}
        (x, u) = (x, \sigma_\alpha(x) \circ t)
        \quad
        (t \in \GL(r, \R))
    \end{equation}
    の形に書けることに注意すれば、
    \begin{alignat}{1}
            &((x, u), s) \in p^{-1}(U_\alpha) \times \GL(r, \R) \\
        \overset{
            \mathclap{\id \times (\mathrm{pr}_2 \circ \varphi_\alpha)}
        }{\mapsto} \qquad
            &((x, u), s, t)
            \in p^{-1}(U_\alpha) \times \GL(r, \R) \times \GL(r, \R) \\
        \overset{\mathclap{\text{$\GL(r, \R)$での積}}}{\mapsto} \qquad
            &((x, u), ts)
            \in p^{-1}(U_\alpha) \times \GL(r, \R) \\
        \overset{p}{\mapsto} \qquad
            &(x, ts)
            \in U_\alpha \times \GL(r, \R) \\
        \overset{\mathclap{\varphi_\alpha^{-1}}}{\mapsto} \qquad
            &(x, \sigma_\alpha(x) \circ ts)
            = (x, u \circ s)
            \in p^{-1}(U_\alpha)
    \end{alignat}
    の各写像が{\smooth}であることから、
    $\beta$は$U_\alpha$上{\smooth}であることがわかる。
    したがって$\beta$は{\smooth}である。

    $\sigma_\alpha$が$U_\alpha$上の$P$の切断となることを示す。
    $p \circ \sigma_\alpha(x) = x$となることは定義から明らか。
    {\smooth}性は
    \begin{equation}
        \sigma_\alpha(x)
            = \varphi_\alpha^{-1}(x, 1)
    \end{equation}
    よりわかる。
    したがって$\sigma_\alpha$は$U_\alpha$上の$P$の切断である。
    さらに$\varphi_\alpha$の定義と$\varphi_\alpha$が diffeo であることから
    主ファイバー束の定義の局所自明性も満たされる。

    最後に、$x \in U_\alpha \cap U_\beta, \; \alpha, \beta \in A$に対し
    \begin{equation}
        \sigma_\beta(x)
            = \sigma_\alpha(x) \circ \sigma_\alpha^{-1} \circ \sigma_\beta(x)
            = \sigma_\alpha(x) \circ \rho_{\alpha\beta}(x)
    \end{equation}
    が成り立つことから、
    $\{ \sigma_\alpha \}$により定まる$P$の変換関数は
    $\{ \rho_{\alpha\beta} \}$である。

    以上で$P$は
    $\{ \sigma_\alpha \}$を切断の族とし、
    これにより定まる$P$の変換関数を
    $\{ \rho_{\alpha\beta} \}$として
    $M$上の主$\GL(r, \R)$束となることが示せた。
\end{proof}

\begin{example}[構造群の縮小]
    $E$をベクトル束、
    $g$を$E$の内積とする。
    フレーム束の定義の$P_x$を
    \begin{equation}
        Q_x \coloneqq \{ u \colon \R^r \to E_x
            \mid u \text{ は線型同型かつ内積を保つ}
        \}
    \end{equation}
    に置き換えると、$Q$は
    直交群$O(r)$を構造群とする$M$上の主束となる。
    このとき$Q$は$P$の部分束であり、
    $Q$は$P$の構造群$\GL(r, \R)$を$O(r)$に
    \term{縮小}[reduction]{縮小}[しゅくしょう]
    して得られたという。
\end{example}

\subsection{主ファイバー束からベクトル束へ}
\label[subsection]{subsec:principal-fiber-bundle-to-vector-bundle}

逆に主$G$束$P$と
表現$\rho \colon G \to \GL(r, \R)$が与えられると、
ランク$r$ベクトル束$E$が構成できる。

\begin{definition}[同伴するベクトル束]
    $M$を多様体、$P \to M$を主$G$束、
    $\rho \colon G \to \GL(r, \R)$を Lie 群の表現とする。
    直積多様体$P \times \R^r$への
    $G$の{\smooth}右作用を
    \begin{equation}
        (P \times \R^r) \times G \to P \times \R^r,
        \quad
        ((u, y), s) \mapsto (u.s, \rho(s)^{-1} y)
    \end{equation}
    で定め、軌道空間$(P \times \R^r) / G$を
    \begin{equation}
        P \times_\rho \R^r
    \end{equation}
    と書く。
    このとき、$P \times_\rho \R^r$は
    $M$上のベクトル束となり、
    $P$のある変換関数$\{ \psi_{\alpha\beta} \}$に対し
    $\{ \rho \circ \psi_{\alpha\beta} \}$が
    $P \times_\rho \R^r$の変換関数のひとつとなる
    (このあとすぐ示す)。
    これを$P$に
    \term{同伴する}[associated]{同伴する}[どうはんする]
    ベクトル束という。
\end{definition}

\begin{proof}
    $P \times_\rho \R^r$が$M$上のベクトル束になることを、
    Vector Bundle Chart Lemma を用いて示す。
   標準射影$P \to M$および
    $P \times \R^r \to P \times_\rho \R^r$を
    それぞれ$p, q$とおく。

    まず射影を構成する。図式
    \begin{equation}
        \begin{tikzcd}
            P \times \R^r
                \ar{d}[swap]{\mathrm{pr}_1}
                \ar{r}{q}
                & P \times_\rho \R^r
                \ar[dashed]{d}{\pi} \\
            P \ar{r}[swap]{p}
                & M
        \end{tikzcd}
    \end{equation}
    において、$p \circ \mathrm{pr}_1$は$q$のファイバー上定値である。
    \begin{innerproof}
        $u \in P_x, \; u' \in P_{x'} \; (x, x' \in M),
        \; y, y' \in \R^r$について
        $q(u, y) = q(u', y')$ならば、
        $q$の定義から
        ある$s \in G$が存在して
        $(u, y) = (u' . s, \rho(s)^{-1} y')$が成り立ち、
        とくに$u = u' . s$だが、
        $G$の$P$への作用がファイバーを保つことから
        $x = x'$が成り立つ。
    \end{innerproof}
    したがって
    写像$\pi \colon P \times_\rho \R^r \to M$が誘導される。
    このとき$p \circ \mathrm{pr}_1$が全射であることより
    $\pi$も全射である。

    つぎに$P \times_\rho \R^r$の局所自明化を構成する。
    $P$の切断の族$\{ \sigma_\alpha \colon U_\alpha \to P \}_{\alpha \in A}$
    であって$\bigcup U_\alpha = P$なるものをひとつ選ぶ。
    これにより定まる$P$の局所自明化の族を$\{ \varphi_\alpha \}$とおき、
    さらにこれにより定まる$P$の変換関数を$\{ \psi_{\alpha\beta} \}$とおく。
    このとき、各$\alpha \in A$に対し図式
    \begin{equation}
        \begin{tikzcd}[column sep=large]
            U_\alpha  \times \R^r
                \ar[dashed]{drr}
                \ar{r}{\substack{(x, y) \\ \; \mapsto (x, 1, y)}}
                & U_\alpha \times G \times \R^r
                \ar{r}{\varphi_\alpha^{-1} \times \id}
                & p^{-1}(U_\alpha) \times \R^r
                \ar{d}{q} \\
            && p^{-1}(U_\alpha) \times_\rho \R^r
                = \pi^{-1}(U_\alpha)
        \end{tikzcd}
    \end{equation}
    の破線部の写像は全単射である。
    \begin{innerproof}
        $(u, y), (u', y') \in U_\alpha \times \R^r$について
        \begin{alignat}{1}
                &q(\varphi_\alpha^{-1}(u, 1), y)
                    = q(\varphi_\alpha^{-1}(u', 1), y') \\
            \iff
                &\exists s \in G
                \quad \text{s.t.} \quad
                \begin{cases}
                    \varphi_\alpha^{-1}(u, 1) = \varphi_\alpha^{-1}(u', 1) . s \\
                    y = \rho(s)^{-1} y'
                \end{cases} \\
            \iff
                &\exists s \in G
                \quad \text{s.t.} \quad
                \begin{cases}
                    \varphi_\alpha^{-1}(u, 1) = \varphi_\alpha^{-1}(u', s) \\
                    y = \rho(s)^{-1} y'
                \end{cases} \\
            \iff
                &\exists s \in G
                \quad \text{s.t.} \quad
                \begin{cases}
                    (u, 1) = (u', s) \\
                    y = \rho(s)^{-1} y'
                \end{cases} \\
            \iff
                &\begin{cases}
                    u = u' \\
                    y = y'
                \end{cases}
        \end{alignat}
    \end{innerproof}
    ただし、図式の右下が
    $p^{-1}(U_\alpha) \times_\rho \R^r = \pi^{-1}(U_\alpha)$であることは
    次のようにしてわかる。
    \begin{innerproof}
        $(\subset)$ \quad
        \begin{align}
            \pi(p^{-1}(U_\alpha) \times_\rho \R^r)
                &= \pi \circ q(p^{-1}(U_\alpha) \times \R^r) \\
                &= p \circ \mathrm{pr}_1 (p^{-1}(U_\alpha) \times \R^r) \\
                &= p \circ p^{-1}(U_\alpha) \\
                &\subset U_\alpha
        \end{align}
        より$p^{-1}(U_\alpha) \times_\rho \R^r \subset \pi^{-1}(U_\alpha)$である。

        \noindent
        $(\supset)$ \quad
        $(u, y) \in p^{-1}(U_\alpha) \times \R^r$について
        $\pi(q(u, y)) \in U_\alpha$ならば
        \begin{equation}
            p(u) = p \circ \mathrm{pr}_1(u, y) \in U_\alpha
        \end{equation}
        だから$(u, y) \in p^{-1}(U_\alpha) \times \R^r$、
        したがって$q(u, y) \in p^{-1}(U_\alpha) \times_\rho \R^r$である。
    \end{innerproof}
    そこで、破線矢印の逆向きの写像$\pi^{-1}(U_\alpha) \to U_\alpha \times \R^r$を
    $\Phi_\alpha$とおく。
    各$x \in M$に対し、
    $x \in U_\alpha$なる$\alpha \in A$をひとつ選べば、
    $\Phi_\alpha(x) \colon \pi^{-1}(x) \to \{ x \} \times \R^r = \R^r$
    は可逆である。
    実際、
    \begin{equation}
        \{ x \} \times \R^r \to \pi^{-1}(x),
        \quad
        (x, y) \mapsto q(\varphi^{-1}(x, 1), y)
    \end{equation}
    が逆写像を与える。
    そこで、この 1:1 対応により$\pi^{-1}(x)$に
    $r$次元$\R$-ベクトル空間の構造を入れる。

    最後に、$U_\alpha \cap U_\beta \neq \emptyset$なる$\alpha, \beta \in A$と
    $(x, y) \in (U_\alpha \cap U_\beta) \times \R^r$に対し
    \begin{alignat}{1}
        \Phi_\alpha \circ \Phi_\beta^{-1} (x, y)
            = (x, \rho \circ \psi_{\alpha\beta} y)
    \end{alignat}
    が成り立つ。
    \begin{innerproof}
        まず
        \begin{alignat}{1}
            \Phi_\alpha \circ \Phi_\beta^{-1} (x, y)
                &= \Phi_\alpha(q(\varphi_\beta^{-1}(x, 1), y)) \\
                &= \Phi_\alpha(q(\varphi_\beta(x)^{-1}(1), y))
        \end{alignat}
        である。このとき
        \begin{align}
            (\varphi_\beta(x)^{-1}(1), y)
            &= (\sigma_\beta(x), y) \\
            &= (\sigma_\alpha(x) . \psi_{\alpha\beta}(x), y)
        \end{align}
        が成り立つから
        \begin{align}
            \Phi_\alpha(q(\varphi_\beta(x)^{-1}(1), y))
                &= \Phi_\alpha(q(\sigma_\alpha(x) . \psi_{\alpha\beta}(x), y)) \\
                &= \Phi_\alpha(
                    \sigma_\alpha(x),
                    \rho(\psi_{\alpha\beta}(x)^{-1})^{-1} y
                ) \\
                &= \Phi_\alpha(
                    \varphi_\alpha(x)^{-1}(1),
                    \rho(\psi_{\alpha\beta}(x)) y
                ) \\
                &= \Phi_\alpha \circ \Phi_\alpha^{-1}(
                    x,
                    \rho(\psi_{\alpha\beta}(x)) y
                ) \\
                &= (x, \rho \circ \psi_{\alpha\beta} y)
        \end{align}
        となる。
    \end{innerproof}
    $\rho, \psi_{\alpha\beta}$はいずれも{\smooth}だから
    $\rho \circ \psi_{\alpha\beta} \colon U_\alpha \cap U_\beta \to \GL(r, \R)$
    も{\smooth}である。

    以上で Vector Bundle Chart Lemma の条件が確認できた。
    したがって$P \times_\rho \R^r$は$M$上のベクトル束となり、
    $\{ \Phi_\alpha \}$は$P \times_\rho \R^r$の局所自明化の族となり、
    これにより定まる$P \times_\rho \R^r$の変換関数は
    $\{ \rho \circ \psi_{\alpha\beta} \}$である。
\end{proof}

\begin{example}[ベクトル束のフレーム束に同伴するベクトル束]
    $E \to M$をランク$r$ベクトル束、
    $\{ \psi_{\alpha\beta} \}$を$E$の変換関数、
    $P$を$E$から構成されたフレーム束とする。
    フレーム束の定義より、
    $\{ \psi_{\alpha\beta} \}$も$P$の変換関数であった。
    よって表現$\rho \colon \GL(r, \R) \to \GL(r, \R)$を
    恒等写像とすれば、
    $P \times_\rho \R^r$の変換関数は
    $\{ \rho \circ \psi_{\alpha\beta} = \psi_{\alpha\beta} \}$となり、
    $P \times_\rho \R^r$が$E$に一致することがわかる。
\end{example}

\begin{example}[直和束]
    \TODO{}
\end{example}

\begin{example}[テンソル積束]
    \TODO{$\rho(s) = s \otimes s$}
\end{example}

\begin{example}[双対束]
    \TODO{$\rho(s) = \up{t}s^{-1}$}
\end{example}


% ============================================================
%
% ============================================================
\chapter{アファイン接続}

接続の概念に慣れるため、
まずは多様体の接束の接続であるアファイン接続からはじめる。

\section{アファイン接続}
\label[section]{sec:affine-connection}

\begin{definition}[ベクトル束の接続]
    $M$を多様体とする。
    $M$の\term{アファイン接続}[affine connection]{アファイン接続}[あふぁいんせつぞく]とは、
    $\R$-線型写像$A^0(TM) \to A^1(TM)$であって、
    Leibniz の公式
    \begin{equation}
        \nabla(fY) = df \otimes Y + f \nabla Y
            \quad (f \in A^0(M),\; Y \in A^0(TM))
    \end{equation}
    をみたすものである。
    各$Y \in A^0(M),\; X \in \frakX(M)$に対し、
    $\nabla Y (X) \in A^0(TM)$を$\nabla_X Y$とも書き、
    $Y$の$X$方向の\term{共変微分}[covariant derivative]{共変微分}[きょうへんびぶん]と呼ぶ。
\end{definition}

\begin{example}[アファイン接続の例]
    ~
    \begin{itemize}
        \item \TODO{座標を明示せよ} $\R^n$のアファイン接続$\wb{\nabla}$を
            \begin{equation}
                \wb{\nabla}_X Y
                    \coloneqq X(Y^1) \deldel{x^1} + \dots + X(Y^n) \deldel{x^n}
                    = X(Y^i) \deldel{x^i}
            \end{equation}
            で定めることができる。
            $\wb{\nabla}$を
            \term{Euclid 接続}[Euclidean connection]{Euclid 接続}[Euclid せつぞく]
            という。
            $X$を書かずに表せば
            \begin{equation}
                \wb{\nabla} Y
                    = dY^i \deldel{x^i}
            \end{equation}
            となる。
            \cref{def:vector-bundle-connection}の Leibniz の公式の成立を確かめると、
            \begin{alignat}{1}
                \wb{\nabla}(fY)
                    &= d(fY^i) \deldel{x^i} \\
                    &= d(fY^i) \otimes \deldel{x^i}
                        \quad (\text{$TM$-値微分形式の同一視}) \\
                    &= (Y^i \,df + f \,dY^i) \otimes \deldel{x^i} \\
                    &= df \otimes Y + f \cdot \wb{\nabla} Y
            \end{alignat}
            より確かに成り立つ。
            \TODO{接続係数が0であることを述べる}
        \item \TODO{tangential connection}
    \end{itemize}
\end{example}

次に定義する接続形式とは、
局所フレームに関する接続の行列表示のようなものである。

\begin{definition}[接続形式]
    $M$を多様体、$U \opensubset M$、
    $(E_i)$を$U$上の$TM$の局所フレームとする。
    \begin{itemize}
        \item 各$j$に対し、
            \begin{equation}
                \nabla E_j = \omega^k_j E_k
            \end{equation}
            と表したときの$1$-形式$\omega^k_j$らの族$\omega \coloneqq (\omega^k_j)$を
            $\nabla$の\term{接続形式}[connection form]{接続形式}[せつぞくけいしき]という。
        \item {\smooth}関数$\Gamma_{ij}^k \colon U \to \R,$
            \begin{equation}
                \Gamma^k_{ij} \coloneqq \omega^k_j(E_i)
            \end{equation}
            を$\nabla$の
            \term{接続係数}[connection coefficient]{接続係数}[せつぞくけいすう]
            という。
            定義から明らかに、接続係数は
            \begin{equation}
                \nabla_{E_i} E_j = \Gamma^k_{ij} E_k
            \end{equation}
            をみたす。
    \end{itemize}
\end{definition}


% ------------------------------------------------------------
%
% ------------------------------------------------------------
\section{捩率と曲率}

この節では捩率テンソルと曲率テンソルを定義する。

\begin{definition}[捩率テンソル]
    $M$を多様体、
    $\nabla$を$M$のアファイン接続とする。
    このとき
    \begin{equation}
        T \colon \frakX(M) \times \frakX(M) \to \frakX(M),
        \quad
        (X, Y) \mapsto \frac{1}{2} (\nabla_X Y - \nabla_Y X - [X, Y])
    \end{equation}
    と定義すると
    $T$は交代$\smooth(M)$-双線型写像となる (このあと示す)。
    そこで$T$は$M$上の$TM$に値をもつ2次形式とみなせて、
    これを接続$\nabla$の
    \term{捩率テンソル}[torsion tensor]{捩率テンソル}[れいりつてんそる]
    という。
    捩率の値が$M$上恒等的に$0$であるとき、
    接続$\nabla$は
    \term{捩れなし}[torsion-free]{捩れなし}[ねじれなし]
    あるいは
    \term{対称}[symmetric]{対称}[たいしょう]であるという\footnote{
        「対称」という語は、
        接続が対称であるための必要十分条件
        $\Gamma^k_{ij} = \Gamma^k_{ji}$からきている
        \cite[p.121]{Lee18}。
    }。
\end{definition}

\begin{proof}
    \TODO{}
\end{proof}

\begin{definition}[曲率テンソル]
    $M$を多様体、
    $\nabla$を$M$のアファイン接続とする。
    このとき、
    \begin{equation}
        R \colon
            \frakX(M) \times \frakX(M) \times \frakX(M)
            \to \frakX(M),
        \quad
        (X, Y, Z)
            \mapsto \nabla_X \nabla_Y Z - \nabla_Y \nabla_X Z - \nabla_{[X, Y]} Z
    \end{equation}
    は$M$上の$(1, 3)$-テンソル場となり、これを
    接続$\nabla$の
    \term{曲率テンソル}[curvature tensor]{曲率テンソル}[きょくりつてんそる]
    という。
\end{definition}

捩率テンソルは、局所的には接続形式を用いて表せる。

\begin{proposition}[第1構造方程式]
    $e_1, \dots, e_n$を$TM$の局所フレーム、
    $\theta^1, \dots, \theta^n$をその双対フレームとする。
    捩率$T$は$A^2(TM)$の元だから
    \begin{equation}
        T = \sum_{i = 1}^n \Theta^i \otimes e_i
            \quad (\Theta^i \in A^2(M))
    \end{equation}
    と表せる。
    すると
    \begin{equation}
        \Theta^i(e_k, e_l) = d\theta^i(e_k, e_l) + \omega^i_j \wedge \theta^j (e_k, e_l)
    \end{equation}
    すなわち
    \begin{equation}
        \Theta^i = d\theta^i + \omega^i_j \wedge \theta^j
    \end{equation}
    が成り立つ。これをアファイン接続$\nabla$の
    \term{第1構造方程式}[first structure equation]{第1構造方程式}[だい1こうぞうほうていしき]
    という。
\end{proposition}

\begin{proof}
    \TODO{}
\end{proof}

Bianchi の第1恒等式は、
曲率テンソルの非対称性を捩率を用いて表すものである。

\begin{proposition}[Bianchi の第1恒等式]
    \begin{equation}
        R(X, Y)Z + R(Y, Z)X + R(Z, X)Y = 2(DT)(X, Y, Z)
    \end{equation}
    接続形式で書けば
    \begin{equation}
        \Omega^i_j \wedge \theta^j = d\Theta^i + \omega^i_j \wedge \Theta^j
    \end{equation}
    \TODO{}
\end{proposition}

\begin{proof}
    \TODO{}
\end{proof}

次に定義する曲率形式とは、
局所フレームに関する曲率テンソルの行列表示のようなものである。

\begin{definition}[曲率形式]
    $M$を多様体、$U \opensubset M$とし、
    $(E_i)$を$U$上の$TM$の局所フレームとする。
    各$j$に対し、
    \begin{equation}
        R(X, Y)(E_j) = \Omega^k_j(X, Y) E_k
    \end{equation}
    と表したときの$2$-形式$\Omega^k_j$らの族$\Omega \coloneqq (\Omega^k_j)$を
    $\nabla$の\term{曲率形式}[curvature form]{曲率形式}[きょくりつけいしき]という。
\end{definition}

曲率は、局所的には接続形式を用いて表せる。

\begin{proposition}[第2構造方程式]
    \label[proposition]{prop:second-structure-equation}
    $M$を多様体、$U \opensubset M$とし、
    $(E_i)$を$U$上の$TM$の局所フレームとする。
    このとき、曲率形式に関する方程式
    \begin{equation}
        \Omega^k_j
            = d\omega^k_j + \omega^k_i \wedge \omega^i_j
    \end{equation}
    が成り立つ。これを接続$\nabla$の
    \term{第2構造方程式}[second structure equation]{第2構造方程式}[だい2こうぞうほうていしき]
    という。
\end{proposition}

\begin{proof}
    \TODO{}
\end{proof}

Bianchi の第2恒等式は、
曲率テンソルの共変外微分が消えることを表す。

\begin{proposition}[Bianchi の第2恒等式]
    \begin{equation}
        DR = 0
    \end{equation}
    接続形式で書けば
    \begin{equation}
        d\Omega^\mu_\lambda
            - \Omega^\mu_\nu \wedge \omega^\nu_\lambda
            + \omega^\mu_\nu \wedge \Omega^\nu_\lambda
            = 0
    \end{equation}
    \TODO{}
\end{proposition}

\begin{proof}
    \TODO{}
\end{proof}

Ricci の恒等式は、
共変微分の非可換性を表すものである。

\begin{proposition}[Ricci の恒等式]
    \begin{equation}
        \frac{1}{2}(\nabla^2 K(X, Y) - \nabla^2 K(Y, X))
            = -R(X, Y) K + \nabla_{T(X, Y)} K
    \end{equation}
    \TODO{}
\end{proposition}

\begin{proof}
    \TODO{}
\end{proof}


% ------------------------------------------------------------
%
% ------------------------------------------------------------
\section{平行移動}

この節では測地線について述べた後、その一般化として平行の概念を導入する。

\begin{definition}[測地線]
    $M$を多様体とし、$\nabla$を$M$のアファイン接続とする。
    $M$上の曲線$\gamma$が$M$の\term{測地線}[geodesic]{測地線}[そくちせん]であるとは、
    $\gamma'$方向の$\gamma'$の共変微分が恒等的に$0$であること、すなわち
    \begin{equation}
        \nabla_{\gamma'} \gamma' \equiv 0
    \end{equation}
    が成り立つことをいう。
\end{definition}

\begin{example}[測地線の例]
    \TODO{}
\end{example}

\begin{definition}[平行]
    $M$を多様体とし、$\nabla$を$M$のアファイン接続とする。
    $\gamma$を$M$上の曲線とする。
    $\gamma$に沿ったテンソル場$V$が
    $\gamma$に沿って\term{平行}[parallel]{平行}[へいこう]であるとは、
    $\gamma'$方向の$V$の共変微分が恒等的に$0$であること、すなわち
    \begin{equation}
        \nabla_{\gamma'} V \equiv 0
    \end{equation}
    が成り立つことをいう。
\end{definition}

\begin{example}[平行なテンソル場の例]
    \TODO{}
    ~
    \begin{itemize}
        \item $\gamma$が測地線であるとは、
            その速度ベクトル場が$\gamma$自身に沿って平行であることと同値である。
    \end{itemize}
\end{example}

\begin{definition}[平行移動]
    $M$を多様体とし、$\nabla$を$M$のアファイン接続とする。
    さらに、$\gamma \colon I \to M$を$M$上の曲線とし、
    $t_0 \in I,\; v \in T_{\gamma'(t_0)} M$とする。
    このとき、$\gamma$に沿って平行なベクトル場$V$であって
    $V(t_0) = v$をみたすものがただひとつ存在し、
    このような$V$を$\gamma$に沿った$v$の
    \term{平行移動}[parallel transport]{平行移動}[へいこういどう]という。
\end{definition}




% ============================================================
%
% ============================================================
\chapter{ベクトル束の接続}

ベクトル束の接続について考える。

% ------------------------------------------------------------
%
% ------------------------------------------------------------
\section{ベクトル束の接続}

\begin{definition}[ベクトル束の接続]
    \label[definition]{def:vector-bundle-connection}
    $M$を多様体、
    $\pi \colon E \to M$をベクトル束とする。
    $E$の\term{接続}[connection]{接続}[せつぞく]とは、
    $\R$-線型写像$\nabla \colon A^0(E) \to A^1(E)$であって、
    Leibniz の公式
    \begin{equation}
        \nabla(f\xi) = df \otimes \xi + f \nabla\xi
            \quad (f \in A^0(M),\; \xi \in A^0(E))
    \end{equation}
    をみたすものである。
    各$\xi \in A^0(E),\; X \in \frakX(M)$に対し、
    $\nabla\xi(X) \in A^0(E)$を$\nabla_X\xi$とも書き、
    $\xi$の$X$方向の\term{共変微分}[covariant derivative]{共変微分}[きょうへんびぶん]と呼ぶ。
\end{definition}

定義からわかるように、
接続$\nabla$は$\smooth(M)$-線型ではないが、
2つの接続$\nabla, \nabla'$の差$\nabla - \nabla'$は$\smooth(M)$-線型である。
この事実はたとえば情報幾何学において、
$m$-接続と$e$-接続の差によって Amari-Chentsov テンソルを定義する際に用いられる。

\begin{proposition}[接続の差は$\smooth(M)$-線型]
    $M$を多様体、
    $E$を$M$上のベクトル束、
    $\nabla, \nabla'$を$E$の接続とする。
    このとき
    $\nabla - \nabla' \colon A^0(E) \to A^1(E), \;
        \xi \mapsto \nabla\xi - \nabla'\xi$
    は$\smooth(M)$-線型である。
\end{proposition}

\begin{proof}
    \begin{alignat}{1}
        (\nabla - \nabla')(f\xi)
            &= \nabla(f\xi) - \nabla'(f\xi) \\
            &= df \otimes \xi + f \nabla\xi - df \otimes \xi - f \nabla'\xi \\
            &= f (\nabla - \nabla')\xi
    \end{alignat}
    より従う。
\end{proof}

% ------------------------------------------------------------
%
% ------------------------------------------------------------
\section{接続形式}

接続形式を導入する。接続形式は、接続の座標表示にあたるものである。

\begin{definition}[接続形式]
    $M$を多様体、
    $E \to M$をランク$r$のベクトル束、
    $\nabla$を$E$の接続とする。
    さらに$U \opensubset M$、
    $\calE \coloneqq (e_1, \dots, e_r)$を$U$上の$E$のフレームとする。
    このとき、
    $U$上の$1$-形式の族$\omega = (\omega_\lambda^\mu)_{\lambda, \mu}$により
    \begin{equation}
        \nabla e_\lambda
            = \sum_{\mu} \omega_\lambda^\mu \otimes e_\mu
            \quad (\lambda = 1, \dots, r)
    \end{equation}
    と書ける。
    $\omega$をフレーム$\calE$に関する$\nabla$の
    \term{接続形式}[connection form]{接続形式}[せつぞくけいしき]という。
\end{definition}

もうひとつのフレームに関する接続形式を考えると、
ふたつの接続形式の間の変換規則が立ち現れる。

\begin{proposition}[接続形式の変換規則]
    上の定義の状況で、
    さらに$\calE' \coloneqq (e'_1, \dots, e'_r)$も$U$上の$E$のフレームとし、
    $\calE'$に関する$\nabla$の接続形式を$\omega'$とする。
    フレームの取り替えの行列$(a_\lambda^\mu)$は
    \begin{equation}
        e'_\lambda = \sum_{\mu} a_\lambda^\mu e_\mu
        \quad (a_\lambda^\mu \in A^0(U))
    \end{equation}
    とおく。
    このとき、接続形式の変換規則は
    \begin{equation}
        \omega' = a^{-1} \omega a + a^{-1} da
    \end{equation}
    となる。
\end{proposition}

\begin{proof}
    \TODO{}
\end{proof}

逆に、$\mathfrak{gl}(r, \R)$に値をもつ1-形式の族から接続を構成できる。

\begin{proposition}[接続形式から定まる接続]
    \label[proposition]{prop:connection-from-1-forms}
    $M$を多様体、
    $E \to M$をランク$r$のベクトル束とする。
    $\{ U_\alpha \}_{\alpha \in A}$を$M$の open cover であって
    各$U_\alpha$上で$g$に関するフレーム
    $\calE_\alpha = (e^{(\alpha)}_1, \dots, e^{(\alpha)}_r)$
    を持つものとする。
    さらに、各$U_\alpha$上の局所自明化$\varphi_\alpha$を
    $\calE_\alpha$から定め、
    変換関数を$\{ \psi_{\alpha\beta} \}$とおく。
    このとき、$\mathfrak{gl}(r, \R)$に値をもつ$1$-形式の族
    \begin{equation}
        \omega = \{ \omega_\alpha \}_{\alpha \in A}
    \end{equation}
    であって、変換規則
    \begin{equation}
        \omega_\beta
            = \psi_{\alpha\beta}^{-1} \omega_\alpha \psi_{\alpha\beta}
            + \psi_{\alpha\beta}^{-1} \, d \psi_{\alpha\beta}
            \quad
            \text{on $U_\alpha \cap U_\beta$}
    \end{equation}
    をみたすものが与えられたならば、
    次をみたす$E$の接続が構成できる:
    \begin{enumerate}
        \item 各フレーム$\calE_\alpha$に関する
            $\nabla$の接続形式は$\omega_\alpha$である。
    \end{enumerate}
\end{proposition}

\begin{proof}
    \TODO{}
\end{proof}

\TODO{大域的な与え方と接続形式による与え方}

\begin{definition}[直和束の接続]
    \TODO{}
\end{definition}

\begin{definition}[テンソル積束の接続]
    \TODO{}
\end{definition}

\begin{definition}[双対束の接続]
    \TODO{}
\end{definition}

\begin{definition}[引き戻し束の接続]
    \TODO{}
\end{definition}

% ------------------------------------------------------------
%
% ------------------------------------------------------------
\section{ベクトル束の共変外微分と曲率}

微分形式に対する外微分を一般化し、
ベクトル束に値をもつ微分形式に対し共変外微分とよばれる演算を定義する。
さらに共変外微分から曲率を定義する。

\begin{definition}[共変外微分]
    $M$を多様体、
    $E \to M$をベクトル束、
    $\nabla$を$E$の接続、
    $p \in \Z_{\ge 0}$とする。
    $\R$-線型写像$D \colon A^p(E) \to A^{p + 1}(E)$を
    \begin{equation}
        D(\theta \otimes \xi)
            \coloneqq d\theta \otimes \xi + \theta \wedge \nabla\xi
            \quad (\theta \in A^p(M), \; \xi \in A^0(E))
    \end{equation}
    で定め、$D$を
    \term{共変外微分}[covariant exterior derivative]{共変外微分}[きょうへんがいびぶん]
    という。
\end{definition}

\begin{remark}
    \label[remark]{remark:covariant-exterior-derivative-and-vector-fields}
    とくに$p = 1$のとき、
    $\varphi \in A^1(E)$に対し
    \begin{equation}
        (D\varphi)(X, Y)
            = \nabla_X (\varphi(Y))
            - \nabla_Y (\varphi(X))
            - \varphi([X, Y])
            \quad
            (X, Y \in \Gamma(TM))
    \end{equation}
    が成り立つ。
    これは通常の外微分の公式\cref{remark:exterior-derivative-and-vector-fields}
    の拡張になっている。
\end{remark}

共変外微分は次の性質をみたす。

\begin{proposition}[共変外微分の anti-derivation 性 (外積に関して)]
    $M$を多様体、
    $E \to M$をベクトル束、
    $\nabla$を$E$の接続、
    $D$を$\nabla$から定まる共変外微分とする。
    $p, q \in \Z_{\ge 0}$に対し
    \begin{equation}
        D(\theta \wedge \varphi)
            = d\theta \wedge \varphi + (-1)^p \theta \wedge D\varphi
            \quad
            (\theta \in A^p(M), \; \varphi \in A^q(E))
    \end{equation}
    が成り立つ\footnote{
        [小林]では$E$-値形式を表すときの$\xi$と$\theta$の順序が逆なので
        \begin{equation}
            D(\varphi \wedge \theta)
                = D\varphi \wedge \theta + (-1)^p \varphi \wedge d\theta
        \end{equation}
        という形になっている。
    }。
\end{proposition}

\begin{proof}
    \TODO{}
\end{proof}

ある性質を満たす双線型写像に対し、
共変外微分は anti-derivation 性をみたす。

\begin{proposition}[共変外微分の anti-derivation 性 (双線型写像に関して)]
    \label[proposition]{prop:signed-leibniz-rule}
    $M$を多様体、
    $E \to M, \; E' \to M$をベクトル束、
    $\nabla, \nabla'$をそれぞれ$E, E'$の接続、
    $D, D'$をそれぞれ$\nabla, \nabla'$から定まる共変外微分、
    $g \colon A^0(E) \times A^0(E') \to A^0(M)$を
    $\smooth(M)$-双線型写像とする。
    $D, D'$が条件
    \begin{equation}
        d(g(\xi, \eta)) = g(\nabla\xi, \eta) + g(\xi, \nabla'\eta)
            \quad
            (\xi \in A^0(E), \; \eta \in A^0(E'))
    \end{equation}
    をみたすならば\footnote{
        とくに$g$が$M$の Riemann 計量で
        $E = E' = TM$の状況でこの条件が成り立っているならば、
        $\nabla'$は$g$に関する$\nabla$の
        \term{双対接続}[dual connection]{双対接続}[そうついせつぞく]
        であるという。
    }、
    $p, q \in \Z_{\ge 0}$に対し
    \begin{equation}
        d(g(\xi, \eta))
            = g(D\xi, \eta) + (-1)^p g(\xi, D'\eta)
            \quad
            (\xi \in A^p(E), \; \eta \in A^q(E'))
    \end{equation}
    が成り立つ。
\end{proposition}

\begin{proof}
    Einstein の記法を用いる。
    $A^p(E), A^q(E')$の元はそれぞれ
    \begin{align}
        &\alpha \otimes \xi \in A^p(E)
            \quad (\alpha \in A^p(M), \; \xi \in A^0(E)) \\
        &\beta \otimes \eta \in A^q(E')
            \quad (\beta \in A^q(M), \; \eta \in A^0(E'))
    \end{align}
    の形の元の有限和で書けるから、
    このような形の元について示せば十分である。
    \begin{align}
        \nabla \xi = \alpha^i \otimes \xi_i,
            \quad
            (\alpha^i \in A^1(M), \; \xi_i \in A^0(E)) \\
        \nabla \eta = \beta^j \otimes \eta_j
            \quad
            (\beta^j \in A^1(M), \; \eta_j \in A^0(E'))
    \end{align}
    とおいておく (ただし、Einstein の記法を使うために共変・反変による
    添字の上下の慣例を一時的に無視している)。
    まず
    \begin{alignat}{1}
        &\quad d(g(\alpha \otimes \xi, \beta \otimes \eta)) \\
        &= d(g(\xi, \eta) \alpha \wedge \beta)
            \quad (\text{$\because$ ベクトル束値形式の内積の定義}) \\
        &= d(g(\xi, \eta)) \alpha \wedge \beta
            + g(\xi, \eta) d\alpha \wedge \beta
            + (-1)^p g(\xi, \eta) \alpha \wedge d\beta \\
        &= d(g(\xi, \eta)) \alpha \wedge \beta
            + g(\xi \otimes d\alpha, \eta \otimes \beta)
            + (-1)^p g(\xi \otimes \alpha, \eta \otimes d\beta)
            \label[equation]{eq:signed-leibniz-rule-1}
    \end{alignat}
    となる。ここで、第1項は
    \begin{alignat}{1}
        &\quad d(g(\xi, \eta)) \alpha \wedge \beta \\
        &= g(\nabla \xi, \eta) \alpha \wedge \beta
            + g(\xi, \nabla' \eta) \alpha \wedge \beta
            \quad (\text{$\because$ 命題の仮定}) \\
        &= g(\xi_i, \eta) \alpha^i \wedge \alpha \wedge \beta
            + g(\xi, \eta_j) \beta^j \wedge \alpha \wedge \beta \\
        &= g(\xi_i, \eta) \alpha^i \wedge \alpha \wedge \beta
            + (-1)^p g(\xi, \eta_j) \alpha \wedge \beta^j \wedge \beta \\
        &= g(\xi_i \otimes \alpha^i \wedge \alpha, \eta \otimes \beta)
            + (-1)^p g(\xi \otimes \alpha, \eta_j \beta^j \wedge \beta) \\
        &= g(\nabla \xi \wedge \alpha, \eta \otimes \beta)
            + (-1)^p g(\xi \otimes \alpha, \nabla' \eta \wedge \beta)
    \end{alignat}
    となる。
    したがって、\cref{eq:signed-leibniz-rule-1}より
    \begin{alignat}{1}
        d(g(\alpha \otimes \xi, \beta \otimes \eta))
            &= g(\nabla \xi \wedge \alpha, \eta \otimes \beta)
                + g(\xi \otimes d\alpha, \eta \otimes \beta) \\
            &\qquad \quad + (-1)^p g(\xi \otimes \alpha, \nabla' \eta \wedge \beta)
                + (-1)^p g(\xi \otimes \alpha, \eta \otimes d\beta) \\
            &= g(D(\xi \wedge \alpha), \eta \otimes \beta)
                + (-1)^p g(\xi \otimes \alpha, D'(\eta \wedge \beta))
    \end{alignat}
    が成り立つ。
\end{proof}

曲率を定義する。

\begin{definition}[曲率]
    $M$を多様体、
    $E \to M$をベクトル束、
    $\nabla$を$E$の接続、
    $D$を$\nabla$により定まる共変外微分とする。
    $R \coloneqq D^2$とおき、
    $R$を$\nabla$の
    \term{曲率}[curvature]{曲率}[きょくりつ]
    という。
\end{definition}

\begin{proposition}
    写像$R = D^2 \colon A^0(E) \to A^2(E)$は
    $A^2(\End E)$の元ともみなせる。
    \TODO{why?}
\end{proposition}

\begin{proof}
    \TODO{}
\end{proof}

曲率は、局所的には接続形式を用いて表せる。
ここで登場する構造方程式は、アファイン接続の場合の第2構造方程式
(\cref{prop:second-structure-equation})
に他ならない。
それでは第1構造方程式はどこにいったのかと気になるが、
一般の接続では捩率が定義できないから第1構造方程式にあたるものは登場しない。
Bianchi の恒等式に関しても同様である。

\begin{proposition}[構造方程式]
    \label[proposition]{prop:structure-equation}
    \begin{equation}
        \Omega^\mu_\lambda = d\omega^\mu_\lambda + \omega^\mu_\nu \wedge \omega^\nu_\lambda
    \end{equation}
    \TODO{}
\end{proposition}

\begin{proof}
    \TODO{}
\end{proof}

\begin{proposition}[Bianchi の恒等式]
    \begin{equation}
        DR = 0
    \end{equation}
    接続形式で書けば
    \begin{equation}
        d\Omega^\mu_\lambda
            - \Omega^\mu_\nu \wedge \omega^\nu_\lambda
            + \omega^\mu_\nu \wedge \Omega^\nu_\lambda
            = 0
    \end{equation}
    \TODO{}
\end{proposition}

\begin{proof}
    \TODO{}
\end{proof}

\begin{proposition}[Ricci の恒等式]
    共変外微分の公式\cref{remark:covariant-exterior-derivative-and-vector-fields}で
    $\varphi = D\xi, \; \xi \in A^0(E)$とおいて計算すると
    \begin{equation}
        R(X, Y) \xi
            = D(D\xi)(X, Y)
            = (\nabla_X \nabla_Y - \nabla_Y \nabla_X - \nabla_{[X, Y]}) \xi
    \end{equation}
    すなわち
    \begin{equation}
        R(X, Y) = \nabla_X \nabla_Y - \nabla_Y \nabla_X - \nabla_{[X, Y]}
    \end{equation}
    を得る。
    これを\term{Ricci の恒等式}[Ricci's identity]{Ricci の恒等式}[Ricci のこうとうしき]
    という。
\end{proposition}

\begin{proof}
    \TODO{}
\end{proof}

\begin{definition}[直和束の曲率]
    \TODO{}
\end{definition}

\begin{definition}[テンソル積束の曲率]
    \TODO{}
\end{definition}

\begin{definition}[双対束の曲率]
    \TODO{}
\end{definition}

\begin{definition}[引き戻し束の曲率]
    \TODO{}
\end{definition}



% ============================================================
%
% ============================================================
\chapter{主ファイバー束の接続}

前章ではベクトル束の接続について考えた。
この章ではまず主ファイバー束の接続について2通りの定義を述べた後、
主ファイバー束の接続とベクトル束の接続との対応について調べる。

% ------------------------------------------------------------
%
% ------------------------------------------------------------
\section{主ファイバー束の接続 (微分形式)}

主$G$束$P$の接続の定義の方法は2つあり、
\begin{enumerate}
    \item 1つ目は$P$上の$\frakg$値1形式としての定義である。
    \item 2つ目は$T_u P$から垂直部分空間への射影としての定義である。
        こちらは幾何学的な様子がわかりやすいという利点がある。
\end{enumerate}
この節ではまずは$\frakg$値1形式としての定義で接続を導入する。

\subsection{主ファイバー束の接続形式}

\cref{prop:connection-from-1-forms} で見たように、
ベクトル束の接続は
$\GL(r; \R)$値の変換関数と
$\mathfrak{gl}(r; \R)$値の1形式により定めることができた。
そこで、主$G$束でも同様の方法により
$\frakg$値1形式として接続を定義する。

\TODO{こちらはむしろ特徴付けにするべきでは?
    ゲージポテンシャルを主役とみる立場ならこちらが定義?}

\begin{definition}[主ファイバー束の接続形式]
    $M$を多様体、
    $G$を Lie 群、$\frakg$を$G$の Lie 代数、
    $p \colon P \to M$を主$G$束とする。
    $\{ (U_\alpha, \varphi_\alpha) \}_{\alpha \in A}$
    を$\bigcup U_\alpha = P$なる$P$の局所自明化の族とし、
    これにより定まる切断の族を$\{ \sigma_\alpha \}$とおく。
    さらに各$\alpha$に対し、
    $\omega_\alpha$を$U_\alpha$上の
    $\frakg$値$1$-形式であって関係式
    \begin{equation}
        \omega_\beta = \varphi_{\alpha\beta}^{-1} \omega_\alpha \varphi_{\alpha\beta}
            + \varphi_{\alpha\beta}^{-1} d\varphi_{\alpha\beta}
            \quad \text{on} \quad
            U_\alpha \cap U_\beta
    \end{equation}
    をみたすものとする。
    このとき、$P$上の$\frakg$値$1$-形式$\omega$を
    各$p^{-1}(U_\alpha) \subset P$上で
    \begin{equation}
        \omega \coloneqq
            s_\alpha^{-1} (\pi^* \omega_\alpha) s_\alpha
            + s_{\alpha}^{-1} ds_\alpha
    \end{equation}
    と定めることができる (このあとすぐ示す)。
    ただし右辺の積は$TG$における積であり、
    $s_\alpha$は
    \begin{equation}
        s_\alpha \coloneqq \mathrm{pr}_2 \circ \varphi_\alpha
        \colon \pi^{-1}(U_\alpha) \to G,
        \quad
        (x, \sigma_\alpha(x) . s) \mapsto s
    \end{equation}
    と定めた。
    $\omega$を$P$の
    \term{接続形式}[connection form]{接続形式!主ファイバー束の---}[せつぞくけいしき]
    という。
\end{definition}

\begin{proof}
    \uline{($\Rightarrow$)} \quad
    $\{ (U_\alpha, \varphi_\alpha) \}_{\alpha \in A}$
    を$\bigcup U_\alpha = P$なる$P$の局所自明化の族とし、
    これにより定まる切断の族を$\{ \sigma_\alpha \}$とおき、
    $\omega$はこれにより定まる$P$の接続形式であるとする。
    \TODO{}
\end{proof}

\begin{remark}
    $\sigma_\alpha^* \omega = \omega_\alpha$が成り立つ\footnote{
        $\sigma_\alpha^* \omega$は物理学では
        \term{ゲージポテンシャル}[gauge potential]{ゲージポテンシャル}
        と呼ばれる。
    }。
    \TODO{}
\end{remark}

\begin{theorem}[主ファイバー束の接続形式の特徴付け]
    $M$を多様体、
    $G$を Lie 群、$\frakg$を$G$の Lie 代数、
    $p \colon P \to M$を主$G$束とする。
    $\frakg$に値をもつ$P$上の$1$-形式$\omega$に関し、
    $\omega$が$P$の接続形式であることと
    $\omega$がつぎの条件をみたすこととは同値である:
    \begin{enumerate}
        \item ($G$-同変性) $R_a^* \omega = (\Ad g^{-1}) \omega \quad (a \in G)$
        \item $\omega(A^*) = A \quad (A \in \frakg)$
    \end{enumerate}
\end{theorem}

\begin{proof}
    \TODO{}
\end{proof}

\begin{definition}[Ehresmann 接続]
    $M$を多様体、$p \colon P \to M$を主$G$束とする。
    上の定理の条件 (1), (2) をみたす
    $P$上の$\frakg$値1形式$\omega$を
    $P$上の\term{Ehresmann 接続}[Ehresmann connection]{Ehresmann 接続}[Ehresmann せつぞく]
    あるいは単に
    \term{接続}[connection]{接続!主ファイバー束の---}[せつぞく]
    という。
\end{definition}

主ファイバー束の接続に対し、
Lie 群の構造方程式
(\cref{def:lie-group-structure-equation})
と類似の方程式が成り立つ。
したがって
\begin{itemize}
    \item $P$の接続は$G$の接続の一般化
    \item $P$の接続の構造方程式は$G$の構造方程式の一般化
\end{itemize}
とみることができる。\TODO{どういう意味?}

\begin{proposition}[接続の構造方程式]
    \begin{equation}
        d\omega = - [\omega, \omega]
    \end{equation}
    \TODO{}
\end{proposition}

\begin{proof}
    \TODO{}
\end{proof}

\subsection{主ファイバー束の曲率}

主ファイバー束の曲率を定義する。
ベクトル束の接続の曲率は構造方程式 (\cref{prop:structure-equation})
をみたすのであった。
そこで、主ファイバー束の接続の曲率は逆にこの方程式によって定義する。

\begin{definition}[曲率形式]
    \TODO{}
\end{definition}

% ------------------------------------------------------------
%
% ------------------------------------------------------------
\section{主ファイバー束の接続 (水平部分空間の方法)}

前節では主ファイバー束の接続を微分形式として定義した。
この節では水平部分空間の方法を用いて接続を定義する。

\subsection{接分布}

まず基本的な概念を導入しておく。

\begin{definition}[接分布]
    $M$を多様体とする。
    $D \subset TM$が$M$上の
    \term{接分布}[tangent distribution]{接分布}[せつぶんぷ]
    であるとは、
    $D$が$TM$の部分ベクトル束であることをいう。
\end{definition}

\begin{definition}[積分多様体]
    $M$を多様体、$D \subset TM$を$M$上の接分布とする。
    部分多様体$N \subset M$が$D$の
    \term{積分多様体}[integral manifold]{積分多様体}[せきぶんたようたい]
    であるとは、
    \begin{equation}
        T_xN = D_x
            \quad
            (\forall x \in N)
    \end{equation}
    が成り立つことをいう。

    各$x \in M$に対し
    $D$のある積分多様体$N \subset M$が存在して
    $x \in N$となるとき、
    $D$は\term{積分可能}[integrable]{積分可能}[せきぶんかのう]
    であるという。
\end{definition}

\begin{definition}[包合的]
    $M$を多様体、$D \subset TM$を$M$上の接分布とする。
    $D$が\term{包合的}[involutive]{包合的}[ほうごうてき]であるとは、
    $D$の任意の局所切断$X, Y$に対し
    $[X, Y]$も$D$の局所切断となることをいう。
\end{definition}

\begin{theorem}[Frobenius]
    \TODO{}
\end{theorem}

\subsection{垂直接分布と水平接分布}

垂直接分布を定義する。
垂直接分布は接続とは関係なく主ファイバー束の構造のみによって決まる。

\begin{definition}[垂直接分布]
    $M$を多様体、$p \colon P \to M$を主$G$束とする。
    各$u \in P$に対し、$\R$-部分ベクトル空間
    \begin{equation}
        V_u \coloneqq \Ker p_*
    \end{equation}
    を$T_uP$の\term{垂直部分空間}[vertical subspace]{垂直部分空間}[すいちょくぶぶんくうかん]
    という。
    さらに$\coprod_{u \in P} V_u$は$TP$の部分ベクトル束となり、
    これを\term{垂直接分布}[vertical distribution]{垂直接分布}[すいちょくせつぶんぷ]という。

    垂直接分布に属する元は
    \term{垂直}[vertical]{垂直}[すいちょく]であるという。
\end{definition}

垂直部分空間は次のように表せる。
これにより$V_u$と$\frakg$を同一視すれば、
主ファイバー束の接続形式の条件$\omega(A^*) = A$とは
$\omega$が$V_u$上恒等写像であるという条件に他ならない。

\begin{proposition}
    $V_u = \{ A^*_u \mid A \in \frakg \}$
    \TODO{}
\end{proposition}

\begin{proof}
    \TODO{}
\end{proof}

\begin{theorem}[垂直接分布は積分可能]
    \TODO{}
\end{theorem}

\begin{proof}
    \TODO{}
\end{proof}

次に水平部分空間を定義する。
水平部分空間とは$T_u P = V_u \oplus H_u$なる部分空間$H_u$のことであるが、
$H_u$は主ファイバー束の構造のみからは決定されない。
後で詳しく見るが、主ファイバー束に接続を与えることは、
本質的には右不変な水平部分空間を選ぶのと同じことである。

\TODO{あとで接続から水平部分空間が定まることをいうのだから、
    水平部分空間の定義には接続を含むべきでないのでは?}

\begin{definition}[水平部分空間]
    $M$を多様体、$p \colon P \to M$を主$G$束、
    $\omega$を$P$の接続形式とする。
    各$u \in P$に対し、$\R$-部分ベクトル空間
    \begin{equation}
        H_u \coloneqq \Ker \omega_u
    \end{equation}
    を$T_uP$の\term{水平部分空間}[horizontal subspace]{水平部分空間}[すいへいぶぶんくうかん]
    という。
\end{definition}

\subsection{水平接分布と接続}

水平接分布により主ファイバー束の接続を特徴付ける。

\begin{theorem}[水平接分布から接続へ]
    $p \colon P \to M$を主$G$束、
    $H$を右不変な水平接分布とする。
    このとき、$P$上の$\frakg$値1形式$\omega$を
    \begin{equation}
        \omega_u \colon T_u P \to V_u \to \frakg
    \end{equation}
    により定めると、$\omega$は$P$上の接続となる。
\end{theorem}

\begin{proof}
    \TODO{cf. [Tu] p.255}
\end{proof}

\begin{theorem}[接続から水平接分布へ]
    $p \colon P \to M$を主$G$束、
    $\omega$を$P$上の接続とする。
    このとき、$H_u \coloneqq \Ker \omega_u$は
    $P$の右不変な水平部分空間である。
\end{theorem}

\begin{proof}
    \TODO{cf. [Tu] p.257}
\end{proof}

主ファイバー束の曲率形式も
水平接分布を用いて表すことができる。

\begin{proposition}
    接続形式$\omega$の曲率形式$\Omega$は
    \begin{equation}
        \Omega(X, Y) = d\omega(X^H, Y^H)
            \quad
            (X, Y \in T_uP)
    \end{equation}
    をみたす。
\end{proposition}

\begin{proof}
    \TODO{}
\end{proof}

水平接分布の積分可能性と曲率には密接な関係がある。

\begin{theorem}[水平接分布の積分可能性]
    $P$の水平接分布$\coprod_{u \in P} H_u$に関し次は同値である:
    \begin{enumerate}
        \item $\coprod H_u$は積分可能である。
        \item $P$の曲率は$0$である。
    \end{enumerate}
\end{theorem}

\begin{proof}
    \TODO{}
\end{proof}


% ------------------------------------------------------------
%
% ------------------------------------------------------------
\section{同伴ベクトル束の接続}

同伴ベクトル束を思い出そう。
\cref{subsec:principal-fiber-bundle-to-vector-bundle}
で見たように、主ファイバー束$P$と表現$\rho$から
同伴ベクトル束$E = P \times_\rho \R^r$が構成できるのであった。
このとき、$P$の共変外微分から$E$に共変外微分が誘導される。
とくに$P$の接続形式から$E$の接続形式が定まる。

\begin{theorem}[微分形式の対応]
    $P$上の$\R^r$値$p$-形式$\widetilde{\xi}$で
    \begin{enumerate}
        \item $R_a^* \widetilde{\xi} = \rho(a)^{-1} \widetilde{\xi} \quad (a \in G)$
        \item ある$i$で$X_i$が垂直ならば
            $\widetilde{\xi}(X_1, \dots, X_p) = 0$
    \end{enumerate}
    をみたすもの全体の空間を$\widetilde{A}^p(P)$とおく。
    $A^p(E)$と$\widetilde{A}^p(P)$は
    次の対応により1:1に対応する。
    \begin{equation}
        \widetilde{\xi}(X_1, \dots, X_p)
            = u^{-1} (\xi (\pi_* X_1, \dots, \pi_* X_p))
            \quad
            (X_1, \dots, X_p \in T_u P)
    \end{equation}
    \TODO{}
\end{theorem}

\begin{proof}
    \TODO{}
\end{proof}

\begin{proposition}
    上の定理の対応は
    外積代数$A(E)$から$\widetilde{A}(P)$への
    $A(M)$-加群同型である。
    \TODO{}
\end{proposition}

\begin{proof}
    \TODO{}
\end{proof}

$\widetilde{A}(P)$に共変外微分を定義し、
上の同型により$A(E)$に共変外微分を誘導する。

\begin{definition}[$\widetilde{A}(P)$の共変外微分]
    \begin{equation}
        D\widetilde{\xi}(X_1, \dots, X_{p + 1})
            = d\widetilde{\xi}(X^H_1, \dots, X^H_{p + 1})
            \quad
            (X_1, \dots, X_{p + 1} \in T_u P)
    \end{equation}
    \TODO{}
\end{definition}

$D\widetilde{\xi}$は次のように書くこともできる。

\begin{proposition}
    \begin{equation}
        D\widetilde{\xi} = d\widetilde{\xi} + \rho(\omega) \wedge \widetilde{\xi}
    \end{equation}
    \TODO{}
\end{proposition}

\begin{proof}
    \TODO{}
\end{proof}

% ------------------------------------------------------------
%
% ------------------------------------------------------------
\section{平行移動とホロノミー}

この節では、平行移動とホロノミーについて述べる。

\subsection{ベクトル束の平行移動とホロノミー}

さて、ここで曲線$\gamma$に沿う$\xi$の共変微分
「$\nabla_{\dot{\gamma}(t)} \xi$」を定義したい。
ところが、ややこしいことに「曲線$\gamma$に沿う$E$の切断」は「$E$の切断」ではないため、
「$\nabla_{\dot{\gamma}(t)} \xi$」という文字列に
正確な意味を与えるにはさらなる定義が必要となる。

\TODO{引き戻しを用いて定義したほうがよさそう cf. [Tu] p. 262}

\begin{definition}[曲線に沿う切断の拡張可能性]
    $M$を多様体、
    $\pi \colon E \to M$をベクトル束、
    $J$を$\R$の区間、
    $\gamma \colon J \to M$を{\smooth}曲線とする。
    曲線$\gamma$に沿う$E$の切断$\xi \colon J \to E$が
    \term{拡張可能}[extendible]{拡張可能}[かくちょうかのう]
    であるとは、
    $\gamma$の像$\gamma(J)$を含む$U \opensubset M$と
    $U$上の$E$の切断$\widetilde{\xi}$が存在して
    \begin{equation}
        \widetilde{\xi}_{\gamma(t)} = \xi_t
            \quad
            (\forall t \in J)
    \end{equation}
    が成り立つことをいう。
    $\widetilde{\xi}$を
    \term{$\xi$の拡張}{拡張!曲線に沿う切断の---}[かくちょう]という。
\end{definition}

\begin{example}[拡張可能でない例]
    8の字曲線$\gamma \colon (-\pi, \pi) \to \R^2, \;
    t \mapsto (\sin t, \sin t \cos t)$の
    速度ベクトル$\dot{\gamma}$は拡張可能でない。
    なぜならば、8の字の中央部分で速度ベクトルが2方向に出ているからである。
\end{example}

\begin{definition}[曲線に沿う共変微分]
    上の定義の状況で、
    さらに$\nabla$を$E$の接続とし、
    $\xi$は拡張可能であるとする。
    このとき、$\xi$の拡張$\widetilde{\xi} \in \Gamma(E)$をひとつ選び
    \begin{equation}
        \nabla_{\dot{\gamma}(t)} \xi
            \coloneqq \nabla_{\dot{\gamma}(t)} \widetilde{\xi}
            \quad
            (t \in J)
    \end{equation}
    と定義し、これを
    \term{曲線$\gamma$に沿う共変微分}[covariant derivative along $\gamma$]
    {曲線に沿う共変微分}[きょくせんにそうきょうへんびぶん]という。
    これは$\widetilde{\xi}$の選び方によらず well-defined に定まる (このあと示す)。
\end{definition}

\begin{proposition}
    上の定義の状況で、
    さらに$\widetilde{\xi} \in \Gamma(E)$を$\xi$の拡張、
    $t \in J$、
    $U$を$M$における$\gamma(t)$の開近傍、
    $e_1, \dots, e_r$を$U$上の$E$の局所フレーム、
    $x^1, \dots, x^n$を$U$上の$M$の局所座標とする。
    $\widetilde{\xi}$を局所的に
    \begin{alignat}{1}
        \widetilde{\xi} &= \widetilde{\xi}^\lambda e_\lambda
            \quad
            (\widetilde{\xi}^\lambda \in \smooth(U))
    \end{alignat}
    と表し、$\xi^\lambda \coloneqq \widetilde{\xi}^\lambda \circ \gamma$とおく。
    また$\nabla e_\mu$を局所的に
    \begin{alignat}{1}
        \nabla e_\mu
            &= \omega_\mu^\lambda \otimes e_\lambda
                \quad
                (\omega_\mu^\lambda \in A^1(U)) \\
            &= \Gamma^\lambda_{\mu i} dx^i \otimes e_\lambda
                \quad
                (\Gamma^\lambda_{\mu i} \in \smooth(U))
    \end{alignat}
    と表す。
    このとき
    \begin{equation}
        \nabla_{\dot{\gamma}(t)} \xi
            = \left\{
                \frac{d\xi^\lambda}{dt}(t)
                +
                \xi^\mu (t)
                \Gamma^\lambda_{\mu i} (\gamma(t))
                \frac{d\gamma^i}{dt}(t)
            \right\}
            (e_\lambda)_{\gamma(t)}
    \end{equation}
    が成り立つ。
    したがってとくに$\nabla_{\dot{\gamma}(t)} \xi$の値は
    $\widetilde{\xi}$の選び方によらず well-defined に定まる。
\end{proposition}

\begin{proof}
    まず記法を整理すると
    \begin{equation}
        \nabla_{\dot{\gamma}(t)} \xi
            = \nabla_{\dot{\gamma}(t)} \widetilde{\xi}
            = (\nabla \widetilde{\xi}) (\dot{\gamma}(t))
            = \underbrace{
                (\nabla \widetilde{\xi})_{\gamma(t)}
            }_{\in \, T^*_{\gamma(t)} M \otimes E_{\gamma(t)}}
            (\underbrace{\dot{\gamma}(t)}_{\in \, T_{\gamma(t)} M})
    \end{equation}
    と書けることに注意する。
    そこで$\nabla \widetilde{\xi}$を変形すると
    \begin{alignat}{1}
        \nabla \widetilde{\xi}
            &= d\widetilde{\xi}^\lambda \otimes e_\lambda
                + \widetilde{\xi}^\mu \nabla e_\mu \\
            &= d\widetilde{\xi}^\lambda \otimes e_\lambda
                + \widetilde{\xi}^\mu \Gamma^\lambda_{\mu i} dx^i \otimes e_\lambda \\
            &= \left\{
                d\widetilde{\xi}^\lambda
                + \widetilde{\xi}^\mu \Gamma^\lambda_{\mu i} dx^i
            \right\} \otimes e_\lambda
    \end{alignat}
    となるから、点$\gamma(t)$での値は
    \begin{alignat}{1}
        (\nabla \widetilde{\xi})_{\gamma(t)}
            &= \left\{
                d\widetilde{\xi}^\lambda_{\gamma(t)}
                + \widetilde{\xi}^\mu (\gamma(t))
                \Gamma^\lambda_{\mu i} (\gamma(t))
                dx^i_{\gamma(t)}
            \right\} \otimes (e_\lambda)_{\gamma(t)} \\
            &= \left\{
                d\widetilde{\xi}^\lambda_{\gamma(t)}
                + \xi^\mu (\gamma(t))
                \Gamma^\lambda_{\mu i} (\gamma(t))
                dx^i_{\gamma(t)}
            \right\} \otimes (e_\lambda)_{\gamma(t)}
    \end{alignat}
    である。
    ここで
    \begin{alignat}{1}
        (d\widetilde{\xi}^\lambda)_{\gamma(t)} (\dot{\gamma}(t))
            &= \dd{t}\bigg|_{t = t} \widetilde{\xi}^\lambda \circ \gamma(t)
            = \frac{d\xi^\lambda}{dt}(t) \\
        (dx^i)_\gamma(t) (\dot{\gamma}(t))
            &= \dd{t}\bigg|_{t = t} x^i \circ \gamma(t)
            = \frac{d\gamma^i}{dt}(t)
    \end{alignat}
    だから
    \begin{alignat}{1}
        \nabla_{\dot{\gamma}(t)} \xi
            = (\nabla \widetilde{\xi})_{\gamma(t)}
            = \left\{
                \frac{d\xi^\lambda}{dt}(t)
                +
                \xi^\mu (t)
                \Gamma^\lambda_{\mu i} (\gamma(t))
                \frac{d\gamma^i}{dt}(t)
            \right\}
            (e_\lambda)_{\gamma(t)}
    \end{alignat}
    を得る。
    関数$\xi^\lambda$は
    $\widetilde{\xi}$の選び方によらないから
    well-defined 性もいえた。
\end{proof}

測地線の一般化として、
平行の概念を定義する。

\begin{definition}[平行]
    $M$を多様体、$E \to M$をベクトル束、
    $\nabla$を$E$の接続、
    $J$を$\R$の区間、
    $\gamma \colon J \to M$を{\smooth}曲線、
    $\xi$を曲線$\gamma$に沿う$E$の切断とする。
    $\xi$が
    \begin{equation}
        \nabla_{\dot{\gamma}(t)} \xi = 0
            \quad
            (\forall t \in J)
    \end{equation}
    をみたすとき、$\xi$は
    \term{曲線$\gamma$に沿って平行}[parallel along $\gamma$]{平行}[へいこう]
    であるという。
    上の命題より、これは次の斉次1階常微分方程式系が成り立つことと同値である:
    \begin{equation}
        \frac{d\xi^\lambda}{dt}(t)
            + \xi^\mu (t)
            \Gamma^\lambda_{\mu i} (\gamma(t))
            \frac{d\gamma^i}{dt}(t)
            = 0
            \quad
            (\lambda = 1, \ldots, r)
    \end{equation}
    $\bm{\xi} \coloneqq \up{t}(\xi^1, \dots, \xi^r), \;
    A \coloneqq \left(
        \Gamma^{\lambda}_{\mu i} \frac{d\gamma^i}{dt}
    \right)_{\lambda, \mu}$
    とおけば
    \begin{equation}
        \frac{d\bm{\xi}}{dt} = - A \bm{\xi}
    \end{equation}
    と書ける。
\end{definition}

\begin{remark}
    測地線とは、
    その速度ベクトルが自身に沿って平行な曲線のことである。
\end{remark}

\begin{definition}[平行移動]
    上の命題の状況で
    さらに$J = [a, b], \; a, b \in \R$とするとき、
    初期値問題の解の存在と一意性より
    任意の$\xi_a \in E_{\gamma(a)}$に対し
    $\xi(a) = \xi_a$なる
    解$\xi$が一意に定まる。
    このとき、$\xi$は
    $\xi_a$を曲線$\gamma$に沿って
    \term{平行移動}[parallel displacement]{平行移動}[へいこういどう]
    して得られたという\footnote{
        最適化の分野では、
        指数写像や平行移動の数値計算のために、
        これらの代替となる
        \term{レトラクション}[retraction]{レトラクション}[れとらくしょん]
        や
        \term{ベクトル輸送}[vector transport]{ベクトル輸送}[べくとるゆそう]
        が用いられる。
    }。
\end{definition}

\begin{proposition}
    上の定義の状況で、
    写像
    \begin{equation}
        E_{\gamma(a)} \to E_{\gamma(b)},
        \quad
        \xi_a \mapsto \xi_b \coloneqq \xi(b)
    \end{equation}
    は$\R$-線型同型である。
\end{proposition}

\begin{proof}
    初期値問題の解の存在と一意性より、写像であることはよい。
    全射性は$t = b$での$\xi$の値を指定した初期値問題を考えればよい。
    $\R$-スカラー倍を保つことは次のようにしてわかる:
    $\xi$が$\xi(a) = \xi_a$なる解であったとすると、
    各$c \in \R$に対し
    $\eta(t) \coloneqq c \xi(t)$は$\eta(a) = c \xi_a$をみたすただひとつの解であるから、
    $c \xi_a = \eta(a)$を曲線$\gamma$に沿って平行移動して得られる値は
    $\eta(b) = c \xi(b) = c \xi_b$に他ならない。
    和を保つことも同様にして示せる。
    よって命題の写像は全射$\R$-線型写像である。
    $\dim_\R E_{\gamma(a)} = \dim_\R E_{\gamma(b)}$より
    $\R$-線型同型であることが従う。
\end{proof}

\begin{definition}[ベクトル束の接続のホロノミー群]
    $x_0 \in M$とする。
    $x_0$を基点とする区分的に{\smooth}な任意の閉曲線$c$に対し、
    平行移動により$\R$-ベクトル空間$E_{x_0}$の自己同型写像
    ($\tau_c$とおく) が得られる。
    そこで
    \begin{equation}
        \Psi_{x_0} \coloneqq \{
            \tau_c \in GL(E_{x_0})
            \mid
            \text{$c$は$x_0$を基点とする区分的に{\smooth}な閉曲線}
        \}
    \end{equation}
    とおくと、
    $\Psi_{x_0}$は$GL(E_{x_0})$の部分群となる (このあと示す)。
    $\Psi_{x_0}$を$x_0$を基点とする
    \term{ホロノミー群}[holonomy group]{ホロノミー群}[ほろのみーぐん]という。
\end{definition}

\begin{proof}
    \uline{写像の合成について閉じていること} \quad
    $\tau_c, \tau_{c'} \in \Psi_{x_0}$とすると
    $c \circ c'$は$x_0$を基点とする区分的に{\smooth}な閉曲線であり、
    $\tau_c \circ \tau_{c'} = \tau_{c \circ c'}$が成り立つ。

    \uline{単位元を含むこと} \quad
    定値曲線$x_0$に対し$\tau_{x_0} \in \Psi_{x_0}$が恒等写像となる。

    \uline{逆元を含むこと} \quad
    $\tau_c \in \Psi_{x_0}$とする。
    $c$を逆向きに動く曲線$d$を
    $d(t) \coloneqq c(a + b - t) \; t \in [a, b]$で定め、
    $\xi$を逆向きに動く曲線$\eta$を
    $\eta(t) \coloneqq \xi(a + b - t) \; t \in [a, b]$で定める。
    このとき$d$は$x_0$を基点とする区分的に{\smooth}な閉曲線だから
    $\tau_d \in \Psi_{x_0}$である。
    また、$\eta$は$d$に沿う$E$の切断である。
    さらに$\eta$が$d$に沿って平行であることは、
    $\xi$の拡張を$\widetilde{\xi}$として
    (これは$\eta$の拡張でもある)
    \begin{alignat}{1}
        \nabla_{\dot{d}(t)} \eta
            &= \nabla_{\dot{d}(t)} \widetilde{\xi} \\
            &= \nabla_{- \dot{c}(a + b - t)} \widetilde{\xi} \\
            &= - \nabla_{\dot{c}(a + b - t)} \widetilde{\xi} \\
            &= - \nabla_{\dot{c}(a + b - t)} \xi \\
            &= 0
    \end{alignat}
    よりわかる。
    よって$\xi_b = \eta(a)$を$d$に沿って平行移動すると
    $\eta(b) = \xi(a) = \xi_a$が得られる。
    したがって$\eta_d = \eta_c^{-1}$である。
\end{proof}

%\begin{proposition}
%    $M$を多様体、$E \to M$をベクトル束、
%    $g$を$E$の内積、
%    $\nabla$を$g$を保つ$E$の接続とする。
%    このとき、内積は平行移動で不変である\TODO{どういう意味?}。
%\end{proposition}
%
%\begin{proof}
%    \TODO{}
%\end{proof}

\subsection{主ファイバー束の平行移動とホロノミー}

\begin{definition}[水平な曲線]
    $M$を多様体、
    $G$を Lie 群、
    $p \colon P \to M$を主$G$束、
    $\omega$を$P$の接続形式、
    $J \subset \R$を区間とする。
    {\smooth}曲線$u \colon J \to P$が
    \term{水平}[horizontal]{水平}[すいへい]であるとは、
    $u$の速度ベクトル$\dot{u}$がつねに水平部分空間に含まれること、すなわち
    \begin{equation}
        \omega(\dot{u}(t)) = 0
            \quad (\forall t \in J)
    \end{equation}
    が成り立つことをいう。
\end{definition}

\begin{definition}[平行移動]
    $M$を多様体、
    $G$を Lie 群、
    $p \colon P \to M$を主$G$束、
    $\omega$を$P$の接続形式、
    $J \subset \R$を区間、
    $x \colon J \to M$を$x_0 \in M$を始点とする{\smooth}曲線
    とする。
    このとき各$u_0 \in P_{x_0}$に対し、
    $u_0$を始点とする水平な{\smooth}曲線$u \colon J \to P$であって
    \begin{equation}
        \pi(u(t)) = x(t) \quad (t \in J)
    \end{equation}
    をみたすものが一意に存在する (証明略)。
    このとき、
    $u$は曲線$x$に沿った$u_0$の
    \term{平行移動}[parallel displacement]{平行移動}[へいこういどう]
    であるという。
    \begin{equation}
        \begin{tikzcd}
            & P \ar{d}{p} \\
            J \ar{ru}{u} \ar{r}[swap]{x}
                & M
        \end{tikzcd}
    \end{equation}
\end{definition}

\begin{proposition}
    $u$が水平ならば、任意の$s \in G$に対し
    $u(t) . s$も水平である。
\end{proposition}

\begin{proof}
    水平接分布が$G$の作用で保たれることより明らか。
\end{proof}

\begin{definition}[主ファイバー束の接続のホロノミー群]
    $u_0 \in P$とし、$x_0 = p(u_0)$とおく。
    $x_0$を始点とする$M$内の任意の閉曲線$c$に対し、
    $x$に沿った$u_0$の平行移動を$u$とおくと
    \begin{equation}
        u(b) = u_0 . \tau_c
    \end{equation}
    なる$\tau_c \in G$が一意に定まる。
    このような$\tau_c$全体の集合を$\Psi_{u_0}$とおくと、
    $\Psi_{u_0}$は$G$の部分群となる。
    $\Psi_{u_0}$を$u_0$を始点とする$\omega$の
    \term{ホロノミー群}[holonomy group]{ホロノミー群}[ほろのみーぐん]という。
\end{definition}

\begin{proposition}[ホロノミー群の共役]
    $u_0, u_1 \in P$とし、
    $x_0 = p(u_0), \; x_1 = p(u_1)$とおく。
    $c_0$を$x_0$から$x_1$への区分的に{\smooth}な曲線とし、
    曲線$c_0$に沿った$u_0$の平行移動を$\widetilde{c}_0$とおく。
    すると$\widetilde{c}_0(b) = u_1 . a$なる$a \in G$がただひとつ存在するが、
    このとき$\Psi_{u_1} = a \Psi_{u_0} a^{-1}$が成り立つ。
\end{proposition}

\begin{proof}
    $a \Psi_{u_0} a^{-1} \subset \Psi_{u_1}$および
    $a^{-1} \Psi_{u_1} a \subset \Psi_{u_0}$を示せばよい。
    実際、これらが示されたならば
    $a \Psi_{u_0} a^{-1} \subset \Psi_{u_1}
    = aa^{-1} \Psi_{u_1} aa^{-1} \subset a \Psi_{u_0} a^{-1}$
    より$a \Psi_{u_0} a^{-1} = \Psi_{u_1}$が従う。
    さらに$u_0, u_1$に関する対称性より
    $a \Psi_{u_0} a^{-1} \subset \Psi_{u_1}$を示せば十分。
    そこで$\tau_c \in \Psi_{u_1}$とし、
    $c$に沿う$u_0$の平行移動を$\widetilde{c}$とおき、
    $a \tau_c a^{-1} \in \Psi_{u_0}$を示す。
    そのためには$a \tau_c a^{-1} = \tau_{c_0 \circ c \circ c_0^{-1}}$であること、
    すなわち$c_0 \circ c \circ c_0^{-1}$に沿う
    $u_1$の平行移動の終点が$u_1 . a \tau_c a^{-1}$であることをいえばよい。

    まず$c_0^{-1}$に沿う$u_1$の平行移動は
    $R_{a^{-1}} \circ \widetilde{c}_0^{-1}$であり、
    その終点は$u_0 . a^{-1}$である。

    つぎに$c$に沿う$u_0 . a^{-1}$の平行移動は
    $R_{a^{-1}} \circ \widetilde{c}$であり、
    その終点は$u_0 . \tau_c a^{-1}$である。

    最後に$c_0$に沿う$u_0 . \tau_c a^{-1}$の平行移動は
    $R_{\tau_c a^{-1}} \circ \widetilde{c}_0$であり、
    その終点は$u_1 . a \tau_c a^{-1}$である。
    これが示したいことであった。
\end{proof}






% ============================================================
%
% ============================================================
\chapter{特性類}

特性類について述べる。
特性類はベクトル束の位相不変量である。

% ------------------------------------------------------------
%
% ------------------------------------------------------------
\section{複素ベクトル束}

\begin{definition}[Complex Vector Bundles]
    \TODO{}
\end{definition}

% ------------------------------------------------------------
%
% ------------------------------------------------------------
\section{Euler 類}

\TODO{}

% ------------------------------------------------------------
%
% ------------------------------------------------------------
\section{Chern 類}

\TODO{}




\end{document}

\part{計量と Riemann 多様体}
\documentclass[report]{jlreq}
\usepackage{global}
\usepackage{./local}
\subfiletrue
%\makeindex
\begin{document}




% ============================================================
%
% ============================================================
\chapter{体}

体について述べる。

% ------------------------------------------------------------
%
% ------------------------------------------------------------
\section{体}

\begin{definition}[素体]
    $k$を体とする。
    $k$の部分体すべての共通部分を$k$の
    \term{素体}[prime field]{素体}[そたい]という。
\end{definition}

\begin{definition}[標数]
    $k$を体とし、
    環準同型$\Z \to k, \; n \mapsto n1_k$を$\phi$とおく。
    $1_k = \phi(1) \in \Im\phi$ゆえに$\Im\phi \neq 0$であり、
    また$\Im\phi$は整域だから、
    準同型定理より$\Ker\phi$は$\Z$の素イデアルである。
    よって$\Ker\phi = (p) \; (\text{$p$は$0$または素数})$と表せる。
    $p$を$k$の\term{標数}[characteristic]{標数}[ひょうすう]という。
\end{definition}


% ------------------------------------------------------------
%
% ------------------------------------------------------------
\section{有限体}

\begin{definition}[有限体]
    濃度が有限の体を
    \term{有限体}[finite field]{有限体}[ゆうげんたい]という。
\end{definition}

\begin{theorem}[有限体の濃度]
    \label[theorem]{thm:cardinality-of-finite-field}
    有限体の濃度は素数の冪である。
\end{theorem}

\begin{proof}
    $k$を有限体とし、
    $k$の標数を$p$とおく。
    $p = 0$だとすると$k$が$\Z$と同型な部分環を含むことになり
    $k$の濃度が有限であることに反するから、
    $p$は素数である。
    よって$k$は$\Z / p\Z$と同型な部分環、
    より強く部分体をもつ。
    $k$を左正則加群とみなせば、
    係数制限により$k$は$\Z / p\Z$上のベクトル空間となり、
    いま$k$の濃度は有限だから
    $\dim_{\Z / p\Z} k \eqqcolon n \in \Z_{\ge 1}$である。
    よって$k$の濃度は$\sharp k = p^n$である。
\end{proof}



% ============================================================
%
% ============================================================
\chapter{体の拡大}

% ------------------------------------------------------------
%
% ------------------------------------------------------------
\section{体の拡大}

多角形の対称変換と多項式の Galois 群との関連は次のように整理できる:

\TODO{なぜここに書いてある?}

\begin{figure}[h]
    \centering
    \begin{tabular}{ll}
        多角形$P$ & 多項式$f(x) \in F[x]$ \\
        平面 & $f(x)$の分離体$E$ \\
        頂点$\Vert(P) = \{v_1, \dots, v_n\}$ & 根$\alpha_1, \dots, \alpha_n$ \\
        線型変換 & $E$の自己同型 \\
        直交変換 & $F$を固定する$E$の自己同型 \\
        $P$を固定する直交変換の群 & Galois 群$\Gal(f) = \Gal(E/F)$ \\
        正多角形 & 既約多項式
    \end{tabular}
\end{figure}

\begin{definition}[体の拡大]
    \idxsym{degree of field extension}{$[L : K]$}{$L$の$K$上の拡大次数}
    $L$を体とする。
    $L$の部分環$K$が体であるとき、
    $K$を$L$の\term{部分体}[subfield]{部分体}[ぶぶんたい]といい、
    $L$を$K$の\term{拡大体}[extension field]{拡大体}[かくだいたい]という。
    \emph{$L/K$は体の拡大である}ともいう。
    $L$の$K$-ベクトル空間としての次元を$[L : K]$と書き、
    $L$の$K$上の
    \term{拡大次数}[degree of field extensioni]{拡大次数}[かくだいじすう]という。
\end{definition}

\begin{example}[拡大体の例]
    ~
    \begin{itemize}
        \item $\R$は$\Q$の拡大体である。
        \item $\C$は$\R$の拡大体である。
            $\C$は$\R$-ベクトル空間として基底$\{1, \sqrt{-1}\}$がとれるので
            $[\C \colon \R] = 2$である。
            したがって$\C$は$\R$の2次拡大である。
        \item $d \neq 1$を square-free な整数とする(e.g. $d = 6$)。
            $L = \Q[\sqrt{d}] = \{ a + b\sqrt{d} \in \Q \colon a, b \in \Q \}$は
            $\C$の部分体である
            (実際、$\Q[\sqrt{d}] \cong \Q[x]/(x^2 - d)$であり、
            $x^2 - d$は$\Q[x]$の既約元
            ($\because$ $L$は$\C$の部分環ゆえに整域)
            だから、$\Q$が体であることと併せて$\Q[x]/(x^2 - d)$は体である)。
            $\sqrt{d} \not\in \Q$ゆえに$[L \colon \Q] \ge 2$である。
            $L$は$\Q$-ベクトル空間として基底$\{1, \sqrt{d}\}$がとれるので
            $[L \colon \Q] = 2$である。
        \item $K$を体とする。$A = K[x_1, \dots, x_n]$を$n$変数多項式環、
            $L = K(x_1, \dots, x_n)$を$n$変数有理関数体とする。
            $A$の$K$-ベクトル空間としての次元は$\infty$である。
            さらに$A$は整域なので、その商体$K(x_1, \dots, x_n)$への自然な準同型は単射、
            したがって$A$は$K(x_1, \dots, x_n)$に含まれる。
            よって$K(x_1, \dots, x_n)/K$は無限次拡大である。
    \end{itemize}
\end{example}

\begin{definition}[代数体]
    $\Q$の有限次拡大体を
    \term{代数体}[algebraic field]{代数体}[だいすうたい]という。
\end{definition}

\begin{definition}[合成体]
    $L$を体とし、$M_1, M_2$を$L$の部分体とする。
    \TODO{}
\end{definition}

\begin{proposition}[体の準同型]
    \label[proposition]{prop:field-homomorphism}
    $K$を体とし、$L, M$を$K$の拡大体とする。
    \begin{enumerate}
        \item $S \subset L$に対し包含写像$\begin{tikzcd}
                S \ar[hook]{r} & K(S)
            \end{tikzcd}$は$K$の拡大体の圏のエピ射である。
            \begin{equation}
                \begin{tikzcd}
                    S \ar[hook]{r}
                        & K(S) \ar[shift left]{r} \ar[shift right]{r}
                        & \bullet
                \end{tikzcd}
            \end{equation}
            すなわち、$K$の拡大体の間の準同型$K(S) \to \bullet$は$S$上の値で決まる。
        \item \TODO{}
    \end{enumerate}
\end{proposition}

\begin{proof}
    cf. [雪江] p.163
\end{proof}


% ------------------------------------------------------------
%
% ------------------------------------------------------------
\section{添加}

\begin{definition}[添加]
    $L/K$を体の拡大、$S \subset L$を部分集合とする。
    \begin{itemize}
        \item $S$が有限集合$S = \{\alpha_1, \dots, \alpha_n\}$なら
            \begin{equation}
                K(S) \coloneqq \left\{
                    \frac{f(\alpha_1, \dots, \alpha_n)}{g(\alpha_1, \dots, \alpha_n)} \in L
                    \colon
                    \frac{f(x_1, \dots, x_n)}{g(x_1, \dots, x_n)} \text{ は$K$係数有理式},\;
                    g(\alpha_1, \dots, \alpha_n) \neq 0
                \right\}
            \end{equation}
        \item $S$が無限集合なら
            \begin{equation}
                K(S) \coloneqq \bigcup_{\substack{S' \subset S \\ |S'| < \infty}} K(S')
            \end{equation}
    \end{itemize}
    と定義する。
    $K(S)$を$K$に$S$を
    \term{添加}[adjunction]{添加}[てんか]した体という。
    \begin{itemize}
        \item $S$が有限集合ならば
            $K(S)$は$K$上
            \term{有限生成}[finitely-generated]{有限生成}[ゆうげんせいせい]
            といい、
        \item $S$が1元集合ならば
            $K(S)$は$K$の
            \term{単拡大}{単拡大}[たんかくだい]
            であるという。
    \end{itemize}
\end{definition}

\begin{example}[有限生成だが有限次拡大でない例]
    $K$を体とする。
    $K$上の1変数有理関数体$K(x)$は$K$上有限生成である。
    しかし拡大次数は$\infty$である。
\end{example}

\begin{definition}[代数拡大と超越拡大]
    $L/K$を体の拡大、$x \in L$とする。
    $a_0, \dots, a_n \in K$、少なくともひとつは$0$でない、が存在して
    \begin{equation}
        a_n x^n + a_{n-1} x^{n-1} + \dots + a_0 = 0
    \end{equation}
    が成り立つとき、
    $x$は$K$上
    \term{代数的}[algebraic]{代数的}[だいすうてき]
    であるといい、
    そうでなければ$x$は$K$上
    \term{超越的}[trancendental]{超越的}[ちょうえつてき]
    であるという。
    $L$のすべての元が$K$上代数的ならば、
    $L/K$は
    \term{代数拡大}[algebraic extension]{代数拡大}[だいすうかくだい]
    といい、
    そうでなければ$L/K$は
    \term{超越拡大}[trancendental extension]{超越拡大}[ちょうえつかくだい]
    という。
\end{definition}

\begin{example}[有限生成と代数拡大]
    ~
    \begin{itemize}
        \item $\Q(\pi)/\Q$は有限生成だが代数拡大でない。
        \item $\Q(\{ \sqrt[n]{2} \colon n = 1, 2, \dots \})$は
            代数拡大だが有限生成でない。
    \end{itemize}
\end{example}

\begin{proposition}[有限次拡大は代数拡大]
    体の拡大$L/K$が有限次拡大ならば、$L/K$は代数拡大である。
\end{proposition}

\begin{proof}
    省略
\end{proof}

\begin{proposition}[有限群の Lagrange の定理の類似]
    $L/M, M/K$を体の有限次拡大とする。
    このとき、$L/K$も有限次拡大で
    \begin{equation}
        [L \colon K] = [L \colon M] [M \colon K]
    \end{equation}
    が成り立つ。
\end{proposition}

\begin{proof}
    省略
\end{proof}

\begin{definition}[最小多項式]
    $L/K$を体の代数拡大とし、$\alpha \in L$とする。
    $K$上の$0$でないモニック多項式$f$で$f(\alpha) = 0$をみたすもののうち
    $\deg f(x)$が最小となるものが一意に存在する(証明略)。
    これを$\alpha$の$K$上の
    \term{最小多項式}[minimal polynomial]{最小多項式}[さいしょうたこうしき]
    という。
\end{definition}

\begin{definition}[共役]
    $L, M$を$K$の拡大体、$\alpha \in L$とする。
    $\alpha$の$K$上の最小多項式を$f$とするとき、
    $f$の根で$M$に属するものを、
    $\alpha$の$M$における$K$上の
    \term{共役}[conjugate]{共役}[きょうやく]、
    あるいは単に$K$上の共役という。
    \begin{equation}
        \begin{tikzcd}
            \alpha && \\[-4ex]
            \rotatebox[origin=c]{-90}{\in} && \\[-4ex]
            L && M \\
            & K \ar[dash]{lu} \ar[dash]{ru}
        \end{tikzcd}
    \end{equation}
\end{definition}

\begin{example}[共役の例]
    \label[example]{ex:conjugate}
    $d \neq 1$を square-free な整数とする (e.g. $d = 6$)。
    $\sqrt{d}$の$\Q$上の最小多項式は
    $x^2 - d = (x - \sqrt{d})(x + \sqrt{d})$なので、
    $\sqrt{d}$の$\Q$上の共役は$\pm \sqrt{d}$である。
    \begin{equation}
        \begin{tikzcd}
            \sqrt{d} && -\sqrt{d} \\[-4ex]
            \rotatebox[origin=c]{-90}{\in} && \rotatebox[origin=c]{-90}{\in} \\[-4ex]
            \Q[\sqrt{d}] && \Q[\sqrt{d}] \\
            & \Q \ar[dash]{lu} \ar[dash]{ru}
        \end{tikzcd}
    \end{equation}
\end{example}

\begin{proposition}[共役は$K$準同型で保たれる]
    \label[proposition]{prop:conjugate-preserved-under-morphism}
    $L/K$を代数拡大、$F/K$を拡大とする。
    各$\alpha \in L$と$\phi \in \Hom_K^\al (L, F)$に対し、
    $\phi(\alpha)$は$\alpha$の共役である。
    \begin{equation}
        \begin{tikzcd}
            \alpha \ar[mapsto]{rr} && \phi(\alpha) \\[-4ex]
            \rotatebox[origin=c]{-90}{\in} && \rotatebox[origin=c]{-90}{\in} \\[-4ex]
            L \ar{rr}{\phi} && F \\
            & K \ar[dash]{lu} \ar[dash]{ru}
        \end{tikzcd}
    \end{equation}
\end{proposition}

\begin{proof}
    cf. [雪江] p.167
\end{proof}




% ------------------------------------------------------------
%
% ------------------------------------------------------------
\section{代数閉包}

\begin{definition}[代数閉包]
    $K$を体とする。
    $L/K$が代数拡大であり$L$が代数的閉体であるとき、
    $L$を$K$の
    \term{代数閉包}[algebraic closure]{代数閉包}[だいすうへいほう]
    という。
\end{definition}

\begin{theorem}[代数閉包の存在 (Steinitz)]
    \TODO{}
\end{theorem}

\begin{proof}
    省略
\end{proof}



% ------------------------------------------------------------
%
% ------------------------------------------------------------
\section{分離拡大}

\begin{definition}[分離拡大]
    ~
    \begin{itemize}
        \item $f(x) \in K[x], \alpha \in \wb{K}$で、
            $f(x)$が$\wb{K}[x]$で$(x - \alpha)^2$で割り切れるとき、
            $\alpha$を$f(x)$の
            \term{重根}[multiple root]{重根}[じゅうこん]
            という。
        \item $f(x)$が$\wb{K}$に重根を持たないとき、
            $f(x)$を
            \term{分離多項式}[separable polynomial]{分離多項式}[ぶんりたこうしき]
            という。
        \item $\alpha \in \wb{K}$の$K$上の最小多項式が分離多項式であるとき、
            $\alpha$は$K$上
            \term{分離的}[separable]{分離的}[ぶんりてき]
            であるといい、
            そうでなければ
            \term{非分離的}[inseparable]{非分離的}[ひぶんりてき]
            であるという。
        \item $K$の代数拡大$L$のすべての元が$K$上分離的であるとき、
            $L$を$K$の
            \term{分離拡大}[separable extension]{分離拡大}[ぶんりかくだい]
            といい、
            そうでなければ
            \term{非分離拡大}[inseparable extension]{非分離拡大}[ひぶんりかくだい]
            であるという。
        \item $K$の任意の代数拡大が$K$の分離拡大ならば、
            $K$を
            \term{完全体}[perfect field]{完全体}[かんぜんたい]
            という。
    \end{itemize}
\end{definition}

多項式が分離多項式かどうかは、微分をみて判定することができる。

\begin{proposition}[分離多項式と微分]
    $K$を体とし、$f(x) \in K[x]$とする。
    このとき、次は同値である:
    \begin{enumerate}
        \item $f(x)$は分離多項式である。
        \item $f(x)$と$f'(x)$は互いに素である。
    \end{enumerate}
\end{proposition}

\begin{proof}
    省略
\end{proof}

\begin{example}[分離的な元]
    $p$を素数、$K$を標数$p$の体とする。
    $a \in K, f(x) = x^p - x - a$とおく。
    $\alpha \in \wb{K}$が$f(x)$の根なら、
    $f'(x) = -1$なので、
    $\alpha$は$K$上分離的である
    (実際、
    もし$\alpha$が$K$上分離的でなかったとすれば、
    $\alpha$の$K$上の最小多項式$g(x)$は$\wb{K}$に重根を持つ。
    よって、いま$f(\alpha) = 0$ゆえに$f$は$g$で割り切れることから、
    $f$は$\wb{K}$に重根を持つ。
    一方、$f(x)$と$f'(x)$は互いに素だから、
    $f(x)$は$\wb{K}$に重根を持たず、矛盾
    )。
\end{example}

\begin{example}[非分離拡大の例]
    \TODO{}
\end{example}

代数拡大が分離拡大かどうかを考えるとき、
もとの体が完全体ならば話は簡単である。
次の命題は体が完全体であるための十分条件を与える。

\begin{proposition}[完全体であるための十分条件]
    \label[proposition]{prop:perfect-field}
    標数$0$の体と有限体は完全体である。
\end{proposition}

\begin{proof}
    省略
\end{proof}

\begin{definition}[分離閉包]
    $L/K$を代数拡大とする。
    $L$の元で$K$上分離的なもの全体の集合を$L_s$と書き、
    $L$における$K$の
    \term{分離閉包}[separable closure]{分離閉包}[ぶんりへいほう]
    という。
    また、$\wb{K}$における$K$の分離閉包を$K^s$と書き、
    $K$の\emph{分離閉包}という。
\end{definition}

\begin{definition}[分離次数]
    $L/K$を有限次拡大とする。
    \begin{itemize}
        \item $[L_s \colon K]$を
            $L$の$K$上の
            \term{分離次数}[separable degree]{分離次数}[ぶんりじすう]
            といい、
            $[L \colon K]_s$と書く。
        \item $[L \colon L_s]$を$L$の$K$上の
            \term{非分離次数}[inseparable degree]{非分離次数}[ひぶんりじすう]
            といい、
            $[L \colon K]_i$と書く。
    \end{itemize}
\end{definition}

\begin{proposition}[分離次数とホムセットの濃度]
    \label[proposition]{prop:separable-degree-homset-cardinality}
    $L/K$を有限次拡大とする。
    \begin{enumerate}
        \item \TODO{}
        \item $[L \colon K]_s = |\Hom_K^\al (L, \wb{K})|$
    \end{enumerate}
\end{proposition}

\begin{proof}
    cf. [雪江] p.183
\end{proof}

\begin{example}[$\Q(\sqrt{d})$のホムセット]
    \label[example]{ex:q-sqrt-d-homset}
    $d \neq 1$を square-free な整数とし (e.g. $d = 6$)、$L = \Q(\sqrt{d})$とする。
    $\ch L = 0$なので、$L/\Q$は分離拡大である (\cref{prop:perfect-field})。
    よって$|\Hom_\Q^\al (L, \wb{\Q})| = 2$である
    (\cref{prop:separable-degree-homset-cardinality})\TODO{?}。
    $\sigma \in \Hom_\Q^\al (L, \wb{\Q})$とすると、
    $L$が$\Q$の代数拡大であることから、\cref{prop:conjugate-preserved-under-morphism}より
    $\sigma(\sqrt{d})$は$\sqrt{d}$の$\Q$上の共役、
    すなわち$\sigma(\sqrt{d}) = \pm \sqrt{d}$である (\cref{ex:conjugate})。
    $L$は$\Q$上$\sqrt{d}$で生成されるので、
    $\sigma$は$\sqrt{d}$での値で定まる (\cref{prop:field-homomorphism})。
    $\sigma$はちょうど2通りあるので、両方の可能性が起きなければならない。
    そこで$\sigma$を$\sigma(\sqrt{d}) = -\sqrt{d}$なるものとすれば、
    $\Hom_\Q^\al (L, \wb{\Q}) = \{ \id_L, \sigma \}$と決まる。
\end{example}



% ------------------------------------------------------------
%
% ------------------------------------------------------------
\section{正規拡大}

\begin{definition}[正規拡大]
    $L/K$を代数拡大とする。
    すべての$\alpha \in L$に対し
    $\alpha$の$K$上の最小多項式が$L$上で1次式の積になるとき、
    $L/K$を
    \term{正規拡大}[normal extension]{正規拡大}[せいきかくだい]
    という。
\end{definition}

次の定理により、正規拡大かどうかはホムセットをみることで判定できる。

\begin{theorem}[正規拡大とホムセット]
    \label[theorem]{thm:normal-extension-homset}
    $L/K$を体の有限次拡大とする。
    このとき、次は同値である:
    \begin{enumerate}
        \item $L/K$は正規拡大である。
        \item $\Hom_K^\al (L, \wb{K})$の元は$L$の元を固定する。
    \end{enumerate}
\end{theorem}

\begin{proof}
    cf. [雪江] p.185
\end{proof}

正規拡大のうちとくに重要なのは、
ホムセットが自己同型となる場合である。

\begin{proposition}[ホムセットが自己同型群となる場合]
    \label[proposition]{prop:homset-automorphism}
    $L/K$を正規代数拡大とする。
    このとき$\Hom_K^\al (L, L) = \Aut_K^\al L$である。
\end{proposition}

\begin{proof}
    cf. [雪江] p.185
\end{proof}

\begin{example}[正規拡大の例]
    \label[example]{ex:normal-extension}
    $d \neq 1$を square-free な整数とする (e.g. $d = 6$)。
    \cref{ex:q-sqrt-d-homset}より各$\phi \in \Hom_K^\al (L, \wb{K})$は
    $\phi(\Q(\sqrt{d})) \subset \Q(\sqrt{d})$をみたすから、
    \cref{thm:normal-extension-homset}より
    $\Q(\sqrt{d})/\Q$は正規拡大である。
\end{example}

\begin{definition}[最小分解体]
    $K$を体とし、$f(x) \in K[x]$とする。$f(x)$を
    \begin{equation}
        f(x) = a_0 (x - \alpha_1) \dots (x - \alpha_n)
        \quad (a_0 \in K^\times,\; \alpha_i \in \wb{K})
    \end{equation}
    と表すとき、
    $K(\alpha_1, \dots, \alpha_n)$を$f$の$K$上の
    \term{最小分解体}[splitting field]{最小分解体}[さいしょうぶんかいたい]
    という。
\end{definition}

\begin{example}[最小分解体の例]
    \TODO{cf. [雪江] p.186}
\end{example}



% ------------------------------------------------------------
%
% ------------------------------------------------------------
\section{Galois 拡大}

\TODO{キーワード:
    Galois の基本定理、円分体、有限体、Kummer 理論、Artin-Schreier 理論、可解性、作図}

分離性と正規性を兼ね備えた拡大が Galois 拡大である。

\begin{definition}[Galois 拡大]
    $L/K$を代数拡大とする。
    \begin{itemize}
        \item $L/K$が分離拡大かつ正規拡大なら
        \term{Galois 拡大}[Galois extension]{Galois 拡大}[Galoisかくだい]
        という。
    \end{itemize}
    $L/K$をさらにガロア拡大とする。
    \begin{itemize}
        \item $\Aut_K^\al L$を$\Gal(L/K)$と書き、
            $L$の$K$上の
            \term{Galois 群}[Galois group]{Galois 群}[Galoisぐん]という。
        \item $\Gal(L/K)$がアーベル群なら、$L/K$を
            \term{アーベル拡大}[abelian extension]{アーベル拡大}[あーべるかくだい]という。
        \item $\Gal(L/K)$が巡回群なら、$L/K$を
            \term{巡回拡大}[cyclic extension]{巡回拡大}[じゅんかいかくだい]という。
    \end{itemize}
\end{definition}

\begin{definition}[多項式の Galois 群]
    $K$を体、$f(x) \in K[x]$とし、
    $L$を$f(x)$の$K$上の最小分解体とする。
    $\Gal(L/K)$を$f(x)$の$K$上の
    \term{Galois 群}[Galois group]{Galois 群}[Galoisぐん]という。
\end{definition}

次の例より、Galois 群の元は複素共役の一般化とみなせることがわかる。

\begin{example}[Galois 拡大の例1]
    体の拡大$\C/\R$は
    \cref{prop:perfect-field}と\cref{thm:normal-extension-homset}により
    分離拡大かつ正規拡大だから、Galois 拡大である。
    \cref{ex:q-sqrt-d-homset}と同様の議論により
    $|\Hom_\R^\al(\C, \C)| = 2$であるから、
    \cref{prop:homset-automorphism}より
    $|\Gal(\C/\R)| = |\Aut_\R^\al(\C)| = 2$である。
    したがって$\Gal(\C/\R) \cong \Z/2\Z$である。
\end{example}

\begin{example}[Galois 拡大の例2]
    $d \neq 1$は square-free な整数とする(e.g. $d = 6$)。
    \cref{ex:q-sqrt-d-homset}と\cref{ex:normal-extension}により、
    代数拡大$\Q(\sqrt{d})$は分離拡大かつ正規拡大だから、
    Galois 拡大である。
    \cref{prop:homset-automorphism}より$\Gal(\Q(\sqrt{d})/\Q) \cong \Z/2\Z$が従う。
\end{example}

\begin{theorem}[Galois 群は対称群の部分群]
    $K$を体とし、$f(x) \in K[x]$を$\deg f(x) = n$なる分離多項式とする。
    このとき、$f(x)$の$K$上の Galois 群は
    対称群$S_n$の部分群と同型である。
\end{theorem}

\begin{proof}
    $f(x)$の相異なる$n$個の根を$\alpha_1, \dots, \alpha_n \in \wb{K}$とおくと、
    $f(x)$の$K$上の Galois 群は$L \coloneqq K(\alpha_1, \dots, \alpha_n)$と表せる。
    $\sigma \in \Gal(L/K)$は$\sigma$の
    $A \coloneqq \{\alpha_1, \dots, \alpha_n\}$上での値で決まるから、
    \begin{equation}
        \Gal(L/K) \to S_n,
        \quad \sigma \mapsto \sigma|_A
    \end{equation}
    は単射準同型である。
\end{proof}

% ------------------------------------------------------------
%
% ------------------------------------------------------------
\section{不変体と Artin の定理}

\begin{definition}[不変体]
    \label[definition]{def:fixed-field}
    $L$を体、$G$を有限群とし、
    $G$は$L$に忠実に作用しているとする。
    このとき、
    \begin{equation}
        L^G \coloneqq \{ \alpha \in L \colon g \cdot \alpha = \alpha \; (\forall g \in G) \}
    \end{equation}
    を$G$の
    \term{不変体}[fixed field]{不変体}[ふへんたい]という。
\end{definition}

\begin{proposition}[Artin の定理]
    \cref{def:fixed-field}の設定のもとで、
    $L/L^G$は Galois 拡大であり、$\Gal(L/L^G) \cong G$が成り立つ。
\end{proposition}

% ------------------------------------------------------------
%
% ------------------------------------------------------------
\section{Galois 理論の基本定理}

\begin{proposition}[中間体の束]
    $L/K$を体の拡大とする。
    $\Lat(L/K)$を$L/K$の中間体全体の集合とし、
    $\Lat(L/K)$上に半順序$\preceq$を
    \begin{equation}
        B \preceq C \quad \Leftrightarrow \quad B \subset C
    \end{equation}
    で定めると、$(\Lat(L/K), \preceq)$は
    共通部分を交わり、合成体を結びとして束となる。
\end{proposition}

\begin{proof}
    省略
\end{proof}

次の補題は Galois 拡大の分離性と正規性を利用するもので、
Galois 理論の基本定理の証明に重要な役割を果たす。

\begin{lemma}[中間体と Galois 拡大]
    $L/K$を有限次 Galois 拡大とし、
    $M \in \Lat(L/K)$とする。
    このとき、$L/M$は Galois 拡大である。
\end{lemma}

\begin{proof}
    \TODO{}
\end{proof}

\begin{theorem}[Galois 理論の基本定理]
    $L/K$を有限次 Galois 拡大とし、Galois 群を$G = \Gal(L/K)$とする。
    \begin{enumerate}
        \item 写像$\gamma \colon \Sub(G) \to \Lat(L/K),$
            \begin{equation}
                H \mapsto L^H
            \end{equation}
            は order-reversing な全単射であり、逆写像は
            \begin{equation}
                \Gal(L/M) \mapsfrom M
            \end{equation}
            で与えられる。
        \item $M \in \Lat(L/K)$に関し
            \begin{equation}
                M/K \text{ が Galois 拡大}
                \Longleftrightarrow
                \Gal(L/M) \text{ が } G \text{ の正規部分群}
            \end{equation}
            が成り立つ。
    \end{enumerate}
\end{theorem}

\begin{proof}
    不変体の定義から order-reversing であることは明らか。
    \TODO{}
\end{proof}

% ------------------------------------------------------------
%
% ------------------------------------------------------------
\section{Hilbert の定理90}

\begin{proof}
    cf. [雪江] p.197
\end{proof}

\begin{theorem}[Galois 拡大の推進定理]
    \TODO{cf. [雪江] p.219}
\end{theorem}

\begin{definition}[Galois コホモロジー]
    \TODO{}
\end{definition}

\begin{theorem}[Hilbert の定理90]
    \TODO{}
\end{theorem}








\end{document}

% ============================================================
%
% ============================================================
\newpage
\phantomsection
\addcontentsline{toc}{part}{演習問題の解答}
\part*{演習問題の解答}

\includecollection{answers}

% ============================================================
%
% ============================================================
\newpage
\phantomsection
\addcontentsline{toc}{part}{参考文献}
\renewcommand{\bibname}{参考文献}
\markboth{\bibname}{}
\bibliographystyle{amsalpha}
\bibliography{../mybibliography}

% ============================================================
%
% ============================================================
\newpage
\phantomsection
\addcontentsline{toc}{part}{記号一覧}
\printglossary[title={記号一覧}]

% ============================================================
%
% ============================================================
\newpage
\phantomsection
\addcontentsline{toc}{part}{索引}
\printindex

\end{document}