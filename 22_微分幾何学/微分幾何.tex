\documentclass[report, notitlepage]{jlreq}
\usepackage{docmute}
\usepackage{global}
\usepackage{./sub/local}
\def\assetspath{./}
\makeindex
\makeglossaries

\title{可微分多様体と微分幾何学の基礎}
\author{Yahata}
\date{}

\begin{document}

\maketitle
\begin{abstract}
    多様体とは、曲線や曲面といった図形の概念を一般化したものであり、
    微分幾何学に限らず様々な幾何学の土俵となっている。
    微分幾何学は、多様体の上で微積分を用いて展開される幾何学であり、
    一般相対性理論をはじめとして物理学に多くの応用がある。

    本稿では可微分多様体および微分幾何学の基礎事項を整理する。
    第1部では可微分多様体の基礎と de Rham コホモロジーについて述べる。
    内容は\cite{Lee12}を参考にしている。
    第2部ではベクトル束と主ファイバー束の接続について述べる。
    扱う内容は\cite{小林04}をベースとし、
    定義などは\cite{Lee18}や\cite{Tu17}を参考にしている。
    第3部では擬 Riemann 多様体と計量について述べる。
    内容は\cite{Lee18}を参考にしている。
\end{abstract}

\setcounter{tocdepth}{1}
\tableofcontents
\markboth{\contentsname}{}

% ============================================================
%
% ============================================================
\part{可微分多様体の基礎}
\documentclass[report]{jlreq}
\usepackage{global}
\usepackage{./local}
\subfiletrue
%\makeindex
\begin{document}


% ============================================================
%
% ============================================================
\chapter{群}

群について述べる。

% ------------------------------------------------------------
%
% ------------------------------------------------------------
\section{群}

\begin{definition}[モノイド]
    $M$を集合、
    $e \in M$、
    $\cdot \colon M \times M \to M$を写像とし、
    各$x, y \in M$に対し$\cdot (x, y)$を
    $x \cdot y$や$xy$と書くことにする。
    組$(M, \cdot, e)$が
    \term{モノイド}[monoid]{モノイド}
    であるとは、次が成り立つことをいう:
    \begin{description}
        \item[(M1) 結合律]
            各$x, y, z \in M$に対して
            $(x \cdot y) \cdot z = x \cdot (y \cdot z)$
            が成り立つ。
        \item[(M2) 単位元]
            各$x \in M$に対して
            $x \cdot e = x = e \cdot x$
            が成り立つ。
    \end{description}
    組$(M, \cdot, e)$のことを
    記号の濫用で単に$(M, \cdot)$や$M$と書くことがある。
    さらに
    \begin{itemize}
        \item $e$を$M$の
            \term{単位元}[unit]{単位元}[たんいげん]という。
    \end{itemize}
\end{definition}

\begin{definition}[群]
    モノイド$(G, \cdot, e)$が
    \term{群}[group]{群}[ぐん]であるとは、
    次が成り立つことをいう:
    \begin{description}
        \item[(G1) 逆元]
            各$x \in G$に対して
            ある$y \in G$が存在して
            $x \cdot y = e = y \cdot x$
            が成り立つ。
    \end{description}
    さらに
    \begin{itemize}
        \item $y$を$x$の
            \term{逆元}[inverse]{逆元}[ぎゃくげん]といい、
            $x^{-1}$と書く。
    \end{itemize}
\end{definition}

\begin{definition}[アーベル群]
    群$(G, +, 0)$が
    \term{アーベル群}[abelian group]{アーベル群}[あーべるぐん]であるとは、
    次が成り立つことをいう:
    \begin{description}
        \item[(A1) 可換性]
            各$x, y \in G$に対して
            $x + y = y + x$
            が成り立つ。
    \end{description}
\end{definition}

\begin{definition}[群準同型]
    \TODO{}
\end{definition}



% ------------------------------------------------------------
%
% ------------------------------------------------------------
\section{部分群}

\begin{proposition}[部分群の特徴付け]
    \TODO{}
\end{proposition}

\begin{proof}
    \TODO{}
\end{proof}

\begin{definition}[生成された部分群]
    $G$を群、$S \subset G$とする。
    このとき、集合
    \begin{equation}
        \langle S \rangle
            \coloneqq \{
                g_1^{\eps_1} \cdots g_n^{\eps_n}
                \mid
                n \in \Z_{\ge 1}, \;
                g_i \in S, \;
                \eps_i \in \{ \pm 1 \}
            \}
    \end{equation}
    は定義から明らかに$G$の部分群となる。
    $\langle S \rangle$を
    \term{$S$により生成された$G$の部分群}[subgroup of $G$ generated by $S$]
        {生成された部分群}[せいせいされたぶぶんぐん]
    といい、
    $S$を$\langle S \rangle$の
    \term{生成系}[generating set]{生成系}[せいせいけい]
    という。

    $G$が有限集合$S$により生成されるとき、
    $G = \langle S \rangle$は
    \term{有限生成}[finitely generated]{有限生成}[ゆうげんせいせい]
    であるといい、
    さらに$S$が1点集合$S = \{ x \}$のとき
    波括弧を省略して$\langle x \rangle$と書き、
    $G = \langle x \rangle$はa
    \term{巡回群}[cyclic group]{巡回群}[じゅんかいぐん]
    であるという。
\end{definition}

\begin{proposition}[生成された部分群の特徴付け]
    $G$を群、$S \subset G$とする。
    このとき
    \begin{equation}
        \langle S \rangle
            = \bigcap_{\substack{
                G' \subset G \colon \text{部分群} \\
                G' \supset S
            }} G'
    \end{equation}
    が成り立つ。
\end{proposition}

\begin{proof}
    \TODO{}
\end{proof}



% ------------------------------------------------------------
%
% ------------------------------------------------------------
\section{群作用}

群の作用について述べる。

\begin{definition}[作用]
    $G$を群、$X$を集合とする。
    写像
    \begin{equation}
        G \times X \to X,
        \quad
        (g, x) \mapsto gx
    \end{equation}
    が与えられていて
    \begin{enumerate}
        \item 各$g_1, g_2 \in G, \; x \in X$に対して
            $(g_1 g_2) x = g_1 (g_2 x)$が成り立つ。
        \item 各$x \in X$に対して$e_G x = x$が成り立つ。
    \end{enumerate}
    をみたすとき、
    $G$は$X$に左から\term{作用}[act]{作用}[さよう]するという。
    $G$が左から作用している集合を
    \term{左$G$-集合}[left $G$-set]{$G$-集合}[Gしゅうごう]
    という。
    右からの作用も同様に定まる。
\end{definition}

\begin{definition}[軌道]
    $G$を群、$X$を左$G$-集合とする。
    $X$上の同値関係を
    \begin{equation}
        \text{$x$と$y$が同値}
        \quad \logeq \quad
        \exists g \in G \quad \text{s.t.} \quad gx = y
    \end{equation}
    で定めることができ、
    この同値関係に関する同値類を
    \term{軌道}[orbit]{軌道}[きどう]
    という。
\end{definition}

\begin{definition}[固定部分群]
    \idxsym{stabilizer}{$\Stab_G(x)$}{$x$の固定部分群}
    $G$を群、$X$を左$G$-集合とする。
    各$x \in X$に対し、$G$の部分群
    \begin{equation}
        \Stab_G(x) \coloneqq \{ g \in G \colon xg = x \}
    \end{equation}
    を$x$の
    \term{固定部分群}[stabilizer]{固定部分群}[こていぶぶんぐん]
    という。
\end{definition}

\begin{definition}[忠実作用]
    $G$を群、$X$を左$G$-集合とする。
    $G$の$X$への作用が
    \term{忠実}[faithful]{忠実}[ちゅうじつ]
    あるいは
    \term{効果的}[effective]{効果的}[こうかてき]
    であるとは、次が成り立つことをいう:
    \begin{itemize}
        \item すべての$x \in X$を
            固定する$g \in G$は単位元のみである。
    \end{itemize}
    定義から明らかに、作用が忠実であることは
    作用の定める表現$G \to \Aut(X)$が単射であることと同値である。
\end{definition}

\begin{definition}[自由作用]
    $G$を群、$X$を左$G$-集合とする。
    $G$の$X$への作用が
    \term{自由}[free]{自由}[じゆう]
    であるとは、
    単位元以外の$g \in G$はすべての$x \in X$を動かすように作用すること、すなわち
    \begin{equation}
        \forall g \in G \; (g \neq 1 \Rightarrow (\forall x \in X \; (xg \neq x)))
    \end{equation}
    が成り立つことをいう。
    これはすべての$x \in X$に対し
    $\Stab_G(x)$が自明群であることと同値である。
\end{definition}

\begin{definition}[推移的作用]
    $G$を群、$X$を左$G$-集合とする。
    各$x \in X$に対し$xG \coloneqq \{ xg \in X \colon g \in G \}$と書く。
    $G$の$X$への作用が
    \term{推移的}[transitive]{推移的}[すいいてき]
    であるとは、
    \begin{equation}
        X = xG \quad (\forall x \in X)
    \end{equation}
    が成り立つことをいう。これは次と同値である:
    \begin{itemize}
        \item $\forall x_0 \in X$を固定すると、
            $\forall y \in X$に対し$\exists g \in G$がとれて$y = x_0 g$が成り立つ。
    \end{itemize}
\end{definition}

\subsection{$G$-torsor}

\begin{definition}[$G$-torsor]
    $G$を群、$X$を非空な左$G$-集合とする。
    \term{shear map}{shear map}
    と呼ばれる写像
    \begin{equation}
        G \times X \to X \times X,
        \quad
        (g, x) \mapsto (gx, x)
    \end{equation}
    が全単射であるとき、
    $X$を\term{$G$-torsor}{$G$-torsor}[G-torsor]
    という。
\end{definition}

\begin{proposition}[$G$-torsor の特徴付け]
    $G$を群、$X$を左$G$-集合とする。
    このとき、次は同値である:
    \begin{enumerate}
        \item $X$は$G$-torsorである。
        \item $G$の$X$への作用は推移的かつ自由である。
        \item $G$の$X$への作用は推移的であり、さらに
            固定部分群が自明群であるような$x \in X$が存在する。
        \item $X$と$G$は左$G$-集合として同型である。
    \end{enumerate}
\end{proposition}

\begin{proof}
    \TODO{}
\end{proof}

\begin{theorem}[類等式]
    \TODO{}
\end{theorem}

\begin{proof}
    \TODO{}
\end{proof}

\begin{theorem}[Lagrange]
    \TODO{}
\end{theorem}

\begin{proof}
    \TODO{}
\end{proof}



% ------------------------------------------------------------
%
% ------------------------------------------------------------
\section{商群}



% ------------------------------------------------------------
%
% ------------------------------------------------------------
\section{準同型定理}

\begin{theorem}[準同型定理]
    \TODO{}
\end{theorem}

\begin{proof}
    \TODO{}
\end{proof}

\begin{theorem}[部分群の対応原理]
    \TODO{}
\end{theorem}

\begin{proof}
    \TODO{}
\end{proof}


% ------------------------------------------------------------
%
% ------------------------------------------------------------
\section{Sylow の定理}

\begin{theorem}[Sylow]
    \TODO{}
\end{theorem}

\begin{proof}
    \TODO{}
\end{proof}



% ------------------------------------------------------------
%
% ------------------------------------------------------------
\section{群の表現}
\label[section]{section:group-action}

\TODO{群の作用とはどう違う?}

\begin{definition}[群の表現]
    $G$を群、$\calC$を圏とする。
    $G$は、射を群の元とし単一の対象$*$からなる圏とみなせる。
    $\calC$における$G$の
    \term{表現}[representation]{表現}[ひょうげん]
    とは、圏$G$から$\calC$への関手のことである。
    $T \colon G \to \calC$を表現とするとき、
    各射$T(g)$は$\calC$の対象$X \coloneqq T(*)$上の自己同型射を与えるから、
    群準同型$G \to \Aut(X)$が定まる。
    この群準同型も\term{表現}[representation]{表現}[ひょうげん]と呼ぶ。
\end{definition}

\begin{remark}
    群の作用は
    集合の圏における群の表現
    (これを\term{置換表現}[permutation representation]{置換表現}[ちかんひょうげん]という)
    に他ならない。
\end{remark}

\begin{example}
    ~
    \begin{itemize}
        \item 有限群の表現
        \item 位相群の表現
        \item Lie 群の表現
        \item \TODO{}
    \end{itemize}
\end{example}

% ------------------------------------------------------------
%
% ------------------------------------------------------------
\section{自由群}

% ------------------------------------------------------------
%
% ------------------------------------------------------------
\section{自由積と融合積}

% ------------------------------------------------------------
%
% ------------------------------------------------------------
\section{アーベル化}

\begin{theorem}[アーベル化の普遍性]
    \TODO{}
\end{theorem}

\begin{proof}
    \TODO{}
\end{proof}

% ------------------------------------------------------------
%
% ------------------------------------------------------------
\section{可解群}




% ============================================================
%
% ============================================================
\chapter{基本的な群}

% ------------------------------------------------------------
%
% ------------------------------------------------------------
\section{対称群}

% ------------------------------------------------------------
%
% ------------------------------------------------------------
\section{2面体群}

% ------------------------------------------------------------
%
% ------------------------------------------------------------
\section{4元数群}

% ------------------------------------------------------------
%
% ------------------------------------------------------------
\section{一般線型群}




\end{document}
\documentclass[report]{jlreq}
\usepackage{global}
\usepackage{./local}
\subfiletrue
\begin{document}

第1部では Riemann 積分および Lebesgue 積分について整理する。

% ============================================================
%
% ============================================================
\chapter{Riemann 積分}

% ------------------------------------------------------------
%
% ------------------------------------------------------------
\section{Riemann 積分の定義}

点付き分割の概念を定義する。
なお、ここで定義する「分割」という言葉の意味は
一般の集合論における分割とは異なることに注意すべきである。

\begin{definition}[点付き分割]
    $I = [a_1, b_1] \times \cdots \times [a_n, b_n] \subset \R^n$を有界閉区間とする。
    組$(X, T)$が
    $I$の\term{点付き分割}[tagged partition]{点付き分割}[てんつきぶんかつ]であるとは、
    次が成り立つことをいう:
    \begin{enumerate}
        \item $X$は順序組$X = (X_1, \dots, X_n)$であって、
            各$X_i \; (i = 1, \dots, n)$は
            $[a_i, b_i]$の有限部分集合
            \begin{equation}
                X_i = \mybrace{
                        a_i = x_{i, 0} < x_{i, 1} < \cdots < x_{i, k_i} = b_i
                    }
                    \quad
                    (k_i \in \Z_{\ge 0})
            \end{equation}
            である。
            $X$を$I$の\term{分割}[partition]{分割}[ぶんかつ]といい、
            \begin{equation}
                I_{i_1 \dots i_n}
                    \coloneqq
                    [x_{1, i_1 - 1}, x_{1, i_1}]
                    \times \cdots \times
                    [x_{n, i_n - 1}, x_{n, i_n}]
                    \quad
                    (1 \le i_1 \le k_1, \dots, 1 \le i_n \le k_n)
            \end{equation}
            を$X$の定める$I$の
            \term{小区間}[subinterval]{小区間}[しょうくかん]という。
        \item $T$は$X$の定める各小区間を "代表" する点の集まりである。
            すなわち
            \begin{equation}
                T \coloneqq \mybrace{
                    t_{i_1 \dots i_n}
                    \mid
                    t_{i_1 \dots i_n} \in I_{i_1 \dots i_n}, \;
                    1 \le i_1 \le k_1, \dots, 1 \le i_n \le k_n
                }
            \end{equation}
            である。
    \end{enumerate}
    $I$の点付き分割全部の集合を$\calP(I)$と書くことにする。
\end{definition}

\begin{definition}[細分]
    $I \subset \R^n$を有界閉区間、
    $(X, T), (X', T')$を$I$の点付き分割とする。
    $(X', T')$が$(X, T)$の\term{細分}[refinement]{細分}[さいぶん]であるとは、
    $X_i \subset X'_i \; (i = 1, \dots, n)$
    が成り立つことをいう。
    このとき$(X, T) \preceq (X', T')$と書く。
\end{definition}

\begin{lemma}
    上の定義の状況で
    $(\calP(I), \preceq)$は有向集合となる。
\end{lemma}

\begin{proof}
    反射性と推移性は包含関係の性質から明らか。
    また、2つの点付き分割$(X, T), (X', T')$に対して
    $X'' \coloneqq X \cup X'$は$I$の分割であり、
    $X''$の定める各小区間から1個ずつ点を選んだものを$T''$とおけば、
    $(X'', T'')$は$I$の点付き分割であって
    $(X, T), (X', T')$の細分である。
    したがって共通上界の存在もいえた。
    よって$(\calP(I), \preceq)$は有向集合である。
\end{proof}

\begin{definition}[Riemann 和]
    $I = [a_1, b_1] \times \cdots \times [a_n, b_n] \subset \R^n$を有界閉区間、
    $f \colon I \to \R$を関数とする。
    $\R$内のネット$S_f \colon \calP(I) \to \R$を次のように定める:
    \begin{equation}
        S_f(X, T)
            \coloneqq
            \sum_{i_1, \dots, i_n} f(t_{i_1 \dots i_n})
            (x_{1, i_1} - x_{1, i_1 - 1}) \cdots (x_{n, i_n} - x_{n, i_n - 1})
    \end{equation}
    各$S_f(X, T)$を$(X, T)$に関する$f$の
    \term{Riemann 和}[Riemann sum]{リーマン和}[Riemann わ]という。
\end{definition}

Riemann 積分をネット$S_f$の極限として定義する。

\begin{definition}[Riemann 積分]
    \idxsym{Riemann integral}{$\int_I f(x) dx$}{$f$の$I$上の Riemann 積分}
    上の定義の状況で、
    ネット$S_f$が収束するとき
    $f$は$I$上
    \term{Riemann 可積分}[Riemann integrable]{リーマン可積分}[Riemann かせきぶん]
    であるといい、
    $S_f$の極限\footnote{
        $\R$は Hausdorff だから$S_f$の極限は存在すれば一意である。
    }を$f$の$I$上の
    \term{Riemann 積分}[Riemann integral]{リーマン積分}[Riemann せきぶん]
    といい、$\int_I f(x) dx$と書く。
\end{definition}

通常、Riemann 積分といったら
点付き分割の「幅」を$0$に近づけるときの Riemann 和の極限を指す\footnote{
    点付き分割の幅を$0$に近づける Riemann 積分の定義は
    ネットの収束の定義と相性が悪い。
    すなわち、ネットの収束は
    「任意の$\eps > 0$に対し、ある$\lambda_0 \in \Lambda$が存在して、
    $\lambda \succeq \lambda_0$なるすべての$\lambda \in \Lambda$に対し、…」
    という形で定式化され、
    「ある〜」の部分$\lambda_0$と「…なるすべての〜」
    の部分$\lambda$が同じ有向集合に属しているが、
    一方で点付き分割の幅を$0$に近づける Riemann 積分の定義は
    「任意の$\eps > 0$に対し、ある$\delta > 0$が存在して、
    $|(X, T)| < \delta$なるすべての$(X, T)$に対し、…」
    の形だから、
    $\delta$と$(X, T)$が同じ有向集合に属していない。
}。
そのような定義と上の定義との同値性を示そう。

\begin{proposition}
    \TODO{}
\end{proposition}

\begin{proof}
    \TODO{}
\end{proof}

\begin{definition}[Darboux 積分]
    \TODO{}
\end{definition}

% ------------------------------------------------------------
%
% ------------------------------------------------------------
\newpage
\section{演習問題}

\begin{problem}[東大数理 2006A]
    
\end{problem}

\begin{answer}
    
\end{answer}




% ============================================================
%
% ============================================================
\chapter{測度}

% ------------------------------------------------------------
%
% ------------------------------------------------------------
\section{
    \texorpdfstring{%
        有限加法族と$\sigma$-加法族%
    }{%
        有限加法族と sigma 加法族%
    }%
}

位相空間における開集合系のように、
与えられた集合$X$に対して "良い" 性質を持った部分集合系を定義することを考えよう。

\begin{definition}[有限加法族]
    \TODO{}
\end{definition}

\begin{definition}[$\sigma$-加法族]
    \TODO{}
\end{definition}

\begin{definition}[可測空間]
    集合$X$と$X$の部分集合の$\sigma$-加法族$\calF$の組
    $(X, \calF)$を
    \term{可測空間}[measurable space]{可測空間}[かそくくうかん]
    という。
\end{definition}

% ------------------------------------------------------------
%
% ------------------------------------------------------------
\section{有限加法的な実数値集合関数}

測度の導入の準備として、
有限加法的な実数値集合関数について述べる。

\begin{definition}[集合関数]
    $X$を集合、
    $\calA$を$X$の部分集合の族とする。
    $\calA$上の関数であって
    $[-\infty, +\infty]$の部分集合に値を持つものを
    $X$上の (あるいは$\calA$上の)
    \term{集合関数}[set function]{集合関数}[しゅうごうかんすう]
    という。
\end{definition}

\subsection{有限加法的な実数値集合関数}

実数値集合関数のなかでもとくに、
次に定義する有限加法性を持つものは
有限加法族と相性が良い。

\begin{definition}[有限加法性]
    $X$を集合、
    $\calA$を$X$の部分集合の族\TODO{最初から有限加法族ではだめ?}、
    $\mu$を$\calA$上の集合関数とする。
    \begin{enumerate}
        \item $\mu$が$(-\infty, \infty)$に値を持つとする。
            $\mu$が
            \term{有限加法的}[finitely additive]{有限加法的}[ゆうげんかほうてき]
            であるとは、
            有限個の互いに素な任意の$A_1, \dots, A_n \in \calA$
            であって$\bigcup_{i = 1}^n A_i \in \calA$なるものに対し
            \begin{equation}
                \mu\myparen{\bigcup_{i = 1}^n A_i} = \sum_{i = 1}^n \mu(A_i)
            \end{equation}
            が成り立つことをいう。
    \end{enumerate}
\end{definition}

\subsection{有限加法的な非負実数値集合関数}

有限加法的な実数値集合関数がさらに非負の場合を考える。

\begin{proposition}[有限加法的な非負実数値集合関数の基本性質]
    $X$を集合、
    $\calA$を$X$の部分集合の有限加法族、
    $\mu$を$\calA$上の有限加法的な非負実数値集合関数とする。
    このとき次が成り立つ:
    \begin{enumerate}
        \item (単調性) 各$A, B \in \calA$に対し、
            $A \subset B \implies \mu(A) \leq \mu(B)$
            が成り立つ。
        \item (有限劣加法性) 各$A_1, \dots, A_n \in \calA$に対し、
            \begin{equation}
                \mu\myparen{\bigcup_{i = 1}^n A_i}
                    \le \sum_{i = 1}^n \mu(A_i)
            \end{equation}
            が成り立つ。
    \end{enumerate}
\end{proposition}

\begin{proof}
    \TODO{}
\end{proof}


% ------------------------------------------------------------
%
% ------------------------------------------------------------
\section{測度}

有限の加法性だけでは理論的に弱すぎるため、
加法性を可算まで強めたものを考える必要がある。

\begin{definition}[可算加法性]
    $X$を集合、
    $\calA$を$X$の部分集合の族\TODO{最初から有限加法族ではだめ?}、
    $\mu$を$\calA$上の集合関数とする。
    \begin{enumerate}
        \item $\mu$が\highlight{$(-\infty, +\infty)$}に値を持つとする。
            $\mu$が
            \term{可算加法的}[countably additive]{可算加法的}[かさんかほうてき]
            であるとは、
            互いに素な任意の列$(A_i)_{i = 1}^\infty, \, A_i \in \calA$
            であって$\bigcup_{i = 1}^\infty A_i \in \calA$なるものに対し
            \begin{equation}
                \mu\myparen{\bigcup_{i = 1}^\infty A_i}
                    = \sum_{i = 1}^\infty \mu(A_i)
            \end{equation}
            が成り立つことをいう。
        \item $\mu$が\highlight{$[0, +\infty]$}に値を持つとする。
            $\mu$が
            \term{可算加法的}[countably additive]{可算加法的}[かさんかほうてき]
            であるとは、
            互いに素な任意の列$(A_i)_{i = 1}^\infty, \, A_i \in \calA$
            であって$\bigcup_{i = 1}^\infty A_i \in \calA$なるものに対し
            \begin{equation}
                \mu\myparen{\bigcup_{i = 1}^\infty A_i}
                    = \sum_{i = 1}^\infty \mu(A_i)
            \end{equation}
            が成り立つことをいう。
            ただし、両辺が同時に$\infty$となることも許す。
    \end{enumerate}
\end{definition}

\begin{remark}
    可算加法性の定義 (1) の右辺に現れる級数は
    絶対収束が要求されていることに注意すべきである。
    実際、$\bigcup A_i \in \calA$ゆえに
    $\mu\myparen{\bigcup A_i}$は実数だから
    右辺$\sum \mu(A_i)$は収束する。
    さらに左辺、したがって右辺は添字の並べ替えで値が変わらないから
    $\sum \mu(A_i)$は無条件収束であり、
    Riemann の級数定理
    (\cref{thm:riemann-series-theorem})
    より絶対収束となる。
\end{remark}

符号付き測度を定義する。

\begin{definition}[符号付き測度]
    $(X, \calA)$を可測空間とする。
    \begin{itemize}
        \item $(X, \calA)$上の可算加法的な実数値集合関数$\mu$を
            $(X, \calA)$上の
            \term{符号付き測度}[signed measure]{符号付き測度}[ふごうつきそくど]
            という\footnote{
                符号付き測度は
                \term{複素測度}[complex measure]{複素測度}[ふくそそくど]
                の実数値の場合である。
            }。
    \end{itemize}
\end{definition}

測度を定義する。

\begin{definition}[測度]
    $(X, \calA)$を可測空間とする。
    \begin{itemize}
        \item $\calA$上の集合関数$\mu$であって
            次をみたすものを、$(X, \calA)$上の
            \term{測度}[measure]{測度}[そくど]
            という:
            \begin{enumerate}
                \item $[0, +\infty]$に値を持つ。
                \item 可算加法性をみたす。
                \item $\mu(\emptyset) = 0$
            \end{enumerate}
        \item $(X, \calA)$上の測度$\mu$であって次をみたすものを
            $(X, \calA)$上の
            \term{$\sigma$-有限測度}[$\sigma$-finite measure]
                {$\sigma$-有限測度}[sigmaゆうげんそくど]
            という:
            \begin{enumerate}
                \item $\mu(E_n) < \infty$
                    かつ$X = \bigcup_{n = 1}^\infty E_n$なる
                    $E_n \in \calA \; (n = 1, 2, \dots)$が存在する。
            \end{enumerate}
        \item $(X, \calA)$上の測度$\mu$であって
            $[0, +\infty)$に値を持つものを
            $(X, \calA)$上の
            \term{有限測度}[finite measure]
                {有限測度}[ゆうげんそくど]
            という。
    \end{itemize}
\end{definition}

\begin{example}[測度の例]
    ~
    \begin{itemize}
        \item 後で述べる Lebesgue 測度は
            $(\R, \calB(\R))$ ($\calB(\R)$は Borel 集合族)
            上の測度である。
        \item 数え上げ測度は
            $(\R, \calB(\R))$上の測度である。
        \item $(X, \calA)$を可測空間、$x \in X$とする。
            $\delta_x(A) \coloneqq \chi_A(x)$で定義される
            $(X, \calA)$上の測度を
            \term{Dirac 測度}[Dirac measure]{Dirac 測度}[Dirac そくど]
            という。
    \end{itemize}
\end{example}

とくに有限測度により測度空間が定義される。

\begin{definition}[測度空間]
    $(X, \calA)$を可測空間、
    $\mu$を$\calA$上の有限測度とする。
    このとき3つ組$(X, \calA, \mu)$を
    \term{測度空間}[measure space]{測度空間}[そくどくうかん]
    という。
\end{definition}

\begin{remark}
    $\mu$が単に測度というだけでは
    測度空間の要件を満たさないことに注意すべきである。
    しかし、可測空間$(X, \calA)$と$\calA$上の ($+\infty$も値に許す) 測度を
    あわせて扱う場面もしばしばあるから、
    混同しないようにしなければならない。
\end{remark}

\begin{definition}[完備な測度]
    $(X, \calA, \mu)$を測度空間とする\TODO{可測空間ではなく?}。
    $\mu$が\term{完備}[complete]{完備}[かんび]であるとは、
    $\mu(A) = 0$なる任意の$A \in \calA$に対し、
    すべての部分集合$B \subset A$が$\calA$に含まれることをいう。
\end{definition}

\begin{definition}[ほとんどいたるところ]
    $(X, \calA)$を可測空間、
    $\mu$を$\calA$上の測度とする。
    $x \in X$に関する命題$P(x)$が
    \begin{equation}
        \mu(\{ x \in X \mid P(x) \text{ は偽} \}) = 0
    \end{equation}
    をみたすとき、
    命題$P(x)$は
    $X$上
    \term{$\mu$-ほとんどいたるところ}[$\mu$-almost everywhere; $\mu$-a.e.]
        {ほとんどいたるところ}
    で成立するという。
\end{definition}


% ------------------------------------------------------------
%
% ------------------------------------------------------------
\section{外測度}

\TODO{どこに書くべき?}

% ------------------------------------------------------------
%
% ------------------------------------------------------------
\section{拡張定理}

\TODO{積測度もここ?}

\TODO{Riesz の表現定理?}




% ============================================================
%
% ============================================================
\chapter{Lebesgue 積分}

% ------------------------------------------------------------
%
% ------------------------------------------------------------
\section{可測関数}

可測関数の概念を定義する。
可測関数は「可測」という名前に反して
測度とは無関係に定義される概念であることに注意すべきである。

\begin{definition}[可測関数]
    $(X, \calA)$を可測空間、
    $f \colon X \to \R$を関数とする。
    このとき$f$が$\calA$に関し
    \term{可測}[measurable]{可測関数}[かそくかんすう]
    であるとは、
    任意の$c \in \R$に対し
    $f^{-1}((-\infty, c)) \in \calA$
    が成り立つことをいう。
\end{definition}

% ------------------------------------------------------------
%
% ------------------------------------------------------------
\section{単関数の積分}

単関数を定義する。
\TODO{$\sigma$-加法族や測度によらず定義されるべき?}

\begin{definition}[単関数]
    $(X, \calA)$を可測関数、
    $\mu$を$\calA$上の$[0, +\infty]$に値をもつ測度とする。
    可測関数$f \colon X \to \R$が
    $\mu$に関し
    \term{単関数}[simple function]{単関数}[たんかんすう]
    であるとは、
    $\mu(\{ x \in X \mid f(x) \neq 0 \}) < \infty$であって\footnote{
        $\mu$が有限測度の場合は
        $\mu(\{ x \in X \mid f(x) \neq 0 \}) < \infty$
        の条件は自然に満たされる。
    }、
    $f$を
    \begin{equation}
        f = \sum_{i = 1}^n c_i \chi_{A_i},
            \quad
            c_i \in \R,
            \quad
            A_i \in \calA
    \end{equation}
    と$\calA$の元の指示関数の$\R$-線型結合に書けることをいう。
\end{definition}

文献によっては単関数の定義に
「$\mu(\{ x \in X \mid f(x) \neq 0 \}) < \infty$」
を含めないものもあるが、
そのような流儀では、
$+\infty$を許す測度を扱う場合にのみこの条件を課すことになり、
議論が煩雑になってしまう。
したがって本稿では最初から単関数の定義にこの条件を含めることにした。

次に単関数の積分を定義する。

\begin{definition}[単関数の積分]
    $(X, \calA)$を可測空間、
    $\mu$を$\calA$上の$[0, +\infty]$に値をもつ測度とする。
    $\mu$に関する任意の単関数$f \colon X \to \R$
    \begin{equation}
        f = \sum_{i = 1}^n c_i \chi_{A_i},
            \quad
            c_i \in \R,
            \quad
            A_i \in \calA
    \end{equation}
    に対し、
    $f$の\term{積分}[integral]{積分}[せきぶん]を
    \begin{equation}
        \int_X f(x) \, \mu(dx)
            \coloneqq \sum_{i = 1}^n c_i \mu(A_i)
    \end{equation}
    と定義する。
\end{definition}

\begin{proposition}[単関数の積分の基本性質]
    \begin{enumerate}
        \item (三角不等式)
    \end{enumerate}
    \TODO{}
\end{proposition}

\begin{definition}[単関数の平均 Cauchy 列]
    $(X, \calA)$を可測空間、
    $\mu$を$X$上の測度、
    $(f_n)_{n \in \N}$を$X$上の$\mu$に関する単関数の列とする。
    $(f_n)_n$が
    \term{平均 Cauchy 列}[mean Cauchy sequence]{平均 Cauchy 列}[へいきん Cauchy れつ]
    であるとは、
    任意の$\eps > 0$に対し、
    ある$n \in \N$が存在して、
    すべての$i, j \ge n$に対し
    \begin{equation}
        \int_X |f_i(x) - f_j(x)| \, \mu(dx) < \eps
    \end{equation}
    が成り立つことをいう。
\end{definition}

\begin{proposition}[平均 Cauchy 列の性質]
    \label[proposition]{prop:mean-fundamental-sequence-of-simple-functions}
    $(X, \calA)$を可測空間、
    $\mu$を$X$上の測度、
    $(f_n)_{n \in \N}$を$X$上の$\mu$に関する単関数の列とする。
    このとき次が成り立つ:
    \begin{enumerate}
        \item $(f_n)_{n \in \N}$が平均 Cauchy 列ならば、
            数列$\myparen{\int_X f_n(x) \, \mu(dx)}_{n \in \N}$は
            ある実数に収束する。
        \item \TODO{}
    \end{enumerate}
\end{proposition}

\begin{proof}
    \uline{(1)} \quad
    単関数の積分の三角不等式と実数の完備性より従う。

    \uline{(2)} \quad
    \TODO{}
\end{proof}

% ------------------------------------------------------------
%
% ------------------------------------------------------------
\section{一般の関数の積分}

一般の関数の積分を定義する。
ここで「一般の」関数というのは次の定義で述べるもののことであって、
決して可測関数を指すわけではないという点に注意すべきである。
実際、可測関数は実数値関数だから$\pm\infty$を値に持つことはないが、
今から定義する積分では$\pm\infty$を値に持つ関数も扱うことになる。
ただし、後で示すように可積分関数はほとんど至るところ可測関数に一致するから、
この違いは実際上はそれほど問題にはならない。

\begin{definition}[一般の関数の積分]
    $(X, \calA)$を可測空間、
    $\mu$を$X$上の測度、
    $f$を$\mu$-a.e. $x \in X$で定義された
    \highlight{関数}であって、
    $[-\infty, +\infty]$に値をもち、
    $\mu$-a.e. $x \in X$で有限なものとする。
    このとき$f$が
    \term{$\mu$-可積分}[$\mu$-integrable]{可積分}[かせきぶん]
    であるとは、
    $X$上の単関数の列$(f_n)_{n \in \N}$であって
    次をみたすものが存在することをいう:
    \begin{enumerate}
        \item $\mu$-a.e. $x \in X$に対し
            $\lim_{n \to \infty} f_n(x) \to f(x)$
            が成り立つ。
        \item $(f_n)_{n \in \N}$は平均 Cauchy 列である。
    \end{enumerate}
    このとき、
    \cref{prop:mean-fundamental-sequence-of-simple-functions}
    より
    \begin{equation}
        \int_X f(x) \, \mu(dx)
            \coloneqq \lim_{n \to \infty} \int_X f_n(x) \, \mu(dx)
            \in \R
    \end{equation}
    が存在するが、これを
    $f$の\term{積分}[integral]{積分}[せきぶん]という。
    積分の値は上の条件をみたす単関数列$(f_n)_{n \in \N}$のとり方によらず
    well-defined に定まる (このあと示す)。
\end{definition}

\begin{proof}
    \TODO{}
\end{proof}

\begin{proposition}
    可積分関数はほとんど至るところ可測関数に一致する。
\end{proposition}

\begin{proof}
    \TODO{}
\end{proof}

\begin{proposition}[可積分関数の基本性質]
    \begin{enumerate}
        \item ($\R$-線型性)
        \item (単調性)
    \end{enumerate}
    \TODO{}
\end{proposition}

\begin{proof}
    \TODO{}
\end{proof}

次の Markov の不等式は確率論において重要な役割を果たす。
確率論的にいえば、Markov の不等式は
確率分布の裾の確率を上から評価するものである\footnote{
    Markov の不等式とは反対に裾の確率を下から評価する不等式としては、
    Salem-Zygmund の不等式がある\cite[p.142]{Bog07}。
}。

\begin{theorem}[Markov の不等式]
    \termhidden[Markov's inequality]{Markov の不等式}[Markov のふとうしき]
    $(X, \calA)$を可測空間、
    $\mu$を$\calA$上の$[0, +\infty]$に値をもつ測度、
    $f$を$\mu$-可積分関数とする。
    このとき、任意の$R > 0$に対し
    \begin{equation}
        \mu(\{ x \in X \mid |f(x)| \ge R \})
            \le \frac{1}{R} \int_X |f(x)| \, \mu(dx)
    \end{equation}
    が成り立つ。
\end{theorem}

\begin{proof}
    $f$は$\mu$-可積分だから、
    $X$上の可測関数であるとしてよい\TODO{why?}。
    $A_R \coloneqq \{ x \in X \mid |f(x)| \ge R \}$とおくと、
    各$x \in X$に対し$R \chi_{A_R}(x) \le |f(x)|$である。
    いま$f$の可積分性より
    $\mu(A_R) \le \mu(\{ x \in X \mid |f(x)| > 0 \}) < \infty$
    だから、積分の単調性より
    \begin{equation}
        0
            \le R \mu(A_R)
            = \int_X R \chi_{A_R}(x) \, \mu(dx)
            \le \int_X |f(x)| \, \mu(dx)
            < \infty
    \end{equation}
    が成り立つ。$R$を移項して定理の主張の式を得る。
\end{proof}

\begin{corollary}[Chebyshev の不等式]
    \TODO{}
\end{corollary}

\begin{proof}
    \TODO{}
\end{proof}

% ------------------------------------------------------------
%
% ------------------------------------------------------------
\section{収束定理}

この節では、極限と積分の交換に関する3つの重要な定理を述べる。
優収束定理は優関数 (dominant function) を用いて
収束を導くものであり、多くの場面で重宝する。
なお、優関数を見つけられない場合に一般化した収束定理としては、
一様可積分性を用いる Lebesgue-Vitali の定理がある\cite[p.268]{Bog07}。

\begin{theorem}[優収束定理; DCT]
    \termhidden[dominated convergence theorem]{優収束定理}[ゆうしゅうそくていり]
    $(X, \calA)$を可測空間、
    $\mu$を$X$上の測度、
    $(f_n)_{n \in \N}$を$\mu$-可積分関数$X \to \R$の列とする。
    このとき、条件
    \begin{enumerate}
        \item $\mu$-a.e.$x \in X$に対し、
            $\lim_{n \to \infty} f_n(x)$が
            $[-\infty, +\infty]$内に存在する。
            \TODO{a.e.でいいのか?}
        \item ある$\mu$-可積分関数$\Phi$が存在して、
            すべての$n \in \N$に対し
            $|f_n(x)| \le \Phi(x) \; \text{a.e.$x$}$
            をみたす。
    \end{enumerate}
    が成り立つならば、
    $f \colon X \to [-\infty, +\infty], \;
        x \mapsto \lim_{n \to \infty} f_n(x)$
    は$\mu$-可積分であり\TODO{値域あってる?}、
    \begin{equation}
        \int_X f(x) \, \mu(dx)
            = \lim_{n \to \infty} \int_X f_n(x) \, \mu(dx)
    \end{equation}
    が成り立つ。
\end{theorem}

\begin{proof}
    \TODO{}
\end{proof}

\begin{theorem}[単調収束定理; MCT]
    \termhidden[monotone convergence theorem]{単調収束定理}[たんちょうしゅうそくていり]
    $(X, \calA)$を可測空間、
    $\mu$を$\calA$上の$[0, +\infty]$に値をもつ測度、
    $(f_n)_{n \in \N}$を$\mu$-可積分関数の列であって
    すべての$n \in \N$と
    $\mu$-a.e. $x \in X$に対し
    $f_n(x) \le f_{n + 1}(x)$をみたすものとする。
    このとき、
    $\sup_n \int_X |f_n(x)| \, \mu(dx) < \infty$
    ならば、
    $f(x) \coloneqq \lim_{n \to \infty} f_n(x)$は
    $\mu$-a.e. $x \in X$に対し有限の値をとる
    $\mu$-可積分であり、
    \begin{equation}
        \int_X f(x) \, \mu(dx)
            = \lim_{n \to \infty} \int_X f_n(x) \, \mu(dx)
    \end{equation}
    が成り立つ。
\end{theorem}

\begin{proof}
    \TODO{}
\end{proof}

\begin{theorem}[Fatou の定理]
    \TODO{}
\end{theorem}

\begin{proof}
    \TODO{}
\end{proof}

% ------------------------------------------------------------
%
% ------------------------------------------------------------
\newpage
\section{演習問題}

\begin{problem}[ChatGPT]
    $(E, \calA)$を可測空間、
    $\mu$を$\calA$上の$[0, +\infty]$に値をもつ測度、
    $f$を非負の可測関数とし、
    $\int_E f(x) \, dx = 0$が成り立つとする。
    このとき、ほとんどすべての $x \in E$ に対して
    $f(x) = 0$ となることを示せ。
\end{problem}

\begin{answer}
    任意の正整数$n$に対し、
    Markov の不等式より
    $0 \le \mu(\{ x \in E \mid f(x) \ge 1/n \}) \le n \int_E f(x) dx = 0$、
    したがって
    $\mu(\{ x \in E \mid f(x) \ge 1/n \}) = 0$
    が成り立つ。よって
    $\mu(\{ x \in E \mid f(x) \neq 0 \})
        = \mu(\bigcup_{n = 1}^\infty \{ x \in E \mid f(x) \ge 1/n \})
        = \sum_{n = 1}^\infty \mu(\{ x \in E \mid f(x) \ge 1/n \})
        = 0$
    が成り立つ。
    したがって
    $\mu$-a.e. $x \in E$に対し
    $f(x) = 0$が成り立つ。
\end{answer}




% ============================================================
%
% ============================================================
\chapter{Lebesgue 測度}

ここまでは一般の可測空間を考えてきたが、
実用上は$\R^n$上での積分がとくに重要である。
この章では Lebesgue 測度を導入し、
Riemann 積分と Lebesgue 積分との関係性について調べる。

% ------------------------------------------------------------
%
% ------------------------------------------------------------
\section{Lebesgue 測度の構成}

\begin{definition}
    \TODO{}
\end{definition}

% ------------------------------------------------------------
%
% ------------------------------------------------------------
\section{Riemann 積分との関連}

\TODO{これが一番大事?}



% ============================================================
%
% ============================================================
\chapter{符号付き測度}

% ------------------------------------------------------------
%
% ------------------------------------------------------------
\section{全変動}

\TODO{符号付き測度全体の空間は全変動をノルムとして Banach 空間となる?}

\begin{definition}[全変動]
    \TODO{}
\end{definition}

% ------------------------------------------------------------
%
% ------------------------------------------------------------
\section{Radon-Nikodym の定理}

\begin{definition}[絶対連続と特異]
    \TODO{}
\end{definition}

\begin{proposition}[Lebesgue 分解]
    $(X, \calB)$を可測空間、
    $\mu$を$\sigma$-有限な測度空間、
    $\Phi$を$\calB$上の測度とする。
    このとき次が成り立つ:
    \begin{enumerate}
        \item \TODO{}
            \begin{equation}
                \Phi(E) = F(E) + \Psi(E)
            \end{equation}
        \item 上の分解は一意である。
    \end{enumerate}
\end{proposition}

\begin{proof}
    \TODO{}
\end{proof}

\begin{theorem}[Radon-Nikodym の定理]
    $(X, \calB)$を可測空間、
    $\mu$を$X$上の$\sigma$-有限測度、
    $F$を$\mu$に関し絶対連続な$X$上の測度とする。
    このとき、
    $\mu$-a.e. $x \in X$に対し定義された
    可積分関数$f$が存在して
    \begin{equation}
        F(E) = \int_E f(x) \, d\mu(x)
            \quad
            (E \in \calB)
    \end{equation}
    が成り立つ。
    この$f$を$\mu$に関する$F$の
    \term{Radon-Nikodym 微分}[Radon-Nikodym derivative]
        {Radon-Nikodym 微分}[Radon-Nikodym びぶん]
    という。
\end{theorem}

\begin{proof}
    \TODO{}
\end{proof}

% ------------------------------------------------------------
%
% ------------------------------------------------------------
\section{Fubini の定理}

\begin{theorem}[Fubini の定理]
    \TODO{}
\end{theorem}

\begin{proof}
    \TODO{}
\end{proof}



% ============================================================
%
% ============================================================
\chapter{関数空間}

% ------------------------------------------------------------
%
% ------------------------------------------------------------
\section{測度収束}

\TODO{その他の収束概念もまとめるべき?}

\begin{definition}[測度 Cauchy 列と測度収束]
    \TODO{}
\end{definition}

% ------------------------------------------------------------
%
% ------------------------------------------------------------
\section{$L^p$ノルム}

\begin{theorem}[Minkowski の不等式]
    \TODO{}
\end{theorem}

\begin{proof}
    \TODO{}
\end{proof}

% ------------------------------------------------------------
%
% ------------------------------------------------------------
\section{Hilbert 空間$L^2$}



\end{document}

\part{接続}
\documentclass[report]{jlreq}
\usepackage{../../global}
\usepackage{./local}
\subfiletrue
%\makeindex
\begin{document}

\TODO{接続とは一体何なのか?}

この部では接続について論じる。
接続とは、ベクトル場を方向微分して
新たなベクトル場を作る手続きのようなものである。
接束の接続はアファイン接続と呼ばれ、とくに重要である。


% ============================================================
%
% ============================================================
\newpage
\chapter{ベクトル値微分形式}

% ------------------------------------------------------------
%
% ------------------------------------------------------------
\section{ベクトル値微分形式}
\label[section]{sec:vector-valued-forms}

微分形式の概念をベクトル束に値をもつように一般化する。
これは後に主ファイバー束の接続を定義するために用いる。

\begin{definition}[ベクトル束に値をもつ微分形式]
    $M$を多様体、$E \to M$をベクトル束とし、$p \in \Z_{\ge 0}$とする。
    ベクトル束$\bigwedge^p T^*M \otimes E$の切断を
    \term{$E$に値をもつ$p$-形式}
    {ベクトル束に値をもつ微分形式}[べくとるそくにあたいをもつびぶんけいしき]
    あるいは
    \term{$E$-値$p$-形式}[$E$-valued $p$-form]
    {ベクトル束に値をもつ微分形式}[べくとるそくにあたいをもつびぶんけいしき]
    という。
    $E$-値$p$-形式全体のなす集合を
    \begin{equation}
        A^p(E) \coloneqq \Gamma\Bigl(
            \Bigl(\bigwedge^p T^*M\Bigr) \otimes E
        \Bigr)
    \end{equation}
    と書く。
    $E$-値$p$-形式は
    $\theta \otimes \xi \; (\theta \in A^p(M), \; \xi \in A^0(E))$の形
    の元の和に (一意ではないが) 書ける。
\end{definition}

\begin{remark}
    \TODO{どういうこと?}
    ベクトル空間の同型
    \begin{equation}
        \Hom(\Lambda^k T_xM, V)
            \cong (\Lambda^k T_xM)^* \otimes V
            \cong (\Lambda^k T_x^*M) \otimes V
    \end{equation}
    に注意すれば、$V$に値をもつ$k$-形式の値は、
    確かに$\Lambda^k T_xM \to V$の$\R$線型写像とみなせることがわかる。
\end{remark}

\begin{remark}
    テキストでは$\theta$と$\xi$の順序が逆になったりしているが、
    ここでは$\theta \otimes \xi$の順序に統一する。
\end{remark}

ベクトル値形式は
従来の意味での微分形式ではなく、
したがって外積は定義されていないが、
通常の外積から自然に定義が拡張される。

\begin{definition}[ベクトル値形式の外積]
    $M$を多様体、$E \to M$をベクトル束、
    $p, q \in \Z_{\ge 0}$とする。
    $\wedge \colon A^p(M) \times A^q(M) \to A^{p + q}(M)$を
    通常の外積とし、
    その一般化として
    $\wedge \colon A^p(M) \times A^q(E) \to A^{p + q}(E)$を
    \begin{alignat}{1}
        (\omega, \xi)
            = \left(
                \omega,
                \sum_{i} \alpha_i \otimes \xi_i
            \right)
            \mapsto
            \omega \wedge \xi
            &\coloneqq
            \sum_{i} \omega \wedge \alpha_i \otimes \xi_i \\
        &\qquad \quad
            (\alpha_i \in A^q(M), \; \xi_i \in A^0(E))
    \end{alignat}
    と定める。
    これは明らかに$\xi$の表し方によらず well-defined に定まる。
\end{definition}

\begin{definition}[ベクトル値形式の内積]
    $M$を多様体、
    $E \to M, \; F \to M$をベクトル束、
    $g \colon A^0(E) \times A^0(F) \to A^0(M)$を
    $\smooth(M)$-双線型写像とする。
    $g$の一般化として、同じ記号で写像
    $g \colon A^p(E) \times A^q(F) \to A^{p + q}(M)$を
    \begin{alignat}{1}
        (\omega, \xi)
            = \left(
                \sum_{i} \alpha_i \otimes \omega_i,
                \sum_{j} \beta_j \otimes \xi_j
            \right)
            &\mapsto
            g(\omega, \xi)
            \coloneqq
            \sum_{i, j}
            g(\omega_i, \xi_j)
            \alpha_i \wedge \beta_j \\
        &
            (
                \alpha_i \in A^p(M), \; \beta_j \in A^q(M), \;
                \omega_i \in A^0(E), \; \xi_j \in A^0(F)
            )
    \end{alignat}
    と定める。
    これは$\omega, \xi$の表し方によらず well-defined に定まり (証明略)、
    また$\smooth(M)$-双線型写像である。
\end{definition}

\begin{remark}
    上の定義の双線型写像$g \colon A^0(E) \times A^0(F) \to A^0(M)$の例としては、
    \begin{itemize}
        \item 双対の定める内積
            $\langle , \rangle \colon A^0(E^*) \times A^0(E) \to A^0(M)$
        \item 計量
            $g \colon A^0(E) \times A^0(E) \to A^0(M)$
    \end{itemize}
    などがある。
\end{remark}



% ============================================================
%
% ============================================================
\newpage
\chapter{主ファイバー束}

主ファイバー束は、多様体$M$上局所自明な群の族である。
ここで主ファイバー束という概念を持ち出す理由はベクトル束を調べるためであるが、
実際ベクトル束と主ファイバー束の間には良い関係がある。
というのも、ベクトル束はフレーム束と呼ばれる主ファイバー束と対応し、
逆に主ファイバー束はその構造群の表現を通してベクトル束と対応する。
したがって、あるベクトル束について調べたいときに
代わりに主ファイバー束を考えることで議論の見通しがよくなることがある。
そこで、この章では主ファイバー束とベクトル束の基本的な関係を調べることにする。

% ------------------------------------------------------------
%
% ------------------------------------------------------------
\section{ファイバー束}

ファイバー束を定義する。

\TODO{ファイバー束は構造群付きを基本として、
    修飾しない場合は自明な構造群を持つものと定義したい}

\begin{definition}[ファイバー束]
    $M, F$を多様体とする。
    多様体$E$が
    \term{ファイバー束}[fiber bundle]{ファイバー束}[ふぁいばーそく]
    であるとは、
    $E$が次をみたすことである:
    \begin{enumerate}
        \item 全射な{\smooth}写像$\pi \colon E \to M$が与えられている。
        \item \TODO{局所自明性}
    \end{enumerate}
\end{definition}

\TODO{主ファイバー束をファイバー束の特別な場合として定義したい}

\begin{definition}[主ファイバー束]
    $M$を多様体、
    $G$を Lie 群とする。
    多様体$P$が
    $G$を\term{構造群}[structure group]{構造群}[こうぞうぐん]とする$M$上の
    \term{主ファイバー束}[principal fiber bundle]{主ファイバー束}[しゅふぁいばーそく]、
    あるいは\term{主$G$束}[principal $G$-bundle]{主$G$束}[しゅGそく]であるとは、
    $P$が次をみたすことである:
    \begin{enumerate}
        \item 全射な{\smooth}写像$p \colon P \to M$が与えられている。
        \item $G$は$P$に右から{\smooth}に作用しており、
            さらに次をみたす:
            \begin{enumerate}[label=(\arabic{enumi}-\alph*)]
                \item 作用はファイバーを保つ。
                \item 作用はファイバー上単純推移的\footnote{
                        作用が\term{単純推移的}[simply transitive]
                        {単純推移的}[たんじゅんすいいてき]
                        であるとは、自由かつ推移的であることをいう。
                    }である。
            \end{enumerate}
        \item $M$のある開被覆$\{ U_\alpha \}_{\alpha \in A}$が存在して、
            各$U_\alpha$上に
            次をみたす写像
            $\sigma_\alpha \colon U_\alpha \to p^{-1}(U_\alpha)$
            が存在する:
            \begin{enumerate}[label=(\arabic{enumi}-\alph*)]
                \item $\sigma_\alpha$は{\smooth}であって
                    $p \circ \sigma_\alpha = \id_{U_\alpha}$をみたす。
                    すなわち$\sigma_\alpha$は$U_\alpha$上の
                    $P$の切断である。
                \item (局所自明性) 写像
                    \begin{equation}
                        \varphi_\alpha \colon p^{-1}(U_\alpha) \to U_\alpha \times G,
                        \quad
                        \underbrace{\sigma_\alpha(x) . s}_{
                            \mathclap{\text{群作用を「$.$」で書く。}}
                        } \mapsto (x, s)
                    \end{equation}
                    が diffeo である
                    (写像として well-defined に定まることはすぐ後で確かめる)\footnote{
                        このように定めた写像$\varphi_\alpha$が diffeo かどうか
                        (とくに{\smooth}かどうか) は
                        他の条件からはおそらく導かれない気がするので (\TODO{本当に?})、
                        独立な条件として与えておくことにする。
                        \TODO{cf. \url{https://math.stackexchange.com/questions/2930299/trivialization-from-a-smooth-frame}}
                    }。
            \end{enumerate}
    \end{enumerate}
    ここで
    \begin{itemize}
        \item $\varphi_\alpha$を$U_\alpha$上の$P$の
            \term{局所自明化}[local trivialization]{局所自明化}[きょくしょじめいか]
            という。
    \end{itemize}
\end{definition}

\begin{lemma}[$G$-torsor の特徴付け]
    $G$を群、
    $X$を空でない集合とし、
    $G$は$X$に右から作用しているとする。
    このとき次は同値である:
    \begin{enumerate}
        \item $G$の作用が単純推移的である。
        \item 写像
            \begin{equation}
                \theta \colon X \times G \to X \times X,
                \quad
                (x, g) \mapsto (x.g, x)
            \end{equation}
            が全単射である\footnote{
                写像$\theta$を shear map といい、
                shear map が全単射のとき$X$を$G$-torsor という。
            }。
    \end{enumerate}
    したがって、とくに上の定義の$\varphi_\alpha$が確かに写像として定まる。
\end{lemma}

\begin{proof}
    \begin{alignat}{1}
        \theta \colon \text{ 全射}
            &\iff \forall x, y \in X \; \exists g \in G \; [x.g = y] \\
            &\iff \text{$G$の作用が推移的} \\
        \theta \colon \text{ 単射}
            &\iff \forall x \in X \;
                \forall g, g' \in G \;
                [x.g = x.g' \implies g = g'] \\
            &\iff \forall x \in X \;
                \forall g, g' \in G \;
                [x = x.g'g^{-1} \implies g'g^{-1} = 1] \\
            &\iff \forall x \in X \;
                \forall g \in G \;
                [x = x.g \implies g = 1] \\
            &\iff \text{$G$の作用が自由}
    \end{alignat}
\end{proof}

\begin{definition}[変換関数]
    $M$を多様体、
    $p \colon P \to M$を主$G$束とすると、
    主$G$束の定義より、$M$の open cover $\{U_\alpha\}_{\alpha \in A}$であって
    各$U_\alpha$上に切断
    $\sigma_\alpha \colon U_\alpha \to p^{-1}(U_\alpha)$
    を持つものがとれる。
    各$\alpha, \beta \in A, \; U_\alpha \cap U_\beta \neq \emptyset$
    に対し、
    写像$\psi_{\alpha\beta} \colon U_\alpha \cap U_\beta \to G$を
    $x \in U_\alpha \cap U_\beta$を
    $\sigma_\beta(x) = \sigma_\alpha(x) . s$なる$s \in G$
    に写す写像、すなわち
    \begin{equation}
        x \overset{\sigma_\beta}{\mapsto} \sigma_\beta(x) = \sigma_\alpha(x) . s
            \overset{
                \substack{\sigma_\alpha \text{ より定まる} \\ \text{局所自明化}}
            }{\mapsto} (x, s)
            \overset{\mathrm{pr}_2}{\mapsto} s
    \end{equation}
    で定めると、これは{\smooth}である。
    {\smooth}写像の族$\{ \psi_{\alpha\beta} \}$を、
    切断の族$\{ \sigma_\alpha \}$から定まる
    $P$の\term{変換関数}[transition function]{変換関数}[へんかんかんすう]という。
\end{definition}

% ------------------------------------------------------------
%
% ------------------------------------------------------------
\section{ベクトル束と主ファイバー束の同伴}

\subsection{ベクトル束から主ファイバー束へ}

多様体上のランク$r$ベクトル束が与えられると、
フレーム束とよばれる主$\GL(r, \R)$束を構成できる。

\TODO{フレーム束はフレーム多様体をファイバーとする主$\GL(r, \R)$束?}

\TODO{フレーム束の主ファイバー束構造は全単射により誘導する?}

\begin{definition}[フレーム束]
    $M$を$n$次元多様体、
    $E \to M$をランク$r$ベクトル束とする。
    $M$の atlas
    $\{ (U_\alpha, \psi_\alpha) \}_{\alpha \in A}$であって、
    各$\alpha$に対して$U_\alpha$上の$E$の局所自明化$\rho_\alpha$が存在するものがとれる。
    \begin{innerproof}
        各$x \in M$に対し、
        多様体の定義とベクトル束の定義より、
        $x$の$M$における開近傍$V_x, W_x$であって
        $V_x$を定義域とするチャートが存在し、
        かつ$W_x$上の$E$の局所自明化が存在するようなものがとれる。
        そこで$U_x \coloneqq V_x \cap W_x$とおけば
        $\{ U_x \}_{x \in M}$が求める atlas となる。
    \end{innerproof}
    $E$の局所自明化の族$\{ \rho_\alpha \}$により定まる
    $E$の変換関数を$\{ \rho_{\alpha\beta} \}$とおく。
    $E$の\term{フレーム束}[frame bundle]{フレーム束}[ふれーむそく]
    とよばれる主$\GL(r, \R)$束
    $p \colon P \to M$を次のように構成する:
    \begin{enumerate}
        \item 各$x \in M$に対し、集合$P_x$を
            \begin{equation}
                P_x \coloneqq \{
                    u \colon \R^r \to E_x
                    \mid
                    \text{$u$は線型同型}
                \}
            \end{equation}
            で定める。
            $P_x$は$E_x$の基底全体の集合とみなせる。
        \item $P_x$らの disjoint union を
            \begin{equation}
                P \coloneqq \coprod_{x \in M} P_x
            \end{equation}
            とおく。
        \item 射影$p \colon P \to M$を
            \begin{equation}
                p((x, u)) \coloneqq x 
            \end{equation}
            で定義する。
        \item $\GL(r, \R)$の$P$への右作用$\beta$を
            次のように定める:
            \begin{equation}
                \beta \colon P \times \GL(r, \R) \to P,
                \quad
                ((x, u), s) \mapsto (x, u \circ s)
            \end{equation}
        \item 各$\alpha \in A$に対し、
            $U_\alpha$上の$E$の局所自明化$\rho_\alpha$をひとつ選び、
            それにより定まる$E$のフレームを
            $e_1^{(\alpha)}, \dots, e_r^{(\alpha)}$とおく。
            写像$\sigma_\alpha \colon U_\alpha \to p^{-1}(U_\alpha)$を
            次のように定める:
            \begin{itemize}
                \item 各$x \in U_\alpha$に対し、
                    $E_x$の基底$e_1^{(\alpha)}(x), \dots, e_r^{(\alpha)}(x)$により
                    定まる線型同型$\R^r \to E_x$を
                    一時的な記号で$\sigma_\alpha(x)_2$と書く。
                \item $\sigma_\alpha(x) \coloneqq (x, \sigma_\alpha(x)_2)$と定める。
                    記号の濫用で$\sigma_\alpha(x)_2$も$\sigma_\alpha(x)$と書く。
            \end{itemize}
        \item 写像$\varphi_\alpha$を
            \begin{equation}
                \varphi_\alpha
                    \colon p^{-1}(U_\alpha) \to U_\alpha \times \GL(r, \R),
                    \quad
                    (x, \sigma_\alpha(x) \circ s) \mapsto (x, s)
            \end{equation}
            と定める。
            ただし、$(x, \sigma_\alpha(x) \circ s)$から
            $s$が一意に定まることは
            $s = \sigma_\alpha(x)^{-1} \circ \sigma_\alpha(x) \circ s$
            と表せることよりわかる。
            また、$\varphi_\alpha$は明らかに可逆である。
        \item 写像族$\{ \varphi_\alpha \}$を用いて
            $P$に多様体構造が入る (このあとすぐ示す)。
        \item $p \colon P \to M$は、
            $\{ \sigma_\alpha \colon U_\alpha \to p^{-1}(U_\alpha) \}$
            を切断の族、
            これにより定まる変換関数を$\{ \rho_{\alpha\beta} \}$として
            $M$上の主$\GL(r, \R)$束となる (このあとすぐ示す)。
    \end{enumerate}
    $P$は$E$に\term{同伴する}[associated]{同伴する}[どうはんする]
    主ファイバー束と呼ばれる。
\end{definition}

\begin{proof}
    $\GL(r, \R) = \R^{r^2}$と同一視する。
    まず$P$に多様体構造が入ることを示す。
    $M$の atlas $\{ (U_\alpha, \psi_\alpha) \}$は、
    小さい範囲に制限した chart、すなわち
    \begin{equation}
        (U'_\alpha, \psi_\alpha|_{U'_\alpha})
        \quad
        (\alpha \in A, \; U'_\alpha \opensubset U_\alpha)
    \end{equation}
    をすべて含むとしてよい。
    写像族$\{ \Phi_\alpha \colon p^{-1}(U_\alpha) \to \R^{n + r^2} \}$を
    \begin{equation}
        \begin{tikzcd}
            p^{-1}(U_\alpha)
                \ar{r}{\varphi_\alpha}
                \ar[bend right=30, end anchor=south west]{rr}[swap]{\Phi_\alpha}
                & U_\alpha \times \GL(r, \R)
                \ar{r}{\psi_\alpha \times \id}
                & \psi_\alpha(U_\alpha) \times \R^{r^2}
                \subset \R^{n + r^2}
        \end{tikzcd}
    \end{equation}
    を可換にするものとして定める。
    $P$に$\{ \Phi_\alpha \}$を atlas とする多様体構造が入ることを示すため、
    Smooth Manifold Chart Lemma (\cref{lemma:smooth-manifold-chart-lemma})
    の条件を確認する。
    $\varphi_\alpha$が可逆であることと
    $\psi_\alpha$が$M$の chart であることから、
    $\Phi_\alpha$は$\R^{n + r^2}$の開部分集合
    $\psi_\alpha(U_\alpha) \times \R^{r^2}$への全単射である。
    よって (i) が満たされる。

    各$\alpha, \beta \in A$に対し
    $\psi_\alpha, \psi_\beta$が$M$の chart であることから
    \begin{align}
        \Phi_\alpha(p^{-1}(U_\alpha) \cap p^{-1}(U_\beta))
            = \psi_\alpha(U_\alpha \cap U_\beta) \times \R^{r^2} \\
        \Phi_\beta(p^{-1}(U_\alpha) \cap p^{-1}(U_\beta))
            = \psi_\beta(U_\alpha \cap U_\beta) \times \R^{r^2}
    \end{align}
    はいずれも$\R^{n + r^2}$の開部分集合である。
    よって (ii) が満たされる。

    各$\alpha, \beta \in A$に対し
    合成写像$\varphi_\beta \circ \varphi_\alpha^{-1}$は
    \begin{equation}
        \begin{tikzcd}
            (U_\alpha \cap U_\beta) \times \GL(r, \R)
                \ar{r}{\varphi_\alpha^{-1}}
                & \pi^{-1}(U_\alpha \cap U_\beta)
                \ar{r}{\varphi_\beta}
                & (U_\alpha \cap U_\beta) \times \GL(r, \R) \\[-1em]
            (x, s)
                \ar[mapsto]{r}
                & (x, \sigma_\alpha(x) \circ s)
                \ar[mapsto]{r}
                & (x, \sigma_\beta(x)^{-1} \circ \sigma_\alpha(x) \circ s)
        \end{tikzcd}
    \end{equation}
    という対応を与えるが、
    ここで$\sigma_\beta(x)^{-1} \circ (\sigma_\alpha(x)) \circ s$は
    $(x, s)$に関し{\smooth}である。
    \begin{innerproof}
        $s$を右から合成する演算は
        Lie 群$\GL(r, \R)$における積なので{\smooth}である。
        そこで$\sigma_\beta(x)^{-1} \circ \sigma_\alpha(x)$について考える。
        いま各$x \in U_\alpha \cap U_\beta$に対し
        \begin{equation}
            \begin{tikzcd}
                \R^r \ar{rr}{\sigma_\beta(x)^{-1} \circ \sigma_\alpha(x)}
                    \ar{dr}[swap]{\sigma_\beta(x)}
                    & & \R^r \ar{dl}{\sigma_\alpha(x)} \\
                & E_x
            \end{tikzcd}
        \end{equation}
        は可換であるが、
        $\sigma_\alpha, \sigma_\beta$は定め方から
        $E$の局所自明化の$E_x$への制限$\rho_\alpha(x), \rho_\beta(x)$の逆写像である。
        よって写像
        \begin{equation}
            U_\alpha \cap U_\beta \to \GL(r, \R),
            \quad
            x \mapsto \sigma_\beta(x)^{-1} \circ \sigma_\alpha(x)
        \end{equation}
        は$E$の変換関数$\rho_{\beta\alpha}$に他ならず、
        したがってこれは{\smooth}である。
        よって、$\sigma_\beta(x)^{-1} \circ (\sigma_\alpha(x)) \circ s$は
        $(x, s)$に関し{\smooth}である。
    \end{innerproof}
    したがって
    \begin{equation}
        \Phi_\beta \circ \Phi_\alpha^{-1}
            = (\psi_\beta \times \id) \circ \varphi_\beta
                \circ \varphi_\alpha^{-1}
                \circ (\psi_\alpha \times \id)^{-1}
    \end{equation}
    は$\Phi_\alpha(p^{-1}(U_\alpha) \cap p^{-1}(U_\beta))$上{\smooth}である。
    よって (iii) が満たされる。

    $\{ (U_\alpha, \psi_\alpha) \}$は
    小さい範囲に制限した chart をすべて含むことから
    明らかに (iv) が満たされる。

    以上で Smooth Manifold Chart Lemma の条件が確認できた。
    したがって$P$は
    $\{ (p^{-1}(U_\alpha), \Phi_\alpha) \}$を atlas として多様体となる。

    つぎに、$P$は
    $\{ \sigma_\alpha \colon U_\alpha \to p^{-1}(U_\alpha) \}$
    を切断の族として
    $M$上の主$\GL(r, \R)$束となることを示す。
    そのためには次を示せばよい:
    \begin{enumerate}
        \item $p$が{\smooth}であること
        \item 作用$\beta$がファイバーを保つこと
        \item 作用$\beta$がファイバー上単純推移的であること
        \item 作用$\beta$が{\smooth}であること
        \item $\sigma_\alpha$が$U_\alpha$上の$P$の切断となること
        \item 主ファイバー束の定義の局所自明性が満たされること
        \item $\{ \sigma_\alpha \}$により定まる$P$の変換関数が
            $\{ \rho_{\alpha\beta} \}$であること
    \end{enumerate}
    ここで、$\varphi_\alpha$らは diffeo である。実際、図式
    \begin{equation}
        \begin{tikzcd}
            p^{-1}(U_\alpha)
                \ar{r}{\varphi_\alpha}
                \ar[bend right=30, end anchor=south west]{rr}[swap]{\Phi_\alpha}
                & U_\alpha \times \GL(r, \R)
                \ar{r}{\psi_\alpha \times \id}
                & \psi_\alpha(U_\alpha) \times \R^{r^2}
                \subset \R^{n + r^2}
        \end{tikzcd}
    \end{equation}
    が可換であることと$\psi_\alpha \times \id, \; \Phi_\alpha$が
    diffeo であることから従う。

    $p$が{\smooth}であることは
    各点の近傍での{\smooth}性を示せばよいが、これは
    各$(x, u) \in P$に対し$p^{-1}(U_\alpha)$が開近傍となるような
    $\alpha \in A$がとれて
    \begin{equation}
        \begin{tikzcd}
            p^{-1}(U_\alpha)
                \ar{rd}[swap]{p}
                \ar{r}{\Phi_\alpha}
                & p^{-1}(U_\alpha) \times \R^{n + r^2}
                \ar{d}{\mathrm{pr}_1} \\
            & P
        \end{tikzcd}
    \end{equation}
    が可換となることから従う。

    $\GL(r, \R)$の$P$への作用
    \begin{equation}
        \beta((x, u), s)
            = (x, u \circ s)
    \end{equation}
    がファイバーを保つことは定義から明らか。

    $\beta$がファイバー$P_x = p^{-1}(x) \; (x \in M)$上単純推移的であることは、
    shear map
    \begin{equation}
        P_x \times \GL(r, \R) \to P_x \times P_x,
        \quad
        ((x, u), s) \mapsto ((x, u \circ s), (x, u))
    \end{equation}
    が逆写像
    \begin{equation}
        P_x \times P_x \to P_x \times \GL(r, \R),
        \quad
        ((x, t), (x, u)) \mapsto ((x, u), u^{-1} \circ t)
    \end{equation}
    を持つことから従う。

    $\beta$が{\smooth}であることを示す。
    $(x, u) \in P$の近傍$U_\alpha$上で
    \begin{equation}
        (x, u) = (x, \sigma_\alpha(x) \circ t)
        \quad
        (t \in \GL(r, \R))
    \end{equation}
    の形に書けることに注意すれば、
    \begin{alignat}{1}
            &((x, u), s) \in p^{-1}(U_\alpha) \times \GL(r, \R) \\
        \overset{
            \mathclap{\id \times (\mathrm{pr}_2 \circ \varphi_\alpha)}
        }{\mapsto} \qquad
            &((x, u), s, t)
            \in p^{-1}(U_\alpha) \times \GL(r, \R) \times \GL(r, \R) \\
        \overset{\mathclap{\text{$\GL(r, \R)$での積}}}{\mapsto} \qquad
            &((x, u), ts)
            \in p^{-1}(U_\alpha) \times \GL(r, \R) \\
        \overset{p}{\mapsto} \qquad
            &(x, ts)
            \in U_\alpha \times \GL(r, \R) \\
        \overset{\mathclap{\varphi_\alpha^{-1}}}{\mapsto} \qquad
            &(x, \sigma_\alpha(x) \circ ts)
            = (x, u \circ s)
            \in p^{-1}(U_\alpha)
    \end{alignat}
    の各写像が{\smooth}であることから、
    $\beta$は$U_\alpha$上{\smooth}であることがわかる。
    したがって$\beta$は{\smooth}である。

    $\sigma_\alpha$が$U_\alpha$上の$P$の切断となることを示す。
    $p \circ \sigma_\alpha(x) = x$となることは定義から明らか。
    {\smooth}性は
    \begin{equation}
        \sigma_\alpha(x)
            = \varphi_\alpha^{-1}(x, 1)
    \end{equation}
    よりわかる。
    したがって$\sigma_\alpha$は$U_\alpha$上の$P$の切断である。
    さらに$\varphi_\alpha$の定義と$\varphi_\alpha$が diffeo であることから
    主ファイバー束の定義の局所自明性も満たされる。

    最後に、$x \in U_\alpha \cap U_\beta, \; \alpha, \beta \in A$に対し
    \begin{equation}
        \sigma_\beta(x)
            = \sigma_\alpha(x) \circ \sigma_\alpha^{-1} \circ \sigma_\beta(x)
            = \sigma_\alpha(x) \circ \rho_{\alpha\beta}(x)
    \end{equation}
    が成り立つことから、
    $\{ \sigma_\alpha \}$により定まる$P$の変換関数は
    $\{ \rho_{\alpha\beta} \}$である。

    以上で$P$は
    $\{ \sigma_\alpha \}$を切断の族とし、
    これにより定まる$P$の変換関数を
    $\{ \rho_{\alpha\beta} \}$として
    $M$上の主$\GL(r, \R)$束となることが示せた。
\end{proof}

\begin{example}[構造群の縮小]
    $E$をベクトル束、
    $g$を$E$の内積とする。
    フレーム束の定義の$P_x$を
    \begin{equation}
        Q_x \coloneqq \{ u \colon \R^r \to E_x
            \mid u \text{ は線型同型かつ内積を保つ}
        \}
    \end{equation}
    に置き換えると、$Q$は
    直交群$O(r)$を構造群とする$M$上の主束となる。
    このとき$Q$は$P$の部分束であり、
    $Q$は$P$の構造群$\GL(r, \R)$を$O(r)$に
    \term{縮小}[reduction]{縮小}[しゅくしょう]
    して得られたという。
\end{example}

\subsection{主ファイバー束からベクトル束へ}
\label[subsection]{subsec:principal-fiber-bundle-to-vector-bundle}

逆に主$G$束$P$と
表現$\rho \colon G \to \GL(r, \R)$が与えられると、
ランク$r$ベクトル束$E$が構成できる。

\begin{definition}[同伴するベクトル束]
    $M$を多様体、$P \to M$を主$G$束、
    $\rho \colon G \to \GL(r, \R)$を Lie 群の表現とする。
    直積多様体$P \times \R^r$への
    $G$の{\smooth}右作用を
    \begin{equation}
        (P \times \R^r) \times G \to P \times \R^r,
        \quad
        ((u, y), s) \mapsto (u.s, \rho(s)^{-1} y)
    \end{equation}
    で定め、軌道空間$(P \times \R^r) / G$を
    \begin{equation}
        P \times_\rho \R^r
    \end{equation}
    と書く。
    このとき、$P \times_\rho \R^r$は
    $M$上のベクトル束となり、
    $P$のある変換関数$\{ \psi_{\alpha\beta} \}$に対し
    $\{ \rho \circ \psi_{\alpha\beta} \}$が
    $P \times_\rho \R^r$の変換関数のひとつとなる
    (このあとすぐ示す)。
    これを$P$に
    \term{同伴する}[associated]{同伴する}[どうはんする]
    ベクトル束という。
\end{definition}

\begin{proof}
    $P \times_\rho \R^r$が$M$上のベクトル束になることを、
    Vector Bundle Chart Lemma を用いて示す。
   標準射影$P \to M$および
    $P \times \R^r \to P \times_\rho \R^r$を
    それぞれ$p, q$とおく。

    まず射影を構成する。図式
    \begin{equation}
        \begin{tikzcd}
            P \times \R^r
                \ar{d}[swap]{\mathrm{pr}_1}
                \ar{r}{q}
                & P \times_\rho \R^r
                \ar[dashed]{d}{\pi} \\
            P \ar{r}[swap]{p}
                & M
        \end{tikzcd}
    \end{equation}
    において、$p \circ \mathrm{pr}_1$は$q$のファイバー上定値である。
    \begin{innerproof}
        $u \in P_x, \; u' \in P_{x'} \; (x, x' \in M),
        \; y, y' \in \R^r$について
        $q(u, y) = q(u', y')$ならば、
        $q$の定義から
        ある$s \in G$が存在して
        $(u, y) = (u' . s, \rho(s)^{-1} y')$が成り立ち、
        とくに$u = u' . s$だが、
        $G$の$P$への作用がファイバーを保つことから
        $x = x'$が成り立つ。
    \end{innerproof}
    したがって
    写像$\pi \colon P \times_\rho \R^r \to M$が誘導される。
    このとき$p \circ \mathrm{pr}_1$が全射であることより
    $\pi$も全射である。

    つぎに$P \times_\rho \R^r$の局所自明化を構成する。
    $P$の切断の族$\{ \sigma_\alpha \colon U_\alpha \to P \}_{\alpha \in A}$
    であって$\bigcup U_\alpha = P$なるものをひとつ選ぶ。
    これにより定まる$P$の局所自明化の族を$\{ \varphi_\alpha \}$とおき、
    さらにこれにより定まる$P$の変換関数を$\{ \psi_{\alpha\beta} \}$とおく。
    このとき、各$\alpha \in A$に対し図式
    \begin{equation}
        \begin{tikzcd}[column sep=large]
            U_\alpha  \times \R^r
                \ar[dashed]{drr}
                \ar{r}{\substack{(x, y) \\ \; \mapsto (x, 1, y)}}
                & U_\alpha \times G \times \R^r
                \ar{r}{\varphi_\alpha^{-1} \times \id}
                & p^{-1}(U_\alpha) \times \R^r
                \ar{d}{q} \\
            && p^{-1}(U_\alpha) \times_\rho \R^r
                = \pi^{-1}(U_\alpha)
        \end{tikzcd}
    \end{equation}
    の破線部の写像は全単射である。
    \begin{innerproof}
        $(u, y), (u', y') \in U_\alpha \times \R^r$について
        \begin{alignat}{1}
                &q(\varphi_\alpha^{-1}(u, 1), y)
                    = q(\varphi_\alpha^{-1}(u', 1), y') \\
            \iff
                &\exists s \in G
                \quad \text{s.t.} \quad
                \begin{cases}
                    \varphi_\alpha^{-1}(u, 1) = \varphi_\alpha^{-1}(u', 1) . s \\
                    y = \rho(s)^{-1} y'
                \end{cases} \\
            \iff
                &\exists s \in G
                \quad \text{s.t.} \quad
                \begin{cases}
                    \varphi_\alpha^{-1}(u, 1) = \varphi_\alpha^{-1}(u', s) \\
                    y = \rho(s)^{-1} y'
                \end{cases} \\
            \iff
                &\exists s \in G
                \quad \text{s.t.} \quad
                \begin{cases}
                    (u, 1) = (u', s) \\
                    y = \rho(s)^{-1} y'
                \end{cases} \\
            \iff
                &\begin{cases}
                    u = u' \\
                    y = y'
                \end{cases}
        \end{alignat}
    \end{innerproof}
    ただし、図式の右下が
    $p^{-1}(U_\alpha) \times_\rho \R^r = \pi^{-1}(U_\alpha)$であることは
    次のようにしてわかる。
    \begin{innerproof}
        $(\subset)$ \quad
        \begin{align}
            \pi(p^{-1}(U_\alpha) \times_\rho \R^r)
                &= \pi \circ q(p^{-1}(U_\alpha) \times \R^r) \\
                &= p \circ \mathrm{pr}_1 (p^{-1}(U_\alpha) \times \R^r) \\
                &= p \circ p^{-1}(U_\alpha) \\
                &\subset U_\alpha
        \end{align}
        より$p^{-1}(U_\alpha) \times_\rho \R^r \subset \pi^{-1}(U_\alpha)$である。

        \noindent
        $(\supset)$ \quad
        $(u, y) \in p^{-1}(U_\alpha) \times \R^r$について
        $\pi(q(u, y)) \in U_\alpha$ならば
        \begin{equation}
            p(u) = p \circ \mathrm{pr}_1(u, y) \in U_\alpha
        \end{equation}
        だから$(u, y) \in p^{-1}(U_\alpha) \times \R^r$、
        したがって$q(u, y) \in p^{-1}(U_\alpha) \times_\rho \R^r$である。
    \end{innerproof}
    そこで、破線矢印の逆向きの写像$\pi^{-1}(U_\alpha) \to U_\alpha \times \R^r$を
    $\Phi_\alpha$とおく。
    各$x \in M$に対し、
    $x \in U_\alpha$なる$\alpha \in A$をひとつ選べば、
    $\Phi_\alpha(x) \colon \pi^{-1}(x) \to \{ x \} \times \R^r = \R^r$
    は可逆である。
    実際、
    \begin{equation}
        \{ x \} \times \R^r \to \pi^{-1}(x),
        \quad
        (x, y) \mapsto q(\varphi^{-1}(x, 1), y)
    \end{equation}
    が逆写像を与える。
    そこで、この 1:1 対応により$\pi^{-1}(x)$に
    $r$次元$\R$-ベクトル空間の構造を入れる。

    最後に、$U_\alpha \cap U_\beta \neq \emptyset$なる$\alpha, \beta \in A$と
    $(x, y) \in (U_\alpha \cap U_\beta) \times \R^r$に対し
    \begin{alignat}{1}
        \Phi_\alpha \circ \Phi_\beta^{-1} (x, y)
            = (x, \rho \circ \psi_{\alpha\beta} y)
    \end{alignat}
    が成り立つ。
    \begin{innerproof}
        まず
        \begin{alignat}{1}
            \Phi_\alpha \circ \Phi_\beta^{-1} (x, y)
                &= \Phi_\alpha(q(\varphi_\beta^{-1}(x, 1), y)) \\
                &= \Phi_\alpha(q(\varphi_\beta(x)^{-1}(1), y))
        \end{alignat}
        である。このとき
        \begin{align}
            (\varphi_\beta(x)^{-1}(1), y)
            &= (\sigma_\beta(x), y) \\
            &= (\sigma_\alpha(x) . \psi_{\alpha\beta}(x), y)
        \end{align}
        が成り立つから
        \begin{align}
            \Phi_\alpha(q(\varphi_\beta(x)^{-1}(1), y))
                &= \Phi_\alpha(q(\sigma_\alpha(x) . \psi_{\alpha\beta}(x), y)) \\
                &= \Phi_\alpha(
                    \sigma_\alpha(x),
                    \rho(\psi_{\alpha\beta}(x)^{-1})^{-1} y
                ) \\
                &= \Phi_\alpha(
                    \varphi_\alpha(x)^{-1}(1),
                    \rho(\psi_{\alpha\beta}(x)) y
                ) \\
                &= \Phi_\alpha \circ \Phi_\alpha^{-1}(
                    x,
                    \rho(\psi_{\alpha\beta}(x)) y
                ) \\
                &= (x, \rho \circ \psi_{\alpha\beta} y)
        \end{align}
        となる。
    \end{innerproof}
    $\rho, \psi_{\alpha\beta}$はいずれも{\smooth}だから
    $\rho \circ \psi_{\alpha\beta} \colon U_\alpha \cap U_\beta \to \GL(r, \R)$
    も{\smooth}である。

    以上で Vector Bundle Chart Lemma の条件が確認できた。
    したがって$P \times_\rho \R^r$は$M$上のベクトル束となり、
    $\{ \Phi_\alpha \}$は$P \times_\rho \R^r$の局所自明化の族となり、
    これにより定まる$P \times_\rho \R^r$の変換関数は
    $\{ \rho \circ \psi_{\alpha\beta} \}$である。
\end{proof}

\begin{example}[ベクトル束のフレーム束に同伴するベクトル束]
    $E \to M$をランク$r$ベクトル束、
    $\{ \psi_{\alpha\beta} \}$を$E$の変換関数、
    $P$を$E$から構成されたフレーム束とする。
    フレーム束の定義より、
    $\{ \psi_{\alpha\beta} \}$も$P$の変換関数であった。
    よって表現$\rho \colon \GL(r, \R) \to \GL(r, \R)$を
    恒等写像とすれば、
    $P \times_\rho \R^r$の変換関数は
    $\{ \rho \circ \psi_{\alpha\beta} = \psi_{\alpha\beta} \}$となり、
    $P \times_\rho \R^r$が$E$に一致することがわかる。
\end{example}

\begin{example}[直和束]
    \TODO{}
\end{example}

\begin{example}[テンソル積束]
    \TODO{$\rho(s) = s \otimes s$}
\end{example}

\begin{example}[双対束]
    \TODO{$\rho(s) = \up{t}s^{-1}$}
\end{example}


% ============================================================
%
% ============================================================
\chapter{アファイン接続}

接続の概念に慣れるため、
まずは多様体の接束の接続であるアファイン接続からはじめる。

\section{アファイン接続}
\label[section]{sec:affine-connection}

\begin{definition}[ベクトル束の接続]
    $M$を多様体とする。
    $M$の\term{アファイン接続}[affine connection]{アファイン接続}[あふぁいんせつぞく]とは、
    $\R$-線型写像$A^0(TM) \to A^1(TM)$であって、
    Leibniz の公式
    \begin{equation}
        \nabla(fY) = df \otimes Y + f \nabla Y
            \quad (f \in A^0(M),\; Y \in A^0(TM))
    \end{equation}
    をみたすものである。
    各$Y \in A^0(M),\; X \in \frakX(M)$に対し、
    $\nabla Y (X) \in A^0(TM)$を$\nabla_X Y$とも書き、
    $Y$の$X$方向の\term{共変微分}[covariant derivative]{共変微分}[きょうへんびぶん]と呼ぶ。
\end{definition}

\begin{example}[アファイン接続の例]
    ~
    \begin{itemize}
        \item \TODO{座標を明示せよ} $\R^n$のアファイン接続$\wb{\nabla}$を
            \begin{equation}
                \wb{\nabla}_X Y
                    \coloneqq X(Y^1) \deldel{x^1} + \dots + X(Y^n) \deldel{x^n}
                    = X(Y^i) \deldel{x^i}
            \end{equation}
            で定めることができる。
            $\wb{\nabla}$を
            \term{Euclid 接続}[Euclidean connection]{Euclid 接続}[Euclid せつぞく]
            という。
            $X$を書かずに表せば
            \begin{equation}
                \wb{\nabla} Y
                    = dY^i \deldel{x^i}
            \end{equation}
            となる。
            \cref{def:vector-bundle-connection}の Leibniz の公式の成立を確かめると、
            \begin{alignat}{1}
                \wb{\nabla}(fY)
                    &= d(fY^i) \deldel{x^i} \\
                    &= d(fY^i) \otimes \deldel{x^i}
                        \quad (\text{$TM$-値微分形式の同一視}) \\
                    &= (Y^i \,df + f \,dY^i) \otimes \deldel{x^i} \\
                    &= df \otimes Y + f \cdot \wb{\nabla} Y
            \end{alignat}
            より確かに成り立つ。
            \TODO{接続係数が0であることを述べる}
        \item \TODO{tangential connection}
    \end{itemize}
\end{example}

次に定義する接続形式とは、
局所フレームに関する接続の行列表示のようなものである。

\begin{definition}[接続形式]
    $M$を多様体、$U \opensubset M$、
    $(E_i)$を$U$上の$TM$の局所フレームとする。
    \begin{itemize}
        \item 各$j$に対し、
            \begin{equation}
                \nabla E_j = \omega^k_j E_k
            \end{equation}
            と表したときの$1$-形式$\omega^k_j$らの族$\omega \coloneqq (\omega^k_j)$を
            $\nabla$の\term{接続形式}[connection form]{接続形式}[せつぞくけいしき]という。
        \item {\smooth}関数$\Gamma_{ij}^k \colon U \to \R,$
            \begin{equation}
                \Gamma^k_{ij} \coloneqq \omega^k_j(E_i)
            \end{equation}
            を$\nabla$の
            \term{接続係数}[connection coefficient]{接続係数}[せつぞくけいすう]
            という。
            定義から明らかに、接続係数は
            \begin{equation}
                \nabla_{E_i} E_j = \Gamma^k_{ij} E_k
            \end{equation}
            をみたす。
    \end{itemize}
\end{definition}


% ------------------------------------------------------------
%
% ------------------------------------------------------------
\section{捩率と曲率}

この節では捩率テンソルと曲率テンソルを定義する。

\begin{definition}[捩率テンソル]
    $M$を多様体、
    $\nabla$を$M$のアファイン接続とする。
    このとき
    \begin{equation}
        T \colon \frakX(M) \times \frakX(M) \to \frakX(M),
        \quad
        (X, Y) \mapsto \frac{1}{2} (\nabla_X Y - \nabla_Y X - [X, Y])
    \end{equation}
    と定義すると
    $T$は交代$\smooth(M)$-双線型写像となる (このあと示す)。
    そこで$T$は$M$上の$TM$に値をもつ2次形式とみなせて、
    これを接続$\nabla$の
    \term{捩率テンソル}[torsion tensor]{捩率テンソル}[れいりつてんそる]
    という。
    捩率の値が$M$上恒等的に$0$であるとき、
    接続$\nabla$は
    \term{捩れなし}[torsion-free]{捩れなし}[ねじれなし]
    あるいは
    \term{対称}[symmetric]{対称}[たいしょう]であるという\footnote{
        「対称」という語は、
        接続が対称であるための必要十分条件
        $\Gamma^k_{ij} = \Gamma^k_{ji}$からきている
        \cite[p.121]{Lee18}。
    }。
\end{definition}

\begin{proof}
    \TODO{}
\end{proof}

\begin{definition}[曲率テンソル]
    $M$を多様体、
    $\nabla$を$M$のアファイン接続とする。
    このとき、
    \begin{equation}
        R \colon
            \frakX(M) \times \frakX(M) \times \frakX(M)
            \to \frakX(M),
        \quad
        (X, Y, Z)
            \mapsto \nabla_X \nabla_Y Z - \nabla_Y \nabla_X Z - \nabla_{[X, Y]} Z
    \end{equation}
    は$M$上の$(1, 3)$-テンソル場となり、これを
    接続$\nabla$の
    \term{曲率テンソル}[curvature tensor]{曲率テンソル}[きょくりつてんそる]
    という。
\end{definition}

捩率テンソルは、局所的には接続形式を用いて表せる。

\begin{proposition}[第1構造方程式]
    $e_1, \dots, e_n$を$TM$の局所フレーム、
    $\theta^1, \dots, \theta^n$をその双対フレームとする。
    捩率$T$は$A^2(TM)$の元だから
    \begin{equation}
        T = \sum_{i = 1}^n \Theta^i \otimes e_i
            \quad (\Theta^i \in A^2(M))
    \end{equation}
    と表せる。
    すると
    \begin{equation}
        \Theta^i(e_k, e_l) = d\theta^i(e_k, e_l) + \omega^i_j \wedge \theta^j (e_k, e_l)
    \end{equation}
    すなわち
    \begin{equation}
        \Theta^i = d\theta^i + \omega^i_j \wedge \theta^j
    \end{equation}
    が成り立つ。これをアファイン接続$\nabla$の
    \term{第1構造方程式}[first structure equation]{第1構造方程式}[だい1こうぞうほうていしき]
    という。
\end{proposition}

\begin{proof}
    \TODO{}
\end{proof}

Bianchi の第1恒等式は、
曲率テンソルの非対称性を捩率を用いて表すものである。

\begin{proposition}[Bianchi の第1恒等式]
    \begin{equation}
        R(X, Y)Z + R(Y, Z)X + R(Z, X)Y = 2(DT)(X, Y, Z)
    \end{equation}
    接続形式で書けば
    \begin{equation}
        \Omega^i_j \wedge \theta^j = d\Theta^i + \omega^i_j \wedge \Theta^j
    \end{equation}
    \TODO{}
\end{proposition}

\begin{proof}
    \TODO{}
\end{proof}

次に定義する曲率形式とは、
局所フレームに関する曲率テンソルの行列表示のようなものである。

\begin{definition}[曲率形式]
    $M$を多様体、$U \opensubset M$とし、
    $(E_i)$を$U$上の$TM$の局所フレームとする。
    各$j$に対し、
    \begin{equation}
        R(X, Y)(E_j) = \Omega^k_j(X, Y) E_k
    \end{equation}
    と表したときの$2$-形式$\Omega^k_j$らの族$\Omega \coloneqq (\Omega^k_j)$を
    $\nabla$の\term{曲率形式}[curvature form]{曲率形式}[きょくりつけいしき]という。
\end{definition}

曲率は、局所的には接続形式を用いて表せる。

\begin{proposition}[第2構造方程式]
    \label[proposition]{prop:second-structure-equation}
    $M$を多様体、$U \opensubset M$とし、
    $(E_i)$を$U$上の$TM$の局所フレームとする。
    このとき、曲率形式に関する方程式
    \begin{equation}
        \Omega^k_j
            = d\omega^k_j + \omega^k_i \wedge \omega^i_j
    \end{equation}
    が成り立つ。これを接続$\nabla$の
    \term{第2構造方程式}[second structure equation]{第2構造方程式}[だい2こうぞうほうていしき]
    という。
\end{proposition}

\begin{proof}
    \TODO{}
\end{proof}

Bianchi の第2恒等式は、
曲率テンソルの共変外微分が消えることを表す。

\begin{proposition}[Bianchi の第2恒等式]
    \begin{equation}
        DR = 0
    \end{equation}
    接続形式で書けば
    \begin{equation}
        d\Omega^\mu_\lambda
            - \Omega^\mu_\nu \wedge \omega^\nu_\lambda
            + \omega^\mu_\nu \wedge \Omega^\nu_\lambda
            = 0
    \end{equation}
    \TODO{}
\end{proposition}

\begin{proof}
    \TODO{}
\end{proof}

Ricci の恒等式は、
共変微分の非可換性を表すものである。

\begin{proposition}[Ricci の恒等式]
    \begin{equation}
        \frac{1}{2}(\nabla^2 K(X, Y) - \nabla^2 K(Y, X))
            = -R(X, Y) K + \nabla_{T(X, Y)} K
    \end{equation}
    \TODO{}
\end{proposition}

\begin{proof}
    \TODO{}
\end{proof}


% ------------------------------------------------------------
%
% ------------------------------------------------------------
\section{平行移動}

この節では測地線について述べた後、その一般化として平行の概念を導入する。

\begin{definition}[測地線]
    $M$を多様体とし、$\nabla$を$M$のアファイン接続とする。
    $M$上の曲線$\gamma$が$M$の\term{測地線}[geodesic]{測地線}[そくちせん]であるとは、
    $\gamma'$方向の$\gamma'$の共変微分が恒等的に$0$であること、すなわち
    \begin{equation}
        \nabla_{\gamma'} \gamma' \equiv 0
    \end{equation}
    が成り立つことをいう。
\end{definition}

\begin{example}[測地線の例]
    \TODO{}
\end{example}

\begin{definition}[平行]
    $M$を多様体とし、$\nabla$を$M$のアファイン接続とする。
    $\gamma$を$M$上の曲線とする。
    $\gamma$に沿ったテンソル場$V$が
    $\gamma$に沿って\term{平行}[parallel]{平行}[へいこう]であるとは、
    $\gamma'$方向の$V$の共変微分が恒等的に$0$であること、すなわち
    \begin{equation}
        \nabla_{\gamma'} V \equiv 0
    \end{equation}
    が成り立つことをいう。
\end{definition}

\begin{example}[平行なテンソル場の例]
    \TODO{}
    ~
    \begin{itemize}
        \item $\gamma$が測地線であるとは、
            その速度ベクトル場が$\gamma$自身に沿って平行であることと同値である。
    \end{itemize}
\end{example}

\begin{definition}[平行移動]
    $M$を多様体とし、$\nabla$を$M$のアファイン接続とする。
    さらに、$\gamma \colon I \to M$を$M$上の曲線とし、
    $t_0 \in I,\; v \in T_{\gamma'(t_0)} M$とする。
    このとき、$\gamma$に沿って平行なベクトル場$V$であって
    $V(t_0) = v$をみたすものがただひとつ存在し、
    このような$V$を$\gamma$に沿った$v$の
    \term{平行移動}[parallel transport]{平行移動}[へいこういどう]という。
\end{definition}




% ============================================================
%
% ============================================================
\chapter{ベクトル束の接続}

ベクトル束の接続について考える。

% ------------------------------------------------------------
%
% ------------------------------------------------------------
\section{ベクトル束の接続}

\begin{definition}[ベクトル束の接続]
    \label[definition]{def:vector-bundle-connection}
    $M$を多様体、
    $\pi \colon E \to M$をベクトル束とする。
    $E$の\term{接続}[connection]{接続}[せつぞく]とは、
    $\R$-線型写像$\nabla \colon A^0(E) \to A^1(E)$であって、
    Leibniz の公式
    \begin{equation}
        \nabla(f\xi) = df \otimes \xi + f \nabla\xi
            \quad (f \in A^0(M),\; \xi \in A^0(E))
    \end{equation}
    をみたすものである。
    各$\xi \in A^0(E),\; X \in \frakX(M)$に対し、
    $\nabla\xi(X) \in A^0(E)$を$\nabla_X\xi$とも書き、
    $\xi$の$X$方向の\term{共変微分}[covariant derivative]{共変微分}[きょうへんびぶん]と呼ぶ。
\end{definition}

定義からわかるように、
接続$\nabla$は$\smooth(M)$-線型ではないが、
2つの接続$\nabla, \nabla'$の差$\nabla - \nabla'$は$\smooth(M)$-線型である。
この事実はたとえば情報幾何学において、
$m$-接続と$e$-接続の差によって Amari-Chentsov テンソルを定義する際に用いられる。

\begin{proposition}[接続の差は$\smooth(M)$-線型]
    $M$を多様体、
    $E$を$M$上のベクトル束、
    $\nabla, \nabla'$を$E$の接続とする。
    このとき
    $\nabla - \nabla' \colon A^0(E) \to A^1(E), \;
        \xi \mapsto \nabla\xi - \nabla'\xi$
    は$\smooth(M)$-線型である。
\end{proposition}

\begin{proof}
    \begin{alignat}{1}
        (\nabla - \nabla')(f\xi)
            &= \nabla(f\xi) - \nabla'(f\xi) \\
            &= df \otimes \xi + f \nabla\xi - df \otimes \xi - f \nabla'\xi \\
            &= f (\nabla - \nabla')\xi
    \end{alignat}
    より従う。
\end{proof}

% ------------------------------------------------------------
%
% ------------------------------------------------------------
\section{接続形式}

接続形式を導入する。接続形式は、接続の座標表示にあたるものである。

\begin{definition}[接続形式]
    $M$を多様体、
    $E \to M$をランク$r$のベクトル束、
    $\nabla$を$E$の接続とする。
    さらに$U \opensubset M$、
    $\calE \coloneqq (e_1, \dots, e_r)$を$U$上の$E$のフレームとする。
    このとき、
    $U$上の$1$-形式の族$\omega = (\omega_\lambda^\mu)_{\lambda, \mu}$により
    \begin{equation}
        \nabla e_\lambda
            = \sum_{\mu} \omega_\lambda^\mu \otimes e_\mu
            \quad (\lambda = 1, \dots, r)
    \end{equation}
    と書ける。
    $\omega$をフレーム$\calE$に関する$\nabla$の
    \term{接続形式}[connection form]{接続形式}[せつぞくけいしき]という。
\end{definition}

もうひとつのフレームに関する接続形式を考えると、
ふたつの接続形式の間の変換規則が立ち現れる。

\begin{proposition}[接続形式の変換規則]
    上の定義の状況で、
    さらに$\calE' \coloneqq (e'_1, \dots, e'_r)$も$U$上の$E$のフレームとし、
    $\calE'$に関する$\nabla$の接続形式を$\omega'$とする。
    フレームの取り替えの行列$(a_\lambda^\mu)$は
    \begin{equation}
        e'_\lambda = \sum_{\mu} a_\lambda^\mu e_\mu
        \quad (a_\lambda^\mu \in A^0(U))
    \end{equation}
    とおく。
    このとき、接続形式の変換規則は
    \begin{equation}
        \omega' = a^{-1} \omega a + a^{-1} da
    \end{equation}
    となる。
\end{proposition}

\begin{proof}
    \TODO{}
\end{proof}

逆に、$\mathfrak{gl}(r, \R)$に値をもつ1-形式の族から接続を構成できる。

\begin{proposition}[接続形式から定まる接続]
    \label[proposition]{prop:connection-from-1-forms}
    $M$を多様体、
    $E \to M$をランク$r$のベクトル束とする。
    $\{ U_\alpha \}_{\alpha \in A}$を$M$の open cover であって
    各$U_\alpha$上で$g$に関するフレーム
    $\calE_\alpha = (e^{(\alpha)}_1, \dots, e^{(\alpha)}_r)$
    を持つものとする。
    さらに、各$U_\alpha$上の局所自明化$\varphi_\alpha$を
    $\calE_\alpha$から定め、
    変換関数を$\{ \psi_{\alpha\beta} \}$とおく。
    このとき、$\mathfrak{gl}(r, \R)$に値をもつ$1$-形式の族
    \begin{equation}
        \omega = \{ \omega_\alpha \}_{\alpha \in A}
    \end{equation}
    であって、変換規則
    \begin{equation}
        \omega_\beta
            = \psi_{\alpha\beta}^{-1} \omega_\alpha \psi_{\alpha\beta}
            + \psi_{\alpha\beta}^{-1} \, d \psi_{\alpha\beta}
            \quad
            \text{on $U_\alpha \cap U_\beta$}
    \end{equation}
    をみたすものが与えられたならば、
    次をみたす$E$の接続が構成できる:
    \begin{enumerate}
        \item 各フレーム$\calE_\alpha$に関する
            $\nabla$の接続形式は$\omega_\alpha$である。
    \end{enumerate}
\end{proposition}

\begin{proof}
    \TODO{}
\end{proof}

\TODO{大域的な与え方と接続形式による与え方}

\begin{definition}[直和束の接続]
    \TODO{}
\end{definition}

\begin{definition}[テンソル積束の接続]
    \TODO{}
\end{definition}

\begin{definition}[双対束の接続]
    \TODO{}
\end{definition}

\begin{definition}[引き戻し束の接続]
    \TODO{}
\end{definition}

% ------------------------------------------------------------
%
% ------------------------------------------------------------
\section{ベクトル束の共変外微分と曲率}

微分形式に対する外微分を一般化し、
ベクトル束に値をもつ微分形式に対し共変外微分とよばれる演算を定義する。
さらに共変外微分から曲率を定義する。

\begin{definition}[共変外微分]
    $M$を多様体、
    $E \to M$をベクトル束、
    $\nabla$を$E$の接続、
    $p \in \Z_{\ge 0}$とする。
    $\R$-線型写像$D \colon A^p(E) \to A^{p + 1}(E)$を
    \begin{equation}
        D(\theta \otimes \xi)
            \coloneqq d\theta \otimes \xi + \theta \wedge \nabla\xi
            \quad (\theta \in A^p(M), \; \xi \in A^0(E))
    \end{equation}
    で定め、$D$を
    \term{共変外微分}[covariant exterior derivative]{共変外微分}[きょうへんがいびぶん]
    という。
\end{definition}

\begin{remark}
    \label[remark]{remark:covariant-exterior-derivative-and-vector-fields}
    とくに$p = 1$のとき、
    $\varphi \in A^1(E)$に対し
    \begin{equation}
        (D\varphi)(X, Y)
            = \nabla_X (\varphi(Y))
            - \nabla_Y (\varphi(X))
            - \varphi([X, Y])
            \quad
            (X, Y \in \Gamma(TM))
    \end{equation}
    が成り立つ。
    これは通常の外微分の公式\cref{remark:exterior-derivative-and-vector-fields}
    の拡張になっている。
\end{remark}

共変外微分は次の性質をみたす。

\begin{proposition}[共変外微分の anti-derivation 性 (外積に関して)]
    $M$を多様体、
    $E \to M$をベクトル束、
    $\nabla$を$E$の接続、
    $D$を$\nabla$から定まる共変外微分とする。
    $p, q \in \Z_{\ge 0}$に対し
    \begin{equation}
        D(\theta \wedge \varphi)
            = d\theta \wedge \varphi + (-1)^p \theta \wedge D\varphi
            \quad
            (\theta \in A^p(M), \; \varphi \in A^q(E))
    \end{equation}
    が成り立つ\footnote{
        [小林]では$E$-値形式を表すときの$\xi$と$\theta$の順序が逆なので
        \begin{equation}
            D(\varphi \wedge \theta)
                = D\varphi \wedge \theta + (-1)^p \varphi \wedge d\theta
        \end{equation}
        という形になっている。
    }。
\end{proposition}

\begin{proof}
    \TODO{}
\end{proof}

ある性質を満たす双線型写像に対し、
共変外微分は anti-derivation 性をみたす。

\begin{proposition}[共変外微分の anti-derivation 性 (双線型写像に関して)]
    \label[proposition]{prop:signed-leibniz-rule}
    $M$を多様体、
    $E \to M, \; E' \to M$をベクトル束、
    $\nabla, \nabla'$をそれぞれ$E, E'$の接続、
    $D, D'$をそれぞれ$\nabla, \nabla'$から定まる共変外微分、
    $g \colon A^0(E) \times A^0(E') \to A^0(M)$を
    $\smooth(M)$-双線型写像とする。
    $D, D'$が条件
    \begin{equation}
        d(g(\xi, \eta)) = g(\nabla\xi, \eta) + g(\xi, \nabla'\eta)
            \quad
            (\xi \in A^0(E), \; \eta \in A^0(E'))
    \end{equation}
    をみたすならば\footnote{
        とくに$g$が$M$の Riemann 計量で
        $E = E' = TM$の状況でこの条件が成り立っているならば、
        $\nabla'$は$g$に関する$\nabla$の
        \term{双対接続}[dual connection]{双対接続}[そうついせつぞく]
        であるという。
    }、
    $p, q \in \Z_{\ge 0}$に対し
    \begin{equation}
        d(g(\xi, \eta))
            = g(D\xi, \eta) + (-1)^p g(\xi, D'\eta)
            \quad
            (\xi \in A^p(E), \; \eta \in A^q(E'))
    \end{equation}
    が成り立つ。
\end{proposition}

\begin{proof}
    Einstein の記法を用いる。
    $A^p(E), A^q(E')$の元はそれぞれ
    \begin{align}
        &\alpha \otimes \xi \in A^p(E)
            \quad (\alpha \in A^p(M), \; \xi \in A^0(E)) \\
        &\beta \otimes \eta \in A^q(E')
            \quad (\beta \in A^q(M), \; \eta \in A^0(E'))
    \end{align}
    の形の元の有限和で書けるから、
    このような形の元について示せば十分である。
    \begin{align}
        \nabla \xi = \alpha^i \otimes \xi_i,
            \quad
            (\alpha^i \in A^1(M), \; \xi_i \in A^0(E)) \\
        \nabla \eta = \beta^j \otimes \eta_j
            \quad
            (\beta^j \in A^1(M), \; \eta_j \in A^0(E'))
    \end{align}
    とおいておく (ただし、Einstein の記法を使うために共変・反変による
    添字の上下の慣例を一時的に無視している)。
    まず
    \begin{alignat}{1}
        &\quad d(g(\alpha \otimes \xi, \beta \otimes \eta)) \\
        &= d(g(\xi, \eta) \alpha \wedge \beta)
            \quad (\text{$\because$ ベクトル束値形式の内積の定義}) \\
        &= d(g(\xi, \eta)) \alpha \wedge \beta
            + g(\xi, \eta) d\alpha \wedge \beta
            + (-1)^p g(\xi, \eta) \alpha \wedge d\beta \\
        &= d(g(\xi, \eta)) \alpha \wedge \beta
            + g(\xi \otimes d\alpha, \eta \otimes \beta)
            + (-1)^p g(\xi \otimes \alpha, \eta \otimes d\beta)
            \label[equation]{eq:signed-leibniz-rule-1}
    \end{alignat}
    となる。ここで、第1項は
    \begin{alignat}{1}
        &\quad d(g(\xi, \eta)) \alpha \wedge \beta \\
        &= g(\nabla \xi, \eta) \alpha \wedge \beta
            + g(\xi, \nabla' \eta) \alpha \wedge \beta
            \quad (\text{$\because$ 命題の仮定}) \\
        &= g(\xi_i, \eta) \alpha^i \wedge \alpha \wedge \beta
            + g(\xi, \eta_j) \beta^j \wedge \alpha \wedge \beta \\
        &= g(\xi_i, \eta) \alpha^i \wedge \alpha \wedge \beta
            + (-1)^p g(\xi, \eta_j) \alpha \wedge \beta^j \wedge \beta \\
        &= g(\xi_i \otimes \alpha^i \wedge \alpha, \eta \otimes \beta)
            + (-1)^p g(\xi \otimes \alpha, \eta_j \beta^j \wedge \beta) \\
        &= g(\nabla \xi \wedge \alpha, \eta \otimes \beta)
            + (-1)^p g(\xi \otimes \alpha, \nabla' \eta \wedge \beta)
    \end{alignat}
    となる。
    したがって、\cref{eq:signed-leibniz-rule-1}より
    \begin{alignat}{1}
        d(g(\alpha \otimes \xi, \beta \otimes \eta))
            &= g(\nabla \xi \wedge \alpha, \eta \otimes \beta)
                + g(\xi \otimes d\alpha, \eta \otimes \beta) \\
            &\qquad \quad + (-1)^p g(\xi \otimes \alpha, \nabla' \eta \wedge \beta)
                + (-1)^p g(\xi \otimes \alpha, \eta \otimes d\beta) \\
            &= g(D(\xi \wedge \alpha), \eta \otimes \beta)
                + (-1)^p g(\xi \otimes \alpha, D'(\eta \wedge \beta))
    \end{alignat}
    が成り立つ。
\end{proof}

曲率を定義する。

\begin{definition}[曲率]
    $M$を多様体、
    $E \to M$をベクトル束、
    $\nabla$を$E$の接続、
    $D$を$\nabla$により定まる共変外微分とする。
    $R \coloneqq D^2$とおき、
    $R$を$\nabla$の
    \term{曲率}[curvature]{曲率}[きょくりつ]
    という。
\end{definition}

\begin{proposition}
    写像$R = D^2 \colon A^0(E) \to A^2(E)$は
    $A^2(\End E)$の元ともみなせる。
    \TODO{why?}
\end{proposition}

\begin{proof}
    \TODO{}
\end{proof}

曲率は、局所的には接続形式を用いて表せる。
ここで登場する構造方程式は、アファイン接続の場合の第2構造方程式
(\cref{prop:second-structure-equation})
に他ならない。
それでは第1構造方程式はどこにいったのかと気になるが、
一般の接続では捩率が定義できないから第1構造方程式にあたるものは登場しない。
Bianchi の恒等式に関しても同様である。

\begin{proposition}[構造方程式]
    \label[proposition]{prop:structure-equation}
    \begin{equation}
        \Omega^\mu_\lambda = d\omega^\mu_\lambda + \omega^\mu_\nu \wedge \omega^\nu_\lambda
    \end{equation}
    \TODO{}
\end{proposition}

\begin{proof}
    \TODO{}
\end{proof}

\begin{proposition}[Bianchi の恒等式]
    \begin{equation}
        DR = 0
    \end{equation}
    接続形式で書けば
    \begin{equation}
        d\Omega^\mu_\lambda
            - \Omega^\mu_\nu \wedge \omega^\nu_\lambda
            + \omega^\mu_\nu \wedge \Omega^\nu_\lambda
            = 0
    \end{equation}
    \TODO{}
\end{proposition}

\begin{proof}
    \TODO{}
\end{proof}

\begin{proposition}[Ricci の恒等式]
    共変外微分の公式\cref{remark:covariant-exterior-derivative-and-vector-fields}で
    $\varphi = D\xi, \; \xi \in A^0(E)$とおいて計算すると
    \begin{equation}
        R(X, Y) \xi
            = D(D\xi)(X, Y)
            = (\nabla_X \nabla_Y - \nabla_Y \nabla_X - \nabla_{[X, Y]}) \xi
    \end{equation}
    すなわち
    \begin{equation}
        R(X, Y) = \nabla_X \nabla_Y - \nabla_Y \nabla_X - \nabla_{[X, Y]}
    \end{equation}
    を得る。
    これを\term{Ricci の恒等式}[Ricci's identity]{Ricci の恒等式}[Ricci のこうとうしき]
    という。
\end{proposition}

\begin{proof}
    \TODO{}
\end{proof}

\begin{definition}[直和束の曲率]
    \TODO{}
\end{definition}

\begin{definition}[テンソル積束の曲率]
    \TODO{}
\end{definition}

\begin{definition}[双対束の曲率]
    \TODO{}
\end{definition}

\begin{definition}[引き戻し束の曲率]
    \TODO{}
\end{definition}



% ============================================================
%
% ============================================================
\chapter{主ファイバー束の接続}

前章ではベクトル束の接続について考えた。
この章ではまず主ファイバー束の接続について2通りの定義を述べた後、
主ファイバー束の接続とベクトル束の接続との対応について調べる。

% ------------------------------------------------------------
%
% ------------------------------------------------------------
\section{主ファイバー束の接続 (微分形式)}

主$G$束$P$の接続の定義の方法は2つあり、
\begin{enumerate}
    \item 1つ目は$P$上の$\frakg$値1形式としての定義である。
    \item 2つ目は$T_u P$から垂直部分空間への射影としての定義である。
        こちらは幾何学的な様子がわかりやすいという利点がある。
\end{enumerate}
この節ではまずは$\frakg$値1形式としての定義で接続を導入する。

\subsection{主ファイバー束の接続形式}

\cref{prop:connection-from-1-forms} で見たように、
ベクトル束の接続は
$\GL(r; \R)$値の変換関数と
$\mathfrak{gl}(r; \R)$値の1形式により定めることができた。
そこで、主$G$束でも同様の方法により
$\frakg$値1形式として接続を定義する。

\TODO{こちらはむしろ特徴付けにするべきでは?
    ゲージポテンシャルを主役とみる立場ならこちらが定義?}

\begin{definition}[主ファイバー束の接続形式]
    $M$を多様体、
    $G$を Lie 群、$\frakg$を$G$の Lie 代数、
    $p \colon P \to M$を主$G$束とする。
    $\{ (U_\alpha, \varphi_\alpha) \}_{\alpha \in A}$
    を$\bigcup U_\alpha = P$なる$P$の局所自明化の族とし、
    これにより定まる切断の族を$\{ \sigma_\alpha \}$とおく。
    さらに各$\alpha$に対し、
    $\omega_\alpha$を$U_\alpha$上の
    $\frakg$値$1$-形式であって関係式
    \begin{equation}
        \omega_\beta = \varphi_{\alpha\beta}^{-1} \omega_\alpha \varphi_{\alpha\beta}
            + \varphi_{\alpha\beta}^{-1} d\varphi_{\alpha\beta}
            \quad \text{on} \quad
            U_\alpha \cap U_\beta
    \end{equation}
    をみたすものとする。
    このとき、$P$上の$\frakg$値$1$-形式$\omega$を
    各$p^{-1}(U_\alpha) \subset P$上で
    \begin{equation}
        \omega \coloneqq
            s_\alpha^{-1} (\pi^* \omega_\alpha) s_\alpha
            + s_{\alpha}^{-1} ds_\alpha
    \end{equation}
    と定めることができる (このあとすぐ示す)。
    ただし右辺の積は$TG$における積であり、
    $s_\alpha$は
    \begin{equation}
        s_\alpha \coloneqq \mathrm{pr}_2 \circ \varphi_\alpha
        \colon \pi^{-1}(U_\alpha) \to G,
        \quad
        (x, \sigma_\alpha(x) . s) \mapsto s
    \end{equation}
    と定めた。
    $\omega$を$P$の
    \term{接続形式}[connection form]{接続形式!主ファイバー束の---}[せつぞくけいしき]
    という。
\end{definition}

\begin{proof}
    \uline{($\Rightarrow$)} \quad
    $\{ (U_\alpha, \varphi_\alpha) \}_{\alpha \in A}$
    を$\bigcup U_\alpha = P$なる$P$の局所自明化の族とし、
    これにより定まる切断の族を$\{ \sigma_\alpha \}$とおき、
    $\omega$はこれにより定まる$P$の接続形式であるとする。
    \TODO{}
\end{proof}

\begin{remark}
    $\sigma_\alpha^* \omega = \omega_\alpha$が成り立つ\footnote{
        $\sigma_\alpha^* \omega$は物理学では
        \term{ゲージポテンシャル}[gauge potential]{ゲージポテンシャル}
        と呼ばれる。
    }。
    \TODO{}
\end{remark}

\begin{theorem}[主ファイバー束の接続形式の特徴付け]
    $M$を多様体、
    $G$を Lie 群、$\frakg$を$G$の Lie 代数、
    $p \colon P \to M$を主$G$束とする。
    $\frakg$に値をもつ$P$上の$1$-形式$\omega$に関し、
    $\omega$が$P$の接続形式であることと
    $\omega$がつぎの条件をみたすこととは同値である:
    \begin{enumerate}
        \item ($G$-同変性) $R_a^* \omega = (\Ad g^{-1}) \omega \quad (a \in G)$
        \item $\omega(A^*) = A \quad (A \in \frakg)$
    \end{enumerate}
\end{theorem}

\begin{proof}
    \TODO{}
\end{proof}

\begin{definition}[Ehresmann 接続]
    $M$を多様体、$p \colon P \to M$を主$G$束とする。
    上の定理の条件 (1), (2) をみたす
    $P$上の$\frakg$値1形式$\omega$を
    $P$上の\term{Ehresmann 接続}[Ehresmann connection]{Ehresmann 接続}[Ehresmann せつぞく]
    あるいは単に
    \term{接続}[connection]{接続!主ファイバー束の---}[せつぞく]
    という。
\end{definition}

主ファイバー束の接続に対し、
Lie 群の構造方程式
(\cref{def:lie-group-structure-equation})
と類似の方程式が成り立つ。
したがって
\begin{itemize}
    \item $P$の接続は$G$の接続の一般化
    \item $P$の接続の構造方程式は$G$の構造方程式の一般化
\end{itemize}
とみることができる。\TODO{どういう意味?}

\begin{proposition}[接続の構造方程式]
    \begin{equation}
        d\omega = - [\omega, \omega]
    \end{equation}
    \TODO{}
\end{proposition}

\begin{proof}
    \TODO{}
\end{proof}

\subsection{主ファイバー束の曲率}

主ファイバー束の曲率を定義する。
ベクトル束の接続の曲率は構造方程式 (\cref{prop:structure-equation})
をみたすのであった。
そこで、主ファイバー束の接続の曲率は逆にこの方程式によって定義する。

\begin{definition}[曲率形式]
    \TODO{}
\end{definition}

% ------------------------------------------------------------
%
% ------------------------------------------------------------
\section{主ファイバー束の接続 (水平部分空間の方法)}

前節では主ファイバー束の接続を微分形式として定義した。
この節では水平部分空間の方法を用いて接続を定義する。

\subsection{接分布}

まず基本的な概念を導入しておく。

\begin{definition}[接分布]
    $M$を多様体とする。
    $D \subset TM$が$M$上の
    \term{接分布}[tangent distribution]{接分布}[せつぶんぷ]
    であるとは、
    $D$が$TM$の部分ベクトル束であることをいう。
\end{definition}

\begin{definition}[積分多様体]
    $M$を多様体、$D \subset TM$を$M$上の接分布とする。
    部分多様体$N \subset M$が$D$の
    \term{積分多様体}[integral manifold]{積分多様体}[せきぶんたようたい]
    であるとは、
    \begin{equation}
        T_xN = D_x
            \quad
            (\forall x \in N)
    \end{equation}
    が成り立つことをいう。

    各$x \in M$に対し
    $D$のある積分多様体$N \subset M$が存在して
    $x \in N$となるとき、
    $D$は\term{積分可能}[integrable]{積分可能}[せきぶんかのう]
    であるという。
\end{definition}

\begin{definition}[包合的]
    $M$を多様体、$D \subset TM$を$M$上の接分布とする。
    $D$が\term{包合的}[involutive]{包合的}[ほうごうてき]であるとは、
    $D$の任意の局所切断$X, Y$に対し
    $[X, Y]$も$D$の局所切断となることをいう。
\end{definition}

\begin{theorem}[Frobenius]
    \TODO{}
\end{theorem}

\subsection{垂直接分布と水平接分布}

垂直接分布を定義する。
垂直接分布は接続とは関係なく主ファイバー束の構造のみによって決まる。

\begin{definition}[垂直接分布]
    $M$を多様体、$p \colon P \to M$を主$G$束とする。
    各$u \in P$に対し、$\R$-部分ベクトル空間
    \begin{equation}
        V_u \coloneqq \Ker p_*
    \end{equation}
    を$T_uP$の\term{垂直部分空間}[vertical subspace]{垂直部分空間}[すいちょくぶぶんくうかん]
    という。
    さらに$\coprod_{u \in P} V_u$は$TP$の部分ベクトル束となり、
    これを\term{垂直接分布}[vertical distribution]{垂直接分布}[すいちょくせつぶんぷ]という。

    垂直接分布に属する元は
    \term{垂直}[vertical]{垂直}[すいちょく]であるという。
\end{definition}

垂直部分空間は次のように表せる。
これにより$V_u$と$\frakg$を同一視すれば、
主ファイバー束の接続形式の条件$\omega(A^*) = A$とは
$\omega$が$V_u$上恒等写像であるという条件に他ならない。

\begin{proposition}
    $V_u = \{ A^*_u \mid A \in \frakg \}$
    \TODO{}
\end{proposition}

\begin{proof}
    \TODO{}
\end{proof}

\begin{theorem}[垂直接分布は積分可能]
    \TODO{}
\end{theorem}

\begin{proof}
    \TODO{}
\end{proof}

次に水平部分空間を定義する。
水平部分空間とは$T_u P = V_u \oplus H_u$なる部分空間$H_u$のことであるが、
$H_u$は主ファイバー束の構造のみからは決定されない。
後で詳しく見るが、主ファイバー束に接続を与えることは、
本質的には右不変な水平部分空間を選ぶのと同じことである。

\TODO{あとで接続から水平部分空間が定まることをいうのだから、
    水平部分空間の定義には接続を含むべきでないのでは?}

\begin{definition}[水平部分空間]
    $M$を多様体、$p \colon P \to M$を主$G$束、
    $\omega$を$P$の接続形式とする。
    各$u \in P$に対し、$\R$-部分ベクトル空間
    \begin{equation}
        H_u \coloneqq \Ker \omega_u
    \end{equation}
    を$T_uP$の\term{水平部分空間}[horizontal subspace]{水平部分空間}[すいへいぶぶんくうかん]
    という。
\end{definition}

\subsection{水平接分布と接続}

水平接分布により主ファイバー束の接続を特徴付ける。

\begin{theorem}[水平接分布から接続へ]
    $p \colon P \to M$を主$G$束、
    $H$を右不変な水平接分布とする。
    このとき、$P$上の$\frakg$値1形式$\omega$を
    \begin{equation}
        \omega_u \colon T_u P \to V_u \to \frakg
    \end{equation}
    により定めると、$\omega$は$P$上の接続となる。
\end{theorem}

\begin{proof}
    \TODO{cf. [Tu] p.255}
\end{proof}

\begin{theorem}[接続から水平接分布へ]
    $p \colon P \to M$を主$G$束、
    $\omega$を$P$上の接続とする。
    このとき、$H_u \coloneqq \Ker \omega_u$は
    $P$の右不変な水平部分空間である。
\end{theorem}

\begin{proof}
    \TODO{cf. [Tu] p.257}
\end{proof}

主ファイバー束の曲率形式も
水平接分布を用いて表すことができる。

\begin{proposition}
    接続形式$\omega$の曲率形式$\Omega$は
    \begin{equation}
        \Omega(X, Y) = d\omega(X^H, Y^H)
            \quad
            (X, Y \in T_uP)
    \end{equation}
    をみたす。
\end{proposition}

\begin{proof}
    \TODO{}
\end{proof}

水平接分布の積分可能性と曲率には密接な関係がある。

\begin{theorem}[水平接分布の積分可能性]
    $P$の水平接分布$\coprod_{u \in P} H_u$に関し次は同値である:
    \begin{enumerate}
        \item $\coprod H_u$は積分可能である。
        \item $P$の曲率は$0$である。
    \end{enumerate}
\end{theorem}

\begin{proof}
    \TODO{}
\end{proof}


% ------------------------------------------------------------
%
% ------------------------------------------------------------
\section{同伴ベクトル束の接続}

同伴ベクトル束を思い出そう。
\cref{subsec:principal-fiber-bundle-to-vector-bundle}
で見たように、主ファイバー束$P$と表現$\rho$から
同伴ベクトル束$E = P \times_\rho \R^r$が構成できるのであった。
このとき、$P$の共変外微分から$E$に共変外微分が誘導される。
とくに$P$の接続形式から$E$の接続形式が定まる。

\begin{theorem}[微分形式の対応]
    $P$上の$\R^r$値$p$-形式$\widetilde{\xi}$で
    \begin{enumerate}
        \item $R_a^* \widetilde{\xi} = \rho(a)^{-1} \widetilde{\xi} \quad (a \in G)$
        \item ある$i$で$X_i$が垂直ならば
            $\widetilde{\xi}(X_1, \dots, X_p) = 0$
    \end{enumerate}
    をみたすもの全体の空間を$\widetilde{A}^p(P)$とおく。
    $A^p(E)$と$\widetilde{A}^p(P)$は
    次の対応により1:1に対応する。
    \begin{equation}
        \widetilde{\xi}(X_1, \dots, X_p)
            = u^{-1} (\xi (\pi_* X_1, \dots, \pi_* X_p))
            \quad
            (X_1, \dots, X_p \in T_u P)
    \end{equation}
    \TODO{}
\end{theorem}

\begin{proof}
    \TODO{}
\end{proof}

\begin{proposition}
    上の定理の対応は
    外積代数$A(E)$から$\widetilde{A}(P)$への
    $A(M)$-加群同型である。
    \TODO{}
\end{proposition}

\begin{proof}
    \TODO{}
\end{proof}

$\widetilde{A}(P)$に共変外微分を定義し、
上の同型により$A(E)$に共変外微分を誘導する。

\begin{definition}[$\widetilde{A}(P)$の共変外微分]
    \begin{equation}
        D\widetilde{\xi}(X_1, \dots, X_{p + 1})
            = d\widetilde{\xi}(X^H_1, \dots, X^H_{p + 1})
            \quad
            (X_1, \dots, X_{p + 1} \in T_u P)
    \end{equation}
    \TODO{}
\end{definition}

$D\widetilde{\xi}$は次のように書くこともできる。

\begin{proposition}
    \begin{equation}
        D\widetilde{\xi} = d\widetilde{\xi} + \rho(\omega) \wedge \widetilde{\xi}
    \end{equation}
    \TODO{}
\end{proposition}

\begin{proof}
    \TODO{}
\end{proof}

% ------------------------------------------------------------
%
% ------------------------------------------------------------
\section{平行移動とホロノミー}

この節では、平行移動とホロノミーについて述べる。

\subsection{ベクトル束の平行移動とホロノミー}

さて、ここで曲線$\gamma$に沿う$\xi$の共変微分
「$\nabla_{\dot{\gamma}(t)} \xi$」を定義したい。
ところが、ややこしいことに「曲線$\gamma$に沿う$E$の切断」は「$E$の切断」ではないため、
「$\nabla_{\dot{\gamma}(t)} \xi$」という文字列に
正確な意味を与えるにはさらなる定義が必要となる。

\TODO{引き戻しを用いて定義したほうがよさそう cf. [Tu] p. 262}

\begin{definition}[曲線に沿う切断の拡張可能性]
    $M$を多様体、
    $\pi \colon E \to M$をベクトル束、
    $J$を$\R$の区間、
    $\gamma \colon J \to M$を{\smooth}曲線とする。
    曲線$\gamma$に沿う$E$の切断$\xi \colon J \to E$が
    \term{拡張可能}[extendible]{拡張可能}[かくちょうかのう]
    であるとは、
    $\gamma$の像$\gamma(J)$を含む$U \opensubset M$と
    $U$上の$E$の切断$\widetilde{\xi}$が存在して
    \begin{equation}
        \widetilde{\xi}_{\gamma(t)} = \xi_t
            \quad
            (\forall t \in J)
    \end{equation}
    が成り立つことをいう。
    $\widetilde{\xi}$を
    \term{$\xi$の拡張}{拡張!曲線に沿う切断の---}[かくちょう]という。
\end{definition}

\begin{example}[拡張可能でない例]
    8の字曲線$\gamma \colon (-\pi, \pi) \to \R^2, \;
    t \mapsto (\sin t, \sin t \cos t)$の
    速度ベクトル$\dot{\gamma}$は拡張可能でない。
    なぜならば、8の字の中央部分で速度ベクトルが2方向に出ているからである。
\end{example}

\begin{definition}[曲線に沿う共変微分]
    上の定義の状況で、
    さらに$\nabla$を$E$の接続とし、
    $\xi$は拡張可能であるとする。
    このとき、$\xi$の拡張$\widetilde{\xi} \in \Gamma(E)$をひとつ選び
    \begin{equation}
        \nabla_{\dot{\gamma}(t)} \xi
            \coloneqq \nabla_{\dot{\gamma}(t)} \widetilde{\xi}
            \quad
            (t \in J)
    \end{equation}
    と定義し、これを
    \term{曲線$\gamma$に沿う共変微分}[covariant derivative along $\gamma$]
    {曲線に沿う共変微分}[きょくせんにそうきょうへんびぶん]という。
    これは$\widetilde{\xi}$の選び方によらず well-defined に定まる (このあと示す)。
\end{definition}

\begin{proposition}
    上の定義の状況で、
    さらに$\widetilde{\xi} \in \Gamma(E)$を$\xi$の拡張、
    $t \in J$、
    $U$を$M$における$\gamma(t)$の開近傍、
    $e_1, \dots, e_r$を$U$上の$E$の局所フレーム、
    $x^1, \dots, x^n$を$U$上の$M$の局所座標とする。
    $\widetilde{\xi}$を局所的に
    \begin{alignat}{1}
        \widetilde{\xi} &= \widetilde{\xi}^\lambda e_\lambda
            \quad
            (\widetilde{\xi}^\lambda \in \smooth(U))
    \end{alignat}
    と表し、$\xi^\lambda \coloneqq \widetilde{\xi}^\lambda \circ \gamma$とおく。
    また$\nabla e_\mu$を局所的に
    \begin{alignat}{1}
        \nabla e_\mu
            &= \omega_\mu^\lambda \otimes e_\lambda
                \quad
                (\omega_\mu^\lambda \in A^1(U)) \\
            &= \Gamma^\lambda_{\mu i} dx^i \otimes e_\lambda
                \quad
                (\Gamma^\lambda_{\mu i} \in \smooth(U))
    \end{alignat}
    と表す。
    このとき
    \begin{equation}
        \nabla_{\dot{\gamma}(t)} \xi
            = \left\{
                \frac{d\xi^\lambda}{dt}(t)
                +
                \xi^\mu (t)
                \Gamma^\lambda_{\mu i} (\gamma(t))
                \frac{d\gamma^i}{dt}(t)
            \right\}
            (e_\lambda)_{\gamma(t)}
    \end{equation}
    が成り立つ。
    したがってとくに$\nabla_{\dot{\gamma}(t)} \xi$の値は
    $\widetilde{\xi}$の選び方によらず well-defined に定まる。
\end{proposition}

\begin{proof}
    まず記法を整理すると
    \begin{equation}
        \nabla_{\dot{\gamma}(t)} \xi
            = \nabla_{\dot{\gamma}(t)} \widetilde{\xi}
            = (\nabla \widetilde{\xi}) (\dot{\gamma}(t))
            = \underbrace{
                (\nabla \widetilde{\xi})_{\gamma(t)}
            }_{\in \, T^*_{\gamma(t)} M \otimes E_{\gamma(t)}}
            (\underbrace{\dot{\gamma}(t)}_{\in \, T_{\gamma(t)} M})
    \end{equation}
    と書けることに注意する。
    そこで$\nabla \widetilde{\xi}$を変形すると
    \begin{alignat}{1}
        \nabla \widetilde{\xi}
            &= d\widetilde{\xi}^\lambda \otimes e_\lambda
                + \widetilde{\xi}^\mu \nabla e_\mu \\
            &= d\widetilde{\xi}^\lambda \otimes e_\lambda
                + \widetilde{\xi}^\mu \Gamma^\lambda_{\mu i} dx^i \otimes e_\lambda \\
            &= \left\{
                d\widetilde{\xi}^\lambda
                + \widetilde{\xi}^\mu \Gamma^\lambda_{\mu i} dx^i
            \right\} \otimes e_\lambda
    \end{alignat}
    となるから、点$\gamma(t)$での値は
    \begin{alignat}{1}
        (\nabla \widetilde{\xi})_{\gamma(t)}
            &= \left\{
                d\widetilde{\xi}^\lambda_{\gamma(t)}
                + \widetilde{\xi}^\mu (\gamma(t))
                \Gamma^\lambda_{\mu i} (\gamma(t))
                dx^i_{\gamma(t)}
            \right\} \otimes (e_\lambda)_{\gamma(t)} \\
            &= \left\{
                d\widetilde{\xi}^\lambda_{\gamma(t)}
                + \xi^\mu (\gamma(t))
                \Gamma^\lambda_{\mu i} (\gamma(t))
                dx^i_{\gamma(t)}
            \right\} \otimes (e_\lambda)_{\gamma(t)}
    \end{alignat}
    である。
    ここで
    \begin{alignat}{1}
        (d\widetilde{\xi}^\lambda)_{\gamma(t)} (\dot{\gamma}(t))
            &= \dd{t}\bigg|_{t = t} \widetilde{\xi}^\lambda \circ \gamma(t)
            = \frac{d\xi^\lambda}{dt}(t) \\
        (dx^i)_\gamma(t) (\dot{\gamma}(t))
            &= \dd{t}\bigg|_{t = t} x^i \circ \gamma(t)
            = \frac{d\gamma^i}{dt}(t)
    \end{alignat}
    だから
    \begin{alignat}{1}
        \nabla_{\dot{\gamma}(t)} \xi
            = (\nabla \widetilde{\xi})_{\gamma(t)}
            = \left\{
                \frac{d\xi^\lambda}{dt}(t)
                +
                \xi^\mu (t)
                \Gamma^\lambda_{\mu i} (\gamma(t))
                \frac{d\gamma^i}{dt}(t)
            \right\}
            (e_\lambda)_{\gamma(t)}
    \end{alignat}
    を得る。
    関数$\xi^\lambda$は
    $\widetilde{\xi}$の選び方によらないから
    well-defined 性もいえた。
\end{proof}

測地線の一般化として、
平行の概念を定義する。

\begin{definition}[平行]
    $M$を多様体、$E \to M$をベクトル束、
    $\nabla$を$E$の接続、
    $J$を$\R$の区間、
    $\gamma \colon J \to M$を{\smooth}曲線、
    $\xi$を曲線$\gamma$に沿う$E$の切断とする。
    $\xi$が
    \begin{equation}
        \nabla_{\dot{\gamma}(t)} \xi = 0
            \quad
            (\forall t \in J)
    \end{equation}
    をみたすとき、$\xi$は
    \term{曲線$\gamma$に沿って平行}[parallel along $\gamma$]{平行}[へいこう]
    であるという。
    上の命題より、これは次の斉次1階常微分方程式系が成り立つことと同値である:
    \begin{equation}
        \frac{d\xi^\lambda}{dt}(t)
            + \xi^\mu (t)
            \Gamma^\lambda_{\mu i} (\gamma(t))
            \frac{d\gamma^i}{dt}(t)
            = 0
            \quad
            (\lambda = 1, \ldots, r)
    \end{equation}
    $\bm{\xi} \coloneqq \up{t}(\xi^1, \dots, \xi^r), \;
    A \coloneqq \left(
        \Gamma^{\lambda}_{\mu i} \frac{d\gamma^i}{dt}
    \right)_{\lambda, \mu}$
    とおけば
    \begin{equation}
        \frac{d\bm{\xi}}{dt} = - A \bm{\xi}
    \end{equation}
    と書ける。
\end{definition}

\begin{remark}
    測地線とは、
    その速度ベクトルが自身に沿って平行な曲線のことである。
\end{remark}

\begin{definition}[平行移動]
    上の命題の状況で
    さらに$J = [a, b], \; a, b \in \R$とするとき、
    初期値問題の解の存在と一意性より
    任意の$\xi_a \in E_{\gamma(a)}$に対し
    $\xi(a) = \xi_a$なる
    解$\xi$が一意に定まる。
    このとき、$\xi$は
    $\xi_a$を曲線$\gamma$に沿って
    \term{平行移動}[parallel displacement]{平行移動}[へいこういどう]
    して得られたという\footnote{
        最適化の分野では、
        指数写像や平行移動の数値計算のために、
        これらの代替となる
        \term{レトラクション}[retraction]{レトラクション}[れとらくしょん]
        や
        \term{ベクトル輸送}[vector transport]{ベクトル輸送}[べくとるゆそう]
        が用いられる。
    }。
\end{definition}

\begin{proposition}
    上の定義の状況で、
    写像
    \begin{equation}
        E_{\gamma(a)} \to E_{\gamma(b)},
        \quad
        \xi_a \mapsto \xi_b \coloneqq \xi(b)
    \end{equation}
    は$\R$-線型同型である。
\end{proposition}

\begin{proof}
    初期値問題の解の存在と一意性より、写像であることはよい。
    全射性は$t = b$での$\xi$の値を指定した初期値問題を考えればよい。
    $\R$-スカラー倍を保つことは次のようにしてわかる:
    $\xi$が$\xi(a) = \xi_a$なる解であったとすると、
    各$c \in \R$に対し
    $\eta(t) \coloneqq c \xi(t)$は$\eta(a) = c \xi_a$をみたすただひとつの解であるから、
    $c \xi_a = \eta(a)$を曲線$\gamma$に沿って平行移動して得られる値は
    $\eta(b) = c \xi(b) = c \xi_b$に他ならない。
    和を保つことも同様にして示せる。
    よって命題の写像は全射$\R$-線型写像である。
    $\dim_\R E_{\gamma(a)} = \dim_\R E_{\gamma(b)}$より
    $\R$-線型同型であることが従う。
\end{proof}

\begin{definition}[ベクトル束の接続のホロノミー群]
    $x_0 \in M$とする。
    $x_0$を基点とする区分的に{\smooth}な任意の閉曲線$c$に対し、
    平行移動により$\R$-ベクトル空間$E_{x_0}$の自己同型写像
    ($\tau_c$とおく) が得られる。
    そこで
    \begin{equation}
        \Psi_{x_0} \coloneqq \{
            \tau_c \in GL(E_{x_0})
            \mid
            \text{$c$は$x_0$を基点とする区分的に{\smooth}な閉曲線}
        \}
    \end{equation}
    とおくと、
    $\Psi_{x_0}$は$GL(E_{x_0})$の部分群となる (このあと示す)。
    $\Psi_{x_0}$を$x_0$を基点とする
    \term{ホロノミー群}[holonomy group]{ホロノミー群}[ほろのみーぐん]という。
\end{definition}

\begin{proof}
    \uline{写像の合成について閉じていること} \quad
    $\tau_c, \tau_{c'} \in \Psi_{x_0}$とすると
    $c \circ c'$は$x_0$を基点とする区分的に{\smooth}な閉曲線であり、
    $\tau_c \circ \tau_{c'} = \tau_{c \circ c'}$が成り立つ。

    \uline{単位元を含むこと} \quad
    定値曲線$x_0$に対し$\tau_{x_0} \in \Psi_{x_0}$が恒等写像となる。

    \uline{逆元を含むこと} \quad
    $\tau_c \in \Psi_{x_0}$とする。
    $c$を逆向きに動く曲線$d$を
    $d(t) \coloneqq c(a + b - t) \; t \in [a, b]$で定め、
    $\xi$を逆向きに動く曲線$\eta$を
    $\eta(t) \coloneqq \xi(a + b - t) \; t \in [a, b]$で定める。
    このとき$d$は$x_0$を基点とする区分的に{\smooth}な閉曲線だから
    $\tau_d \in \Psi_{x_0}$である。
    また、$\eta$は$d$に沿う$E$の切断である。
    さらに$\eta$が$d$に沿って平行であることは、
    $\xi$の拡張を$\widetilde{\xi}$として
    (これは$\eta$の拡張でもある)
    \begin{alignat}{1}
        \nabla_{\dot{d}(t)} \eta
            &= \nabla_{\dot{d}(t)} \widetilde{\xi} \\
            &= \nabla_{- \dot{c}(a + b - t)} \widetilde{\xi} \\
            &= - \nabla_{\dot{c}(a + b - t)} \widetilde{\xi} \\
            &= - \nabla_{\dot{c}(a + b - t)} \xi \\
            &= 0
    \end{alignat}
    よりわかる。
    よって$\xi_b = \eta(a)$を$d$に沿って平行移動すると
    $\eta(b) = \xi(a) = \xi_a$が得られる。
    したがって$\eta_d = \eta_c^{-1}$である。
\end{proof}

%\begin{proposition}
%    $M$を多様体、$E \to M$をベクトル束、
%    $g$を$E$の内積、
%    $\nabla$を$g$を保つ$E$の接続とする。
%    このとき、内積は平行移動で不変である\TODO{どういう意味?}。
%\end{proposition}
%
%\begin{proof}
%    \TODO{}
%\end{proof}

\subsection{主ファイバー束の平行移動とホロノミー}

\begin{definition}[水平な曲線]
    $M$を多様体、
    $G$を Lie 群、
    $p \colon P \to M$を主$G$束、
    $\omega$を$P$の接続形式、
    $J \subset \R$を区間とする。
    {\smooth}曲線$u \colon J \to P$が
    \term{水平}[horizontal]{水平}[すいへい]であるとは、
    $u$の速度ベクトル$\dot{u}$がつねに水平部分空間に含まれること、すなわち
    \begin{equation}
        \omega(\dot{u}(t)) = 0
            \quad (\forall t \in J)
    \end{equation}
    が成り立つことをいう。
\end{definition}

\begin{definition}[平行移動]
    $M$を多様体、
    $G$を Lie 群、
    $p \colon P \to M$を主$G$束、
    $\omega$を$P$の接続形式、
    $J \subset \R$を区間、
    $x \colon J \to M$を$x_0 \in M$を始点とする{\smooth}曲線
    とする。
    このとき各$u_0 \in P_{x_0}$に対し、
    $u_0$を始点とする水平な{\smooth}曲線$u \colon J \to P$であって
    \begin{equation}
        \pi(u(t)) = x(t) \quad (t \in J)
    \end{equation}
    をみたすものが一意に存在する (証明略)。
    このとき、
    $u$は曲線$x$に沿った$u_0$の
    \term{平行移動}[parallel displacement]{平行移動}[へいこういどう]
    であるという。
    \begin{equation}
        \begin{tikzcd}
            & P \ar{d}{p} \\
            J \ar{ru}{u} \ar{r}[swap]{x}
                & M
        \end{tikzcd}
    \end{equation}
\end{definition}

\begin{proposition}
    $u$が水平ならば、任意の$s \in G$に対し
    $u(t) . s$も水平である。
\end{proposition}

\begin{proof}
    水平接分布が$G$の作用で保たれることより明らか。
\end{proof}

\begin{definition}[主ファイバー束の接続のホロノミー群]
    $u_0 \in P$とし、$x_0 = p(u_0)$とおく。
    $x_0$を始点とする$M$内の任意の閉曲線$c$に対し、
    $x$に沿った$u_0$の平行移動を$u$とおくと
    \begin{equation}
        u(b) = u_0 . \tau_c
    \end{equation}
    なる$\tau_c \in G$が一意に定まる。
    このような$\tau_c$全体の集合を$\Psi_{u_0}$とおくと、
    $\Psi_{u_0}$は$G$の部分群となる。
    $\Psi_{u_0}$を$u_0$を始点とする$\omega$の
    \term{ホロノミー群}[holonomy group]{ホロノミー群}[ほろのみーぐん]という。
\end{definition}

\begin{proposition}[ホロノミー群の共役]
    $u_0, u_1 \in P$とし、
    $x_0 = p(u_0), \; x_1 = p(u_1)$とおく。
    $c_0$を$x_0$から$x_1$への区分的に{\smooth}な曲線とし、
    曲線$c_0$に沿った$u_0$の平行移動を$\widetilde{c}_0$とおく。
    すると$\widetilde{c}_0(b) = u_1 . a$なる$a \in G$がただひとつ存在するが、
    このとき$\Psi_{u_1} = a \Psi_{u_0} a^{-1}$が成り立つ。
\end{proposition}

\begin{proof}
    $a \Psi_{u_0} a^{-1} \subset \Psi_{u_1}$および
    $a^{-1} \Psi_{u_1} a \subset \Psi_{u_0}$を示せばよい。
    実際、これらが示されたならば
    $a \Psi_{u_0} a^{-1} \subset \Psi_{u_1}
    = aa^{-1} \Psi_{u_1} aa^{-1} \subset a \Psi_{u_0} a^{-1}$
    より$a \Psi_{u_0} a^{-1} = \Psi_{u_1}$が従う。
    さらに$u_0, u_1$に関する対称性より
    $a \Psi_{u_0} a^{-1} \subset \Psi_{u_1}$を示せば十分。
    そこで$\tau_c \in \Psi_{u_1}$とし、
    $c$に沿う$u_0$の平行移動を$\widetilde{c}$とおき、
    $a \tau_c a^{-1} \in \Psi_{u_0}$を示す。
    そのためには$a \tau_c a^{-1} = \tau_{c_0 \circ c \circ c_0^{-1}}$であること、
    すなわち$c_0 \circ c \circ c_0^{-1}$に沿う
    $u_1$の平行移動の終点が$u_1 . a \tau_c a^{-1}$であることをいえばよい。

    まず$c_0^{-1}$に沿う$u_1$の平行移動は
    $R_{a^{-1}} \circ \widetilde{c}_0^{-1}$であり、
    その終点は$u_0 . a^{-1}$である。

    つぎに$c$に沿う$u_0 . a^{-1}$の平行移動は
    $R_{a^{-1}} \circ \widetilde{c}$であり、
    その終点は$u_0 . \tau_c a^{-1}$である。

    最後に$c_0$に沿う$u_0 . \tau_c a^{-1}$の平行移動は
    $R_{\tau_c a^{-1}} \circ \widetilde{c}_0$であり、
    その終点は$u_1 . a \tau_c a^{-1}$である。
    これが示したいことであった。
\end{proof}






% ============================================================
%
% ============================================================
\chapter{特性類}

特性類について述べる。
特性類はベクトル束の位相不変量である。

% ------------------------------------------------------------
%
% ------------------------------------------------------------
\section{複素ベクトル束}

\begin{definition}[Complex Vector Bundles]
    \TODO{}
\end{definition}

% ------------------------------------------------------------
%
% ------------------------------------------------------------
\section{Euler 類}

\TODO{}

% ------------------------------------------------------------
%
% ------------------------------------------------------------
\section{Chern 類}

\TODO{}




\end{document}

\part{計量と Riemann 多様体}
\documentclass[report]{jlreq}
\usepackage{global}
\usepackage{./local}
\subfiletrue
\begin{document}

% ============================================================
%
% ============================================================
\chapter{関数列と関数項級数}

% ------------------------------------------------------------
%
% ------------------------------------------------------------
\section{関数列}

関数列の収束について基礎的な事項を整理する。

\TODO{関数族の場合も含めてネットで定義する}

\begin{definition}[各点収束と一様収束、広義一様収束]
    $I \subset \R$、
    $f, f_n \; (n \in \N)$を$I$上の実数値関数とする。
    \begin{enumerate}
        \item 関数列$\{ f_n \}_{n \in \N}$が
            $f$に
            \term{$I$上各点収束}[converge pointwise on $I$]
                {各点収束}[かくてんしゅうそく]
            するとは、
            $\forall \eps > 0$と
            $\forall x \in I$に対し、
            $\exists N \in \R$が存在し、
            $\forall n \ge N$に対し
            \begin{equation}
                |f_n(x) - f(x)| < \eps
            \end{equation}
            が成り立つことをいう。
        \item 関数列$\{ f_n \}_{n \in \N}$が
            $f$に
            \term{$I$上一様収束}[converge uniformly on $I$]
                {一様収束}[いちようしゅうそく]
            するとは、
            $\forall \eps > 0$に対し、
            $\exists N \in \R$が存在し、
            $\forall x \in I$と
            $\forall n \ge N$に対し
            \begin{equation}
                |f_n(x) - f(x)| < \eps
            \end{equation}
            が成り立つことをいう。
        \item 関数列$\{ f_n \}_{n \in \N}$が
            $I$の任意のコンパクト部分集合上で一様収束するとき、
            関数列$\{ f_n \}_{n \in \N}$は
            $f$に
            \term{$I$上広義一様収束する}[converge compactly on $I$]
                {広義一様収束}[こうぎいちようしゅうそく]
            という。
    \end{enumerate}
\end{definition}

\begin{proposition}[一様 Cauchy 条件]
    $I \subset \R$、
    $f_n,\, f \colon I \to \R$とする。
    このとき、次の条件は同値である。
    \begin{enumerate}
        \item $f_n \to f \;\text{in}\; C(I) \;\text{as}\; n \to \infty$
        \item $\| f_n - f \| \to 0 \;\text{as}\; n \to \infty$
        \item $\forall \eps > 0$に対し、
            $\exists N \in \N$が存在し、
            $\forall n, m \ge N$に対し
            \begin{equation}
                \forall x \in I,\, |f_n(x) - f_m(x)| < \eps
            \end{equation}
            が成り立ち、
            $f_n(x)$の$n \to \infty$での各点収束の極限は$f(x)$である
    \end{enumerate}
\end{proposition}

\begin{proof}
    \underline{(1) \Leftrightarrow (2)}\, 一様収束の定義から明らか。

    \underline{(1) \Rightarrow (3)}\,
    $\forall \eps > 0$をとる。ある$N \in \N$が存在して、
    $\forall n \ge N$に対し
    \begin{equation}
        \forall x \in I,\, |f_n(x) - f(x)| \le \eps / 2
    \end{equation}
    なので、$\forall n, m \ge N$に対し
    \begin{equation}
        \forall x \in I,\, |f_n(x) - f_m(x)| \le |f_n(x) - f(x)| + |f_m(x) - f(x)| \le \eps
    \end{equation}
    である。

    \underline{(3) \Rightarrow (1)}\,
    $x$ごとに$\{ f_n(x) \}_{n \in \N}$は Cauchy 列なので、
    実数の完備性より確かに
    $\lim_{n \to \infty} f_n(x) \eqqcolon f(x) \cdots$ (1) が$\forall x \in I$に対し存在する。
    (3)より、$\forall \eps > 0$に対しある$N \in \N$が存在して、
    \begin{equation}
        \forall n, m \ge N,\, \forall x \in I,\, |f_n(x) - f_m(x)| < \eps / 2
    \end{equation}
    すなわち\footnote{
        混乱の恐れがなければ、以降の議論をすっ飛ばして単に
        「$m \to \infty$とすれば$\forall n \ge N, \forall x \in I, |f_n(x) - f(x)| < \eps$」
        と言ってしまう手もあります。
        次の\cref{1:prop:2}や第4回の\cref{4:lemma:1}でも
        これと似たような内容の論証を少しずつ異なる言い回しで試みているので、ぜひ見比べてみてください。
    }
    \begin{equation}
        \forall n \ge N,\, \forall x \in I,\, \forall m \ge N,\, |f_n(x) - f_m(x)| < \eps / 2
    \end{equation}
    である。よって、$\forall n \ge N,\, \forall x \in I$をとって、(1)から定まる$N'$ s.t.
    \begin{equation}
        \forall m \ge N',\, |f_m(x) - f(x)| < \eps / 2
    \end{equation}
    に対し$m \ge \max\{N, N'\}$をひとつ選べば
    \begin{equation}
        |f_n(x) - f(x)| \le |f_n(x) - f_m(x)| + |f_m(x) - f(x)| \le \eps
    \end{equation}
    である。
\end{proof}

\begin{proposition}
    $I$を任意の区間とする。
    $I$上の連続関数列$\{ f_n \}_{n \in \N}$が$f$に$I$上広義一様収束するならば、
    $f$も$I$上の連続関数である。
    \label{1:prop:2}
\end{proposition}

もちろん、$I$は有界や閉区間でなくてもかまいません。

\begin{proof}
    $I$がコンパクトでない場合は
    点$x \in I$を含む$I$のコンパクト部分集合をとりなおせばよいから、
    以下では$I$がコンパクトの場合のみ示す。

    $x' \in I$を固定し、$f$が$x'$で連続であることを示そう。
    広義一様収束の仮定から、$\forall \eps > 0$に対し$\exists N \in \N$\, s.t.
    \begin{equation}
        n \ge N \Rightarrow \| f_n - f \| < \eps / 3
    \end{equation}
    である。$f_N$は$x'$で連続だから、$x'$のある近傍$U$が存在して
    \begin{equation}
        x \in U \cap I \Rightarrow |f_N(x') - f_N(x)| < \eps / 3
    \end{equation}
    が成り立つ。よって$\forall x \in U$に対し
    \begin{equation}
        \begin{split}
            |f(x) - f(x')|
                &\le |f(x) - f_N(x)| + |f_N(x) - f_N(x')| + |f_N(x') - f(x')| \\
                &\le \| f - f_N \| + \eps / 3 + \| f_N - f \| \\
                &< \eps
        \end{split}
    \end{equation}
    である。したがって$f$は$x'$で連続である。
\end{proof}

\begin{theorem}[項別積分]
    $I \coloneqq [a, b]$上の連続関数列$\{ f_n \}_{n \in \N}$が
    $n \to \infty$のとき$f$に$I$上一様収束するならば
    \begin{equation}
        \lim_{n \to \infty} \int_a^b f_n(x)\, dx = \int_a^b f(x)\, dx
    \end{equation}
\end{theorem}

\begin{proof}
    一様性があるので積分を外から抑えられます。
\end{proof}

\begin{theorem}[項別微分]
    $I$を任意の区間とする。このとき
    \begin{enumerate}
        \item $\{ f_n \}_{n \in \N} \subset C^1(I)$が
            $n \to \infty$で$f$に各点収束
        \item $\{ f_n' \}_{n \in \N} \subset C(I)$が
            $n \to \infty$で$g$に$I$上広義一様収束
    \end{enumerate}
    ならば
    \begin{enumerate}
        \item $\{ f_n \}_{n \in \N} \subset C^1(I)$が
            $n \to \infty$で$f$に$I$上一様収束し$C^1$級
        \item $I$の各点で$g(x) = f'(x)$
    \end{enumerate}
\end{theorem}

\begin{proof}
    $I$が有界閉区間の場合を以下の流れに沿って示した後、
    一般の区間$I$に対しては各点$x$を含む有界閉区間がとれることを用いて示します。
    $x$ごとに微積分学の基本定理を使って一様収束を示します。
    $f$の微分可能性は積分の平均値定理を使って示します。
\end{proof}




% ------------------------------------------------------------
%
% ------------------------------------------------------------
\section{関数項級数}

関数項級数の収束について基礎的な事項を整理します。

\begin{definition}[関数項級数の収束]
    簡単なので省略
\end{definition}

\begin{proposition}[関数項級数の項別積分]
    $I \coloneqq [a, b]$上の連続関数列$\{ f_n \}_{n \in \N}$による関数項級数
    $\sum_{k \in \N} f_k(x)$が$I$上一様収束であるとき
    \begin{enumerate}
        \item $\sum_{k \in \N} f_k(x)$も連続関数
        \item $\sum_{k \in \N} \int_a^b f_k(x) dx = \int_a^b \sum_{k \in N} f_k(x) dx$
    \end{enumerate}
\end{proposition}

\begin{proof}
    関数列の場合と同様なので省略
\end{proof}

\begin{proposition}[関数項級数の項別微分]
    $I$を\textcolor{red}{任意の区間}とし、$\{ f_n \}_{n \in \N} \subset C^1(I)$とする。このとき
    \begin{enumerate}
        \item $\sum_{k \in \N} f_k(x)$が
            $n \to \infty$で各点収束
        \item $\sum_{k \in \N} f_k'(x)$が
            $n \to \infty$で$I$上広義一様収束
    \end{enumerate}
    ならば
    \begin{enumerate}
        \item $\sum_{k \in \N} f_k(x)$が
            $n \to \infty$で$I$上一様収束し$C^1$級
        \item $I$の各点で$\dd{x} \sum_{k \in \N} f_k(x) = \sum_{k \in \N} f_k'(x)$
    \end{enumerate}
\end{proposition}

\begin{proof}
    関数列の場合と同様なので省略
\end{proof}

\begin{theorem}[Weierstrass のMテスト]
    $a<b,\, I=[a,b]$とし、$f_n: I\to\R\,(\forall n \in \N)$とする。
    ある実数列$\{M_n\}_{n\in\N}$が存在して次を満たすと仮定する:
    \begin{enumerate}
        \item 十分大きな$\forall n\in\N$に対し$\| f_n \| \le M_n$
        \item 級数$\sum_{k\in\N} M_k$は収束する
    \end{enumerate}
    このとき、級数$\sum_{k\in\N} f_k$は$I$上一様収束する。
\end{theorem}

この定理は$f_n,\, f$が多変数関数の場合にも拡張できます。

\begin{proof}
    一様 Cauchy 条件に帰着させて示します。
\end{proof}

% ------------------------------------------------------------
%
% ------------------------------------------------------------
\newpage
\section{演習問題}

\begin{problem}[東大数理 2006A]
    ~
    \begin{enumerate}
        \item 正の整数$n$に対し、
            定積分$I_n = \int_0^{\pi / 2} \cos^n \theta \, d\theta$
            の値を求めよ。
        \item $\R$上の関数
            \begin{equation}
                f(x) = \int_0^{\pi / 2} \cos(x \cos \theta) \, d\theta
            \end{equation}
            を$x = 0$を中心として Taylor 展開し、
            その$n + 1$次以上の項を無視して得られる多項式を
            $p_n(x)$とする。$p_n(x)$を求めよ。
        \item $K$を$\R$の有界な部分集合とする。
            $n \to \infty$のとき
            $p_n(x)$は$f(x)$に
            $K$上一様収束することを示せ。
    \end{enumerate}
\end{problem}

\begin{proof}
    \uline{(1)} \quad
    $n \ge 2$のとき、
    $\cos^n \theta = \cos^{n - 2} \theta (1 - \sin^2 \theta)$
    に注意すると
    \begin{alignat}{1}
        I_n
            &= \underbrace{
                \int_0^{\pi / 2} \cos^{n - 2} \theta \, d\theta
            }_{= I_{n - 2}}
                + \int_0^{\pi / 2} \cos^{n - 2} \theta \sin^2 \theta \, d\theta
    \end{alignat}
    である。右辺第2項は部分積分により
    \begin{alignat}{1}
        \mybracket{
            - \frac{1}{n - 1} \cos^{n - 1} \theta \sin \theta
        }_0^{\pi / 2}
            + \frac{1}{n - 1}
            \int_0^{\pi / 2} \cos^{n - 1} \theta \cos \theta \, d\theta
            &= \frac{1}{n - 1} \int_0^{\pi / 2} \cos^n \theta \, d\theta \\
            &= \frac{1}{n - 1} I_n
    \end{alignat}
    と変形される。したがって
    $I_n = I_{n - 2} + \frac{1}{n - 1} I_n$となり、
    整理して
    $I_n = \frac{n - 1}{n} I_{n - 2}$を得る。
    具体的計算により$I_0 = \frac{\pi}{2}, \; I_1 = 1$だから、
    求める答えは
    \begin{equation}
        I_n = \begin{cases}
            \frac{n - 1}{n} \frac{n - 3}{n - 2} \cdots \frac{1}{2} \frac{\pi}{2}
                & (\text{$n$が偶数}) \\[1ex]
            \frac{n - 1}{n} \frac{n - 3}{n - 2} \cdots \frac{2}{3}
                & (\text{$n$が奇数})
        \end{cases}
    \end{equation}
    である。

    \uline{(2)} \quad
    \begin{equation}
        p_n(x) = \sum_{0 \le k \le n/2}
            \frac{(-1)^k}{k!} I_{2k} x^k
    \end{equation}
    \TODO{}

    \uline{(3)} \quad
    Taylor 展開の剰余項を
    $R_n(x) \coloneqq f(x) - p_n(x)$とおく。
    $R_n(x)$が$n \to \infty$で$K$上$0$に一様収束することをいえばよい。
    いま$K$は有界だから、
    ある$R > 0$が存在して、
    すべての$x \in K$に対して$|x| < R$が成り立つ。
    また(1)の結果より、
    すべての$k \in \Z_{\ge 0}$に対し
    $I_{2k} < \frac{\pi}{2}$が成り立つ。
    したがって
    \begin{alignat}{1}
        |R_n(x)|
            &\le \sum_{k > n/2} \frac{1}{k!} I_{2k} |x|^k \\
            &\le \frac{\pi}{2} \sum_{k > n/2} \frac{1}{k!} R^k \\
            &= \frac{\pi}{2} \myparen{
                e^R - \sum_{0 \le k \le n/2} \frac{1}{k!} R^k
            } \\
            &\to 0 \quad (\text{$n \to \infty$})
    \end{alignat}
    を得る。
    すなわち$R_n(x)$は$K$上$0$に一様収束する。
    これが示したいことであった。
\end{proof}



% ============================================================
%
% ============================================================
\chapter{Fourier 級数と Fourier 変換}

% ------------------------------------------------------------
%
% ------------------------------------------------------------
\section{熱方程式に対するフーリエの方法}

偏微分方程式
\begin{equation}
    \deldel[u]{t}(x, t) = \frac{k}{c} \frac{\del^2 u}{\del x^2}(x, t)
\end{equation}
を\textbf{熱方程式}と呼びます。以下、簡単のため$c = k = 1$とします。
これに
\begin{equation}
    \begin{split}
        &\text{境界条件}\quad u(0, t) = 0,\, u(1, t) = 0 \\
        &\text{初期条件}\quad u(x, 0) = a(x) \not\equiv 0 \quad \text{for}\, x \in [0, 1]
            \quad \left(\text{ただし } \int_0^1 a(x)^2 dx < \infty \right)
    \end{split}
\end{equation}
を付け加えた問題を考えてみます。
求解の方針は次の3ステップです。
\begin{enumerate}
    \item 解の形を変数分離形$u(x, t) = \phai(x) \eta(t)$に仮定し、
    \item 境界条件から$\phai_n(x)$と$\eta_n(t)$を順に求め、
    \item $\phai_n(x) \eta_n(t)$の無限個の重ね合わせをとり、初期条件をみたす係数を求める。
\end{enumerate}
(3)の無限和の収束性に一旦目をつぶれば、"解"は
\begin{equation}
    u(x, t) = \sum_{n = 1}^\infty c_n e^{-(n\pi)^2 t} \sin (n\pi x),\quad
    c_n = 2 \int_0^1 a(x) \sin(n\pi x) dx
    \label{eq:1:1}
\end{equation}
と求まります\footnote{
    $c_n$の式の先頭に現れる$2$は$\int_0^1 \sin^2(n\pi x) dx = \frac{1}{2}$に由来します。
}。
そして、実はこの級数はきちんと収束します。


\begin{proposition}
    フーリエの方法で求めた解(\ref{eq:1:1})は$[0, 1] \times (0, \infty)$上広義一様収束する。
\end{proposition}

\begin{proof}
    $[0, 1] \times [\tau, T],\, 0 < \tau < T$を任意にとる。
    Weierstrass の定理を用いて示す。
    充分大きな任意の$n$に対し
    \begin{equation}
        \begin{split}
            |c_n u_n(x, t)|
                &\le |c_n|\, e^{-(n\pi)^2 \tau} \sin (n\pi x) \\
                &\le |c_n|\, e^{-(n\pi)^2 \tau} \\
                &\le \frac{|c_n|}{(n\pi)^2 \tau} (n\pi)^2 \tau\, e^{-(n\pi)^2 \tau} \\[0.5em]
            \therefore\quad |c_n u_n(x, t)|
                &= O\left(\frac{|c_n|}{n^2}\right) \quad (n \to \infty)
        \end{split}
    \end{equation}
    であり、
    \begin{equation}
        \begin{split}
            \sum_{n = 1}^\infty \frac{|c_n|}{n^2}
                &\le \left(\sum_{n = 1}^\infty |c_n|^2\right)^{1/2}
                    \left(\sum_{n = 1}^\infty \frac{1}{n^4} \right)^{1/2} \\
                &= \left(2 \int_0^1 a(x)^2 dx\right)^{1/2}
                    \left(\sum_{n = 1}^\infty \frac{1}{n^4} \right)^{1/2} \\
                &< \infty
        \end{split}
    \end{equation}
    なので、(\ref{eq:1:1})の級数は$[0, 1] \times [\tau, T]$上一様収束、
    したがって$[0, 1] \times (0, \infty)$上広義一様収束する。
\end{proof}




% ------------------------------------------------------------
%
% ------------------------------------------------------------
\section{フーリエ級数展開}

周期$2\pi$の関数$f: \R \to \R$に対し、
同じく周期$2\pi$の関数からなる関数系
$\{ \textcolor{red}{1 \big/\! \sqrt{\mathstrut 2}},\, \cos nx,\, \sin nx \}\, (n = 1, 2, \dots)$
による展開\footnote{
    関数系$\{ 1 \big/\! \sqrt{\mathstrut 2},\, \cos nx,\, \sin nx \}$は
    内積$\langle f, g \rangle = \textcolor{red}{\frac{1}{\pi}} \int_{-\pi}^{\pi} f\, g\, dx$のもとで
    正規直交関数系となっています。
}
\begin{equation}
    \frac{1}{2} a_0 + \sum_{n=1}^\infty (a_n \cos(nx) + b_n \sin(nx))
\end{equation}
を$f$の\textbf{フーリエ級数展開}と呼び、
\begin{equation}
    S_N[f](x) := \frac{1}{2} a_0 + \sum_{n=1}^N (a_n \cos(nx) + b_n \sin(nx))
\end{equation}
を$f$の\textbf{第$N$フーリエ部分和}と呼びます。
フーリエ級数展開が$f$に一致するかどうかはまだわかりませんが、
一致すると仮定すれば、三角関数の直交性から\textbf{フーリエ係数}$a_n, b_n$は
\begin{equation}
    \begin{split}
        a_n &= \frac{1}{\pi} \int_{-\pi}^{\pi} f(x) \cos(nx) dx \\
        b_n &= \frac{1}{\pi} \int_{-\pi}^{\pi} f(x) \sin(nx) dx
    \end{split}
\end{equation}
と表せることがわかります\footnote{
    $f$が偶関数ならば$a_n = \frac{2}{\pi} \int_0^\pi,\, b_n = 0$に、
    奇関数ならば$a_n = 0,\, b_n = \frac{2}{\pi} \int_0^\pi$になります。
}。
そこで、フーリエ級数展開が$f$に一致するという仮定は一旦忘れて、
$a_n, b_n$を上のように定義して議論をスタートします。

ここからは、フーリエ級数展開が収束するかどうか、するとしたらどこに収束するか、ということを考えていきます。
まずはフーリエ級数展開が一様収束するための条件をいくつか確認しておきます。

\begin{theorem}
    $f$が$C^2$級の$2\pi$周期関数であるとき、$S_N[f]$は$\R$上一様収束する。
\end{theorem}

\begin{proof}
    フーリエ係数$a_n$に対して
    \begin{alignat}{1}
        a_n &= \frac{1}{\pi} \int_{-\pi}^{\pi} f(x) \cos(nx) dx \\
            &= - \frac{1}{\pi} \int_{-\pi}^{\pi} f'(x) \sin(nx) dx \quad (\text{\because\, 部分積分}) \\
            &= - \frac{1}{n^2\pi} \int_{-\pi}^{\pi} f''(x) \cos(nx) dx \quad (\text{\because\, 部分積分}) \\
            &= O\left(\frac{1}{n^2}\right) \quad (n \to \infty)
    \end{alignat}
    である。
    よって
    \begin{equation}
        |a_n \cos(nx) + b_n \sin(nx)| \le |a_n| + |b_n| = O\left(\frac{1}{n^2}\right) \quad (n \to \infty)
    \end{equation}
    なので、Weierstrassの定理により、$S_N[f]$は$\R$上一様収束する。
\end{proof}



\begin{theorem}
    $f$が$C^{\textcolor{red}{1}}$級の$2\pi$周期関数であるとき、$S_N[f]$は$\R$上一様収束する。
\end{theorem}

\begin{proof}
    $m > n > 0$なる自然数$m, n$を任意にとると
    \begin{alignat}{1}
        \sum_{k=n}^m |a_k|
            &= \sum_{k=n}^m \left| \frac{1}{\pi} \int_{-\pi}^{\pi} f(x) \cos(kx) dx \right| \quad (\text{\because\, 定義}) \\
            &= \frac{1}{\pi} \sum_{k=n}^m \left| \frac{1}{k} \int_{-\pi}^{\pi} f'(x) \cos(kx) dx \right| \quad (\text{\because\, 部分積分}) \\
            &\le \frac{1}{\pi}
                \left\{ \sum_{k=n}^m \frac{1}{k^2} \right\}^{1/2}
                \left\{ \sum_{k=n}^m \left( \int_{-\pi}^{\pi} f'(x) \cos(kx) dx \right)^2 \right\}^{1/2}
                \quad (\text{\because\, Schwartzの不等式}) \\
            &\le \frac{1}{\pi}
                \left\{ \sum_{k=n}^m \frac{1}{k^2} \right\}^{1/2}
                \cdot \left\{ \frac{1}{\pi} \int_{-\pi}^{\pi} f'(x)^2 dx \right\}^{1/2}
                \quad (\text{\because\, Parsevalの等式}) \\
            &< \infty
    \end{alignat}
    なので、Cauchy の収束条件により級数$\sum |a_n|$は収束する。
    同様にして$\sum |b_n|$も収束する。
    したがって
    \begin{equation}
        |a_n \cos(nx) + b_n \sin(nx)| \le |a_n| + |b_n|
    \end{equation}
    の右辺は収束するから、Weierstrassの定理により、$S_N[f]$は$\R$上一様収束する。
\end{proof}

さて、フーリエ級数展開が一様収束するための条件はいくつか確認できたので、
次は肝心の「どこに収束するか?」を考えていきます。
そのためには各点収束の極限を考えればよいのですが、その前にいくつかの補題を準備しておきます。

\begin{lemma}
    フーリエの部分和は
    \begin{equation}
        S_N[f](x) = \frac{1}{2\pi}
            \int_{-\pi}^{\pi} f(x+y) \frac{\sin\left(N+\frac{1}{2}\right)y}{\sin\frac{1}{2}y} dy
    \end{equation}
    と書ける。
    \label{3:lemma1}
\end{lemma}

\begin{proof}
    三角数列の和の公式
    \begin{equation}
        \cos\alpha + \cos 2\alpha + \cdots + \cos n\alpha
            = \frac{\cos\left(\frac{n+1}{2}\alpha\right) \sin\left(\frac{n}{2}\alpha\right)}{\sin\frac{\alpha}{2}}
    \end{equation}
    と、$f$の周期性を利用した置換を用いて示します。
\end{proof}

    \begin{lemma}
        区間$[-\pi, \pi]$上で区分的に連続な関数$g$に対して次が成り立つ:
        \begin{equation}
            \lim_{n\to\infty} \frac{1}{\pi} \int_{-\pi}^{\pi} g(x) \sin\left(N + \frac{1}{2}\right) x dx = 0
        \end{equation}
        \label{3:lemma2}
    \end{lemma}

\begin{proof}
    リーマン・ルベーグの定理\footnote{
        リーマン・ルベーグの定理の主張は以下のとおりです。証明は参考文献\cite[第3章 例題3.3]{杉浦+89}を参照。
        有界閉区間$I=[a,b]$上で関数$f$が可積分であるとき、
        $\lim_{t\to\infty} \int_a^b f(x) \sin(tx) dx = 0$
        および
        $\lim_{t\to\infty} \int_a^b f(x) \cos(tx) dx = 0$
        が成り立つ。
    }
    より明らか。
\end{proof}

\begin{lemma}
    $f$を$\R$上の区分的$C^1$級関数とし、
    $x \in \R$を任意にとる。このとき
    \begin{gather}
        y \mapsto \dfrac{f(x + y) - f(x - 0)}{\sin(y/2)}
            \;\text{は}\; -\pi \le y \le 0 \text{ で、} \label{eq:lem3:a} \\[+1em]
        y \mapsto \dfrac{f(x + y) - f(x + 0)}{\sin(y/2)}
            \;\text{は}\; 0 \le y \le \pi \text{ で、} \label{eq:lem3:b}
    \end{gather}
    それぞれ区分的に連続である。
    \label{3:lemma3}
\end{lemma}

\begin{proof}
    $x \in \R$を任意に固定する。$x$は$f$の不連続点であってもよい。
    $y \neq 0$のときは(\ref{eq:lem3:a}), (\ref{eq:lem3:b})の分母は$0$でないから、
    $f$の区分的連続性により定理の主張が成り立つ。
    したがって$y \to 0$のときを考えればよく、
    以下$y \to -0$の場合を示す。$y \to +0$の場合も同様にして示せる。
    \begin{equation}
        \alpha(y) \coloneqq \dfrac{f(x + y) - f(x - 0)}{\sin(y/2)}
    \end{equation}
    とおくと
    \begin{equation}
        \begin{split}
            \alpha(y)
                &= \frac{f(x + y) - \lim_{\eps \to +0} f(x - \eps)}{\sin(y/2)} \\
                &= 2\, \frac{f(x + y) - \lim_{\eps \to +0} f(x - \eps)}{y} \frac{y/2}{\sin(y/2)} \\
                &= 2 \lim_{\eps \to +0} \frac{f(x - \eps + y) - f(x - \eps)}{y} \frac{y/2}{\sin(y/2)}
        \end{split}
    \end{equation}
    である。ただし、最後の式変形では$|y|$が充分小さいとき$f$が点$x + y$で連続であることを用いた。
    $\lim_{\eps \to +0}$の部分を$\eps$-$\delta$論法で書き直すと、
    $\forall \eta > 0$に対し$\exists \eps_\eta > 0\;$ s.t. $\; 0 < \forall \eps < \eps_\eta$に対し
    \begin{equation}
        \alpha(y) - \eta
            < 2\, \frac{f(x - \eps + y) - f(x - \eps)}{y} \frac{y/2}{\sin(y/2)}
            < \alpha(y) + \eta
            \label{eq:lem3:2}
    \end{equation}
    である。
    $\eps$を充分小さくとれば$f$は点$x - \eps$で微分可能、とくに左微分可能なので、
    式(\ref{eq:lem3:2})の第~2辺には$y \to -0$の極限が存在するが、
    それはとくに下極限と一致するから、
    式(\ref{eq:lem3:2})の各辺の$y \to -0$の下極限をとって
    \begin{equation}
        \liminf_{y \to -0} \alpha(y) - \eta \le 2 f'(x - \eps) \le \liminf_{y \to -0} \alpha(y) + \eta
    \end{equation}
    を得る。
    ふたたび$\eps$-$\delta$論法による極限の定義を思い出せば
    \begin{equation}
        2 f'(x - 0) = \liminf_{y \to -0} \alpha(y)
    \end{equation}
    を得る。
    同様の議論により
    \begin{equation}
        2 f'(x - 0) = \limsup_{y \to -0} \alpha(y)
    \end{equation}
    も示せる。$f$は区分的$C^1$級なので$f'(x - 0)$が存在し、したがって
    \begin{equation}
        \lim_{y \to -0} \alpha(y)
            = \liminf_{y \to -0} \alpha(y)
            = \limsup_{y \to -0} \alpha(y)
            = 2 f'(x - 0)
            \in \R
    \end{equation}
    である。
\end{proof}

\begin{theorem}
    $f$が区分的に$C^1$級の$2\pi$周期関数であるとき、任意の$x \in [-\pi, \pi]$に対して
    \begin{equation}
        S_N[f](x) \to \frac{f(x+0) - f(x-0)}{2} \quad (N \to \infty)
    \end{equation}
    が成り立つ。
\end{theorem}

この定理は、$f$のフーリエ級数展開が
\begin{itemize}
    \item $x$が連続点のときは$f(x)$に
    \item $x$が不連続点のときはそこでの "跳躍" の中央に
\end{itemize}
各点で収束するということを主張しています。

\begin{proof}
    \cref{3:lemma1}より、
    \begin{equation}
        S_N[f](x) = \frac{1}{2\pi}
            \int_{-\pi}^{\pi} f(y+x)\frac{\sin\left(N+\frac{1}{2}\right)y}{\sin\frac{1}{2}y} dy
    \end{equation}
    である。さらに
    \begin{equation}
        \int_{-\pi}^{0} \frac{\sin\left(N+\frac{1}{2}\right)y}{\sin\frac{1}{2}y} dy
        = \int_{0}^{\pi} \frac{\sin\left(N+\frac{1}{2}\right)y}{\sin\frac{1}{2}y} dy
        = \pi
    \end{equation}
    である。よって
    \begin{alignat}{2}
        &&&S_N[f](x) - \frac{f(x-0) + f(x+0)}{2} \\
        &=&&\frac{1}{2\pi} \int_{-\pi}^0 f(y+x) \frac{\sin\left(N+\frac{1}{2}\right)y}{\sin\frac{1}{2}y} dy
            -\frac{f(x-0)}{2} \nonumber \\
        &&+\,&\frac{1}{2\pi} \int_0^\pi f(y+x) \frac{\sin\left(N+\frac{1}{2}\right)y}{\sin\frac{1}{2}y} dy
            -\frac{f(x+0)}{2} \\
        &=&&\frac{1}{2\pi} \int_{-\pi}^0 f(y+x) \frac{\sin\left(N+\frac{1}{2}\right)y}{\sin\frac{1}{2}y} dy
            -\frac{1}{2\pi} \int_{-\pi}^0 f(x-0) \frac{\sin\left(N+\frac{1}{2}\right)y}{\sin\frac{1}{2}y} dy \nonumber \\
        &&+\,&\frac{1}{2\pi} \int_0^\pi f(y+x) \frac{\sin\left(N+\frac{1}{2}\right)y}{\sin\frac{1}{2}y} dy
            -\frac{1}{2\pi} \int_0^\pi f(x+0) \frac{\sin\left(N+\frac{1}{2}\right)y}{\sin\frac{1}{2}y} dy \\
        &=&&\frac{1}{2\pi}
            \int_{-\pi}^0 \frac{f(y+x) - f(x-0)}{\sin\frac{1}{2}y} \sin\left(N+\frac{1}{2}\right)y dy \nonumber \\
        &&+\,&\frac{1}{2\pi}
            \int_0^\pi \frac{f(y+x) - f(x+0)}{\sin\frac{1}{2}y} \sin\left(N+\frac{1}{2}\right)y dy \label{eq:thm3:1}
    \end{alignat}
    である。ここで
    \begin{equation}
        g(y) = \begin{cases}
            \displaystyle \frac{f(y+x) - f(x-0)}{\sin\frac{1}{2}y} & (-\pi \le y \le 0) \\[+1em]
            \displaystyle \frac{f(y+x) - f(x+0)}{\sin\frac{1}{2}y} & (0 < y \le \pi)
        \end{cases}
    \end{equation}
    とおけば、\cref{3:lemma3}により$g$は区間$[-\pi, \pi]$で区分的に連続である。
    したがって、\cref{3:lemma2}により
    \begin{equation}
        \text{(式(\ref{eq:thm3:1})) }
            = \frac{1}{2\pi} \int_{-\pi}^\pi g(y) \sin\left(N+\frac{1}{2}\right)y dy
            \to 0 \quad (N\to\infty)
    \end{equation}
    である。
    すなわち
    \begin{equation}
        S_N[f](x) \to \frac{f(x+0) - f(x-0)}{2} \quad (N \to \infty)
    \end{equation}
    がいえた。
\end{proof}

\begin{theorem}
    $f$が区分的な周期関数で$\int_{-\pi}^{\pi} f(x)^2 dx < \infty\,(f \in L^2(-\pi, \pi))$ならば
    \begin{equation}
        \int_{-\pi}^{\pi} (f(x) - S_N[f])^2 dx \to 0 \quad (N \to \infty)
    \end{equation}
    が成り立つ。
\end{theorem}

\begin{proof}
    難しいので省略\footnote{
        参考文献\cite[定理8.2.1]{吉田21}を参照
    }
\end{proof}




% ------------------------------------------------------------
%
% ------------------------------------------------------------
\section{複素フーリエ級数展開}
周期$2\pi l,\, l > 0$の複素数値関数$f: \R \to \C$に対し、
\begin{equation}
    \sum_{n = -\infty}^\infty c_n e^{i \frac{n}{l} x}
\end{equation}
を$f$の\textbf{複素フーリエ級数展開}と呼びます。
フーリエ係数$c_n$は、指数関数の直交性から
\begin{equation}
    c_n = \frac{1}{2\pi l} \int_{-l\pi}^{l\pi} f(x) e^{-i\frac{n}{l}x} dx
\end{equation}
と表せることがわかります。











\begin{problem}
    関数列
    \begin{equation}
        f_n(x) = \frac{e^{nx} - e^{-nx}}{e^{nx} + e^{-nx}}
    \end{equation}
    の極限$(n \to \infty)$を求めよ。

    解答:
    \begin{equation}
        f_n(x) \to \left\{\begin{alignedat}{2}
            \,&1 \quad &(x > 0) \\
            &0 \quad &(x = 0) \\
            &-1 \quad &(x < 0)
        \end{alignedat}
        \right.
    \end{equation}
\end{problem}

\begin{problem}
    $I = [0, 1],\, f_n(x) = x^n\, (n \in \N)$とおく。
    関数列$\{f_n\}_n$は各点収束するが一様収束しないことを示せ。
\end{problem}

\begin{problem}
    $I = [0, 2]$、
    \begin{equation}
        f_n(x) \coloneqq \begin{cases}
            n^2 x &(x \in [0, 1/n]) \\
            -n^2 (x - 2/n) &(x \in [1/n, 2/n]) \\
            0 &(\text{otherwise})
        \end{cases}
    \end{equation}
    とおく。
    関数列$\{f_n\}_n$は各点収束するが一様収束しないことを示せ。
    また、項別積分が一致しないことを確かめよ。
\end{problem}

\begin{problem}
    $I = [0, \infty),\, f_n(x) \coloneqq \frac{x}{nx + 1}$
    とおく。
    関数列$\{f_n\}_n$は$I$上一様収束することを示せ。
\end{problem}

\begin{problem}
    $I = [0, 1]$上の関数$f_n(x) \coloneqq \frac{x}{nx + 1}$は
    $n \to \infty$で$I$上一様収束することを示せ。
    また、項別積分が一致することを確かめよ。
\end{problem}

\begin{problem}
    $I = [-R, R]\, (R > 0)$上の関数項級数
    \begin{equation}
        \sum_{k \in \N} \frac{1}{x^2 + k^2}
    \end{equation}
    が一様収束することを示せ。
\end{problem}

\begin{problem}
    $I = [0, 2]$上の関数項級数
    \begin{equation}
        \sum_{k \in \N} \frac{1}{x^2 + k^2}
    \end{equation}
    が一様収束することを示せ。
\end{problem}

\begin{problem}
    \,
    \begin{itemize}
        \item \cite[第III章 例題6.1.1]{杉浦+89}
        \item \cite[第III章 問6.1.2 (2)]{杉浦+89}
        \item \cite[第III章 例題6.2.1]{杉浦+89}
        \item \cite[第III章 問6.2.1 (1),(2)]{杉浦+89}
    \end{itemize}
    を読者の演習問題とする。
\end{problem}

\begin{problem}
    次の初期値・境界値問題の解を求めよ。
    \begin{equation}
        \begin{split}
            &u_t(x, t) - u_{xx}(x, t) = 0 \quad \text{for $(x, t) \in (0, 1) \times (0, \infty)$} \\
            &u(0, t) = u(1, t) = 0 \\
            &u(x, 0) = 2 \sin(3\pi x) + 5 \sin(8\pi x) \quad \text{for $x \in [0, 1]$}
        \end{split}
    \end{equation}

    解答:
    \begin{equation}
        u(x, t) = 2 e^{-(3\pi)^2 t} \sin(3\pi x) + 5 e^{-(8\pi)^2 t} \sin(8\pi x)
    \end{equation}
\end{problem}

\begin{problem}
    次の初期値・境界値問題の解を求めよ。
    \begin{equation}
        \begin{split}
            &u_t(x, t) - u_{xx}(x, t) = 0 \quad \text{for $(x, t) \in (0, L) \times (0, \infty)$} \\
            &u(0, t) = u(L, t) = 0 \\
            &u(x, 0) = a(x) \quad \text{for $x \in [0, L]$}
        \end{split}
    \end{equation}
    ただし$\int_0^L a(x)^2 dx < \infty$とする。

    解答:
    \begin{equation}
        u(x, t) = \sum_{n=1}^\infty c_n \exp(-\lambda_n^2 t)\, \sin \frac{n\pi x}{L},\quad
        \lambda = \frac{n\pi}{L},\quad
        c_n = \frac{2}{L} \int_0^L a(x) \sin\frac{n\pi x}{L} dx
    \end{equation}
\end{problem}

\begin{problem}
    次の初期値・境界値問題の解を求めよ。
    \begin{equation}
        \begin{split}
            &u_t(x, t) - \textcolor{red}{u(x, t)} - u_{xx}(x, t) = 0 \quad \text{for $(x, t) \in (0, 10) \times (0, \infty)$} \\
            &u(0, t) = u(10, t) = 0 \\
            &u(x, 0) = 3 \sin(2\pi x) - 7 \sin(4\pi x) \quad \text{for $x \in [0, 10]$}
        \end{split}
    \end{equation}

    解答:
    \begin{equation}
        u(x, t) = e^t \left( 3 e^{-(2\pi)^2 t} \sin(2\pi x) - 7 e^{-(4\pi)^2 t} \sin(4\pi x) \right)
    \end{equation}
\end{problem}

\begin{problem}
    関数$x \mapsto \pi - |x| \quad(x \in [-\pi, \pi])$を$\R$全体に周期拡張した関数を$f$とおく。
    $f$のフーリエ級数展開を求めよ。

    解答:
    \begin{equation}
        f(x) \sim
            \frac{\pi}{2} + \frac{4}{\pi} \sum_{n=1}^\infty \frac{\cos((2n-1)x)}{(2n-1)^2}
    \end{equation}
\end{problem}

\begin{problem}
    関数
    \begin{equation}
        x \mapsto \begin{cases}
            1 &(0 < x \le \pi) \\
            0 &(x = 0) \\
            -1 &(-\pi < x < 0)
        \end{cases}
    \end{equation}
    を$\R$全体に周期拡張した関数を$f$とおく。
    $f$のフーリエ級数展開を求めよ。

    解答:
    \begin{equation}
        f(x) \sim
            \frac{\pi}{4} \sum_{n=1}^\infty \frac{\sin((2n-1)x)}{2n-1}
    \end{equation}
\end{problem}

\begin{problem}
    $0 \le x \le \pi$に対し$f(x) = -x (x - \pi)$とおく。
    $f$の周期$2\pi$の奇関数拡張、偶関数拡張のフーリエ級数展開を求めよ。

    解答:\\
    奇関数拡張:$b_n = \frac{4}{n^3 \pi} (1 - (-1)^n)$\\[0.5em]
    偶関数拡張:$\frac{1}{2} a_0 = \frac{1}{6} \pi^2,\, a_n = - \frac{2}{n^2} (1 + (-1)^n)$
\end{problem}

\begin{problem}
    \,
    \begin{itemize}
        \item \cite[第III章 例題3.3]{杉浦+89}
        \item \cite[第III章 問11.1 (1)-(4)]{杉浦+89}
    \end{itemize}
    を読者の演習問題とする。
\end{problem}


% ------------------------------------------------------------
%
% ------------------------------------------------------------
\section{パラメータを含む積分}

ここからはパラメータを含む積分の一様収束性など基礎的な事項を整理します。

以下では\mbox{2変数}関数$f(x, s)$が登場しますが、$x$の方がパラメータで、$s$は主変数です。
$x$の変域$\Omega$はコンパクトの場合のみを考えます。
一方で$s$の変域$I$は、まずコンパクトの場合を考えてから非有界区間の場合に拡張しますが、
このときに積分の一様収束性が仮定に加わることになります。

\begin{theorem}[$I$がコンパクトの場合]
    $\R$上の有界閉区間$\Omega \coloneq [\alpha, \beta],\, I \coloneq [a, b]$をとる。
    関数$f(x, s)$が$\Omega \times I$上連続ならば次が成り立つ。
    \begin{enumerate}
        \vspace{1em}
        \item $x$の関数$\displaystyle \int_a^b f(x, s)\, ds$は$\Omega$上連続である。
        \item $\displaystyle \int_\alpha^\beta \left( \int_a^b f(x, s)\, ds \right) dx \
            = \int_a^b \left( \int_\alpha^\beta f(x, s)\, dx \right) ds$
        \vspace{1em}
    \end{enumerate}
    さらに$\dfrac{\partial f}{\partial x}(x, s)$が存在して$\Omega \times I$上連続ならば次も成り立つ。
    \begin{enumerate}
        \setcounter{enumi}{2}
        \vspace{1em}
        \item $\displaystyle \dd{x} \int_a^b f(x, s)\, ds = \int_a^b \deldel[f]{x}(x, s)\, ds$
            と書けて$\Omega$上連続
    \end{enumerate}
    \label{4:thm:1}
\end{theorem}

\begin{proof}
    (1)
    $f$はコンパクト集合$\Omega \times I$上連続なので、
    $\Omega \times I$上一様連続でもある。
    したがって$\eps > 0$を任意にとると、$\exists \delta > 0$\, s.t.
    \begin{equation}
        | (x, s) - (y, t) | < \delta
        \quad \Rightarrow \quad
        |f(x, s) - f(y, t)| < \eps / (b - a)
    \end{equation}
    とくに
    \begin{equation}
        |x - y| < \delta
        \quad \Rightarrow \quad
        |f(x, s) - f(y, s)| < \eps / (b - a) \quad (\forall s \in I)
    \end{equation}
    すなわち
    \begin{equation}
        |x - y| < \delta
        \quad \Rightarrow \quad
        \| f(x,\, \cdot) - f(y,\, \cdot) \| \le \eps / (b - a)
    \end{equation}
    である。そこで$|x - y| < \delta$のとき
    \begin{equation}
        \begin{split}
            \left| \int_a^b f(x, s)\, ds - \int_a^b f(y, s)\, ds\right|
                &\le \int_a^b | f(x, s) - f(y, s) |\, ds \\
                &\le \| f(x,\, \cdot) - f(y,\, \cdot) \|\, (b - a) \\
                &\le \eps
        \end{split}
    \end{equation}
    である。
    したがって、$\int_a^b f(x, s)\, ds$は$\Omega$上(一様)連続である。

    (2) 長いので省略\footnote{\cite[第IV章 \S{7}]{杉浦80}}

    (3)
    $\deldel[f]{x}(x, s)$が連続ならば
    (1) より$\int_a^b \deldel[f]{x}(x, s) ds$も連続である。
    よって
    \begin{equation}
        \begin{split}
            \int_\alpha^x \left\{ \int_a^b \deldel[f]{x}(\xi, s)\, ds \right\} d\xi
                &= \int_a^b \left\{ \int_\alpha^x \deldel[f]{x}(\xi, s)\, d\xi \right\} ds \\
                &= \int_a^b (f(x, s) - f(\alpha, s)) ds \\
                &= \int_a^b f(x, s) ds - \underbrace{\int_a^b f(\alpha, s) ds}_{定数}
        \end{split}
    \end{equation}
    この両辺を$x$で微分して定理の式を得る。
\end{proof}

    \begin{theorem}[$I$が非有界区間の場合]
        $\R$上の有界閉区間$\Omega \coloneq [\alpha, \beta]$と
        非有界区間$I \coloneq [a, \infty),\, a \in \R$をとる。
        関数$f(x, s)$が$\Omega \times I$上連続で、
        \textcolor{red}{広義積分$\displaystyle \int_a^\infty f(x, s)\, ds$
        が$\Omega$上広義一様収束する}
        ならば次が成り立つ。
        \begin{enumerate}
            \vspace{1em}
            \item $x$の関数$\displaystyle \int_a^\infty f(x, s)\, ds$は$\Omega$上連続である。
            \item $\displaystyle \int_\alpha^\beta \left( \int_a^\infty f(x, s)\, ds \right) dx \
                = \int_a^\infty \left( \int_\alpha^\beta f(x, s)\, dx \right) ds$
            \vspace{1em}
        \end{enumerate}
        さらに$\dfrac{\partial f}{\partial x}(x, s)$が存在して$\Omega \times I$上連続で、
        \textcolor{red}{$\displaystyle \int_a^\infty \deldel[f]{x}(x, s)\, ds$
        が$\Omega$上広義一様収束する}
        ならば次も成り立つ。
        \begin{enumerate}
            \setcounter{enumi}{2}
            \vspace{1em}
            \item $\displaystyle \dd{x} \int_a^\infty f(x, s)\, ds = \int_a^\infty \deldel[f]{x}(x, s)\, ds$
                と書けて$\Omega$上連続
        \end{enumerate}
    \end{theorem}

実際は(3)だけを言いたいのであれば$\displaystyle \int_a^\infty f(x, s)\, ds$は各点収束でも問題ないのですが、
ここではこのまま進めます。

\begin{proof}
    (1), (2) \cref{4:thm:1}と、連続関数の広義一様収束極限も連続関数であるという定理を用いれば示せます。

    (3) \cref{4:thm:1}と同様の論法で示せます。
\end{proof}

さて、広義積分の一様収束性を判定する定理として、
第2回で登場した関数項級数に関する Weierstrass のMテストの類似が成り立ちます。
補題をひとつ提示してから定理を示します。

    \begin{lemma}[連続関数全体の空間の完備性]
        $\Omega \subset \R^n$とする。$\Omega$上の連続関数全体の集合に
        $\sup$ノルムから誘導される距離を入れた空間$C(\Omega)$は完備である。
        \label{4:lemma:1}
    \end{lemma}

\begin{proof}
    $C(\Omega)$の完備性を示すには、$C(\Omega)$の任意の Cauchy 列が
    $C(\Omega)$に極限を持つことをいえばよい。
    そこで、$\Omega$上の連続関数列$\{ F_n \}_{n \in \N}$であって
    \begin{equation}
        \lim_{n, m \to \infty} \| F_n - F_m \| = 0
        \label{4:eq:1}
    \end{equation}
    であるものを任意にとる。
    すると、$x \in \Omega$ごとに$\{ F_n \}_{n \in \N}$は Cauchy 列なので、
    $\R^n$の完備性より$\lim_{n \to \infty} F_n(x) \eqqcolon F(x) \cdots$ (1) が
    $\forall x \in \Omega$に対し存在する。
    よって、あとは$F \in C(\Omega)$を示せば定理がいえる。
    そこで$x' \in \Omega$を固定し、$x'$での$F$の連続性を示そう。

    (\ref{4:eq:1})より、$\forall \eps > 0$に対しある$N \in \N$が存在して
    \begin{equation}
        \forall n, m \ge N,\, \forall x \in \Omega,\, |F_n(x) - F_m(x)| < \eps / 3
    \end{equation}
    なので、$m \to \infty$として
    \begin{equation}
        \forall n \ge N,\, \forall x \in \Omega,\, |F_n(x) - F(x)| \le \eps / 3
    \end{equation}
    である。$F_n$の連続性より、$x'$の近傍$U$が存在して
    \begin{equation}
        x \in U \in \Omega \Rightarrow |F_n(x) - F_n(x')| < \eps / 3
    \end{equation}
    なので、$\forall x \in U$に対し
    \begin{equation}
        |F(x) - F(x')|
            \le |F(x) - F_n(x)| + |F_n(x) - F_n(x')| + |F_n(x') - F(x')| < \eps
    \end{equation}
    である。したがって$F$は$x'$で連続である。
\end{proof}

\begin{theorem}[広義積分に関する Weierstrass のMテスト]
    $\R$上の有界閉区間$\Omega \coloneq [\alpha, \beta]$と
    非有界区間$I \coloneq [a, \infty),\, a \in \R$をとる。
    与えられた連続関数$f \colon \Omega \times I \to \R$に対し、
    ある関数$\varphi: I \to \R$が存在して次を満たすと仮定する:
    \begin{enumerate}
        \item 十分大きな$\forall s \in I$に対し$\| f(\,\cdot\, , s) \| \le \varphi(s)$
        \item $\int_a^\infty \varphi(s)\, ds$が広義可積分
    \end{enumerate}
    このとき、広義積分$\int_a^\infty f(x, s)\, ds$は
    $\Omega$上一様収束する。
\end{theorem}

\begin{proof}
    $\int_a^\infty f(x, s)\, ds$に関する一様 Cauchy 条件の成立を、
    $\int_a^\infty \varphi(s)\, ds$に関する Cauchy 条件を用いて示します。
    極限関数の存在は、$\Omega$上の連続関数全体の空間が$\sup$ノルムに関して完備であることを用いて示します。
\end{proof}





% ------------------------------------------------------------
%
% ------------------------------------------------------------
\section{フーリエ変換}

フーリエ級数展開は直交関数系$\{ e^{i\frac{n}{l} x} \}_{n \in \Z}$による展開でしたが、
周期を持つとは限らない関数でも展開できるようにするため、関数系を非可算に拡張することを考えます。
天下り的に定義を述べると、$\R$上の複素数値広義可積分関数$f: \R \to \C$に対し、
\begin{equation}
    \calF [f] (\xi) := \frac{1}{\sqrt{2\pi}} \int_\R f(x)\, e^{-i\xi x} dx
    \label{eq:4:1:1}
\end{equation}
を$f$の\textbf{フーリエ変換}と呼び、
\begin{equation}
    \calF^{-1} [f] (x) := \frac{1}{\sqrt{2\pi}} \int_\R f(\xi)\, e^{i\xi x} d\xi
    \label{eq:4:1:2}
\end{equation}
を$f$の\textbf{逆フーリエ変換}と呼びます\footnote{
    係数の$1/\sqrt{2\pi}$は複素指数関数の周期$2\pi$に由来しており、
    この係数が出てこないような関数系を用いる流儀も存在します。
    なお、フーリエ変換も逆フーリエ変換も$\int_\R$が付いているのでどっちがどっちだかややこしいですが、
    フーリエ変換は "展開係数" であり、
    逆フーリエ変換は "重ね合わせ" にあたります。
}。
ここで、ある良い性質を持った$f$に対しては
\begin{equation}
    f(x) = \frac{1}{\sqrt{2\pi}} \int_\R \calF [f](\xi)\, e^{i\xi x} d\xi
\end{equation}
が成り立つことがわかっています。
すなわち、非可算な関数系$\{ e^{i\xi x} \}_{\xi \in \R}$を用いて、
$\{\calF [f](\xi)\}_{\xi \in \R}$を展開係数とした$f$の展開が得られるということです。

% ------------------------------------------------------------
%
% ------------------------------------------------------------
\section{急減少関数空間}

\begin{definition}
    関数$f: \R \to \C$が\textbf{急減少関数}であるとは、$f$が次をみたすことをいう\footnotemark :
    \begin{enumerate}
        \item $f \in C^{\infty}(\R)$
        \item 任意の$m, n \in \N$に対し
        \begin{equation}
            |f|_{m,n} := \sup_{x \in \R} |x|^m |f^{(n)} (x)| < \infty
        \end{equation}
    \end{enumerate}
    急減少関数全体の集合を$\calS = \calS(\R)$と書く。
\end{definition}

\footnotetext{
    「微分して\quad $x$たちを\quad 掛けたとて\quad その絶対値\quad 限りありけり」 — 詠み人知らず
}

    \begin{theorem}
        $f \in \calS(\R)$に対し$\calF [f],\, \calF^{-1} [f] \in \calS(\R)$
    \end{theorem}

\begin{proof}
    ややこしいので省略\footnote{\cite[第VII章 定理6.7]{杉浦85}}
\end{proof}

    \begin{theorem}[反転公式]
        $f \in \calS(\R)$とする。このとき、任意の$x \in \R$に対し\textbf{反転公式}
        \begin{equation}
            f(x) = \calF^{-1} \calF [f](x)
        \end{equation}
        が成り立つ。
    \end{theorem}

\begin{proof}
    ややこしいので省略\footnote{\cite[第VII章 定理6.7]{杉浦85}}
\end{proof}



% ------------------------------------------------------------
%
% ------------------------------------------------------------
\section{フーリエ変換の性質}

まず次の写像を準備しておきます:
\begin{itemize}
    \item 平行移動$\tau_h: x \mapsto x - h$
    \item 拡大・縮小$d_t: x \mapsto tx \quad (t > 0)$
    \item 反転$f^{\vee}(x) = f(-x)$
\end{itemize}

\begin{proposition}
    $f \in \calS(\R)$とする。このとき次が成り立つ:
    \begin{enumerate}
        \item $f \circ \tau_h,\, f \circ d_t,\, f^{\vee} \in \calS(\R)$
        \item $\calF [f \circ \tau_h](\xi) = e^{ih\xi} \calF [f](\xi)$
        \item $\calF [f \circ d_t](\xi) = \frac{1}{t} \calF [f] \left(\dfrac{\xi}{t}\right)$
        \item $\calF [f^{\vee}](\xi) = (\calF [f])^{\vee} (\xi) = \calF^{-1} [f](\xi)$
    \end{enumerate}
\end{proposition}

\begin{proof}
    (1) 簡単, (2) 自明, (4) 明らか

    (3)
    \vspace{-2em}\begin{equation}
        \begin{split}
            \calF[f \circ d_t](\xi)
                &= \frac{1}{\sqrt{2\pi}} \int_\R f \circ d_t (x) e^{-i\xi x} dx \\
                &= \frac{1}{\sqrt{2\pi}} \int_\R f(tx) e^{-i\xi x} dx \\
                &= \frac{1}{t} \frac{1}{\sqrt{2\pi}} \int_\R f(x) e^{-i\xi x / t} dx \\
                &= \frac{1}{t} \calF[f] (\xi / t)
        \end{split}
    \end{equation}
\end{proof}

\begin{theorem}
    $f, g \in \calS(\R)$とする。このとき次が成り立つ:
    \begin{itemize}
        \item 線形性:
            \begin{equation}
                \calF (f + g) = \calF f + \calF g, \quad \calF (\alpha f) = \alpha \calF f
                \tag{F1}
            \end{equation}
        \item 微分演算:
            \begin{equation}
                \calF \left[\dd{x} f\right] = i\xi\calF f,
                \quad \calF^{-1}\left[\dd{\xi} g\right] = -ix\calF^{-1} g
                \tag{F2}
            \end{equation}
        \item 掛け算:
            \begin{equation}
                \calF [xf] = i \dd{\xi} (\calF f)
                \tag{F3}
            \end{equation}
        \item 畳み込み\footnotemark: $\displaystyle (f * g)(x) := \int_\R f(x - t) g(t) dt$とおくと
            \begin{equation}
                \calF (f * g) = \sqrt{2 \pi} (\calF f) (\calF g)
                \tag{F4}
            \end{equation}
    \end{itemize}
\end{theorem}

\footnotetext{畳み込みは確率変数の和$X + Y$の確率分布を表すときに使ったりします。}

\begin{proof}
    (F1)は明らか。

    (F2)について、
    \begin{equation}
        \begin{split}
            \sqrt{2 \pi} \calF [f'](\xi)
                &= \int_\R f'(x) e^{-i\xi x} dx \\
                &= \left[ f(x) e^{-i\xi x} \right]_{x = -\infty}^\infty
                    + i\xi \int_\R f(x) e^{-i\xi x} dx \\
                &= i\xi \sqrt{2 \pi} \calF [f](\xi)
        \end{split}
    \end{equation}
    である。ただし、$f$が急減少関数であることからとくに$\sup_{x \in \R} |x| |f(x)| < \infty$、したがって
    \begin{equation}
        \lim_{x \to \pm\infty} |f(x) e^{i\xi x}| = \lim_{x \to \pm\infty} |f(x)| = 0
    \end{equation}
    であることを用いた。$\calF^{-1}$についても同様に示せる。

    (F3)について、
    \begin{equation}
        \begin{split}
            \sqrt{2\pi} \calF [xf](\xi)
                &= \int_\R xf(x) e^{-i\xi x} dx \\
                &= \int_\R f(x) \left(-\frac{1}{i}\right) \frac{\partial}{\partial \xi} (e^{-i\xi x}) dx \\
                &= \left(-\frac{1}{i}\right) \dd{\xi} \int_\R f(x) e^{-i\xi x} dx \\
                &= i \sqrt{2\pi} \dd{\xi} (\calF f) (\xi)
        \end{split}
    \end{equation}
    である。

    (F4)について、
    \begin{equation}
        \begin{split}
            \calF [f * g](\xi)
                &= \frac{1}{\sqrt{2\pi}} \int_\R \left( \int_\R f(x - t)\, g(t) dt \right) e^{-i\xi x} dx \\
                &= \frac{1}{\sqrt{2\pi}} \int_\R \int_\R f(x - t)\, e^{-i\xi (x - t)} g(t)\, e^{-i\xi t} dt dx \\
                &= \frac{1}{\sqrt{2\pi}} \int_\R
                    \left( \int_\R f(x - t)\, e^{-i\xi (x - t)} dx \right)
                    g(t)\, e^{-i\xi t} dt \\
                &= \frac{1}{\sqrt{2\pi}} \int_\R
                    \left( \int_\R f(x)\, e^{-i\xi x} dx \right)
                    g(t)\, e^{-i\xi t} dt \\
                &= \frac{1}{\sqrt{2\pi}} \int_\R
                    \sqrt{2\pi} (\calF f)(\xi)\,
                    g(t)\, e^{-i\xi t} dt \\
                &= \sqrt{2\pi} (\calF f)(\xi)
                    \frac{1}{\sqrt{2\pi}} \int_\R g(t)\, e^{-i\xi t} dt \\
                &= \sqrt{2\pi} (\calF f)(\xi) (\calF g)(\xi)
        \end{split}
    \end{equation}
    である。
\end{proof}





\begin{problem}
    $e^{-x^2} \in \calS$を確認せよ。
\end{problem}

\begin{problem}
    $\frac{1}{1+x^2} \not\in \calS$を確認せよ。
\end{problem}

\begin{problem}[ポアソン核]
    $f(x) = e^{-k|x|}\, (k > 0)$のフーリエ変換を求めよ。

    解答:
    \begin{equation}
        \calF[f](\xi) = \sqrt{\frac{2}{\pi}} \frac{k}{\xi^2 + k^2}
    \end{equation}
\end{problem}

\begin{problem}[ガウス核]
    $f(x) = e^{-\alpha x^2}\, (\alpha > 0)$のフーリエ変換を求めよ。

    解答:
    \begin{equation}
        \calF[f](\xi) = \sqrt{\frac{1}{2\alpha}} \exp\left(-\frac{\xi^2}{4\alpha}\right)
    \end{equation}
\end{problem}

\begin{problem}[全空間上の熱方程式]
    熱方程式の初期値問題
    \begin{equation}
        \begin{split}
            &u_t = u_{xx} \quad (-\infty < x < \infty,\, t > 0) \\
            &|u| \to 0 \quad (|x| \to \infty) \\
            &u(x, 0) = a(x)
        \end{split}
    \end{equation}
    の解を求めよ。

    解答;
    \begin{equation}
        u(x, t) = \int_{-\infty}^\infty \frac{1}{\sqrt{4\pi t}} e^{-\frac{(x-z)^2}{4t}} a(z) dz
    \end{equation}
\end{problem}

\begin{problem}
    次の関数のフーリエ変換を求めよ。$\alpha > 0$に対して
    \begin{equation}
        f(x) = \begin{cases}
            1 \quad (|x| \le \alpha) \\
            0 \quad (|x| > \alpha)
        \end{cases}
    \end{equation}

    解答:
    \begin{equation}
        \calF[f](\xi) = \sqrt{\frac{2}{\pi}} \frac{\sin(\alpha \xi)}{\xi}
    \end{equation}
\end{problem}

\begin{problem}
    次の関数のフーリエ変換を求めよ。$L > 0$に対して
    \begin{equation}
        f(x) = \begin{cases}
            x \quad (0 \le x \le L) \\
            0 \quad (\text{otherwise})
        \end{cases}
    \end{equation}

    解答:
    \begin{equation}
        \calF[f](\xi) = \frac{1}{\sqrt{2\pi}} \frac{\exp(-iL\xi)\, (1 + iL\xi) - 1}{\xi^2}
    \end{equation}
\end{problem}

\begin{problem}
    次の関数のフーリエ変換を求めよ。
    \begin{equation}
        f(x) = \frac{1}{\sqrt{|x|}}
    \end{equation}
    ただし次のことは用いてよい。
    \begin{equation}
        \int_0^\infty \sin x^2 dx = \int_0^\infty \cos x^2 dx = \sqrt{\frac{\pi}{8}}
    \end{equation}

    解答:
    \begin{equation}
        \calF[f](\xi) = \sqrt{\frac{1}{|\xi|}}
    \end{equation}
\end{problem}

\begin{problem}
    次の積分を計算せよ。
    \begin{equation}
        F(x) = \int_0^\infty e^{-t^2} \cos(xt) dt
    \end{equation}

    解答: $\frac{\sqrt{\pi}}{2} \exp\frac{-x^2}{4}$
\end{problem}

\begin{problem}
    連続関数$a(x)$は定数$A > 0$に対して$|a(x)| \le A\, (x \in \R)$をみたすものとし、
    $(x, t) \in \R \times [0, \infty)$で定義された関数
    \begin{equation}
        u(x, t) = \int_{-\infty}^\infty
            \frac{1}{2\sqrt{\pi t}} \exp\left\{ - \frac{(x - y)^2}{4t} \right\} a(y)\, dy
    \end{equation}
    を考える。
    このとき、$u_x(x, t)$は$\R \times (0, \infty)$上連続で
    \begin{equation}
        u_x(x, t) = \int_{-\infty}^\infty
            \frac{-1}{2\sqrt{\pi t}}
            \frac{x - y}{2t}
            \exp\left\{ - \frac{(x - y)^2}{4t} \right\} a(y)\, dy
    \end{equation}
    と表せることを示せ。
\end{problem}

\begin{problem}
    \,
    \begin{itemize}
        \item \cite[第VII章{\S}6 問題1)]{杉浦85}
        \item \cite[第III章 問7.1 (1)]{杉浦+89}
        \item \cite[第III章 問7.2 (1)]{杉浦+89}
    \end{itemize}
    を読者の演習問題とする。
\end{problem}




% ============================================================
%
% ============================================================
\chapter{Lagrange の未定乗数法}


% ------------------------------------------------------------
%
% ------------------------------------------------------------
\section{条件付き極値問題}

ここでは$U$を$\R^n$の空でない開集合とし、$f \colon U \to \R,\, \bm{g} \colon U \to \R^m$を$C^1$級とします。
さらに
\begin{equation}
    S_g \coloneqq \{ \bm{x} \in U \mid \bm{g}(\bm{x}) = 0 \}
\end{equation}
とおきます。さて、$S_g$は拘束条件$\bm{g}(\bm{x}) = 0$をみたす零点集合ですが、
$\bm{x}$がこの集合に沿って動くときの$f(\bm{x})$の極値を求めたいというのが
Lagrange の未定乗数法のモチベーションです。

\begin{definition}
    $\bm{x_0} \in S_g$とする。$\exists r > 0$\quad s.t.
    \begin{equation}
        f(\bm{x}) \le f(\bm{x_0}) \quad \text{for $\forall \bm{x} \in B_r(\bm{x_0}) \cap S_g$}
    \end{equation}
    が成り立つとき、$f$は\textbf{$\bm{x_0}$において$S_g$上の極大値をとる}という。
\end{definition}

$\bm{x}$が$S_g$に沿って動くときの$f(\bm{x})$の極値を求めるには、
素朴なアイディアとしては$\bm{x}$が$S_g$に沿って動くときの$f$の方向微分を考えればよさそうです。
しかし、そのような条件をきちんと考慮するのは結構面倒です。
そこで登場するのが、方向微分など持ち出さずとも単なる勾配$\nabla f$を考えればよいことを保証してくれる次の定理です。

\begin{theorem}
    $U, f, \bm{g}$と$\bm{a} \in S_g$に対し
    \begin{enumerate}
        \item $\bm{a}$において$f$は$S_g$上の極値をとる
        \item $\rank D\bm{g}(\bm{a}) = m$
            \quad ただし$D\bm{g}(\bm{a})
                = \left( g_{i x_j}(\bm{a}) \right)_{\substack{1 \le i \le m \\ 1 \le j \le n}}$
    \end{enumerate}
    が成り立つならば、$\exists \bm{\lambda} = (\lambda_1, \dots, \lambda_m) \in \R^m$\quad s.t.
    \begin{equation}
        \nabla f(\bm{a}) = \bm{\lambda} D\bm{g}(\bm{a})
            \quad\text{i.e.}\quad f_{x_j}(\bm{a}) = \sum_{i=1}^m \lambda_i g_{i x_j}(\bm{a})
    \end{equation}
\end{theorem}

\begin{proof}
    $\rank D\bm{g}(\bm{a}) = m$なので$n \ge m$であり、
    $\rank$の性質より行列$D\bm{g}(\bm{a})$の$0$でない小行列式の最大次数が$m$である。
    そこで、議論の一般性を失うことなく、必要ならば$x_j$の番号を取り替えて
    \begin{equation}
        \deldel[(g_1, \dots, g_m)]{(x_{n-m+1}, \dots, x_{n})} \neq 0
    \end{equation}
    とできる。
    要するに行列$D\bm{g}(\bm{a})$の "右側" が正則となるように並び替えるわけである。
    ここで、行列$D\bm{g}(\bm{a})$を "右側" と "左側" に分けたのにあわせて、
    定理で与えられたベクトルも成分を書き分けておく。すなわち
    \begin{equation}
        \bm{x} \coloneqq \begin{pmatrix}
            \bm{y} \\
            \bm{z}
        \end{pmatrix},
        \quad
        \bm{y} \coloneqq \begin{pmatrix}
            x_1 \\
            \vdots \\
            x_{n-m}
        \end{pmatrix},
        \quad
        \bm{z} \coloneqq \begin{pmatrix}
            x_{n-m+1} \\
            \vdots \\
            x_{n}
        \end{pmatrix},
        \quad
        \bm{a} \coloneqq \begin{pmatrix}
            \bm{b} \\
            \bm{c}
        \end{pmatrix}
    \end{equation}
    とおく。
    陰関数定理より、方程式$\bm{g}(\bm{y}, \bm{z}) = 0$は点$\bm{a}$の近傍で
    $\bm{z} = \phai(\bm{y})$と解ける。
    ここで$F \colon V \to \R,$
    \begin{equation}
        F(\bm{y}) \coloneqq f(\bm{y}, \phai(\bm{y}))
    \end{equation}
    とおくと、これは$f(\bm{x})$を点$\bm{x}$が$S_g$に沿って動くようにしたものとみなせる。
    よって$F$は$\bm{y} = \bm{b}$で極値をとる。
    すなわち$\nabla F(\bm{b}) = 0$である。
    一方、
    \begin{equation}
        \begin{alignedat}{2}
            \nabla F(\bm{y})
                &= \nabla_y f(\bm{y}, \phai(\bm{y}))
                    + \nabla_z f(\bm{y}, \phai(\bm{y}))\, D \phai(\bm{y})
                    &&\quad (\because\, \text{連鎖律}) \\
                &= \nabla_y f(\bm{y}, \phai(\bm{y}))
                    - \textcolor{blue}{
                        \nabla_z f(\bm{y}, \phai(\bm{y}))\,
                        D_z \bm{g}(\bm{y}, \phai(\bm{y}))^{-1}
                    }
                    D_y \bm{g}(\bm{y}, \phai(\bm{y}))
                    &&\quad (\because\, \text{陰関数定理})
        \end{alignedat}
    \end{equation}
    であるが、青文字の部分に$\bm{y} = \bm{b}$を代入したものを
    \begin{equation}
        \bm{\lambda} \coloneqq
            \nabla_z f(\bm{b}, \phai(\bm{b}))\,
            D_z \bm{g}(\bm{b}, \phai(\bm{b}))^{-1}
    \end{equation}
    とおけば、$\nabla F(\bm{b}) = 0$より
    \begin{equation}
        \begin{split}
            \nabla_y f(\bm{b}, \phai(\bm{b})) &= \bm{\lambda} D_y \bm{g}(\bm{b}, \phai(\bm{b})) \\
            \nabla_z f(\bm{b}, \phai(\bm{b})) &= \bm{\lambda} D_z \bm{g}(\bm{b}, \phai(\bm{b}))
        \end{split}
        \hspace{3em} \text{i.e.} \hspace{3em}
        \begin{split}
            \nabla_y f(\bm{a}) &= \bm{\lambda} D_y \bm{g}(\bm{a}) \\
            \nabla_z f(\bm{a}) &= \bm{\lambda} D_z \bm{g}(\bm{a})
        \end{split}
    \end{equation}
    すなわち
    \begin{equation}
        \nabla f(\bm{a}) = \bm{\lambda} D \bm{g}(\bm{a})
    \end{equation}
    を得る。
\end{proof}

\begin{corollary}[Lagrange の未定乗数法]
    $U, f, \bm{g}$と$\bm{a} \in S_g$に対し、
    \begin{enumerate}
        \item $\bm{a}$において$f$は$S_g$上の極値をとる
    \end{enumerate}
    ならば、次のいずれか一方のみが成り立つ。
    \begin{enumerate}
        \item $F \colon \R^{n+m} \to \R,\;
            F(\bm{x}, \bm{\lambda}) = f(\bm{x}) - \bm{\lambda} \cdot \bm{g}(\bm{x})$
            に対し$\exists \bm{\lambda_0} \in \R^m$\, s.t.
            \begin{equation}
                \nabla F(\bm{a}, \bm{\lambda_0}) = 0
            \end{equation}
        \item $\rank D\bm{g}(\bm{a}) < m$
    \end{enumerate}
\end{corollary}

\begin{proof}
    簡単なので省略
\end{proof}












\begin{problem}
    方程式
    \begin{equation}
        f(x, y, z) \coloneqq x^2 + (x - y^2 + 1) z - z^3 = 0
    \end{equation}
    を満たす点$(x, y, z)$が点$(0, 0, 1)$の近傍で$z = \phai(x, y)$と書けることを示せ。
    また、この点における$\deldel[z]{x},\, \deldel[z]{y}$の値を求めよ。

    解答:
    \begin{equation}
        \deldel[z]{x} = \frac{1}{2},\quad \deldel[z]{y} = 0
    \end{equation}
\end{problem}

\begin{problem}
    変数$x, y, z, u, v$が
    \begin{equation}
        \begin{cases}
            &xy + uv = 0 \\
            &x^2 + y^2 + z^2 = u^2 + v^2
        \end{cases}
    \end{equation}
    を満たすとする。
    点$(2, 0, 1, 0, \sqrt{5})$の近傍で$(u, v) = \phai(x, y, z)$と書けることを示せ。
    また、点$(x, y, z) = (2, 0, 1)$における$\phai$のヤコビ行列を求めよ。

    解答:
    \begin{equation}
        \frac{1}{\sqrt{5}} \begin{bmatrix}
            0 & -2 & 0 \\
            2 & 0 & 1
        \end{bmatrix}
    \end{equation}
\end{problem}

\begin{problem}
    \cite[第II章 問6.2]{杉浦+89}
    を読者の演習問題とする。
\end{problem}



\begin{problem}
    写像$f: \R^3 \to \R^3,$
    \begin{equation}
        f(x_1, x_2, x_3) = \left( \sum_i x_i, \sum_i x_i^2, \sum_i x_i^3\right)
    \end{equation}
    が点$(a_1, a_2, a_3)$の近傍で一対一対応となるような$a_i$の条件を求めよ。
\end{problem}

\begin{problem}
    変数$x = (x_1, x_2, x_3),\, u = (u_1, u_2, u_3),\, v = (v_1, v_2, v_3)$の間に
    \begin{equation}
        \begin{cases}
            u_1 = x_1 + x_2 + x_3 \\
            u_2 = x_1 x_2 + x_2 x_3 + x_3 x_1 \\
            u_3 = x_1 x_2 x_3
        \end{cases}
        \quad
        \begin{cases}
            v_1 = x_1^2 + x_2^2 \\
            v_2 = x_2^2 + x_3^2 \\
            v_3 = x_3^2 + x_1^2
        \end{cases}
    \end{equation}
    という関係があるとし、$a = (a_1, a_2, a_3)$とする。
    $a_i \neq a_j\, (i \neq j)$のとき、
    $x = a$の近傍で$v$は$u$の関数として表せることを示せ。
    さらに$a_1 a_2 a_3 \neq 0$とし、
    $x = a$に対し定まる$u$を$b$とおく。
    このとき$x = a$の近傍で$u$は$v$の関数として表せることを示し、
    写像$v \mapsto u$の点$b$におけるヤコビ行列式を求めよ。

    解答:
    \begin{equation}
        \frac{(a_1 - a_2)(a_2 - a_3)(a_1 - a_3)}{16 a_1 a_2 a_3}
    \end{equation}
\end{problem}

\begin{problem}
    \cite[第II章 例題6.2]{杉浦+89}、
    \cite[第II章 問6.3]{杉浦+89}
    を読者の演習問題とする。
\end{problem}


\begin{problem}
    $a \in \R^n\, (a \neq0),\, b \in \R$とし、
    \begin{equation}
        g(x) \coloneqq \sum_{i = 1}^n a_i x_i + b
    \end{equation}
    とおく。$S_g \coloneqq \{ x \in \R^n \mid g(x) = 0 \}$のもとで
    \begin{equation}
        f(x) \coloneqq |x|^2
    \end{equation}
    の最小値を求めよ。

    解答:$\frac{b^2}{|a|^2}$
\end{problem}


\begin{problem}
    $S_g \coloneqq \{(x, y, z) \mid g(x, y, z) = x^2 + y^2 + z^2 - 1 = 0\}$のもとで
    \begin{equation}
        f(x, y, z) \coloneqq x^2 + y^2 - z^2 + 4xz + 4yz
    \end{equation}
    の極値を求め、極大・極小を判定せよ。

    解答:極小値$-3$、極大値$3$
\end{problem}

\begin{problem}
    $(x, y) \in \R^2$で定義された関数$f(x, y) = xy (x^2 + y^2 - 1)$の極値を求めよ。

    解答:極小値$f(\pm 1/2, \pm 1/2) = -1/8$、極大値$(\pm 1/2, \mp 1/2) = 1/8$
\end{problem}

\begin{problem}
    $(x, y) \in \R^2$に対し
    \begin{equation}
        \begin{split}
            f(x, y) &= x^4 + y^4 \\
            g(x, y) &= xy - 4
        \end{split}
    \end{equation}
    を考える。$M_g \coloneqq \{ (x, y) \mid g(x, y) = 0 \}$上での$f$の最大値、最小値を求めよ。

    解答:最大値なし、最小値$32$
\end{problem}

\begin{problem}
    \cite[第II章 問題7.1, 7.2, 7.4]{杉浦+89}を読者の演習問題とする。
\end{problem}



% ============================================================
%
% ============================================================
\chapter{最小二乗法}

% ------------------------------------------------------------
%
% ------------------------------------------------------------
\section{最小二乗法}

$n$個の点$(x_i, y_i)\, (i = 1, \dots, n)$が与えられたとします。
$x_i, y_i$の間には、パラメータ$a, b \in \R$によって
\begin{equation}
    y_i = a x_i + b\quad (i = 1, \dots, n)
\end{equation}
の関係があると仮定します。このとき、誤差
\begin{equation}
    e_i \coloneqq y_i - (a x_i + b)\quad (i = 1, \dots, n)
\end{equation}
の2乗和
\begin{equation}
    J(a, b) \coloneqq \sum_{i=1}^n e_i^2 = \sum_{i=1}^n (y_i - a x_i - b)^2
\end{equation}
が最小となるような$(a, b)$を求めることを考えます。
この問題は$J(a, b)$の極小値を求める問題に帰着されるので、以上の状況設定で充分といえば充分なのですが、
多変数(すなわち各点が$(x_{i1}, \dots, x_{im}, y_i)$である場合)への拡張を見据えて
もう少し一般性のある形に書き換えてみます。
すなわち、
\begin{equation}
    \begin{split}
        \bm{x} \coloneqq (x_1, \dots, x_n)^\tra,\quad
        \bm{y} \coloneqq (y_1, \dots, y_n)^\tra \\
        Q \coloneqq \begin{bmatrix}
            x_1 & 1 \\
            \vdots & \vdots \\
            x_n & 1
        \end{bmatrix},\quad
        \bm{a} \coloneqq \begin{bmatrix}
            a \\
            b
        \end{bmatrix},\quad
        \bm{e} \coloneqq \bm{y} - Q \bm{a}
    \end{split}
\end{equation}
とおきます。すると簡単な計算から
\begin{equation}
    J(a, b) = \|\bm{y}\|^2 + \langle Q^\tra Q \bm{a}, \bm{a} \rangle - \langle 2Q^\tra \bm{y}, \bm{a} \rangle
\end{equation}
が成り立つので、$J(a, b)$の最小化問題は
\begin{equation}
    F(\bm{v}) \coloneqq \langle Q^\tra Q \bm{v}, \bm{v} \rangle - \langle 2Q^\tra \bm{y}, \bm{v} \rangle
\end{equation}
の最小化問題に帰着されます。
ここで、$J$が極値をとるための必要条件$\nabla J(a, b) = 0$は、少し計算すると
\begin{equation}
    Q^\tra Q \bm{a} = Q^\tra \bm{y}
    \label{eq:10:1}
\end{equation}
と表せることがわかります。
したがって、$Q^\tra Q$が正則ならば$\bm{a}$が一意に定まってくれて嬉しいのですが、
次の命題によれば、実用上ほとんどの場合$Q^\tra Q$は正則だということがわかります。

    \begin{proposition}
        $Q$に対し次が成り立つ。
        \begin{enumerate}
            \item $Q^\tra Q$は非負定値対称行列である
            \item $x_1 = \dots = x_n$ではないとすると、$Q^\tra Q$は正定値対称行列である。
                したがって正則である。
        \end{enumerate}
    \end{proposition}

\begin{proof}
    (1)は内積を行列の積の形に書き直せばすぐわかります。

    (2)は成分ごとの方程式を考えて矛盾をいえば示せます。
\end{proof}

(\ref{eq:10:1})をみたす$\bm{a}$が$J(a, b)$の最小化問題の一意的な解であることを
明確に述べたのが次の定理です。

\begin{theorem}
    $x_1 = \dots = x_n$でなければ、(\ref{eq:10:1})をみたす$\bm{a}$は
    \begin{equation}
        F(\bm{a}) < F(\bm{v}) \quad (\bm{v} \in \R^2,\, \bm{v} \neq \bm{a})
        \label{eq:10:2}
    \end{equation}
    をみたす。
\end{theorem}

\begin{proof}
    $\bm{v} = \bm{a} + (\bm{v} - \bm{a})$と分解して
    $F(\bm{v})$と$F(\bm{a})$の間の不等式を導けば示せます。
    途中で$\bm{a}$が(\ref{eq:10:1})をみたすという性質を使って式を綺麗にします。
\end{proof}

上の定理は逆も成り立ちます。

\begin{theorem}
    (\ref{eq:10:2})をみたす$\bm{a}$は(\ref{eq:10:1})をみたす。
\end{theorem}

\begin{proof}
    $f(t) \coloneqq F(\bm{a} + t\bm{w})$は$t$に関して下に凸な2次関数ですが、
    $f(t)$が$t = 0$で極値をとることから$Q^\tra Q \bm{a} = Q^\tra \bm{y}$を導くことができます。
\end{proof}



% ============================================================
%
% ============================================================
\chapter{Gamma 関数と Beta 関数}

% ------------------------------------------------------------
%
% ------------------------------------------------------------
\section{Gamma 関数とBeta 関数}

ここでは\textbf{Gamma 関数}および\textbf{Beta 関数}という特殊関数を扱います。
本題に入る前に、まず重積分の変数変換公式を確認しておきます。

\begin{theorem}[変数変換公式]
    $U, V$を$\R^n$の有界部分集合とし、
    写像$\Phi \colon U \to V,$
    \begin{equation}
        \Phi(u) = (X_1(u), \dots, X_n(u))
    \end{equation}
    は全単射かつ$C^1$級であるとする。
    さらに$\forall u \in U$に対し
    \begin{equation}
        \deldel[(X_1, \dots, X_n)]{(u_1, \dots, u_n)}(u) \neq 0
    \end{equation}
    とする。
    このとき、$U$の任意の体積確定部分集合$U_1$と
    $V_1 \coloneqq \Phi(U_1)$に対し
    \begin{equation}
        \int_{U_1} f(x) dx = \int_{V_1} f(\Phi(u))\, |\det J_\Phi(u)|\, du
    \end{equation}
    が成り立つ。
\end{theorem}

\begin{proof}
    長いので省略\footnote{
        参考文献\cite[第VII章 \S{4}]{杉浦85}を参照。
    }
\end{proof}

\begin{example*}[$n$次元極座標変換]
    $\R^n$の極座標変換は
    \begin{equation}
        \begin{cases}
            x_1 &= r \cos \theta_1 \\
            x_2 &= r \sin \theta_1 \cos \theta_2 \\
            x_3 &= r \sin \theta_1 \sin \theta_2 \cos \theta_3 \\
            \vdots \\
            x_{n-1} &= r \sin \theta_1 \cdots \sin \theta_{n-2} \cos \theta_{n-1} \\
            x_{n} &= r \sin \theta_1 \cdots \sin \theta_{n-2} \sin \theta_{n-1}
        \end{cases}
    \end{equation}
    ただし
    \begin{equation}
        \begin{split}
            &0 \le r < \infty \\
            &0 \le \theta_i \le \pi \quad (i = 1, \dots, n - 2) \\
            &0 \le \theta_{n-1} \le 2\pi
        \end{split}
    \end{equation}
    で与えられます。ヤコビアンは
    \begin{equation}
        \det J_\Phi(r, \theta_1, \dots, \theta_{n-1})
            = r^{n-1} \sin^{n-2} \theta_1 \sin^{n-3} \theta_2 \cdots \sin \theta_{n-2}
    \end{equation}
    です。
\end{example*}

\begin{theorem}
    \begin{enumerate}
        \item $x > 0$に対し、広義積分
            \begin{equation}
                \int_0^\infty e^{-t} t^{x - 1} dt
            \end{equation}
            は各点で絶対収束する。
        \item $x, y > 0$に対し、広義積分
            \begin{equation}
                \int_0^1 t^{x - 1} (1 - t)^{y - 1} dt
            \end{equation}
            は各点で絶対収束する。
    \end{enumerate}
    \label{11:thm:1}
\end{theorem}

\begin{definition}
    \cref{11:thm:1}の(1)で定義される関数$\Gamma(x)$を\textbf{Gamma 関数}、
    (2)で定義される関数$B(x)$を\textbf{Beta 関数}という。
\end{definition}

見ての通り、$\Gamma(x)$や$B(x, y)$はパラメータを含む広義積分で定義された関数です。
%実はもっと強く広義一様収束までいえるのですが、ここではとりあえず各点収束を示します。

\begin{proof}[\cref{11:thm:1}の証明.]
    (1)
    $x > 0$を任意にとる。
    積分範囲が非有界な$\int_1^\infty e^{-t} t^{x-1} dt$と
    被積分関数が非有界な$\int_0^1 e^{-t} t^{x-1} dt$とに分けて収束性を考える。
    まず$\int_1^\infty e^{-t} t^{x-1} dt$を考える。任意の正整数$n$に対し
    \begin{equation}
        e^{-t} = O(t^{-n})\quad (t \to \infty)
    \end{equation}
    なので、
    \begin{equation}
        e^{-t} t^{x-1} = O(t^{x-n-1})\quad (t \to \infty)
    \end{equation}
    である。$n > x$をひとつ選べば
    \begin{equation}
        \int_1^\infty t^{x-n-1} dt
            = \left[ -\frac{1}{n-x} t^{-(n-x)} \right]_1^\infty
            = \frac{1}{n - x}
            \in \R
    \end{equation}
    なので、優関数の方法により
    \begin{equation}
        \int_1^\infty e^{-t} t^{x - 1} dt
    \end{equation}
    も絶対収束する。

    つぎに$\int_0^1 e^{-t} t^{x-1} dt$を考える。
    $e^{-t}$は$t = 0$の近傍で有界なので
    \begin{equation}
        e^{-t} t^{x-1} = O(t^{x-1})\quad (t \to +0)
    \end{equation}
    である。$0 < x < 1$のとき
    \begin{equation}
        \int_0^1 t^{x-1} dt
            = \left[ \frac{1}{x} t^{x} \right]_0^1
            = \frac{1}{x}
            \in \R
    \end{equation}
    なので、優関数の方法により
    \begin{equation}
        \int_0^1 e^{-t} t^{x - 1} dt
    \end{equation}
    も絶対収束する。$x \ge 1$のときは$\int_0^1 e^{-t} t^{x-1} dt$は広義でない普通の積分である。

    $x > 0$は任意であったから、$\int_0^\infty e^{-t} t^{x - 1} dt$は$x > 0$の各点で絶対収束する。
    \\

    (2) $x, y < 1$のときを考えれば充分です。(1)と同様に広義積分を分けて収束性を議論すれば示せます。
\end{proof}

\begin{proposition}[Gamma 関数と Beta 関数の基本性質]
    $x, y > 0$に対し次が成り立つ。
    \begin{enumerate}
        \item $\Gamma(1) = 1, \Gamma(x + 1) = x \Gamma(x)$
        \item $\Gamma(x + n) = (x + n - 1)(x + n - 2) \cdots x \Gamma(x)$
            \quad とくに \quad
            $\Gamma(n + 1) = n!$
        \item $B(x, y) = B(y, x)$ \vspace{0.5em}
        \item $B(x, y) = \frac{\Gamma(x)\, \Gamma(y)}{\Gamma(x + y)}$
    \end{enumerate}
\end{proposition}

\begin{proof}
    (1), (2), (3) は簡単です。

    (4)
    変数変換によって
    \begin{equation}
        \begin{split}
            \Gamma(x) &= 2 \int_0^\infty e^{-u^2} u^{2x-1} du \\
            B(x, y) &= 2 \int_0^{\pi/2} \sin^{2x-1} \theta \cos^{2y-1} \theta\, d\theta
        \end{split}
    \end{equation}
    と書けることに注意する。
    集合列$\{J_R\}_{R \in \N},\, \{I_R\}_{\R \in \N}$をそれぞれ
    \begin{equation}
        \begin{split}
            J_R &\coloneqq [0, R] \times [0, R] \\
            I_R &\coloneqq \{ (u, v) \in \R^2 \mid u^2 + v^2 \le R^2,\, u \ge 0,\, v \ge 0 \}
        \end{split}
    \end{equation}
    と定めると、これらは$[0, \infty) \times [0, \infty)$のコンパクト近似列\footnote{
        コンパクト近似列の定義は参考文献\cite[第VII章 \S{1}]{杉浦85}を参照。
    }になっている。
    \begin{alignat}{3}
        \Gamma(x) \Gamma(y)
            &= 4 \int_0^\infty e^{-u^2} u^{2x-1} du
                \int_0^\infty e^{-v^2} v^{2y-1} dv \\
            &= 4 \lim_{R \to \infty}
                \int_0^R e^{-u^2} u^{2x-1} du
                \int_0^R e^{-v^2} v^{2y-1} dv \\
            &= 4 \lim_{R \to \infty}
                \int_0^R \int_0^R e^{-(u^2 + v^2)} u^{2x-1} v^{2y-1} du\, dv \\
            &= 4 \lim_{R \to \infty}
                \iint_{J_R} e^{-(u^2 + v^2)} u^{2x-1} v^{2y-1} du\, dv \\
        \intertext{広義重積分可能ならば近似列を交換できるから}
            &= 4 \lim_{R \to \infty}
                \iint_{I_R} e^{-(u^2 + v^2)} u^{2x-1} v^{2y-1} du\, dv \\
            &= 4 \lim_{R \to \infty}
                \int_0^{\pi/2} \int_0^R
                    e^{-r^2} r^{2(x + y) - 1} \cos^{2x - 1} \theta \sin^{2y - 1} \theta dr\, d\theta \\
            &= \lim_{R \to \infty}
                2 \int_0^R e^{-r^2} r^{2(x + y) - 1} dr
                \cdot 2 \int_0^{\pi/2} \cos^{2x - 1} \theta \sin^{2y - 1} \theta d\theta \\
            &= 2 \int_0^\infty e^{-r^2} r^{2(x + y) - 1} dr
                \cdot 2 \int_0^{\pi/2} \cos^{2x - 1} \theta \sin^{2y - 1} \theta d\theta \\
            &= \Gamma(x + y) B(x, y)
    \end{alignat}
    より定理の式が成り立つ。
\end{proof}


\begin{proposition}
    \begin{enumerate}
        \item $\Gamma(x)$は$x > 0$上で{\smooth}級であり
            \begin{equation}
                \Gamma^{(n)}(x) = \int_0^\infty e^{-t} t^{x-1} (\log t)^n dt
                \label{11:eq:2}
            \end{equation}
            である。
        \item $\log \Gamma(x)$は$x > 0$上の凸関数である。
    \end{enumerate}
\end{proposition}

(2)はすこし唐突な印象もありますが、
実はこのあと出てくる Bohr-Mollerup の定理において重要な役割を果たします。

\begin{proof}
    (1)
    $\forall n \in \N$に対し広義積分
    \begin{equation}
        \int_0^\infty e^{-t} t^{x-1} (\log t)^n dt
        \label{11:eq:1}
    \end{equation}
    が$x > 0$上広義一様収束することさえ示せば
    積分記号下の微分ができるので、
    あとは数学的帰納法により (1) が示せる
    (数学的帰納法の部分は簡単なのでここでは省略する)。
    そこでまず$0 < \forall x_0 < \forall x_1 < \infty$を固定し、
    $I \coloneqq [x_0, x_1]$上での一様収束性を示そう。
    表記の簡略化のために
    \begin{equation}
        f_n(x, t) \coloneqq e^{-t} t^{x-1} (\log t)^n
    \end{equation}
    とおくと、$t \to +0$で
    \begin{equation}
        \begin{cases}
            e^{-t} &= O(1) \\
            t^{x-1} &= O(t^{x_0 - 1}) \\
            \log t &= O(t^{-\alpha}) \qquad (\text{ただし$\alpha$は$0 < \alpha < x_0/n$なる適当な定数})
        \end{cases}
    \end{equation}
    なので
    \begin{equation}
        f_n(x, t) = O(t^{x_0 - n\alpha - 1})
    \end{equation}
    である。$x_0 - n\alpha - 1 > -1$ゆえに
    $\int_0^1 t^{\alpha - 1} dt$は収束するから、
    Weierstrass の定理により広義積分
    \begin{equation}
        \int_0^1 f_n(x, t) dt
    \end{equation}
    は$I$上一様収束する。
    一方、$t \to \infty$で
    \begin{equation}
        \begin{cases}
            e^{-t} &= O(t^{- x_1 - n - 1}) \\
            t^{x-1} &= O(t^{x_1 - 1}) \\
            \log t &= O(t)
        \end{cases}
    \end{equation}
    なので
    \begin{equation}
        f_n(x, t) = O(t^{-2})
    \end{equation}
    である。
    $\int_1^\infty t^{-2} dt$は収束するから、
    Weierstrass の定理により広義積分
    \begin{equation}
        \int_1^\infty f_n(x, t) dt
    \end{equation}
    は$I$上一様収束する。
    以上より、広義積分(\ref{11:eq:1})は$x > 0$上広義一様収束する。

    (2)
    上で示した(\ref{11:eq:1})より、$\forall u \in \R$に対し
    \begin{equation}
        \begin{split}
            \Gamma(x) u^2 + 2 \Gamma'(x) u + \Gamma''(x)
                &= \int_0^\infty e^{-t} t^{x-1} (u + \log t)^2 dt \\
                &\ge 0
        \end{split}
    \end{equation}
    である。よって判別式は
    \begin{equation}
        D/4 = \Gamma'(x) - \Gamma(x) \Gamma''(x) \le 0
    \end{equation}
    をみたす。
    したがって
    \begin{equation}
        \begin{split}
            (\log \Gamma(x))''
                &= \left(\frac{\Gamma'(x)}{\Gamma(x)}\right)' \\
                &= \frac{\Gamma(x) \Gamma''(x) - \Gamma'(x) }{\Gamma(x)^2} \\
                &\ge 0
        \end{split}
    \end{equation}
    すなわち$\Gamma(x)$は$x > 0$上の凸関数である。
\end{proof}




% ------------------------------------------------------------
%
% ------------------------------------------------------------
\section{Bohr-Mollerup の定理}

    \begin{theorem}[Bohr-Mollerup の定理、$\Gamma$関数の特徴付け]
        $f \colon (0, \infty) \to \R$が
        \begin{enumerate}
            \item $f(x + 1) = x f(x)$
            \item $f(x) > 0$かつ$\log f(x)$は凸関数
            \item $f(1) = 1$
        \end{enumerate}
        をみたすとする。このとき、任意の$x > 0$に対し
        \begin{equation}
            f(x) = \Gamma(x) = \lim_{n \to \infty} \frac{n! n^x}{x(x + 1) \cdots (x + n)}
            \quad (\text{ガウスの公式})
            \label{eq:12:1}
        \end{equation}
        \label{11:thm:2}
    \end{theorem}

$f(x) = \Gamma(x)$の証明の際は、
$\Gamma$の積分表式を持ち出すのではなく、
$f$も$\Gamma$も式(\ref{eq:12:1})の極限式で書けるから一致するという論法で示します。

\begin{proof}
    条件(1)から、$\forall n \ge 1$に対し
    \begin{equation}
        f(x + n) = (x + n - 1) \cdots (x + 1) f(x)
        \label{11:eq:5}
    \end{equation}
    である。さらに$x = 1$として条件(3)を用いれば
    \begin{equation}
        f(n + 1) = n!
        \label{11:eq:6}
    \end{equation}
    が成り立つ。また、条件(2)より$g(x) \coloneqq \log f(x)$は凸関数であるから、
    $0 < {}^\forall a < {}^\forall t < {}^\forall b$に対し
    \begin{equation}
        \frac{g(t) - g(a)}{t - a}
            < \frac{g(b) - g(a)}{b - a}
            < \frac{g(b) - g(t)}{b - t}
        \label{11:eq:3}
    \end{equation}
    が成り立つ。

    \underline{Step 1:}
    さて、ひとまず$x \in (0, 1]$を固定して
    ガウスの公式(\ref{eq:12:1})を示していこう。
    ここで、任意の自然数$n \ge 2$に対し
    \begin{equation}
        \begin{split}
            \log (n - 1)
                &= \log f(n) - \log f(n - 1) \\
                &\le \frac{\log f(n + 1) - \log f(n)}{x} \\
                &\le \log f(n + 1) - \log f(n) \\
                &= \log n
        \end{split}
        \label{11:eq:4}
    \end{equation}
    が成り立つ。
    ただし、途中の不等式は$(a, t, b) = (n - 1, n, n + x),\, (n, n + x, n + 1)$に対し
    不等式(\ref{11:eq:3})を適用したものである。
    不等式(\ref{11:eq:4})の各辺に$x$を掛け、指数関数の値をとり、さらに$f(x)$を掛ければ
    \begin{equation}
        (n - 1)^x f(n) \le f(n + x) \le n^x f(n)
    \end{equation}
    を得る。これと式(\ref{11:eq:5})から
    \begin{equation}
        \frac{(n - 1)^x f(n)}{x (x + 1) \cdots (x + n - 1)}
            \le f(x)
            \le \frac{n^x f(n)}{x (x + 1) \cdots (x + n - 1)}
    \end{equation}
    であり、左の不等式だけ$n$を$n + 1$に置きなおして式(\ref{11:eq:6})を用いると
    \begin{equation}
        \underbrace{\frac{n! n^x}{x (x + 1) \cdots (x + n)}}_{\text{$a_n(x)$とおく}}
            \le f(x)
            \le \underbrace{\frac{n! n^x}{x (x + 1) \cdots (x + n)}}_{a_n(x)} \frac{x + n}{n}
    \end{equation}
    を得る。したがって、$n \to \infty$の極限を考えれば
    \begin{equation}
        f(x) = \lim_{n \to \infty} a_n(x)
    \end{equation}
    がいえる。
    $x \in (0, 1]$は任意であったから、
    $x \in (0, 1]$においてガウスの公式(\ref{eq:12:1})の成立がいえた。

    \underline{Step 2:}
    $x > 1$の場合は、
    $x = y + m\, (0 < y \le 1,\, m \in \N)$とおけば、
    $\forall n > m$に対し
    \begin{equation}
        \begin{split}
            \frac{n! n^x}{x \cdots (x + n)}
                &= \frac{n! n^y n^m}{(y + m) \cdots (y + m + n)} \\
                &= \frac{n! n^y}{y \cdots (y + n)}
                    \frac{n^m y \cdots (y + n)}{(y + m) \cdots (y + m + n)} \\
                &= \frac{n! n^y}{y \cdots (y + n)}
                    \frac{n^m y \cdots (y + m - 1) \cancel{(y + m) \cdots (y + n)}}
                        {\cancel{(y + m) \cdots (y + n)} (y + n + 1) \cdots (y + n + m)} \\
                &= \frac{n! n^y}{y \cdots (y + n)}
                    \underbrace{\frac{n^m}{(y + n + 1) \cdots (y + n + m)}}_{\to 1\; (n \to \infty)}
                    y \cdots (y + m - 1)
        \end{split}
    \end{equation}
    が成り立つから、
    \begin{equation}
        \begin{split}
            \lim_{n \to \infty} \frac{n! n^x}{x \cdots (x + n)}
                &= y \cdots (y + m - 1) f(y) \\
                &= f(y + m) \\
                &= f(x)
        \end{split}
    \end{equation}
    である。
    したがって
    $x > 1$においてもガウスの公式(\ref{eq:12:1})の成立がいえた。

    \underline{Step 3:}
    $\Gamma$も定理の仮定を満たすから、
    $\Gamma$も$x > 0$でガウスの公式(\ref{eq:12:1})をみたす。
    したがって$x > 0$で$f(x) = \Gamma(x)$である。
\end{proof}

    \begin{corollary}
        上の定理の条件(3)を除くと
        \begin{equation}
            f(x) = f(1) \Gamma(x) \quad (x > 0)
        \end{equation}
        が成り立つ。
        \label{11:cor:1}
    \end{corollary}

\begin{proof}
    $h(x) \coloneqq f(x) / f(1)$が条件(1),(2),(3)をみたすことから直ちに成り立つ。
\end{proof}

    \begin{proposition}
        $\forall x \in D \coloneqq \R - (- \N)$に対し
        \begin{equation}
            \lim_{n \to \infty} \frac{n! n^x}{x(x + 1) \cdots (x + n)} \in \R
        \end{equation}
        が存在する。
        \label{11:prop:1}
    \end{proposition}

\begin{proof}
    $x \in D$を任意にとる。$x > 0$の場合は成立がわかっているから$x < 0$の場合のみ考えればよい。
    すると、ある$m \in \N$が存在して$y = x + m > 0$なので、
    充分大きな任意の$n$に対し
    \begin{equation}
        \begin{split}
            \frac{n! n^x}{x \cdots (x + n)}
                &= \frac{n! n^y}{n^m (y - m) \cdots (y - m + n)} \\
                &= \frac{1}{(y - m) \cdots (y - 1)}
                    \underbrace{\frac{n! n^y}{y \cdots (y + n)}}_{\to \Gamma(y)\; (n \to \infty)}
                    \underbrace{\frac{(y - m + n + 1) \cdots (y + n)}{n^m}}_{\to 1\; (n \to \infty)}
        \end{split}
    \end{equation}
    が成り立つ。したがって
    \begin{equation}
        \lim_{n \to \infty} \frac{n! n^x}{x(x + 1) \cdots (x + n)} \in \R
    \end{equation}
    が存在する。
\end{proof}

\begin{definition}[$\Gamma$の極限式による定義]
    $\forall x \in D \coloneqq \R - (- \N)$に対し
    \begin{equation}
        \lim_{n \to \infty} \frac{n! n^x}{x(x + 1) \cdots (x + n)} \in \R
    \end{equation}
    と定義する。
\end{definition}

\begin{proposition}
    $\forall x \in D$に対し次が成り立つ。
    \begin{enumerate}
        \item \begin{equation}
            \Gamma(x) \neq 0
        \end{equation}
        \item \begin{equation}
            \mathrm{sign}\, \Gamma(x) = \begin{cases}
                1 \quad &(x > 0) \\
                (-1)^m \quad &(-m < x < -m+1,\, m \in \N)
            \end{cases}
        \end{equation}
        \item \begin{equation}
            \lim_{x \to -n} (x + n) \Gamma(x) = \frac{(-1)^n}{n!} \quad (n \in \N)
        \end{equation}
    \end{enumerate}
\end{proposition}

(3)は複素数に拡張された$\Gamma(x)$の点$-n$における留数を表しています。

\begin{proof}
    (1), (2) は\cref{11:prop:1}から明らか。

    (3)
    \begin{equation}
        \begin{split}
            (x + n) \Gamma(x)
                &= \frac{(x + n) \cdots x \Gamma(x)}{(x + n - 1) \cdots x} \\
                &= \frac{\Gamma(x + n - 1)}{(x + n - 1) \cdots x} \\
                &\to \frac{\Gamma(1)}{(-1)(-2) \cdots (-n)} \quad (x \to -n) \\
                &= \frac{(-1)^n}{n!}
        \end{split}
    \end{equation}
\end{proof}


% ------------------------------------------------------------
%
% ------------------------------------------------------------
\section{Stirling の公式}

ここでは$x \to \infty$における$\Gamma$の挙動を漸近的に評価する方法を考えていきます。
まず出発地点として$n!$の値を$\int_1^n \log x\, dx$で評価してみましょう。
$\int_1^n \log x\, dx$の値を "短冊" で近似することを考えると
\begin{equation}
    \int_1^n \log x\, dx
        = \log 2 + \cdots \log (n - 1) + \frac{1}{2} \log n + \delta_n
\end{equation}
が成り立ちます(ただし$\delta_n$は誤差)。
すると、簡単な計算により
\begin{equation}
    \begin{split}
        \log (n-1)! &= \left(n - \frac{1}{2}\right) \log n - n + 1 - \delta_n \\
        \therefore \quad \Gamma(n) &= n^{n - \frac{1}{2}} e^{-n} e^{1-\delta_n}
    \end{split}
\end{equation}
と表せることがわかります。
そこで、$x > 0$に対しても何らかの関数$\mu(x)$によって
\begin{equation}
    f(x) \coloneqq x^{x - \frac{1}{2}} e^{-x} e^{\mu(x)}
    \label{11:eq:7}
\end{equation}
を$\Gamma(x)$の定数倍に一致させられないだろうか?
というのが Stirling の公式の基本的なアイディアです。
ここで Bohr-Mollerup の定理(\cref{11:thm:2})によれば、$f$が$x > 0$で条件
\begin{enumerate}
    \item $f(x + 1) = x f(x)$
    \item $f(x) > 0$かつ$\log f(x)$は凸関数
\end{enumerate}
をみたしてくれれば目標達成です。以下、このことを確認していきます。

\begin{lemma}
    式(\ref{11:eq:7})で定義された関数$f$が$x > 0$で条件(1)をみたすには、
    $\mu(x)$が
    \begin{equation}
        \mu(x) - \mu(x + 1)
            = \underbrace{
                \left(x + \frac{1}{2}\right) \log \left(1 + \frac{1}{x}\right) - 1
            }_{\text{$g(x)$とおく}}
    \end{equation}
    をみたすことが必要十分である。
    \label{11:lem:1}
\end{lemma}

\begin{proof}
    $x > 0$とする。
    $f$の定義式(\ref{11:eq:7})によれば
    \begin{equation}
        \frac{f(x + 1)}{f(x)} = x \left(1 + \frac{1}{2}\right)^{x + \frac{1}{2}} e^{\mu(x + 1) - \mu(x) - 1}
    \end{equation}
    なので、$f$が条件(1)、すなわち
    \begin{equation}
        \frac{f(x + 1)}{f(x)} = x
    \end{equation}
    をみたすには、$\mu$が
    \begin{equation}
        \left(1 + \frac{1}{2}\right) \log \left(1 + \frac{1}{x}\right) - 1 = \mu(x) - \mu(x + 1)
    \end{equation}
    をみたすことが必要十分である。
\end{proof}

    \begin{lemma}
        級数$\sum_{n=0}^\infty g(x + n)$は$x > 0$上で各点収束し、
        $x \to \infty$で$0$に収束する。
    \end{lemma}

\begin{proof}
    関数$\frac{1}{2} \log \frac{1+y}{1-y}\; (= \artanh y)$は$|y| < 1$で
    \begin{equation}
        \frac{1}{2} \log \frac{1+y}{1-y}
            = y + \frac{y^3}{3} + \frac{y^5}{5} + \cdots
    \end{equation}
    と収束級数に展開できる。
    右辺に$y = \frac{1}{2x + 1}$を代入することで
    \begin{equation}
        \begin{split}
            g(x)
                &= \left(x + \frac{1}{2}\right) \log \left(1 + \frac{1}{x}\right) - 1 \\
                &= (2x + 1) \frac{1}{2} \log \left(1 + \frac{1}{x}\right) - 1 \\
                &= (2x + 1) \sum_{n=0}^\infty \frac{1}{(2n + 1) (2x + 1)^{2n + 1}} - 1 \\
                &= \sum_{n=1}^\infty \frac{1}{(2n + 1) (2x + 1)^{2n}}
        \end{split}
    \end{equation}
    を得る。したがって
    \begin{equation}
        \begin{split}
            0 < g(x) &< \frac{1}{3} \sum_{n=1}^\infty \frac{1}{(2x + 1)^{2n}} \\
                &= \frac{1}{3} \left(\frac{1}{2x + 1}\right)^2
                    \frac{1}{1 - \left(\frac{1}{2x + 1}\right)^2} \\
                &= \frac{1}{12x(x+1)} \\
                &= \frac{1}{12x} - \frac{1}{12(x+1)}
        \end{split}
    \end{equation}
    である。よって$\forall N \in \N$に対し
    \begin{equation}
        \begin{split}
            0 < \sum_{n=0}^N g(x + n)
                &< \sum_{n=0}^N \left\{ \frac{1}{12(x + n)} - \frac{1}{12(x+n+1)} \right\} \\
                &\le \sum_{n=0}^\infty \left\{ \frac{1}{12(x + n)} - \frac{1}{12(x+n+1)} \right\} \\
                &= \frac{1}{12x}
        \end{split}
    \end{equation}
    である。したがって部分和$\sum_{n=0}^N g(x + n)$は上に有界な正数列なので
    $\sum_{n=0}^\infty g(x + n)$は収束し、
    上の不等式から$x \to \infty$で$0$に収束することも示せた。
\end{proof}

\begin{lemma}
    $\mu$を以下のように定めれば\cref{11:lem:1}の条件が達成される。
    \begin{equation}
        \begin{split}
            \mu(x) &= \sum_{k=0}^\infty g(x + k)
        \end{split}
    \end{equation}
    \label{11:lem:2}
\end{lemma}

\begin{proof}
    級数$\sum_{k=0}^\infty g(x + k)$が収束することから
    \begin{equation}
        \begin{split}
            \mu(x) - \mu(x + 1)
                &= \sum_{k=0}^\infty g(x + k) - \sum_{k=0}^\infty g(x + k + 1) \\
                &= \sum_{k=0}^\infty (g(x + k) - g(x + k + 1)) \\
                &= \lim_{N \to \infty} \sum_{k=0}^N (g(x + k) 0 g(x + k + 1)) \\
                &= \lim_{N \to \infty} (g(x) - g(x + N + 1)) \\
                &= g(x)
        \end{split}
    \end{equation}
    である。
\end{proof}

\begin{lemma}
    定義式(\ref{11:eq:7})と\cref{11:lem:2}の$\mu$によって定まる$f$は
    条件(2)をみたす。
    \label{11:lem:3}
\end{lemma}

\begin{proof}
    $f$の定義式から明らかに$f(x) > 0$である。
    また
    \begin{equation}
        \log f(x) = \left(x - \frac{1}{2}\right) \log x - x + \mu(x)
    \end{equation}
    であり、右辺の$\mu(x)$を除く項は凸であるから、
    $\log f$が凸であることを示すには$\mu$が凸であることをいえばよい。
    そのためには$g$が凸であることをいえば充分だが、
    \begin{equation}
        g''(x) = \frac{1}{2x^2 (x+1)^2} > 0
    \end{equation}
    なので$g$も$x > 0$で凸である。
    したがって$f$は条件(2)をみたす。
\end{proof}

    \begin{lemma}[Wallis の公式]
        \begin{equation}
            \sqrt{\pi} = \lim_{n \to \infty} \frac{(2n)!!}{(2n - 1)!! \sqrt{n}}
        \end{equation}
    \end{lemma}

\begin{proof}
    長いので省略\footnote{
        参考文献\cite[第IV章 定理15.6系]{杉浦80}を参照。
    }
\end{proof}

    \begin{theorem}[Stirling の公式]
        $x > 0$に対し
        \begin{equation}
            \Gamma(x) = \sqrt{2\pi} x^{x - \frac{1}{2}} e^{-x} e^{\mu(x)}
        \end{equation}
        が成り立つ。
        ただし$\mu$は\cref{11:lem:2}で定めたものである。
    \end{theorem}

\begin{proof}
    \cref{11:lem:2}と\cref{11:lem:3}より、
    定義式(\ref{11:eq:7})と$\mu$によって定まる関数$f$は
    Bohr-Mollerup の定理(\cref{11:thm:2})の条件 (1), (2) をみたす。
    したがって\cref{11:cor:1}より
    \begin{equation}
        \Gamma(x) = a f(x) \quad (x > 0)
    \end{equation}
    となる定数$a > 0$が存在する。$a$は Wallis の公式によって求めることができ、
    $n! = \Gamma(n+1) = n\Gamma(n) = anf(n)$に注意すれば
    \begin{equation}
        \begin{split}
            \sqrt{\pi}
                &= \lim_{n \to \infty} \frac{(2n)!!}{(2n - 1)!! \sqrt{n}} \\
                &= \lim_{n \to \infty} \frac{2^{2n} (n!)^2}{(2n)! \sqrt{n}} \\
                &= \lim_{n \to \infty} \frac{2^{2n} (an f(n))^2}{a \cdot 2n f(2n) \sqrt{n}} \\
                &= \lim_{n \to \infty} \frac{2^{2n} a^2 n^2 \left(n^{n-1/2} e^{-n} e^{\mu(n)}\right)^2}
                    {2 an (2n)^{2n-1/2} e^{-2n} e^{\mu(2n)} \sqrt{n}} \\
                &= \lim_{n \to \infty} \frac{a}{\sqrt{2}} \frac{e^{2\mu(n)}}{e^{\mu(2n)}} \\
                &= \frac{a}{\sqrt{2}}
        \end{split}
    \end{equation}
    よって$a = \sqrt{2\pi}$である。
    以上より定理の主張が示せた。
\end{proof}



\begin{problem}
    $\Gamma(1/2),\, \Gamma(n + 1/2)$の値を求めよ。

    解答:
    \begin{equation}
        \Gamma(1/2) = \sqrt{\pi},\quad \Gamma(n + 1/2) = \frac{(2n - 1)!!}{2^n} \sqrt{\pi}
    \end{equation}
\end{problem}

\begin{problem}
    $p, q > -1$に対し
    \begin{equation}
        \int_0^{\pi/2} \cos^p \theta \sin^q \theta d\theta
            = \frac{1}{2} B\!\left(\frac{p+1}{2}, \frac{q+1}{2}\right)
    \end{equation}
    を示せ。
\end{problem}

\begin{problem}
    $n$次元球の体積と表面積を$\Gamma$を用いて表わせ。

    解答:
    \begin{equation}
        |B_n(R)| = \frac{\sqrt{\pi} R^n}{\Gamma(n/2 + 1)},\quad
        |\del B_n(R)| = \frac{2 \sqrt{\pi} R^{n-1}}{\Gamma(n/2)}
    \end{equation}
\end{problem}

\begin{problem}
    $0 < m+1 < n$とする。
    \begin{equation}
        I = \int_0^\infty \frac{x^m}{1 + x^n} dx
    \end{equation}
    を Gamma 関数、Beta 関数を用いて表わせ。

    解答:
    \begin{equation}
        I = \frac{1}{n} B\left( \frac{m+1}{n}, 1 - \frac{m+1}{n} \right)
            = \frac{1}{n} \Gamma\left(\frac{m+1}{n}\right) \Gamma\left(1 - \frac{m+1}{n}\right)
    \end{equation}
\end{problem}

\begin{problem}
    \begin{equation}
        \int_0^\infty \frac{t^{y-1}}{1+t^x} dt \quad (x > y > 0)
    \end{equation}
    を$\Gamma$関数で表わせ。

    解答:
    \begin{equation}
        \frac{1}{x} \Gamma\left(1 - \frac{y}{x}\right) \Gamma\left(\frac{y}{x}\right)
    \end{equation}
\end{problem}

\begin{problem}
    \begin{equation}
        \int_0^{\pi/2} \frac{\cos^{2x-1} \theta \sin^{2y - 1} \theta}
            {(a \cos^2\theta + b \sin^2\theta)^{x+y}} d\theta \quad (a, b, x, y > 0)
    \end{equation}
    を$\Gamma$関数で表わせ。

    解答:
    \begin{equation}
        \frac{1}{2 a^x b^y} \frac{\Gamma(x) \Gamma(y)}{\Gamma(x + y)}
    \end{equation}
\end{problem}

\begin{problem}
    \begin{equation}
        \left(\int_0^{\pi/2} \sqrt{\sin \theta} d\theta\right)
        \left(\int_0^{\pi/2} \frac{d\theta}{\sqrt{\sin\theta}}\right)
        = \pi
    \end{equation}
    を示せ。
\end{problem}

\begin{problem}[積分]
    \,
    \begin{itemize}
        \item \cite[第III章 例題8.1]{杉浦+89}
        \item \cite[第III章 問8.1 (1)-(6)]{杉浦+89}
        \item \cite[第III章 問8.2 (1),(2)]{杉浦+89}
        \item \cite[第III章 問8.3 (1)]{杉浦+89}
    \end{itemize}
    を読者の演習問題とする。
\end{problem}

\begin{problem}[Stirling の公式]
    \,
    \begin{itemize}
        \item \cite[第III章 例題8.5]{杉浦+89}
    \end{itemize}
    を読者の演習問題とする。
\end{problem}

\end{document}

% ============================================================
%
% ============================================================
\newpage
\phantomsection
\addcontentsline{toc}{part}{演習問題の解答}
\part*{演習問題の解答}

\includecollection{answers}

% ============================================================
%
% ============================================================
\newpage
\phantomsection
\addcontentsline{toc}{part}{参考文献}
\renewcommand{\bibname}{参考文献}
\markboth{\bibname}{}
\bibliographystyle{amsalpha}
\bibliography{../mybibliography}

% ============================================================
%
% ============================================================
\newpage
\phantomsection
\addcontentsline{toc}{part}{記号一覧}
\printglossary[title={記号一覧}]

% ============================================================
%
% ============================================================
\newpage
\phantomsection
\addcontentsline{toc}{part}{索引}
\printindex

\end{document}