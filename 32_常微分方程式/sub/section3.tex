\documentclass[report]{jlreq}
\usepackage{../../global}
\usepackage{./local}
\subfiletrue
\def\assetspath{../}
%\makeindex
\begin{document}


\chapter{定数係数線型常微分方程式}

定数係数という節見出しだが、以下の「3.1 定義と基本性質」の内容だけは変数係数の場合も含んでいる。
なお、係数だけでなく非斉次項も定数ならば、
$y$を定数関数とおいて解けば特殊解がひとつ得られる。

\section{定義と基本性質}

\begin{definition}[3.1.1 線型]
    \,
    \begin{enumerate}
        \item $n$階常微分方程式が\textbf{線型}であるとは
            \begin{equation}
                y^{(n)} + a_{n-1}(x) y^{(n-1)} + \cdots + a_1(x) y' + a_0(x) y = f(x)
            \end{equation}
            という形をしていること。
        \item $n$個の未知関数を持つ連立1階常微分方程式が\textbf{線型}であるとは
            \begin{equation}
                \by' = A(x) \by + \bm{b}(x)
            \end{equation}
            という形をしていること。
    \end{enumerate}
\end{definition}

(1)は(2)の形に書き直せる。

\subsection{$b(x) \equiv 0$のとき}
$\by' = A(x) \by$の解全体の集合$V$は線型空間をなす。

\begin{theorem}[3.1.2]
    $\{ \bbeta_i \coloneqq \by_i(\alpha) \}_{1 \le i \le n}\, (\alpha \in \R)$が$\R^n$の基底であるとき、
    $\{ \by_i \}_{1 \le i \le n}$は$V$の基底である。
\end{theorem}

\begin{proof}
    線型独立性は明らかなので、$V = \Span \{ \by_i \}_{1 \le i \le n}$を示そう。
    $\by \in V$を任意にとる。$\by(\alpha) \eqqcolon \bbeta \in \R^n$とおくと、
    $\{ \bbeta_i \}_{1 \le i \le n}$が$\R^n$の基底であることから
    \begin{equation}
        \bbeta = c_1 \bbeta_1 + \cdots + c_n \bbeta_n
    \end{equation}
    と書ける。このとき、$\by(x)$と$c_1 \by_1(x) + \cdots c_n \by_n(x)$はいずれも初期値問題
    \begin{equation}
        \by' = A(x) \by,\quad \by(\alpha) = \bbeta
    \end{equation}
    の解なので、解の一意性により
    \begin{equation}
        \by(x) = c_1 \by_1(x) + \cdots c_n \by_n(x)
    \end{equation}
    が成り立つ。
    したがって$V = \Span \{ \by_i \}_{1 \le i \le n}$がいえた。
\end{proof}

\begin{definition}[3.1.3]
    $V$の基底を与える解の組を\textbf{基本解}という。
\end{definition}


\subsection{$b(x) \not\equiv 0$のとき}
解全体の集合を$\tilde{V}$とおく。

\begin{theorem}[3.1.4]
    $\varphi(x)$をひとつの解とすると、
    \begin{equation}
        y(x) \in \tilde{V} \iff y(x) - \varphi(x) \in V
    \end{equation}
    である。
\end{theorem}






\section{定数係数$n$階線型常微分方程式}

定数係数$n$階線型常微分方程式
\begin{equation}
    y^{(n)} + a_{n-1} y^{(n-1)} + \cdots + a_0 y = f(y) \tag{(1)}
\end{equation}
を考える。

\subsection{同次の場合}

\begin{definition}[特性多項式]
    微分演算子を$D \coloneqq \dd{x}$とおくと、$(1)$は
    \begin{equation}
        (D^n + a_{n-1} D^{n-1} + \cdots + a_1 D + a_0) y = 0
    \end{equation}
    と書ける。ここで
    \begin{equation}
        \Phi(t) \coloneqq t^n + a_{n-1} t^{n-1} + \cdots + a_1 t + a_0
    \end{equation}
    を$(1)$の\textbf{特性多項式}という。
\end{definition}

以下、特性多項式の因数分解
\begin{equation}
    \Phi(t) = \prod_{i=1}^r (t - \lambda_i)^{n_i}
\end{equation}
が重要である。この因数分解に応じて(1)は
\begin{equation}
    \prod_{i=1}^r (D - \lambda_i)^{n_i} y = 0 \tag{(3)}
\end{equation}
と書ける。

\begin{lemma}[3.2.1]
    $e^{-\lambda x} (D - \lambda) g = D (e^{-\lambda x} g)$
\end{lemma}

\begin{proof}
    簡単なので省略。
\end{proof}

\begin{theorem}[3.2.2]
    $\{ x^j e^{\lambda_i x} \mid 1 \le i \le r,\, 0 \le j \le n_i - 1 \}$は
    (3)の基本解である。
\end{theorem}

\begin{proof}
    任意の解が所与の関数系の線型結合で書けることを示せばよい。
    $r = 1$すなわち特性多項式の根がひとつしかない場合は、補題3.2.1から簡単な計算によりわかる。
    $r > 1$の場合は$r$に関する帰納法によって示す。
\end{proof}

同次の場合の解法は至ってシンプルである。
すなわち、定理3.2.2によって特性多項式の根に対応する基本解がただちに得られる。

\begin{problem}[3.2.4]
    $y''' - y'' - y' + y = 0$を解け。

    解答:
    \begin{equation}
        y(x) = C_1 e^x + C_2 x e^x + C_3 e^{-x}
    \end{equation}
\end{problem}

\begin{problem}[教科書例題3.4]
    $y'' + \omega^2 y = 0\; (\omega > 0)$の一般解を実数値関数として求めよ。
    定理3.2.2から得られる基底$e^{i\omega x},\, e^{-i\omega x}$は複素数値関数なので、
    これらの線型結合によって基底を$\cos \omega x,\, \sin \omega x$と取り直せば
    実数値関数としての一般解が得られる。

    解答:$y = C_1 \cos \omega x + C_2 \sin \omega x$
\end{problem}

\begin{problem}
    教科書の問3.1, 問3.2, 問3.3, 問3.4を読者の演習問題とする。
\end{problem}


\subsection{非同次の場合}

非同次の場合は\textbf{演算子法}によって解く。
これは兎にも角にもひとつの解$\varphi(x)$を求めるのが基本的である。
そこで特性多項式を$\Phi(t) = (t - \lambda_1) \Psi(t)$と変形すると、
補題3.2.1より
\begin{equation}
    \begin{split}
        \Phi(D) \varphi = f(x)
            &\iff e^{-\lambda_1 x} (D - \lambda_1) \Psi(D) \varphi = e^{-\lambda_1 x} f(x) \\
            &\iff \Psi(D) \varphi = e^{\lambda_1 x} \int e^{-\lambda_1 x} f(x) dx
    \end{split}
\end{equation}
となり、階数のひとつ小さい微分方程式が得られる\footnote{
    イメージとしては、$(D - \lambda_1) \Psi(D) \phai = f(x)$を見たら
    右辺の式から$e^{\lambda_1 x}$を「引き剥がして」おいて残りを積分するという感じ。
}。これを繰り返して$\varphi(x)$を求める。
あとは$\varphi(x)$に同次の一般解を足し合わせれば、それが非同次の解となる。

また、非斉次項が特別な形の場合は\textbf{未定係数法}も利用できる。

\begin{problem}[3.2.4]
    $y'' + 3y' + 2y = e^x$を解け。
\end{problem}

まず$\varphi$を求めると
\begin{equation}
    \begin{split}
        (D + 1)(D + 2) \varphi &= e^x \\
        (D + 2) \varphi &= \frac{1}{2} e^x \\
        \varphi &= \frac{1}{6} e^x \\
    \end{split}
\end{equation}
なので、一般解は
\begin{equation}
    y(x) = \frac{1}{6} e^x + C_1 e^{-x} + C_2 e^{-2x}
\end{equation}
である。

\begin{problem}
    教科書の問3.5を読者の演習問題とする。
\end{problem}






\section{定数係数連立1階線型常微分方程式}

以下では定数係数の連立1階型を考える。
これは一見すると$n$階型を行列の言葉で言い換えただけに過ぎないようにも見えるが、
同次形の初期値問題が簡単に解けるという利点がある。

\begin{proposition}[3.3.1 行列の指数関数]
    \begin{enumerate}
        \item $n$次正方行列$A$に対し
            \begin{equation}
                \sum_{n=0}^\infty \frac{1}{n!} A^n \eqqcolon e^{A}
            \end{equation}
            は収束する。
        \item $AB = BA$ならば$e^A e^B = e^{A+B}$が成り立つ。
        \item 行列値関数$e^{xA}$は$C^1$級(実は{\smooth}級)で
            \begin{equation}
                \dd{x} e^{xA} = A e^{xA}
            \end{equation}
            が成り立つ。
    \end{enumerate}
\end{proposition}

\begin{proof}
    (1) コーシーの収束条件を使えば示せる。

    (2), (3) 省略
\end{proof}

\begin{theorem}[3.3.2 非同次形の一般解]
    $\bm{y}' = A\bm{y} + \bm{b}(x)$の解は
    \begin{equation}
        \bm{y}(x) = e^{xA} \left( \int^x e^{-tA} \bm{b}(t)\, dt + \bm{C} \right)
    \end{equation}
    である。
\end{theorem}

\begin{corollary}[3.3.3 非同次形の初期値問題]
    さらに初期条件$\bm{y}(\alpha) = \bm{\beta}$をみたす解は
    \begin{equation}
        \bm{y}(x) = e^{xA} \left( \int_\alpha^x e^{-tA} \bm{b}(t)\, dt + e^{-\alpha A} \bm{\beta} \right)
    \end{equation}
    である。
\end{corollary}

\begin{corollary}[3.3.4 同次形の一般解と初期値問題]
    $\bm{y}' = A\bm{y}$の解は
    \begin{equation}
        \bm{y}(x) = e^{xA} \bm{C}
    \end{equation}
    であり、さらに初期条件$\bm{y}(\alpha) = \bm{\beta}$をみたす解は
    \begin{equation}
        \bm{y}(x) = e^{(x - \alpha)A} \bm{\beta}
    \end{equation}
    である。
\end{corollary}

この系3.3.4により、同次形の場合は初期値に指数関数$e^{xA}$を作用させるだけで
解が求まることがわかった。
$e^{xA}$の計算には\textbf{Jordan 標準形}を用いる。
Jordan 標準形とは、通常の対角化における固有値問題$(A - \alpha I)\, \bm{x} = \bm{o}$を
$(A - \alpha I)^n\, \bm{x} = \bm{o}$へ拡張したものに対応する。
Jordan 標準形の求め方は以下の通りである。
\begin{enumerate}
    \item 特性方程式を解き、
    \item 固有値の重複度ぶんだけ
        \begin{equation}
            \begin{split}
                (A - \alpha I)\, \bm{p}_{1} &= \bm{o} \\
                (A - \alpha I)\, \bm{p}_{2} &= \bm{p}_1 \\
                &\vdots
            \end{split}
        \end{equation}
        という計算を繰り返して広義固有ベクトルを求めていく。
\end{enumerate}
これより$e^{xA}$は Jordan 標準形$J$と対角化$\!{}^\text{?}\!$行列$P$を用いて
\begin{equation}
    e^{xA} = P e^{xJ} P^{-1}
\end{equation}
と求まる。このとき、$P$の成分にできるだけ$0$が現れるようにしたほうが後の計算が楽になる。
また、固有値$\alpha$に対応する$m$次の Jordan 細胞$J(\alpha, m)$は
\begin{equation}
    e^{x J(\alpha, m)}
        = \begin{bmatrix}
            e^{\alpha x} & xe^{\alpha x} & \cdots & \tfrac{x^{m-1}}{(m-1)!} e^{\alpha x} \\
             & e^{\alpha x} & & \vdots \\
             & & \ddots & \vdots \\
             & & & e^{\alpha x}
        \end{bmatrix}
\end{equation}
をみたす。この形は記憶しておいたほうがよい。

\begin{problem}[3.3.6]
    次の微分方程式の解を2通りの方法で求めよ。
    \begin{equation}
        \bm{y}' = A\bm{y},\quad
        A = \begin{bmatrix}
            -1 & 2 \\
            -4 & 5
        \end{bmatrix}
    \end{equation}

    1つ目の方法は、素直に$A$の Jordan 標準形を用いるものである。

    2つ目の方法は、$e^{xA}$の各列がそれぞれ初期値$\bm{y}(0) = {}^t (1, 0),\, {}^t (0, 1)$に対応する
    特殊解であることを利用するものである。
    すなわち、$A$の固有多項式の根に対応する基本解(定理3.2.2)の線型結合で
    これら特殊解$\bm{y}(x)$の各成分を表し、
    続いて微分方程式と初期条件から係数を決めるという方法である\footnote{
        板書の注3.3.5を参照
    }。

    解答:
    \begin{equation}
        \bm{y} = \begin{bmatrix}
            2e^x - e^{3x} & -e^x + e^{3x} \\
            2e^x - 2e^{3x} & -e^x + 2e^{3x}
        \end{bmatrix}
        \begin{bmatrix}
            C_1 \\
            C_2
        \end{bmatrix}
    \end{equation}
\end{problem}

\begin{problem}[3.3.7]
    次の微分方程式の解を2通りの方法で求めよ。
    \begin{equation}
        y' = Ay,\quad
        A = \begin{bmatrix}
            1 & 1 \\
            -1 & 3
        \end{bmatrix}
    \end{equation}

    解答:
    \begin{equation}
        \bm{y} = \begin{bmatrix}
            (1-x) e^{2x} & x e^{2x} \\
            -x e^{2x} & (1+x) e^{2x}
        \end{bmatrix}
        \begin{bmatrix}
            C_1 \\
            C_2
        \end{bmatrix}
    \end{equation}
\end{problem}

\begin{problem}[小テスト8]
    次の微分方程式の解を2通りの方法で求めよ。
    \begin{equation}
        y' = Ay
        + \begin{bmatrix}
            1 \\
            -1
        \end{bmatrix},\quad
        A = \begin{bmatrix}
            2 & -3 \\
            1 & -2
        \end{bmatrix}
    \end{equation}
    この微分方程式は係数も非斉次項も定数なので、定数関数を解に持つ。

    解答:
    \begin{equation}
        \bm{y} = \begin{bmatrix}
            -5 \\
            -3
        \end{bmatrix}
        + \frac{1}{2} \begin{bmatrix}
            3 e^{x} - e^{-x} & -3 e^{x} + 3e^{-x} \\
            e^{x} - e^{-x} & -e^x + 3e^{-x}
        \end{bmatrix}
        \begin{bmatrix}
            C_1 \\
            C_2
        \end{bmatrix}
    \end{equation}
\end{problem}

\begin{problem}
    教科書の問3.7を読者の演習問題とする。
\end{problem}

\begin{problem}
    \cite{寺坂10} 第4章例題1-5を読者の演習問題とする。
\end{problem}
\begin{problem}
    \cite{寺坂10} 第4章例題16-17を読者の演習問題とする。
\end{problem}

\end{document}