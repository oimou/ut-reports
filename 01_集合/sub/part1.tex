\documentclass[report]{jlreq}
\usepackage{global}
\usepackage{./local}
\subfiletrue
\def\assetspath{../}
\begin{document}

% ============================================================
%
% ============================================================
\chapter{集合}

集合論の基礎について述べる。

% ------------------------------------------------------------
%
% ------------------------------------------------------------
\section{集合論の公理}

\begin{definition}[集合論の公理系]
    ~
    \begin{description}
        \item[(空集合の存在公理)] 
        \item[(対集合公理)]
        \item[(和集合公理)]
        \item[(冪集合公理)]
        \item[(無限公理)]
        \item[(置換公理)]
        \item[(選択公理)]
    \end{description}
\end{definition}

% ------------------------------------------------------------
%
% ------------------------------------------------------------
\section{集合の演算}

\TODO{圏論的な扱い?}

\begin{definition}[識別和と非交和]
    \TODO{}
\end{definition}

% ------------------------------------------------------------
%
% ------------------------------------------------------------
\section{関係}

\TODO{}

% ------------------------------------------------------------
%
% ------------------------------------------------------------
\section{順序}

\begin{definition}[半順序]
    \TODO{}
\end{definition}

\begin{definition}[全順序]
    \TODO{}
\end{definition}

\begin{definition}[上界と下界]
    \TODO{}
\end{definition}

\begin{definition}[上限と下限]
    \TODO{}
\end{definition}

有向集合を定義する。
有向集合の概念は、位相空間論におけるネットの定義や
代数学における帰納極限などの定義に用いられる。

\begin{definition}[有向集合]
    集合$X$と$X$上の2項関係$\preceq$の組$(X, \preceq)$が
    \term{有向集合}[directed set]
        {有向集合}[ゆうこうしゅうごう]
    であるとは、次が成り立つことをいう:
    \begin{description}
        \item[(D1)] (反射律)
        \item[(D2)] (推移律)
        \item[(D3)] (共通上界の存在)
    \end{description}
    \TODO{}
\end{definition}

\begin{definition}[帰納的半順序集合]
    半順序集合$X$が
    \term{帰納的半順序集合}{帰納的半順序集合}[きのうてきはんじゅんじょしゅうごう]
    であるとは、
    $X$の任意の全順序部分集合が
    $X$内に上界を持つことをいう。
\end{definition}

\begin{theorem}[Zorn の補題]
    非空な帰納的半順序集合は極大元を持つ。
\end{theorem}

\begin{proof}
    \TODO{}
\end{proof}

% ------------------------------------------------------------
%
% ------------------------------------------------------------
\section{束}

束を定義する。
束には2通りの同値な定義が存在する。
まずは順序集合論的な定義を与える。

\begin{definition}[束 (順序集合論的な定義)]
    半順序集合$(L, \preceq)$であって
    任意の$a, b \in L$が最小上界$a \vee b$と最大下界$a \wedge b$を持つものを
    \term{束}[lattice]{束}[そく]という。
\end{definition}

\begin{definition}[束 (代数的な定義)]
    \TODO{}
\end{definition}

% ------------------------------------------------------------
%
% ------------------------------------------------------------
\section{基数と順序数}

\TODO{}




\end{document}