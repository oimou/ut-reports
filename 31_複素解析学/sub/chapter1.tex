\documentclass[report]{jlreq}
\usepackage{../../global}
\usepackage{./local}
\subfiletrue
\def\assetspath{../}
\begin{document}

% ============================================================
%
% ============================================================
\chapter{解析関数}

% ------------------------------------------------------------
%
% ------------------------------------------------------------
\section{Cauchy-Riemann 方程式}

% ------------------------------------------------------------
%
% ------------------------------------------------------------
\section{孤立特異点と留数}

\begin{definition}[孤立特異点]
    \TODO{}
\end{definition}

\begin{definition}[留数]
    \begin{equation}
        \Res_{z = a} f(z) \coloneqq
            \frac{1}{2\pi i}
            \int_{C(a; r)} f(\zeta) \, d\zeta
    \end{equation}
    \TODO{}
\end{definition}

極における留数は、Laurent 展開を求めなくても計算できる。

\begin{proposition}[極における留数]
    \begin{equation}
        \Res_{z = a} f(z)
            = \frac{1}{(n - 1)!}
            \lim_{\substack{z \to a \\ z \neq a}}
            \frac{d^{n - 1}}{dz^{n - 1}} ((z - a)^n f(z))
    \end{equation}
    \TODO{}
\end{proposition}

\begin{proof}
    \TODO{}
\end{proof}

% ------------------------------------------------------------
%
% ------------------------------------------------------------
\newpage
\section{演習問題}

\begin{problem}[ChatGPT]
    関数$f(z) = \frac{1}{z^2 + 1}$の$z = i$における留数を求めよ。
\end{problem}

\begin{answer}
    $f(z) = \frac{1}{(z + i)(z - i)}$と表せるから、
    $z = i$は$f$の1位の極である。
    したがって、極における留数の公式より
    \begin{equation}
        \Res_{z = i} f(z)
            = \lim_{\substack{z \to i \\ z \neq i}}
                (z - i) f(z)
            = \lim_{\substack{z \to i \\ z \neq i}}
                \frac{1}{z + i}
            = \frac{1}{2i}
            = - \frac{1}{2} i
    \end{equation}
    である。
\end{answer}



\end{document}