\documentclass[report]{jlreq}
\usepackage{global}
\usepackage{./local}
\subfiletrue
\def\assetspath{../}
%\makeindex
\chead{2023/06/13}
\begin{document}

% ============================================================
%
% ============================================================

% ------------------------------------------------------------
%
% ------------------------------------------------------------
\section*{振り返りと導入}

\begin{itemize}
    \item \TODO{}
\end{itemize}

% ------------------------------------------------------------
%
% ------------------------------------------------------------
\section{指数型分布族の構造}

$\calP$には自然な多様体構造が定まることをみる。

\begin{propdef}
    指数型分布族$\calP$に関し、次は同値である:
    \begin{enumerate}
        \item ある最小次元実現$(V, T, \mu)$に対し、
            $\Theta^\calP$は$V^\vee$で開である。
        \item すべての最小次元実現$(V, T, \mu)$に対し、
            $\Theta^\calP$は$V^\vee$で開である。
    \end{enumerate}
    $\calP$がこれらの同値な2条件をみたすとき、
    $\calP$は\termsilent{開}[open]であるという。
\end{propdef}

\begin{proof}
    (1) $\Rightarrow$ (2)は
    \cref{thm:transformation-between-representations}より従う。
    (2) $\Rightarrow$ (1)は
    最小次元実現が存在することから従う。
\end{proof}

\begin{propdef}
    $\calP$は開であるとする。
    $\calP$の最小次元実現$(V, T, \mu)$をひとつ選ぶと、
    自然パラメータ付け$P$により、
    $\calP$上に多様体構造と平坦アファイン接続を定めることができる。
    この多様体構造および平坦アファイン接続は
    最小次元実現のとり方によらない。
    これを$\calP$の
    \termsilent{自然な多様体構造}
    および
    \termsilent{自然な平坦アファイン接続}
    と呼ぶ。
\end{propdef}

\begin{proof}
    下の (多様体の圏における) 図式の可換性と、
    $P^{-1}, P'^{-1}$が diffeo. であること、
    $\up{t}L$が線型同型であることから従う。
    \begin{equation}
        \begin{tikzcd}
            (\calP, \calU_{(V, T, \mu)})
                \ar{d}[swap]{P^{-1}}
                \ar{r}{\id}
                &(\calP, \calU_{(V', T', \mu')})
                    \ar{d}{P'^{-1}}
                \\
            \Theta^\calP
                \ar{r}{\up{t}L}
                &\Theta'^\calP
        \end{tikzcd}
    \end{equation}
\end{proof}

\begin{propdef}[$\calP$の Fisher 計量]
    $\calP$は開であるとする。
    $\calP$の最小次元実現$(V, T, \mu)$をひとつ選ぶと、
    $\Theta^\calP$上の Fisher 計量$g$を
    $\theta$で引き戻して
    $\calP$上の Riemann 計量
    $\theta^* g$が定まる。
    この計量は最小次元実現のとり方によらない。
    これを$\calP$上の
    \termsilent{Fisher 計量}
    と呼ぶ。
\end{propdef}

\begin{proof}
    期待値と分散のペアリングの命題と同様の議論により、
    各$p \in \calP$に対し
    $g_{\theta(p)} = (L \otimes L) g'_{\theta'(p)}$
    が成り立つ。

    示すべきことは
    $\theta^* g = \theta'^* g'$が成り立つことである。
    各$p \in \calP, \; u, v \in T_p\calP$に対し
    \begin{alignat}{1}
        (\theta^* g)_p(u, v)
            &=
                g_{\theta(p)} (d\theta_p(u), d\theta_p(v))
                \\
            &=
                \myangle{
                    g_{\theta(p)}
                }{
                    d\theta_p(u) \otimes d\theta_p(v)
                }
                \\
            &=
                \myangle{
                    (L \otimes L) g'_{\theta'(p)}
                }{
                    d\theta_p(u) \otimes d\theta_p(v)
                }
                \\
            &=
                \myangle{
                    g'_{\theta'(p)}
                }{
                    \up{t}L \circ d\theta_p(u) \otimes \up{t}L \circ d\theta_p(v)
                }
                \\
            &=
                \myangle{
                    g'_{\theta'(p)}
                }{
                    d(\up{t}L \circ \theta)_p (u) \otimes d(\up{t}L \circ \theta)_p (v)
                }
                \\
            &=
                \myangle{
                    g'_{\theta'(p)}
                }{
                    d\theta'_p (u) \otimes d\theta'_p (v)
                }
                \\
            &=
                g'_p (d\theta'_p(u), d\theta'_p(v))
                \\
            &=
                (\theta'^* g')_p(u, v)
    \end{alignat}
    が成り立つ。
\end{proof}

% ------------------------------------------------------------
%
% ------------------------------------------------------------
\section{Amari-Chentsov テンソル}

Amari-Chentsov テンソルを定義する。

\begin{propdef}[Amari-Chentsov テンソル]
    $\calP$は開であるとし、
    $D$を$\Theta^\calP$上の標準的な\TODO{とは?}平坦アファイン接続、
    $\theta^i \; (i = 1, \dots, m)$を
    $\Theta^\calP$上の$D$-アファイン座標とする。
    $\Theta^\calP$上の写像
    $S \colon \Theta^\calP \to T^{(0, 3)} \Theta^\calP$
    を
    \begin{equation}
        \theta
            \mapsto
                S_\theta
                \coloneqq
                E_{P(\theta)}[
                    (T - E_{P(\theta)}[T])^{\otimes 3}
                ]
    \end{equation}
    で定めると\TODO{値域あってる?}、
    座標$\theta^i$に関する$S$の成分表示は
    \begin{equation}
        S
            = \del_i \del_j \del_k \psi \, d\theta^i d\theta^j d\theta^k
    \end{equation}
    となる。
    とくに
    $\psi$の\smooth 性より
    $S$は対称$(0, 3)$-テンソル場となる。
    $S$を
    \term{Amari-Chentsov テンソル}[Amari-Chentsov tensor]
        {Amari-Chentsov テンソル}[Amari-Chentsov テンソル]
    と呼ぶ。
\end{propdef}

\begin{proof}
    成分表示は
    \url{0516_資料.pdf}の系2.4を使って式変形すればわかる。
\end{proof}

\begin{propdef}[$\calP$の Amari-Chentsov テンソル]
    $\calP$は開であるとする。
    $\calP$の最小次元実現$(V, T, \mu)$をひとつ選ぶと、
    $\Theta^\calP_{(V, T, \mu)}$上の Amari-Chentsov テンソル$S$を
    $\theta$で引き戻して
    $\calP$上の対称$(0, 3)$-テンソル場
    $\theta^* S$が定まる。
    このテンソル場は最小次元実現のとり方によらない。
    これを$\calP$上の
    \term{Amari-Chentsov テンソル}[Amari-Chentsov tensor]
        {Amari-Chentsov テンソル}[Amari-Chentsov テンソル]
    と呼ぶ。
\end{propdef}

\begin{proof}
    $S$の定義より
    $L^{\otimes 3} S'_{\theta'} = S_\theta$
    が成り立つから、
    Fisher 計量の場合と同様の議論により
    $\theta^* S = \theta'^* S'$が成り立つ。
\end{proof}


% ------------------------------------------------------------
%
% ------------------------------------------------------------
\section{期待値パラメータ空間}

\begin{definition}[期待値パラメータ空間]
    集合$\calM_{(V, T, \nu)}$
    \begin{equation}
        \calM_{(V, T, \nu)}
            \coloneqq \mybrace{
                \mu \in V
                \mid
                \exists \;
                p \colon \text{$\calX$上の確率分布}
                \; \text{s.t.} \;
                p \ll \nu, \;
                E_p[T] = \mu
            }
    \end{equation}
    を$(V, T, \nu)$の
    \term{期待値パラメータ空間}[mean parameter space]
        {期待値パラメータ空間}[きたいちぱらめーたくうかん]
    という。
\end{definition}

期待値パラメータ空間$\calM$は、
$\calP$に属する確率分布に関する$T$の期待値をすべて含んでいる
(一般には真に含んでいる)。

\begin{proposition}
    $\mu \in V$が
    ある$p \in \calP$に関する
    $T$の期待値ならば (すなわち$\mu = E_p[T]$ならば)、
    $\mu$は$\calM_{(V, T, \nu)}$に属する。
\end{proposition}

\begin{proof}
    \TODO{}
\end{proof}

\begin{proposition}[$\calM$は凸集合]
    $\calM_{(V, T, \nu)}$は$V$の凸集合である。
\end{proposition}

\begin{proof}
    \TODO{}
\end{proof}


% ------------------------------------------------------------
%
% ------------------------------------------------------------
\section{Fisher 計量}

\begin{example}[正規分布族]
    \TODO{ちゃんと書く}
    $\calP$を$\calX = \R$上の正規分布族とし、
    実現$(V, T, \mu)$を
    $V = \R^2, \;
        T(x) = (x, x^2), \;
        \mu = \lambda$
    とおく。
    これは条件Aをみたす。

    自然パラメータ空間は
    $\Theta = \Theta^\circ = \R \times \R_{< 0}$である。

    対数分配関数は
    \begin{equation}
        \psi(\theta)
            = \frac{\mu^2}{2 \sigma^2}
            + \log \sigma
            + \frac{1}{2} \log 2\pi
    \end{equation}
    である。
    ただし$\theta^1 = \frac{\mu}{\sigma^2}, \;
        \theta^2 = -\frac{1}{2 \sigma^2}$
    とおいた。
    よって
    \begin{alignat}{1}
        d\psi
            &=
                \frac{\mu}{\sigma^2}
                d\mu
                + \frac{\sigma^2 - \mu^2}{\sigma^3}
                d\sigma
                \\
            &=
                -\frac{\theta^1}{2\theta^2} d\theta^1
                +\myparen{
                    -\frac{1}{2\theta^2}
                    + \frac{(\theta^1)^2}{4(\theta^2)^2}
                }
                d\theta^2
                \\
        \Hess\psi
            &= Dd\psi \\
            &=
                \myparen{
                    -\frac{1}{2\theta^2}
                    d\theta^1
                    + \frac{\theta^1}{2(\theta^2)^2}
                    d\theta^2
                }
                d\theta^1
                +
                \myparen{
                    \frac{\theta^1}{2(\theta^2)^2}
                    d\theta^1
                    + \myparen{
                        \frac{1}{2(\theta^2)^2}
                        - \frac{(\theta^1)^2}{2(\theta^2)^3}
                    }
                    d\theta^2
                }
                d\theta^2
                \\
            &=
                \frac{1}{\sigma^2} (d\mu)^2
                + \frac{2}{\sigma^2} (d\sigma)^2
    \end{alignat}
    である。
    Fisher 計量$g \coloneqq \Hess\psi$から定まる
    Levi-Civita 接続$\nabla^{(g)}$の、
    座標$\mu, \sigma$に関する接続係数を求めてみる。
    \begin{alignat}{2}
        {\Gamma^{(g)}}_{11}^1
            = 0,
            &\qquad
                {\Gamma^{(g)}}_{12}^1
                    = {\Gamma^{(g)}}_{21}^1
                    = -\frac{1}{\sigma},
            &&\qquad
                {\Gamma^{(g)}}_{22}^1
                    = 0,
            \\
        {\Gamma^{(g)}}_{11}^2
            = \frac{1}{2\sigma},
            &\qquad
                {\Gamma^{(g)}}_{12}^2
                    = {\Gamma^{(g)}}_{21}^2
                    = 0,
            &&\qquad
                {\Gamma^{(g)}}_{22}^2
                    = -\frac{1}{\sigma}
    \end{alignat}
    測地線の方程式は
    \begin{equation}
        \begin{cases}
            x'' - \frac{2}{y} x' y' = 0 \\
            y'' + \frac{1}{2y} (x')^2 - \frac{1}{y} (y')^2 = 0
        \end{cases}
    \end{equation}
    である。
    これを直接解くのは少し大変である。
    その代わりに、
    既知の Riemann 多様体との間の等長同型を利用して測地線を求める。
    $(\Theta, g)$は、
    上半平面$H$に計量
    $\breve{g} = \frac{(dx)^2 + (dy)^2}{2y^2}$
    を入れた Riemann 多様体との間に
    等長同型$(\Theta, g) \to (H, \breve{g}), \;
        (x, y) \mapsto (x, \sqrt{2}y)$
    を持つ。
    Levi-Civita 接続に関する測地線は
    等長同型で保たれるから、
    $(H, \breve{g})$の測地線を求めればよい。
    $(H, \breve{g})$の測地線は、
    $y$軸に平行な直線と
    $x$軸上に中心を持つ半円で尽くされることが知られている。
    これらを等長同型で写して、
    $(\Theta, g)$の測地線として
    $y$軸に平行な直線と
    $x$軸上に長軸を持つ半楕円が得られる。
\end{example}


% ------------------------------------------------------------
%
% ------------------------------------------------------------
\section*{今後の予定}

\begin{itemize}
    \item \TODO{}
\end{itemize}

% ------------------------------------------------------------
%
% ------------------------------------------------------------
\section*{参考文献}

\nocite{amari_information_2016}

{
    \renewcommand{\bibsection}{}
    \bibliographystyle{amsalpha}
    \bibliography{./bibliography,../../mybibliography}
}


\end{document}