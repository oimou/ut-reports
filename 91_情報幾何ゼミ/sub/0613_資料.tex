\documentclass[report]{jlreq}
\usepackage{global}
\usepackage{./local}
\subfiletrue
\def\assetspath{../}
%\makeindex
\chead{2023/06/13}
\begin{document}

% ============================================================
%
% ============================================================

% ------------------------------------------------------------
%
% ------------------------------------------------------------
\section*{振り返りと導入}

前回は最小次元実現の間のアファイン変換について調べた。
本稿では次のことを行う:
\begin{itemize}
    \item 指数型分布族自体に構造を入れる。
    \item Amari-Chentsov テンソルおよび$\alpha$-接続を定義する。
\end{itemize}

% ------------------------------------------------------------
%
% ------------------------------------------------------------
\section{指数型分布族の構造}

本節では、
指数型分布族にいくつかの構造を定め、
さらに Amari-Chentsov テンソルおよび$\alpha$-接続の定義を行う。

以降、本節では
$\calX$を可測空間、
$\calP \subset \calP(\calX)$を$\calX$上の指数型分布族とする。
また、
\url{0606_資料.pdf}定義0.1, 0.2で定めた
写像や真パラメータ空間の記号
「$P_{(V, T, \mu)}$」「$\Theta^\calP_{(V, T, \mu)}$」をよく用いる。
さらに次の記号を定義しておく:

\begin{definition}
    $\calP$の最小次元実現$(V, T, \mu)$に対し、
    $P_{(V, T, \mu)}|_{\Theta^\calP_{(V, T, \mu)}}$の
    逆写像$\calP \to \Theta^\calP_{(V, T, \mu)}$を
    $\theta_{(V, T, \mu)}$と書くことにする。
\end{definition}

\subsection{多様体構造と平坦アファイン接続}

\begin{propdef}[$\calP$が開であること]
    指数型分布族$\calP$に関し、次は同値である:
    \begin{enumerate}
        \item ある最小次元実現$(V, T, \mu)$に対し、
            $\Theta^\calP_{(V, T, \mu)}$は$V^\vee$で開である。
        \item すべての最小次元実現$(V, T, \mu)$に対し、
            $\Theta^\calP_{(V, T, \mu)}$は$V^\vee$で開である。
    \end{enumerate}
    $\calP$がこれらの同値な2条件をみたすとき、
    $\calP$は\termsilent{開}[open]であるという。
\end{propdef}

\begin{proof}
    (1) $\Rightarrow$ (2)は、
    \url{0606_資料.pdf}系1.13より、
    最小次元実現の真パラメータ空間がアファイン変換で写り合うことから従う。
    (2) $\Rightarrow$ (1)は
    最小次元実現が存在することから従う。
\end{proof}

以降、本節では$\calP$は開とする。

\begin{propdef}[$\calP$の自然な多様体構造]
    $\calP$上の多様体構造$\calU$であって
    次をみたすものがただひとつ存在する:
    \begin{itemize}
        \item $\calP$の任意の最小次元実現$(V, T, \mu)$に対し、
            $\calU$は全単射$\theta_{(V, T, \mu)}$により
            $\Theta^\calP_{(V, T, \mu)}$から$\calP$上に誘導された多様体構造に一致する。
    \end{itemize}
    この$\calU$を$\calP$の
    \termsilent{自然な多様体構造}
    という。
\end{propdef}

\begin{proof}
    \uline{Step 1: $\calU$の一意性} \quad
    $\calU$の存在を仮定すれば、
    最小次元実現をひとつ選ぶことで
    $\calU$が決まるから、
    $\calU$は一意である。

    \uline{Step 2: $\calU$の存在} \quad
    最小次元実現$(V, T, \mu)$をひとつ選び、
    $\theta \coloneqq \theta_{(V, T, \mu)}$とおき、
    $\theta$により
    $\Theta^\calP_{(V, T, \mu)}$から$\calP$上に誘導された多様体構造を$\calU$とおく。
    この$\calU$が求めるものであることを示せばよい。
    示すべきことは、
    $(V', T', \mu')$を最小次元実現とし、
    $\theta' \coloneqq \theta_{(V', T', \mu')}$とおき、
    $\calU'$を
    $\theta'$により
    $\Theta^\calP_{(V', T', \mu')}$から
    $\calP$上に誘導された多様体構造とするとき、
    恒等写像$\id \colon (\calP, \calU) \to (\calP, \calU')$が
    微分同相となることである。
    これは図式
    \begin{equation}
        \begin{tikzcd}
            (\calP, \calU)
                \ar{d}[swap]{\theta}
                \ar{r}{\id}
                &(\calP, \calU')
                    \ar{d}{\theta'}
                \\
            \Theta^\calP_{(V, T, \mu)}
                \ar{r}[swap]{F}
                &\Theta^\calP_{(V', T', \mu')}
        \end{tikzcd}
    \end{equation}
    の可換性と、
    $\theta, \theta', F$が微分同相であることから従う。
    ただし$F$とは、
    \url{0606_資料.pdf}系1.13より一意に存在する
    アファイン変換$V^\vee \to V'^\vee$の制限である。
\end{proof}

以降、本節では
$\calP$に自然な多様体構造が定まっているものとする。

\begin{propdef}[$\calP$上の自然な平坦アファイン接続]
    \label[propdef]{propdef:natural-flat-connection}
    $\calP$上の平坦アファイン接続$\nabla$であって
    次をみたすものがただひとつ存在する:
    \begin{itemize}
        \item $\calP$の任意の最小次元実現$(V, T, \mu)$に対し、
            $\Theta^\calP_{(V, T, \mu)}$上の
            標準的な平坦アファイン接続を$\wt{\nabla}$とおくと、
            $\nabla$は$\nabla = \theta_{(V, T, \mu)}^* \wt{\nabla}$をみたす。
    \end{itemize}
    この$\nabla$を$\calP$上の
    \termsilent{自然な平坦アファイン接続}
    という。
\end{propdef}

証明には次の補題を用いる。

\begin{lemma}[アファイン変換によるアファイン接続の引き戻し]
    \label[lemma]{lemma:pullback-affine-connection}
    $V, V'$を有限次元$\R$-ベクトル空間、
    $F \colon V \to V'$をアファイン変換、
    $\nabla, \nabla'$をそれぞれ$V, V'$上の標準的な平坦アファイン接続とする。
    このとき
    $F^* \nabla' = \nabla$が成り立つ。
\end{lemma}

\begin{proof}
    資料末尾の付録に記した。
\end{proof}

\begin{proof}[\cref{propdef:natural-flat-connection}の証明]
    \uline{Step 1: $\nabla$の一意性} \quad
    $\nabla$の存在を仮定すれば、
    最小次元実現をひとつ選ぶことで$\nabla$が決まるから、
    $\nabla$は一意である。

    \uline{Step 2: $\nabla$の存在} \quad
    最小次元実現$(V, T, \mu)$をひとつ選び、
    $\theta \coloneqq \theta_{(V, T, \mu)}$、
    $\Theta^\calP_{(V, T, \mu)}$上の
    標準的な平坦アファイン接続を$\wt{\nabla}$、
    $\nabla \coloneqq \theta^* \wt{\nabla}$と定める。
    この$\nabla$が求めるものであることを示せばよい。
    示すべきことは、
    $(V', T', \mu')$を最小次元実現とし、
    $\theta' \coloneqq \theta_{(V', T', \mu')}$、
    $\Theta^\calP_{(V', T', \mu')}$上の
    標準的な平坦アファイン接続を$\wt{\nabla}'$とおくとき、
    $\theta^* \wt{\nabla} = \theta'^* \wt{\nabla}'$が成り立つことである。
    そこで、
    \url{0606_資料.pdf}系1.13より一意に存在する
    アファイン変換$V^\vee \to V'^\vee$を$F$とおくと、
    \begin{alignat}{1}
        \theta'^* \wt{\nabla}'
            &=
                \theta^* F^* \wt{\nabla}'
                \quad
                (\text{$F$と$\theta, \theta'$の関係})
                \\
            &=
                \theta^* \wt{\nabla}
                \quad
                (\text{\cref{lemma:pullback-affine-connection}})
    \end{alignat}
    が成り立つ。
    したがって$\theta^* \wt{\nabla} = \theta'^* \wt{\nabla}'$が示された。
    よって$\nabla$は命題-定義の主張の条件をみたす。
\end{proof}

以降、本節では
$\calP$に自然な平坦アファイン接続$\nabla$が定まっているものとする。

\subsection{Fisher 計量}

\begin{propdef}[$\calP$上の Fisher 計量]
    \label[propdef]{propdef:Fisher-metric}
    $\calP$上の Riemann 計量$g$であって
    次をみたすものがただひとつ存在する:
    \begin{itemize}
        \item $\calP$の任意の最小次元実現$(V, T, \mu)$に対し、
            $\Theta^\calP_{(V, T, \mu)}$上の Fisher 計量を$\wt{g}$とおくと、
            $g = \theta_{(V, T, \mu)}^* \wt{g}$が成り立つ。
    \end{itemize}
    これを$\calP$上の
    \termsilent{Fisher 計量}
    という。
\end{propdef}

証明には次の補題を用いる。

\begin{lemma}
    \label[lemma]{lemma:relationship-between-Fisher-metrics}
    $(V, T, \mu), (V', T', \mu')$を
    $\calP$の最小次元実現とし、
    $\theta \coloneqq \theta_{(V, T, \mu)}, \;
        \theta' \coloneqq \theta_{(V', T', \mu')}$
    とおき、
    $\Theta^\calP_{(V, T, \mu)}, \;
        \Theta^\calP_{(V', T', \mu')}$上の
    Fisher 計量をそれぞれ$g, g'$とおき、
    \url{0606_資料.pdf}定理1.12より一意に存在する
    線型同型写像$V \to V'$を$L$とおく。
    このとき、
    各$p \in \calP$に対し
    $g_{\theta(p)} = (L \otimes L)(g'_{\theta'(p)})$
    が成り立つ。
\end{lemma}

\begin{proof}
    $L$は
    $T'(x) = L(T(x)) + \text{const.} \; \text{$\mu$-a.e.$x$}$
    をみたし、
    また
    各$p \in \calP$に対し
    $g_{\theta(p)} = \Var_{p}[T], \;
        g'_{\theta'(p)} = \Var_{p}[T']$
    が成り立つから、
    期待値と分散のペアリングの命題
    (\url{0523_資料.pdf}命題1.1)
    と同様の議論により補題の主張の等式が成り立つ。
\end{proof}

\begin{proof}[\cref{propdef:Fisher-metric}の証明]
    \uline{Step 1: $g$の一意性} \quad
    $g$の存在を仮定すれば、
    最小次元実現をひとつ選ぶことで
    $g$が決まるから、
    $g$は一意である。

    \uline{Step 2: $g$の存在} \quad
    最小次元実現$(V, T, \mu)$をひとつ選び、
    $\theta \coloneqq \theta_{(V, T, \mu)}$、
    $\Theta^\calP_{(V, T, \mu)}$上の Fisher 計量を$\wt{g}$とおき、
    $g \coloneqq \theta^* \wt{g}$と定める。
    この$g$が求めるものであることを示せばよい。
    示すべきことは、
    $(V', T', \mu')$を最小次元実現とし、
    $\theta' \coloneqq \theta_{(V', T', \mu')}$、
    $\Theta^\calP_{(V', T', \mu')}$上の Fisher 計量を$\wt{g}'$とおいて、
    $\theta^* g = \theta'^* g'$が成り立つことである。
    そこで
    \url{0606_資料.pdf}定理1.12より一意に存在する
    線型同型写像$V \to V'$を$L$とおくと、
    各$p \in \calP, \; u, v \in T_p\calP$に対し
    \begin{alignat}{1}
        (\theta^* g)_p(u, v)
            &=
                g_{\theta(p)} (d\theta_p(u), d\theta_p(v))
                \\
            &=
                \myangle{
                    g_{\theta(p)}
                }{
                    d\theta_p(u) \otimes d\theta_p(v)
                }
                \\
            &=
                \myangle{
                    (L \otimes L) g'_{\theta'(p)}
                }{
                    d\theta_p(u) \otimes d\theta_p(v)
                }
                \quad
                (\cref{lemma:relationship-between-Fisher-metrics})
                \\
            &=
                \myangle{
                    g'_{\theta'(p)}
                }{
                    \up{t}L \circ d\theta_p(u) \otimes \up{t}L \circ d\theta_p(v)
                }
                \\
            &=
                \myangle{
                    g'_{\theta'(p)}
                }{
                    d(\up{t}L \circ \theta)_p (u) \otimes d(\up{t}L \circ \theta)_p (v)
                }
                \\
            &=
                \myangle{
                    g'_{\theta'(p)}
                }{
                    d\theta'_p (u) \otimes d\theta'_p (v)
                }
                \quad
                (\text{$L$と$\theta, \theta'$の関係})
                \\
            &=
                g'_p (d\theta'_p(u), d\theta'_p(v))
                \\
            &=
                (\theta'^* g')_p(u, v)
    \end{alignat}
    が成り立つ。
    したがって$\theta^* g = \theta'^* g'$が示された。
    よって$g$は命題-定義の主張の条件をみたす。
\end{proof}

以降、本節では
$\calP$に Fisher 計量$g$が定まっているものとする。

\subsection{Amari-Chentsov テンソルと$\alpha$-接続}

\begin{definition}[Amari-Chentsov テンソル]
    $\calP$上の$(0, 3)$-テンソル場$S$を
    $S \coloneqq \nabla g$で定め、
    これを$\calP$上の
    \term{Amari-Chentsov テンソル}[Amari-Chentsov tensor]
        {Amari-Chentsov テンソル}[Amari-Chentsov テンソル]
    という。
    また、
    $\calP$上の$(1, 2)$-テンソル場$A$を
    次の関係式により定める:
    \begin{equation}
        g(A(X, Y), Z)
            = S(X, Y, Z)
            \quad
            (\forall X, Y, Z \in \Gamma(T\calP))
    \end{equation}

    以降、「Amari-Chentsov テンソル」を
    「ACテンソル」と略記することがある。
\end{definition}

以降、本節では
$\calP$に Amari-Chentsov テンソル$S$が定まっているものとする。

\begin{proposition}[ACテンソルの成分]
    \label[proposition]{prop:components-of-AC-tensor}
    $(V, T, \mu)$を$\calP$の最小次元実現、
    $\Theta^\calP \coloneqq \Theta^\calP_{(V, T, \mu)}, \;
        \theta \coloneqq \theta_{(V, T, \mu)}$、
    $(V, T, \mu)$の対数分配関数を$\psi$とおく。
    このとき、
    $\calP$上の任意の
    $\nabla$-アファイン座標
    $x \coloneqq (x^1, \dots, x^m) \colon \calP \to \R^m$に対し、
    $\varphi \coloneqq (\varphi^1, \dots, \varphi^m)
        \coloneqq x \circ \theta^{-1} \colon \Theta^\calP \to \R^m$とおくと、
    $S$の成分は
    \begin{equation}
        S_{ijk}(p)
            = \frac{
                \del^3 \psi
            }{
                \del \varphi^i \del \varphi^j \del \varphi^k
            }(\theta(p))
            = E_p\mybracket{
                (T_i - E_p[T_i])
                (T_j - E_p[T_j])
                (T_k - E_p[T_k])
            }
    \end{equation}
    をみたす。
    ただし$T_i \; (i = 1, \dots, m)$とは、
    同一視$V = V^{\vee\vee} = T^\vee_{\theta(p)} \Theta^\calP$により
    $d\varphi^i \; (i = 1, \dots, m)$を$V$の基底とみなしたときの
    $T$の成分である。
\end{proposition}

\begin{proof}
    左側の等号と右側の等号についてそれぞれ示す。

    \uline{Step 1: 左側の等号} \quad
    $\Theta^\calP$上の標準的な平坦アファイン接続を$\wt{\nabla}$とおき、
    $\psi$の定める$\Theta^\calP$上の Fisher 計量を$\wt{g}$とおくと、
    \begin{alignat}{1}
        S\myparen{
            \deldel{x^i},
            \deldel{x^j},
            \deldel{x^k}
        }
            &=
                \myparen{
                    \nabla_{\deldel{x^i}}
                    g
                }
                \myparen{
                    \deldel{x^j},
                    \deldel{x^k}
                }
                \\
            &=
                \myparen{
                    \myparen{
                        \theta^* \wt{\nabla}
                    }_{\deldel{x^i}}
                    \myparen{
                        \theta^* \wt{g}
                    }
                }
                \myparen{
                    \deldel{x^j},
                    \deldel{x^k}
                }
                \\
            &=
                \myparen{
                    \theta^{-1}_*
                    \myparen{
                        \wt{\nabla}_{\theta_* \deldel{x^i}}
                        \wt{g}
                    }
                }
                \myparen{
                    \deldel{x^j},
                    \deldel{x^k}
                }
                \\
            &=
                \myparen{
                    \wt{\nabla}_{\theta_* \deldel{x^i}}
                    \wt{g}
                }
                \myparen{
                    d\theta\myparen{
                        \deldel{x^j}
                    },
                    d\theta\myparen{
                        \deldel{x^k}
                    }
                }
                \\
            &=
                \myparen{
                    \wt{\nabla}_{\deldel{\varphi^i}}
                    \wt{g}
                }
                \myparen{
                    \deldel{\varphi^j},
                    \deldel{\varphi^k}
                }
                \\
            &=
                \myparen{
                    \deldel{\varphi^i}
                    \myparen{
                        \frac{\del^2 \psi}{\del \varphi^l \del \varphi^n}
                    }
                    d\varphi^l d\varphi^n
                }
                \myparen{
                    \deldel{\varphi^j},
                    \deldel{\varphi^k}
                }
                \quad
                (\text{$\varphi$は$\wt{\nabla}$-アファイン座標})
                \\
            &=
                \frac{\del^3 \psi}{\del \varphi^i \del \varphi^j \del \varphi^k}
    \end{alignat}
    となるから、命題の主張の左側の等号が従う。

    \uline{Step 2: 右側の等号} \quad
    「$E_p$」の下付きの$p$を省略して書けば、
    直接計算より
    \begin{alignat}{1}
        &\phantom{=}
            E[(T_i - E[T_i])(T_j - E[T_j])(T_k - E[T_k])]
            \\
        &=
            E[T_i T_j T_k]
            - E[T_i] E[T_j T_k]
            - E[T_j] E[T_k T_i]
            - E[T_k] E[T_i T_j]
            + 2 E[T_i] E[T_j] E[T_k]
            \locallabel{eq:1}
    \end{alignat}
    が成り立つ。
    一方、
    $\lambda \coloneqq \exp \psi$とおき、
    $\deldel{\varphi^i}$を$\del_i$と略記すれば、
    直接計算より
    \begin{alignat}{1}
        \frac{\del^3 \psi}{\del \varphi^i \del \varphi^j \del \varphi^k}
            &=
                \del_i
                \del_j
                \del_k
                \log \lambda
                \\
            &=
                \frac{\del_i \del_j \del_k \lambda}{\lambda}
                - \frac{(\del_i \lambda) (\del_j \del_k \lambda)}{\lambda^2}
                - \frac{(\del_j \lambda) (\del_k \del_i \lambda)}{\lambda^2}
                - \frac{(\del_k \lambda) (\del_i \del_j \lambda)}{\lambda^2}
                + 2 \frac{(\del_i \lambda) (\del_j \lambda) (\del_k \lambda)}{\lambda^3}
    \end{alignat}
    が成り立つ。
    この右辺を
    \url{0516_資料.pdf}系2.4により期待値の形で表せば
    式\localcref{eq:1}に一致するから、
    命題の主張の右側の等号が従う。
\end{proof}

\begin{definition}[$\alpha$-接続]
    $\alpha \in \R$とする。
    $\calP$上のアファイン接続$\nabla^{(\alpha)}$を
    次の関係式により定める:
    \begin{equation}
        g(\nabla^{(\alpha)}_X Y, Z)
            =
                g(\nabla^{(g)}_X Y, Z)
                -
                \frac{\alpha}{2} S(X, Y, Z)
                \qquad
                (X, Y, Z \in \Gamma(T\calP))
    \end{equation}
    この$\nabla^{(\alpha)}$を
    $(g, S)$の定める
    \term{$\alpha$-接続}[$\alpha$-connection]
        {$\alpha$-接続}[alphaせつぞく]
    という。
    とくに$\alpha = 1, -1$の場合をそれぞれ
    \term{e-接続}[e-connection]
        {e-接続}[eせつぞく]、
    \term{m-接続}[m-connection]
        {m-接続}[mせつぞく]
    という。
\end{definition}

\begin{proposition}[$\nabla^{(g)}, \nabla^{(\alpha)}$のACテンソルによる表示]
    \label[proposition]{prop:connections_by_AC_tensor}
    $\calP$上の任意の$\nabla$-アファイン座標に関し、
    $\nabla^{(g)}$および$\nabla^{(\alpha)}$の
    接続係数は次をみたす:
    \begin{enumerate}
        \item
            \begin{equation}
                {\Gamma^{(g)}}_{ij}^k
                    = \frac{1}{2} A_{ij}^k,
                    \quad
                {\Gamma^{(g)}}_{ijk}
                    = \frac{1}{2} S_{ijk}
                \label{eq:Gamma_g}
            \end{equation}
        \item すべての$\alpha \in \R$に対し
            \begin{equation}
                {\Gamma^{(\alpha)}}_{ij}^k
                    = \frac{1 - \alpha}{2} A_{ij}^k,
                    \quad
                {\Gamma^{(\alpha)}}_{ijk}
                    = \frac{1 - \alpha}{2} S_{ijk}
                \label{eq:Gamma_alpha}
            \end{equation}
            とくに$\alpha = 1$のとき
            ${\Gamma^{(1)}}_{ij}^k = 0, \;
                {\Gamma^{(1)}}_{ijk} = 0$である。
    \end{enumerate}
\end{proposition}

\begin{proof}
    \uline{(1)} \quad
    \cref{eq:Gamma_g}の左側の等式は
    \begin{alignat}{1}
        {\Gamma^{(g)}}_{ij}^k
            &=
                \frac{1}{2} g^{kl}
                \myparen{
                    \del_i g_{jl}
                    + \del_j g_{li}
                    - \del_l g_{ij}
                }
                \\
            &=
                \frac{1}{2} g^{kl}
                \myparen{
                    S_{ijl}
                    + S_{jli}
                    - S_{lij}
                }
                \quad
                (\text{\cref{prop:components-of-AC-tensor}})
                \\
            &=
                \frac{1}{2} g^{kl} S_{ijl}
                \\
            &=
                \frac{1}{2} A_{ij}^k
    \end{alignat}
    より従う。
    $g$で添字を下げて\cref{eq:Gamma_g}の右側の等式も従う。

    \uline{(2)} \quad
    $\alpha$-接続の定義より
    ${\Gamma^{(\alpha)}}_{ijk}
        = {\Gamma^{(g)}}_{ijk}
            - \frac{\alpha}{2} S_{ijk}$
    だから、
    (1)とあわせて\cref{eq:Gamma_alpha}の左側の等式が従う。
    $g$で添字を下げて\cref{eq:Gamma_g}の右側の等式も従う。
\end{proof}

\begin{proposition}[捩率と曲率のACテンソルによる表示]
    $\calP$上の任意の$\nabla$-アファイン座標に関し、
    $\nabla^{(\alpha)}$の捩率テンソル$T^{(\alpha)}$
    および$(1, 3)$-曲率テンソル$R^{(\alpha)}$の成分表示は
    次をみたす:
    \begin{enumerate}
        \item すべての$\alpha \in \R$に対し
            \begin{equation}
                {T^{(\alpha)}}_{ij}^k
                    = 0
            \end{equation}
        \item すべての$\alpha \in \R$に対し
            \begin{equation}
                {R^{(\alpha)}}_{ijk}^l
                    = \frac{1 - \alpha}{2} \myparen{
                        \del_i A_{jk}^l
                        -
                        \del_j A_{ik}^l
                    }
                    + \myparen{\frac{1 - \alpha}{2}}^2
                    \myparen{
                        A_{jk}^m A_{im}^l
                        -
                        A_{ik}^m A_{jm}^l
                    }
            \end{equation}
            とくに$\alpha = 1$のとき
            ${R^{(1)}}_{ijk}^l = 0$である。
    \end{enumerate}
\end{proposition}

\begin{proof}
    \uline{(1)} \quad
    \begin{alignat}{1}
        {T^{(\alpha)}}_{ij}
            &=
                {\Gamma^{(\alpha)}}_{ij}^k
                - {\Gamma^{(\alpha)}}_{ji}^k
                \\
            &=
                \frac{1 - \alpha}{2} A_{ij}^k
                - \frac{1 - \alpha}{2} A_{ji}^k
                \quad
                (\cref{prop:connections_by_AC_tensor} (2))
                \\
            &=
                0
                \quad
                (\text{$A_{ij}^k = A_{ji}^k$})
    \end{alignat}
    より従う。

    \uline{(2)} \quad
    \begin{alignat}{1}
        {R^{(\alpha)}}_{ijk}^l
            &=
                \del_i {\Gamma^{(\alpha)}}_{jk}^l
                - \del_j {\Gamma^{(\alpha)}}_{ik}^l
                + {\Gamma^{(\alpha)}}_{jk}^m
                    {\Gamma^{(\alpha)}}_{im}^l
                - {\Gamma^{(\alpha)}}_{ik}^m
                    {\Gamma^{(\alpha)}}_{jm}^l
                \\
            &=
                \frac{1 - \alpha}{2}
                \myparen{
                    \del_i A_{jk}^l
                    - \del_j A_{ik}^l
                }
                + \myparen{\frac{1 - \alpha}{2}}^2
                \myparen{
                    A_{jk}^m A_{im}^l
                    - A_{ik}^m A_{jm}^l
                }
                \quad
                (\cref{prop:connections_by_AC_tensor} (2))
    \end{alignat}
    より従う。
\end{proof}


% ------------------------------------------------------------
%
% ------------------------------------------------------------
\section*{今後の予定}

\begin{itemize}
    \item 具体例: 有限集合上の full support な確率分布の族
    \item 具体例: 正規分布族
\end{itemize}

% ------------------------------------------------------------
%
% ------------------------------------------------------------
\section*{参考文献}

\nocite{amari_information_2016}

{
    \renewcommand{\bibsection}{}
    \bibliographystyle{amsalpha}
    \bibliography{./bibliography,../../mybibliography}
}

% ------------------------------------------------------------
%
% ------------------------------------------------------------
\newpage
\appendix
\renewcommand\thesection{\Alph{section}}
\setcounter{section}{0}
\section{付録}

\begin{fact}[ベクトル場の押し出しと関数]
    \label[fact]{fact:apx1}
    $M, N$を (有限次元実\smooth) 多様体、
    $F \colon M \to N$を微分同相写像とする。
    このとき、次が成り立つ:
    \begin{enumerate}
        \item 任意の$f \in \smooth(M)$に対し
            $F_*(fX) = f \circ F^{-1} \, F_*X$が成り立つ。
        \item 任意の$g \in \smooth(N)$に対し
            $((F_* X) g) \circ F = X(g \circ F)$が成り立つ。
    \end{enumerate}
    \vspace{-2em}
    \qed
\end{fact}

\begin{fact}[アファイン変換によるベクトル場の押し出し]
    \label[fact]{fact:apx2}
    $V, V'$を$m$次元$\R$-ベクトル空間、
    $\del_i, \del'_i \; (i = 1, \dots, m)$をそれぞれ$V, V'$の
    基底をベクトル場とみなしたもの、
    $F \colon V \to V'$をアファイン変換とし、
    $\del_i, \del'_i$に関する$F$の行列表示を
    $(a^i_j)_{i, j}$とする。
    このとき、
    $F_* \del_j = a_i^j \del'_j$が成り立つ。
    \qed
\end{fact}

\begin{proof}[\cref{lemma:pullback-affine-connection}の証明]
    $\del_i, \del'_i \; (i = 1, \dots, m)$をそれぞれ$V, V'$の
    基底をベクトル場とみなしたものとし、
    $\del_i, \del'_i$に関する$F$の行列表示を
    $(a^i_j)_{i, j}$とおき、
    その逆行列を
    $(\wt{a}^i_j)_{i, j}$とおく。
    任意の$X = X^i \del_i, \; Y = Y^i \del_i \in \Gamma(TV)$に対し
    \begin{alignat}{1}
        (F^* \nabla')_X Y
            &=
                F^{-1}_* \myparen{
                    \nabla'_{F_* X}
                    F_* Y
                }
                \\
            &=
                F^{-1}_* \myparen{
                    \nabla'_{F_* (X^i \del_i)}
                    F_* (Y^j \del_j)
                }
                \\
            &=
                F^{-1}_* \myparen{
                    \nabla'_{X^i \circ F^{-1} \, F_* \del_i}
                    (Y^j \circ F^{-1} \, F_* \del_j)
                }
                \quad
                (\text{\cref{fact:apx1} (1)})
                \\
            &=
                F^{-1}_* \myparen{
                    \nabla'_{X^i \circ F^{-1} \, a^k_i \del'_k}
                    (Y^j \circ F^{-1} \, a^l_j \del'_l)
                }
                \quad
                (\text{\cref{fact:apx2}})
                \\
            &=
                F^{-1}_* \myparen{
                    X^i \circ F^{-1} \, a^k_i a^l_j
                    \nabla'_{\del'_k}
                    (Y^j \circ F^{-1} \del'_l)
                }
                \\
            &=
                F^{-1}_* \myparen{
                    X^i \circ F^{-1} \, a^k_i a^l_j
                    \del'_k (Y^j \circ F^{-1}) \del'_l
                }
                \quad
                (\text{基底$\del'_i$の定める座標は$\nabla'$-アファイン})
                \\
            &=
                F^{-1}_* \myparen{
                    X^i \circ F^{-1} \, a^k_i a^l_j
                    ((F^{-1}_* \del'_k) Y^j) \circ F^{-1} \del'_l
                }
                \quad
                (\text{\cref{fact:apx1} (2)})
                \\
            &=
                X^i a^k_i a^l_j
                (F^{-1}_* \del'_k) (Y^j)
                F^{-1}_* \del'_l
                \quad
                (\text{\cref{fact:apx1} (1)})
                \\
            &=
                X^i a^k_i a^l_j
                \wt{a}^m_k \del_m (Y^j)
                \wt{a}^n_l \del_n
                \quad
                (\text{\cref{fact:apx2}})
                \\
            &=
                X^i \del_i (Y^j) \del_j
                \\
            &=
                \nabla_X Y
    \end{alignat}
    となる。
    よって$F^* \nabla' = \nabla$が成り立つ。
\end{proof}


\end{document}