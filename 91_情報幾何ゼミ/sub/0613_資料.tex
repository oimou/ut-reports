\documentclass[report]{jlreq}
\usepackage{global}
\usepackage{./local}
\subfiletrue
\def\assetspath{../}
%\makeindex
\chead{2023/06/06}
\begin{document}

% ============================================================
%
% ============================================================

% ------------------------------------------------------------
%
% ------------------------------------------------------------
\section{振り返りと導入}

\begin{itemize}
    \item 指数型分布族の "多様体化"
    \item Amari-Chentsov テンソル
\end{itemize}

% ------------------------------------------------------------
%
% ------------------------------------------------------------
\section{指数型分布族の "多様体化"}

\begin{definition}[自然パラメータ付け]
    \begin{equation}
        P \colon \Theta_{(V, T, \mu)} \to \calP(\calX),
            \quad
            \theta
            \mapsto
            \exp (\myangle{\theta}{T(x)} - \psi(\theta)) \, \mu(dx)
    \end{equation}
    を$(V, T, \mu)$による$\calP(\calX)$の
    \term{自然パラメータ付け}[natural parameterization]
        {自然パラメータ付け}[しぜんパラメータづけ]
    という。
\end{definition}

\begin{definition}[真パラメータ空間]
    \begin{equation}
        \Theta^\calP_{(V, T, \mu)}
            \coloneqq P^{-1}(\calP)
    \end{equation}
    を$\calP$の$(V, T, \mu)$に関する
    \term{真パラメータ空間}[strict parameter space]
        {真パラメータ空間}[しんパラメータくうかん]
    という。
\end{definition}

\subsection{最小次元実現}

\begin{propdef}[単射性条件]
    $\calP$の実現$(V, T, \mu)$に関する次の条件は同値である:
    \begin{enumerate}
        \item $P \colon \Theta_{(V, T, \mu)} \to \calP(\calX)$は単射である。
        \item $\forall \theta \in V^\vee$
            に対し
            「
                $\myangle{\theta}{T(x)} = \text{const. $\mu$-a.e.$x$}
                \implies
                \theta = 0$
            」
            が成り立つ。
        \item $V$の任意の真アファイン部分空間$W$に対し、
            「$T(x) \in W \; \text{$\mu$-a.e.$x$}$でない」
            が成り立つ。
    \end{enumerate}
    これらの条件が成り立つとき、
    $(V, T, \mu)$は\termsilent{単射性条件}をみたすという。
\end{propdef}

\begin{proof}
    \TODO{}
\end{proof}

\begin{propdef}[Affine span 条件]
    $\calP$の実現$(V, T, \mu)$に関する条件
    \begin{enumerate}
        \item $\Theta^\calP_{(V, T, \mu)}$は
            $V^\vee$を affine span する。
    \end{enumerate}
    が成り立つとき、
    $(V, T, \mu)$は\termsilent{affine span 条件}をみたすという。
    \TODO{違う名称にすべき?}
\end{propdef}

\begin{remark}[条件をみたさない例]
    \TODO{}
\end{remark}

\begin{proposition}[実現の間の変換]
    \label[proposition]{prop:transformation-between-representations}
    $(V, T, \mu), (V', T', \mu')$を$\calP$の実現、
    $P, P'$をそれぞれの実現による自然パラメータ付けとする。
    このとき、
    $(V, T, \mu)$が
    単射性条件と affine span 条件をみたすならば、
    ある全射$\R$-線型写像$L \colon V' \to V$と
    ベクトル$b \in V$
    が存在して、
    すべての$p \in \calP$に対し
    \begin{equation}
        T(x) = L(T'(x)) + b
            \qquad
            \text{$p$-a.e.$x$}
    \end{equation}
    が成り立つ。
    \TODO{$\mu$-a.e. にすべき?}
\end{proposition}

\begin{proof}
    $P, P'$の右逆写像
    $\theta \colon \calP \to \Theta^\calP_{(V, T, \mu)}$
    および
    $\theta' \colon \calP \to \Theta^\calP_{(V', T', \mu')}$
    をひとつずつ選んでおく。

    まず$\mu = \mu'$の場合を考える。

    まず$(V, T, \mu)$の affine span 条件より、
    ある
    $a^i \in \Theta^\calP_{(V, T, \mu)} \; (i = 0, \dots, m)$
    が存在して、
    $e^i \coloneqq a^i - a^0 \; (i = 1, \dots, m)$は
    $V^\vee$の基底となる。
    そこで$p^i \coloneqq P(a^i) \in \calP \; (i = 0, \dots, m)$とおき、
    線型写像$L \colon V' \to V$を
    $t' \mapsto
        \myangle{\theta'(p^i) - \theta'(p^0)}{t'} e_i$
    で定め、
    ベクトル$b \in V$を
    $b \coloneqq
        \mybrace{
            \psi(\theta(p^i)) - \psi(\theta(p^i))
            - \psi(\theta'(p^0)) + \psi(\theta'(p^0))
        } e_i$
    で定める。
    すべての$p \in \calP$に対し
    \begin{equation}
        T(x) = L(T'(x)) + b
            \quad
            \text{$p$-a.e.$x$}
    \end{equation}
    が成り立つことを示す。

    指数型分布族の定義より、
    すべての$p \in \calP$に対し
    \begin{equation}
        \exp(\myangle{\theta(p)}{T(x)} - \psi(\theta(p)))
            =
                \dd[p]{\mu}(x)
            =
                \exp(\myangle{\theta'(p)}{T'(x)} - \psi'(\theta'(p)))
            \qquad
                \text{$p$-a.e.$x$}
    \end{equation}
    が成り立つ ($p, \mu, \mu'$が互いに絶対連続であることを用いた)。
    このことを
    「a.e.」を使わずに言い換えると次のようになる:
    すべての$p \in \calP$に対し、
    ある$p$-零集合$N \subset \calX$が存在して、
    すべての$x \in \calX \setminus N$に対し
    上の等式が成り立つ。
    $p \in \calP$とする。
    上の等式を整理して
    \begin{equation}
        \myangle{\theta(p)}{T(x)}
            - \myangle{\theta'(p)}{T'(x)}
            =
                \psi(\theta(p))
                - \psi'(\theta'(p))
            \qquad
                (x \in \calX \setminus N)
    \end{equation}
    が成り立つ。
    そこで
    $x_0 \in \calX \setminus N$をひとつ選んで固定すると
    \begin{equation}
        \myangle{\theta(p)}{T(x) - T(x_0)}
            =
                \myangle{\theta'(p)}{T'(x) - T'(x_0)}
            \qquad
                (x \in \calX \setminus N)
    \end{equation}
    が成り立つ。

    各$i = 0, \dots, m$に対し
    \begin{equation}
        \myangle{a^i}{T(x) - T(x_0)}
            =
                \myangle{\theta'(p^i)}{T'(x) - T'(x_0)}
            \qquad
                (x \in \calX \setminus N)
    \end{equation}
    となるから、
    各$i = 1, \dots, m$に対し
    \begin{alignat}{3}
        &\phantom{\therefore}&
            \myangle{e^i}{T(x) - T(x_0)}
            &=
                \myangle{\theta'(p^i) - \theta'(p^0)}{T'(x) - T'(x_0)}
                &&\qquad
                (x \in \calX \setminus N)
                \\
        &\therefore \quad&
            \myangle{e^i}{T(x)}
            &=
                \myangle{\theta'(p^i) - \theta'(p^0)}{T'(x)}
                \\
        &&&\qquad
            -
            \myangle{\theta'(p^i) - \theta'(p^0)}{T'(x_0)}
            +
            \myangle{e^i}{T(x_0)}
            &&\qquad
            (x \in \calX \setminus N)
    \end{alignat}
    が成り立つ。
    よって、すべての$x \in \calX \setminus N$に対し
    \begin{alignat}{1}
        T(x)
            &=
                L(T'(x))
                + \mybrace{
                    \myangle{\theta'(p^i) - \theta'(p^0)}{T'(x_0)}
                    +
                    \myangle{e^i}{T(x_0)}
                } e_i
                \\
            &=
                L(T'(x))
                + b
    \end{alignat}
    が成り立つ。
    もし$L$が全射でなかったとすると、
    $T(x) = L(T'(x)) + b \in \Im L + b$
    が$p$-a.e.$x$すなわち$\mu$-a.e.$x$に対し成り立つことになるが、
    $\Im L + b$は$V$の真アファイン部分空間だから
    $(V, T, \mu)$の単射性条件に反する。
    したがって$L$は全射である。

    $\mu \neq \mu'$の場合も、
    $(V, T, \mu)$の単射性条件を用いて
    $(V, T, \mu') \stackrel{\id, 0}{\leadsto} (V, T, \mu)$
    が成り立つことから明らか。
\end{proof}

\begin{corollary}
    上の命題の状況で
    さらに$(V', T', \mu')$も
    単射性条件と affine span 条件をみたすならば、
    $L$は線型同型であり、
    すべての$p \in \calP$に対し
    \begin{equation}
        \begin{cases}
            T(x) = L(T'(x)) + b & \text{$p$-a.e.$x$} \\
            \theta'(p) = \up{t}L(\theta(p))
        \end{cases}
    \end{equation}
    が成り立つ。
    ただし
    写像$\theta \colon \calP \to \Theta^\calP_{(V, T, \mu)}$
    および
    $\theta' \colon \calP \to \Theta^\calP_{(V', T', \mu')}$は
    $P, P'$の逆写像である。
    \TODO{$L$は一意?}
\end{corollary}

\begin{proof}
    上の証明の続きで、
    すべての$x \in \calX \setminus N$に対し
    \begin{alignat}{2}
        &\phantom{\therefore}&
            \myangle{\theta(p)}{L(T'(x)) + b}
            - \myangle{\theta'(p)}{T'(x)}
            &=
                \psi(\theta(p)) - \psi'(\theta'(p))
                \\
        &\therefore \quad&
            \myanglem{\up{t}L(\theta(p))}{T'(x)}
            - \myangle{\theta'(p)}{T'(x)}
            &=
                \psi(\theta(p)) - \psi'(\theta'(p))
                - \myangle{\theta(p)}{b}
                \\
        &\therefore \quad&
            \myanglem{\up{t}L(\theta(p)) - \theta'(p)}{T'(x)}
            &=
                \psi(\theta(p)) - \psi'(\theta'(p))
                - \myangle{\theta(p)}{b}
    \end{alignat}
    が成り立つ。
    $(V', T', \mu)$の単射性条件より
    $\up{t}L(\theta(p)) = \theta'(p)$
    が成り立つ。

    $\theta, \theta'$の役割を入れ替えて
    $\up{t}L(\theta'(p)) = \theta(p)$も成り立つから、
    とくに$V^\vee$の基底$e^i \; (i = 1, \dots, m)$に対し
    $\up{t}L \up{t}L e^i = e^i$が成り立つ。
    よって$\up{t}L, L$は線型同型写像である。
\end{proof}

\begin{theorem}[最小次元実現の特徴づけ]
    $\calP$の実現$(V, T, \mu)$に関する次の条件は同値である:
    \begin{enumerate}
        \item $(V, T, \mu)$は$\calP$の最小次元実現である。
        \item $(V, T, \mu)$は単射性条件と affine span 条件をみたす。
    \end{enumerate}
\end{theorem}

\begin{proof}
    \uline{(1) \Rightarrow (2)} \quad
    以前示した。\TODO{affine span 条件は示してない?}

    \uline{(2) \Rightarrow (1)} \quad
    $(V, T, \mu)$が単射性条件と affine span 条件をみたすとする。
    $\calP$の任意の実現
    $(V', T', \mu')$に対し、
    \cref{prop:transformation-between-representations}より
    全射線型写像$L: V' \to V$が存在するから、
    $\dim V \le \dim V'$である。
    したがって$V$は$\calP$の最小次元実現である。
\end{proof}

\begin{remark}[正規分布族の最小次元実現]
    \TODO{}
\end{remark}

\begin{propdef}
    指数型分布族$\calP$に関し、次は同値である:
    \begin{enumerate}
        \item ある最小次元実現$(V, T, \mu)$に対し、
            $\Theta^\calP_{(V, T, \mu)}$は$V^\vee$で開である。
        \item すべての最小次元実現$(V, T, \mu)$に対し、
            $\Theta^\calP_{(V, T, \mu)}$は$V^\vee$で開である。
    \end{enumerate}
    $\calP$がこれらの同値な2条件をみたすとき、
    $\calP$は\termsilent{開}[open]であるという。
\end{propdef}

\begin{proof}
    (1) $\Rightarrow$ (2)は
    \cref{prop:transformation-between-representations}より従う。
    (2) $\Rightarrow$ (1)は
    最小次元実現が存在することから従う。
\end{proof}

\subsection{多様体構造}

\begin{propdef}
    $\calP$は開であるとする。
    $\calP$の最小次元実現$(V, T, \mu)$をひとつ選ぶと、
    自然パラメータ付け$P$により、
    $\calP$上に多様体構造と平坦アファイン接続を定めることができる。
    この多様体構造および平坦アファイン接続は
    最小次元実現のとり方によらない。
    これを$\calP$の
    \termsilent{自然な多様体構造}
    および
    \termsilent{自然な平坦アファイン接続}
    と呼ぶ。
\end{propdef}

\begin{proof}
    下の図式の可換性と、
    $P^{-1}, P'^{-1}$が diffeo. であること、
    $\up{t}L$が線型同型であることから従う。
    \begin{equation}
        \begin{tikzcd}
            (\calP, \calU_{(V, T, \mu)})
                \ar{d}[swap]{P^{-1}}
                \ar{r}{\id}
                &(\calP, \calU_{(V', T', \mu')})
                    \ar{d}{P'^{-1}}
                \\
            \Theta^\calP
                \ar{r}{\up{t}L}
                &\Theta'^\calP
        \end{tikzcd}
    \end{equation}
\end{proof}

\begin{propdef}[$\calP$の Fisher 計量]
    $\calP$は開であるとする。
    $\calP$の最小次元実現$(V, T, \mu)$をひとつ選ぶと、
    $\Theta^\calP_{(V, T, \mu)}$上の Fisher 計量$g$を
    $\theta$で引き戻して
    $\calP$上の Riemann 計量
    $\theta^* g$が定まる。
    この計量は最小次元実現のとり方によらない。
    これを$\calP$上の
    \termsilent{Fisher 計量}
    と呼ぶ。
\end{propdef}

\begin{proof}
    期待値と分散のペアリングの命題と同様の議論により、
    各$p \in \calP$に対し
    $g_{\theta(p)} = (L \otimes L) g'_{\theta'(p)}$
    が成り立つ。

    示すべきことは
    $\theta^* g = \theta'^* g'$が成り立つことである。
    各$p \in \calP, \; u, v \in T_p\calP$に対し
    \begin{alignat}{1}
        (\theta^* g)_p(u, v)
            &=
                g_{\theta(p)} (d\theta_p(u), d\theta_p(v))
                \\
            &=
                \myangle{
                    g_{\theta(p)}
                }{
                    d\theta_p(u) \otimes d\theta_p(v)
                }
                \\
            &=
                \myangle{
                    (L \otimes L) g'_{\theta'(p)}
                }{
                    d\theta_p(u) \otimes d\theta_p(v)
                }
                \\
            &=
                \myangle{
                    g'_{\theta'(p)}
                }{
                    \up{t}L \circ d\theta_p(u) \otimes \up{t}L \circ d\theta_p(v)
                }
                \\
            &=
                \myangle{
                    g'_{\theta'(p)}
                }{
                    d(\up{t}L \circ \theta)_p (u) \otimes d(\up{t}L \circ \theta)_p (v)
                }
                \\
            &=
                \myangle{
                    g'_{\theta'(p)}
                }{
                    d\theta'_p (u) \otimes d\theta'_p (v)
                }
                \\
            &=
                g'_p (d\theta'_p(u), d\theta'_p(v))
                \\
            &=
                (\theta'^* g')_p(u, v)
    \end{alignat}
    が成り立つ。
\end{proof}

\TODO{Amari-Chentsov もいける?}

% ------------------------------------------------------------
%
% ------------------------------------------------------------
\section{期待値パラメータ空間}

\begin{definition}[期待値パラメータ空間]
    集合$\calM_{(V, T, \nu)}$
    \begin{equation}
        \calM_{(V, T, \nu)}
            \coloneqq \mybrace{
                \mu \in V
                \mid
                \exists \;
                p \colon \text{$\calX$上の確率分布}
                \; \text{s.t.} \;
                p \ll \nu, \;
                E_p[T] = \mu
            }
    \end{equation}
    を$(V, T, \nu)$の
    \term{期待値パラメータ空間}[mean parameter space]
        {期待値パラメータ空間}[きたいちぱらめーたくうかん]
    という。
\end{definition}

期待値パラメータ空間$\calM$は、
$\calP$に属する確率分布に関する$T$の期待値をすべて含んでいる
(一般には真に含んでいる)。

\begin{proposition}
    $\mu \in V$が
    ある$p \in \calP$に関する
    $T$の期待値ならば (すなわち$\mu = E_p[T]$ならば)、
    $\mu$は$\calM_{(V, T, \nu)}$に属する。
\end{proposition}

\begin{proof}
    \TODO{}
\end{proof}

\begin{proposition}[$\calM$は凸集合]
    $\calM_{(V, T, \nu)}$は$V$の凸集合である。
\end{proposition}

\begin{proof}
    \TODO{}
\end{proof}


% ------------------------------------------------------------
%
% ------------------------------------------------------------
\section{Fisher 計量}

\begin{example}[正規分布族]
    \TODO{ちゃんと書く}
    $\calP$を$\calX = \R$上の正規分布族とし、
    実現$(V, T, \mu)$を
    $V = \R^2, \;
        T(x) = (x, x^2), \;
        \mu = \lambda$
    とおく。
    これは条件Aをみたす。

    自然パラメータ空間は
    $\Theta = \Theta^\circ = \R \times \R_{< 0}$である。

    対数分配関数は
    \begin{equation}
        \psi(\theta)
            = \frac{\mu^2}{2 \sigma^2}
            + \log \sigma
            + \frac{1}{2} \log 2\pi
    \end{equation}
    である。
    ただし$\theta^1 = \frac{\mu}{\sigma^2}, \;
        \theta^2 = -\frac{1}{2 \sigma^2}$
    とおいた。
    よって
    \begin{alignat}{1}
        d\psi
            &=
                \frac{\mu}{\sigma^2}
                d\mu
                + \frac{\sigma^2 - \mu^2}{\sigma^3}
                d\sigma
                \\
            &=
                -\frac{\theta^1}{2\theta^2} d\theta^1
                +\myparen{
                    -\frac{1}{2\theta^2}
                    + \frac{(\theta^1)^2}{4(\theta^2)^2}
                }
                d\theta^2
                \\
        \Hess\psi
            &= Dd\psi \\
            &=
                \myparen{
                    -\frac{1}{2\theta^2}
                    d\theta^1
                    + \frac{\theta^1}{2(\theta^2)^2}
                    d\theta^2
                }
                d\theta^1
                +
                \myparen{
                    \frac{\theta^1}{2(\theta^2)^2}
                    d\theta^1
                    + \myparen{
                        \frac{1}{2(\theta^2)^2}
                        - \frac{(\theta^1)^2}{2(\theta^2)^3}
                    }
                    d\theta^2
                }
                d\theta^2
                \\
            &=
                \frac{1}{\sigma^2} (d\mu)^2
                + \frac{2}{\sigma^2} (d\sigma)^2
    \end{alignat}
    である。
    Fisher 計量$g \coloneqq \Hess\psi$から定まる
    Levi-Civita 接続$\nabla^{(g)}$の、
    座標$\mu, \sigma$に関する接続係数を求めてみる。
    \begin{alignat}{2}
        {\Gamma^{(g)}}_{11}^1
            = 0,
            &\qquad
                {\Gamma^{(g)}}_{12}^1
                    = {\Gamma^{(g)}}_{21}^1
                    = -\frac{1}{\sigma},
            &&\qquad
                {\Gamma^{(g)}}_{22}^1
                    = 0,
            \\
        {\Gamma^{(g)}}_{11}^2
            = \frac{1}{2\sigma},
            &\qquad
                {\Gamma^{(g)}}_{12}^2
                    = {\Gamma^{(g)}}_{21}^2
                    = 0,
            &&\qquad
                {\Gamma^{(g)}}_{22}^2
                    = -\frac{1}{\sigma}
    \end{alignat}
    測地線の方程式は
    \begin{equation}
        \begin{cases}
            x'' - \frac{2}{y} x' y' = 0 \\
            y'' + \frac{1}{2y} (x')^2 - \frac{1}{y} (y')^2 = 0
        \end{cases}
    \end{equation}
    である。
    これを直接解くのは少し大変である。
    その代わりに、
    既知の Riemann 多様体との間の等長同型を利用して測地線を求める。
    $(\Theta, g)$は、
    上半平面$H$に計量
    $\breve{g} = \frac{(dx)^2 + (dy)^2}{2y^2}$
    を入れた Riemann 多様体との間に
    等長同型$(\Theta, g) \to (H, \breve{g}), \;
        (x, y) \mapsto (x, \sqrt{2}y)$
    を持つ。
    Levi-Civita 接続に関する測地線は
    等長同型で保たれるから、
    $(H, \breve{g})$の測地線を求めればよい。
    $(H, \breve{g})$の測地線は、
    $y$軸に平行な直線と
    $x$軸上に中心を持つ半円で尽くされることが知られている。
    これらを等長同型で写して、
    $(\Theta, g)$の測地線として
    $y$軸に平行な直線と
    $x$軸上に長軸を持つ半楕円が得られる。
\end{example}


% ------------------------------------------------------------
%
% ------------------------------------------------------------
\section{今後の予定}

\begin{itemize}
    \item KL ダイバージェンス
    \item Fisher 計量
    \item アファイン接続
\end{itemize}

% ------------------------------------------------------------
%
% ------------------------------------------------------------
\section{参考文献}

\nocite{amari_information_2016}

{
    \renewcommand{\bibsection}{}
    \bibliographystyle{amsalpha}
    \bibliography{./bibliography,../../mybibliography}
}


\end{document}