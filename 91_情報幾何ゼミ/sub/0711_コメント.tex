\documentclass[report]{jlreq}
\usepackage{global}
\usepackage{./local}
\subfiletrue
\def\assetspath{../}
%\makeindex
\chead{2023/07/11}
\begin{document}

% ============================================================
%
% ============================================================

発表中にコメントがあった事柄を整理する。

\begin{problem}
    双対構造の定義の
    Riemann 計量の部分を
    擬 Riemann 計量に変更したらどうなるか?
\end{problem}

\begin{answer}
    \TODO{}
\end{answer}

\begin{problem}
    $M$を多様体、
    $g$を$M$上の Riemann 計量、
    $\nabla$を$M$上のアファイン接続、
    $S$を$M$上の$(0, 3)$-テンソル場とする。
    このとき、
    $\nabla$の曲率テンソルが$0$
    かつ$S = \nabla g$が成り立つならば、
    $\alpha$-接続
    $\nabla^{(\alpha)} \coloneqq \nabla^g - \frac{\alpha}{2} g^{-1}S$
    の曲率テンソル$R^{(\alpha)}$の成分は
    (1.2)式のように表せるか?

    逆に、
    $R^{(\alpha)}$が$R^{(0)}$のスカラー倍で表せるためには、
    $\nabla$はどのような条件をみたす必要があるか?
\end{problem}

\begin{answer}
    cf. Jun Zhang, A note on curvature of α-connections of a statistical manifold, 2007

    \TODO{}
\end{answer}

\begin{problem}
    $\calP$を指数型分布族、
    $g$を Fisher 計量とする。
    $R^g = 0$となるような指数型分布族はどのようなものか?
\end{problem}

\begin{answer}
    $R^g = 0$ということは$g$が局所的に
    $g = (dx^1)^2 + \cdots + (dx^n)^2$
    と書けるということである。
    すなわち、各$p \in \calP$に対し、
    $p$の近傍$U$とその上の座標$x$が存在して、
    すべての$q \in U$に対し、
    $\Var_q[T]$の$x$-座標に関する行列表示は単位行列となる。
    このことを統計の言葉でいえば、確率変数 (ベクトル) $T$の白色化が
    $U$上で一斉に可能ということになる。
    \TODO{}
\end{answer}

\begin{problem}
    指数型分布族でも$q$-指数型分布族でもない分布族で
    dually flat なものはあるか?
\end{problem}

\begin{definition}[自己平行部分多様体]
    \TODO{}
\end{definition}


\end{document}