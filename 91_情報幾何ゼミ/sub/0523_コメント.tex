\documentclass[report]{jlreq}
\usepackage{global}
\usepackage{./local}
\subfiletrue
\def\assetspath{../}
%\makeindex
\chead{2023/05/23}
\begin{document}

% ============================================================
%
% ============================================================

発表中にコメントがあった事柄を整理する。

各$\omega \in V^\vee$に対し、
$\langle \omega, f \rangle$を
\termsilent{$f$の$\omega$の向きの成分}、
$\Var[\langle \omega, f \rangle]$を
\termsilent{$f$の$\omega$の向きの分散}と呼ぶことにすれば、
次の命題が成り立つ。

\begin{proposition}
    \begin{enumerate}
        \item $f$の分散が$0$であることと、
            すべての$\omega \in V^\vee$に対し
            $f$の$\omega$の向きの分散が$0$であることとは同値である。
        \item $\omega \in V^\vee$に関し、
            $f$の$\omega$の向きの分散が$0$であることと、
            $f$の$\omega$の向きの成分がa.e.定数であることとは同値である。
        \item $f$がa.e.定数であることと、
            すべての$\omega \in V^\vee$に対し
            $f$の$\omega$の向きの成分がa.e.定数であることとは同値である。
    \end{enumerate}
\end{proposition}

\begin{proof}
    \TODO{}
\end{proof}

$\Var[f]$を行列とみなせば、
$f$の$\omega$の向きの分散が$0$であることは
次のように理解できる。

\begin{proposition}
    $V$の基底を固定することで
    $\Var[f] \in M_m(\R), \; \omega \in \R^m$とみなせば、
    $\Var[\langle \omega, f \rangle] = 0$であることと、
    $\Var[f]\omega = 0$であることとは同値である。
\end{proposition}

\begin{lemma}
    $A \in M_m(\R)$を半正定値対称行列とする\footnote{
        半正定値性を除いた場合の反例としては
        $A = \begin{bmatrix}
            1 & 0 \\
            0 & -1
        \end{bmatrix}, \;
            v = \begin{bmatrix}
                1 \\ 1
            \end{bmatrix}$
        がある。
    }。
    このとき、$v \in \R^m$に関し
    $\up{t}v A v = 0$であることと、
    $Av = 0$であることとは同値である。
\end{lemma}

\begin{proof}
    \TODO{}
\end{proof}

\begin{proof}[命題の証明.]
    \TODO{}
\end{proof}

条件Aは
アファイン部分空間の言葉で特徴づけることができる。

\begin{proposition}
    実現$(V, T, \mu)$に関し次は同値である:
    \begin{enumerate}
        \item $(V, T, \mu)$は条件Aをみたす。
        \item $T$の像が$V$をほとんど affine span する\TODO{定義?}。
    \end{enumerate}
\end{proposition}

\begin{proof}
    \TODO{}
\end{proof}

\begin{definition}
    \begin{equation}
        \Theta'_{(V, T, \mu)} \coloneqq \mybrace{
            \theta \in \Theta_{(V, T, \mu)}
            \mid
            P_\theta \in \calP
        }
    \end{equation}
    を$\calP$の実現$(V, T, \mu)$の
    \termsilent{真のパラメータ空間}[strict parameter space]と呼ぶ。
\end{definition}

\begin{propdef}
    指数型分布族$\calP$に関し、次は同値である:
    \begin{enumerate}
        \item ある最小次元実現$(V, T, \mu)$に対し、
            $\Theta'_{(V, T, \mu)}$は$V^\vee$で開である。
        \item すべての最小次元実現$(V, T, \mu)$に対し、
            $\Theta'_{(V, T, \mu)}$は$V^\vee$で開である。
    \end{enumerate}
    $\calP$がこれらの同値な2条件をみたすとき、
    $\calP$は\termsilent{開}[open]であるという。
\end{propdef}

\begin{proof}
    \TODO{}
\end{proof}

\begin{proposition}
    $(V, T, \mu)$を$\calP$の最小次元実現とする。
    このとき、
    写像$\Theta'_{(V, T, \mu)} \to \calP, \;
        \theta \mapsto P_\theta$は全単射である。
\end{proposition}

\begin{proof}
    全射性は$\Theta'$の定義より従う。
    単射性は\url{0502_資料.pdf}の命題2.2(2)より従う。
\end{proof}

\begin{propdef}
    $\calP$は開であるとする。
    $\calP$の最小次元実現$(V, T, \mu)$をひとつ選ぶと、
    上の命題の全単射により
    $\calP$上に多様体構造と平坦アファイン接続を定めることができる。
    この多様体構造および平坦アファイン接続は
    最小次元実現のとり方によらない。
    これを$\calP$の
    \termsilent{自然な多様体構造}
    および
    \termsilent{自然な平坦アファイン接続}
    と呼ぶ。
\end{propdef}

\begin{proof}
    \TODO{cf. [BN78, Lem. 8.1]. ちなみに[BN70]の証明はRN微分のa.e.一致の扱いに誤りがあるらしい}
\end{proof}

\begin{propdef}[$\calP$の Fisher 計量]
    $\calP$は開であるとする。
    $\calP$の最小次元実現$(V, T, \mu)$をひとつ選ぶと、
    $\Theta'_{(V, T, \mu)}$上の Fisher 計量を
    $\calP$上の Riemann 計量とみなすことができる。
    この計量は最小次元実現のとり方によらない。
    これを$\calP$の
    \termsilent{Fisher 計量}
    と呼ぶ。
\end{propdef}

\begin{proof}
    \TODO{}
\end{proof}

\begin{proposition}
    条件Aが成り立つことと
    $\Hess\psi$が正定値であることとは同値である。
\end{proposition}

\begin{proof}
    \TODO{}
\end{proof}

\begin{definition}[スコア関数]
    \begin{alignat}{2}
        p
            &\colon \Theta \times \calX \to \R, \quad
            &&(\theta, x) \mapsto \exp\myparen{
                \langle \theta, T(x) \rangle - \psi(\theta)
            }
            \\
        l
            &\colon \Theta \times \calX \to \R, \quad
            &&(\theta, x) \mapsto \log p(\theta, x)
            \\
        dl
            &\colon \Theta \times \calX \to V, \quad
            &&(\theta, x) \mapsto dl_\theta
            \\
        dl^2
            &\colon \Theta \times \calX \to V \otimes V, \quad
            &&(\theta, x) \mapsto dl_\theta^2
            \\
        E[dl^2]
            &\colon \Theta \to V \otimes V, \quad
            &&\theta \mapsto E_{P_\theta}[dl^2]
    \end{alignat}
    ただし
    $dl$の値域が$V$であるのは、
    各$\theta$ごとに$dl_\theta \in T^\vee_\theta \Theta$を
    $V^{\vee\vee} = V$の元とみなしている。
\end{definition}

\begin{proposition}
    \begin{equation}
        E[dl^2] = \Hess\psi
    \end{equation}
\end{proposition}

\begin{proof}
    \TODO{}
\end{proof}



% ------------------------------------------------------------
%
% ------------------------------------------------------------
\section*{参考文献}

\nocite{amari_information_2016}

{
    \renewcommand{\bibsection}{}
    \bibliographystyle{amsalpha}
    \bibliography{./bibliography,../../mybibliography}
}



\end{document}