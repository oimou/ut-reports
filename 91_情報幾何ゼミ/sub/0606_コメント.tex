\documentclass[report]{jlreq}
\usepackage{global}
\usepackage{./local}
\subfiletrue
\def\assetspath{../}
%\makeindex
\chead{2023/06/06}
\begin{document}

% ============================================================
%
% ============================================================

発表中にコメントがあった事柄を整理する。

\begin{propdef}[測度の押し出し]
    $\calX, \calY$を可測空間、
    $\mu$を$\calX$上の測度、
    $f \colon \calX \to \calY$を可測写像とする。
    このとき、
    $\mu \circ f^{-1}$は$\calY$上の測度となる。
    これを
    \termsilent{$f$による$\mu$の押し出し}と呼び、
    $f_* \mu$と書く。
\end{propdef}

\begin{proof}
    $\mu \circ f^{-1}$が$\calY$上の測度となることは、
    測度の定義を直接確かめればすぐにわかる。
\end{proof}

\begin{problem}
    条件A (3) は次と同値か?
    \begin{enumerate}
        \setcounter{enumi}{3}
        \item $\aspan(\supp T_* \mu) = V$
    \end{enumerate}
\end{problem}

\begin{answer}
    \TODO{}
\end{answer}

\begin{problem}
    条件A (2) は次と同値か?
    \begin{enumerate}
        \setcounter{enumi}{4}
        \item 任意の$\R$-ベクトル空間$V'$および
            線型写像$F \in \Lin(V, V')$に対し
            「$F(T(x)) = \text{const.} \; \text{$\mu$-a.e.$x$}$
                \Rightarrow $F = 0$」が成り立つ。
    \end{enumerate}
\end{problem}

\begin{answer}
    \TODO{任意の$V'$ではなく、有限次元に限定すれば言えそう}
\end{answer}

\begin{proposition}
    $(V, T, \mu), (V', T', \mu')$を
    $\calP$の実現とする。
    このとき、
    ある$c > 0$および$\theta^0 \in V^\vee$であって
    \begin{equation}
        \mu'
            = c \exp\myangle{\theta^0}{T(x)} \cdot \mu
    \end{equation}
    をみたすものがただ1組存在する。
\end{proposition}

\begin{proof}
    $\mu, \mu'$がみたす関係式
    \begin{equation}
        \myangle{\theta}{T(x)} - \psi(\theta)
            = \myangle{\theta'}{T'(x)} - \psi'(\theta')
            + \log \dd[\mu']{\mu}(x)
            \qquad
            \text{$\mu$-a.e.$x$}
    \end{equation}
    に定理1.12と系1.13を合わせて式変形するとわかる。
\end{proof}



% ------------------------------------------------------------
%
% ------------------------------------------------------------
\section*{参考文献}

\nocite{amari_information_2016}
\nocite{bn1970_pdf}
\nocite{BN78}

{
    \renewcommand{\bibsection}{}
    \bibliographystyle{amsalpha}
    \bibliography{./bibliography,../../mybibliography}
}

\end{document}