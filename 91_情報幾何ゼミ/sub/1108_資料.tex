\documentclass[report]{jlreq}
\usepackage{global}
\usepackage{./local}
\subfiletrue
\def\assetspath{../}
%\makeindex
\chead{2023/11/08}
\begin{document}

% ============================================================
%
% ============================================================

% ------------------------------------------------------------
%
% ------------------------------------------------------------
\section*{振り返りと導入}

前回はKLダイバージェンスの双対平坦多様体への一般化を考え始めた。
本稿では次のことを行う:
\begin{itemize}
    \item 双対平坦構造の canonical ダイバージェンスを定義する。
    \item 双対平坦構造からシンプレクティック構造が定まることをみる。
\end{itemize}

% ------------------------------------------------------------
%
% ------------------------------------------------------------
\section{双対平坦構造の canonical ダイバージェンス}

以下$M$を多様体とする。

\begin{definition}[canonical ダイバージェンスの定義域]
    \label[definition]{def:canonical-divergence-domain}
    $(g, \nabla, \nabla^*)$を$M$上の双対平坦構造とし、
    \begin{alignat}{1}
        \calW
            &\coloneqq
                \mybrace{
                    (p, q) \in M \times M
                    \;\Bigg|\;
                    \parbox{11cm}{
                        \centering
                        (i) $p, q$を結ぶ$\nabla$-測地線のうち最短なものがただひとつ存在する。
                        \\
                        (ii) その像を覆う単連結$\nabla$-アファインチャートが存在する。
                    }
                }
                \\
        \calU
            &\coloneqq
                \Int_{M \times M} \calW
    \end{alignat}
    とおく。
    $\calU$を
    双対平坦構造$(g, \nabla, \nabla^*)$の
    \term{canonical ダイバージェンスの定義域}
        {canonical ダイバージェンスの定義域}[canonical ダイバージェンスのていぎいき]
    と呼ぶ。
\end{definition}

\begin{proposition}
    \label[proposition]{prop:canonical-divergence-domain}
    $\calU$は
    $M \times M$における$\Delta_M$の開近傍である。
\end{proposition}

\begin{proof}
    資料末尾の付録を参照。
\end{proof}

\begin{propdef}[canonical ダイバージェンス]
    \label[propdef]{propdef:canonical-divergence}
    $(p, q) \in \calU$を固定し、
    (i)の$\nabla$-測地線を$\gamma \colon I \to M$とおく。
    $\gamma$の像を覆う
    任意の単連結$\nabla$-アファインチャート$(U, \theta)$と
    $U$上の$g$の任意の$\nabla$-ポテンシャル$\psi \colon U \to \R$に対し、
    $\eta_i \coloneqq \del_i \psi \in \smooth(U), \;
        \eta \coloneqq (\eta_i)_i \in \smooth(U, \R^n), \;
        \varphi \coloneqq \myangle{\theta}{\eta} - \psi \in \smooth(U)$
    とおくと、
    \begin{equation}
        \psi(q) + \varphi(p) - \myangle{\theta(q)}{\eta(p)}
    \end{equation}
    の値は$(U, \theta), \psi$の取り方によらない。
    この値を$D(p \| q)$と記す。
    以上により定まる関数$D \colon \calU \to \R$を
    双対平坦構造$(g, \nabla, \nabla^*)$の
    \term{canonical ダイバージェンス}
        {canonical ダイバージェンス}[canonical ダイバージェンス]
    と呼ぶ。
\end{propdef}

\begin{remark}
    $\eta$は$U$上の座標とは保証されていないことに注意。
\end{remark}

\begin{lemma}
    条件(ii)をみたす任意の$(U, \theta)$に対し、
    次をみたす$\nabla$-ポテンシャル$\psi \colon U \to \R$がただひとつ存在する:
    \begin{enumerate}[label=(\alph*)]
        \item $\psi(p) = 0$
        \item $(\nabla \psi)_p = 0$
    \end{enumerate}
    このような$\psi$を$\psi_p$とおくと、
    $U$上の$g$の任意の$\nabla$-ポテンシャル$\psi$に対し
    \begin{equation}
        \psi_p(q)
            =
                \psi(q) + \varphi(p) - \myangle{\theta(q)}{\eta(p)}
                \qquad
                (q \in U)
    \end{equation}
    が成り立つ。
\end{lemma}

\begin{proof}
    (一意性):
    2つの$\nabla$-ポテンシャル$\psi, \psi'$に対し
    $\nabla^2 \psi = g = \nabla^2 \psi'$であることより従う。
    (存在):
    $U$は単連結だから Poincar\'e の補題より
    $U$上の$\nabla$-ポテンシャル$\psi$が存在する。
    このとき
    $\wt{\psi}(q) \coloneqq \psi(q) - \del_i \psi(p) \theta^i(q) - \psi(p)$
    もまた$U$上の$\nabla$-ポテンシャルであり、
    条件(a), (b)をみたす。
    したがって存在が示せた。
\end{proof}

\begin{proof}[\cref{propdef:canonical-divergence}の証明]
    補題より
    $D^{\theta', \psi}(p \| q) = \psi_p(q) = D^{\theta', \psi'}(p \| q)$が成り立つ。
    また、$U \cap U'$のうち$\gamma$の像を含む連結成分上では
    2つの座標$\theta, \theta'$はアファイン変換で移り合うから、
    $D^{\theta, \psi}(p \| q) = D^{\theta', \psi}(p \| q)$が成り立つ。
    よって$D^{\theta, \psi} = D^{\theta', \psi'}$が成り立つ。
\end{proof}

\begin{proposition}[canonical ダイバージェンスの性質]
    $(p, q) \in \calU$に対し次が成り立つ:
    \begin{enumerate}
        \item $D(p \| q) \ge 0$
        \item $D(p \| q) = 0 \iff p = q$
    \end{enumerate}
\end{proposition}

\begin{proof}
    $\psi$の$\nabla$-凸性より従う。
\end{proof}

\begin{definition}[$D$へのベクトル場の作用の記法]
    $X_1, \dots, X_l, Y_1, \dots, Y_m \in \frakX(M), \;
        l, m \in \Z_{\ge 0}$に対し
    \begin{equation}
        D(X_1, \dots, X_l \| Y_1, \dots, Y_m)
            \coloneqq
                (X_1, 0) \dots (X_l, 0) (0, Y_1) \dots (0, Y_m) D
                \in \smooth(\calU)
    \end{equation}
    と定める。
    ただし$X, Y \in \frakX(M)$に対し
    $(X, Y) \in \frakX(M \times M)$は
    ベクトル場の直和を表す。
\end{definition}

\begin{proposition}[canonical ダイバージェンスから双対平坦構造の復元]
    $p \in M$、
    $x = (x_\alpha)_\alpha$を
    $p$のまわりの座標として
    \begin{enumerate}
        \item $g_p(X_p, Y_p)
            = D(\| XY)(p, p)
            = -D(X \| Y)(p, p)
            = D(XY \|)(p, p)$
        \item $\Gamma_{\alpha \beta \gamma}(p)
            = - D(\del_\gamma \| \del_\alpha \del_\beta)(p, p)$
        \item $D(p \| {-}) \colon \calU_p \to \R$は
            $\calU_p$上の$g$の$\nabla$-ポテンシャルである。
        \item $D({-} \| p) \colon \calU_p \to \R$は
            $\calU_p$上の$g$の$\nabla^*$-ポテンシャルである。
    \end{enumerate}
\end{proposition}

\begin{proof}
    \uline{(1)} \quad
    直接計算より。

    \uline{(2)} \quad
    直接計算より。
    ただし$\nabla$が平坦ゆえ
    $\Gamma_{\alpha \beta}^\gamma
        = \frac{\del x^\gamma}{\del x^\alpha \del x^\beta}$
    であることに注意。

    \uline{(3)} \quad
    $D$の定義から
    \begin{equation}
        d(D(p \| {-}))_q
            = d\psi_q - \eta_i(p) d\theta^i_q
            = (\eta_i(q) - \eta_i(p)) d\theta^i_q
    \end{equation}
    より
    \begin{equation}
        \nabla^2 (D(p \| {-}))
            =
                \del_j (\eta_i) \, d\theta^j d\theta^i
            =
                g
    \end{equation}
    を得る。

    \uline{(4)} \quad
    (3) と同様。
\end{proof}

% ------------------------------------------------------------
%
% ------------------------------------------------------------
\section{双対平坦構造とシンプレクティック構造}

\begin{definition}[シンプレクティックベクトル空間]
    $2n$次元$\R$-ベクトル空間$V$と
    $V$上の非退化交代形式$\omega \colon V \times V \to \R$の組
    $(V, \omega)$を
    \term{シンプレクティックベクトル空間}[symplectic vector space]
        {シンプレクティックベクトル空間}[シンプレクティックベクトル空間]
    という。
\end{definition}

\begin{definition}[シンプレクティック形式]
    $M$を$2n$次元多様体とする。
    $\omega \in \Omega^2(M)$が
    $M$上の
    \term{シンプレクティック形式}[symplectic form]
        {シンプレクティック形式}[シンプレクティックけいしき]
    であるとは、
    $\omega$が閉形式かつ
    各点$x \in M$で
    $(T_x M, \omega_x)$がシンプレクティックベクトル空間であることをいう。
\end{definition}

\begin{example}[標準シンプレクティック形式]
    $\R^{2n}$の標準的な座標
    $(x^1, \ldots, x^n, y_1, \ldots, y_n)$に対し
    $\omega_0 \coloneqq dx^i \wedge dy_i \in \Omega^2(\R^{2n})$
    は$\R^{2n}$上のシンプレクティック構造である。
    $\omega_0$を$\R^{2n}$上の
    \term{標準シンプレクティック形式}[standard symplectic form]
        {標準シンプレクティック形式}[ひょうじゅんシンプレクティックけいしき]
    という。
\end{example}

\begin{example}[余接束の自然シンプレクティック形式]
    $M$を$n$次元多様体とする。
    余接束$\pi \colon T^\vee M \to M$上の
    1-形式$\theta \in \Omega^1(T^\vee M)$を
    \begin{equation}
        \theta_{(q, p)} (v)
            \coloneqq
                p(d\pi_{(q, p)} (v))
    \end{equation}
    で定め、これを
    \term{トートロジカル1-形式}[tautological 1-form]
        {トートロジカル1-形式}[トートロジカル1-けいしき]
    と呼ぶ。
    このとき
    $\omega_0 \coloneqq -d\theta \in \Omega^2(T^\vee M)$は
    $T^\vee M$上のシンプレクティック構造となり、
    これを$T^\vee M$上の
    \term{自然シンプレクティック形式}[canonical symplectic form]
        {自然シンプレクティック形式}[しぜんシンプレクティックけいしき]
    と呼ぶ。
\end{example}

\begin{proposition}[自然シンプレクティック形式の成分表示]
    $M$を$n$次元多様体、
    $x = (x^i)_i$を$M$の局所座標とする。
    $x$により定まる$T^\vee M$の局所座標を
    $(x^1, \dots, x^n, \xi_1, \dots, \xi_n)$とおくと、
    これに関する自然シンプレクティック形式$\omega_0$の成分表示は
    \begin{equation}
        \omega_0
            =
                dx^i \wedge d\xi_i
    \end{equation}
    となる。
\end{proposition}

\begin{proof}
    $\pi(q, p) = q$ゆえ
    $d\pi^* (dx^i) = dx^i$であることに注意すると、
    トートロジカル1-形式の成分表示
    \begin{equation}
        \theta_{(q, p)}
            =
                d\pi_{(q, p)}^* (\xi_i dx^i)
            =
                \xi_i dx^i
    \end{equation}
    より命題の等式が従う。
\end{proof}

\begin{proposition}[双対平坦構造のシンプレクティック構造]
    $M$を多様体、
    $(g, \nabla, \nabla^*)$を$M$上の双対平坦構造、
    $D \colon \calU \to \R$を canonical ダイバージェンス、
    $\omega_0 \in \Omega^2(T^\vee M)$を
    $T^\vee M$上の自然シンプレクティック構造とする。
    写像$d_1 D \colon \calU \to T^\vee M$を
    第1成分に関する微分、すなわち
    $d_1 D \coloneqq D(\tdeldel{x^i} \|) \, dx^i$
    で定め、$\calU$上の2-形式
    $\omega \in \Omega^2(\calU)$を
    $\omega \coloneqq (d_1 D)^* (\omega_0)$
    で定める。
    このとき次が成り立つ:
    \begin{enumerate}
        \item $M$の任意の局所座標$x = (x_i)_i$に対し、
            $x^* \coloneqq x$とおいて
            $\calU$の局所座標$(x, x^*) = (x^1, \dots, x^n, x^{*1}, \dots, x^{*n})$
            を定めると、
            $\omega$の成分表示は
            \begin{equation}
                \omega
                    =
                        D(\tdeldel{x^i} \| \tdeldel{x^{*j}}) \,
                        dx^i \wedge dx^{*j}
            \end{equation}
            となる。
        \item $\omega$は$\calU$上のシンプレクティック形式である。
    \end{enumerate}
\end{proposition}

\begin{proof}
    \uline{(1)} \quad
    $x$により定まる
    $T^\vee M$の局所座標を$(x^1, \dots, x^n, \xi_1, \dots, \xi_n)$
    とおくと
    \begin{alignat}{1}
        \omega
            &=
                (d_1 D)^* (\omega_0)
                \\
            &=
                (d_1 D)^* (dx^i \wedge d\xi_i)
                \\
            &=
                d(x^i \circ d_1 D) \wedge d(\xi_i \circ d_1 D)
                \\
            &=
                dx^i \wedge \myparen{
                    D(\tdeldel{x^j} \tdeldel{x^i} \|) \, dx^j
                    +
                    D(\tdeldel{x^i} \| \tdeldel{x^{*j}}) \, dx^{*j}
                }
                \\
            &=
                D(\tdeldel{x^i} \| \tdeldel{x^{*j}}) \,
                dx^i \wedge dx^{*j}
    \end{alignat}
    を得る。

    \uline{(2)} \quad
    \TODO{要証明}
\end{proof}

% ------------------------------------------------------------
%
% ------------------------------------------------------------
%\section{一般のダイバージェンス}
%
%canonical ダイバージェンスと限らない一般のダイバージェンスを定義し、
%そこから双対平坦構造が誘導されることをみる。
%
%\begin{definition}[ダイバージェンス]
%    $M$を多様体、
%    $\diag M \subset \calE \opensubset M \times M$とする。
%    {\smooth}関数$D \colon \calE \to \R$が
%    $M$上の\term{ダイバージェンス}[divergence]
%        {ダイバージェンス}[ダイバージェンス]
%    であるとは、
%    次が成り立つことをいう:
%    \begin{enumerate}
%        \item 非負性
%        \item 正定値性
%        \item $(D(\del_i \| \del_j))_{ij}$は正定値対称
%    \end{enumerate}
%\end{definition}
%
%\begin{proposition}[ダイバージェンスから誘導される双対平坦構造]
%    \begin{enumerate}
%        \item $g(X, Y) \coloneqq -D(X \| Y)$で$g$を定めると、
%            $g$は$M$上の Riemann 計量となる。
%        \item $\Gamma_{ijk} \coloneqq -D(\del_i \del_j \| \del_k)$と定めると、
%            $\Gamma_{ij}^k \coloneqq g^{kl} \Gamma_{ijl}$
%            を接続係数する
%            $M$上の平坦アファイン接続$\nabla$が定まる。
%        \item $\nabla^*$を$g$に関する$\nabla$の双対接続とすると、
%            $(g, \nabla, \nabla^*)$は双対平坦構造となる。
%    \end{enumerate}
%\end{proposition}
%
%\begin{proof}
%    \TODO{}
%\end{proof}
%
%\begin{proposition}[ダイバージェンスの復元]
%    $D$は定義域上で
%    $(g, \nabla, \nabla^*)$の canonical ダイバージェンスに一致する。
%\end{proposition}
%
%\begin{proof}
%    \TODO{}
%\end{proof}


% ------------------------------------------------------------
%
% ------------------------------------------------------------
\section*{今後の予定}

\begin{itemize}
    \item 双対平坦構造のシンプレクティック構造と双対アファイン座標
\end{itemize}

% ------------------------------------------------------------
%
% ------------------------------------------------------------
\section*{参考文献}

%Legendre 変換については
%\cite{niculescu_convex_2018}
%を参考にした。
%期待値パラメータに関しては
%\cite{wainwright_graphical_2007}を参考にした。

\nocite{amari_information_2016}
\nocite{_bayes_2020}

{
    \renewcommand{\bibsection}{}
    \bibliographystyle{amsalpha}
    \bibliography{./bibliography,../../mybibliography}
}

% ------------------------------------------------------------
%
% ------------------------------------------------------------
\newpage
\appendix
\renewcommand\thesection{\Alph{section}}
\setcounter{section}{0}
\section{付録}

\subsection{\cref{def:canonical-divergence-domain}の条件(i), (ii)について}

$M$を多様体、
$g$を$M$上のRiemann 計量、
$\nabla$を$M$上のアファイン接続とする。

\begin{definition}[simple chain (ここだけの用語)]
    $X$を位相空間とする。
    $X$の開集合の有限列$(U_i)_{i = 1}^N$が
    \term{simple chain}
        {simple chain}[simple chain]
    であるとは、
    $U_i \cap U_j \neq \emptyset \;
        \iff |i - j| \le 1$
    が成り立つことをいう。
    さらにすべての$U_i \cap U_{i + 1}$が連結のとき
    \term{very simple chain}
        {very simple chain}[very simple chain]
    という。
\end{definition}

\begin{lemma}
    $\nabla$-アファインチャートの列
    $(U_i)_{i = 1}^N$が
    very simple chain ならば、
    $\textstyle \bigcup_{i = 1}^N U_i$を定義域とする
    $\nabla$-アファイン座標が存在する。
\end{lemma}

\begin{proof}
    $U_1 \cap U_2$は連結であり、
    2つの座標はアファイン変換で移り合うから、
    それに応じて$U_2$上の座標を調整すれば
    $U_1 \cup U_2$上の$\nabla$-アファイン座標が得られる。
    以下同様にして$U_1 \cup \cdots \cup U_N$上の
    $\nabla$-アファイン座標が得られる。
\end{proof}

\begin{proposition}
    $\gamma \colon I \to M$が単射な$\nabla$-測地線ならば、
    $\gamma(I)$を覆う
    単連結な$\nabla$-アファインチャートが存在する。
\end{proposition}

\begin{proof}
    \TODO{要確認}
    $\gamma(I)$の各点のまわりの$\nabla$-アファインチャートを集めて
    $\gamma(I)$の開被覆$\calU$を作る。
    Lebesgue 数の補題より、
    実数列$0 = t_0 < t_1 < \cdots < t_N = 1$が存在して
    各$S_i \coloneqq \gamma([t_{i - 1}, t_i])$は
    ある$U_i \in \calU$に含まれる。
    $\gamma$の単射性より、
    ある$\varepsilon > 0$であって
    $(U(S_i, \varepsilon))_{i = 1}^N$が
    very simple chain かつ
    $U(S_i, \varepsilon) \subset U_i$
    となるものが存在する
    (ただし$U(S_i, \varepsilon)$は Riemann 距離に関する$\varepsilon$-近傍)。
    そこで$\textstyle U \coloneqq \bigcup_{i = 1}^N U(S_i, \varepsilon)$とおくと、
    補題より$U$上の$\nabla$-アファイン座標$\theta$が存在する。
    $\theta(\gamma(I))$が$\theta(U)$内の線分であることに注意すると、
    $\theta(\gamma(I))$の十分小さい近傍$V$をとれば、
    $\theta^{-1}(V)$は$\gamma(I)$を覆う単連結な$\nabla$-アファインチャートとなる。
\end{proof}

\subsection{\cref{prop:canonical-divergence-domain}の証明}

\begin{proof}
    $p \in M$を固定し、
    $(p, p)$の$M \times M$におけるある開近傍が
    $\calW$に含まれることを示せばよい。
    そのような開近傍を次のように構成する。

    まず$\nabla$の平坦性より
    $p$のまわりの$\nabla$-アファインチャート$(U, \theta)$が存在する。
    $p$の$M$における
    (計量$g$から定まる距離に関する)
    $3r$-近傍が$U$に含まれるように$r > 0$をとり、
    $p$の$M$における$r$-近傍を$U'$とおく。
    さらに$\theta(p)$の$\R^n$における$\eps$-近傍$V_\eps$が
    $\theta(U')$に含まれるように$\eps > 0$をとる。
    $U_\eps \coloneqq \theta^{-1}(V_\eps)$とおくと
    $(p, p)$の$U_\eps \times U_\eps$は$M \times M$における
    開近傍である。

    以下$U_\eps \times U_\eps \subset \calW$を示す。
    すなわち、$(a, b) \in U_\eps \times U_\eps$として
    $(a, b) \in \calW$を示す。
    $U_\eps$は$\nabla$-凸ゆえ、
    $a, b$を結ぶ$U_\eps$内の$\nabla$-測地線$\gamma$が存在する。
    このとき$\gamma$はとくに$U$内の$\nabla$-測地線でもあるが、
    $U$は$\nabla$-アファインチャートだから
    $\gamma$は$a, b$を結ぶ$U$内の唯一の$\nabla$-測地線である。
    $U'$の定め方から、
    $a, b$を結ぶ ($M$内の) 任意の$\nabla$-測地線は
    $\gamma$より真に長いか$\gamma$自身である\TODO{怪しい}。
    したがって、
    $a, b$を結ぶ ($M$内の) $\nabla$-測地線のうち最短なものは
    ただひとつ存在し、それは$\gamma$である。
    よって$(a, b)$は条件(i)をみたす。
    さらに$U_\eps$は$\gamma$の像を覆う
    単連結$\nabla$-アファインチャートだから、
    $(a, b)$は条件(ii)をみたす。
    したがって$(a, b) \in \calW$である。
\end{proof}


\end{document}