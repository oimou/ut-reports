\documentclass[report]{jlreq}
\usepackage{global}
\usepackage{./local}
\subfiletrue
\def\assetspath{../}
%\makeindex
\chead{2023/04/11}
\begin{document}

% ============================================================
%
% ============================================================

発表時にコメントがあった命題などを整理する。

\begin{fact}[確率測度の一致と確率密度関数の一致 (up to a positive scalar)]
    $\calX$を可測空間、$\mu$を$\calX$上の$\sigma$-有限測度とする。
    このとき、
    $\mu$に関し絶対連続な$\calX$上の確率測度$p_1, p_2$に関し、
    次は同値である:
    \begin{enumerate}
        \item $p_1 = p_2$
        \item $\exists \; c > 0$ \quad s.t. \quad
            $\mu$-a.e. $x \in \calX$に対し、
            $\dd[p_1]{\mu}(x) = c \dd[p_2]{\mu}(x)$
    \end{enumerate}
    \qed
\end{fact}


\end{document}