\documentclass[report]{jlreq}
\usepackage{global}
\usepackage{./local}
\subfiletrue
\def\assetspath{../}
%\makeindex
\chead{2023/07/04}
\begin{document}

% ============================================================
%
% ============================================================

% ------------------------------------------------------------
%
% ------------------------------------------------------------
\section*{振り返りと導入}

前回は、指数型分布族にいくつかの構造を定め、
Amari-Chentsov テンソルと$\alpha$-接続を定義した。
本稿では次のことを行う:

\begin{itemize}
    \item 具体例の計算 (有限集合上の full support な確率分布の族)
    \item 具体例の計算 (正規分布族)
\end{itemize}

今回以降、次のように記法を変更する\footnote{
    \cite{BN78}での記法によった。
}。

\begin{definition}[パラメータの空間の記法の変更]
    可測空間$\calX$上の
    指数型分布族$\calP$と
    その実現$(V, T, \mu)$に対し、
    \begin{itemize}
        \item 自然パラメータ空間$\Theta_{(V, T, \mu)}$を
            $\wt{\Theta}_{(V, T, \mu)}$と書くことにし、
        \item 真パラメータ空間$\Theta_{(V, T, \mu)}^\calP$を
            $\Theta_{(V, T, \mu)}$と書くことにする。
    \end{itemize}
    文脈から明らかな場合は添字を省略することがある。
\end{definition}

% ------------------------------------------------------------
%
% ------------------------------------------------------------
\section{具体例: 有限集合上の full support な確率分布の族}

本節では、
有限集合上の full support な確率分布の族について、
$\alpha$-接続に関する測地線方程式を求めてみる。

\begin{settings}[有限集合上の full support な確率分布の族]
    $\calX \coloneqq \{ 1, \dots, n \} \; (n \in \Z_{\ge 1})$とし、
    \begin{equation}
        \calP \coloneqq \mybrace{
            \sum_{i = 1}^n p_i \delta^i
            \in \calP(\calX)
            \, \Big| \,
            0 < p_i < 1 \; (i = 1, \dots, n)
        }
    \end{equation}
    とおく。
    ただし$\delta^i$は
    1点$i \in \calX$での Dirac 測度である。
    \url{0425_資料.pdf}例3.1でみたように
    $\calP$は$\calX$上の指数型分布族である。
\end{settings}

まず$\calP$が開であることを確かめる。

\begin{proposition}[最小次元実現の構成および$\calP$が開であることの確認]
    \label[proposition]{prop:minimal_representation}
    ~
    \begin{enumerate}
        \item $(V, T, \mu)$を次のように定めると、
            これは$\calP$の実現となる:
            \begin{alignat}{1}
                &V = \R^{n - 1}, \\
                &T \colon \calX \to V, \quad
                    k \mapsto \up{t}(\delta_{1k}, \dots, \delta_{(n - 1)k}), \\
                &\mu = \gamma \quad (\text{数え上げ測度})
            \end{alignat}
        \item この実現の対数分配関数$\psi \colon \wt{\Theta} \to \R$は
            $\psi(\theta)
                =
                    \log\myparen{
                        1 + \sum_{l = 1}^{n - 1} \exp\theta^l
                    }$
            となる。
        \item 写像$P \coloneqq P_{(V, T, \mu)} \colon \wt{\Theta} \to \calP$は
            次をみたす:
            \begin{equation}
                P(\theta)
                    =
                        \frac{
                            1
                        }{
                            1 + \sum_{l = 1}^{n - 1} \exp\theta^l
                        }
                        \myparen{
                            \sum_{i = 1}^{n - 1}
                                (\exp\theta^i)
                                \delta^i
                                +
                                \delta^n
                        }
            \end{equation}
        \item 次の写像$\theta \colon \calP \to \Theta$は$P$の逆写像である:
            \begin{alignat}{1}
                \theta
                    \colon
                        \calP \to \Theta,
                    \quad
                        \sum_{i = 1}^n p_i \delta^i
                        \mapsto
                        \myparen{
                            \log \frac{p_i}{p_n}
                        }_{i = 1}^{n - 1}
            \end{alignat}
        \item $\Theta = \wt{\Theta} = \R^{n - 1}$が成り立つ。
        \item $(V, T, \mu)$は最小次元実現である。
            とくに$\calP$は開である。
    \end{enumerate}
\end{proposition}

\begin{proof}
    \uline{(1)} \quad
    実現であることはよい。

    \uline{(2)} \quad
    対数分配関数の定義より
    \begin{alignat}{1}
        \psi(\theta)
            &=
                \log \int_\calX
                    \exp \myangle{\theta}{T(x)}
                    \, \mu(dx)
                \\
            &=
                \log \sum_{i = 1}^n
                    \exp \myparen{
                        \sum_{l = 1}^{n - 1}
                            \theta^l
                            \delta_{li}
                    }
                \\
            &=
                \log \myparen{
                    \sum_{i = 1}^{n - 1}
                        \exp \theta^i
                    + 1
                }
    \end{alignat}
    である。

    \uline{(3)} \quad
    $P(\theta) = \exp(\myangle{\theta}{T(k)} - \psi(\theta)) \cdot \gamma$を
    直接計算すれば得られる。

    \uline{(4)} \quad
    $P \circ \theta, \; \theta \circ P$を
    直接計算すれば確かめられる。

    \uline{(5)} \quad
    すべての$\theta \in \R^{n - 1}$に対し
    $\int_\calX \exp \myangle{\theta}{T(x)} \, \mu(dx)
        = \sum_{l = 1}^{n - 1} \exp \theta^l + 1
        < \infty$
    が成り立つから
    $\Theta = \R^{n - 1}$である。

    \uline{(6)} \quad
    最小次元実現の特徴づけを確かめる。
    条件A(3)と条件Bが成り立つことから、
    最小次元実現であることがわかる。
\end{proof}

以降、
$\calP$には自然な位相および多様体構造が入っているものとして扱い、
$\calP$上の自然な平坦アファイン接続を$\nabla$、
Fisher 計量を$g$とおく。
また、$\theta \colon \calP \to \Theta$は
多様体$\calP$上の座標とみなす。

図形的には、$P$は
$\R^{n - 1}$から$\R^n$内の$(n - 1)$-単体 (の内部) への微分同相写像になっている。

\begin{proposition}[$\calP$の2通りの位相と多様体構造]
    $\calP$上の位相 (resp. 多様体構造) として、
    $\calX$上の符号付き測度全体のなす
    ベクトル空間$\calS(\calX) \cong \R^n$の位相部分空間 (resp. 部分多様体) としての構造と、
    指数型分布族としての自然な構造の2通りを考えられるが、
    これらの構造は互いに一致する。
\end{proposition}

\begin{proof}
    いずれの位相 (resp. 多様体構造) に関しても
    写像$P \colon \wt{\Theta} \to \calP$は同相 (resp. 微分同相) 写像だからである。
\end{proof}

\begin{proposition}[Fisher 計量の成分]
    \label[proposition]{prop:fisher_metric_components}
    座標$\theta = (\theta^1, \dots, \theta^{n - 1})$に関する
    Fisher 計量$g$の成分は
    \begin{equation}
        g_{ij}(p)
            = \delta_{ij} p_i - p_i p_j
            \qquad
            (p \in \calP, \; i, j = 1, \ldots, n - 1)
    \end{equation}
    となる。
\end{proposition}

\begin{proof}
    \cref{prop:minimal_representation}の
    $(V, T, \mu)$に関する Fisher 計量を$\wt{g}$とおくと、
    各点$p \in \calP$に対し
    \begin{alignat}{1}
        g_{ij}(p)
            &=
                g_p \myparen{
                    \deldel{\theta^i},
                    \deldel{\theta^j}
                }
                \\
            &=
                (\theta^* \wt{g})_{p} \myparen{
                    \deldel{\theta^i},
                    \deldel{\theta^j}
                }
                \\
            &=
                \wt{g}_{\theta(p)} \myparen{
                    d\theta \myparen{
                        \deldel{\theta^i}
                    },
                    d\theta \myparen{
                        \deldel{\theta^j}
                    }
                }
                \\
            &=
                (\Hess \psi)_{\theta(p)} \myparen{
                    d\theta \myparen{
                        \deldel{\theta^i}
                    },
                    d\theta \myparen{
                        \deldel{\theta^j}
                    }
                }
                \\
            &=
                (\Var_p [T])(e^i, e^j)
                \\
            &=
                E_p[(T^i - E_p[T^i])(T^j - E_p[T^j])]
                \\
            &=
                \sum_{l = 1}^n
                    (\delta_{il} - p_i)
                    (\delta_{jl} - p_j)
                    p_l
                \\
            &=
                \delta_{ij} p_i - p_i p_j
    \end{alignat}
    が成り立つ。
\end{proof}

\begin{proposition}[ACテンソルの成分]
    \label[proposition]{prop:ac_tensor_components}
    座標$\theta$に関する
    ACテンソル$S$の成分は
    \begin{equation}
        S_{ijk}
            = p_i \delta_{ij} \delta_{jk}
                - p_i p_k \delta_{ij}
                - p_i p_j \delta_{jk}
                - p_j p_k \delta_{ik}
                + 2 p_i p_j p_k
            \qquad
            (p \in \calP, \; i, j, k = 1, \ldots, n - 1)
    \end{equation}
    となる。
\end{proposition}

\begin{proof}
    前回 (\url{0613_資料.pdf}) の命題1.9を用いると
    \begin{equation}
        S_{ijk}(p)
            = E_p [
                (T^i - E_p[T^i])
                (T^j - E_p[T^j])
                (T^k - E_p[T^k])
            ]
    \end{equation}
    となるから、
    \cref{prop:fisher_metric_components}と同様に直接計算して
    命題の主張の等式が得られる。
\end{proof}

以降、$n = 3$の場合を考える。

\begin{proposition}[$n = 3$での$g, S, A$の計算]
    座標$\theta$に関し、
    $g$の行列表示は
    \begin{equation}
        (g_{ij})_{i, j}
            = \begin{pmatrix}
                p_1 (1 - p_1) & - p_1 p_2 \\
                - p_1 p_2 & p_2 (1 - p_2)
            \end{pmatrix},
            \quad
        (g^{ij})_{i, j}
            = \frac{1}{p_3}
                \begin{pmatrix}
                    \frac{p_3}{p_1} + 1 & 1 \\
                    1 & \frac{p_3}{p_2} + 1
                \end{pmatrix}
    \end{equation}
    となる。
    $S$の成分は
    \begin{alignat}{1}
        S_{111}
            &= p_1 - 3 p_1^2 + 2 p_1^3, \\
        S_{112} = S_{121} = S_{211}
            &= - p_1 p_2 + 2 p_1^2 p_2, \\
        S_{122} = S_{212} = S_{221}
            &= - p_1 p_2 + 2 p_1 p_2^2, \\
        S_{222}
            &= p_2 - 3 p_2^2 + 2 p_2^3
    \end{alignat}
    となる。
    $A$の成分は
    \begin{alignat}{2}
        A_{11}^{\hphantom{11}1}
            &=
                1 - 2p_1,
                \qquad
        &A_{11}^{\hphantom{11}2}
            &=
                0
                \\
        A_{12}^{\hphantom{12}1}
            &=
                A_{21}^{\hphantom{21}1}
            =
                - p_2,
                \qquad
        &A_{12}^{\hphantom{12}2}
            &=
                A_{21}^{\hphantom{21}2}
            =
                - p_1
                \\
        A_{22}^{\hphantom{22}1}
            &=
                0,
                \qquad
        &A_{22}^{\hphantom{22}2}
            &=
                1 - 2p_2
    \end{alignat}
    となる。
\end{proposition}

\begin{proof}
    $g$の行列表示は
    \cref{prop:fisher_metric_components}よりわかる。
    その逆行列は直接計算よりわかる。
    $S$の成分は
    \cref{prop:ac_tensor_components}よりわかる。
    $A$の成分は直接計算よりわかる。
\end{proof}

\begin{proposition}[$n = 3$での測地線方程式]
    各$\alpha \in \R$に対し、
    座標$\theta$に関する
    $\nabla^{(\alpha)}$-測地線の方程式は
    \begin{alignat}{1}
        \ddot{\theta^1}
            &=
                - \frac{1 - \alpha}{2}
                \myparen{
                    (1 - 2p_1) \dot{\theta^1}^2
                    - 2p_2 \dot{\theta^1} \dot{\theta^2}
                }
                \\
        \ddot{\theta^2}
            &=
                - \frac{1 - \alpha}{2}
                \myparen{
                    - 2p_1 \dot{\theta^1} \dot{\theta^2}
                    + (1 - 2p_2) \dot{\theta^2}^2
                }
    \end{alignat}
    となる ($p_1, p_2$は$\theta^1, \theta^2$の関数であることに注意)。
    とくに$\alpha = 1$のとき
    \begin{equation}
        \ddot{\theta^1} = 0,
            \quad
            \ddot{\theta^2} = 0
    \end{equation}
    である。
\end{proposition}

\begin{proof}
    測地線の方程式
    \begin{equation}
        \ddot{\theta^k}
            = - \Gamma_{ij}^k \dot{\theta^i} \dot{\theta^j}
    \end{equation}
    に、前回 (\url{0613_資料.pdf}) の命題1.11の等式
    ${\Gamma^{(\alpha)}}_{ij}^k = \frac{1 - \alpha}{2} A_{ij}^{\hphantom{ij}k}$
    を代入して得られる。
\end{proof}

$\alpha \neq 1$の場合に
上の測地線方程式を解くのは難しい(ように思う)。


% ------------------------------------------------------------
%
% ------------------------------------------------------------
\section{具体例: 正規分布族}

本節では、
正規分布族について、
$\alpha$-接続に関する測地線方程式を求めてみる。

\begin{settings}[正規分布族]
    $\calX \coloneqq \R$とし、
    \begin{equation}
        \calP \coloneqq \mybrace{
            \frac{1}{\sqrt{2 \pi \sigma^2}}
            \exp\myparen{
                - \frac{(x - \mu)^2}{2 \sigma^2}
            }
            \lambda(dx)
            \in \calP(\calX)
            \, \Big| \,
            (\mu, \sigma) \in \R \times \R_{> 0}
        }
    \end{equation}
    とおく。
\end{settings}

\begin{proposition}[最小次元実現の構成および$\calP$が開であることの確認]
    \label[proposition]{prop:minimal_representation}
    ~
    \begin{enumerate}
        \item $(V, T, \mu)$を次のように定めると、
            これは$\calP$の実現となる:
            \begin{alignat}{1}
                &V = \R^2, \\
                &T \colon \calX \to V, \quad
                    x \mapsto \up{t}(x, x^2), \\
                &\mu = \lambda \quad (\text{Lebesgue 測度}).
            \end{alignat}
        \item この実現の対数分配関数$\psi \colon \wt{\Theta} \to \R$は
            $\psi(\theta)
                =
                    - \frac{(\theta^1)^2}{8 \theta^2}
                    - \frac{1}{2} \log (- \theta^2)
                    + \frac{1}{2} \log \pi
                $
            となる。
        \item 写像$P \coloneqq P_{(V, T, \mu)} \colon \wt{\Theta} \to \calP$は
            次をみたす:
            \begin{equation}
                P(\theta)
                    =
                        \exp \myparen{
                            \theta^1 x
                            + \theta^2 x^2
                            - \psi(\theta)
                        }
                        \lambda(dx)
            \end{equation}
        \item 次の写像$\theta \colon \calP \to \Theta$は$P$の逆写像である:
            \begin{alignat}{1}
                \theta
                    \colon
                        \calP \to \Theta,
                    \quad
                        p
                        \mapsto
                        \myparen{
                            \frac{E_p[x]}{\Var_p[x]},
                            - \frac{1}{2 \Var_p[x]}
                        }
            \end{alignat}
        \item $\Theta = \wt{\Theta} = \R \times \R_{< 0}$が成り立つ。
        \item $(V, T, \mu)$は最小次元実現である。
            とくに$\calP$は開である。
    \end{enumerate}
\end{proposition}

\begin{proof}
    \uline{(1)} \quad
    実現であることはよい。

    \uline{(2)} \quad
    直接計算よりわかる。

    \uline{(3)} \quad
    $P_{(V, T, \mu)}$の定義よりわかる。

    \uline{(4)} \quad
    $(\theta^1, \theta^2) \in \R \times \R_{< 0}$と
    $(\mu, \sigma) \in \R \times \R_{> 0}$の対応に注意すれば
    直接計算よりわかる。

    \uline{(5)} \quad
    $\theta^2 \ge 0$だと
    $\exp\myparen{
        \theta^1 x
        + \theta^2 x^2
        - \psi(\theta)
    }$
    は積分可能でないから
    $\Theta \subset \wt{\Theta} \subset \R \times \R_{< 0}$
    である。
    逆に$\theta$の定義より明らかに
    $\R \times \R_{< 0} \subset \Theta$である。
    したがって
    $\Theta = \wt{\Theta} = \R \times \R_{< 0}$である。

    \uline{(6)} \quad
    最小次元実現の特徴づけの条件A(3)と条件Bが成り立つことから、
    最小次元実現であることがわかる。
\end{proof}

\begin{proposition}
    Fisher 計量$g$について
    \begin{enumerate}
        \item 座標$(\theta^1, \theta^2)$に関する$g$の成分は
            \begin{equation}
                g_{ij}
                    =
                        \myparen{
                            -\frac{1}{2\theta^2}
                            d\theta^1
                            + \frac{\theta^1}{2(\theta^2)^2}
                            d\theta^2
                        }
                        d\theta^1
                        +
                        \myparen{
                            \frac{\theta^1}{2(\theta^2)^2}
                            d\theta^1
                            + \myparen{
                                \frac{1}{2(\theta^2)^2}
                                - \frac{(\theta^1)^2}{2(\theta^2)^3}
                            }
                            d\theta^2
                        }
                        d\theta^2
            \end{equation}
            である。
        \item 座標$(\mu, \sigma)$に関する$g$の成分は
            \begin{equation}
                g_{ij}
                    =
                        \frac{1}{\sigma^2} (d\mu)^2
                        + \frac{2}{\sigma^2} (d\sigma)^2
            \end{equation}
            である。
    \end{enumerate}
\end{proposition}

\begin{proof}
    座標$(\theta^1, \theta^2)$と
    座標$(\mu, \sigma)$の間の座標変換は
    $\theta^1 = \frac{\mu}{\sigma^2}, \;
        \theta^2 = -\frac{1}{2 \sigma^2}$
    および
    $\mu = -\frac{\theta^1}{2\theta^2}, \;
        \sigma = \sqrt{-\frac{1}{2\theta^2}}$
    である。
    \begin{alignat}{1}
        d\mu
            &=
                - \frac{1}{2\theta^2} d\theta^1
                + \frac{\theta^1}{2(\theta^2)^2} d\theta^2,
                \\
        d\sigma
            &=
                \frac{1}{2\sqrt{2}} (-\theta^2)^{-3/2} d\theta^2,
                \\
        d\theta^1
            &=
                \frac{1}{\sigma^2} d\mu
                - \frac{2\mu}{\sigma^3} d\sigma
                \\
        d\theta^2
            &=
                \frac{1}{\sigma^3} d\sigma
    \end{alignat}
    よって
    \begin{alignat}{1}
        (d\theta^1)^2
            &=
                \frac{1}{\sigma^4} (d\mu)^2
                - \frac{\mu}{\sigma^5} d\mu d\sigma
                + \frac{4\mu^2}{\sigma^6} (d\sigma)^2
                \\
        d\theta^1 d\theta^2
            &=
                \frac{1}{\sigma^5} d\mu d\sigma
                - \frac{2\mu}{\sigma^6} (d\sigma)^2
                \\
        (d\theta^2)^2
            &=
                \frac{1}{\sigma^6} (d\sigma)^2
    \end{alignat}
    である。
    したがって
    \begin{alignat}{1}
        Dd\mu
            &=
                \frac{1}{(\theta^2)^2} d\theta^1 d\theta^2
                - \frac{\theta^1}{(\theta^2)^3} (d\theta^2)^2
                \\
            &=
                \frac{4}{\sigma} d\mu d\sigma,
                \\
        Dd\sigma
            &=
                \frac{3}{4\sqrt{2}} (-\theta^2)^{-5/2} (d\theta^2)^2
                \\
            &=
                \frac{3}{\sigma} (d\sigma)^2
    \end{alignat}
    である。
    よって
    \begin{alignat}{1}
        d\psi
            &=
                \frac{\mu}{\sigma^2}
                d\mu
                + \myparen{
                    - \frac{\mu^2}{\sigma^3}
                    + \frac{1}{\sigma}
                }
                d\sigma
                \\
        \Hess\psi
            &=
                Dd\psi
                \\
            &=
                d\myparen{
                    \frac{\mu}{\sigma^2}
                }
                d\mu
                + \frac{\mu}{\sigma^2} Dd\mu
                + d\myparen{
                    - \frac{\mu^2}{\sigma^3}
                    + \frac{1}{\sigma}
                }
                d\sigma
                + \myparen{
                    - \frac{\mu^2}{\sigma^3}
                    + \frac{1}{\sigma}
                }
                Dd\sigma
                \\
            &=
                \frac{1}{\sigma^2} (d\mu)^2
                + \frac{2}{\sigma^2} (d\sigma)^2
    \end{alignat}
    である。
    座標変換により
    $(\theta^1, \theta^2)$に関する$\Hess\psi$の成分も得られる。
\end{proof}

\begin{proposition}[ACテンソルの成分]
    座標$(\mu, \sigma)$に関するACテンソル$S$の成分は
    \begin{alignat}{1}
        S_{111}
            &=
                0
                \\
        S_{112} = S_{121} = S_{211}
            &=
                \frac{2}{\sigma^3}
                \\
        S_{122} = S_{212} = S_{221}
            &=
                0
                \\
        S_{222}
            &=
                \frac{8}{\sigma^3}
    \end{alignat}
    である。
    座標$(\mu, \sigma)$に関する$A$の成分は
    \begin{alignat}{2}
        A_{11}^{\hphantom{11}1}
            &=
                0,
                \qquad
        &A_{11}^{\hphantom{11}2}
            &=
                \frac{1}{\sigma},
                \\
        A_{12}^{\hphantom{12}1}
            &=
                A_{21}^{\hphantom{21}1}
            =
                \frac{2}{\sigma},
                \qquad
        &A_{12}^{\hphantom{12}2}
            &=
                A_{21}^{\hphantom{21}2}
            =
                0,
                \\
        A_{22}^{\hphantom{22}1}
            &=
                0,
                \qquad
        &A_{22}^{\hphantom{22}2}
            &=
                \frac{4}{\sigma}
    \end{alignat}
    である。
\end{proposition}

\begin{proof}
    \begin{alignat}{1}
        DD\psi
            &=
                D \myparen{
                    \frac{1}{\sigma^2} (d\mu)^2
                    + \frac{2}{\sigma^2} (d\sigma)^2
                }
                \\
            &=
                - \frac{2}{\sigma^3} (d\mu)^2 d\sigma
                + \frac{1}{\sigma^2} D(d\mu)^2
                - \frac{4}{\sigma^3} (d\sigma)^3
                + \frac{2}{\sigma^2} D(d\sigma)^2
    \end{alignat}
    ここで
    \begin{alignat}{1}
        D(d\mu)^2
            &=
                2 d\mu Dd\mu
            =
                \frac{8}{\sigma} (d\mu)^2 d\sigma
                \\
        D(d\sigma)^2
            &=
                2 d\sigma Dd\sigma
            =
                \frac{6}{\sigma} (d\sigma)^3
    \end{alignat}
    だから
    \begin{equation}
        DDd\psi
            =
                \frac{6}{\sigma^3} (d\mu)^2 d\sigma
                + \frac{8}{\sigma^3} (d\sigma)^3
    \end{equation}
    である。
    これより命題の主張の式が得られる。
    $A$の成分は直接計算より得られる。
\end{proof}

\begin{proposition}[接続係数]
    ~
    \begin{enumerate}
        \item 座標$(\mu, \sigma)$に関する
            $\Gamma^g$の成分は
            \begin{alignat}{2}
                {\Gamma^{g}}_{11}^1
                    = 0,
                    &\qquad
                        {\Gamma^{g}}_{12}^1
                            = {\Gamma^{g}}_{21}^1
                            = -\frac{1}{\sigma},
                    &&\qquad
                        {\Gamma^{g}}_{22}^1
                            = 0,
                    \\
                {\Gamma^{g}}_{11}^2
                    = \frac{1}{2\sigma},
                    &\qquad
                        {\Gamma^{g}}_{12}^2
                            = {\Gamma^{g}}_{21}^2
                            = 0,
                    &&\qquad
                        {\Gamma^{g}}_{22}^2
                            = -\frac{1}{\sigma}
            \end{alignat}
            である。
        \item 座標$(\mu, \sigma)$に関する
            $\Gamma^{(\alpha)} \; (\alpha \in \R)$の成分は
            \begin{alignat}{2}
                {\Gamma^{(\alpha)}}_{11}^1
                    = 0,
                    &\qquad
                        {\Gamma^{(\alpha)}}_{12}^1
                            = {\Gamma^{(\alpha)}}_{21}^1
                            = - \frac{1 + \alpha}{\sigma},
                    &&\qquad
                        {\Gamma^{(\alpha)}}_{22}^1
                            = 0,
                    \\
                {\Gamma^{(\alpha)}}_{11}^2
                    = \frac{1 - \alpha}{2 \sigma},
                    &\qquad
                        {\Gamma^{(\alpha)}}_{12}^2
                            = {\Gamma^{(\alpha)}}_{21}^2
                            = 0,
                    &&\qquad
                        {\Gamma^{(\alpha)}}_{22}^2
                            = - \frac{1 + 2\alpha}{\sigma}
            \end{alignat}
            である。
    \end{enumerate}
\end{proposition}

\begin{proof}
    $\Gamma^g$は
    ${\Gamma^g}_{ij}^k
        = \frac{1}{2} g^{kl} \myparen{
            \partial_i g_{jl}
            + \partial_j g_{li}
            - \partial_l g_{ij}
        }$
    を直接計算することで得られる。
    $\Gamma^{(\alpha)}$は
    ${\Gamma^{(\alpha)}}_{ij}^k
        = {\Gamma^g}_{ij}^k - \frac{\alpha}{2} A_{ij}^{\hphantom{ij}k}$
    より得られる。
\end{proof}

\begin{proposition}[測地線方程式]
    $(\mu, \sigma)$座標に関する$\nabla^{(\alpha)}$-測地線の方程式は
    \begin{equation}
        \begin{cases}
            \ddot{\mu}
                - \frac{2 (1 + \alpha)}{\sigma} \dot{\mu} \dot{\sigma}
                = 0,
                \\
            \ddot{\sigma}
                + \frac{1 - \alpha}{2 \sigma} \dot{\mu}^2
                - \frac{1 + 2 \alpha}{\sigma} \dot{\sigma}^2
                = 0
        \end{cases}
    \end{equation}
    である。
    とくに$\alpha = 0$のとき
    \begin{equation}
        \begin{cases}
            \ddot{\mu}
                - \frac{2}{\sigma} \dot{\mu} \dot{\sigma}
                = 0,
                \\
            \ddot{\sigma}
                + \frac{1}{2 \sigma} \dot{\mu}^2
                - \frac{1}{\sigma} \dot{\sigma}^2
                = 0
        \end{cases}
    \end{equation}
    である。
\end{proposition}

\begin{proof}
    測地線の方程式
    「$\ddot{x^k} = - {\Gamma^k}_{ij} \dot{x^i} \dot{x^j}$」
    に接続係数を代入して得られる。
\end{proof}

\begin{fact}
    $\H \coloneqq \R \times \R_{> 0}$とし、
    $\breve{g}$を$\H$上の双曲計量とする。
    このとき、
    写像$f \colon \bbH \to \calP, \; (x, y) \to (x, \sqrt{2} y)$は
    $(\bbH, \breve{g})$から$(\calP, g)$への等長同型写像である。
    \qed
\end{fact}

\begin{fact}
    局所等長同型写像は測地線を保つ。
    \qed
\end{fact}

\begin{fact}
    $\bbH$の測地線は、
    $x$軸上に中心を持つ円弧または
    $y$軸に平行な直線である。
    \qed
\end{fact}

\begin{proposition}
    $\bbH$の測地線は、
    長軸が$x$軸に重なる$\text{長軸}:\text{短軸} = 2:1$の楕円または
    $y$軸に平行な直線である。
\end{proposition}


% ------------------------------------------------------------
%
% ------------------------------------------------------------
\section*{今後の予定}

\begin{itemize}
    \item \TODO{}
\end{itemize}

% ------------------------------------------------------------
%
% ------------------------------------------------------------
\section*{参考文献}

\nocite{amari_information_2016}

{
    \renewcommand{\bibsection}{}
    \bibliographystyle{amsalpha}
    \bibliography{./bibliography,../../mybibliography}
}

% ------------------------------------------------------------
%
% ------------------------------------------------------------
\newpage
\appendix
\renewcommand\thesection{\Alph{section}}
\setcounter{section}{0}
\section{付録}

% show geodesics_1.png
\begin{figure}[htbp]
    \centering
    \includegraphics[width=0.9\linewidth]{{\assetspath}/sub/geodesics_1.png}
    \caption{$\alpha$を変化させたときの$\nabla^{\alpha}$-測地線の様子}
    \label{fig:geodesics_1}
\end{figure}



\end{document}