\documentclass[report]{jlreq}
\usepackage{global}
\usepackage{./local}
\subfiletrue
\def\assetspath{../}
%\makeindex
\chead{2023/06/13}
\begin{document}

% ============================================================
%
% ============================================================

% ------------------------------------------------------------
%
% ------------------------------------------------------------
\section*{振り返りと導入}

\begin{itemize}
    \item \TODO{}
\end{itemize}


% ------------------------------------------------------------
%
% ------------------------------------------------------------
\section{具体例: 有限集合上の full support な確率分布の族}

本節では、
有限集合上の full support な確率分布の族について、
Fisher 計量や Amari-Chentsov テンソルなどを計算してみる。

\begin{settings}[有限集合上の full support な確率分布の族]
    $\calX = \{ 1, \dots, n \} \; (n \in \Z_{\ge 1})$とし、
    \begin{equation}
        \calP \coloneqq \mybrace{
            p = \sum_{i = 1}^n p_i \delta^i
            \in \calP(\calX)
            \, \Big| \,
            0 < p_i < 1 \; (i = 1, \dots, n), \;
            \sum_{i = 1}^n p_i = 1
        }
    \end{equation}
    とおく。
    ただし$\delta^i$は
    1点$i \in \calX$での Dirac 測度である。
    \url{0425_資料.pdf}例3.1でみたように
    $\calP$は$\calX$上の指数型分布族である。
\end{settings}

まず$\calP$が開であることを確かめる。

\begin{proposition}[最小次元実現の構成および$\calP$が開であることの確認]
    ~
    \begin{enumerate}
        \item $(V, T, \mu)$を次のように定めると、
            これは$\calP$の最小次元実現となる:
            \begin{alignat}{1}
                &V = \R^{n - 1}, \\
                &T \colon \calX \to V, \quad
                    k \mapsto \up{t}(\delta_{1k}, \dots, \delta_{(n - 1)k}), \\
                &\mu = \gamma \quad (\text{数え上げ測度})
            \end{alignat}
        \item この実現の対数分配関数$\psi \colon \Theta_{(V, T, \mu)} \to \R$は
            \begin{equation}
                \psi(\theta)
                    =
                        \log\myparen{
                            1 + \sum_{l = 1}^{n - 1} \exp\theta^l
                        }
            \end{equation}
            となる。
        \item 写像$P \coloneqq P_{(V, T, \mu)} \colon \Theta_{(V, T, \mu)} \to \calP$は
            \begin{equation}
                \theta = (\theta^1, \dots, \theta^{n - 1})
                    \mapsto
                        \exp\myparen{
                            \myangle{\theta}{T(k)}
                            -
                            \psi(\theta)
                        }
                        \cdot
                        \gamma
                        =
                        \frac{
                            1
                        }{
                            1 + \sum_{l = 1}^{n - 1} \exp\theta^l
                        }
                        \sum_{i = 1}^{n - 1}
                            (\exp\theta^i)
                            \delta^i
            \end{equation}
            となる。
        \item 次の写像$\theta$は$P$の逆写像である:
            \begin{alignat}{1}
                \theta
                    \colon
                        \calP \to \Theta_{(V, T, \mu)},
                    \quad
                        \sum_{i = 1}^n p_i \delta^i
                        \mapsto
                        \myparen{
                            \log \frac{p_i}{p_n}
                        }_{i = 1}^{n - 1}
            \end{alignat}
            となる。
        \item $\calP$の$(V, T, \mu)$に関する
            真パラメータ空間$\Theta^\calP$は
            $\Theta^\calP = \R^{n - 1}$となり、
            したがって$\calP$は開である。
    \end{enumerate}
\end{proposition}

\begin{proof}
    \uline{Step 1: $(V, T, \mu)$が最小次元実現であること} \quad
    条件A(3)は$T$の定義から明らか。
    条件Bの成立は$p_i = \frac{\exp\theta^i}{\lambda(\theta)}$
    と表せることからわかる。

    \uline{Step 2: $P$と$\theta$が互いに逆写像であること} \quad
    \TODO{}

    \uline{Step 3: $\Theta^\calP = \R^{n - 1}$であること} \quad
    \TODO{}
\end{proof}

以降、
$\calP$には自然な多様体構造が入っているものとして扱い、
$\calP$上の自然な平坦アファイン接続を$\nabla$、
Fisher 計量を$g$とおく。
また、$\theta \colon \calP \to \Theta^\calP$は
多様体$\calP$上の座標とみなす。

\begin{proposition}[Fisher 計量の成分]
    座標$\theta$に関する
    Fisher 計量$g$の成分は
    \begin{equation}
        g_{ij}
            = p_i \delta_{ij} - p_i p_j
    \end{equation}
    となる。
\end{proposition}

\begin{proof}
    $\psi$の定める
    $(V, T, \mu)$上の Fisher 計量を$\wt{g}$とおく。
    $g$は定義より
    $g = \theta^* \wt{g}$
    をみたすから、
    座標$\theta$に関する$g$の成分は
    \TODO{}
\end{proof}

\begin{proposition}[ACテンソルの成分]
    座標$\theta$に関する
    ACテンソル$S$の成分は
    \begin{equation}
        S_{ijk}
            = p_i \delta_{ij} \delta_{jk}
                - p_i p_k \delta_{ij}
                - p_i p_j \delta_{jk}
                - p_j p_k \delta_{ik}
                + 2 p_i p_j p_k
    \end{equation}
    となる。
\end{proposition}

\begin{proof}
    \TODO{}
\end{proof}

以降、$n = 3$の場合を考える。

\begin{proposition}[$n = 3$での$g, S, A$の計算]
    座標$\theta$に関し、
    $g$の行列表示は
    \begin{equation}
        (g_{ij})_{i, j}
            = \begin{pmatrix}
                p_1 - p_1^2 & - p_1 p_2 \\
                - p_2 p_1 & p_2 - p_2^2
            \end{pmatrix},
            \quad
        (g^{ij})_{i, j}
            = \frac{1}{p_3}
                \begin{pmatrix}
                    \frac{p_3}{p_1} + 1 & 1 \\
                    1 & \frac{p_3}{p_2} + 1
                \end{pmatrix}
    \end{equation}
    となる。
    $S$の成分は
    \begin{alignat}{1}
        S_{111}
            &= p_1 - 3 p_1^2 + 2 p_1^3, \\
        S_{112} = S_{121} = S_{211}
            &= - p_1 p_2 + 2 p_1^2 p_2, \\
        S_{122} = S_{212} = S_{221}
            &= - p_1 p_2 + 2 p_1 p_2^2, \\
        S_{222}
            &= p_2 - 3 p_2^2 + 2 p_2^3
    \end{alignat}
    となる。
    $A$の成分は
    \begin{alignat}{2}
        A_{11}^1
            &=
                1 - 2p_1,
                \qquad
        &A_{11}^2
            &=
                0
                \\
        A_{12}^1
            &=
                A_{21}^1
            =
                - p_2,
                \qquad
        &A_{12}^2
            &=
                A_{21}^2
            =
                - p_1
                \\
        A_{22}^1
            &=
                - p_1,
                \qquad
        &A_{22}^2
            &=
                1 - 2p_2
    \end{alignat}
    となる。
\end{proposition}

\begin{proof}
    \TODO{}
\end{proof}

\begin{proposition}[$n = 3$での測地線方程式]
    各$\alpha \in \R$に対し、
    座標$\theta$に関する
    $\nabla^{(\alpha)}$-測地線の方程式は
    \begin{alignat}{1}
        \ddot{\theta^1}
            &=
                \frac{1 - \alpha}{2}
                \myparen{
                    (1 - 2p_1) \dot{\theta^1}^2
                    - 2p_2 \dot{\theta^1} \dot{\theta^2}
                    - p_1 \dot{\theta^2}^2
                }
                \\
        \ddot{\theta^2}
            &=
                \frac{1 - \alpha}{2}
                \myparen{
                    - 2p_1 \dot{\theta^1} \dot{\theta^2}
                    + (1 - 2p_2) \dot{\theta^2}^2
                }
    \end{alignat}
    となる。
    とくに$\alpha = 1$のとき
    \begin{equation}
        \ddot{\theta^1} = 0,
            \quad
            \ddot{\theta^2} = 0
    \end{equation}
    である。
\end{proposition}

\begin{proof}
    ${\Gamma^{(\alpha)}}_{ij}^k = \frac{1 - \alpha}{2} A_{ij}^k$
    であることより従う。
\end{proof}

$\alpha \neq 1$の場合に
上の測地線方程式を解くのは難しい(ように思う)。



% ------------------------------------------------------------
%
% ------------------------------------------------------------
\section{具体例: 正規分布族}

\begin{example}[正規分布族]
    \TODO{ちゃんと書く}
    $\calP$を$\calX = \R$上の正規分布族とし、
    実現$(V, T, \mu)$を
    $V = \R^2, \;
        T(x) = (x, x^2), \;
        \mu = \lambda$
    とおく。
    これは条件A, Bをみたす。
    真パラメータ空間は
    $\Theta^\calP = \R \times \R_{< 0}$である。
    この実現に関する対数分配関数は
    \begin{equation}
        \psi(\theta)
            = \frac{\mu^2}{2 \sigma^2}
            + \log \sigma
            + \frac{1}{2} \log 2\pi
    \end{equation}
    である。
    ただし$\theta^1 = \frac{\mu}{\sigma^2}, \;
        \theta^2 = -\frac{1}{2 \sigma^2}$
    とおいた
    (したがって
    $\mu = -\frac{\theta^1}{2\theta^2}, \;
        \sigma = \sqrt{-\frac{1}{2\theta^2}}$
    )。
    よって
    Fisher 計量$g = \Hess\psi$を計算すると
    \begin{alignat}{1}
        d\psi
            &=
                \frac{\mu}{\sigma^2}
                d\mu
                + \frac{\sigma^2 - \mu^2}{\sigma^3}
                d\sigma
                \\
            &=
                -\frac{\theta^1}{2\theta^2} d\theta^1
                +\myparen{
                    -\frac{1}{2\theta^2}
                    + \frac{(\theta^1)^2}{4(\theta^2)^2}
                }
                d\theta^2
                \\
        \Hess\psi
            &= Dd\psi \\
            &=
                \myparen{
                    -\frac{1}{2\theta^2}
                    d\theta^1
                    + \frac{\theta^1}{2(\theta^2)^2}
                    d\theta^2
                }
                d\theta^1
                +
                \myparen{
                    \frac{\theta^1}{2(\theta^2)^2}
                    d\theta^1
                    + \myparen{
                        \frac{1}{2(\theta^2)^2}
                        - \frac{(\theta^1)^2}{2(\theta^2)^3}
                    }
                    d\theta^2
                }
                d\theta^2
                \\
            &=
                \frac{1}{\sigma^2} (d\mu)^2
                + \frac{2}{\sigma^2} (d\sigma)^2
    \end{alignat}
    となる。
    ACテンソルは
    \begin{equation}
        ?
    \end{equation}
    となる。
    Levi-Civita 接続$\nabla^{(g)}$の、
    座標$\mu, \sigma$に関する接続係数を求めてみる
    (自然パラメータ座標に関するものではないことに注意)。
    \begin{alignat}{2}
        {\Gamma^{(g)}}_{11}^1
            = 0,
            &\qquad
                {\Gamma^{(g)}}_{12}^1
                    = {\Gamma^{(g)}}_{21}^1
                    = -\frac{1}{\sigma},
            &&\qquad
                {\Gamma^{(g)}}_{22}^1
                    = 0,
            \\
        {\Gamma^{(g)}}_{11}^2
            = \frac{1}{2\sigma},
            &\qquad
                {\Gamma^{(g)}}_{12}^2
                    = {\Gamma^{(g)}}_{21}^2
                    = 0,
            &&\qquad
                {\Gamma^{(g)}}_{22}^2
                    = -\frac{1}{\sigma}
    \end{alignat}
    測地線の方程式は
    \begin{equation}
        \begin{cases}
            x'' - \frac{2}{y} x' y' = 0 \\
            y'' + \frac{1}{2y} (x')^2 - \frac{1}{y} (y')^2 = 0
        \end{cases}
    \end{equation}
    である。
    これを直接解くのは少し大変である。
    その代わりに、
    既知の Riemann 多様体との間の等長同型を利用して測地線を求める。
    $(\Theta, g)$は、
    上半平面$H$に計量
    $\breve{g} = \frac{(dx)^2 + (dy)^2}{2y^2}$
    を入れた Riemann 多様体との間に
    等長同型$(\Theta, g) \to (H, \breve{g}), \;
        (x, y) \mapsto (x, \sqrt{2}y)$
    を持つ。
    Levi-Civita 接続に関する測地線は
    等長同型で保たれるから、
    $(H, \breve{g})$の測地線を求めればよい。
    $(H, \breve{g})$の測地線は、
    $y$軸に平行な直線と
    $x$軸上に中心を持つ半円で尽くされることが知られている。
    これらを等長同型で写して、
    $(\Theta, g)$の測地線として
    $y$軸に平行な直線と
    $x$軸上に長軸を持つ半楕円が得られる。
\end{example}


% ------------------------------------------------------------
%
% ------------------------------------------------------------
\section*{今後の予定}

\begin{itemize}
    \item \TODO{}
\end{itemize}

% ------------------------------------------------------------
%
% ------------------------------------------------------------
\section*{参考文献}

\nocite{amari_information_2016}

{
    \renewcommand{\bibsection}{}
    \bibliographystyle{amsalpha}
    \bibliography{./bibliography,../../mybibliography}
}


\end{document}