\documentclass[report]{jlreq}
\usepackage{global}
\usepackage{./local}
\subfiletrue
\def\assetspath{../}
%\makeindex
\chead{2023/05/16}
\begin{document}

% ============================================================
%
% ============================================================

% ------------------------------------------------------------
%
% ------------------------------------------------------------
\section{振り返りと導入}

前回は期待値と分散を定義した。
本稿では次のことを行う:
\begin{itemize}
    \item 対数分配関数$\psi$の\smooth 性と、微分と積分の順序交換ができることを示す。
    \item 分散の基本的な性質を調べる。
\end{itemize}

前回に引き続き、可測空間$\calX$上の確率測度全体の集合を
$\calP(\calX)$と書くことにする。
また、Einstein の記法を用いる。

% ------------------------------------------------------------
%
% ------------------------------------------------------------
\section{対数分配関数}

本節では
$\calX$を可測空間、
$\calP \subset \calP(\calX)$を$\calX$上の指数型分布族、
$(V, T, \mu)$を$\calP$の次元$m$の実現、
$\Theta \subset V^\vee$を自然パラメータ空間、
$\psi \colon \Theta \to \R$を対数分配関数とする。
$V^\vee$における$\Theta$の内部を$\Theta^\circ$と書くことにする。
さらに
関数$h \colon \calX \times \Theta \to \R$および
$\lambda \colon \Theta \to \R$を
\begin{alignat}{2}
    h(x, \theta)
        &\coloneqq e^{\langle \theta, T(x) \rangle}
        &&\quad ((x, \theta) \in \calX \times \Theta) \\
    \lambda(\theta)
        &\coloneqq \int_\calX h(x, \theta) \, \mu(dx)
        &&\quad (\theta \in \Theta)
\end{alignat}
と定める (つまり$\psi(\theta) = \log \lambda(\theta)$である)。

本節の目標は次の定理を示すことである。

\begin{theorem}[$\lambda$と$\psi$の\smooth 性と積分記号下の微分]
    \label[theorem]{thm:smoothness_of_lambda}
    $\varphi = (\theta_1, \dots, \theta_m) \colon \Theta^\circ \to \R^m$
    を$\Theta^\circ$上のチャートとする。
    このとき、
    任意の$k \in \Z_{\ge 1}, \;
        i_1, \dots, i_k \in \{ 1, \dots, m \}$
    に対し、
    \begin{equation}
        \label{eq:smoothness_of_lambda_1}
        \del_{i_k} \cdots \del_{i_1} \lambda(\theta)
            = \int_\calX
                \del_{i_k} \cdots \del_{i_1} h(x, \theta)
                \, \mu(dx)
            \quad (\theta \in \Theta^\circ)
    \end{equation}
    が成り立つ
    ($\del_i$は$\deldel{\theta_i} \in \Gamma(T\Theta^\circ)$の略記)。
    ただし、
    左辺の微分可能性および
    右辺の可積分性も定理の主張に含まれる。
    とくに$\lambda$および$\psi$は$\Theta^\circ$上の\smooth 関数である。
\end{theorem}

\cref{thm:smoothness_of_lambda}の証明には次の事実を用いる。

\begin{fact}[積分記号下の微分]
    \label[fact]{fact:diff-under-integral}
    $\calY$を可測空間、
    $\nu$を$\calY$上の測度、
    $I \subset \R$を開区間、
    $f \colon \calY \times I \to \R$を
    \begin{enumerate}[label=(\roman*)]
        \item 各$t \in I$に対し$f(\cdot, t) \colon \calY \to \R$が可測
        \item 各$y \in \calY$に対し$f(y, \cdot) \colon I \to \R$が微分可能
    \end{enumerate}
    をみたす関数とする。
    このとき、$f$に関する条件
    \begin{enumerate}
        \item 各$t \in I$に対し
            $f(\cdot, t) \in L^1(\calY, \nu)$である。
        \item ある$\nu$-可積分関数
            $\Phi \colon \calY \to \R$が存在し、
            すべての$t' \in I$に対し
            $\myabs{
                \deldel[f]{t}(y, t')
            } \le \Phi(y) \; \text{a.e.$y$}$
            である。
    \end{enumerate}
    が成り立つならば、
    関数$I \to \R, \; t \mapsto \int_\calY f(y, t) \, \nu(dy)$は微分可能で、
    \begin{equation}
        \deldel{t} \int_\calY f(y, t) \, \nu(dy)
            = \int_\calY \deldel[f]{t}(y, t) \, \nu(dy)
    \end{equation}
    が成り立つ。
    \qed
\end{fact}

\cref{thm:smoothness_of_lambda}の証明において最も重要なステップは、
\cref{fact:diff-under-integral}の前提が満たされることの確認である。
そのための補題を次に示す。

\begin{lemma}[優関数の存在]
    \label[lemma]{lemma:existence_of_dominant_function}
    $e^i \; (i = 1, \dots, m)$を$V^\vee$の基底とし、
    この基底が定める$\Theta^\circ$上のチャートを
    $\varphi = (\theta_1, \dots, \theta_m) \colon \Theta^\circ \to \R^m$
    とおく。
    このとき、
    任意の$k \in \Z_{\ge 1}, \;
        i_1, \dots, i_k \in \{ 1, \dots, m \}$
    に対し、次が成り立つ:
    \begin{enumerate}
        \item 任意の$\theta \in \Theta^\circ$に対し、
            関数
            $\del_{i_k} \cdots \del_{i_1} h(\cdot, \theta)
                \colon \calX \to \R$
            は$L^1(\calX, \mu)$に属する。
        \item 任意の$\theta \in \Theta^\circ$に対し、
            $\Theta^\circ$における$\theta$のある近傍$U$と、
            ある$\mu$-可積分関数$\Phi \colon \calX \to \R$が存在し、
            すべての$\theta' \in U$に対し
            $\myabs{
                \del_{i_k} \cdots \del_{i_1} h(x, \theta')
            } \le \Phi(x) \; \text{a.e.$x$}$
            が成り立つ。
    \end{enumerate}
\end{lemma}

\begin{proof}
    (1)は(2)より直ちに従うから、(2)を示す。
    そこで$\theta \in \Theta^\circ$を任意とする。
    補題の主張は座標$\theta_1, \dots, \theta_m$を
    平行移動して考えても等価だから、
    点$\theta$の座標は
    $\varphi(\theta) = 0 \in \R^m$
    であるとしてよい。

    \uline{Step 1: $U$の構成} \quad
    $\eps > 0$を十分小さく選び、
    $\R^m$内の閉立方体
    \begin{alignat}{1}
        A_{2\eps}
            \coloneqq
            \prod_{i = 1}^m [- 2\eps, 2\eps]
        \quad
        A_{\eps}
            \coloneqq
            \prod_{i = 1}^m [- \eps, \eps]
    \end{alignat}
    が$\varphi(\Theta^\circ)$に含まれるようにしておく。
    すると
    $U \coloneqq \varphi^{-1}(\Int A_{\eps})
        \subset \varphi(\Theta^\circ)$は
    $\theta$の近傍となるが、
    これが求める$U$の条件を満たすことを示す。

    \uline{Step 2: $h$の座標表示} \quad
    まず具体的な計算のために
    $h$の座標表示を求める。
    いま各$\theta' \in U$に対し
    \begin{equation}
        h(x, \theta')
            = \exp\langle \theta', T(x) \rangle
            = \exp\langle \theta_i(\theta') e^i, T(x) \rangle
            = \exp\myparen{\theta_i(\theta') T^i(x)}
    \end{equation}
    が成り立っている。
    ただし
        $T^i \colon \calX \to \R, \;
        x \mapsto \langle e^i, T(x) \rangle \;
        (i = 1, \dots, m)$
    とおいた。
    したがって
    \begin{equation}
        \locallabel{eq:partial-derivative-of-h}
        \del_{i_k} \cdots \del_{i_1} h(x, \theta')
            = T^{i_1}(x) \cdots T^{i_k}(x)
                \exp\myparen{\theta_i(\theta') T^i(x)}
    \end{equation}
    と表せることがわかる。

    \uline{Step 3: $\Phi$の構成} \quad
    $\Phi$を構成するため、
    式\localcref{eq:partial-derivative-of-h}の絶対値を上から評価する。
    表記の簡略化のため
    $t' \coloneqq (t'_1, \dots, t'_m)
        \coloneqq \varphi(\theta')
        \in \R^m$
    とおいておく。
    まず$\frac{k + 1}{\eps} \frac{\eps}{k + 1} = 1$より
    \begin{alignat}{1}
        \myabs{
            T^{i_1}(x) \cdots T^{i_k}(x)
            \exp\myparen{
                \sum_{i = 1}^m
                t'_i T^i(x)
            }
        }
            &=
                \myparen{
                    \frac{k + 1}{\eps}
                }^k
                \myparen{
                    \prod_{\alpha = 1}^k
                        \frac{\eps}{k + 1}
                        |T^{i_\alpha}(x)|
                }
                \exp\myparen{
                    \sum_{i = 1}^m
                    t'_i T^i(x)
                } 
                \locallabel{eq:estimate}
    \end{alignat}
    であり、$\prod$の部分を評価すると
    \begin{alignat}{1}
        \prod_{\alpha = 1}^k
            \frac{\eps}{k + 1}
            |T^{i_\alpha}(x)|
            &\le \prod_{\alpha = 1}^k
                \myparen{
                    \exp\myparen{
                        \frac{\eps}{k + 1}
                        T^{i_\alpha}(x)
                    }
                    + \exp\myparen{
                        - \frac{\eps}{k + 1}
                        T^{i_\alpha}(x)
                    }
                }
                \quad
                (\because s \le e^s + e^{-s} \; (s \in \R))
                \\
            &= \sum_{\sigma \in \{ \pm 1 \}^k}
                \exp\myparen{
                    \sum_{\alpha = 1}^k
                        \frac{\eps}{k + 1}
                        \sigma_\alpha
                        T^{i_\alpha}(x)
                }
                \quad
                (\because \text{式の展開})
                \locallabel{eq:estimate-part}
    \end{alignat}
    (ただし$\sigma_\alpha$は$\sigma$の第$\alpha$成分)
    となるから、
    式\localcref{eq:estimate}と式\localcref{eq:estimate-part}を合わせて
    \begin{alignat}{1}
        \localcref{eq:estimate}
            &\le
                C
                \sum_{\sigma \in \{ \pm 1 \}^k}
                    \exp\myparen{
                        \sum_{\alpha = 1}^k
                            \frac{\eps}{k + 1}
                            \sigma_\alpha
                            T^{i_\alpha}(x)
                    }
                \exp\myparen{
                    \sum_{i = 1}^m
                    t'_i T^i(x)
                }
                \\
            &=
                C
                \sum_{\sigma \in \{ \pm 1 \}^k}
                    \exp\myparen{
                        \sum_{\alpha = 1}^k
                            \frac{\eps}{k + 1}
                            \sigma_\alpha
                            T^{i_\alpha}(x)
                        + \sum_{i = 1}^m
                            t'_i T^i(x)
                    }
                \locallabel{eq:estimate-2}
    \end{alignat}
    となる。
    ただし$C \coloneqq \myparen{\frac{k + 1}{\eps}}^k \in \R_{> 0}$とおいた。
    ここで最終行の$\exp$の中身について、
    各$i = 1, \dots, m$に対し
    $T^i(x)$の係数を評価することで、
    ある$t'' \in A_{2\eps}$が存在して
    \begin{equation}
        \localcref{eq:estimate-2}
            =
                C
                \sum_{\sigma \in \{ \pm 1 \}^k}
                    \exp\myparen{
                        \sum_{i = 1}^m
                            t''_i T^i(x)
                    }
            =
                2^k C
                    \exp\myparen{
                        \sum_{i = 1}^m
                            t''_i T^i(x)
                    }
                \locallabel{eq:estimate-3}
    \end{equation}
    と表せることがわかる。
    そこで
    $|t''_i| \le 2\eps \; (i = 1, \dots, m)$より
    \begin{alignat}{1}
        \localcref{eq:estimate-3}
            &\le
                2^k C
                \prod_{i = 1}^m
                    \myparen{
                        \exp\myparen{
                            2\eps
                            T^i(x)
                        }
                        + \exp\myparen{
                            - 2\eps
                            T^i(x)
                        }
                    }
                \\
            &=
                2^k C
                \sum_{\tau \in \{ \pm 1 \}^m}
                    \exp\myparen{
                        \sum_{i = 1}^m
                            2\eps
                            \tau_i
                            T^i(x)
                    }
    \end{alignat}
    を得る。
    この右辺は
    ($t'$によらないから) $\theta'$によらない$\calX$上の関数であり、
    また$\sum$の各項が
    $2\eps \tau \in A_{2\eps}$ゆえに$\mu$-可積分だから
    式全体も$\mu$-可積分である。
    したがってこれが求める優関数である。
\end{proof}

目標の\cref{thm:smoothness_of_lambda}を証明する。

\begin{proof}[\cref{thm:smoothness_of_lambda}の証明.]
    \cref{thm:smoothness_of_lambda}のステートメントで
    与えられているチャート$\varphi = (\theta_1, \dots, \theta_m)$は
    ($V^\vee$の基底が定めるものとは限らない)
    任意のものであるが、
    実は定理の主張を示すには、
    $V^\vee$の基底をひとつ選び、
    その基底が定めるチャート
    $\wt{\varphi} = (\wt{\theta}_1, \dots, \wt{\theta}_m)$
    に対して定理の主張を示せば十分である。
    その理由は次である:
    \begin{itemize}
        \item 式\cref{eq:smoothness_of_lambda_1}の左辺の微分可能性は、
            $\lambda$が$C^\infty$であればよいから、
            チャート$\wt{\varphi}$で考えれば十分。
        \item 式\cref{eq:smoothness_of_lambda_1}の右辺の可積分性および
            式\cref{eq:smoothness_of_lambda_1}の等号の成立については、
            Leibniz 則より、
            $\lambda$の$\wt{\theta}_1, \dots, \wt{\theta}_m$に関する
            $k$回偏導関数が、
            $\lambda$の$\theta_1, \dots, \theta_m$に関する
            $k$回以下の偏導関数たちの
            ($x$によらない) $C^\infty(\Theta^\circ)$-係数の
            線型結合に書けることから従う。
    \end{itemize}
    そこで、以降$\varphi$は
    $V^\vee$の基底が定めるチャートとする。

    \cref{lemma:existence_of_dominant_function} (1)より、
    式\cref{eq:smoothness_of_lambda_1}の右辺の可積分性はわかっている。
    よって、残りの示すべきことは
    \begin{enumerate}[label=(\roman*)]
        \item 式\cref{eq:smoothness_of_lambda_1}の左辺の微分可能性
        \item 式\cref{eq:smoothness_of_lambda_1}の等号の成立
    \end{enumerate}
    の2点である。

    まず$k = 1, i_k = 1$の場合に(i), (ii)が成り立つことを示す。
    そのためには、
    $t = (t_1, \dots, t_m) \in \varphi(\Theta^\circ)$を任意に固定したとき、
    $t_1$を含む$\R$の十分小さな開区間$I$が存在して、
    関数
    \begin{equation}
        \locallabel{eq:h_restriction}
        g \colon \calX \times I \to \R,
            \quad
            (x, s) \mapsto h(x, \varphi^{-1}(s, t_2, \dots, t_m))
    \end{equation}
    が\cref{fact:diff-under-integral}の仮定(1), (2)をみたすことをいえばよい。

    いま
    $\varphi^{-1}(t) \in \Theta^\circ$だから、
    \cref{lemma:existence_of_dominant_function}(2)のいう
    $\Theta^\circ$における$\varphi^{-1}(t)$の近傍$U$と
    $\mu$-可積分関数$\Phi \colon \calX \to \R$が存在する。
    このとき$\varphi(U)$は$\R^m$における$t$の近傍となるから、
    $t_1$を含む$\R$の十分小さな開区間$I$が存在して
    \begin{equation}
        I \times \{ t_2 \} \times \cdots \times \{ t_m \}
            \subset \varphi(U)
    \end{equation}
    が成り立つ。
    この$I$を用いて定まる関数$g$が
    \cref{fact:diff-under-integral}の仮定(1), (2)をみたすことを確認する。

    まず\cref{lemma:existence_of_dominant_function}の結果(1)より、
    $g$は\cref{fact:diff-under-integral}の仮定(1)をみたす。
    また\cref{lemma:existence_of_dominant_function}の結果(2)より、
    $g$は\cref{fact:diff-under-integral}の仮定(2)をみたす。
    したがって$k = 1, i_k = 1$の場合について(i),(ii)が示された。

    同様にして$i_k = 2, \dots, m$の場合についても示される。
    以降、$k$に関する帰納法で、すべての$k \in \Z_{\ge 1}$および
    $i_1, \dots, i_k \in \{ 1, \dots, m \}$に対して示される。
    これで定理の証明が完了した。
\end{proof}

\cref{thm:smoothness_of_lambda}から次の系が従う。

\begin{corollary}
    $\varphi = (\varphi_1, \dots, \varphi_m) \colon \Theta^\circ \to \R^m$を
    $V^\vee$の基底が定めるチャートとする。
    また、各$\theta \in \Theta$に対し、
    $\calX$上の確率測度$P_\theta$を
    $P_\theta(dx)
        = e^{\langle \theta, T(x) \rangle - \psi(\theta)} \, \mu(dx)$
    と定める。
    このとき、
    任意の$k \in \Z_{\ge 1}, \;
        i_1, \dots, i_k \in \{ 1, \dots, m \}$
    に対し、
    \begin{equation}
        E_{P_\theta}[T^{i_k}(x) \cdots T^{i_1}(x)]
            = \frac{
                \del_{i_k} \cdots \del_{i_1} \lambda(\theta)
            }{
                \lambda(\theta)
            }
            \quad
            (\theta \in \Theta^\circ)
    \end{equation}
    が成り立つ。
    ただし、左辺の期待値の存在も系の主張に含まれる。
    \qed
\end{corollary}

\begin{example}[正規分布族における原点周りのモーメント]
    $\calP$が$\calX = \R$上の正規分布族であるとき、
    任意の$P \in \calP$に対し、
    $P$に関する$x, x^2, \dots$の期待値
    $E_P[x], E_P[x^2], \dots$が存在する。
\end{example}


% ------------------------------------------------------------
%
% ------------------------------------------------------------
\section{分散の性質}

以降、本節では
$\calX$を可測空間、
$V$を$m$次元$\R$-ベクトル空間 ($m \in \Z_{\ge 0}$)、
$p$を$\calX$上の確率測度、
$f \colon \calX \to V$を可測写像とする。

前回の正規分布族の例では、
十分統計量の分散が正定値対称であることをみた。
一般に、分散は次の性質を持つ。

\begin{theorem}[分散の半正定値対称性]
    $f \in L^2(\calX, p; V)$とする。
    このとき、
    $\Var_p[f] \in V \otimes V$は、
    対称かつ半正定値である。
\end{theorem}

\begin{proof}
    まず$\Var_p[f]$が対称であることを示す。
    そこで$V$の基底$e_i \; (i = 1, \dots, m)$をひとつ選んで固定し、
    $f, E_p[f]$の成分表示を
    それぞれ$f^i \colon \calX \to \R$
    および
    $a^i \in \R \; (i = 1, \dots, m)$とおく。
    すると
    \begin{alignat}{1}
        \Var_p[f]
            &= E_p[(f - E_p[f])^2]
            = \myparen{
                \int_\calX
                (f^i(x) - a^i)(f^j(x) - a^j)
                \, p(dx)
            } e_i e_j
    \end{alignat}
    となり、最終行の成分は添字$i, j$の置換に関し不変である。
    したがって$V_p[f]$は対称である。

    つぎに$\Var_p[f]$が半正定値であることを示す。
    示すべきことは、
    $\Var_p[f]$を$V^\vee$上の$\R$-双線型形式とみなして、
    各$\theta \in V^\vee$に対し
    $\Var_p[f](\theta, \theta) \ge 0$が成り立つことであるが、
    これは
    \begin{alignat}{1}
        \Var_p[f](\theta, \theta)
            &= \sum_{i, j = 1}^m
                \myparen{
                    \int_\calX (f^i(x) - a^i)(f^j(x) - a^j) \, p(dx)
                }
                \theta(e_i) \theta(e_j) \\
            &= \int_\calX \myparen{
                \sum_{i = 1}^m
                \theta(e_i)(f^i(x) - a^i)
            }^2 \, p(dx) \\
            &\ge 0
    \end{alignat}
    より従う。
    したがって$\Var_p[f]$は半正定値である。
\end{proof}

分散が0であることの特徴づけを与えておく。

\begin{proposition}[分散が0であるための必要十分条件]
    \label[proposition]{prop:zero_variance_condition}
    $f \in L^2(\calX, p; V)$に関し、
    次は同値である:
    \begin{enumerate}
        \item $\Var_p[f] = 0$
        \item $f$は$p$-a.e.定数
    \end{enumerate}
\end{proposition}

証明には次の事実を用いる。

\begin{fact}
    \label[fact]{fact:nonnegative_func}
    $\calY$を可測空間、
    $\mu$を$\calY$上の測度とする。
    このとき、
    $\mu$-可積分関数$g \colon \calY \to \R$であって
    $g(y) \ge 0 \; \text{$\mu$-a.e. $y \in \calY$}$
    をみたすものに関し、
    次は同値である:
    \begin{enumerate}
        \item $\int_\calY g(y) \, \mu(dy) = 0$
        \item $g(y) = 0 \quad \text{$\mu$-a.e. $y \in \calY$}$
    \end{enumerate}
    \vspace{-2em}
    \qed
\end{fact}

\begin{proof}[\cref{prop:zero_variance_condition}の証明.]
    ここでは「$p$-a.e.」を「a.e.」と略記する。
    $V$の基底$e_i \; (i = 1, \dots, m)$をひとつ選んで固定し、
    $f, E_p[f]$の成分表示を
    それぞれ$f^i \colon \calX \to \R$
    および
    $a^i \in \R \; (i = 1, \dots, m)$とおいておく。

    \uline{(\Leftarrow)} \quad
    $f$がa.e.定数ならば、
    $f^i(x) = a^i \;
        \text{a.e.$x$} \;
        (i = 1, \dots, m)$
    したがって
    $(f^i(x) - a^i)(f^j(x) - a^j) = 0 \;
        \text{a.e.$x$} \;
        (i, j = 1, \dots, m)$
    である。
    よって
    $\int_\calX (f^i(x) - a^i)(f^j(x) - a^j) \, p(dx) = 0 \;
        (i, j = 1, \dots, m)$
    だから
    $\Var_p[f] = 0$である。

    \uline{(\Rightarrow)} \quad
    対偶を示すため、
    $f$はa.e.定数ではないと仮定する。
    すると、
    $f_i$がa.e.定数ではないような
    ある$i \in \{ 1, \dots, m \}$が存在する。
    このとき$(f^i - a^i)^2 = 0 \; \text{a.e.}$ではないから、
    \cref{fact:nonnegative_func}より
    $\int_\calX (f^i(x) - a^i)^2 \, p(dx) > 0$である。
    したがって$\Var_p[f] \neq 0$である。
\end{proof}


% ------------------------------------------------------------
%
% ------------------------------------------------------------
\section{今後の予定}

\begin{itemize}
    \item Hessian の定義と$\psi$の Hessian の正定値性を示す。
    \item KL ダイバージェンスを定義する。
    \item Fisher 計量を定義する。
\end{itemize}

% ------------------------------------------------------------
%
% ------------------------------------------------------------
\section{参考文献}

\nocite{amari_information_2016}
\nocite{dudley_18466_2003}
\nocite{brown_fundamentals_1986}

{
    \renewcommand{\bibsection}{}
    \bibliographystyle{amsalpha}
    \bibliography{./bibliography,../../mybibliography}
}

\end{document}