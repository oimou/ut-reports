\documentclass[report]{jlreq}
\usepackage{global}
\usepackage{./local}
\subfiletrue
\def\assetspath{../}
%\makeindex
\chead{2023/05/16}
\begin{document}

% ============================================================
%
% ============================================================

% ------------------------------------------------------------
%
% ------------------------------------------------------------
\section{振り返りと導入}

前回は期待値と分散を定義した。
本稿では次のことを行う:
\begin{itemize}
    \item 分散の基本的な性質を調べる。
    \item 対数分配関数$\psi$の\smooth 性と、微分と積分の順序交換ができることを示す。
    \item Hessian を定義し、$\psi$の Hessian の正定値性を示す。
\end{itemize}

% ------------------------------------------------------------
%
% ------------------------------------------------------------
\section{分散の性質}

前回の正規分布族の例でみた$V_P[T]$は正定値対称であった。
一般に、分散は次の性質を持つ。

\begin{theorem}[分散の半正定値対称性]
    $V_p[f] \in V \otimes V$は、
    対称かつ半正定値である。
\end{theorem}

\begin{proof}
    まず$V_p[f]$が対称であることを示す。
    \TODO{$V^\vee$の基底をとるべき?}
    そこで$V$の基底$e^i \; (i = 1, \dots, m)$をひとつ選んで固定し、
    $f, E_p[f]$の成分表示を
    それぞれ$f_i, a_i \; (i = 1, \dots, m)$とおく。
    すると
    \begin{alignat}{1}
        V_p[f]
            &= E_p[(f - E_p[f])^2] \\
            &= \myparen{
                \int_\calX
                (f_i(x) - a_i)(f_j(x) - a_j)
                \, p(dx)
            } e^i e^j
    \end{alignat}
    となり、最終行の成分は添字$i, j$の置換に関し不変である。
    したがって$V_p[f]$は対称である。

    つぎに$V_p[f]$が半正定値であることを示す。
    示したいことは、
    $V_p[f]$を$V^\vee$上の$\R$-双線型形式とみなして、
    各$\theta \in V^\vee$に対し
    $V_p[f](\theta, \theta) \ge 0$が成り立つことであるが、
    これは
    \begin{alignat}{1}
        V_p[f](\theta, \theta)
            &= \sum_{i, j}
                \myparen{
                    \int_\calX (f_i(x) - a_i)(f_j(x) - a_j) \, p(dx)
                }
                \theta(e^i) \theta(e^j) \\
            &= \int_\calX \myparen{
                \sum_{i, j}
                \theta(e^i)(f_i(x) - a_i)
                \theta(e^j)(f_j(x) - a_j)
            } \, p(dx) \\
            &= \int_\calX \myparen{
                \sum_{i}
                \theta(e^i)(f_i(x) - a_i)
            }^2 \, p(dx) \\
            &\ge 0
    \end{alignat}
    より従う。
    したがって$V_p[f]$は半正定値である。
\end{proof}

分散が0であることの特徴づけを与えておく。

\begin{proposition}[分散が0であるための必要十分条件]
    \label[proposition]{prop:zero_variance_condition}
    分散を持つ可測写像$f \colon \calX \to V$に関し、
    次は同値である:
    \begin{enumerate}
        \item $V_p[f] = 0$
        \item $f$は$p$-a.e.定数
    \end{enumerate}
\end{proposition}

証明には次の事実を用いる。

\begin{fact}
    \label[fact]{fact:nonnegative_func}
    $\calY$を可測空間、
    $\mu$を$\calY$上の測度とする。
    このとき、
    $\mu$-可積分関数$g \colon \calY \to \R$であって
    $g(y) \ge 0 \; \text{$\mu$-a.e. $y \in \calY$}$
    をみたすものに関し、
    次は同値である。
    \begin{enumerate}
        \item $\int_\calY g(x) \, \mu(dx) = 0$
        \item $g(y) = 0 \quad \text{$\mu$-a.e. $y \in \calY$}$
    \end{enumerate}
    \qed
\end{fact}

\begin{proof}[\cref{prop:zero_variance_condition}の証明.]
    ここでは「$p$-a.e.」を「a.e.」と略記する。
    \TODO{$V^\vee$の基底をとるべき?}
    $V$の基底$e^i \; (i = 1, \dots, m)$をひとつ選んで固定し、
    この基底に関する
    $f$の成分を$f_i \colon \calX \to \R \; (i = 1, \dots, m)$、
    $E_p[f]$の成分を$a_i \in \R \; (i = 1, \dots, m)$とおいておく。

    \uline{(\Leftarrow)} \quad
    $f$がa.e.定数ならば、
    $f_i(x) = a_i \;
        \text{a.e.$x$} \;
        (i = 1, \dots, m)$
    したがって
    $(f_i(x) - a_i)(f_j(x) - a_j) = 0 \;
        \text{a.e.$x$} \;
        (i, j = 1, \dots, m)$
    である。
    よって
    $\int_\calX (f_i(x) - a_i)(f_j(x) - a_j) \, p(dx) = 0 \;
        (i, j = 1, \dots, m)$
    だから
    $V_p[f] = 0$である。

    \uline{(\Rightarrow)} \quad
    対偶を示すため、
    $f$はa.e.定数ではないと仮定する。
    すると、
    $f_i$がa.e.定数ではないような
    ある$i \in \{ 1, \dots, m \}$が存在する。
    このとき$(f_i - a_i)^2 = 0 \; \text{a.e.}$ではないから、
    \cref{fact:nonnegative_func}より
    $\int_\calX (f_i(x) - a_i)^2 \, p(dx) > 0$である。
    したがって$V_p[f] \neq 0$である。
\end{proof}

% ------------------------------------------------------------
%
% ------------------------------------------------------------
\section{対数分配関数}

本節では、対数分配関数が
自然パラメータ空間の内部において\smooth であって、
積分記号下の微分が可能であることを示したい。

\TODO{$\Theta$は内部を持つだろうか?}

以降、本節では
$\calX$を可測空間、
$\calP \subset \calP(\calX)$を$\calX$上の指数型分布族、
$(V, T, \mu)$を$\calP$の次元$m$の実現、
$\Theta \subset V^\vee$を自然パラメータ空間、
$\psi \colon \Theta \to \R$を対数分配関数とする。
$V^\vee$における$\Theta$の内部を$\Theta^\circ$と書くことにする。
さらに
関数$h \colon \calX \times \Theta \to \R$および
$\lambda \colon \Theta \to \R$を
\begin{alignat}{2}
    h(x, \theta)
        &\coloneqq e^{\langle \theta, T(x) \rangle}
        &&\quad ((x, \theta) \in \calX \times \Theta) \\
    \lambda(\theta)
        &\coloneqq \int_\calX h(x, \theta) \, \mu(dx)
        &&\quad (\theta \in \Theta)
\end{alignat}
と定める (つまり$\lambda(\theta) = e^{\psi(\theta)}$である)。

\TODO{
\cite{brown_fundamentals_1986}
}

\begin{lemma}[優関数の存在]
    $\varphi = (\theta_1, \dots, \theta_m) \colon \Theta^\circ \to \R^m$
    を$\Theta^\circ$上のチャートとする。
    このとき、
    任意の$\theta \in \Theta^\circ$および
    $k \in \Z_{\ge 0}, \;
        i_1, \dots, i_k \in \{ 1, \dots, m \}$
    に対し、
    $\Theta^\circ$における$\theta$のある近傍$U$が存在して、
    $\calX$上の関数族
    \begin{equation}
        \mybrace{
            \frac{\del^k h}{\del \theta_{i_1} \dots \del \theta_{i_k}}(x, \theta')
        }_{\theta' \in U}
    \end{equation}
    はある$\mu$-可積分関数$\calX \to \R$により支配される。
\end{lemma}

\begin{proof}
    $\theta \in \Theta^\circ, \;
        k \in \Z_{\ge 0}, \;
        i_1, \dots, i_k \in \{ 1, \dots, m \}$
    とし、
    $\theta$の座標を
    $t = (t_1, \dots, t_m) \coloneqq \varphi(\theta) \in \R^m$
    とおく。
    $\Theta^\circ$における$\theta$の開近傍$U'$をひとつ選ぶと、
    $\varphi(U')$は$\R^m$における$t$の開近傍となる。
    そこで、
    $\eps > 0$を十分小さく選び、
    $\R^m$内の開立方体
    $A_{4\eps}
        \coloneqq
        \prod_{i = 1}^m (t_i - 2\eps, t_i + 2\eps)$
    が$\varphi(U')$に含まれるようにしておく。
    さらに$\R^m$内の開立方体$A_{2\eps}$を
    $A_{2\eps}
        \coloneqq
        \prod_{i = 1}^m (t_i - \eps, t_i + \eps)$
    と定める。
    すると$U \coloneqq \varphi^{-1}(A_{2\eps}) \subset U'$は
    $\Theta^\circ$における$\theta$の近傍となるが、
    これが求める$U$の条件を満たすことを示す。

    まず$h$の偏導関数の座標表示を求める。
    そこで、自然な同一視により
    $\deldel{\theta_i}_\theta \in T_\theta \Theta^\circ \;
        (i = 1, \dots, m)$
    を$V^\vee$の基底とみなしたものを
    $e^i \in V^\vee \; (i = 1, \dots, m)$
    とおいておく。
    すると各$\theta' \in U$に対し、
    $\theta'$の座標を
    $t' = (t'_1, \dots, t'_m) \coloneqq \varphi(\theta') \in A_{2\eps}$
    として
    \begin{equation}
        h(x, t')
            = e^{\langle t', T(x) \rangle}
            = e^{\langle t'_i e^i, T(x) \rangle}
            = e^{t'_i T^i(x)}
    \end{equation}
    が成り立つ
    (ただし
        $T^i \colon \calX \to \R, \;
        x \mapsto \langle e^i, T(x) \rangle \;
        (i = 1, \dots, m)$
        とおいた)。
    したがって
    \begin{equation}
        \locallabel{eq:partial-derivative-of-h}
        \frac{\del^k h}{\del \theta_{i_1} \dots \del \theta_{i_k}}(x, \theta')
            = T^{i_1}(x) \cdots T^{i_k}(x) e^{t'_i T^i(x)}
    \end{equation}
    と表せることがわかる。

    次に、式\localcref{eq:partial-derivative-of-h}の絶対値を上から評価する。
    \begin{alignat}{1}
        |T^{i_1}(x) \cdots T^{i_k}(x) e^{t'_i T^i(x)}|
            &=
                \myparen{
                    \frac{k + 1}{\eps}
                }^k
                \exp\myparen{t'_i T^i(x)} 
                \prod_{\alpha = 1}^k
                    \frac{\eps}{k + 1}
                    |T^{i_\alpha}(x)|
                \locallabel{eq:estimate}
    \end{alignat}
    であり、末尾の部分は
    \begin{alignat}{1}
        \prod_{\alpha = 1}^k
            \frac{\eps}{k + 1}
            |T^{i_\alpha}(x)|
            &\le \prod_{\alpha = 1}^k
                \myparen{
                    \exp\myparen{
                        \frac{\eps}{k + 1}
                        T^{i_\alpha}(x)
                    }
                    + \exp\myparen{
                        - \frac{\eps}{k + 1}
                        T^{i_\alpha}(x)
                    }
                }
                \\
            &= \sum_{(\sigma_1, \dots, \sigma_k) \in \{ \pm 1 \}^k}
                \exp\myparen{
                    \sum_{\alpha = 1}^k
                        \frac{\eps}{k + 1}
                        \sigma_\alpha
                        T^{i_\alpha}(x)
                }
    \end{alignat}
    となるから、
    \begin{alignat}{1}
        \localcref{eq:estimate}
            &\le
                \myparen{
                    \frac{k + 1}{\eps}
                }^k
                \exp\myparen{t'_i T^i(x)} 
                \sum_{(\sigma_1, \dots, \sigma_k) \in \{ \pm 1 \}^k}
                    \exp\myparen{
                        \sum_{\alpha = 1}^k
                            \frac{\eps}{k + 1}
                            \sigma_\alpha
                            T^{i_\alpha}(x)
                    }
                \\
            &=
                \myparen{
                    \frac{k + 1}{\eps}
                }^k
                \sum_{(\sigma_1, \dots, \sigma_k) \in \{ \pm 1 \}^k}
                    \exp\myparen{
                        \sum_{\alpha = 1}^k
                            \frac{\eps}{k + 1}
                            \sigma_\alpha
                            T^{i_\alpha}(x)
                        + \sum_{i = 1}^m
                            t'_i T^i(x)
                    }
                \\
            &=
                \myparen{
                    \frac{k + 1}{\eps}
                }^k
                \sum_{\sigma = (\sigma_1, \dots, \sigma_k) \in \{ \pm 1 \}^k}
                    \exp\myparen{
                        \sum_{i = 1}^m
                            t''_{\sigma, i} T^i(x)
                    }
                \locallabel{eq:estimate-2}
    \end{alignat}
    となる。ただし最終行の$t''_{\sigma, i}$は各$i = 1, \dots, m$に対し
    \begin{equation}
        t''_{\sigma, i} \coloneqq
            t_i + \sum_{\substack{
                \alpha \in \{ 1, \dots, k \} \\
                i_\alpha = i
            }}
                \frac{\eps}{k + 1}
                \sigma_\alpha
            \in \R
    \end{equation}
    とおいた。
    ここで、$t''_\sigma = (t''_{\sigma, 1}, \dots, t''_{\sigma, m}) \in \R^m$は
    $A_{4\eps}$に属している。
    なぜならば、各$i = 1, \dots, m$に対し
    \begin{alignat}{1}
        |t''_{\sigma, i} - t_i|
            &\le |t'_i - t_i| + \sum_{\substack{
                \alpha \in \{ 1, \dots, k \} \\
                i_\alpha = i
            }}
                \frac{\eps}{k + 1} \\
            &< \eps + \eps \\
            &= 2\eps
    \end{alignat}
    が成り立つからである。
    したがって
    \localcref{eq:estimate-2}は
    $\calX$上の$\mu$-可積分関数の$\R$-線型結合だから
    $\mu$-可積分であり、
    また ($t'$によらないから) $\theta'$によらない。
    したがってこれが求める優関数である。
\end{proof}

\begin{theorem}[積分記号下の微分]
    $\lambda$に関し次が成り立つ。
    \begin{enumerate}
        \item $\lambda$は$\Theta^\circ$上$C^\infty$級である。
        \item 各$\theta \in \Theta^\circ$と
            $\theta$の近傍上で定義された
            任意のベクトル場$X^{(1)}, \dots, X^{(n)} \; (n \in \Z_{\ge 1})$
            に対し
            \begin{equation}
                (X^{(1)} \cdots X^{(n)} \lambda)(\theta)
                    = \int_\calX
                        (X^{(1)} \cdots X^{(n)} h)(x, \theta) \, \mu(dx)
            \end{equation}
            が成り立つ。
    \end{enumerate}
\end{theorem}

\begin{proof}
    \TODO{}
\end{proof}

% ------------------------------------------------------------
%
% ------------------------------------------------------------
\section{Hessian}

本節では、
$\R^n$の開部分集合上の関数に対する Hessian の概念を、
一般の有限次元$\R$-ベクトル空間$W$の開部分集合$U$上の関数にまで拡張したい。
そこで、まず$U$上に平坦アファイン接続を導入しておく。

以降、本節では
$W$を$m$次元$\R$-ベクトル空間 ($m \in \Z_{\ge 0}$)、
$U \opensubset W$を開部分集合とする。

\begin{definition}[$W^\vee$の基底が定める$U$上の座標]
    $W^\vee$の任意の基底$(f^i)_{i = 1}^m, \; f^i \colon W \to \R$に対し、
    その$U$上への制限
    $(x^i)_{i = 1}^m, \; x^i \colon U \to \R$は
    $U$上の座標となる。
    この$(x^i)_i$を、
    本稿だけの呼び方として、
    \termsilent{$W^\vee$の基底$(f^i)_i$が定める$U$上の座標}
    と呼ぶことにする。
\end{definition}

\begin{propdef}[$W$の開部分多様体としての$U$上の平坦アファイン接続]
    $U$上のアファイン接続
    $D \colon \Gamma(TU) \to \Gamma(T^\vee U \otimes TU)$を、
    次の規則で well-defined に定めることができる:
    \begin{itemize}
        \item 各$X \in \Gamma(TU)$に対し、
            $W^\vee$の基底が定める$U$上の座標$(x^i)_{i = 1}^m$をひとつ選び、
            この座標に関する$X$の成分を$X^i \in \smooth(U)$とおいて
            \begin{equation}
                \label{eq:def_connection_d}
                DX
                    \coloneqq dX^i \otimes \deldel{x^i}
                    \in \Gamma(T^\vee U \otimes TU)
            \end{equation}
            と定める。
    \end{itemize}
    さらに、この$D$は$U$上のアファイン接続として平坦である。

    以上により定まる$D$を、
    本稿だけの呼び方として、
    \termsilent{$W$の開部分多様体としての
        $U$上の平坦アファイン接続}
    と呼ぶことにする。
\end{propdef}

\begin{proof}
    写像として well-defined であることを一旦認め、
    先に$\R$-線型性、Leibniz 則、平坦性を確かめる。
    $D$の$\R$-線型性と Leibniz 則は、
    外微分$d$の$\R$-線型性と Leibniz 則から従う。
    平坦性、
    すなわち$D$の接続係数がすべて$0$になるような座標の存在は、
    \cref{eq:def_connection_d}より明らかである。
    最後に、$D$が写像として well-defined であることを示す。
    \TODO{well-defined 性}
\end{proof}

\TODO{1-form の共変微分をどう定義するかの remark?}

\begin{definition}[Hessian]
    \TODO{Fr\'echet 微分で定義すべき?しかしノルムは入ってない。}
    \smooth 関数$f \colon U \to \R$に対し、
    $f$の\term{Hessian}{Hessian}[Hessian]
    $\Hess f \in \Gamma(T^\vee U \otimes T^\vee U)$を
    \begin{equation}
        \Hess f \coloneqq Ddf
    \end{equation}
    と定義する。
\end{definition}

\begin{proposition}
    Hessian は対称テンソルである。
\end{proposition}

\begin{proof}
    $W^\vee$の基底が定める$U$上の座標$(x^i)_{i = 1}^m$をひとつ選び、
    この座標に関する$\Hess f$の成分表示を具体的に計算する。
    そこで$\deldel{x^i}$を$\del_i$と略記することにすれば
    \begin{alignat}{1}
        (\Hess f)(\del_i, \del_j)
            &= (D_{\del_i} df)(\del_j) \\
            &= \del_i (df(\del_j)) - df(\underbrace{D_{\del_i} \del_j}_{= 0}) \\
            &= \del_i (\del_j f) \\
            &= \frac{\del^2 f}{\del x^i \del x^j}
    \end{alignat}
    となり、
    いま$f$が\smooth であることより
    これは添字$i, j$の置換に関し対称である。
    したがって$\Hess f$は対称テンソルである。
\end{proof}

\begin{proposition}
    $U$がさらに$W$の凸集合であるとする。
    このとき、
    $\Hess f$が正定値ならば
    $f$は$U$上で狭義凸である。
    逆は成立しない。
\end{proposition}

\begin{proof}
    \TODO{}
\end{proof}

\begin{theorem}[対数分配関数の Hessian の正定値性]
    すべての$\theta \in \Int\Theta$に対し、
    $\Hess_{\theta} \psi
        = \myparen{
            \frac{\del^2 \psi}{\del\theta_i \del\theta_j} (\theta)
        }_{i, j}$
    は非負定値かつ対称である。
    さらに$(T, \mu)$が最小表現ならば、
    $\Hess_\theta \psi$は
    正定値である。
\end{theorem}

\begin{proof}
    $\psi(\theta) = \log \int_\calX e^{\langle \theta, T(x) \rangle} \mu(dx)$
    だから、直接計算により
    \begin{alignat}{1}
        \deldel[\psi]{\theta_i}(\theta)
            &= \int_\calX
                T_i(x)
                e^{\langle \theta, T(x) \rangle - \psi(\theta)}
                \, \mu(dx), \\
        \frac{\del^2 \psi}{\del\theta_i \del\theta_j} (\theta)
            &= \int_\calX
                T_i(x) T_j(x)
                e^{\langle \theta, T(x) \rangle - \psi(\theta)}
                \, \mu(dx) \nonumber \\
            &\qquad - \myparen{
                \int_\calX
                    T_i(x)
                    e^{\langle \theta, T(x) \rangle - \psi(\theta)}
                    \, \mu(dx)
            }
            \myparen{
                \int_\calX
                    T_j(x)
                    e^{\langle \theta, T(x) \rangle - \psi(\theta)}
                    \, \mu(dx)
            } \\
            &= E_\theta[T_i(x) T_j(x)] - E_\theta[T_i(x)] E_\theta[T_j(x)] \\
            &= E_\theta[T_i(x) T_j(x)] - \alpha_i \alpha_j \\
            &= E_\theta[(T_i(x) - \alpha_i)(T_j(x) - \alpha_j)]
    \end{alignat}
    を得る。
    ただし$\alpha_k \coloneqq E_\theta[T_k(x)] \; (k = 1, \dots, m)$とおいた。
    よって
    $\Hess_\theta \psi
        = \myparen{
            \frac{\del^2 \psi}{\del\theta_i \del\theta_j} (\theta)
        }_{i, j}$
    は$\theta$に関する$T$の分散$V_\theta[T(x)]$に一致する。
    したがって
    \cref{lem:variance-nonnegative-definite-and-symmetric}
    より$\Hess_\theta \psi$は非負定値かつ対称である。
    \TODO{}
\end{proof}

\begin{corollary}[対数分配関数の凸性]
    \TODO{正定値性より従う。あるいは H\"older}
\end{corollary}

\begin{proof}
    \TODO{}
\end{proof}



% ------------------------------------------------------------
%
% ------------------------------------------------------------
\section{今後の予定}

\begin{itemize}
    \item KL ダイバージェンス
    \item Fisher 計量
    \item アファイン接続
\end{itemize}

% ------------------------------------------------------------
%
% ------------------------------------------------------------
\section{参考文献}

\nocite{amari_information_2016}
\nocite{bn1970_pdf}

{
    \renewcommand{\bibsection}{}
    \bibliographystyle{amsalpha}
    \bibliography{./bibliography,../../mybibliography}
}


\end{document}