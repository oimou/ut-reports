\documentclass[report]{jlreq}
\usepackage{global}
\usepackage{./local}
\subfiletrue
\def\assetspath{../}
%\makeindex
\chead{2023/04/11}
\begin{document}

% ============================================================
%
% ============================================================

% ------------------------------------------------------------
%
% ------------------------------------------------------------
\section{Introduction}

情報幾何学とは、
可微分多様体上の双対構造と呼ばれる幾何学的構造を扱う、
幾何学の一分野である。
情報幾何の重要な応用先のひとつは統計的推定であり、
推定の問題を幾何学的に捉えなおすことで、より深い理解につながることが期待される。

情報幾何が統計的推定へどのように応用されるかをみるために、
本稿では指数型分布族および混合型分布族について説明する。
指数型/混合型分布族とは、ある可測空間上の確率測度からなる族の一種である。
指数型分布族は、正規分布や Poisson 分布など基本的な確率分布が属するクラスであり、
数学的にもよい性質を備えている\TODO{どのような?}。
混合型分布族は、いくつかの確率分布の重み付け和からなる族であり、
応用上重要なクラスである。

指数型/混合型分布族は確率密度関数の形によって定義される。
そこで本稿では、まず測度論の基本事項を復習した後、
確率空間や確率変数などの確率論の基本事項を整理する。
その後、指数型分布族と混合型分布族の定義および具体例を与える。

% ------------------------------------------------------------
%
% ------------------------------------------------------------
\section{Radon-Nikod\'ym の定理と H\"older の不等式}

Radon-Nikod\'ym の定理について復習する。

\begin{definition}[絶対連続]
    $(\calX, \calB)$を可測空間、
    $\mu, \nu$を$\calX$上の測度とする。
    $\nu$が$\mu$に関し
    \term{絶対連続}[absolutely continuous]{絶対連続}[ぜったいれんぞく]
    であるとは、
    任意の$E \in \calB$に対し
    $\mu(E) = 0$ならば$\nu(E) = 0$が成り立つことをいう。
\end{definition}

\begin{theorem}[Radon-NIkod\'ym の定理]
    $(X, \calB)$を可測空間、
    $\mu$を$X$上の$\sigma$-有限測度、
    $\nu$を$X$上の測度とする。
    このとき、
    $\nu$が$\mu$に関して絶対連続であるための必要十分条件は、
    $\mu$-a.e. $x \in X$に対し定義された
    可積分関数$f$が存在して
    \begin{equation}
        \nu(E) = \int_E f(x) \, d\mu(x)
            \quad
            (E \in \calB)
    \end{equation}
    が成り立つことである。
    この$f$を$\mu$に関する$\nu$の
    \term{Radon-Nikod\'ym 微分}[Radon-Nikod\'ym derivative]
        {Radon-Nikod\'ym 微分}[Radon-Nikod\'ym びぶん]
    といい、
    $\frac{d\nu}{d\mu}$と書く。
\end{theorem}

Radon-Nikod\'ym 微分のイメージとしては、
$\mu$の変化に対する$\nu$の変化率を表している。

\begin{proof}
    関数族の$\sup$として$f$を構成する。
\end{proof}

H\"older の不等式について復習する。

\begin{proposition}[H\"older の不等式]
    $(\calX, \calB)$を可測空間、
    $\mu$を$\calX$上の測度とする。
    $1 < p < \infty$、$q = p(p - 1)^{-1}$、
    $f$を$p$乗$\mu$-可積分関数、
    $g$を$q$乗$\mu$-可積分関数とする。
    このとき、$fg$は$\mu$-可積分であり、かつ
    \begin{equation}
        \int_\calX |fg| \mu(dx)
            \le \myparen{
                \int_\calX |f|^{p} \mu(dx)
            }^{\frac{1}{p}}
            \myparen{
                \int_\calX |g|^{q} \mu(dx)
            }^{\frac{1}{q}}
    \end{equation}
    が成り立つ。
\end{proposition}

\begin{proof}
    Young の不等式を使う。
\end{proof}

% ------------------------------------------------------------
%
% ------------------------------------------------------------
\section{確率論の基本事項}

確率論の基本事項を整理する。

\subsection{確率空間}

\begin{definition}[確率空間]
    測度空間$(\Omega, \calF, P)$であって
    \begin{enumerate}
        \item 各$E \in \calF$に対し$P(E) \ge 0$
        \item $P(\Omega) = 1$
    \end{enumerate}
    をみたすものを
    \term{確率空間}[probability space]{確率空間}[かくりつくうかん]
    といい、
    $P$を$(\Omega, \calF)$上の
    \term{確率測度}[probability measure]{確率測度}[かくりつそくど]
    あるいは
    \term{確率分布}[probability distribution]{確率分布}[かくりつぶんぷ]
    という。
\end{definition}

\begin{definition}[確率変数]
    $(\Omega, \calF, P)$を確率空間、
    $(\calX, \calA)$を可測空間とする。
    可測関数$X \colon (\Omega, \calF) \to (\calX, \calA)$を
    $(\calX, \calA)$に値をもつ
    \term{確率変数}[random variable; r.v.]{確率変数}[かくりつへんすう]
    という。
\end{definition}

\begin{definition}[確率変数の確率分布]
    $(\Omega, \calF, P)$を確率空間、
    $X \colon (\Omega, \calF) \to (\calX, \calA)$を確率変数とする。
    このとき、写像
    \begin{equation}
        P^X \colon \calA \to [0, +\infty),
            \quad
            E \mapsto P(X^{-1}(E))
            \quad
            (E \in \calA)
    \end{equation}
    は$(\calX, \calA)$上の確率測度となる。
    これを
    \term{$X$の確率分布}[probability distribution of $X$]
        {確率分布!確率変数の---}[かくりつぶんぷ]
    という。

    $X$の確率分布が$(\calX, \calA)$上のある確率分布$\nu$に等しいとき、
    $X$は
    \term{$\nu$に従う}{確率分布に従う}[かくりつぶんぷにしたがう]
    という。
\end{definition}

\begin{definition}[確率密度関数]
    $(\calX, \calA)$を可測空間、
    $\mu$を$\calX$上の$\sigma$-有限測度、
    $\nu$を$\mu$に関し絶対連続な$(\calX, \calA)$上の確率測度とする。
    このとき、
    $\nu$の$\mu$に関する Radon-NIkod\'ym 微分
    $\dd[\nu]{\mu}$を、
    $\nu$の\term{確率密度関数}[probability density function; PDF]
        {確率密度関数}[かくりつみつどかんすう]
    という。
\end{definition}



% ------------------------------------------------------------
%
% ------------------------------------------------------------
\section{指数型分布族と混合型分布族}

\subsection{指数型分布族}

指数型分布族について述べる。

\begin{definition}[指数型分布族]
    $(\calX, \calA)$を可測空間、
    $\mu$を$(\calX, \calA)$上の$\sigma$-有限測度、
    $\Theta \subset \R^n$を部分集合、
    $\calP = (P_\theta)_{\theta \in \Theta}$を
    $\calX$上の確率分布族とする。
    ここで、ある
    \begin{itemize}
        \item $m \in \Z_{\ge 1}$
        \item $\calX$上の可測関数$g \colon \calX \to \R_{\ge 0}$
        \item $\calX$上の可測関数$T_i \colon \calX \to \R \; (i = 1, \dots, m)$
        \item $\Theta$上の関数$\psi \colon \Theta \to \R$
            および$a_i \colon \Theta \to \R \; (i = 1, \dots, m)$
    \end{itemize}
    が存在し、各$P_\theta$が$\mu$に関し絶対連続で、
    確率密度関数が
    \begin{equation}
        \locallabel{eq:expfam}
        \frac{dP_\theta}{d\mu}(x)
            = g(x) \exp\myparen{
                \sum_{i = 1}^m a_i(\theta) T_i(x)
                - \psi(\theta)
            }
            \quad
            \text{$\mu$-a.e. $x \in \calX$}
    \end{equation}
    となるとき、
    $\calP$を\term{指数型分布族}[exponential family]
        {指数型分布族}[しすうがたぶんぷぞく]
    という。
    \localcref{eq:expfam}を
    \begin{equation}
        P_\theta(dx) = g(x) \exp\myparen{
            \sum_{i = 1}^m a_i(\theta) T_i(x)
            - \psi(\theta)
        } \mu(dx)
    \end{equation}
    とも書く。
    \begin{itemize}
        \item $g \colon \calX \to \R_{\ge 0}$
        \item $a_i(\theta) = \theta_i \; (i = 1, \dots, m)$のとき
            \term{正準形}[canonical form]{正準形}[せいじゅんけい]
            という。
            $a_i(\theta) \; (i = 1, \dots, m)$を
            \term{自然パラメータ}[natural parameter]{自然パラメータ}[しぜんぱらめーた]
            という。
        \item 十分統計量$T_i \colon \calX \to \Theta$
        \item 対数分配関数$\psi \colon \Theta \to \R$
    \end{itemize}
\end{definition}

対数分配関数の形は$g, a_i, T_i$によって決まる。

\begin{proposition}[対数分配関数の式]
    対数分配関数$\psi$は
    \begin{equation}
        \psi(\theta)
            = \log \int_\calX g(x) \exp\myparen{
                \langle a(\theta), T(x) \rangle
            } \mu(dx)
    \end{equation}
    と表せる\footnote{
        $e^{\psi(\theta)} = \int_\calX g(x) \exp\myparen{
            \langle a(\theta), T(x) \rangle
        } \mu(dx)$は0より大きい実数だから$\psi$は確かに定義される。
    }。
    ただし$\langle a(\theta), T(x) \rangle$は
    $\sum_{i = 1}^m a_i(\theta) T_i(x)$の意味である。
\end{proposition}

\begin{proof}
    $P_\theta$が確率測度であることより
    \begin{equation}
        1
            = P_\theta(\calX)
            = \int_\calX g(x) \exp\myparen{
                \langle a(\theta), T(x) \rangle
                - \psi(\theta)
            } \mu(dx)
    \end{equation}
    が成り立つ。$\exp(\psi(\theta))$を両辺にかけて
    \begin{equation}
        \exp(\psi(\theta))
            = \int_\calX g(x) \exp\myparen{
                \langle a(\theta), T(x) \rangle
            } \mu(dx)
    \end{equation}
    となり、対数をとって命題の式が得られる。
\end{proof}

以下に指数型分布族に関する具体例を挙げる。

\begin{example}[正規分布]
    $(\calX, \calA) = (\R, \calB(\R))$、
    $\mu$は$\R$上の Lebesgue 測度、
    $\Theta = \R \times \R_{> 0}$の場合を考える。
    $(\mu, \sigma^2) \in \Theta$に対し、
    \begin{equation}
        P_{(\mu, \sigma^2)}(dx)
            = \frac{1}{\sqrt{2\pi\sigma^2}} \exp\myparen{
                -\frac{(x - \mu)^2}{2\sigma^2}
            } \mu(dx)
    \end{equation}
    で定義される確率分布$P_{(\mu, \sigma^2)}$を\term{正規分布}[normal distribution]
        {正規分布}[せいきぶんぷ]
    という。
    このとき$\{ P_{\mu, \sigma^2} \}_{(\mu, \sigma^2) \in \Theta}$が指数型分布族であることを確かめる。
    \begin{alignat}{1}
        P_{(\mu, \sigma^2)}(dx)
            &= \frac{1}{\sqrt{2\pi\sigma^2}} \exp\myparen{
                -\frac{(x - \mu)^2}{2\sigma^2}
            } \mu(dx) \\
            &= \exp\myparen{
                -\frac{1}{2\sigma^2} (x^2 - 2\mu x + \mu^2)
                -\frac{1}{2} \log 2\pi\sigma^2
            } \mu(dx) \\
            &= \exp\myparen{
                \begin{bmatrix}
                    -\frac{1}{2\sigma^2} & \frac{\mu}{\sigma^2}
                \end{bmatrix}
                \begin{bmatrix}
                    x^2 \\ x
                \end{bmatrix}
                - \frac{\mu^2}{2\sigma^2}
                - \frac{1}{2} \log 2\pi\sigma^2
            } \mu(dx)
    \end{alignat}
    となるから、
    $g(x) = 1, \;
        a_1(\mu, \sigma^2) = -\frac{1}{2\sigma^2}, \;
        a_2(\mu, \sigma^2) = \frac{\mu}{\sigma^2}, \;
        T_1(x) = x^2, \;
        T_2(x) = x, \;
        \psi(\mu, \sigma^2) = \frac{\mu^2}{2\sigma^2} + \frac{1}{2} \log 2\pi\sigma^2$
    とおけばよいことがわかる。
    \TODO{十分統計量から自然パラメータを推定できる?}
\end{example}

\begin{example}[Poisson 分布]
    $(\calX, \calA) = (\N, 2^\N)$、
    $\mu$は$\N$上の数え上げ測度、
    $\Theta = \R_{> 0}$の場合を考える。
    \begin{equation}
        P_\lambda(dk)
            = \frac{\lambda^k}{k!} \exp(-\lambda) \mu(dk)
    \end{equation}
    で定義される確率分布$P_\lambda$を\term{Poisson 分布}[Poisson distribution]
        {Poisson 分布}[Poisson 分布]
    という。
    このとき$\{ P_\lambda \}_{\lambda > 0}$が指数型分布族であることを確かめる。
    \begin{alignat}{1}
        P_\lambda(dk)
            &= \frac{\lambda^k}{k!} \exp(-\lambda) \mu(dk) \\
            &= \frac{1}{k!} \exp\myparen{
                k \log\lambda - \lambda
            } \mu(dk)
    \end{alignat}
    となるから、
    $g(k) = \frac{1}{k!}, \;
        a_1(\lambda) = \log\lambda, \;
        T_1(k) = k, \;
        \psi(\lambda) = \lambda$
    とおけばよいことがわかる。
\end{example}

\begin{example}[指数型分布族でない例]
    $(\calX, \calA) = (\R, \calB(\R))$、
    $\mu$は$\R$上の Lebesgue 測度、
    $\Theta = \R_{> 0}$の場合を考える。
    $\R$上の一様分布の族
    \begin{equation}
        \locallabel{eq:uniform}
        \mybrace{ P_\theta }_{\theta > 0},
            \qquad
            \dd[P_\theta]{\mu}(x) = \begin{cases}
                \frac{1}{\theta} & 0 \leq x \leq \theta \\
                0 & \text{otherwise}
            \end{cases}
    \end{equation}
    を考える。
    これは指数型分布族ではないことを確かめる。
    もし指数型分布族であったとすると
    $\dd[P_\theta]{\mu}(x) = g(x) e^{\varphi(x, \theta)}$
    と表せる ($g$は$\theta$によらない関数)。
    そこでとくに$\theta = 1, x = 2$に対し
    \begin{equation}
        0 = \dd[P_1]{\mu}(2) = g(2) e^{\varphi(2, 1)}
    \end{equation}
    より$g(2) = 0$となる。一方$\theta = 2, x = 2$に対し
    \begin{equation}
        \frac{1}{2} = \dd[P_2]{\mu}(2) = g(2) e^{\varphi(2, 2)}
    \end{equation}
    より$g(2) \neq 0$となり矛盾が従う。
    よって\localcref{eq:uniform}は指数型分布族ではない。
\end{example}

\begin{example}[Cauchy 分布]
    \TODO{}
\end{example}

さて指数型分布族の定義では$g, a_i, T_i, \psi$などいくつかの関数が出てきた。
ここではとくに$\psi$に着目してその性質を調べる。

\begin{proposition}[対数分配関数の凸性]
    $\Theta \subset \R^n$が凸集合で
    $a_i(\theta) = \theta_i \; (i = 1, \dots, n)$ならば、
    対数分配関数$\psi$は凸関数である。
\end{proposition}

\begin{proof}
    示したいことは
    \begin{equation}
        \psi(t\theta + (1 - t)\theta')
            \leq t \psi(\theta) + (1 - t) \psi(\theta')
            \quad
            (\forall t \in (0, 1), \; \theta, \theta' \in \Theta)
    \end{equation}
    である。
    \begin{alignat}{1}
        \psi(t\theta + (1 - t)\theta')
            &= \log \int_\calX
                g(x)
                \exp \langle t\theta + (1 - t)\theta', T(x) \rangle
                \mu(dx) \\
            &= \log \int_\calX
                g(x)^t
                \exp t \langle \theta, T(x) \rangle \;
                g(x)^{1 - t}
                \exp (1 - t) \langle \theta', T(x) \rangle
                \mu(dx) \\
            &\le \log \myparen{
                \int_\calX
                g(x)
                \exp \langle \theta, T(x) \rangle
                \mu(dx)
            }^t
            \myparen{
                \int_\calX
                g(x)
                \exp \langle \theta', T(x) \rangle
                \mu(dx)
            }^{1 - t}
                \quad \text{(H\"older の不等式)} \\
            &= t \log \int_\calX
                g(x)
                \exp \langle \theta, T(x) \rangle
                \mu(dx)
            + (1 - t) \log \int_\calX
                g(x)
                \exp \langle \theta', T(x) \rangle
                \mu(dx) \\
            &= t \psi(\theta) + (1 - t) \psi(\theta')
    \end{alignat}
    よって示せた。
\end{proof}

\begin{proposition}[対数分配関数の微分可能性]
    $\Theta \subset \R^n$が開集合で
    $a_i(\theta) = \theta_i \; (i = 1, \dots, n)$ならば、
    対数分配関数$\psi$は$\Theta$上で任意回微分可能である。
\end{proposition}

\begin{proof}
    証明の要点に集中するために、$n = 1$の場合のみ示す。
    積分記号下での微分を行うため、次の claim を示す。
    \begin{itemize}
        \item 任意の$k \in \Z_{\ge 0}$と$\theta \in \Theta$に対し、
            $g(x) T(x)^k e^{\theta T(x)}$は$\calX$上$\mu$-可積分である。
    \end{itemize}
    \begin{innerproof}
        $k \in \Z_{\ge 0}$に関する帰納法で示す。
        $k = 0$で成り立つことは、
        すべての$\theta \in \Theta$に対し
        $\psi(\theta)$が定義されることからわかる。

        与えられた$k \in \Z_{\ge 0}$で成立を仮定し、
        $k + 1$での成立を示す。
        $\theta \in \Theta$を任意に固定する。
        $\Theta$が開集合であることから、
        ある$\theta' > 0$が存在して$\theta \pm \theta' \in \Theta$が成り立つ。
        $g(x) T(x)^{k + 1} e^{\theta T(x)}$が$\calX$上$\mu$-可積分であることを示すため、
        $|g(x) T(x)^{k + 1} e^{\theta T(x)}| \le \Phi_\theta(x)$なる
        $\mu$-可積分関数$\Phi_\theta \colon \calX \to \R$を構成する。

        ここで各$x \in \calX$に対し$T(x)$の符号で場合分けすると次が成り立つ。
        $T(x) > 0$のとき
        \begin{alignat}{1}
            |g(x) T(x)^{k + 1} e^{\theta T(x)}|
                &= |g(x) T(x)^k| T(x) e^{\theta T(x)} \\
                &= \frac{1}{\theta'} |g(x) T(x)^k| \theta' T(x) e^{\theta T(x)} \\
                &\le \frac{1}{\theta'} |g(x) T(x)^k| e^{(\theta + \theta') T(x)}
                    \quad
                    (\because \theta' T(x) \le e^{\theta' T(x)}) \\
                &= \frac{1}{\theta'} |g(x) T(x)^k e^{(\theta + \theta') T(x)}|
        \end{alignat}
        $T(x) \le 0$のとき
        \begin{align}
            |g(x) T(x)^{k + 1} e^{\theta T(x)}|
                &= |g(x) T(x)^k| (- T(x)) e^{\theta T(x)} \\
                &= \frac{1}{\theta'} |g(x) T(x)^k| (- \theta' T(x)) e^{\theta T(x)} \\
                &\le \frac{1}{\theta'} |g(x) T(x)^k| e^{(\theta - \theta') T(x)}
                    \quad
                    (\because - \theta' T(x) \le e^{- \theta' T(x)}) \\
                &= \frac{1}{\theta'} |g(x) T(x)^k e^{(\theta - \theta') T(x)}|
        \end{align}
        そこで$\Phi_\theta(x)$を上の2つの不等式の右辺の和と定めれば、
        帰納法の仮定よりそれぞれは$\calX$上$\mu$-可積分だから、
        $\Phi_\theta$も$\calX$上$\mu$-可積分である。
        したがって$g(x) T(x)^{k + 1} e^{\theta T(x)}$も$\calX$上$\mu$-可積分である。
        これで帰納法が完成した。
    \end{innerproof}
    $\psi(\theta) = \log \int g(x) e^{\theta T(x)} \mu(dx)$の被積分関数は
    $\theta$に関し微分可能で、
    導関数$g(x) T(x) e^{\theta T(x)}$は補題より$\calX$上$\mu$-可積分だから、
    積分記号下の微分ができる。したがって
    \begin{alignat}{1}
        \psi'(\theta)
            &= \frac{\int_\calX g(x) T(x) e^{\theta T(x)} \mu(dx)}
                {\int_\calX g(x) e^{\theta T(x)} \mu(dx)} \\
            &= \frac{e^{-\psi(\theta)} \int_\calX g(x) T(x) e^{\theta T(x)} \mu(dx)}
                {\int_\calX g(x) e^{\theta T(x) - \psi(\theta)} \mu(dx)} \\
            &= e^{-\psi(\theta)} \int_\calX g(x) T(x) e^{\theta T(x)} \mu(dx)
    \end{alignat}
    を得る。
    上の claim より、右辺に現れる積分も積分記号下の微分ができるから、
    帰納的に$\psi$は任意回微分可能であることがわかる。
\end{proof}

\subsection{混合型分布族}

混合型分布族について述べる。

\begin{definition}[混合型分布族]
    $(\calX, \calA)$を可測空間、
    $\mu$を$(\calX, \calA)$上の$\sigma$-有限測度、
    $H \subset (0, 1)^n$を部分集合、
    $\calP = \myparen{\calP_\eta}_{\eta \in H}$を
    $\calX$上の確率分布族とする。
    ここで、ある
    \begin{itemize}
        \item $m \in \Z_{\ge 1}$
        \item $\calX$上の確率分布$q_i \; (i = 1, \dots, k)$
    \end{itemize}
    が存在し、各$P_\eta$が$\mu$に関し絶対連続で、確率密度関数が
    \begin{equation}
        \dd[P_\eta]{\mu}(x) = \sum_{i = 1}^m \eta_i q_i(x)
            \quad
            \text{$\mu$-a.e. $x \in \calX$},
            \qquad
            \eta_i \in H,
            \qquad
            \eta_1 + \cdots + \eta_m = 1
    \end{equation}
    の形に表されるとき、$\calP$を
    \term{混合型分布族}[mixture family]{混合型分布族}[こんごうがたぶんぷぞく]
    という。
    \TODO{1次独立性は何に使う?}
\end{definition}

\begin{proposition}[負のエントロピー]
    \TODO{}
\end{proposition}

\begin{proof}
    \TODO{}
\end{proof}



% ------------------------------------------------------------
%
% ------------------------------------------------------------
\section{今後の予定}

\begin{itemize}
    \item \TODO{}
\end{itemize}


\end{document}