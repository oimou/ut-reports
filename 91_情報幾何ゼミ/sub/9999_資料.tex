\documentclass[report]{jlreq}
\usepackage{global}
\usepackage{./local}
\subfiletrue
\def\assetspath{../}
%\makeindex
\chead{2023/??/??}
\begin{document}

% ============================================================
%
% ============================================================

% ------------------------------------------------------------
%
% ------------------------------------------------------------
\section*{振り返りと導入}

% ------------------------------------------------------------
%
% ------------------------------------------------------------
\section{統計構造}

\begin{definition}[統計構造]
    $(g, \nabla)$を
    \term{統計構造}[statistical structure]
        {統計構造}[とうけいこうぞう]
    という。
\end{definition}

\begin{propdef}[canonical ダイバージェンス]
    \TODO{}
\end{propdef}

\begin{theorem}[$D$からの復元]
    \TODO{}
\end{theorem}

% ------------------------------------------------------------
%
% ------------------------------------------------------------
\section{シンプレクティック多様体}

\TODO{
    シンプレクティック多様体が余接バンドルでないとき、
    それはどのような空間のどのような運動を記述しているのだろうか?
}

\begin{definition}[シンプレクティック多様体]
    \TODO{}
\end{definition}

\begin{example}[双対平坦空間]
    \TODO{}
\end{example}

\begin{theorem}[統計構造との両立]
    \begin{equation}
        \nabla^* = \nabla - J(\nabla J),
            \qquad
            (\nabla_X J) Y = (\nabla_Y J) X
    \end{equation}
\end{theorem}

% ------------------------------------------------------------
%
% ------------------------------------------------------------
\section{Bayes 更新}

\begin{proposition}[Bayes の定理]
    \TODO{}
\end{proposition}

% ------------------------------------------------------------
%
% ------------------------------------------------------------
\section{Hamiltonian}

\begin{definition}[シンプレクティック同相写像, 正準変換]
    \TODO{}
\end{definition}

\begin{definition}[Hamiltonian]
    \TODO{}
\end{definition}

% ------------------------------------------------------------
%
% ------------------------------------------------------------
\section{Lagrange 部分多様体}

\begin{definition}[Lagrange 部分多様体]
    \TODO{}
\end{definition}

Lagrange 部分多様体は
シンプレクティック多様体の間の写像概念の一般化とみなせる。

\begin{definition}[Lagrange 対応]
    \TODO{}
\end{definition}

% ------------------------------------------------------------
%
% ------------------------------------------------------------
\section*{今後の予定}

% ------------------------------------------------------------
%
% ------------------------------------------------------------
\section*{参考文献}

% ------------------------------------------------------------
%
% ------------------------------------------------------------
%\newpage
%\appendix
%\renewcommand\thesection{\Alph{section}}
%\setcounter{section}{0}
%\section{付録}

\end{document}