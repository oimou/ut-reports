\documentclass[report]{jlreq}
\usepackage{global}
\usepackage{./local}
\subfiletrue
\def\assetspath{../}
%\makeindex
\chead{2023/11/01}
\begin{document}

% ============================================================
%
% ============================================================

% ------------------------------------------------------------
%
% ------------------------------------------------------------
\section*{振り返りと導入}

前回は最尤推定量とKLダイバージェンスを定義した。
本稿では次のことを行う:
\begin{itemize}
    \item KLダイバージェンスと最尤推定との関係を調べる。
    \item KLダイバージェンスの双対平坦多様体への一般化を考える。
\end{itemize}

% ------------------------------------------------------------
%
% ------------------------------------------------------------
\section{KL ダイバージェンスと最尤推定}

本節では1点$x$での Dirac 測度を$\delta^x$と記す。

\begin{definition}[Kullback-Leibler ダイバージェンス]
    関数$D \colon \calP(\calX) \times \calP(\calX) \to [0, \infty],$
    \begin{equation}
        D(p \| q)
            \coloneqq
                \begin{cases}
                    E_q \mybracket{
                        \frac{dp}{dq}
                        \log \frac{dp}{dq}
                    }
                        =
                            E_p \mybracket{
                                \log \frac{dp}{dq}
                            }
                        & (p \ll q) \\
                    \infty
                        & (p \not\ll q)
                \end{cases}
    \end{equation}
    を$\calP(\calX)$上の
    \term{Kullback-Leibler ダイバージェンス}
        {Kullback-Leibler ダイバージェンス}[Kullback-Leibler ダイバージェンス]
    と呼ぶ。
\end{definition}

\begin{example}[有限でない例]
    $p \ll q$であっても
    $D(p \| q) < \infty$とは限らない。
    $\calX \coloneqq \R$として
    $p$を標準 Cauchy 分布、
    $q$を標準正規分布とした場合が反例のひとつ。
\end{example}

\begin{example}[連続でない例]
    $\calP(\calX)$に
    全変動で定まる位相を入れたとき、
    KLダイバージェンスは連続とは限らない。
    $\calX \coloneqq \{ 0, 1 \}$として
    $p_n \coloneqq \frac{1}{n} \delta^0 + \myparen{1 - \frac{1}{n}} \delta^1, \;
        q_n \coloneqq \frac{1}{e^n} \delta^0 + \myparen{1 - \frac{1}{e^n}} \delta^1$
    が反例のひとつ。
\end{example}

$\calX = \{ 1, \ldots, n \}, n \in \N$の場合に
最尤推定量とKLダイバージェンスの関係を考える。

\begin{definition}[経験分布]
    $x = (x_1, \dots, x_k) \in \calX^k$が与えられたとする。
    $\calX$上の確率測度
    \begin{equation}
        \hat{p}_x
            \coloneqq
                \frac{1}{k}
                \sum_{j = 1}^k
                    \delta^{x_j}
            =
                \sum_{i = 1}^n
                    a_i \delta^i,
                \qquad
        a_i
            \coloneqq
                \frac{1}{k}
                \#
                \{
                    j \in \{ 1, \ldots, k \}
                    \mid
                    x_j = i
                \}
    \end{equation}
    を、$x$により定まる
    \term{経験分布}[empirical distribution]
        {経験分布}[けいけんぶんぷ]
    という。
\end{definition}

次の定理により、
尤度最大化の問題は
KLダイバージェンスの言葉で表せることがわかる。

\begin{theorem}[最尤推定量とKLダイバージェンス]
    $(\Theta, \bfp)$を$\calX$上の統計モデル、
    $(\Theta, \bfp^k)$をその$k$個のi.i.d.拡張とする。
    $x = (x_1, \dots, x_k) \in \calX^k$が与えられたとし、
    $\hat{p}_x$を$x$により定まる経験分布とする。
    このとき、
    集合$\bfp(\Theta)$が
    $\hat{p}_x$を支配する確率測度を
    少なくともひとつ含むならば、
    次が成り立つ:
    \begin{equation}
        \argmin_{\theta \in \Theta} D(\hat{p}_x \| \bfp(\theta))
            =
                \argmax_{\theta \in \Theta} p_\theta^k(x).
    \end{equation}
    ただし右辺で
    $p_\theta^k(x) \coloneqq \prod_{j = 1}^k p_\theta(x_j), \;
        p_\theta(x_j) \coloneqq \dd[(\bfp(\theta))]{i}(x_j) \;
        (\text{数え上げ測度に関する確率密度関数})$
    である。
\end{theorem}

\begin{proof}
    定理の仮定より
    $\exists \theta_0 \in \Theta \; \text{s.t.} \; \hat{p}_x \ll \bfp(\theta_0)$
    であるが、
    $(\Theta, \bfp)$が統計モデルゆえに
    $\bfp(\Theta)$に属する測度はすべて同値だから
    $\forall \theta \in \Theta, \; \hat{p}_x \ll \bfp(\theta)$
    である。
    そこで$\forall \theta \in \Theta$に対し、
    $\hat{p}_x \eqqcolon \sum_{i = 1}^n a_i \delta^i, \;
        a_i \in [0, 1]$
    とおくと
    \begin{alignat}{1}
        D(\hat{p}_x \| \bfp(\theta))
            &=
                E_{\hat{p}_x} \mybracket{
                    \log \dd[\hat{p}_x]{(\bfp(\theta))}
                }
                \\
            &=
                \int_\calX
                    \log \dd[\hat{p}_x]{(\bfp(\theta))}(i)
                    \, d\hat{p}_x(i)
                \\
            &=
                \sum_{
                    \substack{
                        i \in \calX \\
                        a_i > 0
                    }
                }
                    a_i \log \frac{a_i}{p_\theta(i)}
                \\
            &=
                -
                \sum_{
                    \substack{
                        i \in \calX \\
                        a_i > 0
                    }
                }
                    a_i \log p_\theta(i)
                + C
                \qquad
                (\text{$C$は$\theta$によらない実定数})
                \\
            &=
                -
                \frac{1}{k}
                \sum_{j = 1}^k
                    \log p_\theta(x_j)
                + C
                \\
            &=
                -
                \frac{1}{k}
                \log p_\theta^k(x)
                + C
    \end{alignat}
    ゆえに定理の主張が従う。
\end{proof}

% ------------------------------------------------------------
%
% ------------------------------------------------------------
\section{双対平坦多様体の性質}

本節では Einstein の記法を用いる。
以下、再び一般の$\calX$を考える。

\begin{proposition}[指数型分布族とKLダイバージェンス]
    $\calP$を$\calX$上の open な指数型分布族とし、
    $(V, T, \mu)$を$\calP$の最小次元実現、
    $\psi \colon \Theta \to \R$を対数分配関数、
    $\theta \colon \calP \to V^\vee$を自然パラメータ座標、
    $\Theta = \theta(\calP)$を自然パラメータ空間とする。
    このとき
    \begin{equation}
        D(p \| q)
            =
                \psi(\theta(q)) - \psi(\theta(p))
                    - \deldel[\psi]{\theta^i}(p)
                    (
                        \theta^i(q) - \theta^i(p)
                    )
                \quad
                (\forall p, q \in \calP)
    \end{equation}
    が成り立つ。
\end{proposition}

\begin{proof}
    \begin{alignat}{1}
        \psi(\theta(q)) - \psi(\theta(p))
            - \deldel[\psi]{\theta^i}(p)
            (
                \theta^i(q) - \theta^i(p)
            )
            &=
                \psi(\theta(q)) - \psi(\theta(p))
                    - E_p[T_i]
                    (
                        \theta^i(q) - \theta^i(p)
                    )
                \\
            &=
                E_p \mybracket{
                    \psi(\theta(q)) - \psi(\theta(p))
                        - \myangle{\theta(q)}{T}
                        + \myangle{\theta(p)}{T}
                }
                \\
            &=
                E_p \mybracket{
                    \log \frac{dp}{dq}
                }
                \\
            &=
                D(p \| q)
    \end{alignat}
\end{proof}

一般の双対平坦多様体にも
上の命題の$\theta, \psi$のようなものが存在するかどうかを考える。

\begin{definition}[ポテンシャル]
    $M$を多様体、
    $g$を$M$上の Riemann 計量、
    $\nabla$を$M$のアファイン接続とする。
    $\psi \in \smooth(M)$が$g$の
    \term{$\nabla$-ポテンシャル}
        {$\nabla$-ポテンシャル}[$\nabla$-ポテンシャル]
    であるとは、
    $g = \nabla d\psi$が成り立つことをいう。
\end{definition}

\begin{definition}[Hessian チャート]
    $M$を多様体、
    $g$を$M$上の Riemann 計量、
    $\nabla$を$M$のアファイン接続とする。
    $M$のチャート$(U, \theta)$が
    $\nabla$に関する
    \term{Hessian チャート}
        {Hessian チャート}[Hessian チャート]
    であるとは、
    $\theta$が$\nabla$-アファイン座標であり、
    $U$上の$g$の$\nabla$-ポテンシャル$\psi \in \smooth(U)$が存在することをいう。
\end{definition}

\begin{theorem}[Hessian チャートの存在]
    $M$を多様体、
    $(g, \nabla, \nabla^*)$を$M$上の双対平坦構造とする。
    このとき、各$q \in M$に対し次が成り立つ:
    \begin{enumerate}
        \item $q$のまわりの Hessian チャート$(U, \theta)$が存在する。
        \item $U$上の$g$の$\nabla$-ポテンシャル$\psi \in \smooth(U)$をひとつ選んで
            $\eta_i \coloneqq \deldel[\psi]{\theta^i}$とおくと、
            $\eta \coloneqq (\eta_i)_i$は
            $U$上の$\nabla^*$-アファイン座標であり、
            $(\theta, \eta)$は$(g, \nabla, \nabla^*)$に関する双対アファイン座標である。
        \item $\bar{\psi} \coloneqq \psi \circ \theta^{-1} \colon \theta(U) \to \R$
            の Legendre 変換を
            $\bar{\varphi} \coloneqq \bar{\psi}^\vee \colon \eta(U) \to \R$
            とおき、
            $\varphi \coloneqq \bar{\varphi} \circ \eta \colon U \to \R$とおくと、
            $\varphi$は$U$上の$g$の$\nabla^*$-ポテンシャルである。
    \end{enumerate}
\end{theorem}

\begin{proof}
    \uline{(1)} \quad
    $(g, \nabla, \nabla^*)$が双対平坦ゆえ、とくに
    $q$のまわりの$\nabla$-アファイン座標$\theta = (\theta^i)_i$が存在する。
    以下$\del_i \coloneqq \deldel{\theta^i}$と記す。
    ここで、$g$が対称であることと、
    $\nabla, \nabla^*$が torsion-free であることと、
    $(g, \nabla, \nabla^*)$が双対構造であることから、
    $\nabla g \in \Gamma(T^{(0, 3)} M)$
    は対称である。
    そこで$h_j \coloneqq g_{ij} d\theta^i \; (j = 1, \ldots, n)$とおくと、
    $\nabla g$の対称性より$h_j$は閉形式となるから、
    Poincar\'e の補題より局所的に
    $h_j = d\psi_j \; (\exists \psi_j)$
    と表せる。
    さらに$h \coloneqq \psi_j d\theta^j$とおくと、
    再び$\nabla g$の対称性より$h$は閉形式となるから、
    Poincar\'e の補題より局所的に
    $h = d\psi \; (\exists \psi)$
    と表せる。
    したがって、
    $q$の十分小さな開近傍$U$が存在し、
    $\psi$を$\nabla$-ポテンシャルとして
    $(U, \theta)$は Hessian チャートとなる。

    \uline{(2)} \quad
    以下$\del^i \coloneqq \deldel{\eta_i}$と記す。
    まず$g$の正定値性より
    $\eta \colon U \to \R^n$は局所微分同相ゆえ、
    必要ならば$U$を小さく取り直して
    $\eta$は微分同相となる。
    したがって$\eta$は$U$上の座標となる。
    また
    \begin{equation}
        g(\del_i, \del^j)
            =
                g\myparen{
                    \del_i,
                    \deldel[\theta^k]{\eta_j} \del_k
                }
            =
                g^{jk}
                g(\del_i, \del_k)
            =
                \delta_i^j
    \end{equation}
    が成り立つから、
    あとは$\eta$が$\nabla^*$-アファイン座標であることを示せばよい。
    これは次の計算により従う:
    \begin{alignat}{1}
        \Gamma_k^{ij}
            &=
                g(\del_k, \Gamma_l^{ij} \del^l)
                \\
            &=
                g(\del_k, \nabla^*_{\del^i} \del^j)
                \\
            &=
                \del^i (g(\del_k, \del^j))
                - g(\nabla_{\del^i} \del_k, \del^j)
                \\
            &=
                - g(g^{il} \nabla_{\del_l} \del_k, \del^j)
                \\
            &=
                0
    \end{alignat}

    \uline{(3)} \quad
    $\eta$の定義より$d\psi = \eta_i d\theta^i$であり、
    また Legendre 変換の定義より$\psi + \varphi = \theta^i \eta_i$であることから、
    $d\varphi = \theta^i d\eta_i$が成り立つ。
    したがって
    \begin{alignat}{1}
        \nabla^* d\varphi
            =
                \nabla^* (\theta^i d\eta_i)
            =
                d\theta^i \otimes d\eta_i
            =
                g_{ij} d\theta^i d\theta^j
            =
                g
    \end{alignat}
    が成り立つ。
    よって$\varphi$は$g$の$\nabla^*$-ポテンシャルである。
\end{proof}

%
%これを踏まえて次のように定義をする。
%
%\begin{propdef}[canonical ダイバージェンス]
%    各$q \in M$に対し、
%    $U$すべての和集合を$U_q$と記す。
%    このとき、
%    各$p \in U_q$に対し
%    \begin{equation}
%        \psi(p) - \psi(q)
%            - \deldel[\psi]{\theta^i}(q)
%            (
%                \theta^i(p) - \theta^i(q)
%            )
%    \end{equation}
%    の値は$\nabla$-アファイン座標$\theta$や
%    $\nabla$-ポテンシャル$\psi$のとり方によらず定まる。
%    これを$D(p \| q)$と記し、
%    双対平坦構造$(g, \nabla, \nabla^*)$の
%    \term{canonical ダイバージェンス}
%        {canonical ダイバージェンス}[canonical ダイバージェンス]
%    という。
%\end{propdef}
%
%\begin{proof}
%    $\nabla d\psi = g = \nabla d\psi'$ゆえ
%    $D^{\theta', \psi} = D^{\theta', \psi'}$が成り立つ。
%    また2つの座標$\theta, \theta'$はアファイン変換で移り合うから
%    $D^{\theta, \psi} = D^{\theta', \psi}$が成り立つ。
%    よって$D^{\theta, \psi} = D^{\theta', \psi'}$が成り立つ。
%\end{proof}


% ------------------------------------------------------------
%
% ------------------------------------------------------------
\section*{今後の予定}

\begin{itemize}
    \item 双対平坦多様体の canonical ダイバージェンス
    \item 一般のダイバージェンスと、そこから誘導される双対平坦構造・シンプレクティック構造
    \item Bayes 更新
\end{itemize}

% ------------------------------------------------------------
%
% ------------------------------------------------------------
\section*{参考文献}

%Legendre 変換については
%\cite{niculescu_convex_2018}
%を参考にした。
%期待値パラメータに関しては
%\cite{wainwright_graphical_2007}を参考にした。

\nocite{amari_information_2016}

{
    \renewcommand{\bibsection}{}
    \bibliographystyle{amsalpha}
    \bibliography{./bibliography,../../mybibliography}
}

% ------------------------------------------------------------
%
% ------------------------------------------------------------
%\newpage
%\appendix
%\renewcommand\thesection{\Alph{section}}
%\setcounter{section}{0}
%\section{付録}

\end{document}