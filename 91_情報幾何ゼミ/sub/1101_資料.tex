\documentclass[report]{jlreq}
\usepackage{global}
\usepackage{./local}
\subfiletrue
\def\assetspath{../}
%\makeindex
\chead{2023/11/01}
\begin{document}

% ============================================================
%
% ============================================================

% ------------------------------------------------------------
%
% ------------------------------------------------------------
\section*{振り返りと導入}

前回は最尤推定量とKLダイバージェンスを定義した。
本稿では次のことを行う:
\begin{itemize}
    \item KLダイバージェンスの性質を調べる。
    \item 双対平坦多様体への一般化を考える。
\end{itemize}

% ------------------------------------------------------------
%
% ------------------------------------------------------------
\section{Kullback-Leibler ダイバージェンス}

\begin{definition}[Kullback-Leibler ダイバージェンス]
    関数$D \colon \calP(\calX) \times \calP(\calX) \to [0, \infty],$
    \begin{equation}
        D(p \| q)
            \coloneqq
                \begin{cases}
                    E_q \mybracket{
                        \frac{dp}{dq}
                        \log \frac{dp}{dq}
                    }
                        =
                            E_p \mybracket{
                                \log \frac{dp}{dq}
                            }
                        & (p \ll q) \\
                    \infty
                        & (p \not\ll q)
                \end{cases}
    \end{equation}
    を$\calP(\calX)$上の
    \term{Kullback-Leibler ダイバージェンス}
        {Kullback-Leibler ダイバージェンス}[Kullback-Leibler ダイバージェンス]
    と呼ぶ。
\end{definition}

\begin{proposition}
    $\calP(\calX)$に
    全変動で定まる位相を入れると、
    KLダイバージェンスは連続とは限らない。
\end{proposition}

\begin{proof}
    $\calX \coloneqq \{ 0, 1 \}$として
    $p_n \coloneqq \frac{1}{n} \delta^0 + \myparen{1 - \frac{1}{n}} \delta^1, \;
        q_n \coloneqq \frac{1}{e^n} \delta^0 + \myparen{1 - \frac{1}{e^n}} \delta^1$
    が反例のひとつ。
\end{proof}

$\calX = \{ 1, \ldots, n \}, n \in \N$ (カテゴリカル分布\TODO{}) の場合に
最尤推定量とKLダイバージェンスの関係を考える。

\begin{definition}[経験分布]
    $x = (x_1, \dots, x_k) \in \calX^k$に対し
    \begin{equation}
        \hat{p}_x
            \coloneqq
                \frac{1}{k}
                \sum_{i = 1}^k
                    \delta^{x_i}
    \end{equation}
    を$x$により定まる
    \term{経験分布}[empirical distribution]
        {経験分布}[けいけんぶんぷ]
    という。
\end{definition}

\begin{proposition}[最尤推定量とKLダイバージェンス]
    $(\Theta, \bfp)$を$\calX$上の統計モデルとし、
    $k$個のi.i.d.拡張$(\Theta, \bfp^k)$を考える。
    $x = (x_1, \dots, x_k) \in \calX^k$とし、
    $\hat{p}_x$を$x$により定まる経験分布とする。
    このとき、
    $\bfp^k(\Theta)$が
    $\hat{p}_x$を支配する確率測度を
    少なくともひとつ含むならば、
    次が成り立つ:
    \begin{equation}
        \argmin_{\theta \in \Theta} D(\hat{p}_x \| \bfp^k(\theta))
            =
                \argmax_{\theta \in \Theta} p_\theta^k(x)
    \end{equation}
\end{proposition}

\begin{proof}
    \TODO{もう少し丁寧に}
    $\forall \theta \in \Theta$に対し
    \begin{alignat}{1}
        D(\hat{p}_x \| \bfp^k(\theta))
            &=
                E_{\hat{p}_x} \mybracket{
                    \log \frac{d\hat{p}_x}{d(\bfp^k(\theta))}
                } \\
            &=
                E_{\hat{p}_x} \mybracket{
                    \log \frac{d\hat{p}_x}{di}
                }
                -
                E_{\hat{p}_x} \mybracket{
                    \log \frac{d(\bfp^k(\theta))}{di}
                } \\
            &=
                (\text{$\theta$によらない項})
                -
                \frac{1}{k}
                \log p_\theta^k(x)
    \end{alignat}
    ゆえに命題の主張が従う。
\end{proof}

% ------------------------------------------------------------
%
% ------------------------------------------------------------
\section{双対平坦多様体とダイバージェンス}

次のことを思い出しておく。

\begin{proposition}
    $\calP$を可測空間$\calX$上の open な指数型分布族とし、
    $\calP$には自然な位相・多様体構造を入れる。
    このとき次が成り立つ:
    \begin{enumerate}[label=(\alph*)]
        \item $\calP$の Fisher 計量を$g$、
            自然な平坦アファイン接続を$\nabla$、
            $g$に関する$\nabla$の双対接続を$\nabla^*$とおくと、
            $(g, \nabla, \nabla^*)$は
            $\calP$上の双対平坦構造となる。
    \end{enumerate}
    さらに次が成り立つ:
    \begin{enumerate}
        \item $(g, \nabla, \nabla^*)$に関する双対アファイン座標$(\theta, \eta)$が存在する。
        \item ある関数$\psi, \varphi \in \smooth(\calP)$であって
            $d\psi = \eta_i d\theta^i, \;
                d\varphi = \theta^i d\eta_i$
            をみたすものが存在する。
        \item $\psi$は$g$のポテンシャルである。
            $\varphi$は$g^{-1}$のポテンシャルである。
    \end{enumerate}
\end{proposition}

\begin{proof}
    (a) \url{便覧.pdf} 定理3.7.3 より。
    (1), (2), (3) \url{便覧.pdf} 定理3.8.4 より。
\end{proof}

\begin{theorem}[双対アファイン座標と双対ポテンシャルの存在]
    $M$を多様体、
    $(g, \nabla, \nabla^*)$を$M$上の双対平坦構造とする。
    このとき、次が成り立つ:
    \begin{enumerate}
        \item $(g, \nabla, \nabla^*)$に関する双対アファイン座標$(\theta, \eta)$が存在する。
        \item ある関数$\psi, \varphi \in \smooth(M)$であって
            $d\psi = \eta_i d\theta^i, \;
                d\varphi = \theta^i d\eta_i$
            をみたすものが存在する。
        \item $\psi$は$g$のポテンシャルである。
            $\varphi$は$g^{-1}$のポテンシャルである。
    \end{enumerate}
\end{theorem}

\begin{proof}
    \TODO{}
\end{proof}

\begin{proposition}[指数型分布族とKLダイバージェンス]
    $\calP$を指数型分布族とする。
    最小次元実現$(V, T, \mu)$に対し
    対数分配関数$\psi$、自然パラメータ座標$\theta$、期待値パラメータ座標$\eta$を考える。
    このとき
    \begin{equation}
        D(p \| q)
            =
                \psi(\theta_q) + \psi^\vee(\eta_p)
                    - \myangle{\theta_q}{\eta_p}
                \quad
                (\forall p, q \in \calP)
    \end{equation}
    が成り立つ。
    ただし$\psi^\vee$は$\psi$の Legendre 変換である。
\end{proposition}

\begin{proof}
    Legendre 変換の定義より
    $\psi(\theta_p) + \psi^\vee(\eta_p)
        = \myangle{\theta_p}{\eta_p}$
    ゆえに
    \begin{alignat}{1}
        \psi(\theta_q) + \psi^\vee(\eta_p)
            - \myangle{\theta_q}{\eta_p}
            &=
                \psi(\theta_q) - \psi(\theta_p)
                    + \myangle{\theta_p}{\eta_p}
                    - \myangle{\theta_q}{\eta_p}
                \\
            &=
                E_p \mybracket{
                    \psi(\theta_q) - \psi(\theta_p)
                        + \myangle{\theta_p}{T}
                        - \myangle{\theta_q}{T}
                }
                \\
            &=
                E_p \mybracket{
                    \log \frac{dp}{dq}
                }
                \\
            &=
                D(p \| q)
    \end{alignat}
\end{proof}

\begin{propdef}[canonical ダイバージェンス]
    \begin{equation}
        \psi(\theta_q) - \varphi(\eta_p)
            - \myangle{\theta_q}{\eta_p}
    \end{equation}
    の値は$\psi, \varphi$のとり方によらず定まる。
    これを$D(p \| q)$とおき、
    双対平坦多様体$(M, g, \nabla, \nabla^*)$の
    \term{canonical ダイバージェンス}
        {canonical ダイバージェンス}[canonical ダイバージェンス]
    という。
\end{propdef}

\begin{proof}
    \TODO{}
\end{proof}


% ------------------------------------------------------------
%
% ------------------------------------------------------------
\section*{今後の予定}

\begin{itemize}
    \item 一般の多様体上のダイバージェンス
    \item ダイバージェンスから誘導される双対平坦構造
    \item ダイバージェンスから誘導されるシンプレクティック構造
    \item Bayes 更新
\end{itemize}

% ------------------------------------------------------------
%
% ------------------------------------------------------------
\section*{参考文献}

%Legendre 変換については
%\cite{niculescu_convex_2018}
%を参考にした。
%期待値パラメータに関しては
%\cite{wainwright_graphical_2007}を参考にした。

\nocite{amari_information_2016}
\nocite{amari_methods_2007}
\nocite{ay_information_2017}

{
    \renewcommand{\bibsection}{}
    \bibliographystyle{amsalpha}
    \bibliography{./bibliography,../../mybibliography}
}

% ------------------------------------------------------------
%
% ------------------------------------------------------------
%\newpage
%\appendix
%\renewcommand\thesection{\Alph{section}}
%\setcounter{section}{0}
%\section{付録}

\end{document}