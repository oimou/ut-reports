\documentclass[report]{jlreq}
\usepackage{global}
\usepackage{./local}
\subfiletrue
\def\assetspath{../}
%\makeindex
\chead{2023/05/23}
\begin{document}

% ============================================================
%
% ============================================================

% ------------------------------------------------------------
%
% ------------------------------------------------------------
\section{振り返りと導入}

\begin{itemize}
    \item KL ダイバージェンス
    \item Fisher 情報量
    \item アファイン接続
\end{itemize}

% ------------------------------------------------------------
%
% ------------------------------------------------------------
\section{期待値パラメータ空間}

指数型分布族の話題に戻る。
以降、本節では$\calX$を可測空間、
$\calP \subset \calP(\calX)$を$\calX$上の指数型分布族、
$(V, T, \nu)$を$\calP$の実現、
$\Theta \coloneqq \Theta_{(V, T, \nu)}$を
$(V, T, \nu)$の自然パラメータ空間とする。

\begin{definition}[期待値パラメータ空間]
    集合$\calM_{(V, T, \nu)}$
    \begin{equation}
        \calM_{(V, T, \nu)}
            \coloneqq \mybrace{
                \mu \in V
                \mid
                \exists \;
                p \colon \text{$\calX$上の確率分布}
                \; \text{s.t.} \;
                p \ll \nu, \;
                E_p[T] = \mu
            }
    \end{equation}
    を$(V, T, \nu)$の
    \term{期待値パラメータ空間}[mean parameter space]
        {期待値パラメータ空間}[きたいちぱらめーたくうかん]
    という。
\end{definition}

期待値パラメータ空間$\calM$は、
$\calP$に属する確率分布に関する$T$の期待値をすべて含んでいる
(一般には真に含んでいる)。

\begin{proposition}
    $\mu \in V$が
    ある$p \in \calP$に関する
    $T$の期待値ならば (すなわち$\mu = E_p[T]$ならば)、
    $\mu$は$\calM_{(V, T, \nu)}$に属する。
\end{proposition}

\begin{proof}
    \TODO{}
\end{proof}

\begin{proposition}[$\calM$は凸集合]
    $\calM_{(V, T, \nu)}$は$V$の凸集合である。
\end{proposition}

\begin{proof}
    \TODO{}
\end{proof}


% ------------------------------------------------------------
%
% ------------------------------------------------------------
\section{今後の予定}

\begin{itemize}
    \item KL ダイバージェンス
    \item Fisher 計量
    \item アファイン接続
\end{itemize}

% ------------------------------------------------------------
%
% ------------------------------------------------------------
\section{参考文献}

\nocite{amari_information_2016}
\nocite{bn1970_pdf}

{
    \renewcommand{\bibsection}{}
    \bibliographystyle{amsalpha}
    \bibliography{./bibliography,../../mybibliography}
}


\end{document}