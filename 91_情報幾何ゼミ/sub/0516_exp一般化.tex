\documentclass[report]{jlreq}
\usepackage{global}
\usepackage{./local}
\subfiletrue
\def\assetspath{../}
%\makeindex
\chead{2023/05/16}
\begin{document}

% ============================================================
%
% ============================================================

\url{0516_資料.pdf}で述べた微分と積分の順序交換について、
$\exp$を他の関数に置き換えた場合、
どのくらい同じことがいえるかを考えてみる。

\begin{problem}
    $\calX$を可測空間、
    $\mu$を$\calX$上の測度、
    $V$を$m$次元$\R$-ベクトル空間、
    $T \colon \calX \to V$を可測関数、
    $f \colon \R \to \R$を微分可能な関数、
    $h \colon V^\vee \times \calX \to \R, \; (t, x) \mapsto f(\myangle{t}{T(x)})$とし、
    $V^\vee$のある開部分集合$\Theta$上で
    $\lambda \colon \Theta \to \R, \; t \mapsto \int_\calX h(t, x) \, \mu(dx)$が
    定義されているとする。
    このとき、
    $\lambda'(t) = \int_\calX \deldel[h]{t} (t, x) \, \mu(dx) \; (t \in \Theta)$
    が成り立つような$f$の条件 ($C^1$級、凸など) はどのようなものか?
    あるいは、どのような反例があるか?
\end{problem}

もう少し簡単な設定で考えてみる。

\begin{problem}
    $\calX$を可測空間、
    $\mu$を$\calX$上の測度、
    $T \colon \calX \to \R$を可測関数、
    $f \colon \R \to \R$を微分可能で上または下に凸な関数、
    $h \colon \R \times \calX \to \R, \; (t, x) \mapsto f(tT(x))$とし、
    $\R$のある開部分集合$\Theta$上で
    $\lambda \colon \Theta \to \R, \; t \mapsto \int_\calX h(t, x) \, \mu(dx)$が
    定義されているとする。
    このとき、
    $\lambda'(t) = \int_\calX \deldel[h]{t} (t, x) \, \mu(dx) \; (t \in \Theta)$
    は成り立つか?
\end{problem}

\begin{answer}
    $t \in \Theta$とし、
    $\lambda'(t) = \int_\calX \deldel[h]{t} (t, x) \, \mu(dx)$
    が成り立つことを示す。
    そのために示すべきことは
    偏導関数に対する優関数の存在、すなわち
    \begin{description}
        \vspace{-1em}
        \setstretch{1.5}
        \item[(A)] $t$のある開近傍$U \opensubset \Theta$と、
            ある$\mu$-可積分関数$\Phi \colon \calX \to \R$が存在し、
            すべての$t' \in U$に対し
            $\myabs{
                \deldel[h]{t}(t', x)
            } \le \Phi(x) \; \text{a.e.$x$}$
            が成り立つ。
    \end{description}
    である。
    $f$が上に凸の場合は
    $f$を$-f$に置き換えて同様の議論をすればよいから、
    以降$f$が下に凸の場合のみ示す。

    \uline{Step 1: $U, \Phi$の構成} \quad
    $r > 0$を十分小さく選び、$\R$の閉区間
    \begin{equation}
        A_{2r} \coloneqq [t - 2r, t + 2r],
            \quad
            A_r \coloneqq [t - r, t + r]
    \end{equation}
    が$\Theta$に含まれるようにしておく。
    そこで
    $U \coloneqq \Int_\Theta A_r = (t - r, t + r)$
    とおき、
    $\Phi \colon \calX \to \R$を
    \begin{equation}
        \Phi(x)
            \coloneqq
            \frac{5}{2r}
            \Big(
                \myabs{
                    h(t - 2r, x)
                }
                +
                \myabs{
                    h(t + 2r, x)
                }
            \Big)
    \end{equation}
    と定める。
    以下、この$U, \Phi$が条件(A)をみたすものであることを示す。

    まず$U$は$\Theta$における$t$の開近傍であり、
    また
    $t \pm 2r \in A_{2r} \subset \Theta$ゆえに
    $h(t \pm 2r, \cdot) \colon \calX \to \R$は
    $\mu$-可積分だから、
    $\Phi$は$\mu$-可積分である。
    したがって
    残りの示すべきことは、
    すべての$t' \in U$に対し
    $\myabs{
        \deldel[h]{t}(t', x)
    }
        \le \Phi(x) \; \text{a.e.$x$}$
    すなわち
    $\myabs{
        f'(t'T(x)) T(x)
    }
        \le \Phi(x) \; \text{a.e.$x$}$
    が成り立つことである。

    \uline{Step 2: $\Phi$による不等式評価} \quad
    $t' \in U$とする。
    まず各$x \in \calX$に対し、$T(x)$の符号で場合分けして不等式評価を与える。
    そこで複号同順でそれぞれの場合を一度に書くと、
    $T(x) \gtrless 0$の場合、
    $t'T(x) < (t \pm 2r) T(x)$だから、
    $f$の微分可能性と凸性より
    \begin{alignat}{2}
        &\phantom{\therefore} \qquad&
            f'(t'T(x))
            &\le
                \frac{
                    f((t \pm 2r)T(x)) - f(t'T(x))
                }
                {
                    ((t \pm 2r)T(x) - t'T(x))
                }
            \\
        &\therefore&
            |f'(t'T(x))T(x)|
            &\le
                \frac{1}{|t \pm 2r - t'|}
                \Big(
                    |f((t \pm 2r)T(x))| + |f(t'T(x))|
                \Big)
            \\
        &&
            &\le
                \frac{1}{r}
                \Big(
                    |f((t \pm 2r)T(x))| + |f(t'T(x))|
                \Big)
    \end{alignat}
    が成り立つ。
    したがって、
    $T(x) > 0$と$T(x) < 0$の場合を合わせると、
    $T(x) \neq 0$のとき
    \begin{equation}
        \locallabel{eq:1}
        |f'(t'T(x)) T(x)|
            \le
                \frac{1}{r}
                \Big(
                    |f((t - 2r)T(x))|
                    + |f(t'T(x))|
                    + |f((t + 2r)T(x))|
                    + |f(t'T(x))|
                \Big)
    \end{equation}
    が成り立つ。
    この不等式は$T(x) = 0$の場合も成り立つから、
    すべての$x \in \calX$に対して成り立つ。

    さらに
    \localcref{eq:1}の右辺の
    $|f(t'T(x))|$の項を評価することを考える。
    いま$t' \in A_r$ゆえに、
    ある$s \in [1/4, 3/4]$が存在して
    $t' = (1 - s)(t - 2r) + s(t + 2r)$が成り立つから、
    $f$の凸性より
    \begin{alignat}{1}
        |f(t'T(x))|
            &=
                \myabs{
                    f((1 - s)(t - 2r)T(x) + s(t + 2r)T(x))
                }
                \\
            &\le
                \myabs{
                    (1 - s)f((t - 2r)T(x)) + sf((t + 2r)T(x))
                }
                \\
            &\le
                (1 - s) |f((t - 2r)T(x))| + s |f((t + 2r)T(x))|
                \\
            &\le
                \frac{3}{4}
                \Big(
                    |f((t - 2r)T(x))| + |f((t + 2r)T(x))|
                \Big)
    \end{alignat}
    が成り立つ。
    この不等式を\localcref{eq:1}に合わせれば、
    すべての$x \in \calX$に対して
    \begin{alignat}{1}
        |f'(t'T(x)) T(x)|
            &=
                \frac{10}{4r}
                \Big(
                    |f((t - 2r)T(x))|
                    +
                    |f((t + 2r)T(x))|
                \Big)
                \\
            &=
                \frac{5}{2r}
                \Big(
                    |h(t - 2r, x)|
                    +
                    |h(t + 2r, x)|
                \Big)
                \\
            &=
                \Phi(x)
    \end{alignat}
    が成り立つことがわかる。
    したがって$U, \Phi$が条件(A)をみたすことが示されて、
    証明が完了した。
\end{answer}

\end{document}