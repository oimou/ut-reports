\documentclass[report]{jlreq}
\usepackage{global}
\usepackage{./local}
\subfiletrue
\def\assetspath{../}
%\makeindex
\chead{2023/05/16}
\begin{document}

% ============================================================
%
% ============================================================

\url{0516_資料.pdf}で述べた微分と積分の順序交換について、
$\exp$を他の関数に置き換えた場合、
どのくらい同じことがいえるかを考えてみる。

\begin{problem}
    $\calX$を可測空間、
    $\mu$を$\calX$上の測度、
    $V$を$m$次元$\R$-ベクトル空間、
    $T \colon \calX \to V$を可測関数、
    $f \colon \R \to \R$を微分可能な関数、
    $h \colon V^\vee \times \calX \to \R, \; (t, x) \mapsto f(\myangle{t}{T(x)})$とし、
    $V^\vee$のある開部分集合$\Theta$上で
    $\lambda \colon \Theta \to \R, \; t \mapsto \int_\calX h(t, x) \, \mu(dx)$が
    定義されているとする。
    このとき、
    $\lambda'(t) = \int_\calX \deldel[h]{t} (t, x) \, \mu(dx) \; (t \in \Theta)$
    が成り立つような$f$の条件 ($C^1$級、凸など) はどのようなものか?
    あるいは、どのような反例があるか?
\end{problem}

もう少し簡単な設定で考えてみる。

\begin{problem}
    $\calX$を可測空間、
    $\mu$を$\calX$上の測度、
    $T \colon \calX \to \R$を可測関数、
    $f \colon \R \to \R$を微分可能な凸関数、
    $h \colon \R \times \calX \to \R, \; (t, x) \mapsto f(tT(x))$とし、
    $\R$のある開部分集合$\Theta$上で
    $\lambda \colon \Theta \to \R, \; t \mapsto \int_\calX h(t, x) \, \mu(dx)$が
    定義されているとする。
    このとき、
    $\lambda'(t) = \int_\calX \deldel[h]{t} (t, x) \, \mu(dx) \; (t \in \Theta)$
    は成り立つか?
\end{problem}

\begin{answer}
    $t \in \Theta$とし、
    $\lambda'(t) = \int_\calX \deldel[h]{t} (t, x) \, \mu(dx)$
    が成り立つことを示す。
    そのために示すべきことは
    偏導関数に対する優関数の存在、すなわち
    \begin{description}
        \vspace{-1em}
        \setstretch{1.5}
        \item[(A)] $t$のある開近傍$U \opensubset \Theta$と、
            ある$\mu$-可積分関数$\Phi \colon \calX \to \R$が存在し、
            すべての$t' \in U$に対し
            $\myabs{
                \deldel[h]{t}(t', x)
            } \le \Phi(x) \; \text{a.e.$x$}$
            が成り立つ。
    \end{description}
    である。

    \uline{Step 1: $U, \Phi$の構成} \quad
    $r > 0$を十分小さく選び、$\R$の閉区間
    \begin{equation}
        A_{2r} \coloneqq [t - 2r, t + 2r],
            \quad
            A_r \coloneqq [t - r, t + r]
    \end{equation}
    が$\Theta$に含まれるようにしておく。
    そこで
    $U \coloneqq \Int_\Theta A_r = (t - r, t + r)$
    とおき、
    $\Phi \colon \calX \to \R$を
    \begin{equation}
        \Phi(x)
            \coloneqq
                \frac{1}{r}
                \Big(
                    \myabs{
                        h(t + 2r, x)
                    }
                    +
                    \myabs{
                        h(t + r, x)
                    }
                    +
                    \myabs{
                        h(t - r, x)
                    }
                    +
                    \myabs{
                        h(t - 2r, x)
                    }
                \Big)
    \end{equation}
    と定める。
    以下、この$U, \Phi$が条件(A)をみたすものであることを示す。

    まず$U$は$\Theta$における$t$の開近傍であり、
    また
    $t \pm r, t \pm 2r \in \Theta$ゆえに
    $h(t \pm r, \cdot), h(t \pm 2r, \cdot) \colon \calX \to \R$は
    $\mu$-可積分だから、
    $\Phi$は$\mu$-可積分である。
    したがって
    残りの示すべきことは、
    すべての$t' \in U$に対し
    $\myabs{
        \deldel[h]{t}(t', x)
    }
        \le \Phi(x) \; \text{a.e.$x$}$
    すなわち
    $\myabs{
        f'(t'T(x)) T(x)
    }
        \le \Phi(x) \; \text{a.e.$x$}$
    が成り立つことである。

    \uline{Step 2: $\Phi$による不等式評価} \quad
    $t' \in U$とする。
    まず各$x \in \calX$に対し、$T(x)$の符号で場合分けして不等式評価を与える。

    $T(x) > 0$の場合、
    $(t - 2r)T(x) < (t - r)T(x) < t'T(x) < (t + r)T(x) < (t + 2r)T(x)$
    だから、
    $f$の凸性より
    \begin{alignat}{4}
        \frac{
            f((t - r)T(x)) - f((t - 2r)T(x))
        }{
            (t - r)T(x) - (t - 2r)T(x)
        }
            &\le
                f'(t'T(x))
            &&\le
                \frac{
                    f((t + 2r)T(x)) - f((t + r)T(x))
                }{
                    (t + 2r)T(x) - (t + r)T(x)
                }
    \end{alignat}
    $T(x) > 0$より
    \begin{alignat}{2}
        \frac{
            f((t - r)T(x)) - f((t - 2r)T(x))
        }{
            r
        }
            &\le
                f'(t'T(x)) T(x)
            &&\le
                \frac{
                    f((t + 2r)T(x)) - f((t + r)T(x))
                }{
                    r
                }
    \end{alignat}
    が成り立つ。
    さらに
    $\R_{\ge 0}$ (resp. $\R_{\le 0}$) 上での
    $| \cdot |$の単調増加性 (resp. 単調減少性) より
    \begin{alignat}{1}
        f'(t'T(x)) T(x) \ge 0
            &\implies
                |f'(t'T(x)) T(x)|
                    \le
                        \frac{1}{r}
                        \myabs{
                            f((t + 2r)T(x)) - f((t + r)T(x))
                        },
            \\
        f'(t'T(x)) T(x) < 0
            &\implies
                |f'(t'T(x)) T(x)|
                    \le
                        \frac{1}{r}
                        \myabs{
                            f((t - r)T(x)) - f((t - 2r)T(x))
                        }
    \end{alignat}
    が成り立つ。
    したがって、これら2つの不等式を合わせて
    \begin{alignat}{1}
        |f'(t'T(x)) T(x)|
            &\le
                \frac{1}{r}
                \Big(
                    \myabs{
                        f((t + 2r)T(x)) - f((t + r)T(x))
                    }
                    +
                    \myabs{
                        f((t - r)T(x)) - f((t - 2r)T(x))
                    }
                \Big)
                \\
            &\le
                \frac{1}{r}
                \Big(
                    \myabs{
                        f((t + 2r)T(x))
                    }
                    +
                    \myabs{
                        f((t + r)T(x))
                    }
                    +
                    \myabs{
                        f((t - r)T(x))
                    }
                    +
                    \myabs{
                        f((t - 2r)T(x))
                    }
                \Big)
                \\
            &=
                \frac{1}{r}
                \Big(
                    \myabs{
                        h(t + 2r, x)
                    }
                    +
                    \myabs{
                        h(t + r, x)
                    }
                    +
                    \myabs{
                        h(t - r, x)
                    }
                    +
                    \myabs{
                        h(t - 2r, x)
                    }
                \Big)
                \\
            &=
                \Phi(x)
    \end{alignat}
    が成り立つ。

    $T(x) < 0$の場合も同様にして
    $|f'(t'T(x)) T(x)| \le \Phi(x)$
    が成り立ち、
    また$T(x) = 0$の場合も明らかに成り立つ。
    したがって$U, \Phi$が条件(A)をみたすことが示されて、
    証明が完了した。
\end{answer}

凸でない場合の反例を考えてみる。

\begin{problem}
    \TODO{凸でない場合の反例}
\end{problem}

\begin{answer}
    \TODO{}
\end{answer}

具体的な非凸多項式関数で反例が作れるかどうか考えてみる。

\begin{problem}
    $f \colon \R \to \R, \; x \mapsto x - x^3$とする。
    このとき、
    次をみたす組$(\calX, \mu, T)$であって
    「$\lambda'(t) = \int_\calX \deldel[h]{t}(t, x) \, \mu(dx)$」
    が成立しないものを構成できるか?
    \begin{itemize}
        \item $\calX$は可測空間、
            $\mu$は$\calX$上の測度、
            $T \colon \calX \to \R$は可測関数。
        \item $h \colon \R \times \calX \to \R, \; (t, x) \mapsto f(tT(x))$と定めると、
            $\R$のある開部分集合$\Theta$上で
            $\lambda \colon \Theta \to \R, \; t \mapsto \int_\calX h(t, x) \, \mu(dx)$が
            定義されている。
    \end{itemize}
\end{problem}

\begin{answer}
    \TODO{}
\end{answer}

より強い主張を考えてみる。

\begin{problem}
    $f \colon \R \to \R$を任意の非凸関数とする。
    このとき、
    次をみたす組$(\calX, \mu, T)$であって
    「$\lambda'(t) = \int_\calX \deldel[h]{t}(t, x) \, \mu(dx)$」
    が成立しないものを構成できるか?
    \begin{itemize}
        \item $\calX$は可測空間、
            $\mu$は$\calX$上の測度、
            $T \colon \calX \to \R$は可測関数。
        \item $h \colon \R \times \calX \to \R, \; (t, x) \mapsto f(tT(x))$と定めると、
            $\R$のある開部分集合$\Theta$上で
            $\lambda \colon \Theta \to \R, \; t \mapsto \int_\calX h(t, x) \, \mu(dx)$が
            定義されている。
    \end{itemize}
\end{problem}

\begin{answer}
    \TODO{}
\end{answer}

\end{document}