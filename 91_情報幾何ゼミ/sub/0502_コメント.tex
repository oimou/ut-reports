\documentclass[report]{jlreq}
\usepackage{global}
\usepackage{./local}
\subfiletrue
\def\assetspath{../}
%\makeindex
\chead{2023/05/02}
\begin{document}

% ============================================================
%
% ============================================================

発表中にコメントがあった命題などを整理する。

ベクトル値関数の可積分性は、
次のように双対空間を用いると簡潔に定義できる。

\begin{definition}[ベクトル値関数の可積分性]
    $\calX$を可測空間、
    $V$を有限次元$\R$-ベクトル空間、
    $\mu$を$\calX$上の測度、
    $f \colon \calX \to V$を可測写像とする。
    すべての$g \in V^\vee$に対し
    $g \circ f \in L^1(\calX, \mu)$が成り立つとき、
    $f$は$\mu$に関し
    \term{可積分}[integrable]
        {可積分}[かせきぶん]
    であるという。
\end{definition}

アファイン部分空間の定義を整理する。

\begin{definition}[アファイン部分空間]
    $K$を体、
    $V$を$K$-ベクトル空間、
    $A \subset V$を部分集合とする。
    $A$が$V$の$K$上の
    \term{アファイン部分空間}[affine subspace]
        {アファイン部分空間}[あふぁいんぶぶんくうかん]であるとは、
    次が成り立つことをいう:
    $\exists \; (S, v)$ s.t.
    \begin{description}
        \item[(A1)] $S \subset V$は$K$-部分ベクトル空間である。
        \item[(A2)] $v \in V$であり、$A = v + S$が成り立つ。
    \end{description}
\end{definition}

次の命題より、
アファイン部分空間に値を持つ写像の期待値も定義できる。

\begin{proposition}[アファイン部分空間に値を持つ写像の積分]
    $\calX$を可測空間、
    $p$を$\calX$上の確率測度、
    $V$を有限次元$\R$-ベクトル空間、
    $f \colon \calX \to V$を$p$-可積分写像とする。
    このとき、
    $V$のあるアファイン部分空間$A$が存在して
    $f(x) \in A \; \text{$p$-a.e.$x$}$が成り立つならば、
    $\int_\calX f(x) \, p(dx) \in A$となる。
\end{proposition}

\begin{proof}
    $f(x) \in A \; \text{$p$-a.e.$x$}$より、
    $f(x) = v + \tilde{f}(x)$なる
    $p$-可積分写像$\tilde{f} \colon \calX \to S$が存在する。
    これより
    \begin{alignat}{1}
        \int_\calX f(x) \, p(dx)
            &= \int_\calX (v + \tilde{f}(x)) \, p(dx) \\
            &= v + \int_\calX \tilde{f}(x) \, p(dx)
                \quad
                (\text{$p$は確率測度}) \\
            &\in v + S \\
            &= A
    \end{alignat}
    が成り立つ。
\end{proof}

ベクトル値関数の可積分性の定義を$q$-乗可積分まで拡張し、
また$q$-乗可積分関数全体の集合を$L^q(\calX, p; V)$と書くことにすれば、
次が成り立つ。

\begin{lemma}[分散の存在条件]
    \label[lemma]{lemma:f_otimes_f}
    可測写像$f \colon \calX \to V$に関し
    次の2条件は同値である:
    \begin{enumerate}
        \item $f$および
            $(f - E_p[f]) \otimes (f - E_p[f])$が
            $p$-可積分
        \item $f \otimes f$が$p$-可積分
        \highlight{\item $f \in L^2(\calX, p; V)$}
    \end{enumerate}
\end{lemma}

\begin{proof}
    \uline{(2) \Rightarrow (3)} \quad 明らか。

    \uline{(3) \Rightarrow (2)} \quad
    H\"older の不等式を用いて示せる。
\end{proof}

\begin{proposition}
    $V$にノルム$\| \cdot \|$が与えられているとする。
    このとき次は同値である。
    \begin{enumerate}
        \item $\| f \| \in L^1(\calX, p)$
        \item $f \in L^1(\calX, p; V)$
    \end{enumerate}
\end{proposition}

\begin{proof}
    \uline{(2) \Rightarrow (1)} \quad
    三角不等式より明らか。

    \uline{(1) \Rightarrow (2)} \quad
    $V$の基底$E$をひとつ選んで固定し、
    成分に関する2-ノルムを$\| \cdot \|_E$とおく。
    有限次元$\R$-ベクトル空間上のノルムの同値性を用いて
    $|f_i(x)| \le \| f(x) \|_E \le C\| f(x) \| \; \text{$p$-a.e.$x$}$
    が成り立つから
    $f_i \in L^1(\calX, p)$が従う。
\end{proof}

自然パラメータ空間$\Theta$は、
指数型分布族の定義の条件(E3)をみたす$\theta$をすべて含んでいる
(一般には真に含んでいる)。

\begin{proposition}
    $\theta \in V^\vee$が
    ある$p \in \calP$に対し
    指数型分布族の条件(E3)をみたすならば、
    $\theta$は$\Theta$に属する。
    すなわち、
    \begin{equation}
        \mybrace{
            \theta \in V^\vee
            \;\Big|\;
            \exists \; p \in \calP
            \quad \text{s.t.} \quad
            \dd[p]{\nu}(x)
                = \frac{
                    \exp{\langle \theta, T(x) \rangle}
                }{
                    \int_\calX \exp{\langle \theta, T(y) \rangle} \, \nu(dy)
                }
                \quad
                (\text{$\nu$-a.e.$x$})
        }
            \subset \Theta
    \end{equation}
    が成り立つ。
\end{proposition}

\begin{proof}
    まず$\theta \in V^\vee$とし、
    $\theta$はある$p \in \calP$に対し
    \cref{def:exponential-family}の条件(E3)をみたすものとする。
    背理法のため$\theta \notin \Theta$と仮定する。
    すると$\Theta$の定義より
    $\int_\calX \exp{\langle \theta, T(y) \rangle} \, \nu(dy) = +\infty$
    が成り立つから、
    条件(E3)より
    $\dd[p]{\nu}(x) = \frac{\exp{\langle \theta, T(x) \rangle}}{\infty} = 0 \;
        (\text{$\nu$-a.e.$x$})$
    が成り立つ。
    よって
    \begin{equation}
        1 = \int_\calX \dd[p]{\nu}(x) \, \nu(dx)
            = \int_\calX 0 \, \nu(dx)
            = 0
    \end{equation}
    となり矛盾が従う。
    背理法より$\theta \in \Theta$が成り立つ。
\end{proof}


% ------------------------------------------------------------
%
% ------------------------------------------------------------
\section{参考文献}

\nocite{amari_information_2016}
\nocite{wainwright_graphical_2007}
\nocite{bn1970_pdf}

{
    \renewcommand{\bibsection}{}
    \bibliographystyle{amsalpha}
    \bibliography{./bibliography,../../mybibliography}
}


\end{document}