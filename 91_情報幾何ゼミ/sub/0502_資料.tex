\documentclass[report]{jlreq}
\usepackage{global}
\usepackage{./local}
\subfiletrue
\def\assetspath{../}
%\makeindex
\chead{2023/05/02}
\begin{document}

% ============================================================
%
% ============================================================

% ------------------------------------------------------------
%
% ------------------------------------------------------------
\section{振り返りと導入}

前回は指数型分布族を定義した。
本稿では、前回発表時の指摘を取り入れながら、指数型分布族と最小次元実現について簡単に整理しなおす。
その後、新たなトピックとして期待値と分散の定義を行う。

前回に引き続き、可測空間$\calX$上の確率測度全体の集合を
$\calP(\calX)$と書くことにする。
また、Einstein の記法を用いる。

% ------------------------------------------------------------
%
% ------------------------------------------------------------
\section{指数型分布族と最小次元実現}

本稿では、指数型分布族の定義は次のものを用いる。
注意点として、前回は有限次元$\R$-ベクトル空間として
$V$ではなく$\R^m$を用いていたが、
より一般の状況に対応するため、
本稿からは$V$を用いるように変更する。
なお、この変更により前回の定理の証明は修正が必要となるが、
その内容については本稿では扱わない
(修正内容は \url{0425_コメント.pdf} に書きました)。

\begin{definition}[指数型分布族]
    \label[definition]{def:exponential-family}
    $\calX$を可測空間、
    $\emptyset \neq \calP \subset \calP(\calX)$とする。
    $\calP$が$\calX$上の
    \term{指数型分布族}[exponential family]
        {指数型分布族}[しすうがたぶんぷぞく]
    であるとは、次が成り立つことをいう:
    $\exists \; (\highlight{V}, T, \mu)$ s.t.
    \begin{description}
        \item[\highlight{(E0)}] \highlight{$V$は有限次元$\R$-ベクトル空間である。}
        \item[(E1)] $T \colon \calX \to \highlight{V}$は可測写像である。
        \item[(E2)] $\mu$は$\calX$上の$\sigma$-有限測度であり、
            $\forall \; p \in \calP$に対し$p \ll \mu$をみたす。
        \item[(E3)] $\forall \; p \in \calP$に対し、
            $\exists \; \theta \in \highlight{V^\vee}$ s.t.
            \begin{equation}
                \dd[p]{\mu}(x)
                    = \frac{
                        \exp \langle \theta, T(x) \rangle
                    }{
                        \int_\calX \exp \langle \theta, T(y) \rangle \, \mu(dy)
                    }
                    \quad
                    \text{$\mu$-a.e. $x \in \calX$}
            \end{equation}
            である。
            ただし$\langle \cdot, \cdot \rangle$は
            \highlight{自然なペアリング$V^\vee \times V \to \R$である。}
    \end{description}
    さらに次のように定める:
    \begin{itemize}
        \item $(\highlight{V}, T, \mu)$を$\calP$の
            \term{実現}[representation]
            {実現}[じつげん]
            という。
            \begin{itemize}
                \item $m$を$(\highlight{V}, T, \mu)$の
                    \term{次元}[dimension]{次元}[じげん]
                    という。
                \item $T$を$(\highlight{V}, T, \mu)$の
                    \term{十分統計量}[sufficient statistic]
                    {十分統計量}[じゅうぶんとうけいりょう]
                    という。
                \item $\mu$を$(\highlight{V}, T, \mu)$の
                    \term{基底測度}[base measure]
                    {基底測度}[きていそくど]
                    という。
            \end{itemize}
        \item 集合$\Theta_{(\highlight{V}, T, \mu)}$
            \begin{equation}
                \Theta_{(\highlight{V}, T, \mu)}
                    \coloneqq \mybrace{
                        \theta \in \highlight{V^\vee}
                        \;\Big|\;
                        \int_\calX \exp \langle \theta, T(y) \rangle \, \mu(dy) < +\infty
                    }
            \end{equation}
            を$(\highlight{V}, T, \mu)$の
            \term{自然パラメータ空間}[natural parameter space]
            {自然パラメータ空間}[しぜんぱらめーたくうかん]
            という。
        \item 関数$\psi \colon \Theta_{(\highlight{V}, T, \mu)} \to \R,$
            \begin{equation}
                \psi(\theta)
                    \coloneqq
                    \log \int_\calX \exp \langle \theta, T(y) \rangle \, \mu(dy)
            \end{equation}
            を$(\highlight{V}, T, \mu)$の
            \term{対数分配関数}[log-partition function]
            {対数分配関数}[たいすうぶんぱいかんすう]
            という。
    \end{itemize}
\end{definition}

以降、本節では
$\calX$を可測空間、
$\calP \subset \calP(\calX)$を$\calX$上の指数型分布族、
$(V, T, \mu)$を$\calP$の実現とする。

\begin{proposition}
    \label[proposition]{prop:as-a}
    $(V, T, \mu)$に関する次の条件は同値である:
    \begin{enumerate}
        \item $\langle \theta, T(x) \rangle$が
            $\calX$上$\mu$-a.e.定数であるような
            $\theta \in V^\vee$は$\theta = 0$のみである。
        \item 各$p \in \calP$に対し、
            \cref{def:exponential-family}の条件(E3)をみたす$\theta \in V^\vee$は
            ただひとつである。
    \end{enumerate}
\end{proposition}

\begin{proof}
    \uline{(2) \Rightarrow (1)} \quad
    前回示した。

    \uline{(1) \Rightarrow (2)} \quad
    $\theta, \theta' \in V^\vee$が
    \cref{def:exponential-family}の条件(E3)をみたすとすると、
    \begin{equation}
        e^{\langle \theta, T(x) \rangle - \psi(\theta)}
            = \dd[p]{\mu}(x)
            = e^{\langle \theta', T(x) \rangle - \psi(\theta')}
            \qquad
            \text{$\mu$-a.e.$x \in \calX$}
    \end{equation}
    が成り立つ。式を整理して
    \begin{equation}
        \langle \theta - \theta', T(x) \rangle
            = \psi(\theta) - \psi(\theta')
            \qquad
            \text{$\mu$-a.e.$x \in \calX$}
    \end{equation}
    が成り立つ。
    したがって(1)より$\theta = \theta'$である。
\end{proof}

\begin{definition}[最小次元実現]
    実現$(V, T, \mu)$が
    $\calP$の実現のうちで次元が最小のものであるとき、
    $(V, T, \mu)$を$\calP$の
    \term{最小次元実現}[minimal representation]
        {最小次元実現}[さいしょうじげんじつげん]という。
\end{definition}

次の事実は前回示した。

\begin{fact}
    最小次元実現は存在する。
    \qed
\end{fact}

\begin{fact}
    最小次元実現は\cref{prop:as-a}の2条件をみたす。
    \qed
\end{fact}

% ------------------------------------------------------------
%
% ------------------------------------------------------------
\section{期待値と分散}

ここで一旦指数型分布族の話題から離れて、
可測関数 (あるいは確率変数) の期待値と分散を定義しておく。
これらの概念は、期待値パラメータ空間の定義や、
対数分配関数の性質を調べる際に重要な役割を果たす。

ここでは期待値および分散の概念を
ベクトル値関数に対して定義したい。
そこで、まずベクトル値関数の積分および可積分性を
次のように定めておく。

\begin{definition}[ベクトル値関数の積分]
    $\calX$を可測空間、
    $V$を有限次元$\R$-ベクトル空間、
    $p$を$\calX$上の確率測度、
    $f \colon \calX \to V$を可測写像とする。
    $V$のある基底$e^1, \dots, e^m$が存在して、
    この基底に関する$f$の成分
    $f_i \colon \calX \to \R \; (i = 1, \dots, m)$が
    すべて$p$-可積分であるとき、
    $f$は$p$に関し
    \term{可積分}[integrable]
        {可積分!ベクトル値関数の---}[かせきぶん]
    であるという
    (well-defined 性はこのあと示す)。

    $f$が$p$-可積分であるとき、
    $f$の$p$に関する\term{積分}[integral]
        {積分}[せきぶん]
    を
    \begin{equation}
        \int_\calX f(x) \, p(dx)
            \coloneqq \myparen{
                \int_\calX f_i(x) \, p(dx)
            } e^i
            \in V
    \end{equation}
    で定義する
    (well-defined 性はこのあと示す)。

    ただし$\dim V = 0$の場合は
    $f$は$p$-可積分で$\int_\calX f(x) \, p(dx) = 0$と約束する。
\end{definition}

\begin{remark}
    $V = \R$の場合は
    $\R$-値関数の通常の積分に一致する。
\end{remark}

\begin{proof}[well-defined 性の証明.]
    $f$が$p$-可積分であるかどうかは
    $V$の基底の取り方によらないことを示す。
    そこで、$e^1, \dots, e^m$および$\tilde{e}^1, \dots, \tilde{e}^m$を
    それぞれ$V$の基底とし、
    それぞれの基底に関する$f$の成分を
    $f_i, \; \tilde{f}_i \colon \calX \to \R \; (i = 1, \dots, m)$とおく。
    示すべきことは
    「$\tilde{f}_i \; (i = 1, \dots, m)$がすべて$L^1(\calX, p)$に属するならば
    $f_i \; (i = 1, \dots, m)$もすべて$L^1(\calX, p)$に属する」ということである。
    このことは、
    $L^1(\calX, p)$が$\R$-ベクトル空間であることと、
    $f_i$たちが$\tilde{f}_i$たちの
    $\R$-線型結合であることから従う。
    よって$f$が$p$-可積分であるかどうかは
    $V$の基底の取り方によらない。

    次に、$f$の$p$に関する積分は
    $V$の基底の取り方によらないことを示す。
    $e^i, \; \tilde{e}^i$をそれぞれ$V$の基底とする。
    いま、ある$a_i^j \in \R \; (i, j = 1, \dots, m)$が存在して
    $f_i = a_i^j \tilde{f}_j \; (i = 1, \dots, m)$
    および
    $\tilde{e}^j = a_i^j e^j \; (j = 1, \dots, m)$
    が成り立っているから、
    \begin{alignat}{1}
        \myparen{
            \int_\calX \tilde{f}_j \, p(dx)
        } \tilde{e}^j
            &= \myparen{
                \int_\calX \tilde{f}_j \, p(dx)
            } a_i^j e^i \\
            &= \myparen{
                \int_\calX a_i^j \tilde{f}_j \, p(dx)
            } e^i
                \quad (\text{積分の$\R$-線型性}) \\
            &= \myparen{
                \int_\calX f_i \, p(dx)
            } e^i
    \end{alignat}
    が成り立つ。これで積分の well-defined 性も示せた。
\end{proof}

以降、本節では
$\calX$を可測空間、
$V$を有限次元$\R$-ベクトル空間 (次元を$m \in \Z_{\ge 0}$とおく)、
$p$を$\calX$上の確率測度、
$f \colon \calX \to V$を可測写像とする。

ベクトル値関数の積分を用いて、期待値および分散を定義する。

\begin{definition}[期待値]
    $f$が$p$-可積分であるとき、
    $f$の$p$に関する
    \term{期待値}[expected value]
        {期待値}[きたいち]
    $E_p[f]$を
    \begin{equation}
        E_p[f] \coloneqq \int_\calX f(x) \, p(dx)
            \in V
    \end{equation}
    と定義する。
\end{definition}

つぎに分散を定義する。
$\R$-値関数の通常の分散との類似で、
ベクトル値関数$f$の分散を
"$V_p[f] \coloneqq E_p[(f - E_p[f]) \otimes (f - E_p[f])]$"
と定義したい。
その準備として次の補題を示しておく。

\begin{lemma}[分散の存在条件]
    \label[lemma]{lemma:f_otimes_f}
    可測写像$f \colon \calX \to V$に関し
    次の2条件は同値である:
    \begin{enumerate}
        \item $f$および
            $(f - E_p[f]) \otimes (f - E_p[f])$が
            $p$-可積分
        \item $f \otimes f$が$p$-可積分
    \end{enumerate}
\end{lemma}

この補題の証明には次の事実を用いる。

\begin{fact}
    \label[fact]{fact:l2_subset_l1}
    $\calY$を可測空間、
    $\mu$を$\calY$上の有限測度とする。
    このとき、
    任意の実数$1 < p < +\infty$に対し
    $L^p(\calY, \mu) \subset L^1(\calY, \mu)$が成り立つ。
\end{fact}

\begin{proof}[\cref{lemma:f_otimes_f}の証明.]
    $\dim V = 0$の場合は明らかに成り立つ。
    以後$\dim V \ge 1$の場合を考える。
    $V$の基底$e^1, \dots, e^m$をひとつ選んで固定し、
    この基底に関する$f$の成分を
    $f_i \colon \calX \to \R \; (i = 1, \dots, m)$とおいておく。

    \uline{(1) \Rightarrow (2)} \quad
    $f$が$p$-可積分であることより
    $E_p[f] \in V$が存在するから、
    これを$a \coloneqq E_p[f]$とおき、
    $V$の基底$e^i$に関する$a$の成分を
    $a_i \in \R \; (i = 1, \dots, m)$とおいておく。
    示すべきことは、
    すべての$i, j = 1, \dots, m$に対し
    $f_i f_j \in L^1(\calX, p)$が成り立つことである。
    そこで次のことに注意する:
    \begin{enumerate}[label=(\roman*)]
        \item $p$が確率測度であることより$1 \in L^1(\calX, p)$である。
        \item $f$が$p$-可積分であることより
            $f_i \in L^1(\calX, p) \; (i = 1, \dots, m)$である。
        \item $(f - a) \otimes (f - a)$が$p$-可積分であることより
            $(f_i - a_i)(f_j - a_j)
                = f_i f_j - a_i f_j - a_j f_i + a_i a_j \in L^1(\calX, p) \;
                (i, j = 1, \dots, m)$である。
    \end{enumerate}
    したがって、
    $L^1(\calX, p)$が$\R$-ベクトル空間であることとあわせて
    $f_i f_j \in L^1(\calX, p) \; (i, j = 1, \dots, m)$が成り立つ。
    よって$f \otimes f$は$p$-可積分である。

    \uline{(2) \Rightarrow (1)} \quad
    まず$f$が$p$-可積分であることを示す。
    そのためには、
    $f_i \in L^1(\calX, p) \; (i = 1, \dots, m)$
    が成り立つことをいえばよい。
    いま$f \otimes f$が$p$-可積分であるから、
    $f_i f_j \in L^1(\calX, p) \; (i, j = 1, \dots, m)$
    が成り立つ。
    とくにすべての$i = 1, \dots, m$に対し
    $f_i \in L^2(\calX, p)$が成り立つから、
    \cref{fact:l2_subset_l1}とあわせて
    $f_i \in L^1(\calX, p)$が成り立つ。
    よって$f$は$p$-可積分である。

    つぎに$(f - E_p[f]) \otimes (f - E_p[f])$が$p$-可積分であることを示す。
    いま$f$が$p$-可積分であることより
    $E_p[f] \in V$が存在するから、
    これを$a \coloneqq E_p[f]$とおき、
    $V$の基底$e^i$に関する$a$の成分を
    $a_i \in \R \; (i = 1, \dots, m)$とおいておく。
    示したいことは、
    $(f_i - a_i)(f_j - a_j)
        = f_i f_j - a_i f_j - f_i a_j + a_i a_j
        \in L^1(\calX, p) \; (i, j = 1, \dots, m)$
    が成り立つことである。
    そこで次のことに注意する:
    \begin{enumerate}[label=(\roman*)]
        \item $p$が確率測度であることより$1 \in L^1(\calX, p)$である。
        \item $f$が$p$-可積分であることより
            $f_i \in L^1(\calX, p) \; (i = 1, \dots, m)$である。
        \item $f \otimes f$が$p$-可積分であることより
            $f_i f_j
                \in L^1(\calX, p) \;
                (i, j = 1, \dots, m)$である。
    \end{enumerate}
    したがって、
    $L^1(\calX, p)$が$\R$-ベクトル空間であることとあわせて
    $(f_i - a_i)(f_j - a_j)
        = f_i f_j - a_i f_j - f_i a_j + a_i a_j
        \in L^1(\calX, p) \; (i, j = 1, \dots, m)$
    が成り立つ。
    よって$(f - E_p[f]) \otimes (f - E_p[f])$は$p$-可積分である。
\end{proof}

この補題を踏まえて分散を定義する。

\begin{definition}[分散]
    $f \otimes f \colon \calX \to V \otimes_\R V$が$p$-可積分であるとき、
    $f$の$p$に関する
    \term{分散}[variance]
        {分散}[ぶんさん]
    $V_p[f]$を
    \begin{equation}
        V_p[f]
            \coloneqq E_p[(f - E_p[f]) \otimes (f - E_p[f])]
            \in V \otimes V
    \end{equation}
    と定義する
    (\cref{lemma:f_otimes_f}よりこれは存在する)。
\end{definition}

正規分布族の十分統計量について期待値と分散を求めてみる。

\begin{example}[正規分布族の十分統計量の期待値と分散]
    $\calX = \R$、
    $\lambda$を$\R$上の Lebesgue 測度とし、
    正規分布族
    \begin{equation}
        \calP \coloneqq \mybrace{
            P_{(\mu, \sigma^2)}(dx)
                = \frac{1}{\sqrt{2\pi\sigma^2}} \exp\myparen{
                    -\frac{(x - \mu)^2}{2\sigma^2}
                } \lambda(dx)
            \;\Big|\;
            \mu \in \R, \; \sigma^2 > 0
        }
    \end{equation}
    と$\calP$の実現$(V, T, \mu), \;
        V = \R^2, \;
        T \colon \calX \to V, \;
        x \mapsto \up{t}(x, x^2)$を考える。
    各$P = P_{(\mu, \sigma^2)} \in \calP$に対し、
    $T$の期待値$E_p[T] \in V$と
    分散$V_p[T] \in V \otimes V$を求めてみる。
    ただし、以下$x, \dots, x^4$の$P$に関する可積分性は仮定する
    (可積分性は次回示す)。

    まず期待値を求める。
    求めるべきものは、
    $V = \R^2$の標準基底を$e_1, e_2$として
    \begin{equation}
        E_P[T]
            = E_P[x] \, e_1 + E_P[x^2] \, e_2
    \end{equation}
    である。
    各成分は$E_P[x] = \mu, \;
        E_P[x^2]
            = E_P[(x - \mu)^2] + E_P[x]^2
            = \sigma^2 + \mu^2 \in \R$
    と求まるから
    \begin{equation}
        E_P[T]
            = \mu \, e_1 + (\sigma^2 + \mu^2) \, e_2
    \end{equation}
    である。

    次に分散を求める。
    求めるべきものは
    \begin{equation}
        V_P[T]
            = E_P[(T - E_P[T]) \otimes (T - E_P[T])]
    \end{equation}
    である。
    これを$V \otimes V$の基底
    $e_i \otimes e_j \; (i, j = 1, 2)$
    に関して成分表示すると
    \begin{alignat}{1}
        V_P[T]
            &=
                E_P[(x - \mu)^2] \, e_1 \otimes e_1 \\
            &\quad +
                E_P[(x - \mu)(x^2 - (\sigma^2 + \mu^2))] \,
                (e_1 \otimes e_2 + e_2 \otimes e_1) \\
            &\quad +
                E_P[(x^2 - (\sigma^2 + \mu^2))^2] \, e_2 \otimes e_2
    \end{alignat}
    と表される。
    そこで原点周りのモーメント
    $a_3 \coloneqq E_P[x^3], \; a_4 \coloneqq E_P[x^4] \in \R$とおくと、
    各成分は
    \begin{alignat}{1}
        E_P[(x - \mu)^2]
            &= \sigma^2 \\
        E_P[(x - \mu)(x^2 - (\sigma^2 + \mu^2))]
            &= a_3 - \mu (\sigma^2 + \mu^2) \\
        E_P[(x^2 - (\sigma^2 + \mu^2))^2]
            &= a_4 - (\sigma^2 + \mu^2)^2
    \end{alignat}
    と求まる。
    したがって$V_P[T]$は
    \begin{alignat}{1}
        V_P[T]
            &= \sigma^2 \, e_1 \otimes e_1 \\
            &\quad +
                (a_3 - \mu (\sigma^2 + \mu^2)) \,
                (e_1 \otimes e_2 + e_2 \otimes e_1) \\
            &\quad +
                (a_4 - (\sigma^2 + \mu^2)^2) \, e_2 \otimes e_2
    \end{alignat}
    と表される。
    最後に原点周りのモーメント
    $a_3, a_4$を具体的に求める。
    これは期待値周りのモーメントの計算に帰着される。
    そこで標準正規分布を$P_0 \coloneqq P_{(0, 1)} \in \calP$とおくと、
    $E_P\mybracket{\myparen{
        \frac{x - \mu}{\sigma}
    }^k} = E_{P_0}[x^k] \; (k = 3, 4)$より
    $E_P[(x - \mu)^k] = \sigma^k E_{P_0}[x^k] \; (k = 3, 4)$が成り立つ。
    ここで$P_0$に関する期待値を
    部分積分などを用いて直接計算すると
    $E_{P_0}[x^3] = 0, \; E_{P_0}[x^4] = 3$となるから、
    $E_P[(x - \mu)^3] = 0, \; E_P[(x - \mu)^4] = 3 \sigma^4$を得る。
    これらを用いて$a_3, a_4$を計算すると
    \begin{alignat}{1}
        0
            &= E_P[(x - \mu)^3] \\
            &= E_P[x^3] - 3 E_P[x^2] \mu + 3 E_P[x] \mu^2 - \mu^3 \\
            &= a_3 - 3 (\sigma^2 + \mu^2) \mu + 3 \mu^3 - \mu^3 \\
            &= a_3 - 3 \sigma^2 \mu - \mu^3 \\
        \therefore a_3
            &= 3 \sigma^2 \mu + \mu^3
        \intertext{および}
        3 \sigma^4
            &= E_P[(x - \mu)^4] \\
            &= E_P[x^4] - 4 E_P[x^3] \mu + 6 E_P[x^2] \mu^2
                - 4 E_P[x] \mu^3 + \mu^4 \\
            &= a_4 - 4 a_3 \mu + 6 (\sigma^2 + \mu^2) \mu^2
                - 4 \mu^4 + \mu^4 \\
            &= a_4 - 6 \sigma^2 \mu^2 - \mu^4 \\
        \therefore a_4
            &= 3 \sigma^4 + 6 \sigma^2 \mu^2 + \mu^4
    \end{alignat}
    を得る。
    これらを$V_P[T]$の成分表示に代入して
    \begin{alignat}{1}
        V_P[T]
            &= \sigma^2 \, e_1 \otimes e_1 \\
            &\quad +
                2 \sigma^2 \mu \,
                (e_1 \otimes e_2 + e_2 \otimes e_1) \\
            &\quad +
                (4 \sigma^2 \mu^2 + 2 \sigma^4) \, e_2 \otimes e_2
    \end{alignat}
    となる。
    行列表示は$\begin{bmatrix}
        \sigma^2 & 2 \sigma^2 \mu \\
        2 \sigma^2 \mu & 4 \sigma^2 \mu^2 + 2 \sigma^4
    \end{bmatrix} \in M_2(\R)$となり、
    これは対称かつ正定値である。
\end{example}

\begin{remark}[Fisher 情報行列に関する (あまり厳密でない) 補足]
    上の例の$V_P[T]$は
    座標変換
    $\R \times \R_{> 0} \to \R^2, \;
        (\mu, \sigma)
            \mapsto
            (\theta_1, \theta_2)
            = \myparen{\frac{\mu}{\sigma^2}, - \frac{1}{2 \sigma^2}}$
    の Jacobi 行列$J$による合同変換により
    $\up{t}J V_P[T] J = \begin{bmatrix}
        \frac{1}{\sigma^2} & 0 \\
        0 & \frac{2}{\sigma^2}
    \end{bmatrix}$に写る。
    これは正規分布のパラメータ$(\mu, \sigma)$に関する Fisher 情報行列に一致する
    (詳細は次回以降で Fisher 計量の話題のときに触れたい)。
\end{remark}

上の例でみた$V_P[T]$は正定値対称であった。
より一般に分散は次の性質を持つ。

\begin{theorem}[分散の半正定値対称性]
    $V_p[f] \in V \otimes_\R V$は、
    $V^\vee$上の$\R$-双線型形式とみて
    対称かつ半正定値である。
\end{theorem}

\begin{remark}[双線型形式としてのテンソル]
    一般に、$\omega \in V \otimes V$を
    $V^\vee$上の$\R$-双線型形式とみるとは、
    $V$の基底$e^i \; (i = 1, \dots, m)$をひとつ選んで
    $\omega$の成分表示を
    $\omega = \omega_{ij} e^i \otimes e^j, \;
        \omega_{ij} \in \R$
    とおくとき、
    \begin{equation}
        \omega(\theta, \theta')
            \coloneqq \omega_{ij} \theta(e^i) \theta'(e^j)
            \in \R
    \end{equation}
    と定めるということである (これは well-defined である)。
\end{remark}

\begin{proof}
    まず$V_p[f]$が対称であることを示す。
    そこで$V$の基底$e^i \; (i = 1, \dots, m)$をひとつ選んで固定し、
    $f, E_p[f]$の成分表示を
    それぞれ$f_i, m_i \; (i = 1, \dots, m)$とおく。
    すると
    \begin{alignat}{1}
        V_p[f]
            &= E_p[(f - E_p[f]) \otimes (f - E_p[f])] \\
            &= \myparen{
                \int_\calX
                (f_i(x) - m_i)(f_j(x) - m_j)
                \, p(dx)
            } e^i \otimes e^j
    \end{alignat}
    となり、最後の行の
    $\int_\calX (f_i(x) - m_i)(f_j(x) - m_j) \, p(dx)$
    は添字の置換に関し不変である。
    したがって$V_p[f]$は対称である。

    つぎに$V_p[f]$が半正定値であることを示す。
    示したいことは、各$\theta \in V^\vee$に対し
    $V_p[f](\theta, \theta) \ge 0$が成り立つことであるが、
    これは
    \begin{alignat}{1}
        V_p[f](\theta, \theta)
            &= \sum_{i, j}
                \myparen{
                    \int_\calX (f_i(x) - m_i)(f_j(x) - m_j) \, p(dx)
                }
                \theta(e^i) \theta(e^j) \\
            &= \int_\calX \myparen{
                \sum_{i, j}
                \theta(e^i)(f_i(x) - m_i)
                \theta(e^j)(f_j(x) - m_j)
            } \, p(dx) \\
            &= \int_\calX \myparen{
                \sum_{i}
                \theta(e^i)(f_i(x) - m_i)
            }^2 \, p(dx) \\
            &\ge 0
    \end{alignat}
    より従う。
    したがって$V_p[f]$は半正定値である。
\end{proof}

最後に、分散が0であることの特徴づけを与えておく。

\begin{proposition}[分散が0であるための必要十分条件]
    \label[proposition]{prop:zero_variance_condition}
    分散を持つ可測写像$f \colon \calX \to V$に関し、
    $V_p[f] = 0$であることと
    $f$が$p$-a.e.定数であることは同値である。
\end{proposition}

証明には次の事実を用いる。

\begin{fact}
    \label[fact]{fact:nonnegative_func}
    $\calY$を可測空間、
    $\mu$を$\calY$上の測度とする。
    このとき、
    $g(x) \ge 0 \; \text{$\mu$-a.e.$x$}$なる
    $\mu$-可積分関数$g \colon \calY \to \R$に関し、
    $\int_\calY g(x) \, \mu(dx) = 0$であることと
    $g = 0 \; \text{$\mu$-a.e.$x$}$であることは同値である。
\end{fact}

\begin{proof}[\cref{prop:zero_variance_condition}の証明.]
    ここでは「$p$-a.e.」を「a.e.」と略記する。
    $V$の基底$e^i \; (i = 1, \dots, m)$をひとつ選んで固定し、
    この基底に関する
    $f$の成分を$f_i \colon \calX \to \R \; (i = 1, \dots, m)$、
    $E_p[f]$の成分を$a_i \in \R \; (i = 1, \dots, m)$とおいておく。

    \uline{(\Leftarrow)} \quad
    $f$がa.e.定数ならば、
    $f_i(x) = a_i \;
        \text{a.e.$x$} \;
        (i = 1, \dots, m)$
    したがって
    $(f_i(x) - m_i)(f_j(x) - m_j) = 0 \;
        \text{a.e.$x$} \;
        (i, j = 1, \dots, m)$
    である。
    よって
    $\int_\calX (f_i(x) - m_i)(f_j(x) - m_j) \, p(dx) = 0 \;
        (i, j = 1, \dots, m)$
    だから
    $V_p[f] = 0$である。

    \uline{(\Rightarrow)} \quad
    対偶を示すため、
    $f$はa.e.定数ではないと仮定する。
    すると、
    $f_i$がa.e.定数ではないような
    ある$i \in \{ 1, \dots, m \}$が存在する。
    このとき$(f_i - m_i)^2 = 0 \; \text{a.e.}$ではないから、
    \cref{fact:nonnegative_func}より
    $\int_\calX (f_i(x) - m_i)^2 \, p(dx) > 0$である。
    したがって$V_p[f] \neq 0$である。
\end{proof}

% ------------------------------------------------------------
%
% ------------------------------------------------------------
\section{今後の予定}

\begin{itemize}
    \item 期待値パラメータ空間
    \item 対数分配関数の微分可能性および Hessian の正定値性
    \item 対数分配関数から誘導されるダイバージェンス
\end{itemize}

\TODO{このような流れでよい?}

% ------------------------------------------------------------
%
% ------------------------------------------------------------
\section{参考文献}

\nocite{amari_information_2016}
\nocite{wainwright_graphical_2007}
\nocite{bn1970_pdf}

{
    \renewcommand{\bibsection}{}
    \bibliographystyle{amsalpha}
    \bibliography{./bibliography,../../mybibliography}
}


\end{document}