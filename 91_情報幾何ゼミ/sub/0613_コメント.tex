\documentclass[report]{jlreq}
\usepackage{global}
\usepackage{./local}
\subfiletrue
\def\assetspath{../}
%\makeindex
\chead{2023/06/13}
\begin{document}

% ============================================================
%
% ============================================================

発表中にコメントがあった事柄を整理する。

\begin{problem}
    $\wt{g}$を勝手な Riemann 計量とし、
    何らかの方法で座標を取り替えたとして、
    同じ方法で$\calP$上に引き戻した$g$は
    well-defined となるか?
    \TODO{もっときちんと定式化する}
\end{problem}

\begin{proposition}
    $M$を多様体とする。
    このとき、
    $M$上のアファイン接続全体の集合$\calA(M)$は、
    $\R$-ベクトル空間
    $\Hom_\R(\Gamma(TM), \Gamma(T^\vee M \otimes TM))$
    のアファイン部分空間であり、
    そのモデル空間は
    $\Gamma(T^\vee M \otimes T^\vee M \otimes TM)$である。
\end{proposition}

\begin{proof}
    $M$上のアファイン接続$\nabla^0$をひとつ選んで固定する。
    このとき、
    $\nabla^0$に任意の$A \in \Gamma(T^\vee M \otimes T^\vee M \otimes TM)$を
    加えた$\nabla^0 + A$は
    $M$上のアファイン接続となるから、
    $\nabla_0 + \Gamma(T^\vee M \otimes T^\vee M \otimes TM)
        \subset \calA(M)$
    が成り立つ。
    逆に
    任意の$\nabla \in \calA(M)$に対し
    $\nabla - \nabla_0$は
    $\Gamma(T^\vee M \otimes T^\vee M \otimes TM)$の元となるから、
    $\calA(M) \subset \nabla_0 + \Gamma(T^\vee M \otimes T^\vee M \otimes TM)$
    が成り立つ。
    したがって
    $\calA(M) = \nabla_0 + \Gamma(T^\vee M \otimes T^\vee M \otimes TM)$
    となり、
    $\calA(M)$は
    $\Hom_\R(\Gamma(TM), \Gamma(T^\vee M \otimes TM))$の
    アファイン部分空間となることがわかる。
\end{proof}

\begin{problem}
    $R^{(-1)} = 0$を示せ。
\end{problem}

\begin{answer}
    $\nabla$-アファイン座標をひとつ選ぶと、
    \url{0613_資料.pdf}命題1.12 (2) より
    \begin{equation}
        {R^{(-1)}}_{ijk}^l
            =
                \del_i A_{jk}^l
                -
                \del_j A_{ik}^l
                +
                A_{jk}^m A_{im}^l
                -
                A_{ik}^m A_{jm}^l
                \locallabel{eq:1}
    \end{equation}
    と表せるから、
    この右辺が$0$となることを示せばよい。
    まず
    \begin{alignat}{1}
        &\phantom{=}
            (\text{式\localcref{eq:1}の右辺第1項})
            \\
        &=
            \del_i A_{jk}^l
            \\
        &=
            \del_i (g^{ln} S_{jkn})
            \\
        &=
            \del_i (g^{ln}) S_{jkn}
            +
            g^{lm} \del_i S_{jkm}
            \\
        &=
            - \del_i (g_{mn}) g^{mn} g^{ln} S_{jkn}
            +
            g^{lm} \del_i S_{jkm}
            \qquad
            (\del_i(g_{nm} g^{lm}) = 0)
            \\
        &=
            - S_{imn} g^{mn} g^{ln} S_{jkn}
            +
            g^{lm} \del_i S_{jkm}
            \\
        &=
            - A_{im}^l A_{jk}^m
            +
            g^{lm} \del_i S_{jkm}
    \end{alignat}
    同様にして
    \begin{alignat}{1}
        &\phantom{=}
            (\text{式\localcref{eq:1}の右辺第2項})
            \\
        &=
            - \del_j A_{ik}^l
            \\
        &=
            \cdots
            \\
        &=
            A_{jm}^l A_{ik}^m
            -
            g^{lm} \del_j S_{ikm}
    \end{alignat}
    を得る。
    これらを合わせて
    \begin{alignat}{1}
        &\phantom{=}
            (\text{式\localcref{eq:1}の右辺第1項})
                + (\text{式\localcref{eq:1}の右辺第2項})
            \\
        &=
            - A_{im}^l A_{jk}^m
            +
            g^{lm} \del_i S_{jkm}
            +
            A_{jm}^l A_{ik}^m
            -
            g^{lm} \del_j S_{ikm}
            \\
        &=
            - A_{im}^l A_{jk}^m
            +
            A_{jm}^l A_{ik}^m
            \qquad
            (\del_i S_{jkm} = \del_j S_{ikm})
    \end{alignat}
    となる。
    これは
    式\localcref{eq:1}の右辺第3, 4項
    の符号を反転させたものに一致するから、
    ${R^{(-1)}}_{ijk}^l = 0$が従う。
    よって$R^{(-1)} = 0$である。
\end{answer}

\begin{definition}[weak-* topology]
    \TODO{cf. \url{https://math.stackexchange.com/a/2134037}}
\end{definition}

% ------------------------------------------------------------
%
% ------------------------------------------------------------
\section*{参考文献}

\nocite{amari_information_2016}

{
    \renewcommand{\bibsection}{}
    \bibliographystyle{amsalpha}
    \bibliography{./bibliography,../../mybibliography}
}


\end{document}