\documentclass[report]{jlreq}
\usepackage{global}
\usepackage{./local}
\subfiletrue
\def\assetspath{../}
%\makeindex
\chead{2023/06/06}
\begin{document}

% ============================================================
%
% ============================================================

% ------------------------------------------------------------
%
% ------------------------------------------------------------
\section*{振り返りと導入}

前回は自然パラメータ空間に Fisher 計量を定義した。
本稿では次のことを行う:
\begin{itemize}
    \item 最小次元実現の間のアファイン変換の一意存在を述べる。
\end{itemize}

$\calX$を可測空間、
$\calP \subset \calP(\calX)$を$\calX$上の指数型分布族とする。
新たな用語として次の2つを導入しておく。

\begin{definition}[自然パラメータ付け]
    $(V, T, \mu)$を$\calP$の実現、
    $\psi$を$(V, T, \mu)$の対数分配関数とする。
    \begin{equation}
        P_{(V, T, \mu)} \colon \Theta_{(V, T, \mu)} \to \calP(\calX),
            \quad
            \theta
            \mapsto
            \exp (\myangle{\theta}{T(x)} - \psi(\theta)) \cdot \mu
    \end{equation}
    を$(V, T, \mu)$による$\calP(\calX)$の
    \term{自然パラメータ付け}[natural parametrization]
        {自然パラメータ付け}[しぜんパラメータづけ]
    という。
\end{definition}

\begin{definition}[真パラメータ空間]
    $(V, T, \mu)$を$\calP$の実現とする。
    \begin{equation}
        \Theta^\calP_{(V, T, \mu)}
            \coloneqq P_{(V, T, \mu)}^{-1}(\calP)
    \end{equation}
    を$\calP$の$(V, T, \mu)$に関する
    \term{真パラメータ空間}[strict parameter space]
        {真パラメータ空間}[しんパラメータくうかん]
    という。
\end{definition}

以降、
$\calP$の実現$(V, T, \mu), (V', T', \mu')$に対し、
それぞれによる自然パラメータ付けを$P, P'$と略記したり、
それぞれに関する真パラメータ空間を$\Theta, \Theta'$と略記したりすることがある。

% ------------------------------------------------------------
%
% ------------------------------------------------------------
\section{最小次元実現の間のアファイン変換}

本節の目標は、
最小次元実現の間のアファイン変換の一意存在を述べた
\cref{thm:transformation-between-representations}
の証明である。
本節では、
ステートメントを簡潔にするために圏の言葉を用いる。

\begin{propdef}[$\calP$の実現とアファイン写像の圏]
    次のデータにより圏が定まる:
    \begin{itemize}
        \item 対象: $\calP$の実現$(V, T, \mu)$全体
        \item 射: $(V, T, \mu)$から$(V', T', \mu')$への射は、
            $V$から$V'$への全射アファイン写像
            $(L, b) \; (L \in \Lin(V, V'), \; b \in V')$
            であって
            $T'(x) = L(T(x)) + b \; \text{$\mu$-a.e.$x$}$をみたすもの
        \item 合成: アファイン写像の合成
            $(L, b) \circ (K, c) = (LK, Lc + b)$
    \end{itemize}
    この圏を\termsilent{$\calP$の実現とアファイン写像の圏}と呼び、
    $\bfC_\calP$と書く。
\end{propdef}

\begin{proof}
    示すべきことは、射の合成が射であること、恒等射の存在、結合律の3点である。
    射の合成が射であることは、
    全射と全射の合成が全射であることと、
    $\mu \ll \mu'$かつ$\mu' \ll \mu$
    が成り立つことより従う。
    また、$(V, T, \mu)$の恒等射は明らかに恒等写像$(\id_V, 0)$であり、
    結合律はアファイン写像の合成の結合律より従う。
\end{proof}

最小次元実現を特徴づける2つの条件を導入する。

\begin{propdef}[条件A]
    $\calP$の実現$(V, T, \mu)$に関する次の条件は同値である:
    \begin{enumerate}
        \item $P \colon \Theta \to \calP(\calX)$は単射である。
        \item $\forall \theta \in V^\vee$
            に対し
            「
                $\myangle{\theta}{T(x)} = \text{const. $\mu$-a.e.$x$}
                \implies
                \theta = 0$
            」
            が成り立つ。
        \item $V$の任意の真アファイン部分空間$W$に対し、
            「$T(x) \in W \; \text{$\mu$-a.e.$x$}$でない」
            が成り立つ。
    \end{enumerate}
    これらの条件が成り立つとき、
    $(V, T, \mu)$は\termsilent{条件A}をみたすという。
\end{propdef}

\begin{proof}
    (1) $\iff$ (2) は\url{0502_資料.pdf}の命題2.2で示した。
    (2) $\iff$ (3) は\url{0523_コメント.pdf}の命題0.4に記した。
\end{proof}

\begin{definition}[条件B]
    $\calP$の実現$(V, T, \mu)$に関する条件
    \begin{enumerate}
        \item $\Theta^\calP$は
            $V^\vee$を affine span する。
    \end{enumerate}
    が成り立つとき、
    $(V, T, \mu)$は\termsilent{条件B}をみたすという。
\end{definition}

条件Aは射の一意性を保証する。

\begin{lemma}[射の一意性]
    \label[lemma]{lemma:uniqueness-of-morphism}
    $(V, T, \mu), (V', T', \mu')$を$\bfC_\calP$の対象とする。
    このとき、
    $(V, T, \mu)$が条件Aをみたすならば
    $(V, T, \mu)$から$(V', T', \mu')$への射は一意である。
\end{lemma}

\begin{proof}
    $(L, b), (K, c)$を$(V, T, \mu)$から$(V', T', \mu')$への射とする。
    射の定義より
    \begin{equation}
        \begin{cases}
            T'(x) = L(T(x)) + b & \text{$\mu$-a.e.$x$} \\
            T'(x) = K(T(x)) + c & \text{$\mu$-a.e.$x$}
        \end{cases}
    \end{equation}
    が成り立つから、2式を合わせて
    \begin{equation}
        (K - L)(T(x)) = b - c \qquad \text{$\mu$-a.e.$x$}
    \end{equation}
    となる。
    そこで基底を固定して
    成分ごとに$(V, T, \mu)$の条件A(2)を適用すれば、
    $K = L$を得る。
    よって
    上式で$K = L$として
    $b = c \; \text{$\mu$-a.e.}$
    したがって$b = c$を得る。
    以上より$(L, b) = (K, c)$である。
\end{proof}

射が存在するための十分条件を調べる。
この後の中間目標は
\cref{lemma:affine-map-between-representations}
であり、その証明は次の図のような2段階に分けて行う。
\begin{equation}
    \begin{tikzcd}
        (V', T', \mu')
            \ar{rd}[swap]{\cref{lemma:mu_dash_to_mu}}
            \ar{rr}{\cref{lemma:affine-map-between-representations}}
            && (V, T, \mu)
            \\
        &(V', T', \mu)
            \ar{ru}[swap]{\cref{lemma:V_dash_T_dash_to_V_T}}
    \end{tikzcd}
\end{equation}

\begin{lemma}[基底測度だけを変化させた対象からの射]
    \label[lemma]{lemma:mu_dash_to_mu}
    $(V, T, \mu')$から$(V, T, \mu)$への射が存在する。
\end{lemma}

\begin{proof}
    $(\id, 0)$が求める射である。
\end{proof}

\begin{lemma}[基底測度以外を変化させた対象からの射]
    \label[lemma]{lemma:V_dash_T_dash_to_V_T}
    $(V, T, \mu)$が 条件Aと条件Bをみたすならば、
    任意の対象$(V', T', \mu)$から$(V, T, \mu)$への射が存在する。
\end{lemma}

\begin{proof}
    \uline{Step 0: $V, V^\vee$の基底を選ぶ} \quad
    $(V, T, \mu)$の条件Bより、
    $V^\vee$のあるアファイン基底
    $a^i \in \Theta^\calP \; (i = 0, \dots, m)$
    が存在する。
    そこで
    $e^i \coloneqq a^i - a^0 \in V^\vee \; (i = 1, \dots, m)$
    とおくとこれは$V^\vee$の基底である。
    さらに$e^i$の双対基底を$V$の元と同一視したものを
    $e_i \in V \; (i = 1, \dots, m)$とおいておく。

    \uline{Step 1: 射$(L, b)$の構成} \quad
    $P, P'$の右逆写像
    $\theta \colon \calP \to \Theta^\calP$
    および
    $\theta' \colon \calP \to \Theta'^\calP$
    をひとつずつ選んで
    $p^i \coloneqq P(a^i) \in \calP \; (i = 0, \dots, m)$とおき、
    $(L, b)$を次のように定める:
    \begin{alignat}{1}
        &L \colon V' \to V,
            \quad
            t' \mapsto
                \myangle{\theta'(p^i) - \theta'(p^0)}{t'} e_i
            \\
        &b \coloneqq
            \mybrace{
                \psi(\theta(p^i)) - \psi(\theta(p^i))
                - \psi(\theta'(p^0)) + \psi(\theta'(p^0))
            } e_i
            \in V
    \end{alignat}
    示すべきことは、
    すべての$p \in \calP$に対し
    \begin{equation}
        T(x) = L(T'(x)) + b
            \quad
            \text{$\mu$-a.e.$x$}
    \end{equation}
    が成り立つことと、
    $(L, b)$が全射となることである。

    \uline{Step 2: $T(x) = L(T'(x)) + b$の証明} \quad
    指数型分布族の定義より、
    任意の$p \in \calP$に対し、
    ある$\mu$-零集合$N_p \subset \calX$が存在して
    \begin{alignat}{2}
        &\phantom{\therefore}& \quad
            \exp(\myangle{\theta(p)}{T(x)} - \psi(\theta(p)))
                =
                    \dd[p]{\mu}(x)
                &=
                    \exp(\myangle{\theta'(p)}{T'(x)} - \psi'(\theta'(p)))
                \qquad
                    (x \in \calX \setminus N_p)
            \\
        &\therefore& \quad
            \myangle{\theta(p)}{T(x)}
                - \myangle{\theta'(p)}{T'(x)}
                &=
                    \psi(\theta(p))
                    - \psi'(\theta'(p))
                \qquad
                    (x \in \calX \setminus N_p)
            \label{eq:pairings-and-log-partition}
    \end{alignat}
    が成り立つ。
    とくに各$i = 0, \dots, m$に対し
    \begin{equation}
        \myangle{\theta(p^i)}{T(x)}
            - \myangle{\theta'(p^i)}{T'(x)}
            =
                \psi(\theta(p^i))
                - \psi'(\theta'(p^i))
            \qquad
                (x \in \calX \setminus N_{p^i})
    \end{equation}
    が成り立つから、
    $\text{($i$の場合)} - \text{($0$の場合)}$を考えて
    \begin{equation}
        \begin{alignedat}{1}
            &\phantom{=}
                \myangle{\theta(p^i) - \theta(p^0)}{T(x)}
                - \myangle{\theta'(p^i) - \theta'(p^0)}{T'(x)}
                \\
            &=
                \psi(\theta(p^i))
                - \psi'(\theta'(p^i))
                - \psi(\theta(p^0))
                + \psi'(\theta'(p^0))
                \qquad
                (x \in \calX \setminus (N_{p^0} \cup N_{p^i}))
        \end{alignedat}
        \locallabel{eq:1}
    \end{equation}
    となる。
    ここで$(V, T, \mu)$の条件A (1)より
    $\theta(p^i) = a^i$
    が成り立つから、
    \localcref{eq:1}より
    \begin{equation}
        \begin{alignedat}{1}
            \myangle{a^i - a^0}{T(x)}
                &=
                    \myangle{\theta'(p^i) - \theta'(p^0)}{T'(x)}
                    \\
                &\qquad
                    + \psi(\theta(p^i))
                    - \psi'(\theta'(p^i))
                    - \psi(\theta(p^0))
                    + \psi'(\theta'(p^0))
                    \qquad
                    (x \in \calX \setminus (N_{p^0} \cup N_{p^i}))
        \end{alignedat}
    \end{equation}
    したがって
    \begin{equation}
        T(x) = L(T'(x)) + b
            \qquad
                (x \in \calX \setminus (N_{p^0} \cup \cdots \cup N_{p^m}))
    \end{equation}
    が成り立つ。
    これで Step 2 が完了した。

    \uline{Step 3: $(L, b)$が全射であることの証明} \quad
    $L$が全射であることをいえばよい。
    もし$L$が全射でなかったとすると、
    $T(x) = L(T'(x)) + b \in \Im L + b$
    が$p$-a.e.$x$すなわち$\mu$-a.e.$x$に対し成り立つことになるが、
    $\Im L + b$は$V$の真アファイン部分空間だから
    $(V, T, \mu)$の条件A (3)に反する。
    したがって$L$は全射である。
\end{proof}

\begin{lemma}[2条件をみたす対象への射]
    \label[lemma]{lemma:affine-map-between-representations}
    $(V, T, \mu)$が 条件Aと条件Bをみたすならば、
    任意の対象$(V', T', \mu')$から$(V, T, \mu)$への射が存在する。
\end{lemma}

\begin{proof}
    上の2つの補題より存在する2つの射
    $(V', T', \mu') \to (V', T', \mu) \to (V, T, \mu)$
    を合成すればよい。
\end{proof}

各条件をみたさない場合にも、射の存在が保証される。

\begin{lemma}[単調性条件をみたさない場合の射]
    \label[lemma]{lemma:morphism-existence-a}
    $(V, T, \mu)$が条件Aをみたさないならば、
    $(V, T, \mu)$よりも次元の小さい
    ある対象$(V', T', \mu')$への
    射$(V, T, \mu) \to (V', T', \mu')$が存在する。
\end{lemma}

\begin{proof}
    末尾の付録に載せた。
\end{proof}

\begin{lemma}[条件Bをみたさない場合の射]
    \label[lemma]{lemma:morphism-existence-b}
    $(V, T, \mu)$が条件Bをみたさないならば、
    $(V, T, \mu)$よりも次元の小さい
    ある対象$(V', T', \mu')$への
    射$(V, T, \mu) \to (V', T', \mu')$が存在する。
\end{lemma}

\begin{proof}
    末尾の付録に載せた。
\end{proof}

以上の補題を用いて
最小次元実現の特徴づけが得られる。

\begin{theorem}[最小次元実現の特徴づけ]
    \label[theorem]{thm:characterization-of-minimal-representation}
    $\calP$の実現$(V, T, \mu)$に関する次の条件は同値である:
    \begin{enumerate}
        \item $(V, T, \mu)$は$\calP$の最小次元実現である。
        \item $(V, T, \mu)$は条件Aと条件Bをみたす。
    \end{enumerate}
\end{theorem}

\begin{proof}
    \uline{(1) \Rightarrow (2)} \quad
    最小次元実現$(V, T, \mu)$が条件A, Bのいずれかをみたさなかったとすると、
    \cref{lemma:morphism-existence-a,lemma:morphism-existence-b}
    よりとくに$(V, T, \mu)$よりも次元の小さい実現が存在することになり、
    矛盾が従う。

    \uline{(2) \Rightarrow (1)} \quad
    $(V, T, \mu)$が条件Aと条件Bをみたすとする。
    $\calP$の任意の実現
    $(V', T', \mu')$に対し、
    \cref{lemma:affine-map-between-representations}より
    全射線型写像$L: V' \to V$が存在するから、
    $\dim V \le \dim V'$である。
    したがって$V$は$\calP$の最小次元実現である。
\end{proof}

\begin{example}[正規分布族の最小次元実現]
    \cref{thm:characterization-of-minimal-representation}により、
    \url{0425_資料.pdf}の例3.2でみた正規分布族の例は
    最小次元実現であることがわかる。
    実際、
    $T(x) = \up{t}(x, x^2)$の像は
    $\R^2$のいかなる真アファイン部分空間にも
    含まれないから、
    条件A (3)が成り立つ。
    また、$\Theta^\calP = \R \times \R_{< 0}$となることから
    条件Bも成り立つ。
\end{example}

本節の目標の定理を示す。

\begin{theorem}[最小次元実現の間のアファイン変換]
    $(V, T, \mu), (V', T', \mu')$が
    ともに最小次元実現ならば、
    $(V, T, \mu)$から$(V', T', \mu')$への射$(L, b)$がただひとつ存在する。
    さらに、$L$は線型同型写像である。
\end{theorem}

\begin{proof}
    \cref{lemma:uniqueness-of-morphism,lemma:affine-map-between-representations}
    より、射$(L, b) \colon (V, T, \mu) \to (V', T', \mu')$はただひとつ存在する。
    また、
    \cref{lemma:affine-map-between-representations}
    より存在する射$(V', T', \mu') \to (V, T, \mu)$をひとつ選んで
    $(K, c)$とおくと、
    合成射$(K, c) \circ (L, b), \; (L, b) \circ (K, c)$は
    \cref{lemma:uniqueness-of-morphism}より
    恒等射$(\id_V, 0), \; (\id_{V'}, 0)$に一致する。
    したがって$L$は線型同型写像である。
\end{proof}

同じことを圏の言葉を使わずに言い換えると次のようになる。

\begin{theorem}[最小次元実現の間のアファイン変換]
    \label[theorem]{thm:transformation-between-representations}
    $(V, T, \mu), (V', T', \mu')$が
    ともに最小次元実現ならば、
    全射線型写像$L: V \to V'$と
    ベクトル$b \in V'$であって
    \begin{equation}
        T(x) = L(T'(x)) + b
            \qquad
            \text{$\mu$-a.e.$x$}
    \end{equation}
    をみたすものがただひとつ存在する。
    さらに、$L$は線型同型写像である。
    \qed
\end{theorem}

\begin{corollary}[自然パラメータの変換]
    上の定理の$L$は
    \begin{equation}
        \theta'(p)
            = \up{t}L(\theta(p))
            \qquad
            (\forall p \in \calP)
    \end{equation}
    をみたす。
    ただし
    写像$\theta \colon \calP \to \Theta^\calP$
    および
    $\theta' \colon \calP \to \Theta'^\calP$は
    $P, P'$ (の$\Theta^\calP, \Theta'^\calP$上への制限) の逆写像である。
\end{corollary}

\begin{proof}
    $p \in \calP$を任意とすると、
    指数型分布族の定義より、
    $\mu'$-a.e.$x$に対し
    \begin{alignat}{2}
        &\phantom{\therefore} \quad&
            \exp(\myangle{\theta(p)}{T(x)} - \psi(\theta(p)))
                =
                    \dd[p]{\mu}(x)
                &=
                    \exp(\myangle{\theta'(p)}{T'(x)} - \psi'(\theta'(p)))
            \\
        &\therefore \quad&
            \myangle{\theta(p)}{T(x)}
                - \myangle{\theta'(p)}{T'(x)}
                &=
                    \psi(\theta(p))
                    - \psi'(\theta'(p))
            \\
        &\therefore \quad&
            \myangle{\theta(p)}{L(T'(x)) + b}
            - \myangle{\theta'(p)}{T'(x)}
            &=
                \psi(\theta(p)) - \psi'(\theta'(p))
                \\
        &\therefore \quad&
            \myanglem{\up{t}L(\theta(p))}{T'(x)}
            - \myangle{\theta'(p)}{T'(x)}
            &=
                \psi(\theta(p)) - \psi'(\theta'(p))
                - \myangle{\theta(p)}{b}
                \\
        &\therefore \quad&
            \myanglem{\up{t}L(\theta(p)) - \theta'(p)}{T'(x)}
            &=
                \psi(\theta(p)) - \psi'(\theta'(p))
                - \myangle{\theta(p)}{b}
    \end{alignat}
    が成り立つ。
    したがって、
    $(V', T', \mu')$の条件A (2)より
    $\up{t}L(\theta(p)) = \theta'(p)$
    が成り立つ。
\end{proof}


% ------------------------------------------------------------
%
% ------------------------------------------------------------
\section*{今後の予定}

\begin{itemize}
    \item 指数型分布族$\calP$自体に構造を入れる。
    \item Amari-Chentsov テンソルを定義する。
    \item 正規分布族の場合の具体的な計算を行う
        (Fisher 計量、Levi-Civita 接続、測地線など)。
\end{itemize}

% ------------------------------------------------------------
%
% ------------------------------------------------------------
\section*{参考文献}

\nocite{amari_information_2016}
\nocite{bn1970_pdf}
\nocite{BN78}

{
    \renewcommand{\bibsection}{}
    \bibliographystyle{amsalpha}
    \bibliography{./bibliography,../../mybibliography}
}

% ------------------------------------------------------------
%
% ------------------------------------------------------------
\newpage
\appendix
\section*{付録}

\begin{proof}[\cref{lemma:morphism-existence-a}の証明]
    $(V, T, \mu)$が
    条件Aをみたさないという仮定から、
    ある$\theta \in V^\vee, \; \theta \neq 0$および
    $r \in \R$が存在して
    \begin{equation}
        \locallabel{eq:costant-pairing}
        \langle \theta, T(x) \rangle
            = r
            \qquad
            \text{$\mu$-a.e.$x$}
    \end{equation}
    が成り立つ。
    そこで
    $V' \coloneqq (\R\theta)^\perp
        = \{ v \in V \mid \langle \theta, v \rangle = 0 \}$
    とおくと、
    ある可測写像$T' \colon \calX \to V'$および
    $v_0 \in V$が存在して
    $T(x) = T'(x) + v_0 \; (\text{$\mu$-a.e.$x$})$
    が成り立つ。
    このように定めた組$(V', T', \mu)$が
    $\calP$の実現であることは一旦認めて最後に示すこととし、
    まず次元と射について確かめる。

    まず
    $(V', T', \mu)$の次元は
    $\dim V' = \dim V - 1 < \dim V$
    より
    $(V, T, \mu)$の次元よりも小さい。
    また、
    射影$\pi \colon V \to V'$をひとつ選べば、
    $(\pi, 0)$は明らかに
    $(V, T, \mu)$から$(V', T', \mu)$への射を与える。

    あとは$(V', T', \mu)$が
    $\calP$の実現であることを示せばよい。
    指数型分布族の定義(\url{0502_資料.pdf})の条件
    (E0), (E1), (E2)は明らかに成立しているから、
    あとは条件(E3)を確認すればよい。
    そこで$p \in \calP$を任意とする。
    いま$(V, T, \mu)$が$\calP$の実現であることから、
    ある$\theta \in V^\vee$が存在して
    \begin{equation}
        \dd[p]{\mu}(x)
            = \frac{
                \exp\langle \theta, T(x) \rangle
            }{
                \int_{\calX} \exp\langle \theta, T(y) \rangle \, \mu(dy)
            }
            \qquad
            \text{$\mu$-a.e.$x$}
    \end{equation}
    が成り立つ。
    $T', v_0$を用いて式変形すると、$\mu$-a.e.$x$に対し
    \begin{alignat}{1}
        \dd[p]{\mu}(x)
            &= \frac{
                \exp\myparen{
                    \langle \theta, T(x) \rangle
                }
            }{
                \int_{\calX} \exp\myparen{
                    \langle \theta, T(x) \rangle
                } \, \mu(dy)
            } \\
            &= \frac{
                \exp\myparen{
                    \langle \theta, T'(x) \rangle
                    + \langle \theta, v_0 \rangle
                }
            }{
                \int_{\calX} \exp\myparen{
                    \langle \theta, T'(x) \rangle
                    + \langle \theta, v_0 \rangle
                } \, \mu(dy)
            } \\
            &= \frac{
                \exp\myparen{
                    \langle \theta, T'(x) \rangle
                }
            }{
                \int_{\calX} \exp\myparen{
                    \langle \theta, T'(x) \rangle
                } \, \mu(dy)
            }
    \end{alignat}
    が成り立つ。
    したがって$(V', T', \mu)$は条件(E3)も満たし、
    $\calP$の実現であることがいえた。
\end{proof}

\begin{proof}[\cref{lemma:morphism-existence-b}の証明]
    $(V, T, \mu)$が条件Bをみたさないとする。
    すると、ある真ベクトル部分空間$W \subsetneq V^\vee$および
    $\theta_0 \in \Theta^\calP$が存在して
    $\aspan\Theta^\calP = W + \theta_0$が成り立つ。
    そこで
    $\wt{V} \coloneqq V / W^\perp$と定め、
    $\pi \colon V \to \wt{V}$を自然な射影として
    $\wt{T} \coloneqq \pi \circ T \colon \calX \to \wt{V}$
    と定める。
    また、$\calX$上の測度$\wt{\mu}$を
    $\wt{\mu} \coloneqq \exp\myangle{\theta_0}{T(x)} \cdot \mu$
    と定める。
    このように定めた組$(\wt{V}, \wt{T}, \wt{\mu})$が
    $\calP$の実現であることは一旦認めて最後に示すこととし、
    まず次元と射について確かめる。

    まず
    $(\wt{V}, \wt{T}, \wt{\mu})$の次元は
    $\dim \wt{V} = \dim V - \dim W^\perp = \dim W < \dim V^\vee = \dim V$
    より
    $(V, T, \mu)$の次元よりも小さい。
    また、
    $(\pi, 0)$は明らかに
    $(V, T, \mu)$から$(\wt{V}, \wt{T}, \wt{\mu})$への射を与える。

    あとは$(\wt{V}, \wt{T}, \wt{\mu})$が
    $\calP$の実現であることを示せばよい。
    指数型分布族の定義(\url{0502_資料.pdf})の条件
    (E0), (E1), (E3)の成立は簡単に確かめられるから、
    ここでは条件(E3)だけ確かめる。
    そこで$p \in \calP$を任意とする。
    $(V, T, \mu)$が$\calP$の実現であることから、
    ある$\theta \in V^\vee$が存在して
    \begin{equation}
        p(dx) =
            \frac{
                \exp\myangle{\theta}{T(x)}
            }{
                \int_\calX \exp\myangle{\theta}{T(x)} \, d\mu(x)
            }
            \mu(dx)
    \end{equation}
    が成り立つ。
    ここで線型写像$\myangle{\theta - \theta_0}{\cdot} \colon V \to \R$は
    $\Ker \myangle{\theta_0}{\cdot} \supset W^\perp$をみたすから、
    図式
    \begin{equation}
        \begin{tikzcd}
            V
                \ar[two heads]{d}
                \ar{r}{\myangle{\theta - \theta_0}{\cdot}}
                & \R
                \\
            \wt{V}
                \ar[dashed]{ru}[swap]{\wt{\theta}}
        \end{tikzcd}
    \end{equation}
    を可換にする線型写像$\wt{\theta} \colon \wt{V} \to \R$
    すなわち
    線型形式$\wt{\theta} \in \wt{V}^\vee$が存在する。
    この$\wt{\theta}$が
    条件(E3)をみたすものであることを確かめればよいが、
    各$x \in \calX$に対し
    \begin{alignat}{1}
        \myangle{\wt{\theta}}{\wt{T}(x)}
            &=
                \myangle{\theta - \theta_0}{T(x)}
                \\
            &=
                \myangle{\theta}{T(x)}
                -
                \myangle{\theta_0}{T(x)}
    \end{alignat}
    が成り立つから
    \begin{alignat}{1}
        p(dx)
            &=
                \frac{
                    \exp\myangle{\theta}{T(x)}
                }{
                    \int_\calX
                    \exp\myangle{\theta}{T(x)}
                    \, \mu(dx)
                }
                \mu(dx)
                \\
            &=
                \frac{
                    \exp\myangle{\wt{\theta}}{\wt{T}(x)}
                    \exp\myangle{\theta_0}{T(x)}
                }{
                    \int_\calX 
                    \exp\myangle{\wt{\theta}}{\wt{T}(x)}
                    \exp\myangle{\theta_0}{T(x)}
                    \, \mu(dx)
                }
                \mu(dx)
                \\
            &=
                \frac{
                    \exp\myangle{\wt{\theta}}{\wt{T}(x)}
                }{
                    \int_\calX
                    \exp\myangle{\wt{\theta}}{\wt{T}(x)}
                    \, \wt{\mu}(dx)
                }
                \wt{\mu}(dx)
    \end{alignat}
    となる。
    したがって条件(E3)の成立が確かめられた。
    以上より$(\wt{V}, \wt{T}, \wt{\mu})$は$\calP$の実現である。
    これで証明が完了した。
\end{proof}


\end{document}