\documentclass[report]{jlreq}
\usepackage{global}
\usepackage{./local}
\subfiletrue
\def\assetspath{../}
%\makeindex
\chead{2023/06/06}
\begin{document}

% ============================================================
%
% ============================================================

% ------------------------------------------------------------
%
% ------------------------------------------------------------
\section*{振り返りと導入}

前回は自然パラメータ空間に Fisher 計量を定義した。
本稿では次のことを行う:
\begin{itemize}
    \item 最小次元実現の間のアファイン変換の一意存在を述べる。
    \item 指数型分布族$\calP$自体に構造を入れる。
\end{itemize}

$\calX$を可測空間、
$\calP \subset \calP(\calX)$を$\calX$上の指数型分布族とする。
新たな用語として次の2つを導入しておく。

\begin{definition}[自然パラメータ付け]
    \begin{equation}
        P_{(V, T, \mu)} \colon V^\vee \to \calP(\calX),
            \quad
            \theta
            \mapsto
            \exp (\myangle{\theta}{T(x)} - \psi(\theta)) \, \mu(dx)
    \end{equation}
    を$(V, T, \mu)$による$\calP(\calX)$の
    \term{自然パラメータ付け}[natural parametrization]
        {自然パラメータ付け}[しぜんパラメータづけ]
    という。
    $P_{(V, T, \mu)}$を$P$と略記することがある。
\end{definition}

\begin{definition}[真パラメータ空間]
    \begin{equation}
        \Theta^\calP_{(V, T, \mu)}
            \coloneqq P^{-1}(\calP)
    \end{equation}
    を$\calP$の$(V, T, \mu)$に関する
    \term{真パラメータ空間}[strict parameter space]
        {真パラメータ空間}[しんパラメータくうかん]
    という。
\end{definition}


% ------------------------------------------------------------
%
% ------------------------------------------------------------
\section{最小次元実現の間のアファイン変換}

本節の目標は、
最小次元実現の間のアファイン変換の一意存在を述べた
\cref{thm:transformation-between-representations}
の証明である。
本節では、
ステートメントを簡潔にするために圏の言葉を用いる。

\begin{propdef}[$\calP$の実現とアファイン写像の圏]
    次のデータにより圏が定まる:
    \begin{itemize}
        \item 対象: $\calP$の実現$(V, T, \mu)$全体
        \item 射: $(V, T, \mu)$から$(V', T', \mu')$への射は、
            $V$から$V'$への全射アファイン写像
            $(L, b), \; L \in \Lin(V, V'), \; b \in V'$
            であって
            $T'(x) = L(T(x)) + b \; \text{$\mu$-a.e.$x$}$をみたすもの
        \item 合成: アファイン写像の合成
            $(L, b)(K, c) = (LK, Lc + b)$
    \end{itemize}
    この圏を\termsilent{$\calP$の実現の圏}と呼び、
    $\bfC_\calP$と書く。
\end{propdef}

\begin{proof}
    示すべきことは、射の合成が射であること、恒等射の存在、結合律の3点である。
    射の合成が射であることは、
    全射と全射の合成が全射であることと、
    $\mu \ll \mu'$かつ$\mu' \ll \mu$
    が成り立つことより従う。

    また、$(V, T, \mu)$の恒等射は明らかに恒等写像$(\id_V, 0)$であり、
    結合律はアファイン写像の合成の結合律より従う。
\end{proof}

最小次元実現を特徴づける2つの条件を導入する。

\begin{definition}[Affine span 条件]
    $\calP$の実現$(V, T, \mu)$に関する条件
    \begin{enumerate}
        \item $\Theta^\calP_{(V, T, \mu)}$は
            $V^\vee$を affine span する。
    \end{enumerate}
    が成り立つとき、
    $(V, T, \mu)$は\termsilent{affine span 条件}をみたすという。
\end{definition}

\begin{propdef}[単射性条件]
    $\calP$の実現$(V, T, \mu)$に関する次の条件は同値である:
    \begin{enumerate}
        \item $P \colon V^\vee \to \calP(\calX)$は単射である。
        \item $\forall \theta \in V^\vee$
            に対し
            「
                $\myangle{\theta}{T(x)} = \text{const. $\mu$-a.e.$x$}
                \implies
                \theta = 0$
            」
            が成り立つ。
        \item $V$の任意の真アファイン部分空間$W$に対し、
            「$T(x) \in W \; \text{$\mu$-a.e.$x$}$でない」
            が成り立つ。
    \end{enumerate}
    これらの条件が成り立つとき、
    $(V, T, \mu)$は\termsilent{単射性条件}をみたすという。
\end{propdef}

\begin{proof}
    (2) $\iff$ (3) は\url{0502_資料.pdf}の命題2.2で示した。
    (1) $\iff$ (2) は\url{0523_板書.pdf}に書き留めてある。
\end{proof}

\begin{remark}[条件をみたさない例]
    単射性条件をみたすが affine span 条件をみたさない例として
    \url{0425_資料.pdf}の例4.4がある。
    Affine span 条件をみたさないことは
    後の\cref{thm:characterization-of-minimal-representation}から従う。

    \TODO{affine span 条件をみたすが単射性条件をみたさない例?}
\end{remark}

\begin{lemma}[基底測度だけを変化させた対象からの射]
    $(V, T, \mu')$から$(V, T, \mu)$への射が存在する。
\end{lemma}

\begin{proof}
    $(\id, 0)$が求める射である。
\end{proof}

\begin{lemma}[基底測度以外を変化させた対象からの射]
    $(V, T, \mu)$が affine span 条件と単射性条件をみたすならば、
    任意の対象$(V', T', \mu)$から$(V, T, \mu)$への射が存在する。
\end{lemma}

\begin{proof}
    $P, P'$の右逆写像
    $\theta \colon \calP \to \Theta^\calP$
    および
    $\theta' \colon \calP \to \Theta'^\calP$
    をひとつずつ選んでおく。

    \uline{Step 1: 射$(L, b)$の構成} \quad
    まず$(V, T, \mu)$の affine span 条件より、
    $V^\vee$のあるアファイン基底
    $a^i \in \Theta^\calP \; (i = 0, \dots, m)$
    が存在して、
    $e^i \coloneqq a^i - a^0 \; (i = 1, \dots, m)$は
    $V^\vee$の基底となる。
    そこで$p^i \coloneqq P(a^i) \in \calP \; (i = 0, \dots, m)$とおき、
    $(L, b)$を次のように定める:
    \begin{alignat}{1}
        &L \colon V' \to V,
            \quad
            t' \mapsto
                \myangle{\theta'(p^i) - \theta'(p^0)}{t'} e_i
            \\
        &b \coloneqq
            \mybrace{
                \psi(\theta(p^i)) - \psi(\theta(p^i))
                - \psi(\theta'(p^0)) + \psi(\theta'(p^0))
            } e_i
            \in V
    \end{alignat}
    示すべきことは、
    すべての$p \in \calP$に対し
    \begin{equation}
        T(x) = L(T'(x)) + b
            \quad
            \text{$\mu$-a.e.$x$}
    \end{equation}
    が成り立つことと、
    $(L, b)$が全射となることである。

    \uline{Step 2: $T(x) = L(T'(x)) + b$の証明} \quad
    指数型分布族の定義より、
    任意の$p \in \calP$に対し、
    ある$\mu$-零集合$N_p \subset \calX$が存在して
    \begin{alignat}{2}
        &\phantom{\therefore}& \quad
            \exp(\myangle{\theta(p)}{T(x)} - \psi(\theta(p)))
                =
                    \dd[p]{\mu}(x)
                &=
                    \exp(\myangle{\theta'(p)}{T'(x)} - \psi'(\theta'(p)))
                \qquad
                    (x \in \calX \setminus N_p)
            \\
        &\therefore& \quad
            \myangle{\theta(p)}{T(x)}
                - \myangle{\theta'(p)}{T'(x)}
                &=
                    \psi(\theta(p))
                    - \psi'(\theta'(p))
                \qquad
                    (x \in \calX \setminus N_p)
    \end{alignat}
    が成り立つ。
    とくに各$i = 0, \dots, m$に対し
    \begin{equation}
        \myangle{\theta(p^i)}{T(x)}
            - \myangle{\theta'(p^i)}{T'(x)}
            =
                \psi(\theta(p^i))
                - \psi'(\theta'(p^i))
            \qquad
                (x \in \calX \setminus N_{p^i})
    \end{equation}
    が成り立つから、
    \begin{equation}
        \begin{alignedat}{1}
            &\phantom{=}
                \myangle{\theta(p^i) - \theta(p^0)}{T(x)}
                - \myangle{\theta'(p^i) - \theta'(p^0)}{T'(x)}
                \\
            &=
                \psi(\theta(p^i))
                - \psi'(\theta'(p^i))
                - \psi(\theta(p^0))
                + \psi'(\theta'(p^0))
                \qquad
                (x \in \calX \setminus (N_{p^0} \cup N_{p^i}))
        \end{alignedat}
    \end{equation}
    となる。
    ここで$(V, T, \mu)$の単射性条件より
    $\theta(p^i) = \theta(P(a^i)) = a^i$
    が成り立つから、
    上の式より
    \begin{equation}
        \begin{alignedat}{1}
            \myangle{e^i}{T(x)}
                &=
                    \myangle{\theta'(p^i) - \theta'(p^0)}{T'(x)}
                    \\
                &\qquad
                    + \psi(\theta(p^i))
                    - \psi'(\theta'(p^i))
                    - \psi(\theta(p^0))
                    + \psi'(\theta'(p^0))
                    \qquad
                    (x \in \calX \setminus (N_{p^0} \cup N_{p^i}))
        \end{alignedat}
    \end{equation}
    したがって
    \begin{equation}
        T(x) = L(T'(x)) + b
            \qquad
                (x \in \calX \setminus (N_{p^0} \cup \cdots \cup N_{p^m}))
    \end{equation}
    が成り立つ。
    これで Step 2 が完了した。

    \uline{Step 3: $(L, b)$が全射であることの証明} \quad
    $L$が全射であることをいえばよい。
    もし$L$が全射でなかったとすると、
    $T(x) = L(T'(x)) + b \in \Im L + b$
    が$p$-a.e.$x$すなわち$\mu$-a.e.$x$に対し成り立つことになるが、
    $\Im L + b$は$V$の真アファイン部分空間だから
    $(V, T, \mu)$の単射性条件に反する。
    したがって$L$は全射である。
\end{proof}

\begin{lemma}[2条件をみたす対象への射]
    \label[lemma]{lemma:affine-map-between-representations}
    $(V, T, \mu)$が affine span 条件と単射性条件をみたすならば、
    任意の対象$(V', T', \mu')$から$(V, T, \mu)$への射が存在する。
\end{lemma}

\begin{proof}
    上の2つの補題より存在する2つの射
    $(V', T', \mu') \to (V', T', \mu) \to (V, T, \mu)$
    を合成すればよい。
\end{proof}

\begin{theorem}[最小次元実現の特徴づけ]
    \label[theorem]{thm:characterization-of-minimal-representation}
    $\calP$の実現$(V, T, \mu)$に関する次の条件は同値である:
    \begin{enumerate}
        \item $(V, T, \mu)$は$\calP$の最小次元実現である。
        \item $(V, T, \mu)$は単射性条件と affine span 条件をみたす。
    \end{enumerate}
\end{theorem}

\begin{proof}
    \uline{(1) \Rightarrow (2)} \quad
    以前示した。\TODO{affine span 条件は示してない?}

    \uline{(2) \Rightarrow (1)} \quad
    $(V, T, \mu)$が単射性条件と affine span 条件をみたすとする。
    $\calP$の任意の実現
    $(V', T', \mu')$に対し、
    \cref{lemma:affine-map-between-representations}より
    全射線型写像$L: V' \to V$が存在するから、
    $\dim V \le \dim V'$である。
    したがって$V$は$\calP$の最小次元実現である。
\end{proof}

\begin{lemma}[自然パラメータの変換]
    $(V, T, \mu)$が
    affine span 条件と単射性条件をみたし、
    $(V', T', \mu')$が
    単射性条件をみたすならば、
    射$(L, b) \colon (V', T', \mu') \to (V, T, \mu)$がただひとつ存在して
    \begin{equation}
        \theta'(p)
            = \up{t}L(\theta(p))
            \qquad
            (\forall p \in \calP)
    \end{equation}
    が成り立つ。
    ただし
    写像$\theta \colon \calP \to \Theta^\calP_{(V, T, \mu)}$
    および
    $\theta' \colon \calP \to \Theta^\calP_{(V', T', \mu')}$は
    $P, P'$の逆写像である。
\end{lemma}

\begin{proof}
    上の証明の続きで、
    すべての$x \in \calX \setminus N$に対し
    \begin{alignat}{2}
        &\phantom{\therefore}&
            \myangle{\theta(p)}{L(T'(x)) + b}
            - \myangle{\theta'(p)}{T'(x)}
            &=
                \psi(\theta(p)) - \psi'(\theta'(p))
                \\
        &\therefore \quad&
            \myanglem{\up{t}L(\theta(p))}{T'(x)}
            - \myangle{\theta'(p)}{T'(x)}
            &=
                \psi(\theta(p)) - \psi'(\theta'(p))
                - \myangle{\theta(p)}{b}
                \\
        &\therefore \quad&
            \myanglem{\up{t}L(\theta(p)) - \theta'(p)}{T'(x)}
            &=
                \psi(\theta(p)) - \psi'(\theta'(p))
                - \myangle{\theta(p)}{b}
    \end{alignat}
    が成り立つ。
    $(V', T', \mu)$の単射性条件より
    $\up{t}L(\theta(p)) = \theta'(p)$
    が成り立つ。

    $(V, T, \mu)$の
    affine span 条件より、
    $\up{t}L$による$V^\vee$の基底の写り方が決まるから、
    $\up{t}L$したがって$L$は$(V, T, \mu), (V', T', \mu')$に対し
    一意に決まる。
\end{proof}

\begin{theorem}[最小次元実現の間のアファイン変換]
    \label[theorem]{thm:transformation-between-representations}
    $(V, T, \mu), (V', T', \mu')$が
    ともに最小次元実現ならば、
    同型射$(V, T, \mu) \to (V', T', \mu')$がただひとつ存在する。
\end{theorem}

\begin{proof}
    上の系から存在する射
    $(L, b) \colon (V', T', \mu') \to (V, T, \mu)$
    および
    $(K, c) \colon (V, T, \mu) \to (V', T', \mu')$
    について、
    $(K, c)$が$(L, b)$の逆射であることを示せばよい。
    いま$V^\vee$の基底$e^i \; (i = 1, \dots, m)$に対し
    $\up{t}L \up{t}K e'^i = e'^i$が成り立つから、
    $\up{t}L$は全射かつ単射なので線型同型写像である。
    したがって$L$も線型同型写像であり、
    圏$\bfC_\calP$の射の定義から
    $(L, b)$は$(K, c)$を逆射とする同型射である。
\end{proof}

\begin{remark}[正規分布族の最小次元実現]
    \url{0425_資料.pdf}の例3.2でみた正規分布族の例は
    最小次元実現である。
    単射性条件は、$(\theta_1, \theta_2)$が異なれば
    平均か分散の少なくとも一方が異なることになり、
    異なる確率分布を与えることからわかる。
    Affine span 条件も明らか。
\end{remark}

% ------------------------------------------------------------
%
% ------------------------------------------------------------
\section{指数型分布族の構造}

$\calP$には自然な多様体構造が定まることをみる。

\begin{propdef}
    指数型分布族$\calP$に関し、次は同値である:
    \begin{enumerate}
        \item ある最小次元実現$(V, T, \mu)$に対し、
            $\Theta^\calP_{(V, T, \mu)}$は$V^\vee$で開である。
        \item すべての最小次元実現$(V, T, \mu)$に対し、
            $\Theta^\calP_{(V, T, \mu)}$は$V^\vee$で開である。
    \end{enumerate}
    $\calP$がこれらの同値な2条件をみたすとき、
    $\calP$は\termsilent{開}[open]であるという。
\end{propdef}

\begin{proof}
    (1) $\Rightarrow$ (2)は
    \cref{thm:transformation-between-representations}より従う。
    (2) $\Rightarrow$ (1)は
    最小次元実現が存在することから従う。
\end{proof}

\begin{propdef}
    $\calP$は開であるとする。
    $\calP$の最小次元実現$(V, T, \mu)$をひとつ選ぶと、
    自然パラメータ付け$P$により、
    $\calP$上に多様体構造と平坦アファイン接続を定めることができる。
    この多様体構造および平坦アファイン接続は
    最小次元実現のとり方によらない。
    これを$\calP$の
    \termsilent{自然な多様体構造}
    および
    \termsilent{自然な平坦アファイン接続}
    と呼ぶ。
\end{propdef}

\begin{proof}
    下の図式の可換性と、
    $P^{-1}, P'^{-1}$が diffeo. であること、
    $\up{t}L$が線型同型であることから従う。
    \begin{equation}
        \begin{tikzcd}
            (\calP, \calU_{(V, T, \mu)})
                \ar{d}[swap]{P^{-1}}
                \ar{r}{\id}
                &(\calP, \calU_{(V', T', \mu')})
                    \ar{d}{P'^{-1}}
                \\
            \Theta^\calP
                \ar{r}{\up{t}L}
                &\Theta'^\calP
        \end{tikzcd}
    \end{equation}
\end{proof}

\begin{propdef}[$\calP$の Fisher 計量]
    $\calP$は開であるとする。
    $\calP$の最小次元実現$(V, T, \mu)$をひとつ選ぶと、
    $\Theta^\calP_{(V, T, \mu)}$上の Fisher 計量$g$を
    $\theta$で引き戻して
    $\calP$上の Riemann 計量
    $\theta^* g$が定まる。
    この計量は最小次元実現のとり方によらない。
    これを$\calP$上の
    \termsilent{Fisher 計量}
    と呼ぶ。
\end{propdef}

\begin{proof}
    期待値と分散のペアリングの命題と同様の議論により、
    各$p \in \calP$に対し
    $g_{\theta(p)} = (L \otimes L) g'_{\theta'(p)}$
    が成り立つ。

    示すべきことは
    $\theta^* g = \theta'^* g'$が成り立つことである。
    各$p \in \calP, \; u, v \in T_p\calP$に対し
    \begin{alignat}{1}
        (\theta^* g)_p(u, v)
            &=
                g_{\theta(p)} (d\theta_p(u), d\theta_p(v))
                \\
            &=
                \myangle{
                    g_{\theta(p)}
                }{
                    d\theta_p(u) \otimes d\theta_p(v)
                }
                \\
            &=
                \myangle{
                    (L \otimes L) g'_{\theta'(p)}
                }{
                    d\theta_p(u) \otimes d\theta_p(v)
                }
                \\
            &=
                \myangle{
                    g'_{\theta'(p)}
                }{
                    \up{t}L \circ d\theta_p(u) \otimes \up{t}L \circ d\theta_p(v)
                }
                \\
            &=
                \myangle{
                    g'_{\theta'(p)}
                }{
                    d(\up{t}L \circ \theta)_p (u) \otimes d(\up{t}L \circ \theta)_p (v)
                }
                \\
            &=
                \myangle{
                    g'_{\theta'(p)}
                }{
                    d\theta'_p (u) \otimes d\theta'_p (v)
                }
                \\
            &=
                g'_p (d\theta'_p(u), d\theta'_p(v))
                \\
            &=
                (\theta'^* g')_p(u, v)
    \end{alignat}
    が成り立つ。
\end{proof}


% ------------------------------------------------------------
%
% ------------------------------------------------------------
\section*{今後の予定}

\begin{itemize}
    \item Amari-Chentsov テンソルを定義する。
    \item 正規分布族の場合の具体的な計算を行う
        (Fisher 計量、Levi-Civita 接続、測地線など)。
\end{itemize}

% ------------------------------------------------------------
%
% ------------------------------------------------------------
\section*{参考文献}

\nocite{amari_information_2016}
\nocite{bn1970_pdf}

{
    \renewcommand{\bibsection}{}
    \bibliographystyle{amsalpha}
    \bibliography{./bibliography,../../mybibliography}
}


\end{document}