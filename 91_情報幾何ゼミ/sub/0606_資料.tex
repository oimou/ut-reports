\documentclass[report]{jlreq}
\usepackage{global}
\usepackage{./local}
\subfiletrue
\def\assetspath{../}
%\makeindex
\chead{2023/05/23}
\begin{document}

% ============================================================
%
% ============================================================

% ------------------------------------------------------------
%
% ------------------------------------------------------------
\section{振り返りと導入}

\begin{itemize}
    \item 期待値パラメータ空間
\end{itemize}

% ------------------------------------------------------------
%
% ------------------------------------------------------------
\section{Fisher 計量}

\begin{example}[正規分布族]
    \TODO{ちゃんと書く}
    $\calP$を$\calX = \R$上の正規分布族とし、
    実現$(V, T, \mu)$を
    $V = \R^2, \;
        T(x) = (x, x^2), \;
        \mu = \lambda$
    とおく。
    これは条件Aをみたす。

    自然パラメータ空間は
    $\Theta = \Theta^\circ = \R \times \R_{< 0}$である。

    対数分配関数は
    \begin{equation}
        \psi(\theta)
            = \frac{\mu^2}{2 \sigma^2}
            + \log \sigma
            + \frac{1}{2} \log 2\pi
    \end{equation}
    である。
    ただし$\theta^1 = \frac{\mu}{\sigma^2}, \;
        \theta^2 = -\frac{1}{2 \sigma^2}$
    とおいた。
    よって
    \begin{alignat}{1}
        d\psi
            &=
                \frac{\mu}{\sigma^2}
                d\mu
                + \frac{\sigma^2 - \mu^2}{\sigma^3}
                d\sigma
                \\
            &=
                -\frac{\theta^1}{2\theta^2} d\theta^1
                +\myparen{
                    -\frac{1}{2\theta^2}
                    + \frac{(\theta^1)^2}{4(\theta^2)^2}
                }
                d\theta^2
                \\
        \Hess\psi
            &= Dd\psi \\
            &=
                \myparen{
                    -\frac{1}{2\theta^2}
                    d\theta^1
                    + \frac{\theta^1}{2(\theta^2)^2}
                    d\theta^2
                }
                d\theta^1
                +
                \myparen{
                    \frac{\theta^1}{2(\theta^2)^2}
                    d\theta^1
                    + \myparen{
                        \frac{1}{2(\theta^2)^2}
                        - \frac{(\theta^1)^2}{2(\theta^2)^3}
                    }
                    d\theta^2
                }
                d\theta^2
                \\
            &=
                \frac{1}{\sigma^2} (d\mu)^2
                + \frac{2}{\sigma^2} (d\sigma)^2
    \end{alignat}
    である。
    Fisher 計量$g \coloneqq \Hess\psi$から定まる
    Levi-Civita 接続$\nabla^{(g)}$の、
    座標$\mu, \sigma$に関する接続係数を求めてみる。
    \begin{alignat}{2}
        {\Gamma^{(g)}}_{11}^1
            = 0,
            &\qquad
                {\Gamma^{(g)}}_{12}^1
                    = {\Gamma^{(g)}}_{21}^1
                    = -\frac{1}{\sigma},
            &&\qquad
                {\Gamma^{(g)}}_{22}^1
                    = 0,
            \\
        {\Gamma^{(g)}}_{11}^2
            = \frac{1}{2\sigma},
            &\qquad
                {\Gamma^{(g)}}_{12}^2
                    = {\Gamma^{(g)}}_{21}^2
                    = 0,
            &&\qquad
                {\Gamma^{(g)}}_{22}^2
                    = -\frac{1}{\sigma}
    \end{alignat}
    測地線の方程式は
    \begin{equation}
        \begin{cases}
            x'' - \frac{2}{y} x' y' = 0 \\
            y'' + \frac{1}{2y} (x')^2 - \frac{1}{y} (y')^2 = 0
        \end{cases}
    \end{equation}
    である。
    これを直接解くのは少し大変である。
    その代わりに、
    既知の Riemann 多様体との間の等長同型を利用して測地線を求める。
    $(\Theta, g)$は、
    上半平面$H$に計量
    $\breve{g} = \frac{(dx)^2 + (dy)^2}{2y^2}$
    を入れた Riemann 多様体との間に
    等長同型$(\Theta, g) \to (H, \breve{g}), \;
        (x, y) \mapsto (x, \sqrt{2}y)$
    を持つ。
    Levi-Civita 接続に関する測地線は
    等長同型で保たれるから、
    $(H, \breve{g})$の測地線を求めればよい。
    $(H, \breve{g})$の測地線は、
    $y$軸に平行な直線と
    $x$軸上に中心を持つ半円で尽くされることが知られている。
    これらを等長同型で写して、
    $(\Theta, g)$の測地線として
    $y$軸に平行な直線と
    $x$軸上に長軸を持つ半楕円が得られる。
\end{example}

% ------------------------------------------------------------
%
% ------------------------------------------------------------
\section{期待値パラメータ空間}

指数型分布族の話題に戻る。
以降、本節では$\calX$を可測空間、
$\calP \subset \calP(\calX)$を$\calX$上の指数型分布族、
$(V, T, \nu)$を$\calP$の実現、
$\Theta \coloneqq \Theta_{(V, T, \nu)}$を
$(V, T, \nu)$の自然パラメータ空間とする。

\begin{definition}[期待値パラメータ空間]
    集合$\calM_{(V, T, \nu)}$
    \begin{equation}
        \calM_{(V, T, \nu)}
            \coloneqq \mybrace{
                \mu \in V
                \mid
                \exists \;
                p \colon \text{$\calX$上の確率分布}
                \; \text{s.t.} \;
                p \ll \nu, \;
                E_p[T] = \mu
            }
    \end{equation}
    を$(V, T, \nu)$の
    \term{期待値パラメータ空間}[mean parameter space]
        {期待値パラメータ空間}[きたいちぱらめーたくうかん]
    という。
\end{definition}

期待値パラメータ空間$\calM$は、
$\calP$に属する確率分布に関する$T$の期待値をすべて含んでいる
(一般には真に含んでいる)。

\begin{proposition}
    $\mu \in V$が
    ある$p \in \calP$に関する
    $T$の期待値ならば (すなわち$\mu = E_p[T]$ならば)、
    $\mu$は$\calM_{(V, T, \nu)}$に属する。
\end{proposition}

\begin{proof}
    \TODO{}
\end{proof}

\begin{proposition}[$\calM$は凸集合]
    $\calM_{(V, T, \nu)}$は$V$の凸集合である。
\end{proposition}

\begin{proof}
    \TODO{}
\end{proof}


% ------------------------------------------------------------
%
% ------------------------------------------------------------
\section{今後の予定}

\begin{itemize}
    \item KL ダイバージェンス
    \item Fisher 計量
    \item アファイン接続
\end{itemize}

% ------------------------------------------------------------
%
% ------------------------------------------------------------
\section{参考文献}

\nocite{amari_information_2016}

{
    \renewcommand{\bibsection}{}
    \bibliographystyle{amsalpha}
    \bibliography{./bibliography,../../mybibliography}
}


\end{document}