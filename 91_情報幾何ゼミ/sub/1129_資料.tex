\documentclass[report,dvipdfmx]{jlreq}
\usepackage{global}
\usepackage{./local}
\subfiletrue
\def\assetspath{../}
%\makeindex
\chead{2023/11/29}
\begin{document}

% ============================================================
%
% ============================================================

% ------------------------------------------------------------
%
% ------------------------------------------------------------
\section*{振り返りと導入}

前回は双対平坦構造に付随するシンプレクティック構造と Legendre 変換について調べた。
本稿では次のことを行う:
\begin{enumerate}
    \item Hamilton ベクトル場と Poisson 括弧を定義する。
    \item 運動量写像を定義し Noether の定理を示す。
\end{enumerate}

% ------------------------------------------------------------
%
% ------------------------------------------------------------
\section{Hamilton ベクトル場と Poisson 括弧}

以下$(M, \omega)$を (一般の) シンプレクティック多様体とする。

\begin{definition}[Hamilton ベクトル場]
    $f \in \smooth(M)$に対し、
    $X_f \in \mathfrak{X}(M)$であって
    \begin{equation}
        \iota_{X_f} \omega = -df
    \end{equation}
    を満たすもの (これはただひとつ存在する) を
    $f$の
    \term{Hamilton ベクトル場}[Hamiltonian vector field]
        {Hamilton ベクトル場}[Hamilton べくとるば]
    といい、
    $f$を$X_f$の
    \term{Hamilton 関数}[Hamiltonian function]
        {Hamilton 関数}[Hamilton かんすう]
    あるいは
    \term{Hamiltonian}
        {Hamiltonian}[Hamiltonian]
    という。
    $X_f$のフローを$f$の
    \term{Hamilton フロー}
        {Hamilton フロー}[Hamilton ふろー]
    という。
\end{definition}

\begin{proposition}[Hamilton ベクトル場の基本性質]
    \label[proposition]{prop:Hamilton-vector-field-basic-properties}
    $f \in \smooth(M), \; Y \in \mathfrak{X}(M)$として次が成り立つ:
    \begin{enumerate}
        \item $\omega(X_f, Y) = - Yf$
        \item $\omega$は Hamilton フローに沿って不変、
            すなわち$L_{X_f} \omega = 0$である。
        \item $\omega = dx^i \wedge d\xi_i$なる$M$の局所座標
            (これは\termsilent{Darboux 座標}と呼ばれる)
            $(x^i, \xi_i)$に関する$X_f$の成分表示は
            \begin{equation}
                X_f
                    =
                        - \deldel[f]{\xi_i} \deldel{x^i}
                        + \deldel[f]{x^i} \deldel{\xi_i}
            \end{equation}
            である。
    \end{enumerate}
\end{proposition}

\begin{proof}
    cf. 資料末尾の付録
\end{proof}

\begin{definition}[Poisson 括弧]
    演算$\{ {-}, {-} \} \colon \smooth(M) \times \smooth(M) \to \smooth(M)$を
    \begin{equation}
        \{f, g\}
            \coloneqq
                \omega(X_f, X_g)
                \quad
                (f, g \in \smooth(M))
    \end{equation}
    で定め、
    これを
    \term{Poisson 括弧}[Poisson bracket]
        {Poisson かっこ}[Poisson かっこ]
    と呼ぶ。
\end{definition}

\begin{proposition}[Poisson 括弧の基本性質]
    \label[proposition]{prop:Poisson-bracket-basic-properties}
    $f, g, h \in \smooth(M), \; Y \in \mathfrak{X}(M)$として次が成り立つ:
    \begin{enumerate}
        \item $\{ {-}, {-} \}$は交代$\R$-双線型写像であり、
            $\{ f, g \} = X_f g = - X_g f$をみたす。
        \item $[X_f, X_g] = X_{\{f, g\}}$
        \item (Jacobi の恒等式)
            $\{f, \{g, h\}\}
                + \{g, \{h, f\}\}
                + \{h, \{f, g\}\}
                = 0$
    \end{enumerate}
    したがって$(\smooth(M), \{ {-}, {-} \})$は Lie 代数であり、
    写像$\smooth(M) \to \mathfrak{X}(M), \; f \mapsto X_f$は
    Lie 代数準同型である。
    \begin{enumerate}
        \setcounter{enumi}{3}
        \item (Leibniz 則) $\{ f, gh \}
                    =
                        \{ f, g \} h
                        + g \{ f, h \}$
    \end{enumerate}
\end{proposition}

\begin{proof}
    cf. 資料末尾の付録
\end{proof}


\begin{definition}[Hamilton 系]
    $H \in \smooth(M)$に対し、
    組$(M, \omega, H)$を
    \term{Hamilton 系}[Hamiltonian system]
        {Hamilton 系}[Hamilton けい]
    といい、
    $H$を$(M, \omega, H)$の
    \term{Hamilton 関数}
        {Hamilton 関数}[Hamilton かんすう]
    あるいは
    \term{Hamiltonian}
        {Hamiltonian}[Hamiltonian]
    という。
    $f \in \smooth(M)$が
    $L_{X_H} f = 0$をみたすとき、
    $f$を$(M, \omega, H)$の
    \term{保存量}[conserved quantity]
        {保存量}[ほぞんりょう]
    という。
\end{definition}

\begin{proposition}[保存量と Poisson 括弧]
    $f \in \smooth(M)$が$(M, \omega, H)$の保存量であるための必要十分条件は
    $\{ H, f \} = 0$が成り立つことである。
    とくに$H$自身は保存量である。
\end{proposition}

\begin{proof}
    必要十分性は$\{ H, f \} = X_H f$であることより従う。
    また$\{ {-}, {-} \}$の交代性より
    $\{ H, H \} = 0$だから
    $H$は保存量である。
\end{proof}

\begin{example}[canonical ダイバージェンスの Hamilton フロー]
    $M$を多様体、$(g, \nabla, \nabla^*)$を双対平坦構造、
    $D \colon \calU \to \R$を canonical ダイバージェンス、
    $\omega$を$(g, \nabla, \nabla^*)$に付随する$\calU$上のシンプレクティック構造とする。
    $M$の$g$-凸な双対アファインチャート$(U, \theta, \eta)$をひとつ選ぶと、
    $(\theta^*, \eta)$は$\calU$の局所的な Darboux 座標となり、
    この座標に関する$X_D$の成分表示は
    \begin{equation}
        X_D
            =
                (\theta^{*i} - \theta^i) \deldel{\theta^{*i}}
                +
                (\eta^*_i - \eta_i) \deldel{\eta_i}
    \end{equation}
    となる。
\end{example}

% ------------------------------------------------------------
%
% ------------------------------------------------------------
\section{運動量写像と Noether の定理}

以下$(M, \omega)$を (一般の) シンプレクティック多様体、
$G$を Lie 群とし、
$G$が$M$に右から{\smooth}作用しているとする。

\begin{definition}[シンプレクティック作用]
    $G$の作用
    \term{シンプレクティック作用}[symplectic action]
        {シンプレクティックさよう}[シンプレクティックさよう]
    であるとは、
    各$g \in G$に対し右移動$R_g \colon M \to M$が
    シンプレクティック同相写像であることをいう。
\end{definition}

\begin{definition}[運動量写像]
    $G$のシンプレクティック作用が
    \term{Hamilton 作用}[Hamiltonian action]
        {Hamilton 作用}[Hamilton さよう]
    であるとは、
    次の2条件を満たす
    {\smooth}写像$\mu \colon M \to \frakg^\vee$
    が存在することをいう:
    \begin{description}
        \item[(M1)]
            $X \in \frakg$に対し
            $\mu_X \coloneqq \myangle{\mu}{X} \in \smooth(M)$
            は$X$の基本ベクトル場$X^*$の Hamiltonian である。
        \item[(M2)]
            $X, Y \in \frakg$に対し
            $\{\mu_X, \mu_Y\} = \mu_{[X, Y]}$
            が成り立つ。
    \end{description}
    この写像$\mu \colon M \to \frakg^\vee$を
    \term{運動量写像}[momentum map]
        {運動量写像}[うんどうりょうしゃぞう]
    という。
\end{definition}

\begin{remark}
    ちなみに「モーメント (moment)」と「運動量 (momentum)」は
    物理的に全く別の概念である。
\end{remark}

\begin{example}[力学における運動量]
    $M \coloneqq T^\vee \R^3, \;
        \omega = dx^i \wedge d\xi_i$
    を$T^\vee \R^3$に標準的なシンプレクティック構造を入れたものとする。
    $G \coloneqq \R^3$の$M$への作用を
    $G \times M \to M, \;
        (a, (x, \xi)) \mapsto (x + a, \xi)$
    で定めると、これはシンプレクティック作用である。
    運動量写像$\mu \colon M \to \frakg^\vee$は
    次のように構成できる。
    すなわち、
    $\frakg = \R^3$の標準基底を$e_1, e_2, e_3$、
    その双対基底を$e^1, e^2, e^3$とおき
    \begin{equation}
        \mu(x, \xi)
            =
                - \xi_i e^i
    \end{equation}
    と定める。
    (M1)は
    $X = X^i e_i \in \frakg$に対し
    $- d\mu_X
        = d (X^i \xi_i)
        = X^i d\xi_i
        = \iota_{X^i \deldel{x^i}} \omega
        = \iota_{X^*} \omega$
    より従う。
    (M2)は
    $\{ \mu_X, \mu_Y \}
        = X^* \mu_Y
        = X^i \deldel{x^i} (- Y^j \xi_j)
        = 0
        = \mu_0
        = \mu_{[X, Y]}$
    より従う。
    別証明として後の
    \cref{prop:Hamilton-action-induced-on-cotangent-bundle}
    を用いることもできる。
    $\mu$は力学における (線型) 運動量の符号を反転したものである。
\end{example}

%\begin{proposition}[運動量写像の基本性質]
%    写像$\frakg \mapsto \smooth(M), \; X \mapsto \mu_X$は Lie 代数準同型である。
%\end{proposition}
%
%\begin{proof}
%    \TODO{orbit map の方がわかりやすい?}
%    $\mu_X$の定義より線型性は明らか。
%    あとは$X, Y \in \frakg$に対し
%    $\{ \mu_X, \mu_Y \} = \mu_{[X, Y]}$を示せばよい。
%    (M1)より
%    \begin{equation}
%        \myangle{\mu(x \exp tX)}{Y}
%            =
%                \myangle{\ltrans{\Ad}_{\exp tX} (\mu(x))}{Y}
%            =
%                \myangle{\mu(x)}{\Ad_{\exp tX} Y}
%    \end{equation}
%    が成り立つから、最左辺と最右辺を$t$で微分して$t = 0$での値を求める。
%    左辺は
%    \begin{alignat}{1}
%        \dd{t} \myangle{\mu(x \exp tX)}{Y} \Big|_{t = 0}
%            &=
%                \myangle{(d\mu_X)_x}{X^*_x}
%                \\
%            &=
%                X^* \mu_Y (x)
%                \\
%            &=
%                \{ \mu_X, \mu_Y \} (x)
%                \quad
%                (\because \text{(M2)})
%    \end{alignat}
%    右辺は
%    \begin{alignat}{1}
%        \dd{t} \myangle{\mu(x)}{\Ad_{\exp tX} Y} \Big|_{t = 0}
%            &=
%                \myangle{\mu(x)}{\ad_X Y}
%                \\
%            &=
%                \mu_{\ad_X Y} (x)
%                \\
%            &=
%                \mu_{[X, Y]} (x)
%    \end{alignat}
%    したがって
%    $\{ \mu_X, \mu_Y \} = \mu_{[X, Y]}$が成り立つ。
%\end{proof}

\begin{theorem}[Noether の定理]
    $(M, \omega, H)$を Hamilton 系とする。
    また$G$の作用が Hamilton 作用であるとし、
    $\mu \colon M \to \frakg^\vee$を運動量写像とする。
    このとき、
    $H$が$G$-不変ならば、
    すべての$X \in \frakg$に対し
    $\mu_X \in \smooth(M)$は$(M, \omega, H)$の保存量である。
\end{theorem}

\begin{proof}
    $\{ H, \mu_X \} = 0$を示せばよいが
    \begin{alignat}{1}
        \{ H, \mu_X \}(x)
            &=
                - \{ \mu_X, H \}(x)
                \\
            &=
                - X^* H (x)
                \\
            &=
                - \dd{t} H(x \exp tX) \Big|_{t = 0}
                \\
            &=
                - \dd{t} H(x) \Big|_{t = 0}
                \quad
                (\because \text{$H$の$G$-不変性})
                \\
            &=
                0
    \end{alignat}
    より従う。
\end{proof}

\begin{proposition}[余接束上に誘導される Hamilton 作用]
    \label[proposition]{prop:Hamilton-action-induced-on-cotangent-bundle}
    $T^\vee M$上に誘導される$G$の作用
    $G \times T^\vee M \to T^\vee M, \;
        (g, (x, \xi)) \mapsto (R_g (x), d(R_g)_* \xi)$
    は Hamilton 作用である。
\end{proposition}

\begin{proof}
    まず canonical 1-形式$\theta \in \Omega^1(T^\vee M)$が
    $G$-不変であることを示しておく。
    そのために$g \in G$を固定し、
    $\what{R}_g$を$T^\vee M$上の右移動として
    $(\what{R}_g)^* \theta = \theta$を示す。
    \begin{alignat}{1}
        \myangle{(\what{R}_g^* \theta)_{(x, \xi)}}{v}
            &=
                \myangle{\theta_{\what{R}_g(x, \xi)}}{d(\what{R}_g)(v)}
                \\
            &=
                \myangle{d(R_g)_* \xi}{d\pi \circ d(\what{R}_g) (v)}
                \\
            &=
                \myangle{d(R_g)_* \xi}{d(R_g) \circ d\pi (v)}
                \\
            &=
                \myangle{d(R_g)^* d(R_g)_* \xi}{d\pi (v)}
                \\
            &=
                \myangle{\xi}{d\pi (v)}
                \\
            &=
                \myangle{\theta_{(x, \xi)}}{v}
    \end{alignat}
    よって$(R_g)^* \theta = \theta$である。
    したがって$\theta$は$G$-不変である。
    引き戻しと外微分の可換性より
    $\omega = -d\theta$も$G$-不変である。

    次に運動量写像を構成する。
    $\frakg$の基底$E_i \; (i = 1, \dots, \dim G)$を1組選び、
    その双対基底を$E^i \in \frakg^\vee$とおき
    \begin{equation}
        \mu \colon T^\vee M \to \frakg^\vee,
            \qquad
            (x, \xi)
                \mapsto
                    - \myangle{\theta_{(x, \xi)}}{(E_i)^*} E^i
    \end{equation}
    と定めると、これは{\smooth}である。
    (M1)について、$X = X^i E_i \in \frakg$に対し
    \begin{alignat}{1}
        \mu_X
            &\coloneqq
                \myangle{\mu}{X}
                \\
            &=
                \myangle{\mu}{X^i E_i}
                \\
            &=
                X^i \myangle{\mu}{E_i}
                \\
            &=
                - X^i \myangle{\theta}{E_i^*}
                \\
            &=
                - \myangle{\theta}{X^*}
                \\
            &=
                - \iota_{X^*} \theta
                \locallabel{eq:1}
    \end{alignat}
    したがって
    \begin{equation}
        - d\mu_X
            =
                d \circ \iota_{X^*} \theta
            =
                L_{X^*} \theta - \iota_{X^*} \circ d\theta
            =
                0 + \iota_{X^*} \omega
            =
                \iota_{X^*} \omega
    \end{equation}
    となるから、$\mu_X$は$X^*$の Hamiltonian である。
    よって(M1)が成り立つ。
    (M2)について、
    \localcref{eq:1}の関係式$\mu_X = - \iota_{X^*} \theta$も用いれば
    \begin{alignat}{1}
        \{\mu_X, \mu_Y\}
            =
                X^* \mu_Y
            =
                - X^* \iota_{Y^*} \theta
            =
                - L_{X^*} \iota_{Y^*} \theta
    \end{alignat}
    一方
    \begin{alignat}{1}
        \mu_{[X, Y]}
            =
                - \iota_{[X, Y]^*} \theta
            =
                - \iota_{[X^*, Y^*]} \theta
            =
                - L_{X^*} \iota_{Y^*} \theta
    \end{alignat}
    よって$\{\mu_X, \mu_Y\} = \mu_{[X, Y]}$を得る。
    したがって(M2)が成り立つ。
    以上より$\mu$は運動量写像である。
\end{proof}

% ------------------------------------------------------------
%
% ------------------------------------------------------------
\section*{今後の予定}

\begin{itemize}
    \item シンプレクティック商?
\end{itemize}

% ------------------------------------------------------------
%
% ------------------------------------------------------------
\section*{参考文献}

\nocite{_bayes_2020}
\nocite{BB19061613}
\nocite{BB13785508}

{
    \renewcommand{\bibsection}{}
    \bibliographystyle{amsalpha}
    \bibliography{./bibliography,../../mybibliography}
}

% ------------------------------------------------------------
%
% ------------------------------------------------------------
\newpage
\appendix
\renewcommand\thesection{\Alph{section}}
\setcounter{section}{0}
\section{付録}

\begin{proof}[\cref{prop:Hamilton-vector-field-basic-properties}の証明.]
    \uline{(1)} \quad
    Hamilton ベクトル場の定義より明らか。

    \uline{(2)} \quad
    \begin{alignat}{2}
        L_{X_f} \omega
            &=
                (d \circ \iota_{X_f} + \iota_{X_f} \circ d) \omega
                \quad
                &&(\because \text{Cartan の公式})
                \\
            &=
                0 + 0
                \quad
                &&(\because \iota_{X_f} \omega = -df, \; d \omega = 0)
                \\
            &=
                0
    \end{alignat}

    \uline{(3)} \quad
    $\omega = dx^i \wedge d\xi_i$と表せるから
    \begin{equation}
        - df
            =
                \omega(X_f, Y)
            =
                dx^i(X_f) d\xi_i(Y)
                - d\xi_i(X_f) dx^i(Y)
    \end{equation}
    係数を比較して
    $dx^i(X_f) = - \deldel[f]{\xi_i}, \;
        d\xi_i(X_f) = \deldel[f]{x^i}$
    を得る。
\end{proof}

\begin{proof}[\cref{prop:Poisson-bracket-basic-properties}の証明.]
    \uline{(1)} \quad
    $\{ {-}, {-} \}$の交代性は$\omega$の交代性より従う。
    Hamilton ベクトル場の性質より
    $\{ f, g \} = \omega(X_f, X_g) = - X_g f$であり、
    交代性より$\{ f, g \} = X_f g$も従う。
    ベクトル場の関数への作用の$\R$-線型性より
    $\{ {-}, {-} \}$は$\R$-双線型である。

    \uline{(2)} \quad
    $\omega([X_f, X_g], Y) = \omega(X_{\{f, g\}}, Y) \; (\forall Y \in \mathfrak{X}(M))$
    を示せば$\omega$の非退化性より(2)が従う。
    微分形式の Lie 微分の公式より
    \begin{alignat}{1}
        X_f (\omega(X_g, Y))
            &=
                (L_{X_f} \omega)(X_g, Y)
                + \omega(L_{X_f} X_g, Y)
                + \omega(X_g, L_{X_f} Y)
                \\
        \therefore \quad
        - X_f Y g
            &=
                0
                + \omega([X_f, X_g], Y)
                + \omega(X_g, [X_f, Y])
    \end{alignat}
    整理して
    \begin{alignat}{1}
        \omega([X_f, X_g], Y)
            &=
                - X_f Y g
                - \omega(X_g, [X_f, Y])
                \\
            &=
                - X_f Y g
                + [X_f, Y] g
                \\
            &=
                - Y X_f g
                \\
            &=
                - Y \{f, g\}
                \quad
                (\because (1))
                \\
            &=
                \omega(X_{\{f, g\}}, Y)
    \end{alignat}
    を得る。
    したがって$[X_f, X_g] = X_{\{f, g\}}$である。

    \uline{(3)} \quad
    (1), (2) を用いて
    \begin{alignat}{2}
        \{ f, \{ g, h \} \}
            &=
                - X_{\{g, h\}} f
                \quad
                &&(\because (1))
                \\
            &=
                - [X_g, X_h] f
                \quad
                &&(\because (2))
                \\
            &=
                - (X_g X_h f - X_h X_g f)
                \\
            &=
                - (\{ g, \{ h, f \} \} - \{ h, \{ g, f \} \})
                \quad
                &&(\because (1))
                \\
            &=
                - (\{ g, \{ h, f \} \} + \{ h, \{ f, g \} \})
                \quad
                &&(\because (1))
    \end{alignat}
    より従う。

    \uline{(4)} \quad
    ベクトル場の関数への作用が Leibniz 則を満たすことより従う。
\end{proof}

\end{document}