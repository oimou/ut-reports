\documentclass[report]{jlreq}
\usepackage{global}
\usepackage{./local}
\subfiletrue
\def\assetspath{../}
%\makeindex
\chead{2023/11/22}
\begin{document}

% ============================================================
%
% ============================================================

% ------------------------------------------------------------
%
% ------------------------------------------------------------
\section*{振り返りと導入}

前回は双対平坦構造に対して canonical ダイバージェンスと
シンプレクティック構造を定義した。
本稿では次のことを行う:
\begin{itemize}
    \item 双対平坦構造に付随するシンプレクティック構造の性質を調べる。
    \item 接束/余接束上の関数の Legendre 変換を定義する。
    \item 双対平坦構造に付随するシンプレクティック構造と Legendre 変換の関係を調べる。
\end{itemize}

なおシンプレクティック構造の話題は
\cite{_bayes_2020}
をベースにしている。

% ------------------------------------------------------------
%
% ------------------------------------------------------------
\section{双対平坦構造に付随するシンプレクティック構造}

以下、$M$を多様体、
$(g, \nabla, \nabla^*)$を$M$上の双対平坦構造とする。
また、
\cref{propdef:canonical-divergence}
で canonical ダイバージェンスが定義されることは一旦認めることにし、
$D \colon \calU \to \R, \;
    \Delta_M \subset \calU \opensubset M \times M$
を$(g, \nabla, \nabla^*)$の canonical ダイバージェンスとする。

\begin{propdef}[双対平坦構造に付随するシンプレクティック構造]
    \label[propdef]{propdef:symplectic-structure-associated-to-dual-flat-structure}
    $\omega_0 \in \Omega^2(T^\vee M)$を
    $T^\vee M$上の自然シンプレクティック形式とする。
    写像$d_1 D \colon \calU \to T^\vee M$を
    第1成分に関する微分、すなわち
    $d_1 D \coloneqq D(\tdeldel{x^i} \|) \, dx^i$
    で定め、$\calU$上の2-形式
    $\omega \in \Omega^2(\calU)$を
    $\omega \coloneqq (d_1 D)^* \omega_0$
    で定める。
    このとき次が成り立つ:
    \begin{enumerate}
        \item $x$を$M$の局所座標とし、
            記号の濫用で
            $\calU$の局所座標$(x, x^*)$を
            $x(p, q) \coloneqq x(p), \; x^*(p, q) \coloneqq x(q)$
            で定めると、
            座標$(x, x^*)$に関する$\omega$の成分表示は
            \begin{equation}
                \omega
                    =
                        D(\tdeldel{x^i} \| \tdeldel{x^j}) \,
                        dx^i \wedge dx^{*j}
            \end{equation}
            となる。
        \item $\omega$は$\calU$上のシンプレクティック形式である。
    \end{enumerate}
    $\omega$を
    双対平坦構造$(g, \nabla, \nabla^*)$に付随する
    \term{シンプレクティック構造}[symplectic structure]
        {シンプレクティック構造}[シンプレクティックこうぞう]
    と呼ぶ。
\end{propdef}

\begin{proof}
    \uline{(1)} \quad
    $x$を$M$の局所座標とし、
    $\calU$の局所座標$(x, x^*)$を
    $x(p, q) \coloneqq x(p), \; x^*(p, q) \coloneqq x(q)$
    で定める。
    $x$により定まる
    $T^\vee M$の自然な局所座標を$(x^1, \dots, x^n, \xi_1, \dots, \xi_n)$
    とおけば
    \begin{alignat}{1}
        \omega
            &=
                (d_1 D)^* \omega_0
                \\
            &=
                (d_1 D)^* (dx^i \wedge d\xi_i)
                \\
            &=
                d(x^i \circ d_1 D) \wedge d(\xi_i \circ d_1 D)
                \\
            &=
                dx^i \wedge \myparen{
                    D(\tdeldel{x^j} \tdeldel{x^i} \|) \, dx^j
                    +
                    D(\tdeldel{x^i} \| \tdeldel{x^j}) \, dx^{*j}
                }
                \\
            &=
                D(\tdeldel{x^i} \| \tdeldel{x^j}) \,
                dx^i \wedge dx^{*j}
    \end{alignat}
    を得る。

    \uline{(2)} \quad
    $d\omega = 0$であることと$\omega$が非退化であることを示せばよい。
    $d\omega = 0$は$d\omega = (d_1 D)^* d\omega_0 = 0$より従う。
    $\omega$が非退化であることを示す。
    $(U, \theta, \eta)$を$g$-凸な双対アファインチャートとすると
    (1)より
    \begin{alignat}{1}
        \omega
            &=
                D(\del_i \| \del_j) \, d\theta^i \wedge d\theta^{*j}
                \\
            &=
                - g_{ij}(p) \, d\theta^i \wedge d\theta^{*j}
    \end{alignat}
    を得る。
    したがって
    $\calU$の局所座標$(\theta, \theta^*)$に関する
    $\omega$の行列表示は
    $\begin{bmatrix}
        O & (-g_{ij}(p))_{ij} \\
        (g_{ij}(p))_{ij} & O
    \end{bmatrix}$
    となる。
    $g$の非退化性より$\omega$は非退化である。
\end{proof}

\begin{proposition}[$\omega$の成分表示]
    $\omega$を$(g, \nabla, \nabla^*)$に付随する
    シンプレクティック構造とする。
    このとき、
    $g$-凸な任意の双対アファインチャート$(U, \theta, \eta)$に対し、
    $\omega$は次の成分表示をもつ:
    \begin{alignat}{1}
        \omega
            &=
                - g_{ij} \,
                d\theta^i \wedge d\theta^{*j}
            =
                - d\eta_i \wedge d\theta^{*i}
            =
                - g_{ij} g^{*jk} \,
                d\theta^i \wedge d\eta^*_k
            =
                - g^{*ij} \,
                d\eta_i \wedge d\eta^*_j
    \end{alignat}
    ただし記号の濫用で
    $g_{ij}, g^{*ij} \colon U \times U \to \R$は
    $g_{ij}(p, q) \coloneqq g_{ij}(p), \;
        g^{*ij}(p, q) \coloneqq g^{ij}(q)$
    を表す。
\end{proposition}

\begin{remark}
    \textbf{任意の}双対アファインチャート$(U, \theta, \eta)$に対しては
    成り立つとは限らない。
\end{remark}

\begin{proof}
    一番左の等号は
    \cref{propdef:symplectic-structure-associated-to-dual-flat-structure}
    の証明(2)の中で示した。
    残りの等号は$(\theta, \eta)$が
    双対アファイン座標であることから従う。
\end{proof}

%\begin{example}[$M = \R^n$の場合]
%    \label[example]{example:canonical-symplectic-structure-on-Rn}
%    $M \coloneqq \R^n$とし、
%    $g$を Euclid 計量、
%    $\nabla \coloneqq \nabla^* \coloneqq \nabla^g$
%    とすると、
%    $(g, \nabla, \nabla^*)$は双対平坦構造となる。
%    このとき、
%    $(g, \nabla, \nabla^*)$の canonical ダイバージェンスは
%    $D \colon M \times M \to \R, \;
%        (p, q) \mapsto \frac{1}{2} \| p - q \|^2$
%    となり、
%    $(g, \nabla, \nabla^*)$のシンプレクティック構造$\omega$は、
%    同一視$M \times M \cong T^\vee M$のもとで
%    $T^\vee M$上の自然シンプレクティック構造$\omega_0$に対し
%    $\omega = - \omega_0$となる。
%    \begin{alignat}{1}
%        \omega
%            &=
%                D(\tdeldel{x^i} \| \tdeldel{x^{*j}}) \,
%                dx^i \wedge dx^{*j}
%            =
%                - \delta_j^i \,
%                dx^i \wedge dx^{*j}
%            =
%                - dx^i \wedge dx^{*i}
%            =
%                - \omega_0
%    \end{alignat}
%\end{example}

% ------------------------------------------------------------
%
% ------------------------------------------------------------
\section{接束/余接束上の関数の Legendre 変換}

以下、
$M$を多様体、
$E \coloneqq TM$、
$W \opensubset E$とし、
$L \colon W \to \R$を{\smooth}関数とする。
以下では$E = TM$の場合を考えるが、
$E = T^\vee M$の場合も同様である。

\begin{definition}[fiber derivative]
    写像
    $\bbF L \colon W \to E^\vee$を
    $L$のファイバー方向の微分、すなわち
    \begin{equation}
        \myangle{\bbF L(x, v)}{w}
            \coloneqq
                \dd{t} L(x, v + tw) \Big|_{t = 0}
                \qquad
                (x, v) \in W, \; w \in E_x
    \end{equation}
    で定める。
    $\bbF L$を$L$の\term{fiber derivative}
        {fiber derivative}[fiber derivative]
    という。
\end{definition}

\begin{remark}
    $\bbF L$はファイバーごとに線型とは限らない。
\end{remark}

\begin{definition}[接束/余接束上の関数の Legendre 変換]
    $\bbF L \colon W \to E^\vee$が
    像への微分同相であるとき、
    $\bbF L$を
    $L$による\term{Legendre 変換}
        {Legendre 変換}[Legendre へんかん]
    と呼ぶことがある。
    また、
    $W' \coloneqq \bbF L(W) \subset E^\vee$とおき、
    \begin{equation}
        H \colon W' \to \R,
            \qquad
            (x, \xi) \mapsto \myangle{\xi}{v} - L(x, v),
            \qquad
            (x, v) \coloneqq (\bbF L)^{-1}(x, \xi) \in W
    \end{equation}
    を$L$の\term{Legendre 変換}
        {Legendre 変換}[Legendre へんかん]
    と呼ぶ。
\end{definition}

\begin{proposition}[凸関数としての Legendre 変換との関係]
    各$x \in M$に対し$W_x \subset E_x$が凸集合で、
    $L|_{W_x} \colon W_x \to \R$が
    $\Hess (L|_{W_x}) > 0$をみたすとする
    (とくに$L|_{W_x}$は凸関数となる)。
    このとき、
    $\bbF L$は像への微分同相であり、
    $H$はファイバーごとに$L|_{W_x}$の凸関数としての Legendre 変換である。
\end{proposition}

\begin{proof}
    $H$がファイバーごとに$L|_{W_x}$の凸関数としての Legendre 変換であることは
    $H$の定義から明らか。
    あとは$\bbF L$が像への微分同相であることを示せばよい。
    そのためには$\bbF L$が単射かつ局所微分同相であることを示せばよい。
    $\bbF L$が単射であることは、
    $\bbF L$がファイバーを保つことと、
    各$L|_{W_x}$が凸ゆえに$\bbF L|_{W_x}$が単射となることより従う。
    また$M$の局所座標$x$をひとつ選んで固定し、
    $x$により定まる$E, E^\vee$の自然な局所座標を考えると、
    これらの座標に関する$\bbF L$の座標表示は
    $(x^i, v^i) \mapsto (x^i, \deldel[L]{v^i})$となる。
    $\Hess (L|_{W_x}) > 0$より
    この座標表示の Jacobi 行列は正則であるから、
    $\bbF L$は局所微分同相となる。
    以上より$\bbF L$は単射かつ局所微分同相、
    したがって像への微分同相である。
\end{proof}

% ------------------------------------------------------------
%
% ------------------------------------------------------------
\section{双対平坦構造に付随するシンプレクティック構造と Legendre 変換}

本節では、2通りの方法で$\calU \subset T^\vee M$とみなし、
それぞれの場合で$L, H$と$\psi, \varphi$との関係性を調べる。
以下、簡単のため
$M$が単一の$g$-凸な双対アファインチャートで覆われる場合を考える。

\begin{proposition}[$L, H$が$\psi, \varphi$と"整合"する ver.]
    \label[proposition]{prop:L-H-psi-varphi}
    $M$上の双対アファインチャート$(M, \theta, \eta)$をひとつ選び固定する。
    このとき次が成り立つ:
    \begin{enumerate}
        \item 写像
            \begin{equation}
                \Phi \colon \calU \to T^\vee M,
                    \qquad
                    (p, q) \mapsto (p, \theta^i(q) {d\eta_i}_p)
            \end{equation}
            は像への微分同相写像となる。
            また$T^\vee M$上の自然なシンプレクティック構造$\omega_0$に対し
            $\Phi^* \omega_0 = - \omega$が成り立つ。
    \end{enumerate}
    以下$W' \coloneqq \Phi(\calU)$とおく。
    $(M, \theta, \eta)$の
    双対ポテンシャル$(\psi, \varphi)$を1組選び固定する。
    このとき次が成り立つ:
    \begin{enumerate}
        \setcounter{enumi}{1}
        \item 関数
            \begin{equation}
                H \colon W' \to \R,
                    \qquad
                    (p, \theta^i(q) d\eta_i)
                        \mapsto
                            \psi(q)
            \end{equation}
            の fiber derivative $\bbF H \colon W' \to T^{\vee\vee} M = TM$は
            \begin{equation}
                \bbF H(p, \theta^i(q) d\eta_i)
                    =
                        (p, \eta_i(q) \tdeldel{\eta_i}),
                        \qquad
                        (p, q) \in \calU
            \end{equation}
            をみたす。
        \item $H$の Legendre 変換$L \colon W \to \R, \; W \coloneqq \bbF H(W')$は
            \begin{equation}
                L(p, \eta_i(q) \tdeldel{\eta_i})
                    + H(p, \theta^i(q) d\eta_i)
                    =
                        \myangle{\theta(q)}{\eta(q)},
                        \qquad
                        (p, q) \in \calU
            \end{equation}
            をみたす。
            したがって$L(p, \eta_i(q) \tdeldel{\eta_i}) = \varphi(q)$が成り立ち、
            この意味で$L, H$は$\psi, \varphi$と"整合"する。
    \end{enumerate}
\end{proposition}

\begin{proof}
    \uline{(1)} \quad
    像への微分同相であることは明らか。
    また、$\eta$により定まる
    $T^\vee M$上の自然な座標を$(\eta_i, \xi^i)$とおけば
    \begin{alignat}{1}
        \Phi^* \omega_0
            &=
                \Phi^* (d\eta_i \wedge d\xi^i)
                \\
            &=
                d(\eta_i \circ \Phi) \wedge d(\xi^i \circ \Phi)
                \\
            &=
                d\eta_i \wedge d\theta^{*i}
                \\
            &=
                - \omega
    \end{alignat}
    が成り立つ。

    \uline{(2)} \quad
    \begin{alignat}{1}
        \myangle{
            \bbF H(p, \theta^i(q) d\eta_i)
        }{
            d\eta_j
        }
            &=
                \dd{t}
                H(p, \theta^i(q) d\eta_i + t d\eta_j) \Big|_{t = 0}
                \\
            &=
                \dd{t}
                \psi \circ \theta^{-1} \myparen{
                    \theta^1(q), \dots, \theta^j(q) + t, \dots, \theta^n(q)
                }
                \Big|_{t = 0}
                \\
            &=
                \deldel[\psi]{\theta^j}(q)
                \\
            &=
                \eta_j(q)
    \end{alignat}
    より従う。

    \uline{(3)} \quad
    $H$の Legendre 変換の定義と(2)より
    \begin{equation}
        L(p, \eta_i(q) \tdeldel{\eta_i})
            =
                \myangle{\theta(p)}{\eta(q)}
                -
                H(p, \theta^i(q) d\eta_i)
    \end{equation}
    が成り立つ。
\end{proof}

\begin{proposition}[$L, H$が$\psi, \varphi$と"整合"しない ver.]
    次が成り立つ:
    \begin{enumerate}
        \item 写像$d_1 D \colon \calU \to T^\vee M$
            は像へのシンプレクティック同相写像となる。
    \end{enumerate}
    以下$W' \coloneqq d_1 D(\calU)$とおく。
    $M$上の双対アファインチャート$(M, \theta, \eta)$をひとつ選び固定し、
    さらに$(M, \theta, \eta)$の
    双対ポテンシャル$(\psi, \varphi)$を1組選び固定する。
    このとき次が成り立つ:
    \begin{enumerate}
        \setcounter{enumi}{1}
        \item 関数
            \begin{equation}
                H \colon W' \to \R,
                    \qquad
                    (p, \xi) \mapsto \psi(q),
                    \qquad
                    q \coloneqq \theta^{-1}\myparen{
                        \theta^i(p)
                        -
                        \myangle{\xi}{\tdeldel{\eta_i}}
                    }_{i = 1}^n
            \end{equation}
            の fiber derivative $\bbF H \colon W' \to T^{\vee\vee} M = TM$は
            \begin{equation}
                \bbF H(p, (\theta^i(p) - \theta^i(q)) d\eta_i)
                    =
                        (p, \eta_i(q) \tdeldel{\eta_i}),
                        \qquad
                        (p, q) \in \calU
            \end{equation}
            をみたす。
        \item $H$の Legendre 変換$L \colon W \to \R, \; W \coloneqq \bbF H(W')$は
            \begin{equation}
                L(p, \eta_i(q) \tdeldel{\eta_i})
                    + H(p, (\theta^i(p) - \theta^i(q)) d\eta_i)
                    =
                        \myangle{\theta(p) - \theta(q)}{\eta(q)},
                        \qquad
                        (p, q) \in \calU
            \end{equation}
            をみたす。
    \end{enumerate}
\end{proposition}

\begin{proof}
    任意の$(M, \theta, \eta)$に関し
    $d_1 D(p, q) = (p, (\theta^i(p) - \theta^i(q)) d\eta_i)$
    が成り立つことに注意すれば
    \cref{prop:L-H-psi-varphi}の証明と同様。
\end{proof}

% ------------------------------------------------------------
%
% ------------------------------------------------------------
%\section{双対平坦構造に付随する概複素構造/概K\"ahler構造}
%
%\begin{propdef}[双対平坦構造に付随する概複素構造]
%    $\calU$上の概複素構造$J \colon T\calU \to T^\vee \calU$を
%    次のように定めることができる:
%    各$(p, q) \in \calU$に対し、
%    $p, q$を含む
%    $g$-凸な双対アファインチャート$(U, \theta, \eta)$をひとつ選び、
%    \begin{equation}
%        J \deldel{\eta_j} \coloneqq \deldel{\theta^{*j}},
%            \qquad
%            J \deldel{\theta^{*j}} \coloneqq - \deldel{\eta_j}
%    \end{equation}
%    と定める。
%    この$J$は$(U, \theta, \eta)$のとり方によらない。
%\end{propdef}
%
%\begin{propdef}[双対平坦構造に付随する概K\"ahler構造]
%    $(g, \omega, J)$は$M$上の概K\"ahler構造を定める。
%\end{propdef}

% ------------------------------------------------------------
%
% ------------------------------------------------------------
\section*{今後の予定}

\begin{itemize}
    \item 概複素構造と概K\"ahler構造
    \item Hamilton フロー
    \item モーメント写像
\end{itemize}

% ------------------------------------------------------------
%
% ------------------------------------------------------------
\section*{参考文献}

\nocite{amari_information_2016}
\nocite{_bayes_2020}
\nocite{BB19061613}
\nocite{BA87914734}
\nocite{marsden_introduction_1999}

{
    \renewcommand{\bibsection}{}
    \bibliographystyle{amsalpha}
    \bibliography{./bibliography,../../mybibliography}
}

% ------------------------------------------------------------
%
% ------------------------------------------------------------
\newpage
\appendix
\renewcommand\thesection{\Alph{section}}
\setcounter{section}{0}
\section{付録}

\subsection{多様体上の構造}

\begin{definition}[シンプレクティックベクトル空間]
    $2n$次元$\R$-ベクトル空間$V$と
    $V$上の非退化交代形式$\omega \colon V \times V \to \R$の組
    $(V, \omega)$を
    \term{シンプレクティックベクトル空間}[symplectic vector space]
        {シンプレクティックベクトル空間}[シンプレクティックベクトル空間]
    という。
\end{definition}

\begin{definition}[シンプレクティック形式]
    $M$を$2n$次元多様体とする。
    $\omega \in \Omega^2(M)$が
    $M$上の
    \term{シンプレクティック形式}[symplectic form]
        {シンプレクティック形式}[シンプレクティックけいしき]
    であるとは、
    $\omega$が閉形式かつ
    各点$x \in M$で
    $(T_x M, \omega_x)$がシンプレクティックベクトル空間であることをいう。
\end{definition}

\begin{example}[標準シンプレクティック形式]
    $\R^{2n}$の標準的な座標
    $(x^1, \ldots, x^n, y_1, \ldots, y_n)$に対し
    $\omega_0 \coloneqq dx^i \wedge dy_i \in \Omega^2(\R^{2n})$
    は$\R^{2n}$上のシンプレクティック構造である。
    $\omega_0$を$\R^{2n}$上の
    \term{標準シンプレクティック形式}[standard symplectic form]
        {標準シンプレクティック形式}[ひょうじゅんシンプレクティックけいしき]
    という。
\end{example}

\begin{example}[余接束の自然シンプレクティック形式]
    $M$を$n$次元多様体とする。
    余接束$\pi \colon T^\vee M \to M$上の
    1-形式$\theta \in \Omega^1(T^\vee M)$を
    \begin{equation}
        \theta_{(q, p)} (v)
            \coloneqq
                p(d\pi_{(q, p)} (v))
    \end{equation}
    で定め、これを
    \term{トートロジカル1-形式}[tautological 1-form]
        {トートロジカル1-形式}[トートロジカル1-けいしき]
    と呼ぶ。
    このとき
    $\omega_0 \coloneqq -d\theta \in \Omega^2(T^\vee M)$は
    $T^\vee M$上のシンプレクティック構造となり、
    これを$T^\vee M$上の
    \term{自然シンプレクティック形式}[canonical symplectic form]
        {自然シンプレクティック形式}[しぜんシンプレクティックけいしき]
    と呼ぶ。
\end{example}

\begin{proposition}[自然シンプレクティック形式の成分表示]
    $M$を$n$次元多様体、
    $x = (x^i)_i$を$M$の局所座標とする。
    $x$により定まる$T^\vee M$の局所座標を
    $(x^1, \dots, x^n, \xi_1, \dots, \xi_n)$とおくと、
    これに関する自然シンプレクティック形式$\omega_0$の成分表示は
    \begin{equation}
        \omega_0
            =
                dx^i \wedge d\xi_i
    \end{equation}
    となる。
\end{proposition}

\begin{proof}
    $\pi(q, p) = q$ゆえ
    $d\pi^* (dx^i) = dx^i$であることに注意すると、
    トートロジカル1-形式の成分表示
    \begin{equation}
        \theta_{(q, p)}
            =
                d\pi_{(q, p)}^* (\xi_i dx^i)
            =
                \xi_i dx^i
    \end{equation}
    より命題の等式が従う。
\end{proof}

\begin{definition}[概複素構造]
    \TODO{}
\end{definition}

\begin{definition}[概K\"ahler構造]
    \TODO{}
\end{definition}

\subsection{canonical ダイバージェンスの定義域}

\begin{definition}[$\nabla$-凸集合]
    部分集合$S \subset M$が
    \term{$\nabla$-凸}[$\nabla$-convex]
        {$\nabla$-凸集合}[nabla とつしゅうごう]
    であるとは、
    任意の$p, q \in S$に対し、
    $p$から$q$への$S$内の$\nabla$-測地線が
    ただひとつ存在することをいう。
\end{definition}

\begin{definition}[$g$-凸集合]
    部分集合$S \subset M$が
    \term{$g$-凸}[$g$-convex]
        {$g$-凸集合}[g とつしゅうごう]
    であるとは、
    任意の$p, q \in S$に対し、
    $p$から$q$への$M$内の$\nabla^g$-測地線で
    最短なものがただひとつ存在し、
    かつそれが$S$内に含まれることをいう。
\end{definition}

\begin{definition}[canonical ダイバージェンスの定義域]
    \begin{alignat}{1}
        \calU
            &\coloneqq
                \mybrace{
                    (p, q) \in M \times M
                    \;\Bigg|\;
                    \parbox{11cm}{
                        \centering
                        $p, q$を含む$g$-凸開集合を含む、
                        \\
                        $\nabla$-凸または$\nabla^*$-凸な
                        双対アファインチャート$(U, \theta, \eta)$が存在する
                    }
                }
    \end{alignat}
\end{definition}

\begin{proposition}
    次は同値である:
    \begin{enumerate}
        \item $U$は$\nabla$-凸であり、$U$上の双対アファイン座標が存在する。
        \item $U$は$\nabla$-凸であり、$U$上の$\nabla$-アファイン座標が存在する。
    \end{enumerate}
\end{proposition}

\begin{proof}
    \uline{(1) $\Rightarrow$ (2)} \quad
    明らか。

    \uline{(2) $\Rightarrow$ (1)} \quad
    $\nabla$-凸性より
    $\eta \coloneqq (\eta_i)_i, \;
        \eta_i \coloneqq \deldel[\psi]{\theta^i}$
    は$U$上単射である。
    したがって$\eta$は$U$から像への微分同相であり、
    $U$上の座標となる。
    このとき$(\theta, \eta)$は$U$上の双対アファイン座標となる。
    さらに$\varphi \coloneqq \theta^i \eta_i - \psi$とおけば
    $(\psi, \varphi)$は$(U, \theta, \eta)$の双対ポテンシャルとなる。
\end{proof}

\begin{remark}
    $p, q$を含む$g$-凸開集合が存在したとしても、
    それを含む$\nabla$-凸または$\nabla^*$-凸な
    双対アファインチャートが存在するとは限らない。
    たとえば、正規分布族を考え、自然パラメータ空間 (これは上半空間となる) から
    線分$\{ 0 \} \times (0, 2)$を除いた空間を考えると、
    2点$p = (2, 1), q = (-2, 1)$を含む$g$-凸開集合が存在する
    (上半楕円形にとればよい) が、
    2点$p, q$を結ぶ$\nabla$-測地線も$\nabla^*$-測地線も存在しない
    ($\nabla$-測地線は"水平線"、$\nabla^*$-測地線は"下に凸"な曲線)
    ため、
    2点$p, q$を含む$\nabla$-凸または$\nabla^*$-凸な双対アファインチャートは存在しない。
\end{remark}

\begin{lemma}[$g$-凸開近傍の存在]
    各$p \in M$に対し、
    ある$R > 0$が存在して、
    任意の$r \in (0, R)$に対し
    $B_r(p) \subset M$は$g$-凸である。
\end{lemma}

\begin{proof}
    Riemann 多様体の教科書にある。
\end{proof}

\begin{lemma}[$\calU$の多様体構造]
    $\calU$は$\Delta_M$を含む$M \times M$の開集合である。
    したがって$\calU$には$M \times M$の開部分多様体の構造が入る。
\end{lemma}

\begin{proof}
    開集合となることは定義から明らか。
    また、各$p_0 \in M$に対し、
    $p_0$のまわりの双対アファインチャート$(U, \theta, \eta)$が存在するから、
    $p_0$の$\nabla$-凸開近傍$U'$を$U' \subset U$となるようにとれば、
    補題より$U'$は
    $p_0$の$g$-凸開近傍を含む。
    したがって
    $U' \times U'$は$M \times M$における
    $p_0$の近傍であり、$\calU$に含まれる。
    よって$\calU$は$\Delta_M$を含む。
\end{proof}

\subsection{canonical ダイバージェンス}

\begin{propdef}[canonical ダイバージェンス]
    \label[propdef]{propdef:canonical-divergence}
    関数$D \colon \calU \to \R$を次のように定める:
    $(p, q) \in \calU$を固定し、
    $p, q$を含む$g$-凸開集合な
    %$\nabla$-凸または$\nabla^*$-凸な
    双対アファインチャート$(U, \theta, \eta)$をひとつ選び、
    その双対ポテンシャル$(\psi, \varphi)$を1組選ぶ。
    このとき、点$(p, q)$における
    \begin{equation}
        \psi(q) + \varphi(p) - \myangle{\theta(q)}{\eta(p)}
    \end{equation}
    の値は
    $(U, \theta, \eta)$や$(\psi, \varphi)$の選び方によらない。
    この値を$D(p \| q)$と記す。
    以上により定まる関数$D \colon \calU \to \R$を
    双対平坦構造$(g, \nabla, \nabla^*)$の
    \term{canonical ダイバージェンス}
        {canonical ダイバージェンス}[canonical ダイバージェンス]
    と呼ぶ。
\end{propdef}

\begin{proof}
    $(p, q) \in \calU$とし、
    $(U, \theta, \eta), (U', \theta', \eta')$を
    それぞれ条件をみたす双対アファインチャート、
    $(\psi, \varphi), (\psi', \varphi')$を
    それぞれの双対ポテンシャルとする。
    $(p, q) \in \calU$ゆえ
    $p, q$を含む$g$-凸集合が存在するから、
    $p$から$q$への$M$内の$\nabla^g$-測地線$\gamma$がただひとつ存在する。
    ここで
    $U, U'$は$p, q$を含む$g$-凸開集合を含んでいたから、
    $U \cap U'$は$\gamma$の像を含む。
    このとき$U \cap U'$の連結成分$C$であって
    $\gamma$の像を含むものがただ1つ存在する。

    $C$の連結性より
    $\psi'(q) - \psi(q)
        = (\text{$C$上の定数})
        = \psi'(p) - \psi(p)$
    が成り立つ。
    よって
    \begin{alignat}{1}
        \psi'(q) + \varphi'(p) - \myangle{\theta'(q)}{\eta'(p)}
            &=
                \psi'(q) - \psi'(p) - \myangle{\theta'(q) - \theta'(p)}{\eta'(p)}
                \\
            &=
                \psi(q) - \psi(p) - \myangle{\theta'(q) - \theta'(p)}{\eta'(p)}
    \end{alignat}
    が成り立つ。
    あとは
    $\myangle{\theta'(q) - \theta'(p)}{\eta'(p)}
        = \myangle{\theta(q) - \theta(p)}{\eta(p)}$
    を示せばよい。

    \TODO{locally const. の言葉で書き直す}
    $C$の連結性より、
    組$(A = (A_i^j)_{i, j}, b) \in \GL_n(\R) \times \R^n$であって
    $\theta'(r) = A \theta(r) + b \; (\forall r \in C)$
    をみたすものがただ1組存在する。
    よって任意の$r \in C$に対し
    \begin{alignat}{1}
        \eta_i(r)
            &=
                \deldel[\psi]{\theta^i}(r)
                \qquad
                (\because \text{$d\psi = \eta_i d\theta^i$})
                \\
            &=
                \deldel[\psi']{\theta^i}(r)
                \qquad
                (\because \text{$\psi' - \psi$は$C$上定数})
                \\
            &=
                \deldel[\theta'^j]{\theta^i}(r)
                \deldel[\psi']{\theta'^j}(r)
                \\
            &=
                A_i^j \eta'_j(r)
                \qquad
                (\because \text{$d\psi' = \eta'_j d\theta'^j$})
                \\
        \therefore \eta(r)
            &=
                A \eta'(r)
    \end{alignat}
    が成り立つ。
    さらに任意の$r \in C$に対し
    \begin{alignat}{1}
        \theta'^i(r)
            &=
                \deldel[\varphi']{\eta'_i}(r)
                \qquad
                (\because \text{$d\varphi' = \theta'^i d\eta'_i$})
                \\
            &=
                \deldel[\varphi]{\eta'_i}(r)
                \qquad
                (\because \text{$\varphi' - \varphi$は$C$上定数})
                \\
            &=
                \deldel[\eta_j]{\eta'_i}(r)
                \deldel[\varphi]{\eta_j}(r)
                \\
            &=
                A_j^i \theta^j(r)
                \qquad
                (\because \text{$d\varphi = \theta^j d\eta_j, \; \eta = A \eta'$})
                \\
        \therefore \theta'(r)
            &=
                A \theta(r)
    \end{alignat}
    が成り立つ。
    したがって
    \begin{equation}
        \myangle{\theta'(q) - \theta'(p)}{\eta'(p)}
            =
                \myangle{A(\theta(q) - \theta(p))}{A^{-1} \eta(p)}
            =
                \myangle{\theta(q) - \theta(p)}{\eta(p)}
    \end{equation}
    が示された。
\end{proof}

\begin{proposition}[canonical ダイバージェンスの性質]
    $(g, \nabla^*, \nabla)$の canonical ダイバージェンスを$D^*$として
    \begin{enumerate}
        \item $D$は{\smooth}関数である。
        \item $D(p \| q) \ge 0$
        \item $D(p \| q) = 0 \iff p = q$
        \item $D(p \| q) = D^*(q \| p)$
    \end{enumerate}
\end{proposition}

\begin{proof}
    \uline{(1)} \quad
    局所的な{\smooth}性を示せばよい。
    $(p, q) \in \calU$とし、
    $(U, \theta, \eta)$を条件をみたす双対アファインチャートとすれば、
    $(p, q)$の近傍$U \times U$上で$D$は{\smooth}である。

    \uline{(2), (3)} \quad
    $\psi$の$\nabla$-凸性あるいは
    $\varphi$の$\nabla^*$-凸性より従う。

    \uline{(4)} \quad
    定義より明らか。
\end{proof}

\end{document}