\documentclass[report]{jlreq}
\usepackage{global}
\usepackage{./local}
\subfiletrue
\def\assetspath{../}
%\makeindex
\chead{2023/11/22}
\begin{document}

% ============================================================
%
% ============================================================

% ------------------------------------------------------------
%
% ------------------------------------------------------------
\section*{振り返りと導入}

...

% ------------------------------------------------------------
%
% ------------------------------------------------------------
\section{双対平坦構造とシンプレクティック構造}

以下、$M$を多様体、
$(g, \nabla, \nabla^*)$を$M$上の双対平坦構造、
$\calU \opensubset M \times M$を canonical ダイバージェンスの定義域、
$D \colon \calU \to \R$を canonical ダイバージェンスとする。

\begin{definition}[良いチャート]
    次をみたす
    双対アファインチャート$(U, \theta, \eta)$を
    ここだけの用語で
    \termsilent{良いチャート}と呼ぶ:
    \begin{enumerate}
        \item $U \times U \subset \calU$である。
        \item $U$は$g$-凸である。
    \end{enumerate}
\end{definition}

\begin{propdef}[双対平坦構造のシンプレクティック構造]
    $\omega_0 \in \Omega^2(T^\vee M)$を
    $T^\vee M$上の自然シンプレクティック形式とする。
    写像$d_1 D \colon \calU \to T^\vee M$を
    第1成分に関する微分、すなわち
    $d_1 D \coloneqq D(\tdeldel{x^i} \|) \, dx^i$
    で定め、$\calU$上の2-形式
    $\omega \in \Omega^2(\calU)$を
    $\omega \coloneqq (d_1 D)^* (\omega_0)$
    で定める。
    このとき次が成り立つ:
    \begin{enumerate}
        \item $M$の任意の局所座標$x = (x_i)_i$に対し、
            $x^* \coloneqq x$とおいて
            $\calU$の局所座標$(x, x^*) = (x^1, \dots, x^n, x^{*1}, \dots, x^{*n})$
            を定めると、
            $\omega$の成分表示は
            \begin{equation}
                \omega
                    =
                        D(\tdeldel{x^i} \| \tdeldel{x^{*j}}) \,
                        dx^i \wedge dx^{*j}
            \end{equation}
            となる。
        \item $\omega$は$\calU$上のシンプレクティック形式である。
    \end{enumerate}
    $\omega$を
    双対平坦構造$(g, \nabla, \nabla^*)$の
    \term{シンプレクティック構造}
        {シンプレクティック構造}[シンプレクティックこうぞう]
    と呼ぶ。
\end{propdef}

\begin{proof}
    \uline{(1)} \quad
    前回示した。

    \uline{(2)} \quad
    $d_1 D$がはめ込みであることを示せばよい。
    座標$(\theta, \theta^*)$に関する座標表示の
    Jacobi 行列を考えると
    $\begin{bmatrix}
        I & O \\
        O & -g
    \end{bmatrix}$
    の形になることから、$g$の非退化性より
    $d_1 D$がはめ込みであることが従う。
\end{proof}

\begin{proposition}[$\omega$の基本性質]
    $\omega$を双対平坦構造$(g, \nabla, \nabla^*)$の
    シンプレクティック構造、
    $(U, \theta, \eta)$を良いチャートとすると、
    $\forall p, q \in U, \; (p, q) \in \calU$に対し
    次が成り立つ:
    \begin{alignat}{1}
        \omega_{(p, q)}
            &=
                - g_{ij}(p) \,
                d\theta^i \wedge d\theta^{*j}
                \\
            &=
                - d\eta_i \wedge d\theta^{*i}
                \\
            &=
                - g_{ij}(p) g^{jk}(q) \,
                d\theta^i \wedge d\eta^*_k
                \\
            &=
                - g^{ij}(q) \,
                d\eta_i \wedge d\eta^*_j
    \end{alignat}
\end{proposition}

\begin{remark}
    \textbf{任意の}双対アファインチャート$(U, \theta, \eta)$に対しては
    $\omega = - d\eta_i \wedge d\theta^{*i}$
    が成り立つとは限らない。
\end{remark}

\begin{proof}
    \TODO{}
\end{proof}

\begin{example}[$M = \R^n$の場合]
    $M \coloneqq \R^n$とし、
    $g$を Euclid 計量、
    $\nabla \coloneqq \nabla^* \coloneqq \nabla^g$
    とすると、
    $(g, \nabla, \nabla^*)$は双対平坦構造となる。
    このとき、
    $(g, \nabla, \nabla^*)$の canonical ダイバージェンスは
    $D \colon M \times M \to \R, \;
        (p, q) \mapsto \frac{1}{2} \| p - q \|^2$
    となり、
    $(g, \nabla, \nabla^*)$のシンプレクティック構造$\omega$は、
    同一視$M \times M \cong T^\vee M$のもとで
    $T^\vee M$上の自然シンプレクティック構造$\omega_0$に対し
    $\omega = - \omega_0$となる。
    \begin{alignat}{1}
        \omega
            &=
                D(\tdeldel{x^i} \| \tdeldel{x^{*j}}) \,
                dx^i \wedge dx^{*j}
                \\
            &=
                - \delta_j^i \,
                dx^i \wedge dx^{*j}
                \\
            &=
                - dx^i \wedge dx^{*i}
                \\
            &=
                - \omega_0
    \end{alignat}
\end{example}

\begin{propdef}[ダイバージェンスを使わない直接的な定義]
    $g$-凸な任意の双対アファインチャート$(U, \theta, \eta)$に対し、
    \begin{equation}
        d\eta_i \wedge d\theta^{*i}
    \end{equation}
    は$(U, \theta, \eta)$の選び方によらない。
\end{propdef}

\begin{proof}
    $g$-凸な双対アファインチャート$(U, \theta, \eta), (U', \theta', \eta')$に対し、
    \begin{alignat}{1}
        d\eta_i \wedge d\theta^{*i}
            &=
                \deldel[\eta_i]{\eta'_j}(p)
                \deldel[\theta^i]{\theta'^k}(q) \,
                d\eta'_j \wedge d\theta'^{*k}
    \end{alignat}
    が成り立つ。
    $U, U'$の$g$-凸性より、
    $p, q$を結ぶ$M$内の最短測地線は$U \cap U'$に含まれる。
    したがって$p, q$は$U \cap U'$の単一の連結成分$C$に含まれる。
    $C$上で
    $\deldel[\eta_i]{\eta'_j} = \deldel[\theta'^j]{\theta^i}$は
    定数だから、
    $\deldel[\eta_i]{\eta'_j}(p)
        \deldel[\theta^i]{\theta'^k}(q) = \delta_k^j$
    が成り立つ。
\end{proof}

% ------------------------------------------------------------
%
% ------------------------------------------------------------
\section{シンプレクティック幾何における Legendre 変換}

\TODO{}

% ------------------------------------------------------------
%
% ------------------------------------------------------------
\section*{今後の予定}

\begin{itemize}
    \item Legendre 変換
    \item モーメント写像
\end{itemize}

% ------------------------------------------------------------
%
% ------------------------------------------------------------
\section*{参考文献}

%Legendre 変換については
%\cite{niculescu_convex_2018}
%を参考にした。
%期待値パラメータに関しては
%\cite{wainwright_graphical_2007}を参考にした。

\nocite{amari_information_2016}
\nocite{_bayes_2020}

{
    \renewcommand{\bibsection}{}
    \bibliographystyle{amsalpha}
    \bibliography{./bibliography,../../mybibliography}
}

% ------------------------------------------------------------
%
% ------------------------------------------------------------
\newpage
\appendix
\renewcommand\thesection{\Alph{section}}
\setcounter{section}{0}
\section{付録}

\subsection{双対ポテンシャル}

\begin{definition}[双対ポテンシャル]
    $(U, \theta, \eta)$を双対アファインチャートとする。
    関数$\psi, \varphi \colon U \to \R$の組
    $(\psi, \varphi)$が$(U, \theta, \eta)$の
    \term{双対ポテンシャル}[dual potential]
        {双対ポテンシャル}[そうついぽてんしゃる]
    であるとは、$U$上で
    \begin{equation}
        d\psi = \eta_i d\theta^i,
            \quad
            d\varphi = \theta^i d\eta_i,
            \quad
            \psi + \varphi = \theta^i \eta_i
    \end{equation}
    が成り立つことをいう。
\end{definition}

\begin{proposition}[双対ポテンシャルの基本性質]
    $(U, \theta, \eta)$を双対アファインチャート、
    $(\psi, \varphi)$を$(U, \theta, \eta)$の双対ポテンシャルとする。
    このとき次が成り立つ:
    \begin{enumerate}
        \item $U$上で
            $\psi$は$g$の$\nabla$-ポテンシャルであり、
            $\varphi$は$g$の$\nabla^*$-ポテンシャルである。
        \item $(\psi, \varphi), (\psi', \varphi')$を
            $(U, \theta, \eta)$の双対ポテンシャルとすると、
            $U$の連結成分ごとに
            $\psi' - \psi$および$\varphi' - \varphi$は定数である。
    \end{enumerate}
\end{proposition}

\begin{proof}
    \uline{(1)} \quad
    $(\theta, \eta)$が双対アファイン座標であることから
    \begin{equation}
        \nabla^2 \psi
            =
                \deldel[\eta_i]{\theta^j} \, d\theta^i d\theta^j
            =
                g_{ij} \, d\theta^i d\theta^j
            =
                g,
                \quad
        \nabla^{*2} \varphi
            =
                \deldel[\theta^i]{\eta_j} \, d\eta_i d\eta_j
            =
                g^{ij} \, d\eta_i d\eta_j
            =
                g
    \end{equation}
    を得る。

    \uline{(2)} \quad
    $\psi' - \psi$について示す。
    $\psi, \psi'$が$g$の$\nabla$-ポテンシャルであることより
    $U$上で$\nabla^2 (\psi' - \psi) = 0$である。
    したがって$U$の各連結成分$C$に対し、
    組$(a_C, b_C) \in \R^n \times \R$であって
    $\psi'(r) - \psi(r)
        = \myangle{a_C}{\theta(r)} + b_C \;
        (\forall r \in C)$
    なるものがただ1組存在する。
    さらに$\psi, \psi'$が双対ポテンシャルであることより
    $C$上で$d(\psi' - \psi) = 0$だから
    $a_C = 0$が成り立つ。
    よって$C$上で$\psi' - \psi = b_C$が成り立つ。
    $\varphi' - \varphi$についても同様。
\end{proof}

\subsection{canonical ダイバージェンスの定義域}

\begin{definition}[$\nabla$-凸集合]
    部分集合$S \subset M$が
    \term{$\nabla$-凸}[$\nabla$-convex]
        {$\nabla$-凸集合}[nabla とつしゅうごう]
    であるとは、
    任意の$p, q \in S$に対し、
    $p$から$q$への$S$内の$\nabla$-測地線が
    ただひとつ存在することをいう。
\end{definition}

\begin{definition}[$g$-凸集合]
    部分集合$S \subset M$が
    \term{$g$-凸}[$g$-convex]
        {$g$-凸集合}[g とつしゅうごう]
    であるとは、
    任意の$p, q \in S$に対し、
    $p$から$q$への$M$内の$\nabla^g$-測地線で
    最短なものがただひとつ存在し、
    かつそれが$S$内に含まれることをいう。
\end{definition}

\begin{definition}[canonical ダイバージェンスの定義域]
    \begin{alignat}{1}
        \calU
            &\coloneqq
                \mybrace{
                    (p, q) \in M \times M
                    \;\Bigg|\;
                    \parbox{11cm}{
                        \centering
                        $p, q$を含む$g$-凸開集合を含む、
                        \\
                        $\nabla$-凸または$\nabla^*$-凸な
                        双対アファインチャート$(U, \theta, \eta)$が存在する
                    }
                }
    \end{alignat}
\end{definition}

\begin{proposition}
    次は同値である:
    \begin{enumerate}
        \item $U$は$\nabla$-凸であり、$U$上の双対アファイン座標が存在する。
        \item $U$は$\nabla$-凸であり、$U$上の$\nabla$-アファイン座標が存在する。
    \end{enumerate}
\end{proposition}

\begin{proof}
    \uline{(1) $\Rightarrow$ (2)} \quad
    明らか。

    \uline{(2) $\Rightarrow$ (1)} \quad
    $\nabla$-凸性より
    $\eta \coloneqq (\eta_i)_i, \;
        \eta_i \coloneqq \deldel[\psi]{\theta^i}$
    は$U$上の座標となる。
    このとき$(\theta, \eta)$は$U$上の双対アファイン座標となる。
    さらに$\varphi \coloneqq \theta^i \eta_i - \psi$とおけば
    $(\psi, \varphi)$は$(U, \theta, \eta)$の双対ポテンシャルとなる。
\end{proof}

\begin{remark}
    ~
    \begin{itemize}
        \item $p, q$を含む$g$-凸開集合が存在したとしても、
            $p, q$のまわりの良いチャートが存在するとは限らない。
            たとえば、正規分布族を考え、自然パラメータ空間 (これは上半空間となる) から
            $\{ 0 \} \times (0, 2)$を除いた空間を考えると、
            2点$p = (2, 1), q = (-2, 1)$を含む$g$-凸開集合が存在するが、
            2点を結ぶ$\nabla$-測地線も$\nabla^*$-測地線も存在しないため、
            2点を含む$\nabla$-凸または$\nabla^*$-凸な双対アファインチャートは存在しない。
    \end{itemize}
\end{remark}

\begin{lemma}[$g$-凸開近傍の存在]
    各$p \in M$に対し、
    ある$R > 0$が存在して、
    任意の$r \in (0, R)$に対し
    $B_r(p) \subset M$は$g$-凸である。
\end{lemma}

\begin{proof}
    \TODO{cf. Riemann 多様体の教科書}
\end{proof}

\begin{lemma}[$\calU$の多様体構造]
    $\calU$は$\Delta_M$を含む$M \times M$の開集合である。
    したがって$\calU$には$M \times M$の開部分多様体の構造が入る。
\end{lemma}

\begin{proof}
    開集合となることは定義から明らか。
    また、各$p_0 \in M$に対し、
    $p_0$のまわりの双対アファインチャート$(U, \theta, \eta)$が存在するから、
    $p_0$の$\nabla$-凸開近傍$U'$を$U' \subset U$となるようにとれば、
    補題より$U'$は
    $p_0$の$g$-凸開近傍を含む。
    したがって
    $U' \times U'$は$M \times M$における
    $p_0$の近傍であり、$\calU$に含まれる。
    よって$\calU$は$\Delta_M$を含む。
\end{proof}

\subsection{canonical ダイバージェンス}

\begin{propdef}[canonical ダイバージェンス]
    \label[propdef]{propdef:canonical-divergence}
    関数$D \colon \calU \to \R$を次のように定める:
    $(p, q) \in \calU$を固定し、
    $p, q$を含む$g$-凸開集合を含む
    $\nabla$-凸または$\nabla^*$-凸な
    双対アファインチャート$(U, \theta, \eta)$をひとつ選び、
    その双対ポテンシャル$(\psi, \varphi)$を1組選ぶ。
    このとき、点$(p, q)$における
    \begin{equation}
        \psi(q) + \varphi(p) - \myangle{\theta(q)}{\eta(p)}
    \end{equation}
    の値は
    $(U, \theta, \eta)$や$(\psi, \varphi)$の選び方によらない。
    この値を$D(p \| q)$と記す。
    以上により定まる関数$D \colon \calU \to \R$を
    双対平坦構造$(g, \nabla, \nabla^*)$の
    \term{canonical ダイバージェンス}
        {canonical ダイバージェンス}[canonical ダイバージェンス]
    と呼ぶ。
\end{propdef}

\begin{proof}
    $(p, q) \in \calU$とし、
    $(U, \theta, \eta), (U', \theta', \eta')$を
    それぞれ条件をみたす双対アファインチャート、
    $(\psi, \varphi), (\psi', \varphi')$を
    それぞれの双対ポテンシャルとする。
    $(p, q) \in \calU$ゆえ
    $p, q$を含む$g$-凸集合が存在するから、
    $p$から$q$への$M$内の$\nabla^g$-測地線$\gamma$がただひとつ存在する。
    ここで
    $U, U'$は$p, q$を含む$g$-凸開集合を含んでいたから、
    $U \cap U'$は$\gamma$の像を含む。
    このとき$U \cap U'$の連結成分$C$であって
    $\gamma$の像を含むものがただ1つ存在する。

    $C$の連結性より
    $\psi'(q) - \psi(q)
        = (\text{$C$上の定数})
        = \psi'(p) - \psi(p)$
    が成り立つ。
    よって
    \begin{alignat}{1}
        \psi'(q) + \varphi'(p) - \myangle{\theta'(q)}{\eta'(p)}
            &=
                \psi'(q) - \psi'(p) - \myangle{\theta'(q) - \theta'(p)}{\eta'(p)}
                \\
            &=
                \psi(q) - \psi(p) - \myangle{\theta'(q) - \theta'(p)}{\eta'(p)}
    \end{alignat}
    が成り立つ。
    あとは
    $\myangle{\theta'(q) - \theta'(p)}{\eta'(p)}
        = \myangle{\theta(q) - \theta(p)}{\eta(p)}$
    を示せばよい。

    $C$の連結性より、
    組$(A = (A_i^j)_{i, j}, b) \in \GL_n(\R) \times \R^n$であって
    $\theta'(r) = A \theta(r) + b \; (\forall r \in C)$
    をみたすものがただ1組存在する。
    よって任意の$r \in C$に対し
    \begin{alignat}{1}
        \eta_i(r)
            &=
                \deldel[\psi]{\theta^i}(r)
                \qquad
                (\because \text{$d\psi = \eta_i d\theta^i$})
                \\
            &=
                \deldel[\psi']{\theta^i}(r)
                \qquad
                (\because \text{$\psi' - \psi$は$C$上定数})
                \\
            &=
                \deldel[\theta'^j]{\theta^i}(r)
                \deldel[\psi']{\theta'^j}(r)
                \\
            &=
                A_i^j \eta'_j(r)
                \qquad
                (\because \text{$d\psi' = \eta'_j d\theta'^j$})
                \\
        \therefore \eta(r)
            &=
                A \eta'(r)
    \end{alignat}
    が成り立つ。
    さらに任意の$r \in C$に対し
    \begin{alignat}{1}
        \theta'^i(r)
            &=
                \deldel[\varphi']{\eta'_i}(r)
                \qquad
                (\because \text{$d\varphi' = \theta'^i d\eta'_i$})
                \\
            &=
                \deldel[\varphi]{\eta'_i}(r)
                \qquad
                (\because \text{$\varphi' - \varphi$は$C$上定数})
                \\
            &=
                \deldel[\eta_j]{\eta'_i}(r)
                \deldel[\varphi]{\eta_j}(r)
                \\
            &=
                A_j^i \theta^j(r)
                \qquad
                (\because \text{$d\varphi = \theta^j d\eta_j, \; \eta = A \eta'$})
                \\
        \therefore \theta'(r)
            &=
                A \theta(r)
    \end{alignat}
    が成り立つ。
    したがって
    \begin{equation}
        \myangle{\theta'(q) - \theta'(p)}{\eta'(p)}
            =
                \myangle{A(\theta(q) - \theta(p))}{A^{-1} \eta(p)}
            =
                \myangle{\theta(q) - \theta(p)}{\eta(p)}
    \end{equation}
    が示された。
\end{proof}

\begin{proposition}[canonical ダイバージェンスの性質]
    $(g, \nabla^*, \nabla)$の canonical ダイバージェンスを$D^*$として
    \begin{enumerate}
        \item $D$は{\smooth}関数である。
        \item $D(p \| q) \ge 0$
        \item $D(p \| q) = 0 \iff p = q$
        \item $D(p \| q) = D^*(q \| p)$
    \end{enumerate}
\end{proposition}

\begin{proof}
    \uline{(1)} \quad
    局所的な{\smooth}性を示せばよい。
    $(p, q) \in \calU$とし、
    $(U, \theta, \eta)$を条件をみたす双対アファインチャートとすれば、
    $(p, q)$の近傍$U \times U$上で$D$は{\smooth}である。

    \uline{(2), (3)} \quad
    $\psi$の$\nabla$-凸性あるいは
    $\varphi$の$\nabla^*$-凸性より従う。

    \uline{(4)} \quad
    定義より明らか。
\end{proof}

\end{document}