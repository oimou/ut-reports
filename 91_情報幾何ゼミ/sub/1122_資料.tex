\documentclass[report]{jlreq}
\usepackage{global}
\usepackage{./local}
\subfiletrue
\def\assetspath{../}
%\makeindex
\chead{2023/11/22}
\begin{document}

% ============================================================
%
% ============================================================

% ------------------------------------------------------------
%
% ------------------------------------------------------------
\section*{振り返りと導入}

前回はKLダイバージェンスの双対平坦多様体への一般化を考え始めた。
本稿では次のことを行う:
\begin{itemize}
    \item 双対平坦構造の canonical ダイバージェンスを定義する。
    \item 双対平坦構造からシンプレクティック構造が定まることをみる。
\end{itemize}

% ------------------------------------------------------------
%
% ------------------------------------------------------------
\section{双対平坦構造とシンプレクティック構造}

\begin{proposition}[双対平坦構造のシンプレクティック構造]
    $M$を多様体、
    $(g, \nabla, \nabla^*)$を$M$上の双対平坦構造、
    $D \colon \calU \to \R$を canonical ダイバージェンス、
    $\omega_0 \in \Omega^2(T^\vee M)$を
    $T^\vee M$上の自然シンプレクティック構造とする。
    写像$d_1 D \colon \calU \to T^\vee M$を
    第1成分に関する微分、すなわち
    $d_1 D \coloneqq D(\tdeldel{x^i} \|) \, dx^i$
    で定め、$\calU$上の2-形式
    $\omega \in \Omega^2(\calU)$を
    $\omega \coloneqq (d_1 D)^* (\omega_0)$
    で定める。
    このとき次が成り立つ:
    \begin{enumerate}
        \item $M$の任意の局所座標$x = (x_i)_i$に対し、
            $x^* \coloneqq x$とおいて
            $\calU$の局所座標$(x, x^*) = (x^1, \dots, x^n, x^{*1}, \dots, x^{*n})$
            を定めると、
            $\omega$の成分表示は
            \begin{equation}
                \omega
                    =
                        D(\tdeldel{x^i} \| \tdeldel{x^{*j}}) \,
                        dx^i \wedge dx^{*j}
            \end{equation}
            となる。
        \item $\omega$は$\calU$上のシンプレクティック形式である。
    \end{enumerate}
\end{proposition}

\begin{proof}
    \uline{(1)} \quad
    前回示した。

    \uline{(2)} \quad
    $\calU$の局所座標として$(\eta, \theta^*)$をとれば
    $D(\tdeldel{\eta_i} \| \tdeldel{\theta^{*j}}) = - \delta_j^i$となるから
    $d_1 D$ははめ込みである。
    よって$\omega$は$\calU$上のシンプレクティック形式である。
\end{proof}

\begin{example}[$\R^n$の場合]
    \TODO{}
\end{example}

% ------------------------------------------------------------
%
% ------------------------------------------------------------
\section*{今後の予定}

\begin{itemize}
    \item 双対平坦構造のシンプレクティック構造と双対アファイン座標
\end{itemize}

% ------------------------------------------------------------
%
% ------------------------------------------------------------
\section*{参考文献}

%Legendre 変換については
%\cite{niculescu_convex_2018}
%を参考にした。
%期待値パラメータに関しては
%\cite{wainwright_graphical_2007}を参考にした。

\nocite{amari_information_2016}
\nocite{_bayes_2020}

{
    \renewcommand{\bibsection}{}
    \bibliographystyle{amsalpha}
    \bibliography{./bibliography,../../mybibliography}
}

% ------------------------------------------------------------
%
% ------------------------------------------------------------
\newpage
\appendix
\renewcommand\thesection{\Alph{section}}
\setcounter{section}{0}
\section{付録}

\subsection{\cref{def:canonical-divergence-domain}の条件(i), (ii)について}

$M$を多様体、
$g$を$M$上のRiemann 計量、
$\nabla$を$M$上のアファイン接続とする。

\begin{definition}[simple chain (ここだけの用語)]
    $X$を位相空間とする。
    $X$の開集合の有限列$(U_i)_{i = 1}^N$が
    \term{simple chain}
        {simple chain}[simple chain]
    であるとは、
    $U_i \cap U_j \neq \emptyset \;
        \iff |i - j| \le 1$
    が成り立つことをいう。
    さらにすべての$U_i \cap U_{i + 1}$が連結のとき
    \term{very simple chain}
        {very simple chain}[very simple chain]
    という。
\end{definition}

\begin{lemma}
    $\nabla$-アファインチャートの列
    $(U_i)_{i = 1}^N$が
    very simple chain ならば、
    $\textstyle \bigcup_{i = 1}^N U_i$を定義域とする
    $\nabla$-アファイン座標が存在する。
\end{lemma}

\begin{proof}
    $U_1 \cap U_2$は連結であり、
    2つの座標はアファイン変換で移り合うから、
    それに応じて$U_2$上の座標を調整すれば
    $U_1 \cup U_2$上の$\nabla$-アファイン座標が得られる。
    以下同様にして$U_1 \cup \cdots \cup U_N$上の
    $\nabla$-アファイン座標が得られる。
\end{proof}

\begin{proposition}
    $\gamma \colon I \to M$が単射な$\nabla$-測地線ならば、
    $\gamma(I)$を覆う
    単連結な$\nabla$-アファインチャートが存在する。
\end{proposition}

\begin{proof}
    \TODO{要確認}
    $\gamma(I)$の各点のまわりの$\nabla$-アファインチャートを集めて
    $\gamma(I)$の開被覆$\calU$を作る。
    Lebesgue 数の補題より、
    実数列$0 = t_0 < t_1 < \cdots < t_N = 1$が存在して
    各$S_i \coloneqq \gamma([t_{i - 1}, t_i])$は
    ある$U_i \in \calU$に含まれる。
    $\gamma$の単射性より、
    ある$\varepsilon > 0$であって
    $(U(S_i, \varepsilon))_{i = 1}^N$が
    very simple chain かつ
    $U(S_i, \varepsilon) \subset U_i$
    となるものが存在する
    (ただし$U(S_i, \varepsilon)$は Riemann 距離に関する$\varepsilon$-近傍)。
    そこで$\textstyle U \coloneqq \bigcup_{i = 1}^N U(S_i, \varepsilon)$とおくと、
    補題より$U$上の$\nabla$-アファイン座標$\theta$が存在する。
    $\theta(\gamma(I))$が$\theta(U)$内の線分であることに注意すると、
    $\theta(\gamma(I))$の十分小さい近傍$V$をとれば、
    $\theta^{-1}(V)$は$\gamma(I)$を覆う単連結な$\nabla$-アファインチャートとなる。
\end{proof}

\subsection{\cref{prop:canonical-divergence-domain}の証明}

\begin{proof}
    $p \in M$を固定し、
    $(p, p)$の$M \times M$におけるある開近傍が
    $\calW$に含まれることを示せばよい。
    そのような開近傍を次のように構成する。

    まず$\nabla$の平坦性より
    $p$のまわりの$\nabla$-アファインチャート$(U, \theta)$が存在する。
    $p$の$M$における
    (計量$g$から定まる距離に関する)
    $3r$-近傍が$U$に含まれるように$r > 0$をとり、
    $p$の$M$における$r$-近傍を$U'$とおく。
    さらに$\theta(p)$の$\R^n$における$\eps$-近傍$V_\eps$が
    $\theta(U')$に含まれるように$\eps > 0$をとる。
    $U_\eps \coloneqq \theta^{-1}(V_\eps)$とおくと
    $(p, p)$の$U_\eps \times U_\eps$は$M \times M$における
    開近傍である。

    以下$U_\eps \times U_\eps \subset \calW$を示す。
    すなわち、$(a, b) \in U_\eps \times U_\eps$として
    $(a, b) \in \calW$を示す。
    $U_\eps$は$\nabla$-凸ゆえ、
    $a, b$を結ぶ$U_\eps$内の$\nabla$-測地線$\gamma$が存在する。
    このとき$\gamma$はとくに$U$内の$\nabla$-測地線でもあるが、
    $U$は$\nabla$-アファインチャートだから
    $\gamma$は$a, b$を結ぶ$U$内の唯一の$\nabla$-測地線である。
    $U'$の定め方から、
    $a, b$を結ぶ ($M$内の) 任意の$\nabla$-測地線は
    $\gamma$より真に長いか$\gamma$自身である\TODO{怪しい}。
    したがって、
    $a, b$を結ぶ ($M$内の) $\nabla$-測地線のうち最短なものは
    ただひとつ存在し、それは$\gamma$である。
    よって$(a, b)$は条件(i)をみたす。
    さらに$U_\eps$は$\gamma$の像を覆う
    単連結$\nabla$-アファインチャートだから、
    $(a, b)$は条件(ii)をみたす。
    したがって$(a, b) \in \calW$である。
\end{proof}


\end{document}