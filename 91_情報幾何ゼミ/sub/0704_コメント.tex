\documentclass[report]{jlreq}
\usepackage{global}
\usepackage{./local}
\subfiletrue
\def\assetspath{../}
%\makeindex
\chead{2023/07/04}
\begin{document}

% ============================================================
%
% ============================================================

発表中にコメントがあった事柄を整理する。

\begin{axiom}[分出公理 (axiom schema of specification)]
    $x$を自由変項にもつ任意の論理式$\phi(x)$に対し
    \begin{equation}
        \forall x
        \exists y
        \forall z
        \mybracket{
            z \in y
            \leftrightarrow
            z \in x \land \phi(z)
        }.
    \end{equation}
    この公理により、
    $x$を自由変項にもつ任意の論理式$\phi(x)$と
    任意の集合$A$に対し、
    $\phi(a)$を満たす元$a \in A$全体の集合がただひとつ存在する。
    これを$\{ a \in A \mid \phi(a) \}$と書く。
\end{axiom}

\begin{axiom}[置換公理 (axiom schema of replacement)]
    \TODO{
        $\phi(x, y)$を1変項関数論理式とする。
        任意の集合$A$に対し、$A$の元$a$の$\phi$による
        $\llangle\text{像}\rrangle$であるような
        $z$の全体は集合である。
    }
\end{axiom}

\begin{proposition}
    有限集合上の full support な確率分布の族について、
    $n = 3$のとき、
    $\nabla^{(\alpha)}$のスカラー曲率$S^{(\alpha)} \; (\alpha \in \R)$は
    \begin{equation}
        S^{(\alpha)}(p)
            =
                \frac{1 - \alpha}{p_3} \myparen{
                    - 2(p_1^2 + p_2^2)
                    + 3(p_1 + p_2)
                    - 2
                }
            =
                - \frac{2(1 - \alpha)}{p_3} \myparen{
                    (p_1 - 3/4)^2
                    + (p_2 - 3/4)^2
                    - 1/8
                }
    \end{equation}
    をみたす。
    さらに、すべての$p \in \calP$に対し
    \begin{equation}
        \sgn S^{(\alpha)}(p)
            = \begin{cases}
                +1 & \alpha > 1 \\
                0  & \alpha = 1 \\
                -1 & \alpha < 1
            \end{cases}
    \end{equation}
    が成り立つ。
\end{proposition}

\begin{proof}
    スカラー曲率が命題の主張の式で表せることは直接計算によりわかる。
    さらに$0 < p_1 < 1, \; 0 < p_2 < 1, \; 0 < p_1 + p_2 < 1$より
    $(p_1 - 3/4)^2 + (p_2 - 3/4)^2 - 1/8 > 0$だから、
    $S^{(\alpha)}(p)$の符号は命題の主張のようになる。
\end{proof}



\end{document}