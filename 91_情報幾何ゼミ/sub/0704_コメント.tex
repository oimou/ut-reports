\documentclass[report]{jlreq}
\usepackage{global}
\usepackage{./local}
\subfiletrue
\def\assetspath{../}
%\makeindex
\chead{2023/07/04}
\begin{document}

% ============================================================
%
% ============================================================

発表中にコメントがあった事柄を整理する。

\begin{axiom}[分出公理 (axiom schema of specification)]
    $x$を自由変項にもつ任意の論理式$\phi(x)$に対し
    \begin{equation}
        \forall x
        \exists y
        \forall z
        \mybracket{
            z \in y
            \leftrightarrow
            z \in x \land \phi(z)
        }.
    \end{equation}
    この公理により、
    $x$を自由変項にもつ任意の論理式$\phi(x)$と
    任意の集合$A$に対し、
    $\phi(a)$を満たす元$a \in A$全体の集合がただひとつ存在する。
    これを$\{ a \in A \mid \phi(a) \}$と書く。
\end{axiom}

\begin{axiom}[置換公理 (axiom schema of replacement)]
    \TODO{
        $\phi(x, y)$を1変項関数論理式とする。
        任意の集合$A$に対し、$A$の元$a$の$\phi$による
        $\llangle\text{像}\rrangle$であるような
        $z$の全体は集合である。
    }
\end{axiom}

\begin{proposition}
    有限集合上の full support な確率分布の族について、
    $n = 3$のとき、
    $\nabla^{(\alpha)}$の
    Ricci 曲率テンソル$\Ric^{(\alpha)}$の
    $(\mu, \sigma)$-座標に関する成分は
    \begin{alignat}{1}
        \Ric^{(\alpha)}_{11}
            =
                \frac{p_1 (1 - p_1) (1 - a^2)}{2},
                \qquad
        \Ric^{(\alpha)}_{12} = \Ric^{(\alpha)}_{12}
            =
                \frac{p_1 p_2 (1 - a^2)}{2},
                \qquad
        \Ric^{(\alpha)}_{22}
            =
                \frac{p_2 (1 - p_2) (1 - a^2)}{2}
    \end{alignat}
    をみたし、
    $g$に関するスカラー曲率$S^{(\alpha)} \; (\alpha \in \R)$は
    \begin{equation}
        S^{(\alpha)}(p)
            = 1 - a^2
    \end{equation}
    をみたす。
\end{proposition}

\begin{proof}
    直接計算によりわかる。
\end{proof}



\end{document}