\documentclass[report]{jlreq}
\usepackage{global}
\usepackage{./local}
\subfiletrue
\def\assetspath{../}
%\makeindex
\chead{2023/07/11}
\begin{document}

% ============================================================
%
% ============================================================

% ------------------------------------------------------------
%
% ------------------------------------------------------------
\section*{振り返りと導入}

前回は指数型分布族の具体例の計算を行った。
本稿では次のことを行う:
\begin{itemize}
    \item 双対構造を定義し、とくに指数型分布族の$\alpha$-接続の性質を調べる。
    \item Legendre 変換を定義する。
    \item 指数型分布族の期待値パラメータを定義する。
\end{itemize}

% ------------------------------------------------------------
%
% ------------------------------------------------------------
\section{双対構造}

\begin{definition}[双対構造]
    $M$を多様体とする。
    $M$上の
    Riemann 計量$g$と
    アファイン接続$\nabla, \nabla^*$の組
    $(g, \nabla, \nabla^*)$
    が$M$上の
    \term{双対構造}[dualistic structure]
        {双対構造}[そうついこうぞう]
    であるとは、
    すべての$X, Y, Z \in \frakX(M)$に対し
    \begin{equation}
        X(g(Y, Z))
            =
                g(\nabla_X Y, Z) + g(Y, \nabla^*_X Z)
    \end{equation}
    が成り立つことをいう。
    このとき、
    $\nabla, \nabla^*$はそれぞれ$g$に関する$\nabla^*, \nabla$の
    \term{双対接続}[dual connection]
        {双対接続}[そうついせつぞく]
    であるという。

    さらに$\nabla, \nabla^*$がいずれも$M$上平坦であるとき、
    $(g, \nabla, \nabla^*)$は
    \term{双対平坦}[dually flat]
        {双対平坦}[そうついへいたん]
    であるという。
    双対平坦な双対構造を
    \term{双対平坦構造}[dually flat structure]
        {双対平坦構造}[そうついへいたんこうぞう]
    という。
\end{definition}

\begin{proposition}[双対接続の存在と一意性]
    \label[proposition]{prop:dual-connection-existence-uniqueness}
    $M$を多様体、
    $g$を$M$上のRiemann 計量、
    $\nabla$を$M$上のアファイン接続とする。
    このとき、
    $g$に関する$\nabla$の双対接続がただひとつ存在する。
\end{proposition}

\begin{proof}
    証明は付録に記した。
\end{proof}

指数型分布族の$\alpha$-接続について考える。
以降、$\calP$を可測空間$\calX$上の open な指数型分布族、
$\nabla$を$\calP$上の自然な平坦アファイン接続、
$g$を$\calP$上の Fisher 計量、
$S, A$をそれぞれ$(0, 3), (1, 2)$型の Amari-Chentsov テンソル、
$\nabla^{(\alpha)} \; (\alpha \in \R)$を$\alpha$-接続とする。

\begin{proposition}[曲率のACテンソルによる表示]
    \label[proposition]{prop:curvature-AC-tensor}
    $\alpha \in \R$、
    $R^{(\alpha)}$を$\nabla^{(\alpha)}$の
    $(1, 3)$-曲率テンソルとする。
    このとき、
    $\calP$の任意の$\nabla$-アファイン座標に関し、
    $R^{(\alpha)}$の成分は
    \begin{equation}
        {R^{(\alpha)}}_{ijk}^{\hphantom{ijk}l}
            =
                - \frac{1 - \alpha^2}{4}
                \myparen{
                    A_{jk}^{\hphantom{jk}m} A_{im}^{\hphantom{im}l}
                    - A_{ik}^{\hphantom{ik}m} A_{jm}^{\hphantom{jm}l}
                }
    \end{equation}
    となる。
\end{proposition}

\begin{proof}
    \url{0613_資料.pdf}命題1.12の式
    \begin{equation}
        {R^{(\alpha)}}_{ijk}^{\hphantom{ijk}l}
            = \frac{1 - \alpha}{2} \myparen{
                \del_i A_{jk}^{\hphantom{jk}l}
                -
                \del_j A_{ik}^{\hphantom{ik}l}
            }
            + \myparen{\frac{1 - \alpha}{2}}^2
            \myparen{
                A_{jk}^{\hphantom{jk}m} A_{im}^{\hphantom{im}l}
                -
                A_{ik}^{\hphantom{ik}m} A_{jm}^{\hphantom{jm}l}
            }
    \end{equation}
    を変形する。
    \begin{alignat}{1}
        \del_i A_{jk}^{\hphantom{jk}l}
        -
        \del_j A_{ik}^{\hphantom{ik}l}
            &=
                \del_i (
                    g^{la} S_{jka}
                )
                -
                \del_j (
                    g^{la} S_{ika}
                )
                \\
            &=
                \del_i (g^{la})
                S_{jka}
                +
                g^{la}
                \del_i S_{jka}
                -
                \del_j (g^{la})
                S_{ika}
                -
                g^{la}
                \del_j S_{ika}
                \\
            &=
                \del_i (g^{la})
                S_{jka}
                -
                \del_j (g^{la})
                S_{ika}
    \end{alignat}
    である。
    右辺第1項について、
    $0
        =
            \del_i \delta_m^l
        =
            \del_i (g^{la} g_{ma})
        =
            \del_i (g^{la}) g_{ma}
            +
            g^{lb} \del_i (g_{mb})$
    より
    $\del_i (g^{la})
        =
            - g^{ma} g^{lb} \del_i (g_{mb})$
    だから
    \begin{alignat}{1}
        \del_i (g^{la})
            S_{jka}
            &=
                -
                g^{ma}
                g^{lb}
                \del_i (g_{mb})
                S_{jka}
                \\
            &=
                -
                g^{ma}
                g^{lb}
                S_{imb}
                S_{jka}
                \\
            &=
                -
                A_{im}^{\hphantom{im}l}
                A_{jk}^{\hphantom{jk}m}
    \end{alignat}
    同様にして
    \begin{equation}
        \del_j (g^{la})
            S_{ika}
                =
                    -
                    A_{jm}^{\hphantom{jm}l}
                    A_{ik}^{\hphantom{ik}m}
    \end{equation}
    を得る。
    したがって
    $\del_i A_{jk}^{\hphantom{jk}l} - \del_j A_{ik}^{\hphantom{ik}l}
        =
            - A_{im}^{\hphantom{im}l} A_{jk}^{\hphantom{jk}m}
            + A_{jm}^{\hphantom{jm}l} A_{ik}^{\hphantom{ik}m}$
    だから
    \begin{equation}
        {R^{(\alpha)}}_{ijk}^{\hphantom{ijk}l}
            =
                \myparen{
                    - \frac{1 - \alpha}{2}
                    + \myparen{\frac{1 - \alpha}{2}}^2
                }
                \myparen{
                    A_{jk}^{\hphantom{jk}m} A_{im}^{\hphantom{im}l}
                    - A_{ik}^{\hphantom{ik}m} A_{jm}^{\hphantom{jm}l}
                }
            =
                -
                \frac{1 - \alpha^2}{4}
                \myparen{
                    A_{jk}^{\hphantom{jk}m} A_{im}^{\hphantom{im}l}
                    - A_{ik}^{\hphantom{ik}m} A_{jm}^{\hphantom{jm}l}
                }
    \end{equation}
    となる。
\end{proof}

\begin{corollary}
    \label[corollary]{corollary:flatness}
    ~
    \begin{enumerate}
        \item $\forall \alpha \in \R$に対し
            $R^{(\alpha)}
                =
                    (1 - \alpha^2)
                    R^{(0)}
                =
                    R^{(-\alpha)}$.
        \item 次は同値:
            \begin{enumerate}
                \item すべての$\alpha \in \R$に対し、
                    $\nabla^{(\alpha)}$は平坦である。
                \item ある$\alpha \neq \pm 1$が存在し、
                    $\nabla^{(\alpha)}$は平坦である。
            \end{enumerate}
    \end{enumerate}
\end{corollary}

\begin{proof}
    \uline{(1)} \quad
    \cref{prop:curvature-AC-tensor}
    より明らか。

    \uline{(2)} \quad
    まず(1)より次は同値である:
    \begin{enumerate}[label=(\alph*)']
        \item $\forall \alpha \in \R$に対し
            $R^{(\alpha)} = 0$.
        \item $\exists \alpha \neq \pm 1$
            \, s.t. \,
            $R^{(\alpha)} = 0$.
    \end{enumerate}
    さらに$\alpha$-接続はすべて torsion-free だから、
    曲率が$0$であることと平坦であることは同値である。
\end{proof}

\begin{theorem}[$\alpha$-接続による双対構造]
    任意の$\alpha \in \R$に対し、
    3つ組$(g, \nabla^{(\alpha)}, \nabla^{(-\alpha)})$は
    $\calP$上の双対構造となる。
    さらに、
    $\alpha = \pm 1$ならば
    $(g, \nabla^{(\alpha)}, \nabla^{(-\alpha)})$は
    双対平坦である。
\end{theorem}

\begin{proof}
    双対構造であることは、
    すべての$X, Y, Z \in \frakX(\calP)$に対し
    \begin{alignat}{1}
        g(\nabla^{(\alpha)}_X Y, Z)
            + g(Y, \nabla^{(-\alpha)}_X Z)
            &=
                g(\nabla^{g}_X Y, Z)
                - \frac{\alpha}{2} S(X, Y, Z)
                + g(Y, \nabla^{g}_X Z)
                + \frac{\alpha}{2} S(X, Z, Y)
                \\
            &=
                g(\nabla^{g}_X Y, Z)
                + g(Y, \nabla^{g}_X Z)
                \\
            &=
                X(g(Y, Z))
    \end{alignat}
    より従う。
    $\alpha = \pm 1$で双対平坦となることは
    \cref{corollary:flatness}
    よりわかる。
\end{proof}

% ------------------------------------------------------------
%
% ------------------------------------------------------------
\section{Legendre 変換}

本節では
$W$を有限次元$\R$-ベクトル空間とする。

\begin{definition}[Legendre 変換]
    $U \subset W$を開集合、
    $f \colon U \to \R$を$C^\infty$関数であって
    $\nabla f \colon U \to W^\vee$が単射であるものとする。
    関数
    \begin{equation}
        f^\vee \colon U' \to \R,
            \quad
            y
            \mapsto
            \myangle{(\nabla f)^{-1}(y)}{y} - f((\nabla f)^{-1}(y))
            \quad
            \text{where}
            \quad
            U' \coloneqq \nabla f(U)
    \end{equation}
    を$f$の
    \term{Legendre 変換}[Legendre transform]
        {Legendre 変換}[Legendre へんかん]
    という。
\end{definition}

\begin{example}[Legendre 変換の例]
    前回 (\url{0704_資料.pdf}) 扱った具体例について
    対数分配関数の Legendre 変換を計算してみる。
    \begin{itemize}
        \item \uline{Bernoulli 分布族 (i.e. 2元集合上の full support な確率分布の族):} \quad
            対数分配関数は
            $\psi \colon \R \to \R, \; \theta \mapsto \log (1 + \exp \theta)$
            であった。
            よって
            $\nabla \psi(\theta)
                =
                    \frac{\exp \theta}{1 + \exp \theta}$
            であり、
            $(\nabla \psi)^{-1}(\eta)
                =
                    \log \eta - \log (1 - \eta)$
            である。
            したがって
            $\psi^\vee(\eta)
                =
                    \eta \log \eta
                    + (1 - \eta) \log (1 - \eta)$
            である。
        \item \uline{正規分布族:} \quad
            対数分配関数は
            $\psi \colon \R \times \R_{< 0} \to \R, \;
                \theta
                \mapsto
                - \frac{(\theta^1)^2}{4 \theta^2}
                - \frac{1}{2} \log (- \theta^2)
                + \frac{1}{2} \log \pi$
            であった。
            よって
            $\nabla \psi(\theta)
                =
                    \begin{pmatrix}
                        - \frac{\theta^1}{2 \theta^2}
                        &
                        \frac{(\theta^1)^2}{4 (\theta^2)^2} - \frac{1}{2 \theta^2}
                    \end{pmatrix}$
            であり、
            $(\nabla \psi)^{-1}(\eta)
                =
                    \frac{1}{\eta_2 - (\eta_1)^2}
                    \begin{pmatrix}
                        \eta_1
                        \\
                        - 1/2
                    \end{pmatrix}$
            である。
            よって
            $\psi^\vee(\eta)
                =
                    - \frac{1}{2}
                    \myparen{
                        1 + \log 2\pi
                        + \log(\eta_2 - (\eta_1)^2)
                    }$
            である。
    \end{itemize}
\end{example}

本稿では、とくに次の状況を考えることになる。

\begin{proposition}
    \label[proposition]{prop:Legendre-transform-properties}
    $U \subset W$を凸開集合、
    $f \colon U \to \R$を$C^\infty$関数であって
    $\Hess f$が$U$上各点で (対称であることも含む意味で) 正定値であるものとする。
    このとき、次が成り立つ:
    \begin{enumerate}
        \item $\nabla f$は局所微分同相である。
            とくに$U' \coloneqq \nabla f(U)$は$W^\vee$の開集合である。
        \item $\nabla f \colon U \to U'$は微分同相である。
            とくに$\nabla f$は単射である。
    \end{enumerate}
    したがって$f^\vee$が定義でき、$f^\vee$は次をみたす:
    \begin{enumerate}
        \setcounter{enumi}{2}
        \item $f^\vee \colon U' \to \R$は$C^\infty$関数である。
        \item $\nabla f^\vee = (\nabla f)^{-1}$が成り立つ。
            とくに$\nabla f^\vee$は単射である。
        \item 各$y \in U'$に対し
            $(\Hess f^\vee)_y = ((\Hess f)_x)^{-1}$が成り立つ
            (ただし$x \coloneqq (\nabla f)^{-1}(y)$)。
            とくに$(\Hess f^\vee)_y$は正定値である。
    \end{enumerate}
\end{proposition}

\begin{proof}
    \uline{(1)} \quad
    命題の仮定より$\Hess f$は$U$上各点で正定値だから、
    $\nabla f$の微分は各点で線型同型である。
    したがって$\nabla f$は局所微分同相であり、
    とくに開写像である。
    よって$U' = \nabla f(U)$は$W^\vee$の開集合である。

    \uline{(2)} \quad
    $u, \wt{u} \in U, \; u \neq \wt{u}$を固定し、
    $[0, 1]$を含む$\R$の開区間$I$であって、
    すべての$t \in I$に対し
    $(1 - t)u + t\wt{u}$が
    $U$に属するようなものをひとつ選ぶ
    ($U$は$W$の凸開集合だからこれは可能)。
    さらに
    $\varphi \colon I \to U, \; t \mapsto f((1 - t)u + t\wt{u})$と定めると、
    平均値定理より、
    ある$\tau \in (0, 1)$が存在して
    \begin{alignat}{1}
        \myangle{
            \nabla f(\wt{u}) - \nabla f(u)
        }{
            \wt{u} - u
        }
            &=
                \varphi'(1) - \varphi'(0)
                \\
            &=
                \varphi''(\tau)
                \qquad
                (\text{平均値定理})
                \\
            &=
                \myangle{
                    (\Hess f)_{(1 - \tau)u + \tau\wt{u}}
                }{
                    (\wt{u} - u)^2
                }
                \\
            &>
                0
                \qquad
                (\text{$\Hess f$は正定値})
    \end{alignat}
    が成り立つ。
    よって$\nabla f(\wt{u}) \neq \nabla f(u)$である。
    したがって$\nabla f$は単射である。
    このことと (1) より
    $\nabla f \colon U \to U'$は微分同相である。

    \uline{(3)} \quad
    $\nabla f \colon U \to U'$が微分同相ゆえに
    $(\nabla f)^{-1} \colon U' \to U$は{\smooth}だから、
    $f^\vee$は{\smooth}関数である。

    \uline{(4)} \quad
    $f^\vee$の定義式を$\nabla$で微分すると、
    すべての$y \in U'$に対し
    \begin{alignat}{1}
        (\nabla f^\vee)(y)
            &=
                (\nabla f)^{-1}(y)
                + \myangle{
                    y
                }{
                    \nabla
                    (\nabla f)^{-1}
                    (y)
                }
                - \myangle{
                    (\nabla f)(
                        (\nabla f)^{-1}
                        (y)
                    )
                }{
                    \nabla
                    (\nabla f)^{-1}
                    (y)
                }
            =
                (\nabla f)^{-1}(y)
    \end{alignat}
    が成り立つ。
    よって$(\nabla f)^{-1} = \nabla f^\vee$である。

    \uline{(5)} \quad
    (4)より
    \begin{alignat}{1}
        (\Hess f^\vee)_y
            &=
                d(\nabla f^\vee)_y
                \\
            &=
                d((\nabla f)^{-1})_y
                \\
            &=
                (d(\nabla f)_x)^{-1}
                \\
            &=
                ((\Hess f)_x)^{-1}
    \end{alignat}
    となる。
\end{proof}

\begin{corollary}[Legendre 変換の対合性]
    $f^{\vee \vee} = f$.
\end{corollary}

\begin{proof}
    Legendre 変換の定義より、
    すべての$x \in U$に対し
    \begin{alignat}{1}
        f^{\vee\vee}(x)
            &=
                \myangle{
                    x
                }{
                    (\nabla f^\vee)^{-1}
                    (x)
                }
                - f^\vee(
                    (\nabla f^\vee)^{-1}
                    (x)
                )
                \\
            &=
                \myangle{
                    x
                }{
                    \nabla f
                    (x)
                }
                - f^\vee(
                    \nabla f
                    (x)
                )
                \qquad
                (\nabla f^\vee = (\nabla f)^{-1})
                \\
            &=
                \myangle{
                    x
                }{
                    \nabla f
                    (x)
                }
                - \myparen{
                    \myangle{
                        \nabla f
                        (x)
                    }{
                        (\nabla f)^{-1}(
                            \nabla f
                            (x)
                        )
                    }
                    - f(
                        (\nabla f)^{-1}(
                            \nabla f
                            (x)
                        )
                    )
                }
                \\
            &=
                f(x)
    \end{alignat}
    が成り立つ。
    よって$f^{\vee\vee} = f$である。
\end{proof}

% ------------------------------------------------------------
%
% ------------------------------------------------------------
\section{期待値パラメータ}

$\calP$を可測空間$\calX$上の open な指数型分布族、
$\nabla$を$\calP$上の自然な平坦アファイン接続、
$g$を$\calP$上の Fisher 計量、
$S, A$をそれぞれ$(0, 3), (1, 2)$型の Amari-Chentsov テンソル、
$\nabla^{(\alpha)} \; (\alpha \in \R)$を$\alpha$-接続とする。

以降、
$\calP$の最小次元実現$(V, T, \mu)$をひとつ固定し、
この実現に関する対数分配関数を$\psi \colon \wt{\Theta} \to \R$とおく。

\begin{propdef}[期待値パラメータ空間]
    \label[propdef]{propdef:mean-parameter-space}
    集合
    \begin{equation}
        \calM
            \coloneqq
                \mybrace{
                    E_p[T] \in V
                    \mid
                    p \in \calP
                }
    \end{equation}
    は$V$の開部分多様体となり、
    写像$\eta \colon \calP \to \calM, \; p \mapsto E_p[T]$
    は微分同相写像となる。

    $\calM$を
    $(V, T, \mu)$に関する$\calP$の
    \term{期待値パラメータ空間}[mean parameter space]
        {期待値パラメータ空間}[きたいちぱらめーたくうかん]
    といい、
    $\eta$を
    $(V, T, \mu)$に関する$\calP$上の
    \term{期待値パラメータ座標}[mean parameter coordinates]
        {期待値パラメータ座標}[きたいちぱらめーたざひょう]
    という。
\end{propdef}

この証明には次の2つの事実を使う。

\begin{fact}[$\psi$の微分は十分統計量の期待値]
    \label[fact]{fact:mean-parameter-space}
    写像$\nabla \psi \colon \Theta \to V^{\vee\vee} = V$は
    \begin{equation}
        (\nabla \psi)(\theta(p))
            =
                \eta(p)
                \qquad
                (p \in \calP)
    \end{equation}
    をみたす。
    したがって
    $\calM = \nabla \psi(\Theta)$である。
    \qed
\end{fact}

\begin{fact}
    \label[fact]{fact:convexity-of-interior}
    位相ベクトル空間の凸集合の内部は凸集合である。
    \qed
\end{fact}

\begin{proof}[\cref{propdef:mean-parameter-space}の証明]
    まず$\calM$が$V$の開部分多様体となることを示す。
    $\psi$を$\Int \wt{\Theta}$上の関数とみなすと、
    \cref{fact:convexity-of-interior}とあわせて
    $\psi$は\cref{prop:Legendre-transform-properties}の前提をみたすから、
    \cref{prop:Legendre-transform-properties} (1)より
    $\nabla \psi \colon \Int \wt{\Theta} \to V^{\vee\vee} = V$は
    局所微分同相、とくに開写像でもある。
    したがって$\nabla \psi(\Int \wt{\Theta})$は$V$の開部分多様体となる。
    さらに$\Theta$は$\Int \wt{\Theta}$の開集合だから、
    $\nabla \psi(\Theta)$は$\nabla \psi(\Int \wt{\Theta})$の開部分多様体となる。
    このことと\cref{fact:mean-parameter-space}より、
    $\calM = \nabla \psi(\Theta)$は
    $\nabla \psi(\Int \wt{\Theta})$の開部分多様体となり、
    とくに$V$の開部分多様体となる。

    次に$\eta$が微分同相写像であることを示す。
    \cref{prop:Legendre-transform-properties} (2)より
    $\nabla \psi$は
    $\Int \wt{\Theta}$から$\nabla \psi(\Int \wt{\Theta})$への微分同相だから、
    部分多様体への制限により
    $\nabla \psi$は
    $\Theta$から$\calM$への微分同相を与える。
    したがって
    写像$\eta = (\nabla \psi) \circ \theta \colon \calP \to \calM$は
    微分同相である。
\end{proof}

以降、
$\psi|_{\Int \wt{\Theta}}$の Legendre 変換を
$\calM$上に制限したものを$\phi$と書くことにする。

\begin{theorem}[自然パラメータ座標と期待値パラメータ座標の関係]
    関数
    $\psi \colon \Theta \to \R$および
    $\phi \colon \calM \to \R$と、
    $\calP$上の
    自然パラメータ座標$\theta = (\theta^1, \dots, \theta^n)$および
    期待値パラメータ座標$\eta = (\eta_1, \dots, \eta_m)$
    に関し次が成り立つ:
    \begin{enumerate}
        \item
            \begin{equation}
                \deldel[\psi]{\theta^i}(\theta(p)) = \eta_i(p),
                    \qquad
                    \deldel[\phi]{\eta_i}(\eta(p)) = \theta^i(p)
                    \qquad
                    (p \in \calP).
            \end{equation}
        \item $g$の$\theta$-座標に関する成分は
            \begin{equation}
                g_{ij}(p)
                    =
                        \frac{
                            \del^2 \psi
                        }{
                            \del \theta^i
                            \del \theta^j
                        }
                        (\theta(p))
                    =
                        \deldel[\eta_j]{\theta^i}(p),
                        \qquad
                g^{ij}(p)
                    =
                        \frac{
                            \del^2 \phi
                        }{
                            \del \eta_i
                            \del \eta_j
                        }
                        (\eta(p))
                    =
                        \deldel[\theta^i]{\eta_j}(p)
                        \qquad
                        (p \in \calP)
            \end{equation}
            をみたす。
        \item $\delta_i^j$を Kronecker のデルタとして
            \begin{equation}
                g\myparen{
                    \deldel{\theta^i},
                    \deldel{\eta_j}
                }
                    =
                        \delta_i^j
            \end{equation}
            が成り立つ。
    \end{enumerate}
\end{theorem}

\begin{proof}
    \uline{(1)} \quad
    \cref{fact:mean-parameter-space}より
    $\nabla \psi \circ \theta = \eta$であることと、
    \cref{prop:Legendre-transform-properties} (4)より
    $\nabla \phi = (\nabla \psi)^{-1}$であることから従う。

    \uline{(2)} \quad
    $g$の定義および
    \cref{prop:Legendre-transform-properties} (5)より従う。

    \uline{(3)} \quad
    \begin{alignat}{1}
        g\myparen{
            \deldel{\theta^i},
            \deldel{\eta_j}
        }
            =
                g\myparen{
                    \deldel{\theta^i},
                    \deldel[\theta^k]{\eta_j}
                    \deldel{\theta^k}
                }
            =
                g_{ik} \deldel[\theta^k]{\eta_j}
            =
                g_{ik} g^{kj}
            =
                \delta_i^j.
    \end{alignat}
\end{proof}

\begin{theorem}
    期待値パラメータ座標は
    $\calP$上の$\nabla^{(-1)}$-アファイン座標である。
\end{theorem}

\begin{proof}
    $\del_i = \deldel{\theta^i}, \; \del^i = \deldel{\eta_i}$
    と略記すれば、
    上の定理の(3)より
    \begin{alignat}{1}
        0
            =
                \del^i \delta_k^j
            =
                g\myparen{
                    \nabla^{(1)}_{\del^i} \del_k,
                    \del^j
                }
                + g\myparen{
                    \del_k,
                    \nabla^{(1)}_{\del^i} \del^j
                }
    \end{alignat}
    だから
    \begin{alignat}{1}
        {\Gamma^{(-1)}}_k^{ij}
            &=
                g\myparen{
                    \del_k,
                    \nabla^{(-1)}_{\del^i} \del^j
                }
                \\
            &=
                -g\myparen{
                    \nabla^{(1)}_{\del^i} \del_k,
                    \del^j
                }
                \\
            &=
                - \deldel[\theta^l]{\eta_i}
                g\myparen{
                    \nabla^{(1)}_{\del_l} \del_k,
                    \del^j
                }
                \\
            &=
                - \deldel[\theta^l]{\eta_i}
                {\Gamma^{(1)}}_{lk}^j
                \\
            &=
                0
                \qquad
                ({\Gamma^{(1)}}_{lk}^j = 0)
    \end{alignat}
    となる。
\end{proof}

%\begin{example}[$\nabla^{(-1)}$-測地線]
%    \TODO{有限集合上の場合の$\nabla^{(-1)}$-測地線を自然パラメータ座標で表すとどうなる?}
%\end{example}

%\begin{propdef}[KLダイバージェンス]
%    関数$\rho \colon M \times M \to \R$であって、
%    $\calP$の任意の最小次元実現$(V, T, \mu)$に対し
%    \begin{equation}
%        \rho(p, q)
%            \coloneqq
%                \psi(\theta(p)) + \phi(\eta(q))
%                - \myangle{\theta(p)}{\eta(q)}
%                \qquad
%                ((p, q) \in M \times M)
%    \end{equation}
%    をみたすものがただひとつ存在する。
%    $\rho$を$M$上の
%    \term{KLダイバージェンス}[KL divergence]
%        {KLダイバージェンス}[KLダイバージェンス]
%    という。
%    $\rho(p, q)$のことを
%    $\rho(p : q)$とも書く。
%\end{propdef}
%
%\begin{proof}
%    \TODO{}
%\end{proof}
%
%\begin{theorem}
%    $\rho$から$g, S$を復元できる。
%\end{theorem}
%
%\begin{proof}
%    \TODO{}
%\end{proof}

% ------------------------------------------------------------
%
% ------------------------------------------------------------
\section*{今後の予定}

\begin{itemize}
    \item KLダイバージェンス
\end{itemize}

% ------------------------------------------------------------
%
% ------------------------------------------------------------
\section*{参考文献}

Legendre 変換については
\cite{niculescu_convex_2018}
を参考にした。
期待値パラメータに関しては
\cite{wainwright_graphical_2007}を参考にした。

\nocite{amari_information_2016}

{
    \renewcommand{\bibsection}{}
    \bibliographystyle{amsalpha}
    \bibliography{./bibliography,../../mybibliography}
}

% ------------------------------------------------------------
%
% ------------------------------------------------------------
\newpage
\appendix
\renewcommand\thesection{\Alph{section}}
\setcounter{section}{0}
\section{付録}

\begin{proof}[\cref{prop:dual-connection-existence-uniqueness}の証明]
    一意性は$g$の非退化性より明らか。
    以下、存在を示す。
    まず、$X, Z \in \frakX(TM)$を固定すると
    写像$\frakX(TM) \to \smooth(M), \;
        Y \mapsto X(g(Y, Z)) - g(\nabla_X Y, Z)$
    は$\smooth(M)$-線型だから$\Omega^1(M)$に属する。
    これを$g$で添字を上げて得られるベクトル場を
    $\nabla^*_X Z$と書くことにすれば、
    $\nabla^*_X Z$は目的の式をみたす。
    ここまでで、目的の式をみたす
    写像$\nabla^* \colon \Gamma(TM) \to \Map(\Gamma(TM), \Gamma(TM))$
    が得られた。
    $\nabla^*$の像が
    $\Hom_{\smooth(M)}(\Gamma(TM), \Gamma(TM)) = \Gamma(T^\vee M \otimes TM)$
    に属することは、
    各$Z \in \frakX(M)$に対し
    $\nabla^* Z$の$\smooth(M)$-線型性を確かめればよく、すぐにわかる。
    あとは$\nabla^*$の$\R$-線型性と Leibniz 則を確かめればよいが、
    これらも$\nabla^*$の定め方から明らか。
    よって存在が示された。
\end{proof}


\end{document}