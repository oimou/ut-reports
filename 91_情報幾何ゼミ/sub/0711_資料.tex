\documentclass[report]{jlreq}
\usepackage{global}
\usepackage{./local}
\subfiletrue
\def\assetspath{../}
%\makeindex
\chead{2023/07/11}
\begin{document}

% ============================================================
%
% ============================================================

% ------------------------------------------------------------
%
% ------------------------------------------------------------
\section*{振り返りと導入}

前回は指数型分布族の具体例の計算を行った。
本稿では次のことを行う:
\begin{itemize}
    \item 双対構造を定義し、とくに$\alpha$-接続の性質を調べる。
    \item Legendre 変換を定義する。
    \item 指数型分布族の期待値パラメータを定義する。
\end{itemize}

% ------------------------------------------------------------
%
% ------------------------------------------------------------
\section{双対構造}

\subsection{一般の多様体の場合}

\begin{definition}[双対構造]
    $M$を多様体とする。
    3つ組$(g, \nabla, \nabla^*)$
    が$M$上の
    \term{双対構造}[dualistic structure]
        {双対構造}[そうついこうぞう]
    であるとは、
    すべての$X, Y, Z \in \frakX(M)$に対し
    \begin{equation}
        X(g(Y, Z))
            =
                g(\nabla_X Y, Z) + g(Y, \nabla'_X Z)
    \end{equation}
    が成り立つことをいう。
    このとき、
    $\nabla, \nabla'$はそれぞれ$g$に関する$\nabla', \nabla$の
    \term{双対接続}[dual connection]
        {双対接続}[そうついせつぞく]
    であるという。

    さらに$\nabla, \nabla'$がいずれも$M$上平坦であるとき、
    $(g, \nabla, \nabla')$は
    \term{双対平坦}[dually flat]
        {双対平坦}[そうついへいたん]
    であるという。
    双対平坦な双対構造を
    \term{双対平坦構造}[dually flat structure]
        {双対平坦構造}[そうついへいたんこうぞう]
    という。
\end{definition}

\begin{proposition}[双対接続の存在と一意性]
    \label[proposition]{prop:dual-connection-existence-uniqueness}
    $\nabla$を$M$上のアファイン接続とする。
    このとき、
    $g$に関する$\nabla$の双対接続がただひとつ存在する。
\end{proposition}

\begin{proof}
    証明は付録に記した。
\end{proof}

たとえ双対平坦であっても、
両方の接続係数が同一の座標のもとで消えるとは限らない。
実際、それが成り立つためには次のような非常に強い条件が必要となる:

\begin{proposition}
    $(g, \nabla, \nabla')$を双対構造、
    $x = (x_1, \dots, x_m)$を$M$の開集合$U$上の座標とする。
    このとき、
    $x$に関する$\nabla, \nabla'$の接続係数が
    $U$上つねに$0$ならば、
    $(U, g|_U)$は$\R^m$のある開集合と等長同型である。
\end{proposition}

\begin{proof}
    $\del_i g_{jk} = \Gamma_{ij}^l g_{lk} + {\Gamma'}_{ik}^l g_{lj}$
    より従う。
\end{proof}

\subsection{指数型分布族の場合}

指数型分布族の$\alpha$-接続について考える。

\begin{proposition}[曲率のACテンソルによる表示]
    $\alpha \in \R$、
    $R^{(\alpha)}$を$\nabla^{(\alpha)}$の
    $(1, 3)$-曲率テンソルとする。
    このとき、
    $\calP$の任意の$\nabla^{(1)}$-アファイン座標に関し、
    $R^{(\alpha)}$の成分は
    \begin{equation}
        {R^{(\alpha)}}_{ijk}^{\hphantom{ijk}l}
            =
                - \frac{1 - \alpha^2}{4}
                \myparen{
                    A_{jk}^{\hphantom{jk}m} A_{im}^{\hphantom{im}l}
                    - A_{ik}^{\hphantom{ik}m} A_{jm}^{\hphantom{jm}l}
                }
    \end{equation}
    となる。
\end{proposition}

\begin{proof}
    \url{0613_資料.pdf}命題1.12の式
    \begin{equation}
        {R^{(\alpha)}}_{ijk}^{\hphantom{ijk}l}
            = \frac{1 - \alpha}{2} \myparen{
                \del_i A_{jk}^{\hphantom{jk}l}
                -
                \del_j A_{ik}^{\hphantom{ik}l}
            }
            + \myparen{\frac{1 - \alpha}{2}}^2
            \myparen{
                A_{jk}^{\hphantom{jk}m} A_{im}^{\hphantom{im}l}
                -
                A_{ik}^{\hphantom{ik}m} A_{jm}^{\hphantom{jm}l}
            }
    \end{equation}
    を変形する。
    \begin{alignat}{1}
        \del_i A_{jk}^{\hphantom{jk}l}
            &=
                \del_i (g^{ln} S_{jkn})
                \\
            &=
                \del_i (g^{ln}) S_{jkn}
                +
                g^{lm} \del_i S_{jkm}
                \\
            &=
                - \del_i (g_{mn}) g^{mn} g^{ln} S_{jkn}
                +
                g^{lm} \del_i S_{jkm}
                \qquad
                (\del_i(g_{nm} g^{lm}) = 0)
                \\
            &=
                - S_{imn} g^{mn} g^{ln} S_{jkn}
                +
                g^{lm} \del_i S_{jkm}
                \qquad
                (\del_i g_{mn} = S_{imn})
                \\
            &=
                - A_{im}^{\hphantom{im}l} A_{jk}^{\hphantom{jk}m}
                + g^{lm} \del_i S_{jkm}.
    \end{alignat}
    同様にして
    \begin{alignat}{1}
        \del_j A_{ik}^l
            =
                - A_{jm}^{\hphantom{jm}l} A_{ik}^{\hphantom{ik}m}
                + g^{lm} \del_j S_{ikm}.
    \end{alignat}
    したがって命題の主張の式が得られた。
\end{proof}

\begin{corollary}
    ~
    \begin{enumerate}
        \item $\forall \alpha \in \R$に対し
            $R^{(\alpha)}
                =
                    (1 - \alpha^2)
                    R^{(0)}
                =
                    R^{(-\alpha)}$.
        \item 次は同値:
            \begin{enumerate}
                \item すべての$\alpha \in \R$に対し、
                    $\nabla^{(\alpha)}$は平坦である。
                \item ある$\alpha \neq \pm 1$が存在し、
                    $\nabla^{(\alpha)}$は平坦である。
            \end{enumerate}
    \end{enumerate}
\end{corollary}

\begin{proof}
    \uline{(1)} \quad
    上の命題より明らか。

    \uline{(2)} \quad
    まず、上の命題より次は同値である:
    \begin{enumerate}
        \item $\forall \alpha \in \R$に対し
            $R^{(\alpha)} = 0$.
        \item $\exists \alpha \neq \pm 1$
            \, s.t. \,
            $R^{(\alpha)} = 0$.
    \end{enumerate}
    さらに$\alpha$-接続はすべて torsion-free だから、
    曲率が$0$であることと平坦であることは同値である。
\end{proof}

\begin{theorem}
    任意の$\alpha \in \R$に対し、
    3つ組$(g, \nabla^{(\alpha)}, \nabla^{(-\alpha)})$は
    $\calP$上の双対構造となる。
    さらに、
    $\alpha = \pm 1$ならば
    $(g, \nabla^{(\alpha)}, \nabla^{(-\alpha)})$は
    双対平坦である。
\end{theorem}

\begin{proof}
    双対構造であることは
    \begin{alignat}{1}
        g(\nabla^{(\alpha)}_X Y, Z)
            + g(Y, \nabla^{(-\alpha)}_X Z)
            &=
                g(\nabla^{g}_X Y, Z)
                - \frac{1 - \alpha}{2} S(X, Y, Z)
                + g(Y, \nabla^{g}_X Z)
                - \frac{1 + \alpha}{2} S(X, Z, Y)
                \\
            &=
                g(\nabla^{g}_X Y, Z)
                + g(Y, \nabla^{g}_X Z)
                \\
            &=
                X(g(Y, Z))
    \end{alignat}
    より従う。
    $\alpha = \pm 1$で双対平坦となることは
    上の系よりわかる。
\end{proof}

% ------------------------------------------------------------
%
% ------------------------------------------------------------
\section{Legendre 変換}

本節では
$W$を有限次元$\R$-ベクトル空間、
$\nabla$を$W$上の標準的なアファイン接続とする。

\begin{propdef}[古典的な Legendre 変換]
    $U \subset W$を開集合、
    $f \colon U \to \R$を$C^\infty$関数であって
    $\nabla f \colon U \to W^\vee$が可逆であるものとする。
    このとき、
    関数$f^\vee \colon U' \to \R$ \;
    (ただし$U' \coloneqq (\nabla f)(U)$)
    であって
    次をみたすものがただひとつ存在する:
    \begin{equation}
        f(x) + f^\vee(y) = \langle x, y \rangle,
            \quad \text{where} \quad
            x \coloneqq (\nabla f)^{-1}(y)
            \qquad (y \in U').
    \end{equation}
    $f^\vee$を$f$の
    \term{Legendre 変換}[Legendre transform]
        {Legendre 変換}[Legendre へんかん]
    という。
\end{propdef}

\begin{proof}
    一意性は明らか。
    存在は
    $f^\vee(y)
        \coloneqq
            \myangle{(\nabla f)^{-1}(y)}{y} - f((\nabla f)^{-1}(y))$
    と定義すればよい。
\end{proof}

\begin{example}[Legendre 変換の例]
    ~
    \begin{itemize}
        \item $e^x$ (Poisson 分布族の実現の対数分配関数)
            $\to y \log y - y$
        \item $\log (1 + e^x)$ (Bernoulli 分布族の実現の対数分配関数)
            $\to y \log y + (1 - y) \log (1 - y)$
        \item $x^2 / 2$ (分散既知の正規分布族の実現の対数分配関数)
            $\to y^2 / 2$
    \end{itemize}
\end{example}

\begin{proposition}[Legendre 変換の性質]
    \label[proposition]{prop:Legendre-transform-properties}
    ~
    \begin{enumerate}
        \item $(\nabla f)^{-1} = \nabla f^\vee$
        \item $f^{\vee \vee} = f$
    \end{enumerate}
\end{proposition}

\begin{proof}
    \uline{(1)} \quad
    \begin{alignat}{1}
        (\nabla f^\vee)(y)
            &=
                (\nabla f)^{-1}(y)
                + \myangle{
                    y
                }{
                    (\nabla
                        (\nabla f)^{-1}
                    )(y)
                }
                - \myangle{
                    (\nabla f)
                    (\nabla f)^{-1}
                    (y)
                }{
                    (\nabla
                        (\nabla f)^{-1}
                    )(y)
                }
            =
                (\nabla f)^{-1}(y)
    \end{alignat}
    よって$(\nabla f)^{-1} = \nabla f^\vee$である。

    \uline{(2)} \quad
    \begin{alignat}{1}
        f^{\vee\vee}(x)
            &=
                \myangle{
                    x
                }{
                    (\nabla f^\vee)^{-1}
                    (x)
                }
                - f^\vee(
                    (\nabla f^\vee)^{-1}
                    (x)
                )
                \\
            &=
                \myangle{
                    x
                }{
                    (\nabla f)
                    (x)
                }
                - f^\vee(
                    (\nabla f)
                    (x)
                )
                \\
            &=
                \myangle{
                    x
                }{
                    (\nabla f)
                    (x)
                }
                - \myparen{
                    \myangle{
                        (\nabla f)
                        (x)
                    }{
                        (\nabla f)^{-1}
                        (\nabla f)
                        (x)
                    }
                    - f(
                        (\nabla f)^{-1}
                        (\nabla f)
                        (x)
                    )
                }
                \\
            &=
                f(x)
    \end{alignat}
    よって$f^{\vee\vee} = f$である。
\end{proof}

本稿では、とくに次の状況下で Legendre 変換を考えることになる。

\begin{lemma}
    $W$を有限次元$\R$-ベクトル空間、
    $U \subset W$を凸開集合、
    $f \colon U \to \R$を$C^\infty$関数であって
    $\Hess f$が$U$上各点で正定値であるものとする。
    このとき、
    $\nabla f \colon U \to W^\vee$は単射である\footnote{
        逆は成立しない。
        すなわち、$f'$が単射であっても
        $\Hess f$が正定値であるとは限らない。
        $f(x) = x^4$が反例となる。
    }。
\end{lemma}

\begin{proof}
    $u, u' \in U, \; u \neq u'$を固定する。
    $u_t \coloneqq (1 - t)u + tu' \; (t \in [0, 1])$とおき、
    $\varphi \colon [0, 1] \to U, \; t \mapsto f(u_t)$と定める
    ($U$は凸だから$\varphi$の像はたしかに$U$に属する)。
    平均値の定理より、
    ある$\tau \in (0, 1)$が存在して
    \begin{alignat}{1}
        \myangle{
            (\nabla f)(u') - (\nabla f)(u)
        }{
            u' - u
        }
            &=
                \varphi'(1) - \varphi'(0)
                \\
            &=
                \varphi''(\tau)
                \\
            &=
                \myangle{
                    (\Hess f)(u_\tau) (u' - u)
                }{
                    u' - u
                }
                \\
            &> 0
    \end{alignat}
    が成り立つ。
    よって$(\nabla f)(u') \neq (\nabla f)(u)$である。
    したがって$\nabla f$は単射である。
\end{proof}

% ------------------------------------------------------------
%
% ------------------------------------------------------------
\section{期待値パラメータ}

\begin{propdef}[期待値パラメータ空間]
    \label[propdef]{propdef:mean-parameter-space}
    $\calP$は開であるとし、
    $(V, T, \mu)$を$\calP$の最小次元実現とする。
    このとき、集合
    \begin{equation}
        \calM
            \coloneqq
                \mybrace{
                    E_p[T] \in V
                    \mid
                    p \in \calP
                }
    \end{equation}
    は$V$の開部分多様体となり、
    写像$\eta \colon \calP \to \calM, \; p \mapsto E_p[T]$
    は微分同相写像となる。

    $\calM$を
    $(V, T, \mu)$に関する$\calP$の
    \term{期待値パラメータ空間}[mean parameter space]
        {期待値パラメータ空間}[きたいちぱらめーたくうかん]
    といい、
    $\eta$を
    \term{期待値パラメータ座標}[mean parameter coordinates]
        {期待値パラメータ座標}[きたいちぱらめーたざひょう]
    という。
\end{propdef}

\begin{lemma}
    写像$\nabla \psi \colon \Theta \to V^{\vee\vee} = V$は
    \begin{equation}
        (\nabla \psi)(\theta(p))
            =
                \eta(p)
                \qquad
                (p \in \calP)
    \end{equation}
    をみたす。
\end{lemma}

\begin{proof}
    明らか。
\end{proof}

\begin{lemma}
    $(\nabla \psi)|_{\Int \wt{\Theta}}
        \colon \Int \wt{\Theta} \to V^{\vee\vee} = V$
    は{\smooth}埋め込みかつ開写像である。
\end{lemma}

\begin{fact}
    位相ベクトル空間の凸集合の内部は凸集合である。
    \qed
\end{fact}

\begin{proof}[補題の証明]
    $\psi$は{\smooth}だから$\nabla \psi$も{\smooth}である。
    また、
    $\Hess \psi$は正定値だから
    $\nabla \psi$の微分は可逆である。
    逆写像定理より
    $\nabla \psi$は局所微分同相写像であり、
    とくに開写像である。
    また、
    $\wt{\Theta}$が$V^\vee$の凸集合であること (\url{0425_資料.pdf}命題2.2) と
    上の事実より$\Int \wt{\Theta}$は$V^\vee$の凸集合だから、
    $\nabla \psi$は単射である。
    したがって$\nabla \psi$は埋め込みである。
\end{proof}

\begin{proof}[\cref{propdef:mean-parameter-space}の証明]
    まず$\calM = (\nabla \psi)(\Theta)$である。
    いま$\calP$は開だから$\Theta$は$V^\vee$の開集合である。
    このことと$\nabla \psi$が開写像であることから
    $\calM$は$V$の開集合、したがって開部分多様体である。
    $\eta(p) = (\nabla \psi) \circ \theta(p) \; (p \in \calP)$
    が成り立つから、
    $(\nabla \psi), \theta$が微分同相写像であることから
    $\eta$も微分同相写像である。
\end{proof}

\begin{proposition}
    $\calP$は開であるとし、
    $\phi$を$\psi|_\Theta$の Legendre 変換とする。
    このとき次が成り立つ:
    \begin{enumerate}
        \item
            \begin{equation}
                \deldel[\psi]{\theta^i} = \eta_i,
                    \qquad
                    \deldel[\phi]{\eta_i} = \theta^i.
            \end{equation}
        \item $\theta$-座標に関し
            \begin{equation}
                g_{ij}
                    =
                        \frac{
                            \del^2 \psi
                        }{
                            \del \theta^i
                            \del \theta^j
                        }
                    =
                        \deldel[\eta_j]{\theta^i},
                        \qquad
                g^{ij}
                    =
                        \frac{
                            \del^2 \phi
                        }{
                            \del \eta_i
                            \del \eta_j
                        }
                    =
                        \deldel[\theta^i]{\eta_j}.
            \end{equation}
        \item
            \begin{equation}
                g\myparen{
                    \deldel{\theta^i},
                    \deldel{\eta_j}
                }
                    =
                        \delta_i^j.
            \end{equation}
    \end{enumerate}
\end{proposition}

\begin{proof}
    \uline{(1)} \quad
    $\deldel[\psi]{\theta^i}(\theta(p)) = E_p[T^i] = \eta_i(p)$
    である。
    また、Legendre 変換の定義より
    \begin{equation}
        \phi(\eta)
            =
                \myangle{\eta}{(\nabla \psi)^{-1}(\eta)}
                - \psi((\nabla \psi)^{-1}(\eta))
    \end{equation}
    だから
    \begin{alignat}{1}
        (\nabla \phi)(\eta)
            &=
                (\nabla \psi)^{-1}(\eta)
                + \myangle{
                    \eta
                }{
                    \nabla
                    (\nabla \psi)^{-1}
                    (\eta)
                }
                - \myangle{
                    (\nabla \psi)
                    (\nabla \psi)^{-1}
                    (\eta)
                }{
                    \nabla
                    (\nabla \psi)^{-1}
                    (\eta)
                }
                \\
            &=
                (\nabla \psi)^{-1}(\eta)
                + \myangle{
                    \eta
                }{
                    \nabla
                    (\nabla \psi)^{-1}
                    (\eta)
                }
                - \myangle{
                    \eta
                }{
                    \nabla
                    (\nabla \psi)^{-1}
                    (\eta)
                }
                \\
            &=
                (\nabla \psi)^{-1}
                (\eta)
    \end{alignat}
    である。
    したがって
    $(\nabla \phi)(\eta(p)) = (\nabla \psi)^{-1}(\eta(p)) = \theta(p)$
    よって
    $\deldel[\phi]{\eta_i}(\eta(p)) = \theta^i(p)$
    である。

    \uline{(2)} \quad
    $g_{ij}(\theta(p)) = \frac{\del^2 \psi}{\del \theta^i \del \theta^j}(\theta(p))$
    は$g$の定義から明らか。
    また
    $\delta^j_k
        = \deldel[\eta_i]{\theta^k} \deldel[\theta^j]{\eta_i}
        = g_{ik} \deldel[\theta^j]{\eta_i}$
    より
    $g^{ij}
        = \deldel[\theta^j]{\eta_i}
        = \frac{\del^2 \phi}{\del \eta_i \del \eta_j}$
    である。

    \uline{(3)} \quad
    \begin{alignat}{1}
        g\myparen{
            \deldel{\theta^i},
            \deldel{\eta_j}
        }
            =
                g\myparen{
                    \deldel{\theta^i},
                    \deldel[\theta^k]{\eta_j}
                    \deldel{\theta^k}
                }
            =
                g_{ik} \deldel[\theta^k]{\eta_j}
            =
                g_{ik} g^{kj}
            =
                \delta_i^j.
    \end{alignat}
\end{proof}

\begin{theorem}
    期待値パラメータ座標に
    $\calM$上の任意の座標を合成したものは
    $\calP$上の$\nabla^{(-1)}$-アファイン座標である。
\end{theorem}

\begin{proof}
    $\del_i = \deldel{\theta^i}, \; \del^i = \deldel{\eta_i}$
    と略記すれば、
    上の命題の(3)より
    \begin{alignat}{1}
        0
            =
                \del^i \delta_k^j
            =
                g\myparen{
                    \nabla^{(1)}_{\del^i} \del_k,
                    \del^j
                }
                + g\myparen{
                    \del_k,
                    \nabla^{(1)}_{\del^i} \del^j
                }
    \end{alignat}
    だから
    \begin{alignat}{1}
        {\Gamma^{(-1)}}_k^{ij}
            &=
                g\myparen{
                    \del_k,
                    \nabla^{(-1)}_{\del^i} \del^j
                }
                \\
            &=
                -g\myparen{
                    \nabla^{(1)}_{\del^i} \del_k,
                    \del^j
                }
                \\
            &=
                - \deldel[\theta^l]{\eta_i}
                g\myparen{
                    \nabla^{(1)}_{\del_l} \del_k,
                    \del^j
                }
                \\
            &=
                - \deldel[\theta^l]{\eta_i}
                {\Gamma^{(1)}}_{lk}^j
                \\
            &=
                0
                \qquad
                ({\Gamma^{(1)}}_{lk}^j = 0)
    \end{alignat}
    となる。
\end{proof}

\begin{example}[$\nabla^{(-1)}$-測地線]
    \TODO{有限集合上の場合の$\nabla^{(-1)}$-測地線を自然パラメータ座標で表すとどうなる?}
\end{example}

%\begin{propdef}[KLダイバージェンス]
%    関数$\rho \colon M \times M \to \R$であって、
%    $\calP$の任意の最小次元実現$(V, T, \mu)$に対し
%    \begin{equation}
%        \rho(p, q)
%            \coloneqq
%                \psi(\theta(p)) + \phi(\eta(q))
%                - \myangle{\theta(p)}{\eta(q)}
%                \qquad
%                ((p, q) \in M \times M)
%    \end{equation}
%    をみたすものがただひとつ存在する。
%    $\rho$を$M$上の
%    \term{KLダイバージェンス}[KL divergence]
%        {KLダイバージェンス}[KLダイバージェンス]
%    という。
%    $\rho(p, q)$のことを
%    $\rho(p : q)$とも書く。
%\end{propdef}
%
%\begin{proof}
%    \TODO{}
%\end{proof}
%
%\begin{theorem}
%    $\rho$から$g, S$を復元できる。
%\end{theorem}
%
%\begin{proof}
%    \TODO{}
%\end{proof}

% ------------------------------------------------------------
%
% ------------------------------------------------------------
\section*{今後の予定}

\begin{itemize}
    \item KLダイバージェンス
\end{itemize}

% ------------------------------------------------------------
%
% ------------------------------------------------------------
\section*{参考文献}

\nocite{amari_information_2016}

{
    \renewcommand{\bibsection}{}
    \bibliographystyle{amsalpha}
    \bibliography{./bibliography,../../mybibliography}
}

% ------------------------------------------------------------
%
% ------------------------------------------------------------
\newpage
\appendix
\renewcommand\thesection{\Alph{section}}
\setcounter{section}{0}
\section{付録}

\begin{proof}[\cref{prop:dual-connection-existence-uniqueness}の証明]
    一意性は$g$の非退化性より明らか。
    以下、存在を示す。
    まず、$X, Z \in \frakX(TM)$を固定すると
    写像$\frakX(TM) \to \smooth(M), \;
        Y \mapsto X(g(Y, Z)) - g(\nabla_X Y, Z)$
    は$\smooth(M)$-線型だから$\Omega^1(M)$に属する。
    これを$g$で添字を上げて得られるベクトル場を
    $\nabla'_X Z$と書くことにすれば、
    $\nabla'_X Z$は目的の式をみたす。
    これで写像$\nabla' \colon \Gamma(TM) \to \Map(\Gamma(TM), \Gamma(TM))$
    が得られた。
    $\nabla'$の像が
    $\Hom_{\smooth(M)}(\Gamma(TM), \Gamma(TM)) = \Gamma(T^\vee M \otimes TM)$
    に属することは、
    各$Z \in \frakX(M)$に対し
    $\nabla' Z$の$\smooth(M)$-線型性を確かめればよく、すぐにわかる。
    あとは$\nabla'$の$\R$-線型性と Leibniz 則を確かめればよいが、
    これらも$\nabla'$の定め方から明らか。
    よって存在が示された。
\end{proof}


\end{document}