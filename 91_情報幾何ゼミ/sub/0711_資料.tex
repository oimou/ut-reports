\documentclass[report]{jlreq}
\usepackage{global}
\usepackage{./local}
\subfiletrue
\def\assetspath{../}
%\makeindex
\chead{2023/07/11}
\begin{document}

% ============================================================
%
% ============================================================

% ------------------------------------------------------------
%
% ------------------------------------------------------------
\section*{振り返りと導入}

前回は指数型分布族の具体例の計算を行った。
本稿では次のことを行う:
\begin{itemize}
    \item 一般化を念頭に置きながら、指数型分布族の双対構造の性質を整理する!!!
    \item \TODO{}
\end{itemize}

% ------------------------------------------------------------
%
% ------------------------------------------------------------
\section{双対構造}

\subsection{一般の多様体の場合}

\begin{definition}[双対接続]
    $(M, g)$を Riemann 多様体、
    $\nabla, \nabla'$を$M$上のアファイン接続とする。
    $\nabla'$が
    $g$に関する$\nabla$の
    \term{双対接続}[dual connection]
        {双対接続}[そうついせつぞく]
    であるとは、
    すべての$X, Y, Z \in \frakX(M)$に対し
    \begin{equation}
        X(g(Y, Z))
            =
                g(\nabla_X Y, Z) + g(Y, \nabla'_X Z)
    \end{equation}
    が成り立つことをいう。
\end{definition}

\begin{proposition}
    $\nabla$を$M$上のアファイン接続とする。
    このとき、
    $g$に関する$\nabla$の双対接続がただひとつ存在する。
\end{proposition}

\begin{proof}
    \TODO{}
\end{proof}

\begin{definition}[双対構造]
    $(g, \nabla, \nabla^*)$
    が$M$上の
    \term{双対構造}[dualistic structure]
        {双対構造}[そうついこうぞう]
    であるとは、
    $\nabla$と$\nabla'$が
    $g$に関し互いに双対接続であることをいう。
\end{definition}

\begin{definition}[双対平坦]
    $\nabla, \nabla'$がいずれも$M$上平坦であるとき、
    $(g, \nabla, \nabla')$は
    \term{双対平坦}[dually flat]
        {双対平坦}[そうついへいたん]
    であるという。
    双対平坦な双対構造を
    \term{双対平坦構造}[dually flat structure]
        {双対平坦構造}[そうついへいたんこうぞう]
    という。
\end{definition}

たとえ双対平坦であっても、
両方の接続係数が同一の座標のもとで消えるとは限らない。
実際、それが成り立つためには次のような非常に強い条件が必要となる:

\begin{proposition}
    $(g, \nabla, \nabla')$を双対構造、
    $x = (x_1, \dots, x_m)$を$M$の開集合$U$上の座標とする。
    このとき、
    $x$に関する$\nabla, \nabla'$の接続係数が
    $U$上つねに$0$ならば、
    $g$は$U$上定数である。
\end{proposition}

\begin{proof}
    $\del_i g_{jk} = \Gamma_{ij}^l g_{lk} + \Gamma_{ik}^l g_{lj}$
    より従う。
\end{proof}

\subsection{指数型分布族の場合}

指数型分布族の$\alpha$-接続について考える。

\begin{proposition}[曲率のACテンソルによる表示]
    $\alpha \in \R$、
    $R^{(\alpha)}$を$\nabla^{(\alpha)}$の
    $(1, 3)$-曲率テンソルとする。
    このとき、
    $\calP$の任意の$\nabla$-アファイン座標に関し\TODO{$\nabla^{(\alpha)}$ではなく?}、
    $R^{(\alpha)}$の成分は
    \begin{equation}
        {R^{(\alpha)}}_{ijk}^{\hphantom{ijk}l}
            =
                - \frac{1 - \alpha^2}{2}
                \myparen{
                    A_{jk}^{\hphantom{jk}m} A_{im}^{\hphantom{im}l}
                    - A_{ik}^{\hphantom{ik}m} A_{jm}^{\hphantom{jm}l}
                }
    \end{equation}
    となる。
    とくに$\alpha = \pm 1$のとき
    $R^{(\alpha)} = 0$となる。
\end{proposition}

\begin{proof}
    \url{0613_資料.pdf}命題1.12の式
    \begin{equation}
        {R^{(\alpha)}}_{ijk}^{\hphantom{ijk}l}
            = \frac{1 - \alpha}{2} \myparen{
                \del_i A_{jk}^{\hphantom{jk}l}
                -
                \del_j A_{ik}^{\hphantom{ik}l}
            }
            + \myparen{\frac{1 - \alpha}{2}}^2
            \myparen{
                A_{jk}^{\hphantom{jk}m} A_{im}^{\hphantom{im}l}
                -
                A_{ik}^{\hphantom{ik}m} A_{jm}^{\hphantom{jm}l}
            }
    \end{equation}
    を変形する。
    \begin{alignat}{1}
        &\phantom{=}
            \del_i A_{jk}^{\hphantom{jk}l}
            \\
        &=
            \del_i (g^{ln} S_{jkn})
            \\
        &=
            \del_i (g^{ln}) S_{jkn}
            +
            g^{lm} \del_i S_{jkm}
            \\
        &=
            - \del_i (g_{mn}) g^{mn} g^{ln} S_{jkn}
            +
            g^{lm} \del_i S_{jkm}
            \qquad
            (\del_i(g_{nm} g^{lm}) = 0)
            \\
        &=
            - S_{imn} g^{mn} g^{ln} S_{jkn}
            +
            g^{lm} \del_i S_{jkm}
            \qquad
            (\del_i g_{mn} = S_{imn})
            \\
        &=
            - A_{im}^{\hphantom{im}l} A_{jk}^{\hphantom{jk}m}
            + g^{lm} \del_i S_{jkm}.
    \end{alignat}
    同様にして
    \begin{alignat}{1}
        \del_j A_{ik}^l
            =
                - A_{jm}^{\hphantom{jm}l} A_{ik}^{\hphantom{ik}m}
                + g^{lm} \del_j S_{ikm}.
    \end{alignat}
    したがって命題の主張の式が得られた。
\end{proof}

\begin{corollary}
    すべての$\alpha \in \R$に対し
    \begin{equation}
        R^{(\alpha)}
            =
                R^{(-\alpha)}
    \end{equation}
    が成り立つ。
\end{corollary}

\begin{proof}
    $\frac{1 - \alpha}{2} = \frac{1 + \alpha}{2} - \alpha$
    および
    $\myparen{
        \frac{1 - \alpha}{2}
    }^2
        =
            \myparen{
                \frac{1 + \alpha}{2}
            }^2 - \alpha$
    が成り立つことより、
    命題から
    \begin{equation}
        {R^{(\alpha)}}_{ijk}^{\hphantom{ijk}l}
            =
                {R^{(-\alpha)}}_{ijk}^{\hphantom{ijk}l}
                - \alpha {R^{(-1)}}_{ijk}^{\hphantom{ijk}l}
    \end{equation}
    となる。
    さらに
    $R^{(-1)} = 0$だから
    系の主張が得られた。
\end{proof}

\begin{theorem}
    任意の$\alpha \in \R$に対し、
    3つ組$(g, \nabla^{(\alpha)}, \nabla^{(-\alpha)})$は
    $\calP$上の双対構造となる。
    さらに、
    $\alpha = \pm 1$ならば
    $(g, \nabla^{(\alpha)}, \nabla^{(-\alpha)})$は
    双対平坦である。
    \TODO{逆はいえるか?}
\end{theorem}

\begin{proof}
    双対構造であることは
    \begin{alignat}{1}
        g(\nabla^{(\alpha)}_X Y, Z)
            + g(Y, \nabla^{(-\alpha)}_X Z)
            &=
                g(\nabla^{g}_X Y, Z)
                - \frac{1 - \alpha}{2} S(X, Y, Z)
                + g(Y, \nabla^{g}_X Z)
                - \frac{1 + \alpha}{2} S(X, Z, Y
                \\
            &=
                g(\nabla^{g}_X Y, Z)
                + g(Y, \nabla^{g}_X Z)
                \\
            &=
                X(g(Y, Z))
    \end{alignat}
    より従う。
    $\alpha = \pm 1$で双対平坦となることは
    上の系よりわかる。
\end{proof}

% ------------------------------------------------------------
%
% ------------------------------------------------------------
\section{期待値パラメータ}

\begin{lemma}
    $W$を$\R$-ベクトル空間、
    $U \subset W$を開集合、
    $f \colon U \to \R$を$C^\infty$関数であって
    $\Hess f$が$U$上各点で正定値であるものとする。
    このとき、
    $\nabla f \colon U \to W^\vee$は単射である。
\end{lemma}

\begin{proof}
    \TODO{}
\end{proof}

\begin{definition}[古典的な Legendre 変換]
    $W$を$\R$-ベクトル空間、
    $U \subset W$を開集合、
    $f \colon U \to \R$を$C^\infty$関数であって
    $\Hess f$が$U$上各点で正定値であるものとする。
    関数
    \begin{equation}
        f^\vee \colon U' \to \R,
            \quad
            y
            \mapsto
            \myangle{y}{(\nabla f)^{-1}(y)} - f((\nabla f)^{-1}(y))
            \quad \text{where} \quad
            U' \coloneqq (\nabla f)(U)
    \end{equation}
    を$f$の
    \term{凸共役}[convex conjugate]
        {凸共役}[とつきょうやく]
    という。
\end{definition}

\begin{example}[凸共役の例]
    ~
    \begin{itemize}
        \item $e^x$ (Poisson 分布族の実現の対数分配関数)
            $\to y \log y - y$
        \item $\log (1 + e^x)$ (Bernoulli 分布族の実現の対数分配関数)
            $\to y \log y + (1 - y) \log (1 - y)$
        \item $x^2$ (分散既知の正規分布族の実現の対数分配関数)
            $\to y^2 / 4$
    \end{itemize}
\end{example}

\begin{proposition}[凸共役の性質]
    \TODO{}
\end{proposition}

\begin{proof}
    \TODO{何が必要?}
\end{proof}

\begin{proposition}
    $(\nabla \psi)|_{\Int \wt{\Theta}}
        \colon \Int \wt{\Theta} \to V^{\vee\vee} = V$
    は{\smooth}埋め込みかつ開写像である。
\end{proposition}

\begin{lemma}
    $\Int \wt{\Theta}$は$V^\vee$の凸集合である。
\end{lemma}

\begin{proof}
    $\wt{\Theta}$が$V^\vee$の凸集合であることは
    \url{0425_資料.pdf}命題2.2で確かめた。
    一般に、位相ベクトル空間の凸集合の内部は凸集合だから、補題が従う。
\end{proof}

\begin{proof}[命題の証明]
    $\psi$は{\smooth}だから$\nabla \psi$も{\smooth}である。
    また、
    $\nabla \psi$
    $\Hess \psi$は正定値だから
    $\nabla \psi$の微分は全単射である。
    逆写像定理より
    $\nabla \psi$は局所微分同相写像であり、
    とくに開写像である。
    また、補題より$\Int \wt{\Theta}$は$V^\vee$の凸集合だから、
    $\nabla \psi$は単射である。
    したがって$\nabla \psi$は埋め込みである。
\end{proof}

\begin{propdef}[期待値パラメータ空間]
    $\calP$は開であるとし、
    $(V, T, \mu)$を$\calP$の最小次元実現とする。
    このとき、集合
    \begin{equation}
        \calM
            \coloneqq
                \mybrace{
                    E_p[T] \in V
                    \mid
                    p \in \calP
                }
    \end{equation}
    は$V$の開部分多様体となり、
    写像$\eta \colon \calP \to \calM, \; p \mapsto E_p[T]$
    は微分同相写像となる。
    $\calM$を
    $(V, T, \mu)$に関する$\calP$の
    \term{期待値パラメータ空間}[mean parameter space]
        {期待値パラメータ空間}[きたいちぱらめーたくうかん]
    といい、
    $\eta$を
    \term{期待値パラメータ座標}[mean parameter coordinates]
        {期待値パラメータ座標}[きたいちぱらめーたざひょう]
    という。
\end{propdef}

\begin{proof}
    まず$\calM = (\nabla \psi)(\Theta)$である。
    いま$\calP$は開だから$\Theta$は$V^\vee$の開集合である。
    このことと$\nabla \psi$が開写像であることから
    $\calM$は$V$の開集合、したがって開部分多様体である。
    $\eta(p) = (\nabla \psi) \circ \theta(p) \; (p \in \calP)$
    が成り立つから、
    $(\nabla \psi), \theta$が微分同相写像であることから
    $\eta$も微分同相写像である。
\end{proof}

\begin{proposition}
    $\calP$は開であるとし、
    $\phi$を$\psi|_\Theta$の凸共役とする。
    このとき次が成り立つ:
    \begin{enumerate}
        \item
            \begin{equation}
                \deldel[\psi]{\theta^i} = \eta_i,
                    \qquad
                    \deldel[\phi]{\eta_i} = \theta^i.
            \end{equation}
        \item $\theta$-座標に関し
            \begin{equation}
                g_{ij}
                    =
                        \frac{
                            \del^2 \psi
                        }{
                            \del \theta^i
                            \del \theta^j
                        },
                        \qquad
                g^{ij}
                    =
                        \frac{
                            \del^2 \phi
                        }{
                            \del \eta_i
                            \del \eta_j
                        }.
            \end{equation}
    \end{enumerate}
\end{proposition}

\TODO{$g^{ij}$は$T^\vee \calP$上の計量を定める?}

\begin{proof}
    \uline{(1)} \quad
    $\deldel[\psi]{\theta^i}(\theta(p)) = E_p[T^i] = \eta_i(p)$
    である。
    \TODO{}

    \uline{(2)} \quad
    $g_{ij}(\theta(p)) = \frac{\del^2 \psi}{\del \theta^i \del \theta^j}(\theta(p))$
    は$g$の定義から明らか。
    \TODO{}
\end{proof}

\begin{theorem}
    期待値パラメータ座標に
    $\calM$上の任意の座標を合成したものは
    $\calP$上の$\nabla^{(-1)}$-アファイン座標である。
\end{theorem}

\begin{proof}
    \begin{equation}
        {\Gamma^{(-1)}}_{\gamma}^{\alpha\beta}
            =
                \deldel[\eta_\gamma]{\theta^k}
                \myparen{
                    \frac{\del \theta^k}{\del \eta_\alpha \del \eta_\beta}
                    +
                    {\Gamma^{(-1)}}_{ij}^k
                    \deldel[\theta^i]{\eta_\alpha}
                    \deldel[\theta^j]{\eta_\beta}
                }
    \end{equation}
    が$0$となることをいえばよい。
\end{proof}

\begin{example}[$\nabla^{(-1)}$-測地線]
    \TODO{}
\end{example}

%\begin{propdef}[KLダイバージェンス]
%    関数$\rho \colon M \times M \to \R$であって、
%    $\calP$の任意の最小次元実現$(V, T, \mu)$に対し
%    \begin{equation}
%        \rho(p, q)
%            \coloneqq
%                \psi(\theta(p)) + \phi(\eta(q))
%                - \myangle{\theta(p)}{\eta(q)}
%                \qquad
%                ((p, q) \in M \times M)
%    \end{equation}
%    をみたすものがただひとつ存在する。
%    $\rho$を$M$上の
%    \term{KLダイバージェンス}[KL divergence]
%        {KLダイバージェンス}[KLダイバージェンス]
%    という。
%    $\rho(p, q)$のことを
%    $\rho(p : q)$とも書く。
%\end{propdef}
%
%\begin{proof}
%    \TODO{}
%\end{proof}
%
%\begin{theorem}
%    $\rho$から$g, S$を復元できる。
%\end{theorem}
%
%\begin{proof}
%    \TODO{}
%\end{proof}

% ------------------------------------------------------------
%
% ------------------------------------------------------------
\section*{今後の予定}

\begin{itemize}
    \item KLダイバージェンス
\end{itemize}

% ------------------------------------------------------------
%
% ------------------------------------------------------------
\section*{参考文献}

\nocite{amari_information_2016}

{
    \renewcommand{\bibsection}{}
    \bibliographystyle{amsalpha}
    \bibliography{./bibliography,../../mybibliography}
}

% ------------------------------------------------------------
%
% ------------------------------------------------------------
\newpage
\appendix
\renewcommand\thesection{\Alph{section}}
\setcounter{section}{0}
\section{付録}


\end{document}