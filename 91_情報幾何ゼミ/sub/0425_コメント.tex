\documentclass[report]{jlreq}
\usepackage{global}
\usepackage{./local}
\subfiletrue
\def\assetspath{../}
%\makeindex
\chead{2023/04/25}
\begin{document}

% ============================================================
%
% ============================================================

発表時にコメントがあった命題などを整理する。

\begin{definition}[full support]
    $\calX$を位相空間、$\mu$を$\calX$上の Borel 測度とする。
    \begin{equation}
        \supp\mu \coloneqq \Big(
            \bigcup_{\substack{U \opensubset \calX \\ \mu(U) = 0}} U
        \Big)^c
    \end{equation}
    と定める。
    $\mu$が\term{full support}{full support}であるとは、
    $\supp\mu = \calX$であることをいう。
\end{definition}

\begin{problem}
    $\calX = \{ 1, \dots, n \}$とする。
    $\calX$上の full support な確率測度全体の集合$\calP$は
    例3.1で見たように指数型分布族であるが、
    $\calP$は$n - 1$次元の実現を持つか?
\end{problem}

\begin{answer}
    答え: 持つ。
    \begin{alignat}{1}
        p(dk)
            &=
                \exp\mybrace{
                    \sum_{i = 1}^{n - 1}
                        (\log p_i) \delta_{ik}
                    +
                    \myparen{
                        \log\myparen{
                            1 - \sum_{i = 1}^{n - 1} p_i
                        }
                    }
                    \delta_{n, k}
                }
                \gamma(dk)
                \\
            &=
                \exp\mybrace{
                    \sum_{i = 1}^{n - 1}
                        \myparen{
                            \log p_i
                            -
                            \log\myparen{
                                1 - \sum_{i = 1}^{n - 1} p_i
                            }
                        }
                        \delta_{ik}
                    +
                    \log\myparen{
                        1 - \sum_{i = 1}^{n - 1} p_i
                    }
                }
                \gamma(dk)
    \end{alignat}
    と表せるから、
    $T \colon \calX \to \R^{n - 1}, \;
        k \mapsto \up{t}(\delta_{1k}, \dots, \delta_{n - 1, k})$
    を十分統計量、
    数え上げ測度$\gamma$を基底測度として
    $(\R^{n - 1}, T, \gamma)$は
    $n - 1$次元の実現となる。
\end{answer}

\begin{problem}
    $\calX$を可測空間、
    $\calP$を$\calX$上の指数型分布族とする。
    任意の$P_1, P_2 \in \calP$に対し
    $P_1 \sim P_2$ (互いに絶対連続) が成り立つことを示せ。
\end{problem}

\begin{answer}
    役割を入れ替えれば逆向きも示せるから、$P_1 \ll P_2$のみ示せばよい。
    $E \subset \calX$を$P_2(E) = 0$なる可測集合とし、
    $P_1(E) = 0$を示す。
    $\calP$の実現$(T, \mu)$をひとつ選んで固定する。
    $P_2$に対し定義2.1条件(E3)の$\theta_2 \in \R^m$をひとつ選ぶ。
    \begin{alignat}{1}
        0
            &= P_2(E) \\
            &= \int_E \dd[P_2]{\mu}(x) \mu(dx) \\
            &= \int_E e^{\langle \theta_2, T(x) \rangle - \psi(\theta_2)} \mu(dx)
    \end{alignat}
    であるが、
    被積分関数は$\mu$に関しほとんど至るところ正であることから、
    $\mu(E) = 0$でなければならない\footnote{
        ほとんど至るところ正の値をとる関数$f > 0$に対し
        $\calX = f^{-1}(
            (1, +\infty]
            \cup (1/2, 1]
            \cup \cdots
            \cup (1/n, 1/(n - 1)]
            \cup \cdots
        )$と表せることを使って示せる。
    }。
    よって、$P_1 \ll \mu$であることとあわせて
    $P_1(E) = 0$が従う。
\end{answer}

\begin{problem}
    正規分布族は2より小さい次元の実現を持つか?
\end{problem}

\begin{answer}
    条件(as-a), (as-b)を確かめれば、
    2が最小であることがわかる。
    (cf. \url{0606_資料.pdf})
\end{answer}

指数型分布族の定義の$\R^m$を有限次元ベクトル空間$V$に置き換えてみる。
発表時点では$\R^m$で十分だと考えていたが、
$\R^m$の代わりに$V$を使うことで、
議論を簡潔にできるというメリットが指摘によりわかった。

\begin{definition}[指数型分布族]
    \label[definition]{def:exponential-family}
    $\calX$を可測空間、
    $\emptyset \neq \calP \subset \calP(\calX)$とする。
    $\calP$が$\calX$上の
    \term{指数型分布族}[exponential family]
        {指数型分布族}[しすうがたぶんぷぞく]
    であるとは、次が成り立つことをいう:
    $\exists \; (\highlight{V}, T, \mu)$ s.t.
    \begin{description}
        \item[\highlight{(E0)}] \highlight{$V$は有限次元$\R$-ベクトル空間である。}
        \item[(E1)] $T \colon \calX \to \highlight{V}$は可測写像である。
        \item[(E2)] $\mu$は$\calX$上の$\sigma$-有限測度であり、
            $\forall \; p \in \calP$に対し$p \ll \mu$をみたす。
        \item[(E3)] $\forall \; p \in \calP$に対し、
            $\exists \; \theta \in \highlight{V^\vee}$ s.t.
            \begin{equation}
                \dd[p]{\mu}(x)
                    = \frac{
                        \exp \langle \theta, T(x) \rangle
                    }{
                        \int_\calX \exp \langle \theta, T(y) \rangle \, \mu(dy)
                    }
                    \quad
                    \text{$\mu$-a.e. $x \in \calX$}
            \end{equation}
            である。
            ただし$\langle \cdot, \cdot \rangle$は
            \highlight{自然なペアリング$V^\vee \times V \to \R$である。}
    \end{description}
    さらに次のように定める:
    \begin{itemize}
        \item $(\highlight{V}, T, \mu)$を$\calP$の
            \term{実現}[representation]
            {実現}[じつげん]
            という。
            \begin{itemize}
                \item $m$を$(\highlight{V}, T, \mu)$の
                    \term{次元}[dimension]{次元}[じげん]
                    という。
                \item $T$を$(\highlight{V}, T, \mu)$の
                    \term{十分統計量}[sufficient statistic]
                    {十分統計量}[じゅうぶんとうけいりょう]
                    という。
                \item $\mu$を$(\highlight{V}, T, \mu)$の
                    \term{基底測度}[base measure]
                    {基底測度}[きていそくど]
                    という。
            \end{itemize}
        \item 集合$\Theta_{(\highlight{V}, T, \mu)}$
            \begin{equation}
                \Theta_{(\highlight{V}, T, \mu)}
                    \coloneqq \mybrace{
                        \theta \in \highlight{V^\vee}
                        \;\Big|\;
                        \int_\calX \exp \langle \theta, T(y) \rangle \, \mu(dy) < +\infty
                    }
            \end{equation}
            を$(\highlight{V}, T, \mu)$の
            \term{自然パラメータ空間}[natural parameter space]
            {自然パラメータ空間}[しぜんぱらめーたくうかん]
            という。
        \item 関数$\psi \colon \Theta_{(\highlight{V}, T, \mu)} \to \R,$
            \begin{equation}
                \psi(\theta)
                    \coloneqq
                    \log \int_\calX \exp \langle \theta, T(y) \rangle \, \mu(dy)
            \end{equation}
            を$(\highlight{V}, T, \mu)$の
            \term{対数分配関数}[log-partition function]
            {対数分配関数}[たいすうぶんぱいかんすう]
            という。
    \end{itemize}
\end{definition}

上の定義に基づいて次の定理を書き直してみる
(ただし発表時の修正を踏襲し、「極小実現」の語は「$\theta$が一意の実現」に置き換えてある)。
証明の主な変更点としては、
$\R^m$を$V$に置き換えたことによって
ノルムが使えなくなるため、
かわりに annihilated を使うようになっている。
証明は\cite[Lemma 21]{bn1970_pdf}を参考にした。

\begin{theorem}[「$\theta$が一意の実現」の存在]
    $\calX$を可測空間、
    $\calP \subset \calP(\calX)$を
    $\calX$上の指数型分布族とする。
    このとき、$\calP$の「$\theta$が一意の実現」が存在する。
\end{theorem}

\begin{proof}
    $(V, T, \mu)$は$\calP$の実現のうちで次元が最小のものであるとする。
    $(V, T, \mu)$の次元 ($m$とおく) が$0$ならば
    $V^\vee$は1点集合だから証明は終わる。

    以下$m \ge 1$の場合を考え、
    $(V, T, \mu)$が「$\theta$が一意の実現」であることを示す。
    背理法のために$(V, T, \mu)$が「$\theta$が一意の実現」でないこと、
    すなわちある$p_0 \in \calP$および
    $\theta_0, \theta_0' \in V^\vee, \; \theta_0 \neq \theta_0'$が存在して
    \begin{equation}
        \locallabel{eq:assumption}
        \exp\myparen{\langle \theta_0, T(x) \rangle - \psi(\theta_0)}
            = \dd[p_0]{\mu}(x)
            = \exp\myparen{\langle \theta_0', T(x) \rangle - \psi(\theta_0')}
            \qquad
            \text{$\mu$-a.e. $x \in \calX$}
    \end{equation}
    が成り立つことを仮定する。
    証明の方針としては、
    次元$m - 1$の実現$(V', T', \mu)$を具体的に構成することにより、
    $(V, T, \mu)$の次元$m$が最小であることとの矛盾を導く。

    さて、式\localcref{eq:assumption}を整理して
    \begin{equation}
        \langle \theta_0 - \theta_0', T(x) \rangle
            = \psi(\theta_0) - \psi(\theta_0')
            \qquad
            \text{$\mu$-a.e. $x \in \calX$}
    \end{equation}
    を得る。
    表記の簡略化のために
    $\theta_1 \coloneqq \theta_0 - \theta_0' \in V^\vee, \;
        r \coloneqq \psi(\theta_0) - \psi(\theta_0') \in \R$
    とおけば
    \begin{equation}
        \locallabel{eq:costant-pairing}
        \langle \theta_1, T(x) \rangle
            = r
            \qquad
            \text{$\mu$-a.e. $x \in \calX$}
    \end{equation}
    を得る。
    ここで
    $V' \coloneqq (\R\theta)^\top = \{ v \in V \mid \langle \theta, v \rangle = 0 \}$
    とおき、
    次の claim を示す。
    \begin{description}
        \item[Claim] ある可測写像$T' \colon \calX \to V'$および
            $v_0 \in V$が存在して
            $T(x) = T'(x) + v_0 \; (\text{$\mu$-a.e.$x$})$
            が成り立つ。
    \end{description}
    \begin{innerproof}
        いま背理法の仮定より$\theta_1 \neq 0$であるから、
        $\theta_1$を延長した$V^\vee$の基底$\theta_1, \dots, \theta_m$が存在する。
        このとき、$\theta_1, \dots, \theta_m$を双対基底に持つ
        $V$の基底$v_1, \dots, v_m$が存在する。
        この基底$v_1, \dots, v_m$に関する
        $T$の成分表示を
        $T(x) = \sum_{i = 1}^m T^i(x) v_i, \;
            T^i \colon \calX \to \R$とおくと、
        \localcref{eq:costant-pairing}より
        $T^1(x) = \langle \theta_1, T(x) \rangle = r \; (\text{$\mu$-a.e.$x$})$
        が成り立つ。
        そこで$v_0 \coloneqq rv_1 \in V$とおくと
        $\langle \theta_1, T(x) - v_0 \rangle = 0 \; (\text{$\mu$-a.e.$x$})$
        が成り立つから、
        可測写像$T' \colon \calX \to V'$を
        \begin{equation}
            T'(x) \coloneqq \begin{cases}
                T(x) - v_0 & (\langle \theta_1, T(x) - v_0 \rangle = 0) \\
                0 & (\text{otherwise})
            \end{cases}
        \end{equation}
        と定めることができる。
        この$T, v_0$が求めるものである。
    \end{innerproof}
    $(V', T', \mu)$が$\calP$の実現であることを示す。
    \cref{def:exponential-family}の条件(E0)-(E2)は明らかに成立しているから、
    あとは条件(E3)を確認すればよい。
    そこで$p \in \calP$とする。
    いま$(V, T, \mu)$が$\calP$の実現であることより、
    ある$\theta \in V^\vee$が存在して
    \begin{equation}
        \dd[p]{\mu}(x)
            = \frac{
                \exp\langle \theta, T(x) \rangle
            }{
                \int_{\calX} \exp\langle \theta, T(y) \rangle \, \mu(dy)
            }
            \qquad
            \text{$\mu$-a.e. $x \in \calX$}
    \end{equation}
    が成り立つ。
    $T', v_0$を用いて式変形すると、$\mu$-a.e.$x$に対し
    \begin{alignat}{1}
        \dd[p]{\mu}(x)
            &= \frac{
                \exp\myparen{
                    \langle \theta, T(x) \rangle
                }
            }{
                \int_{\calX} \exp\myparen{
                    \langle \theta, T(x) \rangle
                } \, \mu(dy)
            } \\
            &= \frac{
                \exp\myparen{
                    \langle \theta, T'(x) \rangle
                    + \langle \theta, v_0 \rangle
                }
            }{
                \int_{\calX} \exp\myparen{
                    \langle \theta, T'(x) \rangle
                    + \langle \theta, v_0 \rangle
                } \, \mu(dy)
            } \\
            &= \frac{
                \exp\myparen{
                    \langle \theta, T'(x) \rangle
                }
            }{
                \int_{\calX} \exp\myparen{
                    \langle \theta, T'(x) \rangle
                } \, \mu(dy)
            }
    \end{alignat}
    が成り立つ。
    したがって$(V', T', \mu)$は条件(E3)も満たし、
    $\calP$の実現であることがいえた。
    $(V', T', \mu)$は次元$m - 1$だから
    $(V, T, \mu)$の次元$m$の最小性に矛盾する。
    背理法より$(V, T, \mu)$は$\calP$の「$\theta$が一意の実現」である。
\end{proof}

% ------------------------------------------------------------
%
% ------------------------------------------------------------
\section{参考文献}

{
    \renewcommand{\bibsection}{}
    \bibliographystyle{amsalpha}
    \bibliography{./bibliography,../../mybibliography}
}


\end{document}