\documentclass[report]{jlreq}
\usepackage{global}
\usepackage{../sub/local}
\subfiletrue
\def\assetspath{../}
\begin{document}

% ============================================================
%
% ============================================================
\chapter{統計的多様体}

% ------------------------------------------------------------
%
% ------------------------------------------------------------
\section{双対構造}

\begin{definition}[双対構造]
    $M$を多様体とする。
    $M$上の
    Riemann 計量$g$と
    アファイン接続$\nabla, \nabla^*$の組
    $(g, \nabla, \nabla^*)$
    が$M$上の
    \term{双対構造}[dualistic structure]
        {双対構造}[そうついこうぞう]
    であるとは、
    すべての$X, Y, Z \in \frakX(M)$に対し
    \begin{equation}
        X(g(Y, Z))
            =
                g(\nabla_X Y, Z) + g(Y, \nabla^*_X Z)
    \end{equation}
    が成り立つことをいう。
    このとき、
    $\nabla, \nabla^*$はそれぞれ$g$に関する$\nabla^*, \nabla$の
    \term{双対接続}[dual connection]
        {双対接続}[そうついせつぞく]
    であるという。

    さらに$\nabla, \nabla^*$がいずれも$M$上平坦であるとき、
    $(g, \nabla, \nabla^*)$は
    \term{双対平坦}[dually flat]
        {双対平坦}[そうついへいたん]
    であるという。
    双対平坦な双対構造を
    \term{双対平坦構造}[dually flat structure]
        {双対平坦構造}[そうついへいたんこうぞう]
    という。
\end{definition}

\begin{proposition}[双対接続の存在と一意性]
    \label[proposition]{prop:dual-connection-existence-uniqueness}
    $M$を多様体、
    $g$を$M$上のRiemann 計量、
    $\nabla$を$M$上のアファイン接続とする。
    このとき、
    $g$に関する$\nabla$の双対接続がただひとつ存在する。
\end{proposition}

\begin{proof}
    一意性は$g$の非退化性より明らか。
    以下、存在を示す。
    まず、$X, Z \in \frakX(TM)$を固定すると
    写像$\frakX(TM) \to \smooth(M), \;
        Y \mapsto X(g(Y, Z)) - g(\nabla_X Y, Z)$
    は$\smooth(M)$-線型だから$\Omega^1(M)$に属する。
    これを$g$で添字を上げて得られるベクトル場を
    $\nabla^*_X Z$と記すことにすれば、
    $\nabla^*_X Z$は目的の式をみたす。
    ここまでで、目的の式をみたす
    写像$\nabla^* \colon \Gamma(TM) \to \Map(\Gamma(TM), \Gamma(TM))$
    が得られた。
    $\nabla^*$の像が
    $\Hom_{\smooth(M)}(\Gamma(TM), \Gamma(TM)) = \Gamma(T^\vee M \otimes TM)$
    に属することは、
    各$Z \in \frakX(M)$に対し
    $\nabla^* Z$の$\smooth(M)$-線型性を確かめればよく、すぐにわかる。
    あとは$\nabla^*$の$\R$-線型性と Leibniz 則を確かめればよいが、
    これらも$\nabla^*$の定め方から明らか。
    よって存在が示された。
\end{proof}

\begin{definition}[双対アファイン座標]
    $(g, \nabla, \nabla^*)$を$M$上の双対構造とする。
    $\nabla$-アファイン座標$\theta = (\theta^1, \ldots, \theta^n)$と
    $\nabla^*$-アファイン座標$\eta = (\eta_1, \ldots, \eta_n)$の組
    $(\theta, \eta)$が
    $(g, \nabla, \nabla^*)$に関する
    \term{双対アファイン座標}[dual affine coordinate]
        {双対アファイン座標}[そうついアファインざひょう]
    であるとは、
    \begin{equation}
        g(\del_i, \del^j) = \delta_i^j
            \qquad
            (\forall i, j)
    \end{equation}
    が成り立つことをいう。
    ただし$\del_i \coloneqq \deldel{\theta^i}, \;
        \del^i \coloneqq \deldel{\eta_i}$である。
\end{definition}



\end{document}