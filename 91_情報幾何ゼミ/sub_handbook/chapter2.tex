\documentclass[report]{jlreq}
\usepackage{global}
\usepackage{../sub/local}
\subfiletrue
\def\assetspath{../}
\begin{document}

% ============================================================
%
% ============================================================
\chapter{Hessian}

% ------------------------------------------------------------
%
% ------------------------------------------------------------
\section{Hessian}

本節では
$W$を$m$次元$\R$-ベクトル空間 ($m \in \Z_{\ge 0}$)、
$U \opensubset W$を開部分集合とする。

本節では$U$上の$C^\infty$関数に対し Hessian を定義したい。
そこで、まず$U$上にアファイン接続を定義し、それを用いて Hessian を定義する。

\subsection{$U$上のアファイン接続}

一般のアファイン接続の平坦性を定義しておく。

\begin{definition}[平坦アファイン接続]
    $M$を多様体、
    $\nabla$を$M$上のアファイン接続とする。
    \begin{itemize}
        \item $M$の開部分集合$O \opensubset M$上の座標であって、
            それに関する$\nabla$の接続係数がすべて$0$となるものを、
            $O$上の
            \term{$\nabla$-アファイン座標}[$\nabla$-affine coordinates]
                {アファイン座標}[アファインざひょう]
            という。
        \item 各$p \in M$に対し、
            $p$のまわりの$\nabla$-アファイン座標が存在するとき、
            $\nabla$は$M$上
            \term{平坦}[flat]{平坦}[へいたん]であるという。
    \end{itemize}
\end{definition}

今考えている$U$上には、次のような平坦アファイン接続が定まる。

\begin{propdef}[$U$上の平坦アファイン接続]
    $U$上のアファイン接続
    $D \colon \Gamma(TU) \to \Gamma(T^\vee U \otimes TU)$を、
    次の規則で well-defined に定めることができる:
    \begin{itemize}
        \item 各$X \in \Gamma(TU)$に対し、
            $W$の基底が定める$U$上の座標
            $x^i \; (i = 1, \dots, m)$
            をひとつ選び、
            \begin{equation}
                \label{eq:def_connection_d}
                DX
                    \coloneqq dX^i \otimes \deldel{x^i}
                    \in \Gamma(T^\vee U \otimes TU)
            \end{equation}
            と定める。
            ただし、
            $X$の成分表示を$X = X^i \deldel{x^i}$とおいた。
    \end{itemize}
    さらに、この$D$は$U$上のアファイン接続として平坦である。
\end{propdef}

\begin{proof}
    写像として well-defined であることを一旦認め、
    先に$\R$-線型性、Leibniz 則、平坦性を確かめる。
    $D$の$\R$-線型性と Leibniz 則は、
    外微分$d$の$\R$-線型性と Leibniz 則から従う。
    平坦性は、
    式\cref{eq:def_connection_d}で用いた座標$x^i$が
    $D$-アファイン座標となることから従う。
    最後に、$D$が写像として well-defined であることを示す。
    $y^\alpha \; (\alpha = 1, \dots, m)$を
    $W$の基底が定める$U$上の座標とすると、
    \begin{alignat}{1}
        dX^i \otimes \deldel{x^i}
            &=
                d\myparen{
                    X^\alpha \deldel[x^i]{y^\alpha}
                }
                \otimes
                \deldel[y^\alpha]{x^i}
                \deldel{y^\alpha}
                \\
            &=
                \Bigg(
                    \deldel[x^i]{y^\alpha} dX^\alpha
                    +
                    X^\alpha \underbrace{
                        d\myparen{
                            \deldel[x^i]{y^\alpha}
                        }
                    }_{=0}
                \Bigg)
                \otimes
                \deldel[y^\alpha]{x^i}
                \deldel{y^\alpha}
                \\
            &=
                dX^\alpha
                \otimes
                \deldel{y^\alpha}
    \end{alignat}
    となる。ただし「$= 0$」の部分は
    $x^i$と$y^\alpha$の間の座標変換がアファイン変換となることを用いた。
    これで well-defined 性も示された。
\end{proof}

\subsection{Hessian}

$U$上のアファイン接続$D$により、
$T^\vee U$の接続が誘導される。
これを用いて Hessian を定義する。

\begin{definition}[Hessian]
    \smooth 関数$f \colon U \to \R$に対し、
    $f$の\term{Hessian}{Hessian}[Hessian]を
    \begin{equation}
        \Hess f \coloneqq Ddf
            \in \Gamma(T^\vee U \otimes T^\vee U)
    \end{equation}
    と定義する。
\end{definition}

$D$-アファイン座標を用いると、
Hessian の成分表示は簡単な形になる。

\begin{proposition}[Hessian の成分表示]
    \label[proposition]{prop:hessian_components}
    $x^i \; (i = 1, \dots, m)$を
    $U$上の$D$-アファイン座標とする。
    このとき、
    座標$x^i$に関する$\Hess f$の成分表示は
    \begin{equation}
        \Hess f
            = \frac{\del^2 f}{\del x^i \del x^j} \, dx^i \otimes dx^j
    \end{equation}
    となる。
    とくに$f$の\smooth 性より
    $\Hess f$は対称テンソルである。
\end{proposition}

\begin{proof}
    $(\Hess f)(\del_i, \del_j)
        = \langle D_{\del_i} df, \del_j \rangle
        = \del_i \langle df, \del_j \rangle
            - \langle
                df,
                \underbrace{D_{\del_i} \del_j}_{= 0}
            \rangle
        = \del_i (\del_j f)
        = \frac{\del^2 f}{\del x^i \del x^j}$
    より従う。
\end{proof}


\end{document}