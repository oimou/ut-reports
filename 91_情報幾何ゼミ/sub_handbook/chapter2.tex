\documentclass[report]{jlreq}
\usepackage{global}
\usepackage{../sub/local}
\subfiletrue
\def\assetspath{../}
\begin{document}

% ============================================================
%
% ============================================================
\chapter{Hessian, Legendre 変換, Fourier-Laplace 変換}

% ------------------------------------------------------------
%
% ------------------------------------------------------------
\section{Hessian}

本節では
$W$を$m$次元$\R$-ベクトル空間 ($m \in \Z_{\ge 0}$)、
$U \opensubset W$を開部分集合とする。

本節では$U$上の$C^\infty$関数に対し Hessian を定義したい。
そこで、まず$U$上にアファイン接続を定義し、それを用いて Hessian を定義する。

\subsection{ベクトル空間の開部分集合上のアファイン接続}

一般のアファイン接続の平坦性を定義しておく。

\begin{definition}[平坦アファイン接続]
    $M$を多様体、
    $\nabla$を$M$上のアファイン接続とする。
    \begin{itemize}
        \item $M$の開部分集合$O \opensubset M$上の座標であって、
            それに関する$\nabla$の接続係数がすべて$0$となるものを、
            $O$上の
            \term{$\nabla$-アファイン座標}[$\nabla$-affine coordinates]
                {アファイン座標}[アファインざひょう]
            という。
        \item 各$p \in M$に対し、
            $p$のまわりの$\nabla$-アファイン座標が存在するとき、
            $\nabla$は$M$上
            \term{平坦}[flat]{平坦}[へいたん]であるという。
    \end{itemize}
\end{definition}

今考えている$U$上には、次のような平坦アファイン接続が定まる。

\begin{propdef}[$U$上の平坦アファイン接続]
    $U$上のアファイン接続
    $D \colon \Gamma(TU) \to \Gamma(T^\vee U \otimes TU)$を、
    次の規則で well-defined に定めることができる:
    \begin{itemize}
        \item 各$X \in \Gamma(TU)$に対し、
            $W$の基底が定める$U$上の座標
            $x^i \; (i = 1, \dots, m)$
            をひとつ選び、
            \begin{equation}
                \label{eq:def_connection_d}
                DX
                    \coloneqq dX^i \otimes \deldel{x^i}
                    \in \Gamma(T^\vee U \otimes TU)
            \end{equation}
            と定める。
            ただし、
            $X$の成分表示を$X = X^i \deldel{x^i}$とおいた。
    \end{itemize}
    さらに、この$D$は$U$上のアファイン接続として平坦である。
\end{propdef}

\begin{proof}
    写像として well-defined であることを一旦認め、
    先に$\R$-線型性、Leibniz 則、平坦性を確かめる。
    $D$の$\R$-線型性と Leibniz 則は、
    外微分$d$の$\R$-線型性と Leibniz 則から従う。
    平坦性は、
    式\cref{eq:def_connection_d}で用いた座標$x^i$が
    $D$-アファイン座標となることから従う。
    最後に、$D$が写像として well-defined であることを示す。
    $y^\alpha \; (\alpha = 1, \dots, m)$を
    $W$の基底が定める$U$上の座標とすると、
    \begin{alignat}{1}
        dX^i \otimes \deldel{x^i}
            &=
                d\myparen{
                    X^\alpha \deldel[x^i]{y^\alpha}
                }
                \otimes
                \deldel[y^\alpha]{x^i}
                \deldel{y^\alpha}
                \\
            &=
                \Bigg(
                    \deldel[x^i]{y^\alpha} dX^\alpha
                    +
                    X^\alpha \underbrace{
                        d\myparen{
                            \deldel[x^i]{y^\alpha}
                        }
                    }_{=0}
                \Bigg)
                \otimes
                \deldel[y^\alpha]{x^i}
                \deldel{y^\alpha}
                \\
            &=
                dX^\alpha
                \otimes
                \deldel{y^\alpha}
    \end{alignat}
    となる。ただし「$= 0$」の部分は
    $x^i$と$y^\alpha$の間の座標変換がアファイン変換となることを用いた。
    これで well-defined 性も示された。
\end{proof}

\subsection{Hessian}

$U$上のアファイン接続$D$により、
$T^\vee U$の接続が誘導される。
これを用いて Hessian を定義する。

\begin{definition}[Hessian]
    \smooth 関数$f \colon U \to \R$に対し、
    $f$の\term{Hessian}{Hessian}[Hessian]を
    \begin{equation}
        \Hess f \coloneqq Ddf
            \in \Gamma(T^\vee U \otimes T^\vee U)
    \end{equation}
    と定義する。
\end{definition}

$D$-アファイン座標を用いると、
Hessian の成分表示は簡単な形になる。

\begin{proposition}[Hessian の成分表示]
    \label[proposition]{prop:hessian_components}
    $x^i \; (i = 1, \dots, m)$を
    $U$上の$D$-アファイン座標とする。
    このとき、
    座標$x^i$に関する$\Hess f$の成分表示は
    \begin{equation}
        \Hess f
            = \frac{\del^2 f}{\del x^i \del x^j} \, dx^i \otimes dx^j
    \end{equation}
    となる。
    とくに$f$の\smooth 性より
    $\Hess f$は対称テンソルである。
\end{proposition}

\begin{proof}
    $(\Hess f)(\del_i, \del_j)
        = \langle D_{\del_i} df, \del_j \rangle
        = \del_i \langle df, \del_j \rangle
            - \langle
                df,
                \underbrace{D_{\del_i} \del_j}_{= 0}
            \rangle
        = \del_i (\del_j f)
        = \frac{\del^2 f}{\del x^i \del x^j}$
    より従う。
\end{proof}

% ------------------------------------------------------------
%
% ------------------------------------------------------------
\section{Legendre 変換}


\begin{definition}[Legendre 変換]
    $U \subset W$を開集合、
    $f \colon U \to \R$を$C^\infty$関数であって
    $\nabla f \colon U \to W^\vee$が単射であるものとする。
    関数
    \begin{equation}
        f^\vee \colon U' \to \R,
            \quad
            y
            \mapsto
            \myangle{(\nabla f)^{-1}(y)}{y} - f((\nabla f)^{-1}(y))
            \quad
            \text{where}
            \quad
            U' \coloneqq \nabla f(U)
    \end{equation}
    を$f$の
    \term{Legendre 変換}[Legendre transform]
        {Legendre 変換}[Legendre へんかん]
    という。
\end{definition}

\begin{example}[Legendre 変換の例]
    前回 (\url{0704_資料.pdf}) 扱った具体例について
    対数分配関数の Legendre 変換を計算してみる。
    \begin{itemize}
        \item \uline{Bernoulli 分布族 (i.e. 2元集合上の full support な確率分布の族):} \quad
            対数分配関数は
            $\psi \colon \R \to \R, \; \theta \mapsto \log (1 + \exp \theta)$
            であった。
            よって
            $\nabla \psi(\theta)
                =
                    \frac{\exp \theta}{1 + \exp \theta}$
            であり、
            $(\nabla \psi)^{-1}(\eta)
                =
                    \log \eta - \log (1 - \eta)$
            である。
            したがって
            $\psi^\vee(\eta)
                =
                    \eta \log \eta
                    + (1 - \eta) \log (1 - \eta)$
            である。
        \item \uline{正規分布族:} \quad
            対数分配関数は
            $\psi \colon \R \times \R_{< 0} \to \R, \;
                \theta
                \mapsto
                - \frac{(\theta^1)^2}{4 \theta^2}
                - \frac{1}{2} \log (- \theta^2)
                + \frac{1}{2} \log \pi$
            であった。
            よって
            $\nabla \psi(\theta)
                =
                    \begin{pmatrix}
                        - \frac{\theta^1}{2 \theta^2}
                        &
                        \frac{(\theta^1)^2}{4 (\theta^2)^2} - \frac{1}{2 \theta^2}
                    \end{pmatrix}$
            であり、
            $(\nabla \psi)^{-1}(\eta)
                =
                    \frac{1}{\eta_2 - (\eta_1)^2}
                    \begin{pmatrix}
                        \eta_1
                        \\
                        - 1/2
                    \end{pmatrix}$
            である。
            したがって
            $\psi^\vee(\eta)
                =
                    - \frac{1}{2}
                    \myparen{
                        1 + \log 2\pi
                        + \log(\eta_2 - (\eta_1)^2)
                    }$
            である。
    \end{itemize}
\end{example}

本稿では、とくに次の状況を考えることになる。

\begin{proposition}
    \label[proposition]{prop:Legendre-transform-properties}
    $U \subset W$を凸開集合、
    $f \colon U \to \R$を$C^\infty$関数であって
    $\Hess f$が$U$上各点で (対称であることも含む意味で) 正定値であるものとする。
    このとき、次が成り立つ:
    \begin{enumerate}
        \item $\nabla f$は局所微分同相である。
            とくに$U' \coloneqq \nabla f(U)$は$W^\vee$の開集合である。
        \item $\nabla f \colon U \to U'$は微分同相である。
            とくに$\nabla f$は単射である。
    \end{enumerate}
    したがって$f^\vee$が定義でき、$f^\vee$は次をみたす:
    \begin{enumerate}
        \setcounter{enumi}{2}
        \item $f^\vee \colon U' \to \R$は$C^\infty$関数である。
        \item $\nabla f^\vee = (\nabla f)^{-1}$が成り立つ。
            とくに$\nabla f^\vee$は単射である。
        \item 各$y \in U'$に対し
            $(\Hess f^\vee)_y = ((\Hess f)_x)^{-1}$が成り立つ
            (ただし$x \coloneqq (\nabla f)^{-1}(y)$)。
            とくに$(\Hess f^\vee)_y$は正定値である。
    \end{enumerate}
\end{proposition}

\begin{proof}
    \uline{(1)} \quad
    命題の仮定より$\Hess f$は$U$上各点で正定値だから、
    $\nabla f$の微分は各点で線型同型である。
    したがって$\nabla f$は局所微分同相であり、
    とくに開写像である。
    よって$U' = \nabla f(U)$は$W^\vee$の開集合である。

    \uline{(2)} \quad
    $u, \wt{u} \in U, \; u \neq \wt{u}$を固定し、
    $[0, 1]$を含む$\R$の開区間$I$であって、
    すべての$t \in I$に対し
    $(1 - t)u + t\wt{u}$が
    $U$に属するようなものをひとつ選ぶ
    ($U$は$W$の凸開集合だからこれは可能)。
    さらに
    $\varphi \colon I \to U, \; t \mapsto f((1 - t)u + t\wt{u})$と定めると、
    平均値定理より、
    ある$\tau \in (0, 1)$が存在して
    \begin{alignat}{1}
        \myangle{
            \nabla f(\wt{u}) - \nabla f(u)
        }{
            \wt{u} - u
        }
            &=
                \varphi'(1) - \varphi'(0)
                \\
            &=
                \varphi''(\tau)
                \qquad
                (\text{平均値定理})
                \\
            &=
                \myangle{
                    (\Hess f)_{(1 - \tau)u + \tau\wt{u}}
                }{
                    (\wt{u} - u)^2
                }
                \\
            &>
                0
                \qquad
                (\text{$\Hess f$は正定値})
    \end{alignat}
    が成り立つ。
    よって$\nabla f(\wt{u}) \neq \nabla f(u)$である。
    したがって$\nabla f$は単射である。
    このことと (1) より
    $\nabla f \colon U \to U'$は微分同相である。

    \uline{(3)} \quad
    $\nabla f \colon U \to U'$が微分同相ゆえに
    $(\nabla f)^{-1} \colon U' \to U$は{\smooth}だから、
    $f^\vee$は{\smooth}関数である。

    \uline{(4)} \quad
    $f^\vee$の定義式を$\nabla$で微分すると、
    すべての$y \in U'$に対し
    \begin{alignat}{1}
        (\nabla f^\vee)(y)
            &=
                (\nabla f)^{-1}(y)
                + \myangle{
                    y
                }{
                    \nabla
                    (\nabla f)^{-1}
                    (y)
                }
                - \myangle{
                    (\nabla f)(
                        (\nabla f)^{-1}
                        (y)
                    )
                }{
                    \nabla
                    (\nabla f)^{-1}
                    (y)
                }
            =
                (\nabla f)^{-1}(y)
    \end{alignat}
    が成り立つ。
    よって$(\nabla f)^{-1} = \nabla f^\vee$である。

    \uline{(5)} \quad
    (4)より
    \begin{alignat}{1}
        (\Hess f^\vee)_y
            &=
                d(\nabla f^\vee)_y
                \\
            &=
                d((\nabla f)^{-1})_y
                \\
            &=
                (d(\nabla f)_x)^{-1}
                \\
            &=
                ((\Hess f)_x)^{-1}
    \end{alignat}
    となる。
\end{proof}

\begin{corollary}[Legendre 変換の対合性]
    $f^{\vee \vee} = f$.
\end{corollary}

\begin{proof}
    Legendre 変換の定義より、
    すべての$x \in U$に対し
    \begin{alignat}{1}
        f^{\vee\vee}(x)
            &=
                \myangle{
                    x
                }{
                    (\nabla f^\vee)^{-1}
                    (x)
                }
                - f^\vee(
                    (\nabla f^\vee)^{-1}
                    (x)
                )
                \\
            &=
                \myangle{
                    x
                }{
                    \nabla f
                    (x)
                }
                - f^\vee(
                    \nabla f
                    (x)
                )
                \qquad
                (\nabla f^\vee = (\nabla f)^{-1})
                \\
            &=
                \myangle{
                    x
                }{
                    \nabla f
                    (x)
                }
                - \myparen{
                    \myangle{
                        \nabla f
                        (x)
                    }{
                        (\nabla f)^{-1}(
                            \nabla f
                            (x)
                        )
                    }
                    - f(
                        (\nabla f)^{-1}(
                            \nabla f
                            (x)
                        )
                    )
                }
                \\
            &=
                f(x)
    \end{alignat}
    が成り立つ。
    よって$f^{\vee\vee} = f$である。
\end{proof}

% ------------------------------------------------------------
%
% ------------------------------------------------------------
\section{Fourier-Laplace 変換}

\TODO{ちゃんと書く。cf. \cite{BN78}}

\begin{definition}[Fourier-Laplace 変換]
    $V$を有限次元$\R$-ベクトル空間、
    $\mu$を$V$上の測度とする。
    \begin{equation}
        L_\mu(\theta)
            \coloneqq
                \int_{v \in V}
                    e^{\myangle{\theta}{v}}
                    \, d\mu(v)
                \quad
                (\theta \in V^\vee \otimes \C)
    \end{equation}
    と定め、$L_\mu$を
    \term{Fourier-Laplace 変換}[Fourier-Laplace transform]
        {Fourier-Laplace 変換}[Fourier-Laplace へんかん]
    という。
\end{definition}


\end{document}