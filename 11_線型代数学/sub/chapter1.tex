\documentclass[report]{jlreq}
\usepackage{global}
\usepackage{./local}
\subfiletrue
\def\assetspath{../}
\begin{document}

% ============================================================
%
% ============================================================
\chapter{全行列環}

\TODO{ここでは環というより加群としての性質に注目する?}

一般の環の全行列環について調べる。

% ------------------------------------------------------------
%
% ------------------------------------------------------------
\section{Cayley-Hamilton}

\TODO{}

% ------------------------------------------------------------
%
% ------------------------------------------------------------
\section{分解}

\TODO{QR分解は Gram-Schmidt の直交化と等価?}



% ------------------------------------------------------------
%
% ------------------------------------------------------------
\section{行列の相似}

行列の相似の概念を定義する。

\begin{definition}[相似]
    \term{相似}[similar]{相似}[そうじ]
    \begin{equation}
        UAU^{-1} = B
    \end{equation}
    \TODO{}
\end{definition}

\begin{theorem}
    \begin{enumerate}
        \item 対応
            \begin{equation}
                M_n(K) \to \{
                    \text{$K^n$上の$K[X]$-加群構造}
                \},
                \quad
                A \mapsto \down{A}K^n
            \end{equation}
            は全単射である。
        \item $A, B \in M_n(K)$が相似であることを
            $A \sim B$と書くことにすると、
            (1) の全単射は全単射
            \begin{equation}
                M_n(K) / \sim \to \{
                    \text{$K^n$上の$K[X]$-加群構造}
                \} / \cong
            \end{equation}
            を引きおこす。
    \end{enumerate}
\end{theorem}

\begin{proof}
    (1) は明らか。

    (2)
    \TODO{}
\end{proof}

$A$がブロック対角行列と相似ならば、
$\down{A}K^n$の$K[X]$-加群構造は
ブロックごとの部分的な構造に分解することができる。
とくに後で定義する Jordan 標準形はこのようなブロック対角行列の形になっている。

\begin{lemma}
    $K$を体、
    $n = n_1 + n_2$、
    $A_1 \in M_{n_1}(K), \; A_2 \in M_{n_2}(K)$、
    $A \sim \begin{pmatrix}
        A_1 & 0 \\
        0 & A_2
    \end{pmatrix} \in M_n(K)$とする。
    このとき
    \begin{equation}
        \down{A}K^n \cong \up{A_1}K^{n_1} \oplus \up{A_2}K^{n_2}
    \end{equation}
    が成り立つ。
\end{lemma}

\begin{proof}
    \TODO{}
\end{proof}

% ------------------------------------------------------------
%
% ------------------------------------------------------------
\section{行列式}

\begin{definition}[行列式]
    \idxsym{determinant}{$\det B$}{行列$B$の行列式}
    $R$を可換環、
    $B = (b_{ij})_{i, j} \in M_n(R)$とする。
    $B$の\term{行列式}[determinant]{行列式}[ぎょうれつしき]
    $\det B$を
    \begin{equation}
        \det B \coloneqq
            \sum_{\sigma \in \calS_n} \sgn(\sigma)
            b_{\sigma(1)1} \cdots b_{\sigma(n)n}
            \in R
    \end{equation}
    で定義する。
\end{definition}

\begin{definition}[小行列式]
    \begin{itemize}
        \item \TODO{小行列式 (minor)}
        \item \TODO{主小行列式 (principal minor)}
        \item \TODO{首座小行列式 (leading principal minor)}
    \end{itemize}
\end{definition}

行列の基本変形と同様に、
行列式にも基本変形がある。
ただし、行列の基本変形と異なり
基本変形によって係数が現れる場合があるため、
計算には注意が必要である。

\begin{proposition}[行列式の基本変形]
    \TODO{}
\end{proposition}

\begin{proof}
    \TODO{}
\end{proof}

\begin{definition}[余因子行列]
    \idxsym{adjugate matrix}{$\adj(B)$}{行列$B$の余因子行列}
    $R$を可換環、
    $B = (b_{ij})_{i, j} \in M_n(R)$とする。
    $B$の\term{余因子行列}[adjugate matrix]{余因子行列}[よいんしぎょうれつ]
    $\adj(B)$を
    \begin{equation}
        \adj(B) \coloneqq (c_{ij})_{i, j},
        \quad
        c_{ij} \coloneqq (-1)^{i + j} \det (b_{\highlight{j' i'}})_{i, j}
    \end{equation}
    で定義する。
    ただし$(b_{j' i'})_{i, j}$とは
    $j$行目と$i$列目の成分を除いた$n - 1$次正方行列の意味である。
    $c_{ij}$を\term{$(i, j)$-余因子}[$(i, j)$-th cofactor]{余因子}[よいんし]
    という。
\end{definition}

\begin{proposition}
    $R$を可換環、$B = (b_{ij})_{i, j} \in M_n(R)$とする。
    このとき
    \begin{equation}
        B \adj(B) = \det(B) I = \adj(B) B
    \end{equation}
    が成り立つ。
\end{proposition}

\begin{proof}
    \TODO{}
\end{proof}



% ============================================================
%
% ============================================================
\chapter{PID 上の全行列環}

PID 上の全行列環について調べる。

% ------------------------------------------------------------
%
% ------------------------------------------------------------
\section{単因子論}

単因子論について述べる。

\begin{lemma}
    $R$をPID、
    $x, y, r \in R^\times$が
    $(x, y) = (r)$をみたすとする。
    このとき、ある$a, b \in R$で
    $(a, b) = (1)$かつ$ax + by = r$となるものが存在する。
    したがって、ある$c, d \in R$が存在して
    \begin{equation}
        \det\begin{pmatrix}
            a & b \\
            c & d
        \end{pmatrix}
            = \det\begin{pmatrix}
                a & c \\
                b & d
            \end{pmatrix}
            = 1
    \end{equation}
    となる。
\end{lemma}

\begin{proof}
    $r \in (x, y)$だから
    $ax + by = r$なる$a, b \in R$が存在する。
    いま$R$はPIDだから$(a, b) = (t)$なる$t \in R$が存在する。
    $(x, y) = (r)$と
    $r = ax + by \in (t)$とをあわせて
    $x, y \in (t)$である。
    このことと$(a, b) = (t)$より
    $r = ax + by \in (t^2)$である。
    同様の議論を繰り返して$r \in \bigcap_{n \ge 1} (t^n)$となる。
    ここで$(t) \subsetneq R$であったとすると
    $t \in R - R^\times$だから
    \cref{thm:PID-is-UFD}の(2)より$r = 0$となり$r \in R^\times$に矛盾。
    よって$(a, b) = (1)$が成り立つ。
    したがって$ad - bc = 1$なる$c, d \in R$が存在する。
\end{proof}

行列の対等と標準形を定義する。

\begin{definition}[対等と標準形]
    \idxsym{equivalent matrices}{$X \sim Y$}{対等な行列}
    $R$をPID、
    $m, n \in \Z_{\ge 1}$とする。
    \begin{itemize}
        \item $X, Y \in M_{m, n}(R)$が
            \term{対等}[equivalent]{対等}[たいとう]であるとは、
            ある$A \in GL_m(R), \; B \in GL_n(R)$が存在して
            $AXB = Y$が成り立つことをいう。
            すなわち、行および列基本変形で互いに移り合うということである。
            $X, Y$が対等であることを$X \sim Y$と書く。
        \item $(x_{ij})_{i, j} \in M_{m, n}(R)$が
            \term{標準形}[canonical form]{標準形}[ひょうじゅんけい]
            であるとは、
            \begin{equation}
                x_{ij} = 0 \quad (i \neq j),
                \quad
                x_{ii} \mid x_{jj} \quad (i < j)
            \end{equation}
            が成り立つことをいう。
            すなわち、対角成分が右下を順に割り切る対角行列である。
    \end{itemize}
\end{definition}

$R$が Euclid 整域の場合は
以下の手順で標準形を具体的に求めることができる。

\begin{example}[標準形の計算]
    \TODO{割り切れる列があったら割り切ってしまってよいのでは?}
    \begin{alignat}{1}
        \begin{pmatrix}
            6 & 8 & 6 \\
            4 & 3 & 6
        \end{pmatrix}
            &\sim \begin{pmatrix}
                3 & 4 & 6 \\
                8 & 6 & 6
            \end{pmatrix}
                \quad (\text{最小次数と入れ替え}) \\
            &\sim \begin{pmatrix}
                3 & 1 & 6 \\
                8 & -2 & 6
            \end{pmatrix}
                \quad (\text{列2 \% 列1}) \\
            &\sim \begin{pmatrix}
                3 & 1 & 6 \\
                2 & -4 & -6
            \end{pmatrix}
                \quad (\text{行2 \% 行1}) \\
            &\sim \begin{pmatrix}
                1 & 3 & 6 \\
                -4 & 2 & -6
            \end{pmatrix}
                \quad (\text{最小次数と入れ替え})
    \end{alignat}

    なお、計算機を使う場合は
    \href{https://docs.sympy.org/latest/modules/matrices/normalforms.html#sympy.matrices.normalforms.smith_normal_form}
        {SymPy}
    で計算できる。
\end{example}

次の補題は単因子論の基本定理1の証明に用いる。

\begin{lemma}
    $R$をPIDとする。
    $Y \in M_{m, n}(R)$に対し、
    $Y \sim Z$なる$Z \in M_{m, n}(R)$であって次をみたすものが存在する:
    \begin{enumerate}
        \item $Z$は次の形である:
            \begin{equation}
                Z = \begin{bmatrix}
                    z_{11} & 0 & \cdots & 0 \\
                    0 & z_{22} & \cdots & z_{2n} \\
                    \vdots & \vdots & \ddots & \vdots \\
                    0 & z_{m2} & \cdots & z_{mn}
                \end{bmatrix}
            \end{equation}
        \item $z_{11} \mid z_{ij} \quad (i, j \ge 2)$
    \end{enumerate}
\end{lemma}

\begin{proof}
    \TODO{}
\end{proof}

\begin{theorem}[単因子論の基本定理1]
    \label[theorem]{thm:elementary-divisors-1}
    $R$をPIDとする。
    各$Y \in M_{m, n}(R)$に対し、
    ある標準形$X \in M_{m, n}(R)$が存在して$X \sim Y$となる。
\end{theorem}

\begin{proof}
    \TODO{}
\end{proof}

次に単因子論の基本定理2の証明のための補題を述べる。

\begin{lemma}
    $R$をPID、
    $X \in M_{m, n}(R), \;
        A \in M_m(R), \;
        B \in M_n(R), \;
        1 \le r \le \min\{m, n\}$とする。
    \begin{enumerate}
        \item $h_1$を$AX$のある$r \times r$小行列式とすると、
            $h_1$は$X$のいくつかの$r \times r$小行列式の
            $R$-係数1次結合である。
        \item $h_2$を$XB$のある$r \times r$小行列式とすると、
            $h_2$は$X$のいくつかの$r \times r$小行列式の
            $R$-係数1次結合である。
    \end{enumerate}
\end{lemma}

\begin{proof}
    \TODO{}
\end{proof}

小行列式で生成されるイデアルの記号を導入しておく。

\begin{definition}
    $R$をPIDとする。
    $X \in M_{m, n}(R), \; 1 \le r \le \min\{m, n\}$に対し、
    $X$のすべての$r$次小行列式で生成された$R$のイデアルを
    $\delta_r(X)$で表す。
\end{definition}

\begin{theorem}
    $R$をPIDとする。
    $X \in M_{m, n}(R)$に対し、
    \begin{enumerate}
        \item $r \le r'$ならば$\delta_r(X) \supset \delta_{r'}(X)$である。
        \item $X \sim Y$ならば$\delta_r(X) = \delta_r(Y)$である。
    \end{enumerate}
\end{theorem}

\begin{proof}
    \TODO{}
\end{proof}

\begin{theorem}[単因子論の基本定理2]
    $R$をPIDとする。
    \begin{enumerate}
        \item $X \in M_{m, n}(R)$が標準形ならば、
            各$1 \le r \le \min\{m, n\}$に対し
            \begin{equation}
                (x_{11} \cdots x_{rr}) \in \delta_r(X)
            \end{equation}
            が成り立つ。
        \item $X, Y \in M_{m, n}(R)$が対等な標準形ならば、
            ある$c_1, \dots, c_{\min\{m, n\}} \in R^\times$が存在して
            \begin{equation}
                x_{ii} = c_i y_{ii}
                    \quad
                    (1 \le i \le \min\{m, n\})
            \end{equation}
            が成り立つ。
    \end{enumerate}
\end{theorem}

\begin{proof}
    \TODO{}
\end{proof}

行列の単因子は、
標準形の各対角成分により生成されるイデアルの列である。

\begin{definition}[単因子]
    $R$をPID、
    $X \in M_{m, n}(R)$とする。
    $X \sim X' = (x'_{ij})_{i, j}$なる標準形$X'$をひとつ選び、
    各$1 \le r \le \min\{m, n\}$に対し
    \begin{equation}
        e_r(X) \coloneqq (x'_{rr}) \subset R
    \end{equation}
    とおく。
    $e_1(X), \dots, e_{\min\{m, n\}}(X)$を
    $X$の\term{単因子}[elementary divisor]{単因子!行列の---}[たんいんし]
    という。
    これは$X'$の選び方によらない。
\end{definition}



% ------------------------------------------------------------
%
% ------------------------------------------------------------
\section{PID上の有限生成加群の構造定理}

PID上の有限生成加群の構造定理について述べる。

\subsection{構造定理}

\begin{theorem}[単因子分解]
    $R$をPID、$M$を有限生成$R$-加群とする。
    このとき、ある$s \in \Z_{\ge 0}$と
    $R$の$0$でない固有イデアル
    $e_1 \supset e_2 \supset \cdots \supset e_k \; (k \in \Z_{\ge 0})$
    が一意に存在して
    \begin{equation}
        M \cong \down{R}R^{\oplus s}
            \oplus (\down{R}R / e_1)
            \oplus (\down{R}R / e_2)
            \oplus \cdots
            \oplus (\down{R}R / e_k)
    \end{equation}
    をみたす。
    $e_1, \dots, e_k$を$M$の
    \term{単因子}[elementary divisor]{単因子!加群の---}[たんいんし]という。
\end{theorem}

\begin{proof}
    存在を示す。
    $M$は$R$上有限生成だから
    全射$R$-加群準同型$\Phi \colon R^{\oplus m} \to M$が存在する。
    また、$R$はPIDゆえにネーター環だから
    \cref{thm:fg-module-over-noetherian-ring-is-noetherian}より
    $R^{\oplus m}$はネーター加群である。
    よって$R^{\oplus m}$の$R$-部分加群である
    $\Ker \Phi$は有限生成であり、
    $\lMod{R}$の完全列
    \begin{equation}
        \begin{tikzcd}
            R^{\oplus m}
                \ar{r}{\Psi}
                & R^{\oplus n}
                    \ar{r}{\Phi}
                & M
                    \ar{r}
                & 0
        \end{tikzcd}
    \end{equation}
    を得る。
    $R^{\oplus m}$と$R^{\oplus n}$の
    標準基底に関する$\Psi$の行列表示を
    $A \in M_{m, n}(R)$とおく。
    単因子論の基本定理1 (\cref{thm:elementary-divisors-1}) より、
    $r \coloneqq \min\{m, n\}$として
    \begin{equation}
        A \sim \begin{bmatrix}
            e'_1 \\
            & \ddots & & \text{\Large $0$} \\
            && e'_r
        \end{bmatrix},
        \quad
        e'_1 \mid e'_2 \mid \cdots \mid e'_r
    \end{equation}
    が成り立つ。
    $m \ge n$の場合、
    $\Im \Psi = (e'_1) \oplus \dots \oplus (e'_n)$だから
    準同型定理より
    \begin{equation}
        M \cong (\down{R}R / (e'_1)) \oplus \dots \oplus (\down{R}R / (e'_n))
    \end{equation}
    が成り立つ。
    $m < n$の場合、
    $\Im \Psi
        = (e'_1) \oplus \dots \oplus (e'_m)
        \oplus \underbrace{0 \oplus \cdots \oplus 0}_{n - m}$
    だから準同型定理より
    \begin{equation}
        M \cong (\down{R}R / (e'_1)) \oplus \dots \oplus (\down{R}R / (e'_m))
            \oplus \down{R}R^{\oplus (n - m)}
    \end{equation}
    が成り立つ。

    \TODO{}
\end{proof}

\begin{corollary}[有限生成アーベル群の場合]
    \TODO{}
\end{corollary}

\begin{corollary}
    $R$をPIDとする。
    $R$-加群$M$が有限生成かつねじれなしならば
    自由$R$-加群である。
    とくに射影的かつ有限生成ならば自由である。
    \TODO{why?}
\end{corollary}

\begin{theorem}[準素分解]
    \TODO{}
\end{theorem}

\begin{proof}
    \TODO{}
\end{proof}



% ============================================================
%
% ============================================================
\chapter{体上の全行列環}

% ------------------------------------------------------------
%
% ------------------------------------------------------------
\section{基本変形}

\TODO{米田の補題の現れ?}

% ------------------------------------------------------------
%
% ------------------------------------------------------------
\section{階数}

\TODO{}

% ------------------------------------------------------------
%
% ------------------------------------------------------------
\section{掃き出し法}

\TODO{Gauss の消去法と、逆行列を用いた解法との比較?}

% ------------------------------------------------------------
%
% ------------------------------------------------------------
\section{1次方程式系}

本節では1次方程式系
\begin{equation}
    \label[equation]{eq:linear-system}
    \begin{cases}
        a_{11} X_1 + \cdots + a_{1n} X_n &= c_1 \\
        \vdots \\
        a_{m1} X_1 + \cdots + a_{mn} X_n &= c_m
    \end{cases}
\end{equation}
を考える。

\begin{definition}[係数行列]
    $m \times n$行列$A$
    \begin{equation}
        A \coloneqq \begin{bmatrix}
            a_{11} & \cdots & a_{1n} \\
            \vdots & \ddots & \vdots \\
            a_{m1} & \cdots & a_{mn}
        \end{bmatrix}
    \end{equation}
    を1次方程式系\cref{eq:linear-system}の
    \term{係数行列}[coefficient matrix]{係数行列}[けいすうぎょうれつ]という。
    また、$m \times (n + 1)$行列$\wt{A}$
    \begin{equation}
        \wt{A} \coloneqq \begin{bmatrix}
            a_{11} & \cdots & a_{1n} & c_1 \\
            \vdots & \ddots & \vdots & \vdots \\
            a_{m1} & \cdots & a_{mn} & c_m
        \end{bmatrix}
    \end{equation}
    を1次方程式系\cref{eq:linear-system}の
    \term{拡大係数行列}[extended coefficient matrix]
        {拡大係数行列}[かくだいけいすうぎょうれつ]
    という。
\end{definition}

\subsection{斉次1次方程式系}

\TODO{}



% ============================================================
%
% ============================================================
\chapter{ベクトル空間}

この章では、有限次元とは限らない一般のベクトル空間について述べる。
ただし、加群の一般論から直ちに導かれる事実については証明を省略し、
詳細は代数学の資料に譲る。

% ------------------------------------------------------------
%
% ------------------------------------------------------------
\section{ベクトル空間}

ベクトル空間とは体上の加群のことであった。

\begin{definition}[ベクトル空間]
    体$K$上の加群$V$を
    \term{$K$-ベクトル空間}[$K$-vector space]{ベクトル空間}[べくとるくうかん]
    という。
\end{definition}

% ------------------------------------------------------------
%
% ------------------------------------------------------------
\section{線型写像}

\begin{definition}[線型写像]
    $K$-ベクトル空間の間の$K$-加群準同型を
    \term{$K$-線型写像}[$K$-linear map]{線型写像}[せんけいしゃぞう]
    ともいう。
\end{definition}

\TODO{完全列の分裂に関する話?}

% ------------------------------------------------------------
%
% ------------------------------------------------------------
\section{基底}

\begin{definition}[基底]
    \TODO{}
\end{definition}

\begin{remark}
    本稿では「基底」といったら順序基底のことを指すものとする。
\end{remark}

% ------------------------------------------------------------
%
% ------------------------------------------------------------
\section{固有値と固有ベクトル}

ベクトル空間$V$に対しては、
自己準同型$f$をひとつ固定すると
次のように$K[X]$-加群構造が入る。

\begin{definition}
    \label[definition]{def:endo-module}
    \idxsym{vector space on which polynomials act}
        {$\down{f}K^n$}{$K^n$に$K[X]$-加群構造を定めたもの}
    $V$を$K$-ベクトル空間、
    $f$を$V$の自己準同型とする。
    $V$に対し$K[X]$-加群構造を
    \begin{equation}
        X v \coloneqq f(v)
            \quad (v \in V)
    \end{equation}
    で定めたものを$\down{f}V$と書くことにする。
\end{definition}

\begin{definition}[固有値と固有ベクトル]
    $V$を$K$-ベクトル空間、
    $\varphi$を$V$の自己準同型とする。
    $\alpha \in K$について、
    $\varphi(v) = \alpha v$なる$v \in V - \{ 0 \}$
    が存在するとき、$\alpha$を$\varphi$の
    \term{固有値}[eigenvalue]{固有値}[こゆうち]
    といい、
    $v$を$\varphi$の固有値$\alpha$に属する
    \term{固有ベクトル}[eigenvector]{固有ベクトル}[こゆうべくとる]
    という。
\end{definition}

\begin{definition}[固有空間]
    $V$を$K$-ベクトル空間、
    $f$を$V$の自己準同型とする。
    \begin{equation}
        V_{\alpha} \coloneqq \Ker (f - \alpha \id_V)
    \end{equation}
    を$f$の固有値$\alpha$に属する
    \term{固有空間}[eigenspace]{固有空間}[こゆうくうかん]
    という。
\end{definition}

固有空間の概念を拡張し、
一般固有空間の概念を定義する。

\begin{definition}[一般固有空間]
    \idxsym{generalized eigenspace}
        {$V_{[\alpha]}$}{$\alpha$に関する$V$の一般固有空間}
    $V$を$K$-ベクトル空間、
    $f$を$V$の自己準同型とする。
    $\down{f}V$と$\alpha \in K$に対し
    \begin{equation}
        V_{[\alpha]} \coloneqq \{
            v \in V
            \mid
            \exists n \in \Z_{\ge 1}
            \; \text{s.t.} \;
            (X - \alpha)^n v = 0
        \}
    \end{equation}
    は$V_{[\alpha]}$は$\down{f}V$の$K[X]$-部分加群となり、
    これを$f$の固有値$\alpha$に属する
    \term{一般固有空間}[generalized eigenspace]{一般固有空間}[いっぱんこゆうくうかん]
    という。
    $V_{[\alpha]}$の元$v$を ($0$も含めて)
    \term{一般固有ベクトル}[generalized eigenvector]{一般固有ベクトル}[いっぱんこゆうべくとる]
    といい、
    $f^k(v) \neq 0, \; f^{k + 1}(v) = 0$なる$k \in \Z_{\ge 0}$を
    $v$の\term{階数}{階数!一般固有ベクトルの---}[かいすう]
    という。
\end{definition}





% ============================================================
%
% ============================================================
\chapter{有限次元ベクトル空間}

有限次元ベクトル空間について述べる。
一般の次元のベクトル空間と比較して、
有限次元ベクトル空間の理論には以下のような特徴がある。
\begin{itemize}
    \item 次元の概念が重要:
        有限次元ベクトル空間では、次元の概念が非常に重要な役割を果たす。
    \item 線型写像の行列による表現:
        有限次元ベクトル空間上の線型写像に対しては、行列による表現が存在する。
        この表現は、線型写像の性質を行列演算に帰着することができるため、計算上便利である。
    \item 標準基底による同型:
        有限次元ベクトル空間は、固定した基底のもとで座標空間と同型になる。
        これにより、ベクトル空間上の問題を座標空間上の問題に帰着することができる。
    \item 有限生成であること:
        有限次元ベクトル空間はとくに有限生成であるため、
        有限次元ベクトル空間上の線型方程式の解の存在性や一意性を議論することができる
        \TODO{どういうこと?}。
\end{itemize}

これらの特徴によって有限次元ベクトル空間の理論は非常に発展しており、
応用上も非常に重要である。

% ------------------------------------------------------------
%
% ------------------------------------------------------------
\section{次元}

有限次元ベクトル空間では、次元の概念が非常に重要な役割を果たす。

\begin{proposition}
    \label[proposition]{prop:isom-inj-surj}
    $V, W$を次元の等しい有限次元$K$-ベクトル空間とする。
    このとき、$K$-線型写像$f \colon V \to W$に対し
    次は同値である:
    \begin{enumerate}
        \item $f$は同型である。
        \item $f$は単射である。
        \item $f$は全射である。
    \end{enumerate}
\end{proposition}

\begin{proof}
    \TODO{}
\end{proof}

\begin{proposition}[次元と部分空間]
    $\dim_K W = \dim_K V$と$W = V$は同値
    \TODO{}
\end{proposition}

\begin{proof}
    \TODO{}
\end{proof}

\begin{proposition}[商ベクトル空間の基底]
    $V$を有限次元$K$-ベクトル空間、
    $W$を$V$の部分空間とする。
    $x_1, \dots, x_n$が$V$の基底で
    $x_1, \dots, x_m$が$W$の基底ならば
    $x_{m + 1}, \dots, x_n$の像は$V / W$の基底である。
\end{proposition}

\begin{proof}
    \TODO{}
\end{proof}

% ------------------------------------------------------------
%
% ------------------------------------------------------------
\section{行列表示}

\begin{definition}[行列表示]
    $m, n \in \Z_{\ge 0}$、
    $V, W$をそれぞれ$m, n$次元$K$-ベクトル空間、
    $f \colon V \to W$を$K$-線型写像とする。
    $V$の基底$\calV \coloneqq (v_1, \dots, v_m)$と
    $W$の基底$\calW \coloneqq (w_1, \dots, w_n)$を任意に固定すると、
    $f(v_j) = \sum_{i = 1}^n a_{ij} w_i$
    なる$a_{ij} \in K \; (1 \le i \le n, \; 1 \le j \le m)$が
    一意に定まる。
    この行列$(a_{ij})_{i, j} \in M_n(K)$を
    $V$の基底$\calV$と$W$の基底$\calW$に関する
    $f$の\term{行列表示}[matrix representation]{行列表示}[ぎょうれつひょうじ]
    という。
\end{definition}

\begin{proposition}[自己準同型環と全行列環の同型]
    \label[proposition]{prop:isom-between-end-and-mat}
    $m \in \Z_{\ge 0}$、
    $V$を$m$次元$K$-ベクトル空間、
    $f \colon V \to V$を自己準同型とする。
    $V$の任意の基底$\calV \coloneqq (v_1, \dots, v_m)$に対し、
    各$f \in \End_K(V)$を
    $\calV$に関する$f$の行列表示$A_f \in M_m(K)$に写す写像
    \begin{equation}
        \Phi \colon \End_K(V) \to M_m(K),
            \quad
            f \mapsto A_f
    \end{equation}
    は$K$-代数の同型を与える。
\end{proposition}

\begin{proof}
    逆写像$\Psi$は自由加群の普遍性により得られる。
    あとは$\Phi$が$K$-代数準同型であることの定義を確認すればよい。
    \TODO{}
\end{proof}

% ------------------------------------------------------------
%
% ------------------------------------------------------------
\section{最小多項式}

最小多項式を導入する。
最小多項式は、ベクトル空間の自己準同型を調べるために用いられる。

\begin{propdef}[最小多項式]
    $K$を体、
    $n \in \Z_{\ge 1}$、
    $A \in M_n(K)$とし、
    $K^n$を\cref{def:endo-module}のように
    $K[X]$-加群とみなして$\down{A}K^n$と書く。
    このとき、
    $\Ann_{K[X]}(\down{A}K^n)$のモニックな生成元$\varphi_A$がただひとつ存在する。
    $\varphi_A$を$A$の
    \term{最小多項式}[minimal polynomial]{最小多項式}[さいしょうたこうしき]
    という。
\end{propdef}

\begin{proof}
    $K[X]$はPIDだから
    そのイデアル$\Ann_{K[X]}(\down{A}K^n)$は単項イデアルである。
    よってその生成元が存在する。
    一意性はモニックの条件から従う。
\end{proof}

\begin{example}[最小多項式の計算]
    $A$の最小多項式を具体的に計算するには、
    $A$の固有多項式の約元を計算してみるという方法がある。
    たとえば$\Phi_A(X) = (X - 3)(X - 1)^3$なら
    $(A - 3I)(A - I), \; (A - 3I)(A - I)^2$
    を計算してみればよい。
    いずれも$O$でないならば
    最小多項式は$\Phi_A(X)$に一致する。
\end{example}

最小多項式の定義から明らかに次がわかる。

\begin{proposition}
    $f(X) \in K[X]$が
    $f(A) = O$をみたすならば
    $\varphi_A \mid f$である。
\end{proposition}

\begin{proof}
    Annihilator の定義から明らか。
\end{proof}

Jordan 細胞を定義する。

\begin{definition}[Jordan 細胞]
    \begin{equation}
        J_n(\alpha) = \begin{bmatrix}
            \alpha & 1 & \dots & 0 \\
            0 & \alpha & \dots & 0 \\
            \vdots & \vdots & \ddots & \vdots \\
            0 & 0 & \dots & \alpha
        \end{bmatrix}
    \end{equation}
    \TODO{}
\end{definition}

行列$A$が Jordan 細胞の場合、
$\down{A}K^n$の$K[X]$-加群構造は非常に単純な形になる。

\begin{proposition}
    $K$を体、
    $\alpha \in K$とする。
    $M \coloneqq K[X] / ((X - \alpha)^n)$を$K[X]$-加群とみなす。
    すると
    \begin{equation}
        \wb{1}, \wb{X - \alpha}, \wb{(X - \alpha)^2},
            \dots, \wb{(X - \alpha)^{n - 1}}
    \end{equation}
    は$M$の$K$上の基底となり、
    これにより$M \cong K^n$とすると
    $K[X]$-加群として
    $M \cong \up{J_n(\alpha)} K^n$
    が成り立つ。
\end{proposition}

\begin{proof}
    \TODO{}
\end{proof}

有限次元の場合、最小多項式を用いて
一般固有空間を簡単な形に表すことができる。
\TODO{計算上はそれほど簡単というわけでもないか?}

\begin{proposition}[最小多項式の根の重複度と一般固有空間]
    $V$を有限次元$K$-ベクトル空間、
    $f \in \End(V)$とする。
    このとき、$f$の任意の固有値$a \in K$に対し
    \begin{equation}
        V_{[a]} = \Ker (f - a\id)^d
    \end{equation}
    が成り立つ。
    ただし$d$は$f$の最小多項式の根としての$a$の重複度である。
\end{proposition}

\begin{proof}
    \TODO{}
\end{proof}

% ------------------------------------------------------------
%
% ------------------------------------------------------------
\section{固有多項式}

\begin{definition}[固有多項式]
    $K$を体とする。
    \begin{itemize}
        \item $A \in M_n(K)$とする。
            $XI - A \in M_n(K[X])$の行列式
            $\det (XI - A) \in K[X]$を$A$の
            \term{固有多項式}[characteristic polynomial]{固有多項式}[こゆうたこうしき]
            という。
        \item $V$を有限次元$K$-ベクトル空間、
            $f$を$V$の自己準同型とする。
            $V$の基底をひとつ選んで、それに関する$f$の行列表示を$A$とおき、
            $A$の固有多項式を
            $f$の\term{固有多項式}[characteristic polynomial]{固有多項式}[こゆうたこうしき]
            と呼ぶ。
            これは$V$の基底の選び方によらず定まる (このあと示す)。
    \end{itemize}
\end{definition}

\begin{proof}
    $A, B \in M_n(K)$が共役のとき
    $A, B$の固有多項式が一致することを示せばよい。
    \TODO{}
\end{proof}

次の命題により、固有値とは固有多項式の根であると言うことができる。
証明は同値変形を繰り返す形にも書けるが、
次元の有限性が効いてくる部分を明らかにするため必要性と十分性をそれぞれ示す。

\begin{proposition}[固有値と固有多項式の関係]
    $V$を有限次元$K$-ベクトル空間、
    $f$を$V$の自己準同型とする。
    $a \in K$に関し次は同値である:
    \begin{enumerate}
        \item $a$は$f$の固有値である。
        \item $a$は$f$の固有多項式の根である。
    \end{enumerate}
\end{proposition}

\begin{proof}
    \uline{(1) \Rightarrow (2)} \quad
    $a$が$f$の固有値であるとする。
    固有値の定義より、$a$が$f$の固有値であることは
    ある$v \in V - \{ 0 \}$が存在して
    $f(v) = av$、すなわち$(a \id_V - f)(v) = 0$が成り立つことと同値である。
    よって$\Ker (a \id_V - f) \neq 0$だから
    $a \id_V - f$は単射でなく、とくに同型でない。
    したがって、$V$の基底をひとつ選ぶと、
    \cref{prop:isom-between-end-and-mat}より
    その基底に関する$f$の行列表示$A_f$に対し
    $aI - A_f$は可逆でない。
    よって$\det (aI - A_f) = 0$であるから、
    因数定理より$a$は$A_f$の固有多項式の根である。
    したがって$f$の固有多項式の根である。

    \uline{(2) \Rightarrow (1)} \quad
    $a$は$f$の固有多項式の根であるとする。
    このとき、$V$の基底をひとつ選び、
    その基底に関する$f$の行列表示を$A_f$とおくと、
    因数定理より$\det (aI - A_f) = 0$である。
    したがって$aI - A_f$は可逆でないから、
    \cref{prop:isom-between-end-and-mat}より
    $a \id_V - f$は同型でない。
    いま$V$は有限次元だから、
    \cref{prop:isom-inj-surj}より
    $a \id_V - f$は単射でない。
    したがって$\Ker (a \id_V - f) \neq 0$である。
    よって、ある$v \in V - \{ 0 \}$が存在して
    $(a \id_V - f)(v) = 0$、すなわち$f(v) = av$が成り立つ。
    よって$a$は$f$の固有値である。
\end{proof}

\begin{definition}[固有値の重複度]
    $V$を有限次元$K$-ベクトル空間、
    $f$を$V$の自己準同型とする。
    $f$の固有値$a \in K$に対し、
    $f$の固有多項式の根としての$a$の重複度を
    固有値$a$の
    \term{重複度}[multiplicity]{重複度!固有値の---}[ちょうふくど]
    あるいは
    \term{代数的重複度}[algebraic multiplicity]{重複度!代数的---}[だいすうてきちょうふくど]
    という。

    $a$に属する固有空間$V_a$の次元を
    固有値$a$の
    \term{幾何学的重複度}[geometric multiplicity]{重複度!幾何学的---}[きかがくてきちょうふくど]
    という。
\end{definition}

\begin{proposition}
    \begin{equation}
        (\text{幾何学的重複度}) \le (\text{代数的重複度})
    \end{equation}
    であり、
    すべての固有値に対して等号が成り立つことと
    対角化可能性は同値である。
\end{proposition}

\begin{proof}
    \TODO{}
\end{proof}

\begin{proposition}[固有値の重複度と一般固有空間の次元]
    \label[proposition]{prop:multiplicity-and-dim-of-generalized-eigenspace}
    $V$を有限次元$K$-ベクトル空間、
    $f$を$V$の自己準同型とする。
    $f$の任意の固有値$a \in K$に対し、
    固有値$a$に属する一般固有空間$V_{[a]}$の次元は
    $\dim_K V_{[a]} = (\text{固有値$a$の重複度})$をみたす。
\end{proposition}

\begin{proof}
    \TODO{cf. \cref[p.117]{齋藤07}}
\end{proof}

\subsection{同伴行列}

\begin{definition}[同伴行列]
    $f(X) = c_0 + c_1 X + \cdots + c_{n-1} X^{n-1} + X^n \in K[X]$を
    モニックな多項式とする。
    行列
    \begin{equation}
        \begin{pmatrix}
            0 & 0 & \cdots & 0 & -c_0 \\
            1 & 0 & \cdots & 0 & -c_1 \\
            0 & 1 & \cdots & 0 & -c_2 \\
            \vdots & \vdots & \ddots & \vdots & \vdots \\
            0 & 0 & \cdots & 1 & -c_{n-1}
        \end{pmatrix}
    \end{equation}
    を$f$の
    \term{同伴行列}[companion matrix]{同伴行列}[どうはんぎょうれつ]
    という。
\end{definition}

\begin{proposition}
    多項式$f(X) \in K[X]$に対し、
    $f$の同伴行列の固有多項式と最小多項式は$f$に一致する。
\end{proposition}

\begin{proof}
    \TODO{}
\end{proof}

% ------------------------------------------------------------
%
% ------------------------------------------------------------
\section{三角化と対角化}

三角化と対角化について述べる。

\begin{definition}[三角化可能]
    \TODO{}
\end{definition}

代数的閉体上の有限次元ベクトル空間では
つねに三角化が可能である。

\begin{proposition}[三角化]
    \label[proposition]{prop:triangularization}
    $K$を代数的閉体、
    $V$を有限次元$K$-ベクトル空間、
    $f \colon V \to V$を$K$-線型写像とする。
    このとき$V$のある基底に関する$f$の行列表示は
    上三角行列となる。
\end{proposition}

\begin{proof}
    次元に関する帰納法で示す。
    $\dim_K V = 1$の場合は明らかに成り立つ。
    $K$は代数的閉体だから$f$の特性多項式は$K$に根を持つ。
    したがって
    $f$の固有値$\lambda \in K$と
    $\lambda$に属する固有ベクトル$v \in V$が存在する。
    $\langle v \rangle$は$V$の直和因子だから
    $V = \langle v \rangle \oplus V'$と直和分解できる。
    $v$に$V'$の基底を付け加えて$V$の基底が得られ、
    この基底に関する$f$の行列表示は
    \begin{equation}
        \begin{bmatrix}
            \lambda & * & \dots & * \\
            0 \\
            \vdots & & A' \\
            0
        \end{bmatrix},
        \quad
        A' \in M_{n - 1}(K)
    \end{equation}
    の形である。
    帰納法の仮定より$A'$の部分が上三角行列となるように
    $V'$の基底を選ぶことができるから、
    これで帰納法が完成する。
\end{proof}

対角行列に相似な行列は対角化可能であるといい、特に重要である。
\TODO{対角化可能性は固有値問題のためにある?}

\begin{definition}[対角化可能]
    $K$を体、
    $n \in \Z_{\ge 1}$、
    $A \in M_n(K)$とする。
    $A$が\term{対角化可能}[diagonalizable]{対角化可能}[たいかくかかのう]
    であるとは、$A$がある対角行列に相似であることをいう。
\end{definition}

対角化可能性は最小多項式を用いて特徴づけることができる。

\begin{theorem}[対角化可能性の特徴づけ]
    $K$を代数的閉体、
    $n \in \Z_{\ge 1}$、
    $A \in M_n(K)$とする。
    このとき次は同値である:
    \begin{enumerate}
        \item $A$は対角化可能である。
        \item $A$の最小多項式$\varphi_A$は重根を持たない。
    \end{enumerate}
\end{theorem}

\begin{proof}
    \TODO{}
\end{proof}

\begin{corollary}
    $A$の固有多項式$\Phi_A$が重根を持たないならば、
    $A$は対角化可能である。
\end{corollary}

\begin{proof}
    $\Phi_A$が重根を持たないならば、
    $\varphi_A$も重根を持たないから、
    定理より$A$は対角化可能である。
\end{proof}

\subsection{直交対角化とユニタリ対角化}

\TODO{}


% ------------------------------------------------------------
%
% ------------------------------------------------------------
\section{スペクトル定理}

\TODO{}


% ------------------------------------------------------------
%
% ------------------------------------------------------------
\section{Jordan 標準形}

この節では Jordan 標準形について述べる。

\begin{lemma}
    $K$を代数的閉体、
    $V$を$K[X]$-加群で$\dim_K V < \infty$なるものとする。
    このとき$\Ann_{K[X]} V \neq 0$である。
\end{lemma}

\begin{proof}
    $n \coloneqq \dim_K V$とおく。
    $X \in K[X]$は$V$上の$K$-線型変換として作用するから、
    \cref{prop:triangularization}より
    $X$の作用の行列表示$(a_{ij})_{i, j} \in M_{n}(K)$を上三角行列にするような
    $V$の基底$e_1, \dots, e_n$が存在する。
    そこで$(X - a_{11}) \cdots (X - a_{nn})$が
    $\Ann_{K[X]} V$に属することを示せばよい。
    そのためにはすべての$1 \le i \le n$に対し
    $(X - a_{11}) \cdots (X - a_{ii}) e_i = 0$が成り立つことを
    示せばよい。
    まず$(X - a_{11}) e_1 = a_{11} e_1 - a_{11} e_1 = 0$である。
    つぎに
    \begin{alignat}{1}
        (X - a_{11}) \cdots (X - a_{ii}) e_i
            &= (X - a_{11}) \cdots (X - a_{i - 1, i - 1})
                \left( \sum_{1 \le j \le i} a_{ji} e_j - a_{ii} e_i \right) \\
            &= (X - a_{11}) \cdots (X - a_{i - 1, i - 1})
                \sum_{1 \le j \le i - 1} a_{ji} e_j
    \end{alignat}
    であり、帰納法の仮定より右辺は$0$である。
    よって帰納法が完成した。
    したがって$(X - a_{11}) \cdots (X - a_{nn})$は
    $\Ann_{K[X]} V$に属し、
    とくに$\Ann_{K[X]} V \neq 0$である。
\end{proof}

$K$が代数的閉体ならば、
$K$上の有限次元ベクトル空間において
一般固有分解と呼ばれる概念が定義できる。

\begin{theorem}[一般固有分解]
    $K$を代数的閉体、
    $V$を$K[X]$-加群で$\dim_K V < \infty$なるものとする。
    このとき、ある$\alpha_1, \dots, \alpha_k \in K$が存在して
    \begin{equation}
        V = \bigoplus_{i = 1}^k V_{[\alpha_i]}
    \end{equation}
    が成り立つ。
    この直和分解を
    $V$の\term{一般固有分解}[generalized eigendecomposition]{一般固有分解}[いっぱんこゆうぶんかい]
    という。\TODO{本当に?}
\end{theorem}

\begin{proof}
    \TODO{}
\end{proof}

\begin{lemma}
    $K$を代数的閉体とする。
    $V = V_{[\alpha]}$の単因子は
    $((X - \alpha)^k)$の形
    \TODO{}
\end{lemma}

\begin{proof}
    上の補題より$\Ann_{K[X]} V \neq 0$である。
    \TODO{}
\end{proof}

\begin{corollary}[Jordan 標準形の一意存在定理]
    $K$を代数的閉体とする。
    このとき、任意の$A \in M_n(K)$に対し
    $n_1 + \dots + n_k = n$なる$n_1, \dots, n_k \in \Z_{\ge 1}$と
    $\alpha_1, \dots, \alpha_k \in K$が存在し
    \begin{equation}
        A \sim \begin{bmatrix}
            J_{n_1}(\alpha_1) & \cdots & 0 \\
            \vdots & \ddots & \vdots \\
            0 & \cdots & J_{n_k}(\alpha_k)
        \end{bmatrix}
    \end{equation}
    が成り立つ。
    右辺の表示は Jordan 細胞の並べ替えを除いて一意である。
\end{corollary}

\begin{proof}
    \TODO{}
\end{proof}

% ------------------------------------------------------------
%
% ------------------------------------------------------------
\newpage
\section{演習問題}

\begin{problem}[代数学II 10.124]
    $A$を環、$M$を有限生成$A$-加群とする。
    このとき$M$の任意の直和成分は有限生成であることを示せ。
\end{problem}

\begin{answer}
    $M$は有限生成だから生成元$v_1, \dots, v_n$が存在する。
    $M$の直和分解$M = X \oplus Y$を考え、
    この分解に従って$v_i = x_i + y_i \; (i = 1, \dots, n)$と表す。
    このとき$X$は$x_1, \dots, x_n$で$A$上生成される。
    実際、$M = X + Y$であることより$x \in X \subset M$に対し
    \begin{alignat}{1}
        x &= a_1 v_1 + \cdots + a_n v_n
            \quad (a_i \in A) \\
          &= a_1 x_1 + \cdots + a_n x_n
            + a_1 y_1 + \cdots + a_n y_n
    \end{alignat}
    と表せて、$X \cap Y = 0$より
    $x = a_1 x_1 + \cdots + a_n x_n$となる。
    したがって$X$は$A$上有限生成である。
\end{answer}

\begin{problem}[代数学II 10.125]
    \label[problem]{problem:algebra2-10.125}
    $R$をPIDとするとき、
    有限生成直既約$R$-加群を同型を除いて分類せよ。
\end{problem}

\begin{answer}
    有限生成直既約$R$-加群は、同型を除いて
    次のもので尽くされることを示す:
    \begin{enumerate}
        \item $\down{R}R$
        \item $\down{R}R / (p^n)$ \quad
            ($p$は$R$の既約元、$n \in \Z_{\ge 1}$)
    \end{enumerate}

    $\down{R}R$および$\down{R}R / (p^n)$は有限生成$R$-加群だから、
    直既約であることを示す。
    $\down{R}R$について、
    非自明な直和分解$\down{R}R = I \oplus J$が存在したとすると、
    $0 = I \cap J$であるが、
    $R$が整域であることから$0 \neq IJ = I \cap J$なので矛盾である。
    よって$\down{R}R$は直既約である。
    $\down{R}R / (p^n)$について、
    $R / (p^n)$のイデアルは
    $(p^i) / (p^n) \; (0 \le i \le n)$の形に表せることに注意する。
    \begin{innerproof}
        $I'$を$R / (p^n)$のイデアルとする。
        イデアルの対応原理
        (\cref{thm:ideal-correspondence-principle})
        より、$R$のイデアル$I$であって$(p^n)$を含むものがただひとつ存在して
        $I' = I / (p^n)$と表せる。
        $I$は$R$がPIDであることより$I = (x) \; (x \in R)$と表せる。
        $(p^n) \subset I = (x)$より
        $x \mid p^n$だから、
        $p$が$R$の既約元であることと
        既約元分解の一意性より、ある$0 \le i \le n$が存在して
        $x$は$p^i$の同伴元である。
        したがって$I = (x) = (p^i)$が成り立つ。
        よって$I' = (p^i) / (p^n)$である。
    \end{innerproof}
    そこで$\down{R}R / (p^n)$の直和分解
    $\down{R}R / (p^n) = (p^i) / (p^n) \oplus (p^j) / (p^n)$を考える。
    標準射影$R \to R / (p^n)$を$\pi$とおくと
    \begin{alignat}{1}
        0
            &= (p^i) / (p^n) \cap (p^j) / (p^n) \\
            &= \pi((p^i)) \cap \pi((p^j)) \\
            &= \pi((p^i) \cap (p^j))
                \quad (\text{$(p^i), (p^j)$は$\pi$に関し saturated}) \\
            &= \pi((p^{\min\{i, j\}}))
    \end{alignat}
    より$(p^i) / (p^n) = \pi((p^i)) = 0$
    または$(p^j) / (p^n) = \pi((p^j)) = 0$が成り立つ。
    よって$\down{R}R / (p^n)$は直既約である。

    逆に$M$を有限生成直既約$R$-加群とする。
    構造定理より、ある自由$R$-加群$F$が存在して
    $M \cong M_\tor \oplus F$が成り立つ。
    $M$は直既約だから$M \cong M_\tor$または$M \cong F$である。
    $M \cong F$ならばふたたび$M$が直既約であることより
    $\rank F = 1$だから$M \cong F \cong \down{R}R$を得る。
    $M \cong M_\tor$の場合を考える。
    構造定理より$M_\tor$は$\down{R}R$の商加群の直和だから、
    $M \cong M_\tor$が直既約であることとあわせて
    $M \cong M_\tor \cong \down{R}R / (a) \; (\exists a \in R)$が成り立つ。
    ここで$a \in R^\times$であったとすると
    $M = 0$となり$M$が直既約であることに矛盾するから、
    $a \notin R^\times$である。
    また、$a$がいかなる既約元の冪とも同伴でなかったとすると、
    互いに素な$b, c \in R$を用いて$a = bc$と表せるから、
    中国剰余定理より非自明な直和分解
    $M \cong \down{R}R / (a) \cong \down{R}R / (b) \oplus \down{R}R / (c)$
    が得られ、$M$が直既約であることに矛盾する。
    よってある既約元$p \in R$と$n \in \Z_{\ge 1}$が存在して
    $a$は$p^n$の同伴元である。
    したがって$M \cong \down{R}R / (p^n)$が成り立つ。
    以上で証明が完成した。
\end{answer}

\begin{problem}[代数学II 10.126]
    有限生成でない直既約$\Z$-加群の例を挙げよ。
\end{problem}

\begin{answer}
    $\Q$が求める例であることを示す。
    まず$\Q$は$\Z$上有限生成でない (\cref{lemma:Q-is-not-finitely-generated-over-Z})。
    直和分解$\Q = X \oplus Y$を考える。
    $X \neq 0, \; Y \neq 0$であったとすると、
    ある$x \in X - \{ 0 \}, \; y \in Y - \{ 0 \}$が存在する。
    そこで$x = a / b, \;
        y = c / d, \;
        a, c \in \Z, \;
        b, d \in \Z_{\ge 1}$
    と表すと
    $bcx = ady = ac \neq 0$が成り立つから
    $X \cap Y = 0$に矛盾。
    よって$X, Y$の少なくとも一方は$0$である。
    したがって$\Q$は直既約である。
\end{answer}

\begin{problem}[代数学II 10.128]
    $A$を環、$M$を有限の長さを持つ$A$-加群とすると、
    $M$は有限個の直和成分を持つ直既約分解を持つことを示せ。
\end{problem}

\begin{answer}
    $M$は有限の長さを持つから
    組成列$\{ M_i \}_{i = 0}^n$が存在する。
    完全列
    \begin{equation}
        \begin{tikzcd}
            0
                \ar{r}
                & M_n
                    \ar{r}
                & M_{n - 1}
                    \ar{r}
                & M_{n - 1} / M_n
                    \ar{r}
                & 0
        \end{tikzcd}
    \end{equation}
    について、$M_n = 0$および$M_{n - 1} / M_n$は既約だから
    $M_{n - 1}$も既約である。
    よってこの完全列は分裂し
    $M_{n - 1} \cong M_n \oplus (M_{n - 1} / M_n)$となる。
    $M_n, \; M_{n - 1} / M_n$は既約ゆえに直既約だから
    $M_{n - 1}$は有限個の直和成分への直既約分解を持つ。
    組成列の長さは有限だから、帰納的に
    $M = M_0$は有限個の直和成分への直既約分解を持つ。
\end{answer}

\begin{problem}[代数学II 10.129]
    $R$をPIDとするとき、
    有限生成$R$-加群は有限個の直和成分を持つ直既約分解を持つことを示せ。
\end{problem}

\begin{answer}
    $M$を有限生成$R$-加群とする。
    構造定理より、ある$s \in \Z_{\ge 0}$と
    $R$の$0$でない固有イデアル
    $e_1 \supset e_2 \supset \cdots \supset e_k \; (k \in \Z_{\ge 0})$
    が一意に存在して
    \begin{equation}
        M \cong \down{R}R^{\oplus s}
            \oplus \underbrace{
                (\down{R}R / e_1)
                \oplus (\down{R}R / e_2)
                \oplus \cdots
                \oplus (\down{R}R / e_k)
            }_{\cong M_\tor}
    \end{equation}
    が成り立つ。

    まず$\down{R}R^{\oplus s}$について、
    PID上の有限生成直既約加群の分類
    (\cref{problem:algebra2-10.125})
    より$\down{R}R$は直既約だから、
    $\down{R}R^{\oplus s}$は有限個の直和成分への直既約分解を持つ。

    つぎに$M_\tor$について考える。
    各$e_i$は$0$でない固有イデアルだから、
    $0$でも単元でもない$a_i \in R$が存在して
    $e_i = (a_i)$と表せる。
    ここで一般に、$0$でも単元でもない任意の$a \in R$に対し、
    既約元分解により$a$は
    ある$q_1^{n_1} \cdots q_l^{n_l} \; (q_j \text{ は既約元})$
    と同伴であるが、
    $q_1^{n_1}, \dots, q_l^{n_l}$は互いに素だから
    中国剰余定理より
    \begin{equation}
        \down{R}R / (a)
            \cong (\down{R}R / (q_1^{n_1}))
            \otimes \cdots
            \otimes (\down{R}R / (q_l^{n_l}))
    \end{equation}
    が成り立つ。
    PID上の有限生成直既約加群の分類より
    各$\down{R}R / (q_j^{n_j})$は直既約だから、
    $\down{R}R / (a)$は有限個の直和成分への直既約分解を持つことがわかる。
    したがって各$e_i$に対し
    $\down{R}R / e_i$は有限個の直和成分への直既約分解を持つから、
    $M_\tor$も有限個の直和成分への直既約分解を持つ。

    以上より$M$は有限個の直和成分への直既約分解を持つ。
\end{answer}

\begin{problem}[東大数理 2007A]
    $A \coloneqq \myparen{
        \begin{smallmatrix}
            2 & 1 & 1 \\
            -1 & 2 & -1 \\
            -1 & -1 & 0
        \end{smallmatrix}
    }$とする。以下の問いに答えよ。
    \begin{enumerate}
        \item $A$の固有値および広義固有空間をすべて求めよ。
        \item $AB = 2BA$をみたす任意の3次正方行列$B$は$B^3 = 0$をみたすことを示せ。
        \item $AB = 2BA$をみたす$0$でない3次正方行列$B$を1つ求めよ。
    \end{enumerate}
\end{problem}

\begin{answer}
    \TODO{}
\end{answer}

\begin{problem}[東大数理 2011A]
    複素数$a, b$に対して、3次の複素正方行列$A, B$を次のように定める:
    \begin{equation}
        A \coloneqq \begin{pmatrix}
            2 & 0 & 0 \\
            1 & 1 & a \\
            1 & -a & 3
        \end{pmatrix},
        \quad
        B \coloneqq \begin{pmatrix}
            0 & b & -2 \\
            1 & 1 & 1 \\
            3 & -3 & 5
        \end{pmatrix}
    \end{equation}
    \begin{enumerate}
        \item $A$が対角化可能であるための必要十分条件を求めよ。
        \item $A$と$B$が相似であるための必要十分条件を求めよ。
    \end{enumerate}
\end{problem}

\begin{answer}
    \TODO{}
\end{answer}

\begin{problem}[ChatGPT]
    体$K$上の$O$でも$I$でもない
    $n$ 次正方行列 $A$ が $A^2 = A$ を満たすとき、
    $A$ の固有値は 0 と 1 であることを示せ。
\end{problem}

\begin{answer}
    $\lambda \in K$を$A$の固有値とする。
    $A$の固有値$\lambda$に属する固有ベクトル$v \neq 0$をひとつ選ぶと、
    $A^2 = A$の仮定より$\lambda^2 v = A^2 v = Av = \lambda v$が成り立つ。
    したがって$\lambda (\lambda - 1) v = 0$だから、
    $v \neq 0$より$\lambda (\lambda - 1) = 0$、
    したがって$\lambda = 0$または$\lambda = 1$である。
    逆に$\lambda = 0, 1$はたしかに$A$の固有値であることを示す。
    まず、$A^2 = A, \; A \neq I$より$A$は不可逆だから
    $\det A = 0$である。
    このことと、$\det A$が$A$の固有値すべての積であることをあわせて、
    $A$は固有値$0$をもつことがわかる。
    また、$A \neq O$より$Av \neq 0$なる$v \in K^n \setminus \{ 0 \}$が存在するから、
    $A^2 v = Av$であることとあわせて
    $A$は固有値$1$をもつことがわかる。
    したがって$A$の固有値は$0$と$1$である。
\end{answer}

\begin{problem}[東大数理 2006A]
    4行5列の実数を成分とする行列$A$を考える。
    次のそれぞれの場合について、
    $A$の第1列と第2列と第3列の3つのベクトルが生成する
    $\R^4$の部分空間の次元の可能性をすべて求めよ.
    \begin{enumerate}
        \item $A$に行基本変形を何回か行なって
            \begin{equation}
                B = \begin{pmatrix}
                    1 & 3 & 0 & 1 & 0 \\
                    0 & 0 & 1 & -2 & 0 \\
                    0 & 0 & 0 & 0 & 1 \\
                    0 & 0 & 0 & 0 & 0
                \end{pmatrix}
            \end{equation}
            にできる場合.
        \item $A$に行基本変形と列基本変形を何回か行なって
            \begin{equation}
                C = \begin{pmatrix}
                    1 & 0 & 0 & 0 & 0 \\
                    0 & 1 & 0 & 0 & 0 \\
                    0 & 0 & 1 & 0 & 0 \\
                    0 & 0 & 0 & 1 & 0
                \end{pmatrix}
            \end{equation}
            にできる場合.
    \end{enumerate}
\end{problem}

\begin{answer}
    $A = \myparen{\begin{smallmatrix}
        \bm{a}_1 & \bm{a}_2 & \bm{a}_3 & \bm{a}_4 & \bm{a}_5
        \end{smallmatrix}}$
    とおき、$A$の第1列と第2列と第3列の3つのベクトルを並べた行列を
    $A' \coloneqq \myparen{\begin{smallmatrix}
        \bm{a}_1 & \bm{a}_2 & \bm{a}_3
        \end{smallmatrix}}$、
    これらのベクトルが生成する$\R^4$の部分空間を
    $V \coloneqq \R \bm{a}_1 + \R \bm{a}_2 + \R \bm{a}_3$
    とおく。

    \uline{(1)} \quad
    $A$から$B$への行基本変形を表す正則行列を$P$とおく。
    $V$を左から$P$を掛ける線型写像$P \times$で写した像は
    \begin{equation}
        PV
            = \R P \bm{a}_1 + \R P \bm{a}_2 + \R P \bm{a}_3
            = \R \begin{pmatrix}
                1 \\
                0 \\
                0 \\
                0 \\
            \end{pmatrix}
            + \R \begin{pmatrix}
                3 \\
                0 \\
                0 \\
                0 \\
            \end{pmatrix}
            + \R \begin{pmatrix}
                0 \\
                1 \\
                0 \\
                0 \\
            \end{pmatrix}
            = \R \begin{pmatrix}
                1 \\
                0 \\
                0 \\
                0 \\
            \end{pmatrix}
            \oplus \R \begin{pmatrix}
                0 \\
                1 \\
                0 \\
                0 \\
            \end{pmatrix}
    \end{equation}
    であり、この次元は2である。
    このことと線型写像$P \times$が単射であることを合わせて、
    $V$の次元は2である。

    \uline{(2)} \quad
    まず$A'$の行列のサイズより$\rank A' \le 3$である。
    一方、$A$の列ベクトルは
    $A'$の列ベクトルらに2個の列ベクトルを付け加えたものだから
    $\rank A' \le \rank A \le \rank A' + 2$であり、
    また基本変形で行列の階数は不変だから
    $\rank A = \rank C = 4$である。
    以上より$\rank A' = 2, 3$が候補である。
    逆に、
    $A = C$の場合$\rank A' = 3$であり、
    また
    \begin{equation}
        A = \begin{pmatrix}
            1 & 0 & 0 & 0 & 0 \\
            0 & 0 & 0 & 0 & 1 \\
            0 & 0 & 1 & 0 & 0 \\
            0 & 0 & 0 & 1 & 0
        \end{pmatrix}
    \end{equation}
    の場合
    (これは$A$から第3列と第5列の交換で$C$が得られる)
    $\rank A' = 2$である。
    したがって$V$の次元の可能性は
    $\dim V = \rank A' = 2, 3$で尽くされる。
\end{answer}

\begin{problem}[物理工学科]
    次の3次正方行列$A$に関する以下の問いに答えよ。
    \begin{equation}
        A = \begin{pmatrix}
            3 & 0 & 1 \\
            -1 & 2 & -1 \\
            -2 & -2 & 1
        \end{pmatrix}
    \end{equation}
    \begin{enumerate}
        \item 行列$A$の固有値をすべて求めよ。
        \item 行列$A^n$を求めよ。ただし、$n$は自然数とする。
        \item 3次正方行列$B$は対角化可能で、$AB = BA$の関係をみたすとする。
            行列$A$の任意の固有ベクトル$p$は、行列$B$の固有ベクトルであることを示せ。
        \item $B^2 = A$の関係をみたす3次正方行列$B$を求めよ。
            ただし$B$はその固有値がすべて正となる対角化可能な行列とする。
        \item 3次正方行列$X$は対角化可能で、
            $AX = XA$の関係をみたすとする。
            $\tr(AX) = d$のとき、$\det(AX)$の最大値を$d$の関数として求めよ。
            ただし、$d$は正の実数とし、$X$の固有値はすべて正とする。
    \end{enumerate}
\end{problem}

\begin{proof}
    以下ではすべて$\C$上のベクトル空間を考える。

    \uline{(1)} \quad
    $A$の固有多項式を基本変形すると
    \begin{alignat}{1}
        \det(XI - A)
            &= \det \begin{pmatrix}
                X - 3 & 0 & -1 \\
                1 & X - 2 & 1 \\
                2 & 2 & X - 1
            \end{pmatrix} \\
            &= \det \begin{pmatrix}
                X - 3 & 0 & -1 \\
                -X + 3 & X - 2 & 1 \\
                0 & 2 & X - 1
            \end{pmatrix} \\
            &= \begin{pmatrix}
                X - 3 & 0 & -1 \\
                0 & X - 2 & 0 \\
                0 & 2 & X - 1
            \end{pmatrix} \\
            &= (X - 3)(X - 2)(X - 1)
    \end{alignat}
    だから、固有値は$1, 2, 3$である。

    \uline{(2)} \quad
    $A$の固有多項式の形から$A$の最小多項式は重根を持たないから、
    $A$は対角化可能である。
    そこで$A$の対角化を考える。
    \begin{itemize}
        \item 固有値$1$について、$A - 1I$を行基本変形して
            \begin{alignat}{1}
                A - 1I
                    \rightarrow
                        \begin{pmatrix}
                            2 & 0 & 1 \\
                            -1 & 1 & -1 \\
                            -2 & -2 & 0
                        \end{pmatrix}
                    \rightarrow
                        \begin{pmatrix}
                            2 & 0 & 1 \\
                            0 & 2 & -1 \\
                            0 & 0 & 0
                        \end{pmatrix}
            \end{alignat}
            となる。
            よって固有値$1$に属する固有空間は$\C \begin{pmatrix}
                -1 \\ 1 \\ 2
            \end{pmatrix}$である。
        \item 固有値$2$について、$A - 2I$を行基本変形して
            \begin{alignat}{1}
                A - 2I
                    \rightarrow
                        \begin{pmatrix}
                            1 & 0 & 1 \\
                            -1 & 0 & -1 \\
                            -2 & -2 & -1
                        \end{pmatrix}
                    \rightarrow
                        \begin{pmatrix}
                            1 & 0 & 1 \\
                            0 & 0 & 0 \\
                            0 & -2 & -1
                        \end{pmatrix}
            \end{alignat}
            となる。
            よって固有値$2$に属する固有空間は$\C \begin{pmatrix}
                -2 \\ 1 \\ 2
            \end{pmatrix}$である。
        \item 固有値$3$について、$A - 3I$を行基本変形して
            \begin{alignat}{1}
                A - 3I
                    \rightarrow
                        \begin{pmatrix}
                            0 & 0 & 1 \\
                            -1 & -1 & -1 \\
                            -2 & -2 & -2
                        \end{pmatrix}
                    \rightarrow
                        \begin{pmatrix}
                            0 & 0 & 1 \\
                            -1 & -1 & 0 \\
                            0 & 0 & 0
                        \end{pmatrix}
            \end{alignat}
            となる。
            よって固有値$3$に属する固有空間は$\C \begin{pmatrix}
                1 \\ -1 \\ 0
            \end{pmatrix}$である。
    \end{itemize}
    そこで各固有ベクトルを並べた行列を
    $P \coloneqq \begin{pmatrix}
        -1 & -2 & 1 \\
        1 & 1 & -1 \\
        2 & 2 & 0
    \end{pmatrix}$
    とおくと
    \begin{equation}
        A = PDP^{-1},
            \quad
            D \coloneqq \begin{pmatrix}
                1 & 0 & 0 \\
                0 & 2 & 0 \\
                0 & 0 & 3
            \end{pmatrix}
    \end{equation}
    が成り立つ。
    ここで掃き出し法により
    $P^{-1} = \begin{pmatrix}
        1 & 1 & 1/2 \\
        -1 & -1 & 0 \\
        0 & -1 & 1/2 \\
    \end{pmatrix}$
    を得る。
    よって
    \begin{alignat}{1}
        A^n
            &= (PDP^{-1})^{n} \\
            &= P D^n P^{-1} \\
            &= \begin{pmatrix}
                1 + 2^{n + 1} & -1 + 2^{n + 1} - 3^n & \frac{1}{2} (-1 + 3^n) \\
                1 - 2^n & 1 - 2^n + 3^n & \frac{1}{2} (1 - 3^n) \\
                2 - 2^{n + 1} & 2 - 2^{n + 1} & 1
            \end{pmatrix}
    \end{alignat}
    である。

    \uline{(3)} \quad
    $\lambda$を$A$の固有値とし、
    $A$の固有値$\lambda$に属する固有空間を$V_{\lambda}$とおく。
    $p \in V_{\lambda}$とすると、
    問題の仮定$AB = BA$より
    $ABp = BAp = B\lambda p = \lambda Bp$
    が成り立つ。
    したがって$Bp \in V_{\lambda}$である。
    いま (2)での議論より$\dim V_{\lambda} = 1$だから、
    $p \in V_{\lambda}$であることとあわせると、
    ある$\alpha \in \C$が存在して
    $Bp = \alpha p$が成り立つ。
    したがって$p$は$B$の固有ベクトルである。
    \TODO{$B$の対角化可能性をどこで用いた?}

    \uline{(4)} \quad
    $\sqrt{D} \coloneqq \begin{pmatrix}
        1 & 0 & 0 \\
        0 & \sqrt{2} & 0 \\
        0 & 0 & \sqrt{3}
    \end{pmatrix}$とおき、
    $B \coloneqq P \sqrt{D} P^{-1}$とおくと、
    $B^2 = (P \sqrt{D} P^{-1})^2 = P \sqrt{D}^2 P^{-1} = P D P^{-1} = A$である。
    $B$の定め方から$B$は対角化可能であり、
    その固有値$1, \sqrt{2}, \sqrt{3}$はすべて正だから、
    この$B$が求めるものである。
    具体的に成分を計算すれば
    \begin{alignat}{1}
        B
            &= P \sqrt{D} P^{-1} \\
            &= \begin{pmatrix}
                1 + 2\sqrt{2}
                    & -1 + 2\sqrt{2} - \sqrt{3}
                    & \frac{1}{2} (-1 + \sqrt{3}) \\
                1 - \sqrt{2}
                    & 1 - \sqrt{2} + \sqrt{3}
                    & \frac{1}{2} (1 - \sqrt{3}) \\
                2 - 2\sqrt{2}
                    & 2 - 2\sqrt{2}
                    & 1
            \end{pmatrix}
    \end{alignat}
    となる。

    \uline{(5)} \quad
    関数$f, g$を
    \begin{alignat}{1}
        f &\colon \R_{> 0}^3 \to \R,
            \quad (x, y, z) \mapsto 6xyz, \\
        g &\colon \R_{> 0}^3 \to \R,
            \quad (x, y, z) \mapsto x + 2y + 3z
    \end{alignat}
    と定める。

    \noindent
    Claim: 求める最大値は条件$g(x, y, z) = 0$の下での
    $f(x, y, z)$の最大値である。
    \begin{innerproof}
        問題の条件をみたす行列$X$全体の集合$\calX$を
        \begin{equation}
            \calX \coloneqq \mybrace{
                X \in M_3(\C)
                \mid
                \tr AX = d, \;
                \text{$X$の固有値はすべて正の実数}, \;
                \text{$X$は対角化可能}, \;
                XA = AX
            }
        \end{equation}
        とおく。
        本問で求める値は
        関数$\Phi \colon \calX \to \R, \; X \mapsto \det AX = 6 \det X$の最大値である。
        ここで$\Phi$の値は$X$を相似変換しても不変であるから、
        行列の相似が定める同値関係を$\sim$とおき、
        図式
        \begin{equation}
            \begin{tikzcd}
                \calX
                    \ar{r}{\Phi}
                    \ar[twoheadrightarrow]{d}
                    & \R \\
                \calX / \sim
                    \ar[dashed]{ur}[swap]{\wb{\Phi}}
            \end{tikzcd}
        \end{equation}
        の破線部に誘導される関数
        $\wb{\Phi} \colon \calX / \sim \; \to \R, \; [X] \mapsto 6 \det X$
        の最大値をかわりに求めればよいことがわかる。
        ただし$[X]$は$X$の同値類を表す。
        一方、写像$\pi \colon g^{-1}(0) \to \calX / \sim$を
        \begin{equation}
            \pi(x, y, z) \coloneqq \mybracket{
                \begin{pmatrix}
                    x & 0 & 0 \\
                    0 & y & 0 \\
                    0 & 0 & z
                \end{pmatrix}
            }
        \end{equation}
        で定めることができる。
        実際、$(x, y, z) \in g^{-1}(0)$に対し、
        $D' \coloneqq \begin{pmatrix}
            x & 0 & 0 \\
            0 & y & 0 \\
            0 & 0 & z
        \end{pmatrix}, \; X \coloneqq PD'P^{-1}$
        とおけば
        \begin{itemize}
            \item $\tr AX = \tr PDD'P^{-1} = \tr DD' = x + 2y + 3z = d$
            \item $X$の固有値はすべて正の実数
            \item $X$は対角化可能
            \item $XA = PD'DP^{-1} = PDD'P^{-1} = AX$
        \end{itemize}
        が成り立つから
        $\pi(x, y, z) = [D'] = [X]$は確かに$\calX / \sim$に属する。
        また明らかに$\pi(g^{-1}(0)) = \calX / \sim$が成り立つ。
        さらに$f$の定義より、$\pi$は図式
        \begin{equation}
            \begin{tikzcd}
                \calX / \sim
                    \ar{r}{\wb{\Phi}}
                    & \R \\
                g^{-1}(0)
                    \ar{u}{\pi}
                    \ar{ur}[swap]{f}
            \end{tikzcd}
        \end{equation}
        を可換にする。
        よって
        $f(g^{-1}(0))
            = \wb{\Phi}(\pi(g^{-1}(0)))
            = \wb{\Phi}(\calX / \sim)$
        が成り立つ。
        よって$\wb{\Phi}$の最大値は
        $f$の$g^{-1}(0)$上の最大値に他ならない。
        これで claim がいえた。
    \end{innerproof}
    Lanrange の未定乗数法を用いて解を求める。
    \TODO{}
\end{proof}



% ============================================================
%
% ============================================================
\chapter{双対空間}

% ------------------------------------------------------------
%
% ------------------------------------------------------------
\section{双対空間}

\TODO{}

% ------------------------------------------------------------
%
% ------------------------------------------------------------
\newpage
\section{演習問題}

\begin{problem}[東大数理 2006A]
    $V$を3次元複素ベクトル空間とする。
    $f_{1},f_{2},f_{3}\in V^{*}$
    について,
    \begin{alignat}{1}
        V_{1} &= \mathrm{Ker}(f_{1})\cap\mathrm{Ker}(f_{2}), \\
        V_{2} &= \mathrm{Ker}(f_{2})\cap\mathrm{Ker}(f_{3}), \\
        V_{3} &= \mathrm{Ker}(f_{3})\cap\mathrm{Ker}(f_{1})
    \end{alignat}
    と定める。
    このとき、以下の条件 (a) と (b) が同値であることを示せ。
    \begin{enumerate}[label=(\alph*)]
        \item $V$は部分空間$V_1, V_2, V_3$の直和である.
        \item $f_1, f_2, f_3$は一次独立である.
    \end{enumerate}
\end{problem}

\begin{proof}
    (b)から(a)の向きを示す。
    (b)の仮定より$f_i (i = 1, 2, 3)$は$V^*$の基底である。
    いま$V$は有限次元ゆえに$V^*$も有限次元なので、$f_i (i = 1, 2, 3)$の双対基底として$V^{**}$の基底$E_i$がただひとつ存在する。
    $E_i$の自然な同型$V^{**} \cong V$による像を$e_i \in V$とおくと、$e_i$は$V$の基底である。
    いま$f_j(e_i) = E_i(f_j) = \delta_{ij} \; (i, j = 1, 2, 3)$であるから、$f_i$は$e_i$の双対基底である。
    ここで、$V_1 = \C e_3$を示すため、両方の向きの包含を示す。
    $V_1 \supset \C e_3$であることは、
    任意の$\alpha e_3 \in \C e_3$に対し
    双対基底の性質より$f_1(\alpha e_3) = \alpha f_1(e_3) = 0, f_2(\alpha e_3) = \alpha f_2(e_3) = 0$となることから従う。
    $V_1 \subset \C e_3$であることを示す。
    いま$e_i$は$V$の基底であるから、任意の$v \in V_1 \subset V$に対し、
    $v = v_1 e_1 + v_2 e_2 + v_3 e_3$なる$v_i \in \C \; (i = 1, 2, 3)$が一意に存在する。
    よって、$V_1$の定義より$v_1 = f_1(v) = 0, \; v_2 = f_2(v) = 0$となる。
    したがって$v = v_3 e_3 \in \C e_3$である。
    これで$V_1 \subset \C e_3$、ひいては$V_1 = \C e_3$がいえた。
    同様にして$V_2 = \C e_1, V_3 = \C e_2$も得られる。
    いま$e_i$が$V$の基底であることより$V = \C e_1 \oplus \C e_2 \oplus \C e_3$なので、
    $V = V_1 \oplus V_2 \oplus V_3$を得る。

    次に(a)から(b)の向きを示す。
    まず各$i = 1, 2, 3$に対し$\dim \Ker f_i = 2, 3$である。
    実際、Rank-Nullity 定理より
    $3 = \dim V = \dim \Ker f_i + \dim \Im f_i \le \dim \Ker f_i + 1$ゆえに
    $2 \le \dim \Ker f_i \le \dim V = 3$だからである。
    つぎに$f_i \neq 0$を示しておく。
    $i = 1$の場合を考える。
    $f_1 = 0$と仮定すると$\Ker f_1 = V$だから
    \begin{alignat}{1}
        V_1 &= \Ker f_2 \\
        V_2 &= \Ker f_2 \cap \Ker f_3 \\
        V_3 &= \Ker f_3
    \end{alignat}
    である。すると
    \begin{alignat}{1}
        3
            &= \dim V \\
            &= \dim V_1 \oplus V_2 \oplus V_3 \\
            &= \dim V_1 + \dim V_2 + \dim V_3 \\
            &\ge \dim V_1 + \dim V_3 \\
            &= \dim \Ker f_2 + \dim \Ker f_3 \\
            &\ge 2 + 2 \\
            &= 4
    \end{alignat}
    となり矛盾が従う。
    よって$f_1 \neq 0$である。
    同様にして$f_2 \neq 0, f_3 \neq 0$である。
    よって、すべての$i = 1, 2, 3$に対し$\dim \Ker f_i = 2$が成り立つ。
    ここで
    \begin{alignat}{1}
        \dim V_1 &= \dim \Ker f_1 \cap \Ker f_2 \le \dim \Ker f_1 = 2 \\
        \dim V_2 &= \dim \Ker f_2 \cap \Ker f_3 \le \dim \Ker f_2 = 2 \\
        \dim V_3 &= \dim \Ker f_3 \cap \Ker f_1 \le \dim \Ker f_3 = 2
    \end{alignat}
    より$\dim V_i = 0, 1, 2 \; (i = 1, 2, 3)$であるが、
    もし$\dim V_1 = 0$なら
    $\Ker f_1 \cap \Ker f_2 = 0$より
    $\dim V \ge \dim \Ker f_1 \oplus \Ker f_2 = 2 + 2 = 4$となり矛盾。
    したがって$\dim V_1 = 1, 2$である。
    同様にして$\dim V_i = 1, 2 \; (i = 1, 2, 3)$である。
    いま(a)の仮定$V = V_1 \oplus V_2 \oplus V_3$より
    $3 = \dim V = \dim V_1 + \dim V_2 + \dim V_3$だから、
    $\dim V_i = 1 \; (i = 1, 2, 3)$が従う。
    \TODO{}
\end{proof}


% ============================================================
%
% ============================================================
\chapter{双線型形式}

% ------------------------------------------------------------
%
% ------------------------------------------------------------
\section{双線型写像}

\TODO{修正する}

\begin{definition}[多重線型写像]
    $V_1, \dots, V_k$および$W$をベクトル空間とする。
    写像$f \colon V_1 \times \dots \times V_k \to W$が
    \term{$k$-重線型}[$k$-multilinear]{多重線型写像}[たじゅうせんけいしゃぞう]
    であるとは、
    各$i = 1, \dots, k$に対し
    第$i$引数に関する線型性が成り立つこと、すなわち
    \begin{equation}
        f(v_1, \dots, a v_i + a' v_i' , \dots, v_k)
            = a f(v_1, \dots, v_i, \dots, v_k) + a' f(v_1, \dots, v_i', \dots, v_k)
    \end{equation}
    が成り立つことをいう。
    $k = 2$のときとくに
    \term{双線型}[bilinear]{双線型写像}[そうせんけいしゃぞう]であるという。
    また、値域が$W = \R$の場合とくに
    $f$を\term{$k$-重線型形式}[$k$-multilinear form]{線型形式}[せんけいけいしき]
    や\term{双線型形式}[bilinear form]{双線型形式}[そうせんけいけいしき]という。
\end{definition}

\begin{example}[多重線型写像の例]
    \label[example]{ex:multilinear-maps}
    ~
    \begin{enumerate}[label=(\alph*)]
        \item 通常の内積$f \colon \R^2 \times \R^2 \to \R,$
            \begin{equation}
                f(x, y) \coloneqq \up{t} x y
            \end{equation}
            は双線型写像である。
            しかし線型写像ではない。
        \item 行列式$f \colon \R^n \times \dots \times \R^n \to \R,$
            \begin{equation}
                f(v_1, \dots, v_n) \coloneqq \det [v_1, \dots, v_n]
            \end{equation}
            は$n$-重線型写像である。
        \item 通常のベクトル積$f \colon \R^3 \times \R^3 \to \R^3,$
            \begin{equation}
                f(x, y) \coloneqq \begin{bmatrix}
                    x_2 y_3 - x_3 y_2 \\
                    x_3 y_1 - x_1 y_3 \\
                    x_1 y_2 - x_2 y_1
                \end{bmatrix}
            \end{equation}
            は双線型写像である。
    \end{enumerate}
\end{example}

% ------------------------------------------------------------
%
% ------------------------------------------------------------
\section{対称性と交代性}

多重線型写像のうち、次に定義する対称性あるいは交代性をもつものはとくに重要である。

\begin{definition}[対称性と交代性]
    $f \colon V_1 \times \dots \times V_k \to W$を$k$-重線型写像とする。
    \begin{itemize}
        \item $f$が
            \term{対称}[symmetric]{対称}[たいしょう]であるとは、
            2つの引数の入れ替えで値が不変であること、すなわち
            \begin{equation}
                f(v_1, \dots, v_i, \dots, v_j, \dots, v_k)
                    = f(v_1, \dots, v_j, \dots, v_i, \dots, v_k)
            \end{equation}
            がすべての$i \neq j$で成り立つことをいう。
        \item $f$が
            \term{交代的}[alternating]{交代的}[こうたいてき]であるとは、
            2つの引数の入れ替えで符号が反転すること、すなわち
            \begin{equation}
                f(v_1, \dots, v_i, \dots, v_j, \dots, v_k)
                    = -f(v_1, \dots, v_j, \dots, v_i, \dots, v_k)
            \end{equation}
            がすべての$i \neq j$で成り立つことをいう。
    \end{itemize}
\end{definition}

\begin{example}[対称性と交代性の例]
    \cref{ex:multilinear-maps} でみた多重線型写像のうち
    内積は対称であり、行列式とベクトル積は交代的である。
\end{example}

多重線型写像の全体もまたベクトル空間をなすが、
とくに$\R$値の場合が重要である。
次に定義する多重線型形式の空間はテンソル積と密接に関係する。
また、双対空間は微分形式の定義において重要な役割を果たす。

\begin{definition}[多重線型写像のなすベクトル空間]
    $V$をベクトル空間とする。
    \begin{itemize}
        \item 線型写像$V \to \R$全体の集合は、
            通常の和とスカラー倍のもとでベクトル空間となる。
            これを$\Hom(V, \R)$と書き、
            $\Hom(V, \R)$の元を
            \term{線型形式}[linear form]{線型形式}[せんけいけいしき]あるいは
            \term{コベクトル}[covector]{コベクトル}という。
        \item $\Hom(V, \R)$は$V^*$とも書き、
            これを$V$の\term{双対空間}[dual space]{双対空間}[そうついくうかん]
            と呼ぶ。
        \item $k$重線型写像$V_1 \times \dots \times V_k \to \R$全体の集合は、
            通常の和とスカラー倍のもとでベクトル空間となる。
            これを$Mhrm{L}(V_1, \dots, V_k; \R)$と書き、
            $Mhrm{L}(V_1, \dots, V_k; \R)$の元を
            \term{多重線型形式}[multilinear form]{多重線型形式}[たじゅうせんけいけいしき]
            という。
    \end{itemize}
\end{definition}

\begin{definition}[双対基底]
    $V$を有限次元ベクトル空間とし、$e_1, \dots, e_n$をその基底とする。
    $V^*$の基底$e^1, \dots, e^n$であって
    \begin{equation}
        e^i (e_j) = \delta_{ij}
    \end{equation}
    をみたすものがただひとつ存在する。
    これを$e_1, \dots, e_n$の
    \term{双対基底}[dual basis]{双対基底}[そうついきてい]という。
\end{definition}

\begin{problem}[{[斎藤] A4.1.1}]
    $\R^3$の部分空間を
    $W \coloneqq \langle e_1 - e_2, e_2 - e_3 \rangle$で定める。
    $e_1, e_2, e_3$の双対基底を$f_1, f_2, f_3$とおく。
    このとき、$f_1 + f_2 + f_3$の$W$への制限は$0$であることを示せ。
\end{problem}

\begin{answer}
    $W$の生成元が$0$に写ることをいえばよい。
    \begin{alignat}{1}
        (f_1 + f_2 + f_3)(e_1 - e_2)
            &= (f_1 + f_2 + f_3)(e_1)
                - (f_1 + f_2 + f_3)(e_2) \\
            &= f_1(e_1) + f_2(e_1) + f_3(e_1)
                - f_1(e_2) - f_2(e_2) - f_3(e_2) \\
            &= 1 + 0 + 0 - 0 - 1 - 0 \\
            &= 0 \\
        (f_1 + f_2 + f_3)(e_2 - e_3)
            &= (f_1 + f_2 + f_3)(e_2)
                - (f_1 + f_2 + f_3)(e_3) \\
            &= f_1(e_2) + f_2(e_2) + f_3(e_2)
                - f_1(e_3) - f_2(e_3) - f_3(e_3) \\
            &= 0 + 1 + 0 - 0 - 0 - 1 \\
            &= 0
    \end{alignat}
\end{answer}


% ============================================================
%
% ============================================================
\chapter{擬ユークリッド空間と内積空間}

$K$-ベクトル空間$V$上の双線型写像で$K$に値をもつものを双線型形式という。
双線型形式のうち、対称性を備えたもの、すなわち対称双線型形式はとくに重要である。
対称双線型形式の重要な性質として、後で説明する非退化性と、
それを強めた正定値性が挙げられる。
非退化対称双線型形式を備えたベクトル空間は擬ユークリッド空間と呼ばれ、
その特別な場合として、
正定値対称双線型形式を備えたベクトル空間は内積空間と呼ばれる。

\TODO{ノルム空間は?}

% ------------------------------------------------------------
%
% ------------------------------------------------------------
\section{非退化性と正定値性}

双線型形式の非退化性を定義する。

\begin{definition}[非退化形式]
    $V, W$をベクトル空間、
    $b \colon V \times W \to \R$を双線型形式とする。
    \begin{itemize}
        \item $b$が$V$上\term{非退化}[nondegenerate]{非退化}[ひたいか]であるとは、
            \begin{equation}
                \forall x \in V \setminus \{ 0 \}
                \quad \text{に対し} \quad
                \exists y \in W
                \quad \text{s.t.} \quad
                b(x, y) \neq 0
            \end{equation}
            が成り立つことをいう。
        \item $b$が$W$上\term{非退化}{非退化}[ひたいか]であるとは、
            \begin{equation}
                \forall y \in W \setminus \{ 0 \}
                \quad \text{に対し} \quad
                \exists x \in V
                \quad \text{s.t.} \quad
                b(x, y) \neq 0
            \end{equation}
            が成り立つことをいう。
        \item $b$が\term{非退化}{非退化}[ひたいか]であるとは、
            $b$が$V$上非退化かつ$W$上非退化であることをいう。
    \end{itemize}
\end{definition}

\begin{proposition}[非退化性の特徴付け]
    $V, W$をベクトル空間、
    $b \colon V \times W \to \R$を双線型形式とする。
    \begin{enumerate}
        \item $b$が$V$上非退化であることと、
            $l_b \colon V \to W^*$が単射であることとは同値である。
        \item $b$が$W$上非退化であることと、
            $r_b \colon W \to V^*$が単射であることとは同値である。
    \end{enumerate}

    \TODO{行列表示による特徴付け?}
\end{proposition}

\begin{proof}
    \TODO{}
\end{proof}

\begin{definition}[双対対]
    $V, W$をベクトル空間とし、
    $b \colon V \times W \to \R$を双線型形式とする。
    組$(V, W, b)$が\term{双対対}[dual pair]{双対対}[そうついつい]であるとは、
    $b$が非退化であることをいう。
    このとき、$b$は
    \term{$X$と$Y$との間に双対性を定める}[places $X$ and $Y$ in duality]
    {双対性を定める}[そうついせいをさだめる]
    という。
\end{definition}

\begin{definition}[正定値]
    \begin{equation}
        \begin{cases}
            b(x, x) \ge 0 \\
            b(x, x) = 0 \quad \iff \quad x = 0
        \end{cases}
    \end{equation}
    \TODO{}
\end{definition}



% ------------------------------------------------------------
%
% ------------------------------------------------------------
\section{擬ユークリッド空間と内積空間}

\begin{definition}[擬ユークリッド空間]
    $K$を順序体とする。
    $K$-ベクトル空間$V$と
    関数$\langle \, , \, \rangle \colon V \times V \to K$
    であって次をみたすものの組$(V, \langle \, , \, \rangle)$を
    \TODO{}
\end{definition}

\begin{theorem}[Cauchy-Schwarz の不等式]
    \begin{equation}
        |\langle x, y \rangle|^2 \le \langle x, x \rangle \langle y, y \rangle
    \end{equation}
    \TODO{}
\end{theorem}

\begin{proof}
    \TODO{}
\end{proof}

\begin{definition}[直交系と正規直交系]
    \TODO{}
\end{definition}

\TODO{直交化すれば逆行列を転置として求めることができる}

\begin{theorem}[Gram-Schmidt の直交化法]
    $A = [a_1 \dots a_n]$とする。
    \begin{alignat}{1}
        q'_1 &= a_1 \\
        q_1 &= q'_1 / \| q'_1 \| \\
        q'_2 &= a_2 - (a_2, q_1) q_1
            = a_2 - q_1 \up{t}q_1 a_2 \\
        q_2 &= q'_2 / \| q'_2 \| \\
        q'_3 &= a_3 - (a_3, q_1) q_1 - (a_3, q_2) q_2
            = a_3 - q_1 \up{t}q_1 a_3 - q_2 \up{t}q_2 a_3 \\
        q_3 &= q'_3 / \| q'_3 \| \\
        &\vdots
    \end{alignat}
    より$A = QR$
    \begin{equation}
        Q = [q_1 \dots q_n],
        \quad
        R = \begin{bmatrix}
            \| q'_1 \| & \up{t}q_1 a_2 & \cdots & \up{t}q_1 a_n \\
            & \| q'_2 \| & & \up{t}q_2 a_n \\
            & & \ddots & \vdots \\
            & & & \| q'_n \|
        \end{bmatrix}
    \end{equation}
    と分解される。

    \TODO{ただの直交化は三角行列を与え、正規直交化は直交行列を与える?}
\end{theorem}

\begin{proof}
    \TODO{}
\end{proof}

% ------------------------------------------------------------
%
% ------------------------------------------------------------
\section{直交変換とユニタリ変換}

\begin{definition}[Gram 行列]
    $v_1, \dots, v_n$を$V$の基底とする。
    \begin{equation}
        \begin{pmatrix}
            \myangle{v_1}{v_1} & \cdots & \myangle{v_1}{v_n} \\
            \vdots & \ddots & \vdots \\
            \myangle{v_n}{v_1} & \cdots & \myangle{v_n}{v_n}
        \end{pmatrix}
    \end{equation}
    を$v_1, \dots, v_n$の
    \term{Gram 行列}[Gram matrix]{Gram 行列}[Gram ぎょうれつ]という。
\end{definition}

\begin{definition}[直交変換とユニタリ変換]
    $\myangle{\varphi(v)}{\varphi(w)} = \myangle{v}{w}$をみたす
    線型写像$\varphi \colon V \to V$を
    \term{ユニタリ変換}[unitary transformation]{ユニタリ変換}[ユニタリへんかん]という。
    とくに$K = \R$のとき
    \term{直交変換}[orthogonal transformation]{直交変換}[ちょっこうへんかん]という。
\end{definition}

\begin{definition}[対称行列と随伴行列]
    $\up{t}A$を
    $A$の\term{転置行列}[transpose matrix]{転置行列}[てんちぎょうれつ]といい、
    $A^*$を
    $A$の\term{随伴行列}[adjoint matrix]{随伴行列}[ずいはんぎょうれつ]という。
\end{definition}

% ------------------------------------------------------------
%
% ------------------------------------------------------------
\section{2次形式}

\begin{definition}[符号]
    \TODO{}
\end{definition}

\begin{theorem}[Sylvester の慣性法則]
    \TODO{}
    実対称行列の符号は
    任意の正則行列$S$に関する
    合同変換$A \mapsto \up{t}SAS$
    によって不変である。
\end{theorem}

\TODO{符号と非退化性との関係?}

\begin{proposition}[正定値性の判定]
    $A$を$n$次実対称行列とする。
    このとき、2次形式$A[X]$が正定値であるためには、
    $n$個の首座小行列式$\det A_k \; (k = 1, \dots, n)$が
    すべて正であることが必要十分である。
\end{proposition}

\begin{proof}
    \TODO{}
\end{proof}

% ------------------------------------------------------------
%
% ------------------------------------------------------------
\section{Banach 空間と Hilbert 空間}

\begin{definition}[有界線型作用素]
    $X, Y$をノルム空間、
    $f \colon X \to Y$を線型作用素とする。
    \begin{equation}
        \sup_{\| x \| \le 1} \| f(x) \| < \infty
    \end{equation}
    のとき、
    $f$は
    \term{有界}[bounded]{有界}[ゆうかい]
    であるという。
\end{definition}

\begin{theorem}[有界性の特徴づけ]
    $X, Y$をノルム空間、
    $f \colon X \to Y$を線型作用素とする。
    このとき、次は同値である:
    \begin{enumerate}
        \item $f$は有界である。
        \item $f$は連続である。
        \item $f$はある1点で連続である。
    \end{enumerate}
\end{theorem}

連続性は点列連続性で置き換えられることは重要である。
なぜならば、ノルム空間は距離化可能ゆえに第1可算だからである。

\begin{proof}
    \TODO{}
\end{proof}

\begin{theorem}[開写像定理]
    $X, Y$を Banach 空間とする。
    有界線型作用素$f \colon X \to Y$が全射ならば、
    $f$は開写像である。
\end{theorem}

\begin{proof}
    \TODO{}
\end{proof}








% ============================================================
%
% ============================================================
\chapter{テンソル積}

以下、係数体は$\R$とする。

% ------------------------------------------------------------
%
% ------------------------------------------------------------
\section{ベクトル空間のテンソル積}

テンソル積は多重線型写像と密接な関係をもつ。

\begin{definition}[テンソル積]
    $V_1, \dots, V_k$をベクトル空間とする。
    ベクトル空間$T$が
    $V_1, \dots, V_k$の
    \term{テンソル積}[tensor product]{テンソル積}[てんそるせき]
    であるとは、
    $T$が次をみたすことをいう:
    \begin{description}
        \item[(T1)]
            $k$重線型写像$\otimes \colon V_1, \dots, V_k \to T$が存在する。
        \item[(T2)]
            任意のベクトル空間$X$と$k$重線型写像$f \colon V_1 \times \dots \times V_k \to X$に対し、
            図式
            \begin{equation}
                \begin{tikzcd}[row sep=large]
                    V_1 \times \dots \times V_k \ar{r}{\otimes}
                        \ar{dr}[swap]{f}
                        & T \ar[dashed]{d}{\tilde{f}} \\
                    & X
                \end{tikzcd}
            \end{equation}
            を可換にする線型写像$\tilde{f}$がただひとつ存在する。
            この性質を\term{テンソル積の普遍性}[universal property of the tensor product]
            {テンソル積の普遍性}[てんそるせきのふへんせい]という。
    \end{description}
    このような$T$は存在する(証明略)。
    \begin{itemize}
        \item $T$を$V_1 \otimes \dots \otimes V_k$と書く。
        \item 各$v_j \in V_j\; (j = 1, \dots, k)$に対し、
            $\otimes(v_1, \dots, v_k)$を$v_1 \otimes \dots \otimes v_k$と書き、
            $v_1, \dots, v_k$の\term{テンソル積}[tensor product]{テンソル積}[てんそるせき]と呼ぶ。
        \item $V_1 \otimes \dots \otimes V_k$の元を\term{テンソル}[tensor]{テンソル}と呼ぶ。
    \end{itemize}
\end{definition}

\begin{example}[テンソル積の普遍性の例]
    \label[example]{ex:tensor-product-universal-property}
    $f \colon \R^3 \times \R^3 \to \R^3$を通常のベクトル積とする。
    \cref{ex:multilinear-maps} でみたように
    $f$は双線型写像だから、テンソル積の普遍性により、図式
    \begin{equation}
        \begin{tikzcd}
            \R^3 \times \R^3 \ar{r}{\otimes}
                \ar{dr}[swap]{f}
                & \R^3 \otimes \R^3 \ar[dashed]{d}{\tilde{f}} \\
            & \R^3
        \end{tikzcd}
    \end{equation}
    を可換にする線型写像$\tilde{f}$が誘導される。
    $\tilde{f}$は
    \begin{equation}
        \tilde{f}(x \otimes y) = \begin{bmatrix}
            x_2 y_3 - x_3 y_2 \\
            x_3 y_1 - x_1 y_3 \\
            x_1 y_2 - x_2 y_1
        \end{bmatrix}
    \end{equation}
    などをみたす。
\end{example}

テンソル積の基底は非常に簡単な形をしている。

\begin{theorem}[テンソル積の基底]
    $V_1, \dots, V_k$をそれぞれ$n_1, \dots, n_k$次元ベクトル空間とする。
    各$i = 1, \dots, k$に対し、
    $e_1^{i}, \dots, e_{n_i}^{i}$を$V_i$の基底とする。
    このとき、$V_1 \otimes \dots \otimes V_k$の部分集合
    \begin{equation}
        \{
            e_{i_1}^{1} \otimes \dots \otimes e_{i_k}^{k} \colon
            1 \le i_1 \le n_1, \dots, 1 \le i_k \le n_k
        \}
    \end{equation}
    は$V_1 \otimes \dots \otimes V_k$の基底となる。
\end{theorem}

\begin{proof}
    省略
\end{proof}

\begin{example}[基底の例]
    上の定理より、テンソル積$\R^2 \otimes \R^2$の基底として
    \begin{equation}
        e_1 \otimes e_1, \; e_1 \otimes e_2, \;
        e_2 \otimes e_1, \; e_2 \otimes e_2
    \end{equation}
    がとれる。
\end{example}

\begin{remark}[テンソル積の諸性質]
    ~
    \begin{itemize}
        \item $u \otimes v = v \otimes u$とは限らない。
            \begin{innerproof}
                \cref{ex:tensor-product-universal-property} の
                線型写像$\tilde{f}$を考えると、
                \begin{alignat}{1}
                    \tilde{f}(e_1 \otimes e_2) &= e_3 \\
                    \tilde{f}(e_2 \otimes e_1) &= - e_3
                \end{alignat}
                だから$e_1 \otimes e_2 = e_2 \otimes e_1$ではありえない。
            \end{innerproof}
        \item $V_1 \otimes \dots \otimes V_k$の元はすべてが
            $v_1 \otimes \dots \otimes v_k$の形に書けるとは限らない。
            \begin{innerproof}
                $e_1 \otimes e_2 + e_2 \otimes e_1 \in \R^2 \otimes \R^2$が
                \begin{equation}
                    \begin{bmatrix}
                        a \\ b
                    \end{bmatrix} \otimes \begin{bmatrix}
                        c \\ d
                    \end{bmatrix}
                \end{equation}
                の形に書けたとする。
                \begin{alignat}{1}
                    \begin{bmatrix}
                        a \\ b
                    \end{bmatrix} \otimes \begin{bmatrix}
                        c \\ d
                    \end{bmatrix}
                        &= (ae_1 + be_2) \otimes (ce_1 + de_2) \\
                        &= ae_1 \otimes ce_1 + ae_1 \otimes de_2 + be_2 \otimes ce_1 + be_2 \otimes de_2 \\
                        &= ac (e_1 \otimes e_1) + ad (e_1 \otimes e_2) + bc (e_2 \otimes e_1) + bd (e_2 \otimes e_2)
                \end{alignat}
                だから
                \begin{equation}
                    ac = 0, \quad ad = 1, \quad bc = 1, \quad bd = 0
                \end{equation}
                であるが、これをみたす$a, b, c, d \in \R$は存在しない。
            \end{innerproof}
    \end{itemize}
\end{remark}

行列とテンソルは異なる概念だが、
次の例でみるように同一視が可能である。

\begin{example}[行列とテンソル]
    \label[example]{ex:matrix-tensor}
    $V \otimes U^*$は、次の同型により$\Hom(U, V)$と同一視できる(証明略)。
    \begin{equation}
        V \otimes U^* \to \Hom(U, V),
        \quad
        v \otimes f \mapsto f(u)v
    \end{equation}
    とくに$V = \R^m, U = \R^n$とすれば
    $\R^m \otimes (\R^n)^* \cong \Hom(\R^n, \R^m) \cong M_{m \times n}(\R)$
    が成り立つ。
    具体的には
    \begin{equation}
        e_j \otimes f^i
            \mapsto f^i (\square) e_j
            \mapsto ((j, i) \text{ 成分だけが$1$でそれ以外$0$の行列})
    \end{equation}
    と対応する。ただし$e_j$は$\R^m$の標準基底、
    $f^i$は$\R^n$の標準基底の双対基底である。
\end{example}

\begin{definition}[行列の Kronecker 積]
    $A, B \in M_{n \times n}(\R)$とし、
    $A = (a^i_j)_{i,j}, B = (b^i_j)_{i,j}$とおく。
    線型写像$\varphi_A \otimes \varphi_B \colon \R^n \otimes \R^n \to \R^n \otimes \R^n$を
    \begin{equation}
        (\varphi_A \otimes \varphi_B)(e_i \otimes e_j)
            \coloneqq A e_i \otimes B e_j
    \end{equation}
    で定義する。このとき、$\R^n \otimes \R^n$の基底
    \begin{equation}
        e_1 \otimes e_1, \; e_1 \otimes e_2, \dots, e_n \otimes e_n
        \quad (\text{添字に関し辞書順})
    \end{equation}
    に関する線型写像$\varphi_A \otimes \varphi_B$の行列表示は
    \begin{equation}
        \begin{bmatrix}
            a^1_1 b^1_1 & a^1_1 b^1_2 & \dots & a^1_n b^1_n \\
            a^1_1 b^2_1 & a^1_1 b^2_2 & \dots & a^1_n b^2_n \\
            \vdots &&& \vdots \\
            a^n_1 b^n_1 & a^n_1 b^n_2 & \dots & a^n_n b^n_n
        \end{bmatrix}
    \end{equation}
    となる。
    この行列表示を$A, B$の
    \term{Kronecker 積}[Kronecker product]{Kronecker 積}[Kroneckerせき]といい、
    $A \otimes B$と書く。
\end{definition}

\begin{example}[$2 \times 2$行列の Kronecker 積]
    \begin{equation}
        A = \begin{bmatrix}
            a & b \\
            c & d
        \end{bmatrix},
        \quad
        X = \begin{bmatrix}
            x & y \\
            z & w
        \end{bmatrix}
    \end{equation}
    のとき
    \begin{equation}
        A \otimes X
            = \begin{bmatrix}
                aX & bX \\
                cX & dX
            \end{bmatrix}
            = \begin{bmatrix}
                ax & ay & bx & by \\
                az & aw & bz & bw \\
                cx & cy & dx & dy \\
                cz & cw & dz & dw
            \end{bmatrix}
    \end{equation}
    となる。
\end{example}


% ------------------------------------------------------------
%
% ------------------------------------------------------------
\section{1種類のベクトル空間からなるテンソル積}

1種類のベクトル空間(とその双対空間)から構成されるテンソル積は
応用上とくに重要である。

\begin{definition}[共変テンソル]
    $V$を有限次元ベクトル空間とする。
    \begin{equation}
        T^k(V^*) \coloneqq \underbrace{V^* \otimes \dots \otimes V^*}_{k \text{ times}}
    \end{equation}
    と書き、$T^k(V^*)$の元を
    $V$上の\term{共変$k$-テンソル}[covariant $k$-tensor]{共変テンソル}[きょうへんてんそる]と呼ぶ。
    $k = 0$の場合、共変$0$-テンソルは$\R$の元を表すものと約束する。
\end{definition}

共変テンソルは、次の定理により多重線型形式と同一視できる。

\begin{theorem}[テンソルと多重線型形式の対応]
    \label[theorem]{thm:tensor-product-and-multilinear-map}
    $V_1, \dots, V_k$をベクトル空間とする。
    このとき、写像$\Phi \colon V_1^* \times \dots \times V_k^* \to Mhrm{L}(V_1, \dots, V_k; \R),$
    \begin{equation}
        \Phi(\omega^1, \dots, \omega^k)(v_1, \dots, v_k)
            \coloneqq \omega^1(v_1) \dots \omega^k(v_k)
    \end{equation}
    は$k$重線型写像となり、テンソル積の普遍性により誘導される線型写像
    $\tilde{\Phi}$は同型である。
    \begin{equation}
        \begin{tikzcd}[row sep=large]
            V_1^* \times \dots \times V_k^* \ar{r}{\otimes}
                \ar{dr}[swap]{\Phi}
                & V_1^* \otimes \dots \otimes V_k^* \ar[dashed]{d}{\tilde{\Phi}}[swap]{\cong} \\
            & Mhrm{L}(V_1, \dots, V_k; \R)
        \end{tikzcd}
    \end{equation}
\end{theorem}

\begin{proof}
    省略
\end{proof}

\begin{remark}
    \cref{thm:tensor-product-and-multilinear-map}により、
    共変$k$-テンソルは$\underbrace{V \times \dots \times V}_{k \text{ times}}$上の
    $k$重線型形式とみなせる。
\end{remark}

\begin{definition}[反変テンソル]
    $V$を有限次元ベクトル空間とする。
    \begin{equation}
        T^k(V) \coloneqq \underbrace{V \otimes \dots \otimes V}_{k \text{ times}}
    \end{equation}
    と書き、$T^k(V)$の元を
    $V$上の\term{反変$k$-テンソル}[contravariant $k$-tensor]{反変テンソル}[はんぺんてんそる]と呼ぶ。
    $k = 0$の場合、反変$0$-テンソルは$\R$の元を表すものと約束する。
\end{definition}

\begin{definition}[混合テンソル]
    $V$を有限次元ベクトル空間とする。
    \begin{equation}
        T^{(k, l)}(V)
            \coloneqq \underbrace{V \otimes \dots \otimes V}_{k \text{ times}}
            \otimes \underbrace{V^* \otimes \dots \otimes V^*}_{l \text{ times}}
    \end{equation}
    と書き、$T^{(k, l)}(V)$の元を
    $V$上の\term{$(k, l)$型混合テンソル}[mixed tensor of type $(k, l)$]{混合テンソル}[こんごうてんそる]
    あるいは単に\term{$(k, l)$型テンソル}[tensor of type $(k, l)$]{混合テンソル}[こんごうてんそる]と呼ぶ。
    $(0, 0)$型テンソルは$\R$の元を表すものと約束する。
\end{definition}

\begin{remark}[テンソルの係数による表記]
    $V$を有限次元ベクトル空間、$e_1, \dots, e_n$を$V$の基底とし、
    その双対基底を$e^1, \dots, e^n$とする。
    このとき、$V$上の$(k, l)$型テンソル$\alpha$は
    \begin{equation}
        \alpha = \alpha^{\kappa_1 \dots \kappa_k}_{\lambda_1 \dots \lambda_l}
            e_{\kappa_1} \otimes \dots \otimes e_{\kappa_k}
            \otimes e^{\lambda_1} \otimes \dots \otimes e^{\lambda_l}
    \end{equation}
    と表せる。
    そこで、係数の族$\alpha^{\kappa_1 \dots \kappa_k}_{\lambda_1 \dots \lambda_l}$によって
    テンソル$\alpha$を表すことがある。
\end{remark}

% ------------------------------------------------------------
%
% ------------------------------------------------------------
\section{外巾}

\begin{definition}[外巾]
    \TODO{}
\end{definition}

\begin{proposition}
    $V$を$n$次元ベクトル空間、
    $v_1, \dots, v_n$を$V$の基底とする。
    このとき$v_1 \wedge \dots \wedge v_n$は$\bigwedge^n V$の基底である。
\end{proposition}

\begin{proof}
    \TODO{cf. [斎藤] p.227}
\end{proof}

% ------------------------------------------------------------
%
% ------------------------------------------------------------
\section{対称テンソルと交代テンソル}

共変テンソルは多重線型形式とみなせるから、
多重線型形式としての性質に基づいて
対称テンソル・交代テンソルという2つの特別な共変テンソルが定義される。
多様体との関連でいえば、
対称テンソルは Riemann 計量の定義に、交代テンソルは微分形式の定義において
それぞれ重要な役割を果たす。

\begin{definition}[対称テンソル]
    $V$を$n$次元ベクトル空間とする。
    共変$k$-テンソル$\alpha \in T^k(V^*)$が
    $k$-重線型形式として対称であるとき、
    $\alpha$を\term{対称テンソル}[symmetric tensor]{対称テンソル}[たいしょうてんそる]という。
    $V$上の対称$k$-テンソル全体の集合を$\Sigma^k(V^*)$と書く。
\end{definition}

\begin{definition}[交代テンソル]
    $V$を$n$次元ベクトル空間とする。
    共変$k$-テンソル$\alpha \in T^k(V^*)$が
    $k$-重線型形式として交代的であるとき、
    $\alpha$を\term{交代テンソル}[alternating tensor]{交代テンソル}[こうたいてんそる]という。
    $V$上の交代$k$-テンソル全体の集合を$\Lambda^k(V^*)$と書く。
\end{definition}

% ------------------------------------------------------------
%
% ------------------------------------------------------------
\section{テンソルの縮約}

テンソルの縮約とは混合テンソルの型を変える線型写像であり、
トレースの一般化である。

\begin{definition}[縮約]
    $V$を有限次元ベクトル空間とし、$m, n \ge 1$とする。
    線型写像
    \begin{equation}
        \underbrace{V \otimes \dots \otimes V}_{m \text{ times}}
        \otimes
        \underbrace{V^* \otimes \dots \otimes V^*}_{n \text{ times}}
        \to
        \underbrace{V \otimes \dots \otimes V}_{(m - 1) \text{ times}}
        \otimes
        \underbrace{V^* \otimes \dots \otimes V^*}_{(n - 1) \text{ times}}
    \end{equation}
    であって
    \begin{alignat}{1}
        &v_1 \otimes \dots \otimes v_k \otimes \dots \otimes v_m
            \otimes f_1 \otimes \dots \otimes f_l \otimes \dots \otimes f_m \\
        \mapsto f_l(v_k) \;
            &v_1 \otimes \dots \otimes \cancel{v_k} \otimes \dots \otimes v_m
            \otimes f_1 \otimes \dots \otimes \cancel{f_l} \otimes \dots \otimes f_m
    \end{alignat}
    により定まるものを\term{$(k, l)$-縮約}[$(k, l)$ contraction]{縮約}[しゅくやく]という。
\end{definition}

\begin{example}[行列のトレースと縮約]
    \cref{ex:matrix-tensor}でのテンソルと行列の対応を思い出すと、
    行列$(a_{ij})_{i,j} \in M_{n \times n}(\R)$の$(1, 1)$-縮約は
    \begin{alignat}{1}
        \sum_{i, j} a_{ij} e_j \otimes e^i
        \mapsto
        \sum_{i, j} a_{ij} e^i(e_j)
        = a_{11} + \dots + a_{nn}
    \end{alignat}
    となり、行列のトレースに一致する。
\end{example}




\end{document}