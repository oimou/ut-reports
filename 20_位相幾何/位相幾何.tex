\documentclass[report, notitlepage]{jlreq}
\usepackage{docmute}
\usepackage{global}
\usepackage{./sub/local}
\def\assetspath{./}
\makeindex
\makeglossaries

\title{代数的トポロジーの基礎}
\author{Yahata}
\date{}

\begin{document}

\maketitle
\begin{abstract}
    代数的トポロジーは、
    位相空間と連続写像の問題を
    群やベクトル空間などの代数系とその準同型の問題に帰着させて研究する
    数学の一分野である。
    代数的トポロジーでは
    ホモトピーやホモロジーといった手法によって空間の形を捉え、
    その背後にある秩序を代数系に投影する。

    本稿では基本群とホモロジー群について整理する。
    定義や命題にはできるだけその概要や導入の動機、応用先について一言程度添える。
    演習問題にはできるだけ詳細な解答をつける。

    内容は主に\cite{Rot98}や\cite{河澄22}、\cite{Lee10}を参考にしている。
\end{abstract}

\setcounter{tocdepth}{1}
\tableofcontents
\markboth{\contentsname}{}

% ============================================================
%
% ============================================================
\newpage
\documentclass[report]{jlreq}
\usepackage{global}
\usepackage{./local}
\subfiletrue
\def\assetspath{../}
\begin{document}

% ============================================================
%
% ============================================================
\chapter{位相空間}

代数的トポロジーでは位相空間内のパスを扱うから、
必然的に実数体$\R$の位相を扱うことになる。
そこで、代数的トポロジーでとくに重要な位相的性質について述べておくことにする。

% ------------------------------------------------------------
%
% ------------------------------------------------------------
\section{連結空間}

連結性について述べる。
次の事実は次節で導入する弧状連結性の議論に必須である。

\begin{proposition}[$\R$の区間は連結]
    $\R$の空でない任意の区間$J$は連結である。
\end{proposition}

\begin{proof}
    $J_1, J_2 \opensubset J, \;
        J = J_1 \cup J_2, \;
        J_1 \neq \emptyset, \;
        J_2 \neq \emptyset, \;
        J_1 \cap J_2 \neq \emptyset$
    と表せたとして矛盾を導く。
    $x_1 \in J_1, \; x_2 \in J_2$をひとつずつ選ぶ。
    一般性を失うことなく$x_1 < x_2$としてよい。
    \begin{equation}
        c \coloneqq \sup \{
            x \in \R
            \mid
            [x_1, x) \cap J \subset J_1
        \}
        \; (\ge x_1)
    \end{equation}
    とおく。
    すると$c \le x_2$である。
    実際、$c > x_2$とするといま$x_1 < x_2$であったから
    $x_1 < x_2 < c$したがって$x_2 \in [x_1, c) \cap J \subset J_1$となり
    $x_2 \notin J_1$に矛盾する。
    よって$c \in J$である。
    実際、$J$が区間であることより
    $c \in [x_1, x_2] \subset J$だからである。
    したがって上限の性質より$c \in \Cl_J J_1$である。
    このことと$J_1 \; (= J \setminus J_2)$が closed in $J$であることから
    $c \in J_1$である。
    いま$c \in J_1 \opensubset J$ゆえに
    ある$r > 0$が存在して
    $c \in (c - r, c + r) \cap J \subset J_1$が成り立つ。
    したがって$[x_1, c + r) \cap J \subset J_1$となり
    $c$の定義に矛盾する。
\end{proof}

% ------------------------------------------------------------
%
% ------------------------------------------------------------
\section{弧状連結空間}
\label[section]{section:path-connected-space}

弧状連結性を定義する。
弧状連結性は基本群の定義に不可欠な要素である。

\begin{definition}[パスとループ]
    \idxsym{unit interval}{$I$}{$\R$の閉区間$I = [0, 1]$}
    $X$を位相空間とする。
    連続写像$f \colon [0, 1] \to X$を
    $X$内で$f(0)$と$f(1)$をつなぐ\term{パス}[path]{パス}という。
    さらに$f(0) = f(1)$のとき、$f$を\term{ループ}[loop]{ループ}といい、
    点$f(0)$をループ$f$の\term{基点}[base point]{基点}[きてん]という。
    今後は閉区間$[0, 1]$をよく使うので、$I \coloneqq [0, 1]$と書くことがある。
\end{definition}

\begin{definition}[弧状連結]
    \TODO{}
\end{definition}

\begin{definition}[局所弧状連結]
    \TODO{}
\end{definition}

弧状連結空間は連結である。

\begin{proposition}[弧状連結ならば連結]
    \TODO{}
\end{proposition}

\begin{proof}
    \TODO{}
\end{proof}

逆に連結空間が弧状連結であるとは限らないが、
ひとつ条件を加えれば弧状連結性が成り立つ。
証明の流れは連結空間の特徴付けを用いる典型的なものである。

\begin{proposition}[連結かつ局所弧状連結ならば弧状連結]
    \label[proposition]{prop:cnd-and-loc-path-cnd-implies-path-cnd}
    位相空間$X$が連結かつ局所弧状連結ならば
    弧状連結である。
\end{proposition}

\begin{proof}
    $X$は連結だから
    $X \neq \emptyset$である。
    $x_0 \in X$をひとつえらび、
    $x_0$の属する弧状連結成分を$C$とおく。

    まず、$C$が open in $X$であることを示す。
    $X$は局所弧状連結だから、各$x \in C$に対し、
    $x$の$X$での開近傍$U$であって弧状連結であるようなものがとれる。
    このとき、$U$の点は$x$と$U$内のパスでつなぐことができ、
    $x$は$C$の任意の点と$C$内のパスでつなぐことができるから、
    $U$の点は$C$の任意の点とパスでつなぐことができる。
    よって、弧状連結成分の定義より$U \subset C$である。
    ゆえに$x$は$C$の$X$における内点である。
    したがって$C$は open in $X$である。

    次に、$C$が closed in $X$であることを示す。
    $x \in \del C$とすると、
    $x$の$X$での開近傍$U$であって弧状連結であるようなものがとれる。
    このとき、$x \in \del C$であることより$C \cap U \neq \emptyset$である。
    よって、或る$y \in C \cap U$がとれる。
    $x$は$y$と$U$内のパスでつなぐことができ、
    $y$は$C$の任意の点と$C$内のパスでつなぐことができるから、
    $x$は$C$の任意の点とパスでつなぐことができる。
    よって、弧状連結成分の定義より$x \in C$である。
    ゆえに$\del C \in C$、したがって$C$は closed in $X$である。

    以上より$C$は非空かつ clopen in $X$である。
    いま$X$は連結であったから、$C = X$である。
    したがって$X$は弧状連結である。
\end{proof}

\begin{definition}[弧状連結成分]
    \TODO{}
\end{definition}

\begin{proposition}[clopen かつ弧状連結ならば弧状連結成分]
    \label[proposition]{prop:clopen-path-cnd-implies-path-cnd-component}
    位相空間の部分集合は、clopen かつ弧状連結ならば
    弧状連結成分である。
\end{proposition}

\begin{proof}
    $X$を位相空間とし、
    $A \subset X$は clopen in $X$かつ弧状連結であるとする。
    $A$は弧状連結ゆえに$A \neq \emptyset$だから或る$a \in A$がとれる。
    $A$が$a$の属する$X$の弧状連結成分$P$に一致することを示す。
    そのためには、$A$は$a$とパスでつなぐことのできる$X$の点全体からなることを示せばよい。
    まず、$A$は弧状連結だから$A \subset P$である。
    つぎに、$b \in P$とする。
    $a, b$をつなぐパスを$\gamma \colon I \to X$とすると、
    $\gamma^{-1}(A)$は非空かつ clopen in $I$だから、
    $I$が連結であることとあわせて$I = \gamma^{-1}(A)$である。
    よって$\gamma(I) \subset A$、とくに$b \in A$である。
    したがって$P \subset A$である。
    以上より$A = P$がいえた。
\end{proof}



% ------------------------------------------------------------
%
% ------------------------------------------------------------
\newpage
\section{演習問題}

\subsection{問題セット 1}

\begin{problem}[幾何学II 1.1]
    半開区間$(0, 1]$から開区間$(0, 1)$への
    連続な全単射が存在するかどうか調べよ。
\end{problem}

\begin{answer}
    連続全単射$f \colon (0, 1] \to (0, 1)$が存在したとして矛盾を導く。
    $s \coloneqq f(1)$とおく。
    制限$f|_{(0, 1)}$を$F$とおく。
    $F$は$(0, 1)$から$(0, s) \cup (s, 1)$への連続全単射であり、
    \begin{equation}
        F^{-1}((0, s) \cup (s, 1))
            = F^{-1}((0, s)) \cup F^{-1}((s, 1))
            = (0, 1)
    \end{equation}
    をみたす。ここで、$F$の連続性より
    $F^{-1}((0, s)), F^{-1}((s, 1))$は
    disjoint open sets in $(0, 1)$ だから、
    $(0, 1)$は連結でないことになり矛盾が従う。
\end{answer}

\begin{problem}[幾何学II 1.2]
    $[0, 1]$から$[0, 1] \times [0, 1]$への
    連続な全単射が存在するかどうか調べよ。
\end{problem}

\begin{answer}
    連続全単射$f \colon [0, 1] \to [0, 1] \times [0, 1]$が存在するとして矛盾を導く。
    $[0, 1]$のコンパクト性と$[0, 1] \times [0, 1]$の Hausdorff 性より、
    $f$は同相写像である (\cref{thm:compact-to-Hausdorff})。
    いま$[0, 1] \times [0, 1]$は3個以上の点を含むから、
    $[0, 1] \times [0, 1]$の点$a$であって
    $[0, 1]$の端点以外に対応するもの、すなわち
    $f^{-1}(a) \in (0, 1)$なるものが存在する。
    このとき制限$f|_{[0, 1] \setminus \{ f^{-1}(a) \}}$は
    \begin{equation}
        [0, 1] \setminus \{ f^{-1}(a) \}
        \to
        ([0, 1] \times [0, 1]) \setminus \{a\}
    \end{equation}
    の同相写像である。
    ところが、左辺は弧状連結でなく、右辺は弧状連結だから、
    弧状連結性の位相不変性に矛盾する。
\end{answer}

\begin{problem}[幾何学II 1.3]
    直線$\R$から円周$S^1$への連続な全単射が存在するかどうか調べよ。
\end{problem}

\begin{answer}
    題意の連続全単射$f \colon \R \to S^1$が存在するとして矛盾を導く。
    合成
    \begin{equation}
        \begin{tikzcd}
            \R \setminus \{0\} \ar{r}{f}
                & S^1 \setminus \{f(0)\} \ar{r}{\approx}
                & \R
        \end{tikzcd}
    \end{equation}
    を$F$とおくと、$F$も連続全単射である。
    連続単射であることより$F$は狭義単調であり、
    必要ならばさらに同相写像$x \mapsto -x$を合成することで
    $F$は狭義単調増加であるとしてよい。

    ここで、空でない開区間$I$と狭義単調な連続写像$g \colon \R \to \R$に対し
    $J \coloneqq g(I)$が$\R$の開区間であることを示す。
    まず区間であることを示す。
    いま$I$は区間ゆえに弧状連結だから、$g$の連続性より$J$も弧状連結である。
    もし$J$が区間でないとすると、
    $J$の或る2点$x < y$がとれて、$x$と$y$の間の点$c$で$J$に属さないようなものが存在する。
    一方、$J$の弧状連結性により$x, y$をつなぐ$J$内のパス$\gamma$がとれて、
    中間値定理より$\gamma$は$c$を通る。$c$は$J$に属さないから矛盾。
    よって$J$は区間である。
    つぎに$J$が開区間であることを示す。
    $y \in J$を任意の点とすると、
    $g^{-1}(y)$は開区間$I$の点だから、
    或る$x, x' \in I$であって
    $x < g^{-1}(y) < x'$であるものがとれる。
    $g$の狭義単調性より$g(x) < y < g(x')$が成り立つ。
    $g(x), g(x') \in J$だから、$y$は$J$の端点ではない。
    したがって$J$は開区間であることがいえた。

    上の段落の議論から$F((-\infty, 0)), F((0, \infty))$は開区間である。
    したがって
    \begin{equation}
        \R = F(\R \setminus \{0\}) = F((-\infty, 0)) \cup F((0, \infty))
    \end{equation}
    は disjoint open intervals の和の形である。
    これは$\R$の連結性に矛盾。
\end{answer}

\begin{problem}[幾何学II 1.4]
    次の集合が互いに同相かどうか調べよ。
    \begin{enumerate}
        \item $\R^2 \setminus \{(0, 0)\}$
        \item $\R^2 \setminus ([-1, 1] \times \{0\})$
        \item $\R^2 \setminus \{(x, y) \colon x^2 + y^2 \le 1\}$
    \end{enumerate}
\end{problem}

\begin{answer}
    すべて互いに同相となることを示す。

    $\underline{(1) \approx (2)}$\quad
    写像$f \colon \R^2 \setminus \{(0, 0)\} \to \R^2 \setminus ([-1, 1] \times \{0\})$を
    \begin{equation}
        (x, y) \mapsto \left(\frac{1 + r}{r} x, \frac{\sqrt{r (2 + r)}}{r} y\right),
        \quad r = \sqrt{x^2 + y^2}
    \end{equation}
    で定める。このとき$r > 0$であることに注意すれば$f$は連続である。
    図形的には、点$(x, y)$は
    焦点$\pm 1$、長半径$1 + r$の楕円上に写る
    (短半径はこれらの情報から決まる)。
    楕円の方程式を書いておくと
    \begin{equation}
        \left(\frac{p}{1 + r}\right)^2 + \left(\frac{q}{\sqrt{r (2 + r)}}\right)^2 = 1
    \end{equation}
    である。

    念のため、連続写像の族$F_\alpha \colon X \to Y_\alpha\; (\alpha \in A)$の直積
    $F \colon X \to \prod_{\alpha \in A} Y_\alpha$が連続写像であることを示しておく。
    $V \opensubset \prod_{\alpha \in A} Y_\alpha$を任意の開集合とする。
    積位相の準開基の定め方から
    \begin{gather}
        V = \bigcup_{\lambda \in \Lambda}
            \bigcap_{\alpha \in A_\lambda}
            \pi_\alpha^{-1}(V_\alpha) \\
        A_\lambda \subset A \colon \text{finite}, \quad
        V_\alpha \opensubset Y_\alpha
    \end{gather}
    と表せる。よって
    \begin{alignat}{1}
        F^{-1}(V)
            &= \bigcup_{\lambda \in \Lambda}
                \bigcap_{\alpha \in A_\lambda}
                (\pi_\alpha \circ F)^{-1}(V_\alpha) \\
            &= \bigcup_{\lambda \in \Lambda}
                \bigcap_{\alpha \in A_\lambda}
                \underbrace{F_\alpha^{-1}(V_\alpha)}_{\text{open}}
    \end{alignat}
    が成り立ち、これは open in $X$である。
    よって$F$は連続であることがいえた。

    さて、$f$の連続な逆写像を構成する。
    連続写像$g \colon \R^2 \setminus ([-1, 1] \times \{0\}) \to \R^2 \setminus \{(0, 0)\}$を
    \begin{equation}
        (p, q) \mapsto \left(\frac{R}{1 + R} p, \frac{R}{\sqrt{R (2 + R)}} q\right),
        \quad R = -1 + \frac{1}{2}\left(
            \sqrt{(p - 1)^2 + q^2} + \sqrt{(p + 1)^2 + q^2}
        \right)
    \end{equation}
    と定めることができる。
    実際、図形的に考えて$R$の定義式の括弧内は$> 2$だから$R > 0$であり
    (あるいは三角不等式を用いてもう少し厳密にも示せる)、
    さらに$(p, q) \neq (0, 0)$も満たされている。

    $g$が$f$の逆写像であることを示す。まず
    \begin{alignat}{1}
        f \circ g(p, q)
            &= f\left(\frac{R}{1 + R} p, \frac{R}{\sqrt{R (2 + R)}} q\right)
    \end{alignat}
    である。ここで
    \begin{alignat}{1}
        r
            &= \sqrt{
                \left(\frac{R}{1 + R}\right)^2 p^2
                +
                \left(\frac{R}{\sqrt{R (2 + R)}}\right)^2 q^2
            } \\
            &= R \sqrt{
                \left(\frac{p}{1 + R}\right)^2
                +
                \left(\frac{q}{\sqrt{R (2 + R)}}\right)^2
            }
    \end{alignat}
    であるが、$R$の定義から
    \begin{equation}
        \sqrt{(p - 1)^2 + q^2} + \sqrt{(p + 1)^2 + q^2}
            = 2 (R + 1)
            = (1 + R - 1) + (1 + R + 1)
    \end{equation}
    なので、図形的に考えれば、点$(p, q)$は冒頭の楕円の方程式で
    $r$を$R$に置き換えたものを満たす。
    したがって
    \begin{equation}
        r = R
    \end{equation}
    を得る。よって
    \begin{equation}
        f \circ g(p, q)
            = \left(
                \frac{1 + r}{r} \frac{R}{1 + R} p,
                \frac{\sqrt{r (2 + r)}}{r} \frac{R}{\sqrt{R (2 + R)}} q
            \right)
            = (p, q)
    \end{equation}
    である。

    つぎに
    \begin{equation}
        g \circ f(x, y)
            = g\left(
                \frac{1 + r}{r} x,
                \frac{\sqrt{r (2 + r)}}{r} y
            \right)
    \end{equation}
    である。
    ここで
    \begin{alignat}{1}
        R
            &= -1 + \frac{1}{2}\left(
                \sqrt{\left(\frac{1 + r}{r} x - 1\right)^2
                    + \left(\frac{\sqrt{r (2 + r)}}{r} y\right)^2}
                +
                \sqrt{\left(\frac{1 + r}{r} x + 1\right)^2
                    + \left(\frac{\sqrt{r (2 + r)}}{r} y\right)^2}
            \right) \\
            &= -1 + \frac{1}{2}\left(
                \sqrt{\left( (1 + r) - \frac{x}{r} \right)^2}
                +
                \sqrt{\left( (1 + r) + \frac{x}{r} \right)^2}
            \right) \quad (\because r = \sqrt{x^2 + y^2}) \\
            &= -1 + \frac{1}{2}\left(
                (1 + r) - \frac{x}{r} + (1 + r) + \frac{x}{r}
            \right) \\
            &= r
    \end{alignat}
    なので
    \begin{equation}
        g \circ f(x, y)
            = \left(
                \frac{R}{1 + R} \frac{1 + r}{r} x,
                \frac{R}{\sqrt{R (2 + R)}} \frac{\sqrt{r (2 + r)}}{r} y
            \right)
            = (x, y)
    \end{equation}
    である。
    よって$f$は$\R^2 \setminus \{(0, 0)\}$から
    $\R^2 \setminus ([-1, 1] \times \{0\})$への同相写像であることがいえた。

    $\underline{(1) \approx (3)}$\quad
    写像$f \colon \R^2 \setminus \{(0, 0)\}
    \to \R^2 \setminus \{ (x, y) \colon x^2 + y^2 \le 1 \}$を
    \begin{equation}
        (x, y) \mapsto \left(
            \frac{1 + r}{r} x,
            \frac{1 + r}{r} y
        \right),
        \quad r = \sqrt{x^2 + y^2}
    \end{equation}
    で定める。このとき$r > 0$であることに注意すれば$f$は連続である。
    $f$の連続な逆写像は
    \begin{equation}
        (p, q) \mapsto \left(
            \frac{R}{1 + R} p,
            \frac{R}{1 + R} q,
        \right),
        \quad R = \sqrt{p^2 + q^2} - 1
    \end{equation}
    で与えられる。
    よって$f$は$\R^2 \setminus \{(0, 0)\}$から
    $\R^2 \setminus \{ (x, y) \colon x^2 + y^2 \le 1 \}$への同相写像であることがいえた。
\end{answer}


\begin{problem}[幾何学II 1.5]
    次の集合が互いに同相かどうか調べよ。
    \begin{enumerate}
        \item $\{ (x, y) \in \R^2 \colon x^2 + y^2 = 1 \} \cup [-1, 1] \times \{0\}$
        \item $\{ (x, y) \in \R^2 \colon (x + 1)^2 + y^2 = 1 \}
            \cup \{ (x, y) \in \R^2 \colon (x - 1)^2 + y^2 = 1 \}$
    \end{enumerate}
\end{problem}

\begin{answer}
    集合$(1), (2)$をそれぞれ$A, B$とおく。
    $A, B$が同相であるとすると、
    $B$から点$(0, 0)$を抜いた集合は
    $A$から1点を抜いた集合と同相である。
    ところが、$B \setminus \{(0, 0)\}$は2個の弧状連結成分を持つのに対し、
    $A$から1点を抜いた集合は1個しか弧状連結成分を持たない。矛盾。
\end{answer}

\begin{problem}[幾何学II 1.6]
    次の集合が互いに同相かどうか調べよ。
    \begin{enumerate}
        \item $[0, 1]^2 \setminus ([0, 1] \times \{0\})$
        \item $[0, 1]^2 \setminus ((0, 1) \times \{0\})$
    \end{enumerate}
\end{problem}

\begin{answer}
    (1) は局所コンパクトだが
    (2) は点$(0, 0)$のコンパクトな近傍がとれないから
    局所コンパクトでない。
    したがって (1) と (2) は同相でない。
\end{answer}


\begin{problem}[幾何学II 1.8]
    $\R^2$上の関係
    \begin{equation}
        (x, y) \sim (x', y') \quad \Leftrightarrow \quad
            (x - x', y - y') \in \Z \times \Z
    \end{equation}
    は同値関係となる。
    $\R^2/\sim \approx S^1 \times S^1$を示せ。
\end{problem}

\begin{answer}
    商写像$\R^2 \to \R^2/~$を$\pi$とおく。
    題意の同相を示すため、写像$f \colon \R^2 \to S^1 \times S^1,$
    \begin{equation}
        (x, y) \mapsto (e^{2\pi i x}, e^{2\pi i y})
    \end{equation}
    を考える。これは連続写像の積・合成・直積だから連続である。
    また、
    \begin{alignat}{1}
        \pi(x, y) = \pi(x', y')
            &\Rightarrow (x - x', y - y') \in \Z \times \Z \\
            &\Rightarrow (e^{2\pi i x}, e^{2\pi i y}) = (e^{2\pi i x'}, e^{2\pi i y'}) \\
            &\Rightarrow f(x, y) = f(x', y')
    \end{alignat}
    が成り立つから、商位相空間の普遍性より、図式
    \begin{equation}
        \begin{tikzcd}[row sep=large]
            \R^2
                \ar{r}{f} \ar{d}[swap]{\pi}
                & S^1 \times S^1 \\
            \R^2 / \sim
                \ar{ru}[swap]{\wb{f}}
        \end{tikzcd}
    \end{equation}
    を可換にする連続写像$\wb{f}$が誘導される。
    $\wb{f}$は逆写像$g \colon S^1 \times S^1 \to \R^2/\sim,$
    \begin{equation}
        (e^{2\pi i x}, e^{2\pi i y}) \mapsto \pi(x, y)\quad
        (x, y \in [0, 1))
    \end{equation}
    をもつ。実際、
    \begin{alignat}{1}
        \wb{f} \circ g(e^{2\pi i x}, e^{2\pi i y})
            &= \wb{f}(\pi(x, y)) \\
            &= f(x, y) \\
            &= (e^{2\pi i x}, e^{2\pi i y}) \\
        g \circ \wb{f}(\pi(x, y))
            &= g(f(x, y)) \\
            &= g(e^{2\pi i x}, e^{2\pi i y}) \\
            &= g(e^{2\pi i (x - \lfloor x \rfloor)} e^{2\pi i \lfloor x \rfloor},
                e^{2\pi i (y - \lfloor y \rfloor)} e^{2\pi i \lfloor y \rfloor}) \\
            &= g(e^{2\pi i (x - \lfloor x \rfloor)},
                e^{2\pi i (y - \lfloor y \rfloor)} \\
            &= \pi(x - \lfloor x \rfloor, y - \lfloor y \rfloor) \\
            &= \pi(x, y)
    \end{alignat}
    が成り立つ。
    よって$\wb{f}$は$\R^2/\sim$から$S^1 \times S^1$への連続全単射である。
    ここで$\R^2/\sim$はコンパクト空間$[0, 1] \times [0, 1]$の
    連続写像$\pi$による像ゆえにコンパクトである。
    一方、$S^1$は距離空間$\R^2$の部分空間だから Hausdorff である。
    よって$S^1 \times S^1$は Hausdorff である。
    したがって、$\wb{f}$はコンパクト空間から Hausdorff 空間への連続全単射だから
    同相写像である。
\end{answer}

\begin{problem}[幾何学II 1.9]
    写像$f \colon [0, \infty) \to \R^2,$
    \begin{equation}
        f(t) \coloneqq \begin{cases}
            (t \cos(2\pi / t), t \sin(2\pi / t)) & (t > 0) \\
            (0, 0) & (t = 0)
        \end{cases}
    \end{equation}
    は部分集合$(0, \infty)$上連続である。
    $f$が中への同相写像か調べよ。
\end{problem}

\begin{answer}
    \TODO{より「集合と位相」的な議論で連続性を示したい}
    $f$の像を$X \coloneqq f([0, \infty))$とおく。
    $X$の概形は、$t = 1/n$のとき
    $x$軸と$1/n$で交わるような
    時計回りの渦巻きのような図形である。

    まず、$f$が$t = 0$においても連続であることを示す。
    そこで$\eps > 0$を任意とする。
    $\delta \coloneqq \eps^2\; (> 0)$とおけば、
    各$t \in [0, \delta)$に対し
    \begin{alignat}{1}
        |f(t) - f(0)|
            &= |(t \cos(2\pi / t), t \sin(2\pi / t)) - (0, 0)| \\
            &= \sqrt{t^2 \cos^2(2\pi / t) + t^2 \sin^2(2\pi / t)} \\
            &= \sqrt{t} \\
            &< \sqrt{\delta} \\
            &= \eps
    \end{alignat}
    が成り立つ。したがって、$f$は$t = 0$において連続である。

    また、連続写像$g \colon X \to [0, \infty),$
    \begin{equation}
        (x, y) \mapsto \sqrt{x^2 + y^2}
    \end{equation}
    が$f$の逆写像となる。
    以上より、$f$は中への同相写像である。
    ただし、$g$が$f$の逆写像であることは
    \TODO{}
\end{answer}

\begin{problem}[幾何学II 1.10]
    開区間$(-\pi, \pi)$の各点$t$に
    平面上の点$(\sin t, \sin t \cos t) \in \R^2$を対応させる写像
    $f \colon (-\pi, \pi) \to \R^2$は連続である。
    写像$f$が中への同相写像かどうか調べよ。
\end{problem}

\begin{answer}
    $f$は中への同相写像でないことを示す。
    $f$の像を$X \coloneqq f([0, \infty))$とおく。
    $X$の概形は、「$\infty$」のような図形であって、
    原点から左上に出発して一筆書きをしながら原点に右下から帰ってくるようなものである。
    $f$が中への同相写像であったとすると、
    $X$から1点$(1/\sqrt{2}, 1/2) = f(\pi/4)$を抜いた集合
    $X' \coloneqq X \setminus \{(1/\sqrt{2}, 1/2)\}$は
    $(-\pi, \pi/4) \cup (\pi/4, \pi)$と同相である。
    ここで、$X'$の2点$(0, 0), (1, 0)$は
    $X'$内のパス
    \begin{equation}
        [\pi/2, \pi] \to X',\quad
        t \mapsto (\sin t, \sin t \cos t)
    \end{equation}
    でつなぐことができるから、ひとつの弧状連結成分に属する。
    一方、これら2点を$f$で引き戻した
    $0, \pi/2 \in (-\pi, \pi/4) \cup (\pi/4, \pi)$は
    それぞれ相異なる弧状連結成分に属する。矛盾。
\end{answer}

\subsection{幾何学II 練習問題}

\begin{problem}[幾何学II 練習問題6]
    \label[problem]{problem:geometry2-ex-6}
    $\R^n \; (n \ge 2)$から余次元$2$の線型部分空間$V$を
    除いて得られる集合$\R^n \setminus V$は弧状連結であることを示せ。
\end{problem}

\begin{remark}
    この証明と同様の方法で
    $\R^n \; (n \ge 3)$から余次元$3$の部分空間を除いた空間は
    単連結であることが示せる。
\end{remark}

\begin{answer}
    $x, y \in \R^n \setminus V$とする。
    $x, y$の$V$の直交補空間$V^\perp$への射影をそれぞれ$x', y'$とする。
    このとき$x$から$x'$への線分$\gamma_x$は$V$と交わりをもたない。
    実際、ある$t \in I$に対し
    $(1 - t) x + tx' = v \in V$であったとすると、
    $V^\perp$への射影は
    $x' = (1 - t) x' + tx' = 0$
    となるから、
    直和分解$V \oplus V^\perp$に沿って
    $x$をそれぞれの空間の元の和に表すと
    $x = (x - x') + x' = x - x' \in V$
    が成り立ち矛盾。
    同様に$y'$から$y$への線分$\gamma_y$も$V$と交わりをもたない。
    また、同相
    $V^\perp \setminus V = V^\perp \setminus \{ 0 \}$
    は弧状連結空間$\R^2 \setminus \{ 0 \}$と同相だから
    $x'$と$y'$をつなぐ$V^\perp \setminus V$内のパス$\gamma$が存在する。
    以上をまとめると
    パスの合成$\gamma_x \cdot \gamma \cdot \gamma_y$が
    $x$と$y$をつなぐ$\R^n \setminus V$内のパスとなる。
    よって$\R^n \setminus V$は弧状連結である。
\end{answer}

\begin{problem}[幾何学II 練習問題14]
    \label[problem]{problem:geometry2-ex-14}
    直線$\R$の部分集合$X$の部分集合$A$に対して、
    $X$において$A$を1点に縮めて得られる空間$X / A$を考え、
    商写像を$p \colon X \to X / A$とおく。
    このとき$p(X) \setminus p(A)$は
    $X \setminus A$と同相といえるか?
\end{problem}

\begin{answer}
    反例を挙げる。
    $X = [0, 2], \; A = \{ 0 \} \cup (1, 2]$とおく。
    まず$p(X) = X / A$は$S^1$と同相である。
    実際、写像
    \begin{equation}
        f \colon X \to S^1,
        \quad
        t \mapsto \begin{cases}
            e^{2\pi it} & t \in [0, 1] \\
            1 & t \in [1, 2]
        \end{cases}
    \end{equation}
    を考えるとこれは全射で、
    貼り合わせ補題より連続である。
    $f$は$p$のファイバー上定値だから、
    $p(X)$がコンパクト集合$X$の連続像ゆえにコンパクトで
    $S^1$が Hausdorff であることとあわせて
    同相$p(X) \to S^1$が誘導される。
    よって$p(X) \setminus p(A)
        \approx S^1 \setminus \{ 1 \}
        \approx (0, 1)$である。
    したがって$p(X) \setminus p(A) \approx (0, 1]$は
    $X \setminus A \approx (0, 1)$と同相ではない。
    \begin{innerproof}
        同相であったとすると
        $(0, 1]$から$1$を抜いた空間$(0, 1)$ (これは連結である) が
        $(0, 1)$から1点を除いた空間 (これは連結でない) と同相になり矛盾する。
    \end{innerproof}
\end{answer}






% ============================================================
%
% ============================================================
\chapter{基本的な位相空間}

この章ではいくつかの基本的な位相空間について調べる。
まとまりを良くするために、ホモロジーなど現段階でまだ扱っていない概念も含めて述べておく。

% ------------------------------------------------------------
%
% ------------------------------------------------------------
\section{球面}

\begin{definition}
    \TODO{}
\end{definition}

\begin{lemma}
    \label[lemma]{lemma:disk-over-boundary-is-homeo-to-sphere}
    $D^n / \del D^n$は$S^n$と同相である。
\end{lemma}

\begin{proof}
    \TODO{}
\end{proof}

\begin{definition}[立体射影]
    $S^n$の北極を$N = (0, \dots, 0, 1)$とおく。
    連続写像
    \begin{equation}
        \begin{tikzcd}
            S^n \setminus \{ N \}
                \ar{r}
                & \R^n \\
            x = (x_1, \dots, x_n)
                \ar[mapsto]{r}
                & \left( \frac{x_1}{1-x_n}, \dots, \frac{x_n}{1-x_n} \right)
        \end{tikzcd}
    \end{equation}
    は連続逆写像
    \begin{equation}
        \begin{tikzcd}
            \R^n
                \ar{r}
                & S^n \setminus \{ N \} \\
            y
                \ar[mapsto]{r}
                & \frac{\|y\|^2 - 1}{\|y\|^2 + 1} N + \frac{2}{\|y\|^2 + 1} y
        \end{tikzcd}
    \end{equation}
    をもつ。したがって同相である。
    これを\term{立体射影}[stereographic projection]{立体射影}[りったいしゃえい]という。
\end{definition}

% ------------------------------------------------------------
%
% ------------------------------------------------------------
\section{線型空間}

\begin{lemma}
    \label[lemma]{lemma:codimension-2-space-path-connected}
    $\R^n \; (n \ge 2)$から余次元$2$の線型部分空間$V$を
    除いて得られる集合$\R^n \setminus V$は弧状連結である。
\end{lemma}

\begin{proof}
    \cref{problem:geometry2-ex-6}を参照。
\end{proof}

% ------------------------------------------------------------
%
% ------------------------------------------------------------
\section{行列の空間}

実行列や複素行列は線型代数で慣れ親しんだ主題である。
ここでは行列の空間とその位相的性質について調べる。

\begin{definition}
    $n \in \Z_{\ge 1}$とする。
    全行列空間$M(n, \K)$は$\K^{n^2}$と同一視して位相が入っているとする。
    \begin{enumerate}
        \item
            \term{一般線型群}[general linear group]{一般線型群}[いっぱんせんけいぐん]
            \begin{equation}
                \GL(n, \K)
                    \coloneqq \{
                        A \in M(n, \K) \mid \det A \neq 0
                    \}
            \end{equation}
            \term{特殊線型群}[special linear group]{特殊線型群}[とくしゅせんけいぐん]
            \begin{equation}
                \SL(n, \K)
                    \coloneqq \{
                        A \in M(n, \K) \mid \det A = 1
                    \}
            \end{equation}
    \end{enumerate}
\end{definition}

\begin{definition}[直交群とユニタリ群]
    $n \in \Z_{\ge 1}$とする。
    \begin{enumerate}
        \item
            \term{直交群}[orthogonal group]{直交群}[ちょっこうぐん]
            \begin{equation}
                \O(n)
                    \coloneqq \{
                        A \in \GL(n, \R) \mid A \up{t}A = I
                    \}
            \end{equation}
            \term{特殊直交群}[special orthogonal group]{特殊直交群}[とくしゅちょっこうぐん]
            あるいは
            \term{回転群}[rotation group]{回転群}[かいてんぐん]
            \begin{equation}
                \SO(n)
                    \coloneqq \{
                        A \in \SL(n, \R) \mid A \up{t}A = I
                    \}
            \end{equation}
        \item
            \term{ユニタリ群}[unitary group]{ユニタリ群}[ユニタリぐん]
            \begin{equation}
                \U(n)
                    \coloneqq \{
                        A \in \GL(n, \C) \mid AA^* = I
                    \}
            \end{equation}
            \term{特殊ユニタリ群}[special unitary group]{特殊ユニタリ群}[とくしゅユニタリぐん]
            \begin{equation}
                \SU(n)
                    \coloneqq \{
                        A \in \SL(n, \C) \mid AA^* = I
                    \}
            \end{equation}
    \end{enumerate}
\end{definition}

線型代数的な種々の変形操作を利用して
空間の性質を調べよう。
次の命題では行列の基本変形を用いる。

\begin{proposition}
    $\GL(n, \C)$は弧状連結である。
\end{proposition}

\begin{proof}
    基本変形によって示す。
    $A \in \GL(n, \C)$とする。
    $A$と$I_n$をつなぐ$\GL(n, \C)$内のパスの存在をいえばよい。
    $A$は正則だから左右から有限個の基本行列を掛けることで
    $I_n$が得られる。
    そこで、それぞれの基本行列が
    $\GL(n, \C)$内のパスで$I_n$とつながることをいえばよい。
    第$i, j$行の入れ替えの基本行列は
    \begin{equation}
        t \mapsto \left[\begin{smallmatrix}
            1 \\
            & \ddots \\
            & & 1 - t & \cdots & t \\
            & & \vdots & & \vdots \\
            & & t & \cdots & 1 - t \\
            & & & & & & \ddots \\
            & & & & & & & 1
        \end{smallmatrix}\right]
    \end{equation}
    で$I_n$からのパスが得られる。
    第$i$行に第$j$行の$m$倍を加える基本行列は
    \begin{equation}
        t \mapsto \left[\begin{smallmatrix}
            1 \\
            & \ddots \\
            & & 1 \\
            & & \vdots & \ddots \\
            & & tm & \cdots & 1 \\
            & & & & & & \ddots \\
            & & & & & & & 1
        \end{smallmatrix}\right]
    \end{equation}
    で$I_n$からのパスが得られる。
    第$i$行を$m \; (m \neq 0)$倍する基本行列は
    \begin{equation}
        t \mapsto \left[\begin{smallmatrix}
            1 \\
            & \ddots \\
            & & \beta_m(t) \\
            & & & \ddots \\
            & & & & 1
        \end{smallmatrix}\right]
    \end{equation}
    で$I_n$からのパスが得られる。
    ただし$\beta_m$は
    $\C^\times$が弧状連結であることより存在する
    $1$から$m$への$\C^\times$内のパスである。
    以上で主張が示せた。
\end{proof}

逆行列をとる操作は連続である。

\begin{proposition}[逆行列]
    写像$\GL(n, \K) \mapsto \GL(n, \K), \; A \mapsto A^{-1}$は連続である。
\end{proposition}

\begin{proof}
    Cramer の公式を用いる。

    \TODO{}
\end{proof}

Gram-Schmidt の正規直交化は連続写像であり、
より強く同相写像を与える。

\begin{proposition}[Gram-Schmidt の正規直交化]
    Gram-Schmidt の正規直交化は連続写像
    \begin{enumerate}
        \item $\GL(n, \R) \to \O(n) \times T(n, \R)$
        \item $\GL(n, \C) \to \U(n) \times T(n, \C)$
    \end{enumerate}
    を与える。
    さらにこれらはそれぞれ行列の積を逆写像として同相写像となる。
\end{proposition}

\begin{proof}
    \TODO{}
\end{proof}

Gram-Schmidt の正規直交化を用いて次がわかる。

\begin{proposition}[一般線型群の変形レトラクト]
    \begin{enumerate}
        \item $\GL(n, \R)$は$\O(n)$を変形レトラクトにもつ。
        \item $\GL(n, \C)$は$\U(n)$を変形レトラクトにもつ。
    \end{enumerate}
\end{proposition}

\begin{proof}
    $T$が$\{ I_n \}$を変形レトラクトにもつことを使う。

    \TODO{}
\end{proof}

$\SU(2)$は$S^3$と同相であることを示そう。
$\SU(2)$は次のように表せることに注意しておく。

\begin{lemma}
    \begin{equation}
        \SU(2) = \left\{
            \begin{bmatrix}
                a & b \\
                - \wb{b} & \wb{a}
            \end{bmatrix} \in M(2, \C)
            \; \middle| \;
            a, b \in \C, \;
            |a|^2 + |b|^2 = 1
        \right\}
    \end{equation}
    と表せる。
\end{lemma}

\begin{proof}
    右辺が左辺に含まれることは明らか。逆の包含を示す。
    $A = \begin{bmatrix}
        a & b \\
        c & d
    \end{bmatrix} \in \SU(2)$とする。
    $A A^* = I_2, \; \det A = 1$よりとくに次が成り立つ。
    \begin{alignat}{1}
        a \wb{a} + b \wb{b} &= 1 \label{eq:su2-representation-1} \\
        %c \wb{c} + d \wb{d} &= 1 \label{eq:su2-representation-2} \\
        a \wb{c} + b \wb{d} &= 0 \label{eq:su2-representation-3} \\
        ad - bc &= 1 \label{eq:su2-representation-4}
    \end{alignat}
    \cref{eq:su2-representation-1}から$|a|^2 + |b|^2 = 1$を得る。
    $a = 0$の場合、$|b| = 1$だから
    \cref{eq:su2-representation-3}より$d = 0$を得て、
    \cref{eq:su2-representation-4}より$c = -\wb{b}$を得る。
    よって包含が成り立つ。

    $a \neq 0$の場合、
    \cref{eq:su2-representation-4}の両辺に$\wb{a}$を掛けて
    $|a|^2 d - bc\wb{a} = \wb{a}$を得る。
    \cref{eq:su2-representation-3}とあわせて
    $(|a|^2 + |b|^2) d = \wb{a}$を得る。
    さらに\cref{eq:su2-representation-1}とあわせて
    $d = \wb{a}$を得る。
    いま$a \neq 0$だから
    \cref{eq:su2-representation-3}とあわせて
    $c = -\wb{b}$を得る。
    よって包含が成り立つ。
\end{proof}

\begin{proposition}
    $\SU(2)$は$S^3$と同相である。
\end{proposition}

\begin{proof}
    写像$\SU(2) \to S^3, \;
        \begin{bmatrix}
            a & b \\
            - \wb{b} & \wb{a}
        \end{bmatrix}
        \mapsto (a, b)$
    が同相を与える。
\end{proof}

$\SO(3)$は$\R P^3$と同相であることを示そう。

\begin{proposition}
    $\SO(3)$は$\R P^3$と同相である。
\end{proposition}

\begin{proof}
    \TODO{}
\end{proof}


% ------------------------------------------------------------
%
% ------------------------------------------------------------
\section{トーラス}

\begin{definition}
    \TODO{}
\end{definition}

% ------------------------------------------------------------
%
% ------------------------------------------------------------
\section{M\"{o}bius の帯}

M\"{o}bius の帯は
ホモロジーから位相が決まらない例のひとつでもある。

\begin{definition}[M\"{o}bius の帯]
    ~
    \begin{enumerate}
        \item $[0, 1] \times [0, 1]$上の同値関係$\sim$を
            $(0, y) \sim (1, 1 - y)$により生成されるものとして定める。
            この同値関係に関する商空間を
            \term{境界を持つ M\"{o}bius の帯}[M\"{o}bius band with boundary]
                {M\"{o}bius の帯}[M\"{o}bius のおび]
            という。
        \item $[0, 1] \times (0, 1)$上の同値関係$\sim$を
            $(0, y) \sim (1, 1 - y)$により生成されるものとして定める。
            この同値関係に関する商空間を
            \term{境界を持たない M\"{o}bius の帯}[M\"{o}bius band without boundary]
                {M\"{o}bius の帯}[M\"{o}bius のおび]
            という。
    \end{enumerate}
\end{definition}

\begin{proposition}[境界を持つ M\"{o}bius の帯と円柱は同相でない]
    境界を持つ M\"{o}bius の帯と境界を持つ円柱は同相でない。
\end{proposition}

\begin{proof}
    $M$を境界を持つ M\"{o}bius の帯、$C$を境界を持つ円柱とする。
    同相写像$\varphi \colon M \to C$が存在したとすると、
    $\varphi$の制限により多様体としての境界$\del M$と$\del C$は同相となる。
    $\del M$は弧状連結だが$\del C$は弧状連結でないから矛盾。
\end{proof}

\begin{proposition}[境界を持たない M\"{o}bius の帯と円柱は同相でない]
    境界を持たない M\"{o}bius の帯と境界を持たない円柱は同相でない。
\end{proposition}

\begin{proof}
    とくに同相写像$\varphi \colon E \to S^1 \times \R$が存在する。
    $K \coloneqq S^1 \times \{ 0 \} \subset S^1 \times \R$、
    $K' \coloneqq \varphi^{-1}(K)$とおく。
    $K$はコンパクトだから$L$もコンパクトである。
    このとき$p^{-1}(L)$は$[0, 1] \times \R$の有界閉集合である。
    \begin{innerproof}
        閉であることは明らか。
        もし有界でなかったとすると
        任意の$n \in \Z_{\ge 1}$に対し
        $y_n > |n|$なる点$(x_n, y_n)$が$p^{-1}(K')$に含まれ、
        したがって点$p(x_n, y_n)$が$K'$に含まれる。
        すると$K'$の開被覆
        $\{ p([0, 1] \times (-n, n)) \}_{n \in \Z_{\ge 1}}$
        が有限部分被覆を持たないから
        $K'$はコンパクトでないことになり矛盾。
    \end{innerproof}
    よってある$a > 0$が存在して
    $[0, 1] \times [-a, a] \supset p^{-1}(L)$、
    したがって$L' \coloneqq p([0, 1] \times [-a, a]) \supset L$となる。
    $L'$はコンパクトだから
    $K' \coloneqq \varphi(L')$もコンパクトである。
    したがって
    ある$b > 0$が存在して
    $S^1 \times (-b, b) \supset K' \supset K$をみたす。
    よって$(S^1 \times \R) \setminus K'$は連結でない。
    一方$E \setminus L'$は連結である。
    これで矛盾がいえた。
\end{proof}

% ------------------------------------------------------------
%
% ------------------------------------------------------------
\section{射影空間}

射影空間は、位相群の作用による商として定義される重要な空間のひとつである。
射影空間$\K P^n$は$\K^{n + 1}$内の原点を通る直線をひとつの点とみなして
それらを集めた空間とみなすことができる。
すなわち Grassmann 多様体と呼ばれる多様体の特別な場合である。
ここでは複素射影空間と実射影空間について述べる。

\section{基本性質}

\begin{definition}[射影空間]
    $\K = \R$または$\C$とし、$n \ge 1$とする。
    このとき、乗法群$\K^\times$の連続作用
    $\K^\times \curvearrowright \K^{n + 1} \setminus \{ 0 \}$
    に関する軌道空間を
    \begin{equation}
        \K P^n \coloneqq (\K^{n + 1} \setminus \{ 0 \}) / \K^\times
    \end{equation}
    とおき、これを
    \term{$n$次元$\K$射影空間}[$n$-dimensional $\K$-projective space]
    {射影空間}[しゃえいくうかん]
    という。
   標準射影$\K^{n + 1} \setminus \{ 0 \} \to \K P^n$を
    $\varpi$とおく。
\end{definition}

\begin{definition}[斉次座標・非斉次座標]
    \TODO{}
\end{definition}

\begin{lemma}
    射影空間は Hausdorff である。
\end{lemma}

\begin{proof}
    \cref{prop:orbit-space-Hausdorff} より成り立つ。
\end{proof}

斉次多項式により
射影空間上に写像が誘導される。

\begin{theorem}[斉次多項式から誘導される写像]
    \TODO{}
\end{theorem}

\begin{proof}
    \TODO{}
\end{proof}

$\K^{n + 1}$上の線型自己同型は
射影空間上に同相写像を誘導する。

\begin{theorem}[射影変換]
    $A \in \GL(n + 1, \K)$とする。
    写像
    \begin{equation}
        \K P^n \to \K P^n,
        \quad
        [z] \mapsto [Az]
    \end{equation}
    は well-defined であり同相写像となる。
    これを\term{射影変換}[projective transformation]{射影変換}[しゃえいへんかん]という。
\end{theorem}

\begin{proof}
    \TODO{}
\end{proof}


\section{複素射影空間}

Hopf ファイブレーションは
複素射影空間の具体的な計算に役立つ。

\begin{theorem}[Hopf ファイブレーション]
    商写像$\C^{n + 1} \setminus \{ 0 \} \to \C P^n$を$\varpi$とおき、
    \begin{equation}
        S^{2n + 1} = \{
            z \in \C^{n + 1} \mid \| z \| = 1
        \}
    \end{equation}
    とみなして$\pi \coloneqq \varpi|_{S^{2n + 1}}$とおく。
    このとき$\pi$は同相$S^{2n + 1} / S^1 \approx \C P^n$を誘導する。
    ただし、$S^1$の作用$S^1 \curvearrowright S^{2n + 1}$は
    作用$\C^\times \curvearrowright \C^{n + 1} \setminus \{ 0 \}$の制限により定める。
    $\pi$を \term{Hopf ファイブレーション}{Hopf fibration} という。
    \begin{equation}
        \begin{tikzcd}
            S^{2n + 1}
                \ar{r}{\pi = \varpi|_{S^{2n + 1}}}
                \ar[twoheadrightarrow]{d}
                & \C P^n \\
            S^{2n + 1} / S^1
                \ar[dashed]{ru}[swap]{\approx}
        \end{tikzcd}
    \end{equation}
\end{theorem}

\begin{proof}
    $\pi$が連続であることは明らか。
    $\pi$が$S^{2n + 1}$から$\C P^n$の上への全射であることは
    各$[z] \in \C P^n, \; z \in \C^{n + 1} \setminus \{ 0 \}$に対し
    $\pi(z / \| z \|) = [z]$が成り立つことからわかる。
    単射性は、
    $z, z' \in S^{2n + 1}$に関し
    $\pi(z) = \pi(z')$ならば
    $z = \alpha z' \; (\exists \alpha \in \C^\times)$であり、
    $|z| = |z'| = 1$ゆえに$|\alpha| = 1$
    すなわち$\alpha \in S^1$となることより従う。
    したがって$\pi$により連続全単射
    $\wb{\pi} \colon S^{2n + 1} / S^1 \to \C P^n$が誘導されるが、
    いま$S^{2n + 1} / S^1$はコンパクトで$\C P^n$は Hausdorff だから
    $\wb{\pi}$は同相である。
\end{proof}

射影空間の部分集合のうちよく現れるものに次がある:
\begin{enumerate}
    \item $\{ z_n = 0 \} \subset \C P^n$
    \item $\{ z_n \neq 0 \} \subset \C P^n$
    \item $\C P^n$から1点を除いた空間
\end{enumerate}
これらの空間について調べよう。

\begin{lemma}
    $\{ z_n = 0 \} \subset \C P^n$は$\C P^{n - 1}$に同相である。
\end{lemma}

\begin{proof}
    \begin{equation}
        [z_0 : \dots : z_{n - 1} : 0] \mapsto [z_0 : \dots : z_{n - 1}]
    \end{equation}
    \TODO{}
\end{proof}

\begin{lemma}
    $\{ z_n \neq 0 \} \subset \C P^n$は$\C^n$に同相である。
\end{lemma}

\begin{proof}
    \begin{equation}
        [z_0 : \dots : z_{n - 1} : z_n]
            \mapsto [z_0 / z_n : \dots : z_{n - 1} / z_n]
    \end{equation}
    \TODO{}
\end{proof}

以上の2つの補題によりとくに
\begin{equation}
    \C P^n = \C P^{n - 1} \sqcup \C^n
\end{equation}
と表せることがわかった。
この関係は複素射影空間の胞体的ホモロジーを考える際に役立つ。
さらに次の連続写像は基本的である。

\begin{lemma}[$\C P^n$の胞体構造]
    \label[lemma]{lemma:cpn-cell-structure}
    $D^{2n} \subset \C^n$とみなして写像
    \begin{equation}
        \varphi^{2n} = \varphi \colon D^{2n} \to \C P^n,
            \quad
            w \mapsto [w_0 : \dots : w_{n - 1} : \sqrt{1 - \| w \|^2}]
    \end{equation}
    は商写像であり、
    制限$\varphi|_{\mathring{D}^{2n}}$は
    $\{ z_n \neq 0 \}$の上への同相写像となる。
    $\varphi|_{\mathring{D}^{2n}}$の連続逆写像は
    \begin{equation}
        \psi \colon \{ z_n \neq 0 \} \to \mathring{D}^{2n},
            \quad
            [z_0 : \dots : z_{n - 1} : z_n]
                \mapsto \frac{|z_n|}{\| z \|} \left(
                    \frac{z_0}{z_n} : \dots : \frac{z_{n - 1}}{z_n}
                \right)
    \end{equation}
    で与えられる。
\end{lemma}

\begin{proof}
    $\varphi$および$\psi$が連続であることは定義から明らか。
    $\psi$が$\varphi|_{\mathring{D}^{2n}}$の逆写像であることは
    直接計算によりわかる。
    したがって$\varphi|_{\mathring{D}^{2n}}$は
    $\{ z_n \neq 0 \}$の上への同相写像であり、
    逆写像は$\psi$である。
    また、$\varphi$の定義から明らかに
    $\varphi(\del D^{2n}) = \{ z_n = 0 \}$だから
    内部の対応とあわせて$\varphi$は全射である。
    $D^{2n}$はコンパクトで$\C P^n$は Hausdorff だから
    \cref{thm:compact-to-Hausdorff}より
    $\varphi$は閉写像である。
    したがって$\varphi$は全射かつ連続な閉写像だから、
    \cref{prop:surj-closed-cts-map-is-quotient-map}より
    $\varphi$は商写像である。
\end{proof}

次に$\C P^n$から1点を抜いた空間を調べる。

\begin{lemma}
    $\C P^{n}$から任意の1点を除いた空間は互いに同相である。
\end{lemma}

\begin{proof}
    $P \in \C P^{n}$とし
    $P_0 \coloneqq [1 : 0 : \dots : 0]$とおく。
    $\C P^{n} \setminus \{ P_0 \} \approx \C P^{n} \setminus \{ P \}$
    を示せばよい。
    そこで Hopf ファイブレーション
    $S^{2n + 1} \to \C P^n$を$\pi$とおくと
    \begin{alignat}{1}
        P_0 &= \pi(p_0), \quad p_0 \coloneqq \up{t}(1, 0, \dots, 0) \\
        P &= \pi(p), \quad p \in S^{2n + 1}
    \end{alignat}
    と表せる。
    まず$S^{2n + 1}$のレベルで同相写像を構成し、
    そこから求める同相を誘導する。
    いま$p \neq 0 \in \R^{2n + 2}$だから、
    $\R^{2n + 2}$の元を付け加えて
    $\R^{2n + 2}$の基底$p, x_1, \dots, x_{2n + 1}$が得られる。
    Gram-Schmidt の直交化により
    正規直交基底$p, y_1, \dots, y_{2n + 1}$を得る。
    これらを並べた$(2n + 2)$次行列を
    $A \coloneqq [p y_1 \dots y_{2n + 1}]$とおくと、
    $A$は直交行列だから
    同相写像$S^{2n + 1} \to S^{2n + 1}, \; x \mapsto Ax$
    が定まる。
    ここで$Ap_0 = p, \; A(-p_0) = -p$だから
    $S^{2n + 1} \setminus \{ p_0, -p_0 \}
        \approx S^{2n + 1} \setminus \{ p, -p \}$
    すなわち
    $\pi^{-1}(\C P^n \setminus \{ P_0 \})
        \approx \pi^{-1}(\C P^n \setminus \{ P \})$
    が成り立つ。
    この両辺は$\pi$に関し saturated な$S^{2n + 1}$の開集合だから
    それぞれへの$\pi$の制限は等化写像となる。
    これにより同相
    $\C P^{n} \setminus \{ P_0 \} \approx \C P^{n} \setminus \{ P \}$
    が誘導され、主張が示せた。
\end{proof}

上の補題を用いれば、
複素射影空間から1点を除いた空間は
次元をひとつ下げた複素射影空間とホモトピー同値であることが示せる。

\begin{proposition}[$\C P^{n + 1}$から1点を除いた空間]
    \label[proposition]{prop:cp-minus-a-point}
    $\C P^{n + 1}$から1点を除いた空間は
    $\C P^n$とホモトピー同値である。
\end{proposition}

\begin{proof}
    標準射$\C^{n + 2} \setminus \{ 0 \} \to \C P^{n + 1}$を$\pi$とおく。
    上の補題より、$P_0 \coloneqq [0 : \dots : 0 : 1]$とおいて
    $\C P^{n + 1} \setminus \{ P_0 \} \simeqhe \C P^n$を示せば十分。
    そこで埋め込み
    \begin{equation}
        \iota \colon \C P^n \to \C P^{n + 1} \setminus \{ P_0 \},
        \quad
        [z_0 : \dots : z_n] \mapsto [z_0 : \dots : z_n : 0]
    \end{equation}
    を考え、$\iota(\C P^n)$が$\C P^{n + 1} \setminus \{ P_0 \}$の
    変形レトラクトであることを示せばよい。
    集合$U \coloneqq \pi^{-1}(\C P^{n + 1} \setminus \{ P_0 \})$は
    $\pi$に関し saturated な
    $\C^{n + 2} \setminus \{ 0 \}$の開部分集合だから、
    $\pi|_U$は等化写像である。
    そこで連続写像
    \begin{equation}
        r \colon \C P^{n + 1} \setminus \{ P_0 \} \to \iota(\C P^n),
        \quad
        [z_0 : \dots : z_n : z_{n + 1}] \mapsto [z_0 : \dots : z_n : 0]
    \end{equation}
    が誘導される。
    このとき$r$は変形レトラクションとなる。
    実際、J. H. C. Whitehead の補題より連続写像
    \begin{equation}
        H \colon \C P^{n + 1} \setminus \{ P_0 \} \times I
            \to \C P^{n + 1} \setminus \{ P_0 \},
        \quad
        ([z_0 : \dots : z_n : z_{n + 1}], t) \mapsto
            [z_0 : \dots : z_n : tz_{n + 1}]
    \end{equation}
    が誘導され、これがホモトピーとなるからである。
    したがって$\C P^{n + 1} \setminus \{ P_0 \} \simeqhe \C P^n$が示せた。
\end{proof}

$\C P^n$の部分空間$\C P^{n - 1}$を1点に縮めると
$S^{2n}$が得られる。

\begin{lemma}
    $n \in \Z_{\ge 1}$に対し
    $\C P^n / \C P^{n - 1}$は$S^{2n}$と同相である。
    ただし$\C P^{n - 1} = \{ z_n = 0 \} \subset \C P^n$の意味である。
\end{lemma}

\begin{proof}
    \cref{lemma:cpn-cell-structure}の連続写像
    $\varphi^{2n} \colon D^{2n} \to \C P^n$を用いる。
    図式
    \begin{equation}
        \begin{tikzcd}
            D^{2n}
                \ar[twoheadrightarrow]{r}{\varphi^{2n}}
                \ar[twoheadrightarrow]{d}
                & \C P^n
                    \ar[twoheadrightarrow]{d} \\
            S^{2n} \approx D^{2n} / \del D^{2n}
                \ar[dashed]{r}[swap]{\wb{\varphi}}
                & \C P^n / \C P^{n - 1}
        \end{tikzcd}
    \end{equation}
    を可換にする連続全単射$\wb{\varphi}$が誘導され、
    \TODO{$\wb{\varphi}$が単射であるのはなぜ?}
    $S^{2n}$はコンパクトで
    $\C P^n / \C P^{n - 1}$は Hausdorff だから
    $\wb{\varphi}$は同相である。
    \begin{innerproof}
        $\C P^n / \C P^{n - 1}$が Hausdorff であることを示す。
        \TODO{}
    \end{innerproof}
\end{proof}

とくに次の系が従う。

\begin{corollary}
    \label[lemma]{lemma:cp1-s2-homeomorphic}
    $\C P^1$は$S^2$と同相である。
    \qed
\end{corollary}

$\C P^1$と$S^2$が同相であることを利用して
$\C P^2$の単連結性を示す。

\begin{lemma}
    $\C P^2$は単連結である。
\end{lemma}

\begin{proof}
    $\C P^2$から1点を抜いた空間$U, V$で被覆し
    \cref{thm:fundamental-group-mayer-vietoris}を用いる。
    $U \cap V$は$\C^3 \setminus \{ 0 \}$に引き戻すと
    $\C^3$から$(z_0, 0, 0)$や$(0, z_1, 0)$の形の元を除いた空間になる。
    これは弧状連結であることが示せるから
    $U \cap V$も弧状連結である。
    \TODO{}
\end{proof}

$\C P^2$内の平面射影曲線の簡単な例を調べよう。
次の証明は\cite{川又01}を参考にした。
\TODO{Fermat curve についても書きたい}

\begin{proposition}
    空間
    \begin{equation}
        X = \{
            [z_0 : z_1 : z_2] \in \C P^2
            \mid
            z_0^2 + z_1^2 + z_2^2 = 0
        \}
    \end{equation}
    は$S^2$と同相である。
\end{proposition}

\begin{proof}
    \cref{lemma:cp1-s2-homeomorphic}より
    $\C P^1 \approx S^2$だから$X \approx \C P^1$を示せばよい。
    $z_0^2 + z_1^2 + z_2^2 = (z_0 + iz_1)(z_0 - iz_1) - (iz_2)^2$
    と変形できることに着目して、線型同型
    \begin{equation}
        F \colon \C^3 \to \C^3,
        \quad
        z \mapsto \begin{bmatrix}
            1 & i & 0 \\
            1 & -i & 0 \\
            0 & 0 & i
        \end{bmatrix}
        z
    \end{equation}
    により定まる射影変換$\wb{F} \colon \C P^2 \to \C P^2$を考える。
    $\wb{F}(z_0, z_1, z_2) = (z_0 + iz_1, z_0 - iz_1, iz_2)$だから
    \begin{equation}
        \wb{F}(X) = \{
            [w_0 : w_1 : w_2] \in \C P^2
            \mid
            w_0 w_1 - w_2^2 = 0
        \}
    \end{equation}
    である。
    $\wb{F}(X) \approx \C P^1$を示す。
    そこで$f \colon \C P^1 \to \C P^2$を
    $f([\zeta_0 : \zeta_1]) \coloneqq [\zeta_0^2 : \zeta_1^2 : \zeta_0 \zeta_1]$
    で定める。
    右辺は斉次だからこれは well-defined な連続写像であり、
    また像が$\wb{F}(X)$に入ることも明らか。
    つぎに$g \colon \wb{F}(X) \to \C P^1$を
    \begin{equation}
        g([w_0 : w_1 : w_2]) \coloneqq \begin{cases}
            [w_0 : w_2] & (w_0 \neq 0) \\
            [w_2 : w_1] & (w_1 \neq 0)
        \end{cases}
    \end{equation}
    で定める。
    $w_0 w_1 - w_2^2 = 0$ゆえに
    $w_0 = 0$と$w_1 = 0$が同時に成り立つことはないから
    場合分けはこれで十分で、
    共通部分では$w_0 \neq 0, \; w_1 \neq 0, \; w_0 w_1 - w_2^2 = 0$より
    $w_2 \neq 0$も成り立つことから
    \begin{equation}
        [w_0 : w_2]
            = [w_0 w_1 : w_1 w_2]
            = [w_2^2 : w_1 w_2]
            = [w_2 : w_1]
    \end{equation}
    となる。
    よって$g$は well-defined である。
    直接計算により$f, g$は互いに逆写像であることがわかる。
    したがって$f$は連続全単射で、
    $\C P^1$がコンパクト、$\wb{F}(X)$が Hausdorff であることから
    $f$は同相である。
    よって$S^1 \approx \C P^1 \approx \wb{F}(X) \approx X$がいえた。
\end{proof}



\section{実射影空間}

実射影空間も複素射影空間の場合と同様に
Hopf ファイブレーションが考えられる。

\begin{theorem}[Hopf ファイブレーション]
    商写像$\R^{n + 1} \setminus \{ 0 \} \to \R P^n$を$\varpi$とおき、
    $\pi \coloneqq \varpi|_{S^n}$とおく。
    このとき$\pi$は同相$S^n / \{ \pm 1 \} \approx \R P^n$を誘導する。
    $\pi$を \term{Hopf ファイブレーション}{Hopf fibration} という。
\end{theorem}

\begin{proof}
    \TODO{}
\end{proof}

実射影空間の場合、
Hopf ファイブレーションは普遍被覆になっている。

\begin{proposition}[実射影空間の普遍被覆]
    \TODO{}
\end{proposition}

\begin{proof}
    \TODO{}
\end{proof}

実射影空間は次のように表示することもできる。

\begin{theorem}[実射影空間の表示]
    対蹠点の同一視により
    $\R P^n \approx D^n / \sim$
    \TODO{}
\end{theorem}

\begin{proof}
    \TODO{}
\end{proof}

複素射影空間の場合と同様に次の補題が成り立つ。
証明は複素の場合と全く同様だから省略する。

\begin{lemma}
    $\{ x_n = 0 \} \subset \R P^n$は$\R P^{n - 1}$に同相である。
    \qed
\end{lemma}

\begin{lemma}
    $\{ x_n \neq 0 \} \subset \R P^n$は$\R^n$に同相である。
    \qed
\end{lemma}

\begin{lemma}
    $D^{n} \subset \R^n$とみなして写像
    \begin{equation}
        f \colon D^{n} \to \R P^n,
            \quad
            y \mapsto [y_0 : \dots : y_{n - 1} : \sqrt{1 - \| y \|^2}]
    \end{equation}
    は$\{ x_n \neq 0 \}$の上への同相となる。
    逆写像は
    \begin{equation}
        g \colon \{ x_n \neq 0 \} \to D^{n},
            \quad
            [x_0 : \dots : x_{n - 1} : x_n]
                \mapsto \frac{|x_n|}{\| x \|} \left(
                    \frac{x_0}{x_n} : \dots : \frac{x_{n - 1}}{x_n}
                \right)
    \end{equation}
    で与えられる。
    \qed
\end{lemma}


% ------------------------------------------------------------
%
% ------------------------------------------------------------
\section{Klein の壺}

\begin{definition}[Klein の壺]
    \TODO{}
\end{definition}



\end{document}

\newpage
\documentclass[report]{jlreq}
\usepackage{global}
\usepackage{./local}
\subfiletrue
\def\assetspath{../}
\begin{document}


% ============================================================
%
% ============================================================
\chapter{基本群と被覆空間}

基本群と被覆空間について述べる。

% ----------------------------------------------------------------------------
%
% ----------------------------------------------------------------------------
\section{空間対}

空間対の概念を導入する。

\begin{definition}[空間対]
    \idxsym{category of pairs of spaces}{$\CatTopPair$}{空間対の圏}
    圏$\CatTopPair$を次のように定める:
    \begin{itemize}
        \item $\Ob(\CatTopPair)$は
            位相空間$X$とその部分空間$A \subset X$の対$(X, A)$の全体
        \item 各$(X, A), (Y, B) \in \Ob(\CatTopPair)$に対し、
            $\Ar((X, A), (Y, B))$は
            連続写像$f \colon X \to Y$であって$f(A) \subset B$なるものの全体
    \end{itemize}
    $\CatTopPair$を
    \term{空間対の圏}[category of pairs of spaces]{空間対の圏}[くうかんついのけん]
    という。
\end{definition}

\begin{definition}[空間対のホモトピー]
    $f, g \colon (X, A) \to (Y, B)$を空間対の射とする。
    $f, g$が\term{ホモトピック}[homotopic]{ホモトピック!空間対---}[ほもとぴっく]
    であるとは、
    空間対の射$H \colon (X \times I, A \times I) \to (Y, B)$であって
    $H \colon X \times I \to Y$が
    $f \colon X \to Y$を$g \colon X \to Y$に
    つなぐホモトピーであるようなものが存在することをいう。
    空間対の
    \term{ホモトピー同値}[homotopy equivalent]{ホモトピー同値!空間対---}[ほもとぴーどうち]
    や
    \term{ホモトピー同値射}[homotopy equivalence]{ホモトピー同値射!空間対---}[ほもとぴーどうちしゃ]
    も同様に定義する。
\end{definition}

% ------------------------------------------------------------
%
% ------------------------------------------------------------
\section{ホモトピー}

ホモトピーについて述べる。
ホモトピーの概念は基本群やホモロジーの重要な基盤となる。

\subsection{ホモトピーの定義と基本性質}

\TODO{最初から空間対で定義する?}

\TODO{相対ホモトピーは後で導入すべき?
    空間対は包含だけど相対ホモトピーは固定だから違うもの?}

ホモトピーとは、大まかには2つの写像の間の連続的な変形を表すものである。

\begin{definition}[相対ホモトピー]
    \idxsym{homotopic maps rel $A$}{$f \simeq g \; \rel \; A$}{rel $A$ でホモトピックな写像}
    $f, g \colon X \to Y$を連続写像、
    $A \subset X$、
    $f = g \; \text{on} \; A$とする。
    連続写像$H \colon X \times I \to Y$であって
    \begin{alignat}{1}
        H(x, 0) &= f(x) \quad (\forall x \in X) \\
        H(x, 1) &= g(x) \quad (\forall x \in X) \\
        H(a, t) &= f(a) = g(a) \quad (\forall a \in A, \; t \in I)
    \end{alignat}
    をみたすものが存在するとき
    $f \simeqhe g \; \rel \; A$と書き、
    $f$と$g$は
    \term{rel $A$ でホモトピック}[homotopic rel $A$]{ホモトピック!相対---}
    であるという。
    $H$を$f$と$g$をつなぐ rel $A$な
    \term{相対ホモトピー}[relative homotopy]{ホモトピー!相対---}
    という。
\end{definition}

\begin{definition}[自由ホモトピー]
    \idxsym{homotopic maps}{$f \simeq g$}{ホモトピックな写像}
    $f \simeqhe g \; \rel \; \emptyset$のとき
    $f \simeqhe g$と書き、
    $f$と$g$は
    \term{ホモトピック}[homotopic]{ホモトピック}
    であるという。
    $f$と$g$をつなぐ rel $\emptyset$な相対ホモトピーを
    \term{自由ホモトピー}[free homotopy]{ホモトピー!自由---}
    あるいは単に
    \term{ホモトピー}[homotopy]{ホモトピー}
    という
    \footnote{
        連続写像$H$が$f, g \colon X \to Y$をつなぐホモトピーであることは、
        次の可換図式で表される:
        \begin{equation}
            \begin{tikzcd}[column sep=huge, ampersand replacement=\&]
                X \ar{d}[swap]{x \mapsto (x, 0)} \ar{dr}{f} \\
                X \times I \ar{r}{H} \& Y \\
                X \ar{u}{x \mapsto (x, 1)} \ar{ur}[swap]{g}
            \end{tikzcd}
        \end{equation}
    }。
\end{definition}

\TODO{antipodal map の例を挙げたい}

\begin{example}[ホモトピックな写像の例]
    連続写像$f, g \colon \R \to \R^2$を
    \begin{equation}
        f(x) \coloneqq (x, x^2), \quad g(x) \coloneqq (x, x)
    \end{equation}
    で定めると、線型ホモトピー$H(x, t) \coloneqq (x, (1 - t) x^2 + t x)$により
    $f, g$はホモトピックとなる。
\end{example}

\begin{example}[ホモトピックでない写像の例]
    連続写像$f, g \colon S^1 \to S^1$を
    \begin{equation}
        f(z) \coloneqq 1, \quad g(z) \coloneqq z
    \end{equation}
    で定めると、$f, g$はホモトピックではない。
    このことは$S^1$が単連結でないという (後で証明する) 事実を用いて示される。
\end{example}

ホモトピックな写像は
左や右から連続写像を合成してもホモトピックである。

\begin{proposition}[ホモトピックな写像と連続写像の合成]
    \label[proposition]{prop:homotopic-cts-composition}
    $X, Y, Z$を位相空間、
    $f, g \colon X \to Y$および
    $h \colon Y \to Z, \; i \colon Z \to X$を
    連続写像とする。
    $f \simeq g$のとき
    \begin{equation}
        h \circ f \simeq h \circ g, \quad f \circ i \simeq g \circ i
    \end{equation}
    が成り立つ。
\end{proposition}

\begin{proof}
    $f \simeq g$より、図式
    \begin{equation}
        \begin{tikzcd}[column sep=large]
            X \ar{d}[swap]{x \mapsto (x, 0)} \ar{dr}{f} \\
            X \times I \ar{r}{F} & Y \\
            X \ar{u}{x \mapsto (x, 1)} \ar{ur}[swap]{g}
        \end{tikzcd}
    \end{equation}
    を可換にするホモトピー$F$が存在する。そこで、図式
    \begin{equation}
        \begin{tikzcd}[column sep=large]
            X \ar{d}[swap]{x \mapsto (x, 0)} \ar{dr}{f} \\
            X \times I \ar{r}{F} & Y \ar{r}{h} & Z \\
            X \ar{u}{x \mapsto (x, 1)} \ar{ur}[swap]{g}
        \end{tikzcd}
    \end{equation}
    を考えれば、ホモトピー$h \circ F$により
    \begin{equation}
        h \circ f \simeq h \circ g
    \end{equation}
    が成り立つことがわかる。
    また、図式
    \begin{equation}
        \begin{tikzcd}[column sep=large, row sep=large]
            Z \ar{r}{i} \ar{d}[swap]{z \mapsto (z, 0)}
                & X \ar{d}[swap]{x \mapsto (x, 0)} \ar{dr}{f} \\
            Z \times I \ar{r}{i \times \id}
                & X \times I \ar{r}{F} & Y \\
            Z \ar{r}[swap]{i} \ar{u}{z \mapsto (z, 1)}
                & X \ar{u}{x \mapsto (x, 1)} \ar{ur}[swap]{g}
        \end{tikzcd}
    \end{equation}
    を考えれば、ホモトピー$F \circ (i \times \id)$により
    \begin{equation}
        f \circ i \simeq g \circ i
    \end{equation}
    が成り立つことがわかる。
\end{proof}

\begin{lemma}[$\simeq$は同値関係]
    $X, Y$を位相空間、$A \subseteq X$とする。
    このとき、$\simeq \; rel A$は
    $X$から$Y$への連続写像全体の集合上の同値関係である。
\end{lemma}

\begin{proof}
    \TODO{何に使う?}
\end{proof}

\begin{definition}[パスホモトピー類]
    $X$を位相空間、
    $x_0 \in X$とする。
    $x_0$を基点とするループ$\gamma$に対し、
    同値関係$\simeq \; \rel \; \{0, 1\}$に関する
    $\gamma$の同値類を
    \term{パスホモトピー類}[path homotopy class]{パスホモトピー類}といい、
    $[\gamma]$と書く。
\end{definition}



\subsection{ホモトピー同値}

ホモトピーの概念を用いて、
空間のホモトピー同値性を定義する。

\begin{definition}[ホモトピー同値]
    $X, Y$を位相空間とする。
    \begin{itemize}
        \item 連続写像$f \colon X \to Y,\; g \colon Y \to X$が存在して
            $f \circ g \simeq 1_Y,\; g \circ f \simeq 1_X$をみたすとき、
            $X, Y$は\term{ホモトピー同値}[homotopy equivalent]{ホモトピー!---同値}あるいは
            同じ\term{ホモトピー型}[homotopy type]{ホモトピー!---型}を持つという。
        \item 上の$f, g$を\term{ホモトピー同値写像}[homotopy equivalence]
            {ホモトピー!---同値写像}という。
        \item $X, Y$がホモトピー同値であることを
            $X \simeqhe Y$と書く。
    \end{itemize}
\end{definition}

\begin{remark}[同相ならばホモトピー同値]
    上の定義で$X, Y$が同相ならば、
    「$\simeq$」が「$=$」で成り立つから、とくにホモトピー同値である。
\end{remark}

\begin{example}[ホモトピー同値な空間の例]
    $\{0\}$と$\R$はホモトピー同値である。
    実際、$f \colon \{0\} \to \R, x \mapsto 0,\; g \colon \R \to \{0\}, y \mapsto 0$とおけば、
    $g \circ f = 1_{\{0\}}$であるし、
    $f \circ g$は線型ホモトピーにより$1_\R$とホモトピックである。
\end{example}

\begin{example}[ホモトピー同値でない空間の例]
    $\C \setminus \{0\}$と$\C$はホモトピー同値でない。
    このことは$S^1$が単連結でないという (後で証明する) 事実を用いて示される。
\end{example}

\begin{lemma}[ホモトピー同値は同値関係]
    ホモトピー同値は位相空間の間の同値関係である。
\end{lemma}

\begin{proof}
    省略
\end{proof}

\subsection{レトラクト}

基本群やホモロジーと相性の良い部分空間のクラスとして
レトラクトや変形レトラクトがある。

\begin{definition}[レトラクション]
    \label[definition]{def:retraction}
    $X$を位相空間、$A \subset X$を部分空間とする。
    連続写像$r \colon X \to A$が
    \term{レトラクション}[retrcaction]{レトラクション}
    であるとは、
    包含写像$i \colon A \to X$に対し
    $r \circ i = \id_A$が成り立つことをいう。
    このとき、$A$は$X$の\term{レトラクト}[retract]{レトラクト}であるという。
\end{definition}

Hausdorff 空間のレトラクトは閉集合でなければならない。

\begin{proposition}[Hausdorff 空間のレトラクトは閉集合]
    $X$を Hausdorff 位相空間、
    $A \subset X$、
    $r \colon X \to A$をレトラクションとする。
    このとき$A$は閉集合である。
\end{proposition}

\begin{proof}
    $i \colon A \to X$を包含写像とすると
    $A$は$i \circ r$の不動点である。
    したがって\cref{corollary:Hausdorff-fixed-points-closed}より
    $A$は$X$の閉集合である。
\end{proof}

$X$のレトラクト$A$であって$X$を$A$に連続的に変形できるようなものは
変形レトラクトと呼ばれる。

\begin{definition}[変形レトラクション]
    $X$を位相空間、$A \subset X$を部分空間とする。
    ホモトピー$H \colon X \times I \to X$が
    $X$から$A$への
    \term{変形レトラクション}[deformation retraction]
        {変形レトラクション}[へんけいれとらくしょん]
    であるとは、次が成り立つことをいう:
    \begin{alignat}{1}
        H(x, 0) &= x \quad (\forall x \in X) \\
        H(x, 1) &\in A \quad (\forall x \in X) \\
        H(a, 1) &= a \quad (\forall a \in A)
    \end{alignat}
    このとき、$A$は$X$の
    \term{変形レトラクト}[deformation retract]{変形レトラクト}[へんけいれとらくと]
    であるという。
\end{definition}

\begin{lemma}[変形レトラクトとのホモトピー同値]
    $X$を位相空間とし、$A \subset X$を部分空間とする。
    $A$が$X$の変形レトラクトならば、$X$と$A$はホモトピー同値である。
\end{lemma}

\begin{proof}
    $H \colon X \times I \to X$を$X$から$A$への変形レトラクションとし、
    包含写像$A \to X$を$\iota$とおく。
    このとき、連続写像$r \colon X \to A$を
    \begin{equation}
        x \mapsto H(x, 1)
    \end{equation}
    で定めることができる。
    $X$と$A$がホモトピー同値であることを示すには、
    \begin{equation}
        \begin{cases}
            \iota \circ r \simeq 1_X \\
            r \circ \iota \simeq 1_A
        \end{cases}
    \end{equation}
    をいえばよいが、
    1個目は$H$が$1_X$と$\iota \circ r$をつなぐホモトピーであることから成り立ち、
    2個目は$H$が$X$から$A$への変形レトラクションであることから等号で成り立つ。
\end{proof}

\subsection{可縮空間}

ホモトピー同値性を用いて定義される空間のクラスのうち
最も重要なもののひとつが可縮空間である。

\begin{definition}[可縮空間]
    $X$を位相空間とする。$X$が1点からなる空間とホモトピー同値であるとき、
    $X$は\term{可縮}[contractible]{可縮}[かしゅく]であるという。
    \cref{prop:homotopic-cts-composition}より明らかに
    可縮性は位相不変である。
\end{definition}

\begin{proposition}[可縮空間の特徴付け]
    位相空間$X$に対し
    次は同値である:
    \begin{enumerate}
        \item $X$は可縮である。
        \item $\id_X$はある定値写像$X \to X, x \mapsto c$とホモトピックである。
        \item $X$は1点からなる変形レトラクトをもつ。
    \end{enumerate}
\end{proposition}

\begin{proof}
    \uline{(1) \Rightarrow (2)} \quad
    $X$を可縮とすると
    ホモトピー同値写像$f \colon X \to *, \; g \colon * \to X$が存在して
    とくに定値写像$g \circ f \colon x \mapsto g(*)$は$\id_X$とホモトピックである。

    \uline{(2) \Rightarrow (3)} \quad
    明らかに$\{ c \} \subset X$が$X$の変形レトラクトとなる。

    \uline{(3) \Rightarrow (1)} \quad
    $X$の変形レトラクトは$X$とホモトピー同値である。
\end{proof}

\begin{example}[可縮空間の例]
    凸集合は明らかに可縮である。
    また、関数$x \mapsto x^2$のグラフ$\Gamma \coloneqq \{ (x, x^2) \in \R^2 \colon x \in \R \}$は
    凸集合ではないが可縮である。
    実際、ホモトピー$H \colon \Gamma \times I \to \Gamma$を
    \begin{equation}
        H((x, y), t) \coloneqq ((1 - t) x, ((1 - t) x)^2) \quad (\forall (x, y) \in \Gamma, t \in I)
    \end{equation}
    により$1_X$は原点での定値ループとホモトピックとなる。
\end{example}



% ------------------------------------------------------------
%
% ------------------------------------------------------------
\section{基本群}

基本群について述べる。

\subsection{第0ホモトピー集合}

弧状連結空間の基本的な事項は
\cref{subsection:path-connected-space}で述べた。
ここでは基本群の導入への橋渡しとして
空間の弧状連結成分全体の集合を考える。

\begin{definition}[第0ホモトピー集合]
    \idxsym{set of path-connected components}{$\pi_0(X)$}{$X$の弧状連結成分全体の集合}
    $X$を位相空間とする。
    $X$の弧状連結成分全体の集合を
    $\pi_0(X)$と書き、
    $X$の\term{第0ホモトピー集合}{第0ホモトピー集合}[だい0ほもとぴーしゅうごう]
    という。
\end{definition}

\begin{proposition}[第0ホモトピー集合上に誘導される写像]
    位相空間$X, Y$に対して、
    $X$から$Y$への連続写像は$\pi_0(X)$から$\pi_0(Y)$への
    写像を誘導する。
    また、$X$から$Y$への互いにホモトピックな連続写像が誘導する写像は一致する。
\end{proposition}

\begin{proof}
    \cref{problem:geometry2-2.5}を参照。
\end{proof}

\begin{corollary}
    $X, Y$がホモトピー同値ならば
    $\pi_0(X), \pi_0(Y)$の濃度は一致する。
    \qed
\end{corollary}



\subsection{基本群の定義と基本性質}

\TODO{基本亜群を経由した定義に修正する?それは同型類だから別物?}

パスホモトピー類を用いて基本群を定義する。

\begin{definition}[パスの合成・反転]
    \label[definition]{def:path-composition}
    $X$を位相空間、$x_0 \in X$とする。
    この節だけの記号として、$x_0$を基点とする$X$内のループ全体の集合を
    $\mathrm{Loop}(X, x_0)$と書き、
    定値ループ$x_0$を$c_{x_0}$と書くことにする。
    $\alpha, \beta \in \mathrm{Loop}(X, x_0)$に対し、
    パスの\term{合成}{合成!パスの---}[ごうせい] $\alpha * \beta$
    およびパスの\term{反転}{反転!パスの---}[はんてん]$\bar{\alpha}$をそれぞれ
    \begin{alignat}{1}
        \alpha * \beta (s) &\coloneqq \begin{cases}
            \alpha(2s) \quad \text{if } s \in [0, 1/2] \\
            \beta(2s - 1) \quad \text{if } s \in [1/2, 1]
        \end{cases} \\
        \bar{\alpha} (s) &\coloneqq \alpha(1 - s)
    \end{alignat}
    で定義する (写像の合成$\circ$と逆向きであることに注意)。
\end{definition}

\begin{remark}[$\mathrm{Loop}(X, x_0)$は群でない]
    \cref{def:path-composition}の状況で、
    一見$\mathrm{Loop}(X, x_0)$はパスの合成と反転により群になりそうだが、一般にはそうはならない。
    実際、たとえば$X = S^1, x_0 = 1$のとき、$S^1$を反時計回りに1周するループを$\gamma$とすると、
    $\gamma * \bar{\gamma} \neq \bar{\gamma} * \gamma$である。
\end{remark}

\begin{definition}[基本群]
    $X$を位相空間、$x_0 \in X$とする。パスホモトピー類の集合
    \begin{equation}
        \pi_1(X, x_0)
            \coloneqq \{ [\gamma] \colon \gamma \in \mathrm{Loop}(X, x_0) \}
    \end{equation}
    は、パスの合成と反転により定まる演算
    \begin{alignat}{1}
        [\alpha] \cdot [\beta] &\coloneqq [\alpha * \beta] \\
        [\alpha]^{-1} &\coloneqq [\bar{\alpha}]
    \end{alignat}
    をそれぞれ積、逆元として群をなし、単位元は$[c_{x_0}]$となる(証明略)。
    群$\pi_1(X, x_0)$を、$x_0$を基点とする
    $X$の\term{基本群}[fundamental group]{基本群}[きほんぐん]という
    \footnotemark{}。
\end{definition}

\footnotetext{
    $\pi_1$は点付き位相空間の圏$\mathbf{Top_*}$から$\mathbf{Groups}$への共変関手である。
    \begin{equation}
        \begin{tikzcd}[ampersand replacement=\&]
            \mathbf{Top_*} \ar{r}{\pi_1} \& \mathbf{Groups}
        \end{tikzcd}
    \end{equation}
}

\begin{remark}[弧状連結空間の基本群]
    $X$を弧状連結な位相空間、$x_0, x_1 \in X$とすると、
    弧状連結性より$x_0$と$x_1$をつなぐ$X$内のパス$h$がとれる。
    そこで、基点の取り換えの写像$\beta_h \colon \pi_1(X, x_0) \to \pi_1(X, x_1)$を
    \begin{equation}
        \beta_h([\gamma]) \coloneqq [h * \gamma * \bar{h}]
    \end{equation}
    で定めると、$\beta_h$は群同型となる(証明略)。
    これを表す可換図式が次である:
    \begin{equation}
        \begin{tikzcd}[row sep=normal, column sep=huge]
            \mathrm{Loop}(X, x_0)
                \ar{d}[swap]{[\,\cdot\,]}
                \ar{r}{\gamma \mapsto h \cdot \gamma \cdot \bar{h}}
                & \mathrm{Loop}(X, x_1) \ar{d}{[\,\cdot\,]} \\
            \pi_1(X, x_0) \ar{r}{\beta_h}[swap]{\cong} & \pi_1(X, x_1)
        \end{tikzcd}
    \end{equation}
    よって、群構造のみを問題にするときは
    基点を省略して$\pi_1(X)$と書くことがある。
\end{remark}

\begin{definition}[連続写像により誘導される準同型]
    $X, Y$を位相空間、$x_0 \in X$とし、$f \colon X \to Y$を連続写像とする。
    このとき、写像$f_* \colon \pi_1(X, x_0) \to \pi_1(Y, f(x_0)),$
    \begin{equation}
        f_*([\gamma]) \coloneqq [f \circ \gamma]
    \end{equation}
    は群準同型として well-defined である (このあと示す)。
    これを表す可換図式が次である:
    \begin{equation}
        \begin{tikzcd}
            \mathrm{Loop}(X, x_0) \ar{r}{f \circ} \ar{d}[swap]{[\,\cdot\,]}
                & \mathrm{Loop}(Y, f(x_0)) \ar{d}{[\,\cdot\,]} \\
            \pi_1(X, x_0) \ar[dashed]{r}[swap]{f_*} & \pi_1(Y, f(x_0))
        \end{tikzcd}
    \end{equation}
\end{definition}

\begin{proof}
    \TODO{}
\end{proof}

\begin{theorem}[基本群はホモトピー不変量]
    $X, Y$を位相空間、$x_0 \in X$とし、$f \colon X \to Y$をホモトピー同値写像とする。
    このとき、$f$により誘導される準同型$f_*$は同型である。
    とくに、基本群はホモトピー不変量かつ位相不変量である。
\end{theorem}

\begin{proof}
    \TODO{}
\end{proof}

\begin{remark}[基本群の一致はホモトピー同値を意味しない]
    上の定理の逆は必ずしも成り立たない。
    すなわち、2つの空間が同じ基本群を持ったとしても、ホモトピー同値であるとは限らない
    \footnote{
        Whitehead の定理と呼ばれる定理は、特別な場合にこのような逆の成立を主張する。
    }
    。
\end{remark}

\subsection{単連結空間}

基本群を用いて定義される位相空間のクラスのうち
最も重要なもののひとつが単連結空間である。

\begin{definition}[単連結]
    位相空間$X$が
    \term{単連結}[simply connected]{単連結}[たんれんけつ]であるとは、
    $X$が弧状連結かつ
    $\pi_1(X)$が自明群であることをいう。
    \TODO{点付きでなくて良いか?}
\end{definition}

位相空間が単連結でないことを定義から直接示すのは難しいが、
単連結であることを示すのは比較的簡単な場合がある。
\TODO{本当に?}

\begin{example}[単連結な空間の例1]
    $\C$は単連結である。実際、$\C$は弧状連結であるし、
    $1$を基点とするループはすべて(線型ホモトピーによって)定値ループとホモトピックだから
    $\pi_1(X)$は自明である。
\end{example}

次の定理は
ホモロジーの Mayer-Vietoris の定理の
基本群における類似である。

\begin{theorem}[Mayer-Vietoris の類似]
    \label[theorem]{thm:fundamental-group-mayer-vietoris}
    $X$を位相空間とする。
    $X$単連結な開集合$U, V$で被覆され、
    共通部分$U \cap V$が弧状連結ならば、
    $X$は単連結である。
\end{theorem}

\begin{answer}
    \cref{problem:geometry2-4.7}を参照。
\end{answer}

\begin{corollary}[球面の基本群]
    $n \ge 2$ならば$S^n$は単連結である。
\end{corollary}

\begin{proof}
    $S^n$から南極を除いた集合と北極を除いた集合を考えて定理を適用すればよい。
\end{proof}

\subsection{直積空間の基本群}

\begin{theorem}[直積空間の基本群]
    \label[theorem]{thm:product-space-fundamental-group}
    $X, Y$を位相空間、$x_0 \in X, y_0 \in Y$とする。
    このとき、群の同型$\pi_1(X \times Y, (x_0, y_0)) \cong \pi_1(X, x_0) \times \pi_1(Y, y_0)$が成り立つ。
\end{theorem}

\begin{proof}
    \cref{problem:geometry2-4.8}を見よ。
\end{proof}

\begin{example}[トーラスの基本群]\label[example]{ex:torus-fundamental-group}
    $S^1$の基本群$\pi_1(S^1)$が$\Z$に同型であることを一時的に認めると、
    2次元トーラスの基本群は直積群$\Z \times \Z$に同型であることがわかる、
    実際、$T^2 = S^1 \times S^1$だから\cref{thm:product-space-fundamental-group}よりただちに従う。
    この証明は後の\cref{cor:torus-fundamental-group}で与える。
\end{example}









このあとのいくつかの例では、$S^1$が単連結でないという事実を一時的に認める。

\begin{example}[$\R^n$と凸部分集合]
    $X \subset \R^n$を凸部分集合とする。
    $x_0 \in X$を任意に固定する。
    このとき、ホモトピー$H(x, t) \coloneqq (1 - t) x + t x_0$は$A$から$\{x_0\}$への変形レトラクションである。
\end{example}

\begin{example}[$\R^2 \setminus \{0\}$と$S^1$]
    $X \coloneqq \R^2 \setminus \{0\}$とし、$A \coloneqq S^1$とする。
    ホモトピー$H \colon X \times I \to X,$
    \begin{equation}
        H(x, t) \coloneqq (1 - t) x + t \frac{x}{\|x\|}
    \end{equation}
    は$X$から$A$への変形レトラクションである。
    したがって$X$と$A$はホモトピー同値であり、それぞれの基本群は同型となるから、
    $A = S^1$が単連結でないことから$\R^2 \setminus \{0\}$も単連結でない。
\end{example}

\begin{example}[$\C$から2点を除いた空間]
    $X \coloneqq \C \setminus \{ \pm 1 \}$とおき、$C_\pm \coloneqq S^1 \pm 1$とおく。
    \term{8の字空間}[figure-eight space]{8の字空間} $A \coloneqq C_- \cup C_+$を考える。
    ホモトピー$H \colon X \times I \to X$を、
    第1象限の$z = x + iy \in X$に対し
    \begin{equation}
        H(x + iy, t) \coloneqq \begin{cases}
            x + i \left( (1-t) y + t \sqrt{1 - (x - 1)^2}\right) \quad (|z - 1| \ge 1, x \le 1) \\
            1 + (1-t)(z - 1) + t \frac{z - 1}{|z - 1|} \quad (\text{otherwise})
        \end{cases}
    \end{equation}
    と定める。他の象限の$z$に対しても同様に定める(\cref{fig:figure-eight})。
    このとき、$H$は$X$から$A$への変形レトラクションである。
    ここで、8の字空間$A$は単連結でない。
    \begin{innerproof}
        包含写像$C_+ \to A$を$\iota$とおき、
        $r \colon A \to C_+$を$r(x + iy) \coloneqq |x| + iy$で定める。
        このとき$r \circ \iota = 1_{C_+}$が成り立つから、
        誘導される準同型$(r \circ \iota)_* \colon \pi_1(C_+, 0) \to \pi_1(C_+, 0)$は
        恒等写像である。
        $(r \circ \iota)_* = r_* \circ \iota_*$であることにも注意すれば、
        $\iota_* \colon \pi_1(C_+, 0) \to \pi_1(A, 0)$は単射である。
        $\pi_1(C_+, 0) \cong \pi_1(S^1)$は非自明だから、$\pi_1(A, 0)$も非自明である。
        したがって$A$は単連結でない。
    \end{innerproof}
    よって$X = \C \setminus \{ \pm 1 \}$も単連結でないことがわかる。
\end{example}

\begin{figure}
    \centering
    \includegraphics[width=8cm]{\assetspath assets/figure-eight.png}
    \caption{8の字空間}
    \label[figure]{fig:figure-eight}
\end{figure}





% ------------------------------------------------------------
%
% ------------------------------------------------------------
\section{被覆空間}

\TODO{連結性に関する仮定があやしいので書き直したい}

位相空間の被覆空間とは、
大まかには位相空間を局所的な形を保ったまま覆うような位相空間のことである。

\TODO{ファイバー束の特別な場合であることを強調すべき?}

\begin{definition}[被覆空間]
    $E, X$を位相空間とする。
    連続写像$p \colon E \to X$が次を満たすとき、
    $E$は\term{被覆空間}[covering space]{被覆空間}[ひふくくうかん]であるという:
    \begin{enumerate}
        \item $E$は連結かつ局所弧状連結である\footnote{
            したがってとくに弧状連結でもある。
        }。
        \item 各$x \in X$は$p$により
            \term{自明に被覆される}[trivially covered]{自明に被覆される}[じめいにひふくされる]
            近傍をもつ。すなわち、
            各$x \in X$に対し、
            $x \in \exists U \stackrel{\text{open}}{\subset} X$と
            $E$の disjoint な開集合の族
            $\exists \{ V_\lambda \}_{\lambda \in \Lambda}$が存在して
            次が成り立つ:
            \begin{enumerate}
                \item $p^{-1}(U)$は
                    \begin{equation}
                        p^{-1}(U) = \bigcup_{\lambda \in \Lambda} V_\lambda
                    \end{equation}
                    をみたす。
                \item 各$\lambda$に対し
                    $p|_{V_\lambda}$は$U$の上への同相写像である。
            \end{enumerate}
    \end{enumerate}
    このとき、
    \begin{itemize}
        \item $X$を\term{底空間}[base space]{底空間}[ていくうかん]、
        \item $U$を$x$の\term{自明化近傍}{自明化近傍}[じめいかきんぼう]、
        \item $V_\lambda$らを$U$上の$p$の\term{シート}[sheets]{シート}、
        \item $p^{-1}(x)$を$x$上の$p$の\term{ファイバー}[fiber]{ファイバー}
    \end{itemize}
    という(\cref{fig:covering-space-1})。
\end{definition}

\begin{figure}[t]
    \centering
    \includegraphics[width=8cm]{\assetspath assets/covering-space-1.png}
    \caption{被覆空間}
    \label[figure]{fig:covering-space-1}
\end{figure}

\begin{example}[被覆空間の例]
    ~
    \begin{itemize}
        \item $X$を位相空間とする。恒等写像$1_X \colon X \to X$は
            $X$の自明な被覆空間である。
    \end{itemize}
\end{example}

\begin{example}[$S^1$の被覆空間]
    \label[example]{ex:covering-space}
    $\R$は$S^1$の被覆空間である(\cref{fig:s1-covering-space-1})。
    被覆空間$p \colon \R \to S^1$は
    $p(x) \coloneqq e^{2\pi i x}$とおけばよいことを確かめる。
    $\delta > 0$を十分小さく固定しておく。
    各$z = e^{i\theta_0} \in S^1, 0 \le \theta_0 < 2\pi$に対し、
    $U \coloneqq \{ e^{i\theta}
        \in S^1 \colon \theta_0 - \delta < \theta < \theta_0 + \delta \}$とおけば、
    $U$は$S^1$における$z$の開近傍である。$U$は
    \begin{equation}
        p^{-1}(U) = \bigcup_{n \in \Z}
            \left(\theta_0 - \delta + 2n\pi, \theta_0 + \delta + 2n\pi\right)
    \end{equation}
    をみたし、各項は$p$により$U$と同相である。
    したがって、$U$は$p$により自明に被覆されることがわかる。
\end{example}

\begin{figure}[t]
    \centering
    \includegraphics[width=8cm]{\assetspath assets/s1-covering-space-1.png}
    \caption{$S^1$の被覆空間}
    \label[figure]{fig:s1-covering-space-1}
\end{figure}

\begin{proposition}[ファイバーは離散空間]
    $p \colon E \to X$を被覆空間とする。
    このとき、各$x \in X$に対しファイバー$p^{-1}(x)$は離散空間である。
\end{proposition}

\begin{proof}
    各$e \in p^{-1}(x)$に対し$\{e\}$が$p^{-1}(x)$の開集合であることをいえばよい。
    そこで$x$の自明化近傍$U$をひとつ固定すると、
    $U$上のシート$V_\lambda$であって$e$の属するものがただひとつ存在する。
    $\{e\} = V_\lambda \cap p^{-1}(x)$を示せばよい。
    右向きの包含は明らかだから、逆向きを示す。
    そこで$e' \in V_\lambda \cap p^{-1}(x)$とすると、
    $p(e') = x = p(e)$であるが、
    $p$は$V_\lambda$から$U$への単射だから$e' = e$、
    したがって$e' \in \{e\}$である。
    よって逆向きの包含もいえた。
\end{proof}

被覆空間は全射連続開写像である。

\begin{proposition}[被覆空間は全射連続開写像]
    \TODO{}
\end{proposition}

\begin{proof}
    \TODO{}
\end{proof}

\subsection{リフト}

リフトについて述べる。

\TODO{多価関数の枝についての例を念頭に考えたい}

\begin{definition}[リフト]
    $p \colon E \to X$を被覆空間、
    $f \colon Y \to X$を連続写像とする。
    連続写像$\wt{f} \colon Y \to E$が
    $p$による$f$の
    \term{リフト}[lift]{リフト}
    あるいは\term{持ち上げ}{持ち上げ}[もちあげ]
    であるとは、図式
    \begin{equation}
        \begin{tikzcd}
            & E \ar{d}{p} \\
            Y \ar[dashed]{ur}{\wt{f}} \ar{r}[swap]{f} & X
        \end{tikzcd}
    \end{equation}
    が可換となることをいう(\cref{fig:lift-1})。
\end{definition}

\begin{figure}[t]
    \centering
    \includegraphics[width=8cm]{\assetspath assets/lift-1.png}
    \caption{リフト}
    \label[figure]{fig:lift-1}
\end{figure}

\begin{example}[リフトが存在しない例]
    \cref{ex:covering-space}の被覆空間$p \colon \R \to S^1$を考える。
    $f \colon S^1 \to S^1$を恒等写像とすると、
    $p$による$f$のリフト$\wt{f}$は存在しない。
    \begin{equation}
        \begin{tikzcd}
            & \R
                \ar{d}{p} \\
            S^1
                \ar[dashed]{ur}{\wt{f}}
                \ar{r}{f}
                & S^1
        \end{tikzcd}
    \end{equation}
    実際、もしこのようなリフト$\wt{f}$が存在するとしたら、
    $p \circ \wt{f}$から誘導される群準同型
    $(p \circ \wt{f})_* \colon \pi_1(S^1) \to \pi_1(S^1)$
    は恒等写像となるから、$p_* \colon \pi_1(\R) \to \pi_1(S^1)$は全射である。
    ところが、($S^1$が単連結でないことを認めると)
    $\pi_1(S^1)$は非自明で$\pi_1(\R)$は自明だからこれは矛盾である。
    したがって上のようなリフト$\wt{f}$は存在しない。
\end{example}

\begin{example}[リフトが一意でない例]
    $U \subset \C^\times$を単連結領域とする。
    このとき、指数関数$\exp \colon \C \to \C^\times,\;
    z \mapsto \exp(z)$は被覆空間である。
    また、各$n_0 \in \Z$に対し
    写像$\log_{n_0} \colon U \to \C,\;
    z \mapsto \log |z| + i (\mathrm{Arg} (z) + 2n_0\pi)$
    は$\exp$による$f \colon U \to \C^\times,\; z \mapsto z$のリフトである。
    $n_0$をどのようにとっても$\log_{n_0}$は$\exp$のリフトであるが、
    $\log_{n_0}$は$n_0$ごとに異なるから、
    リフトは一意的でないことが確かめられた。
    \begin{equation}
        \begin{tikzcd}
            & \C \ar{d}{\exp} \\
            U \ar{ur}{\log_{n_0}} \ar{r}[swap]{\id_{U}} & C^\times
        \end{tikzcd}
    \end{equation}
\end{example}

連結空間 (たとえば$I$) からの写像がリフトを持つとき、
上の例で見たようにそれは一意とは限らないが、
1点での値を決めれば一意に定まる。
証明の手法は連結性を用いる典型的なものである。

\begin{theorem}[リフトの一意性定理]
    $p \colon E \to X$を被覆空間、
    $Y$を\highlight{連結}位相空間、
    $\varphi \colon Y \to X$を連続写像とする。
    $\wt{\varphi}_1, \wt{\varphi}_2 \colon Y \to E$を
    $p$による$\varphi$のリフトとするとき、
    $\wt{\varphi}_1, \wt{\varphi}_2$が
    1点で一致するならば全体で一致する。
\end{theorem}

\begin{proof}
    $B \coloneqq \{ x \in Y \colon \wt{\varphi}_1(x) = \wt{\varphi}_2(x) \}$
    とおく。問題の仮定より$B$は空でない。
    $Y$は連結だから、$B$が$Y$で開かつ閉であることを示せば$B = Y$となり定理の主張が従う。
    \begin{equation}
        \begin{tikzcd}[row sep=huge, column sep=huge]
            & E
                \ar{d}{p} \\
            Y
                \ar[shift left]{ur}{\wt{\varphi}_1}
                \ar[shift right]{ur}{\wt{\varphi}_2}
                \ar{r}{\varphi}
                & X
        \end{tikzcd}
    \end{equation}

    \noindent
    \uline{$B$が$Y$で開であること} \quad
    $b_0 \in B$とする。
    \begin{equation}
        e \coloneqq \wt{\varphi}_1(b_0) = \wt{\varphi}_2(b_0),
        \quad
        x_0 \coloneqq \varphi(b_0) = p(e)
    \end{equation}
    とおく。
    $x_0$の自明化近傍$U \subset X$と、
    $U$上のシート$\wt{U}$であって$e$を含むものが存在する。
    ここで
    $V \coloneqq \wt{\varphi}_1^{-1}(\wt{U})
        \cap \wt{\varphi}_2^{-1}(\wt{U})$
    とおくと、
    $V$は$Y$における$b_0$の開近傍であり、
    $\wt{\varphi}_1, \wt{\varphi}_2$は$V$上で$\wt{U}$に値をとる。
    したがって
    \begin{itemize}
        \item $\varphi$は$V \to U$の連続写像とみなすことができ、
        \item $\wt{\varphi}_1, \wt{\varphi}_2$は
            $V \to \wt{U}$の連続写像とみなすことができ、
        \item $p$は$\wt{U} \to U$の同相写像とみなすことができる。
    \end{itemize}
    よって、図式
    \begin{equation}
        \begin{tikzcd}[row sep=huge, column sep=huge]
            & \wt{U} \ar{d}{p}[swap, anchor=center, rotate=90, yshift=1ex]{\approx} \\
            V \ar[shift left]{ur}{\wt{\varphi}_1}
                \ar[shift right]{ur}[swap]{\wt{\varphi}_2}
                \ar{r}{\varphi}
                & U
        \end{tikzcd}
    \end{equation}
    は可換となる。$p$の$\wt{U}$上での単射性から
    $V$上$\wt{\varphi}_1 = \wt{\varphi}_2$となることがわかる。
    したがって$b_0 \in V \subset B$であり、$b_0$は$Y$における$B$の内点であることがいえた。
    $b_0 \in B$は任意であったから、$B$は$Y$で開である。

    \noindent
    \uline{$B$が$Y$で閉であること} \quad
    $Y \setminus B$が$Y$で開であることを示す。
    $b_0 \in Y \setminus B$とする。
    $e_1 \coloneqq \wt{\varphi}_1(b_0),\;
    e_2 \coloneqq \wt{\varphi}_2(b_0)$とおくと、
    $b_0 \not\in B$より$e_1 \neq e_2$である。
    $x_0 \coloneqq p(e_1) = p(e_2) = \varphi(b_0)$とおくと、
    $x_0$の自明化近傍$U \subset X$と、
    $U$上のシート$\wt{U}_1, \wt{U}_2$であって
    $e_1, e_2$をそれぞれ含むものが存在する。
    このとき$\wt{U}_1 \cap \wt{U}_2 = \emptyset$である
    \begin{innerproof}
        もし交わりをもったとすれば、シートの disjoint 性より
        $\wt{U}_1 = \wt{U}_2$でなければならないが、
        すると$p$が$\wt{U}_1 = \wt{U}_2$から$U$への全単射となることから
        $e_1 = (p|_{\wt{U}_1})^{-1}(x_0)
        = (p|_{\wt{U}_2})^{-1}(x_0) = e_2$となり矛盾。
    \end{innerproof}
    ここで$V \coloneqq \wt{\varphi}_1^{-1}(\wt{U}_1)
    \cap \wt{\varphi}_2^{-1}(\wt{U}_2)$とおくと、
    $V$は$Y$における$b_0$の近傍であり、
    $\wt{\varphi}_1(V) \subset \wt{U}_1,\;
    \wt{\varphi}_2(V) \subset \wt{U}_2$をみたす。
    よって$V$上$\wt{\varphi}_1 \neq \wt{\varphi}_2$である。
    したがって$V \subset Y \setminus B$であり、
    $b_0$は$Y$における$Y \setminus B$の内点であることがいえた。
    $b_0 \in Y \setminus B$は任意であったから、
    $Y \setminus B$は$Y$で開である。
    よって$B$は$Y$で閉である。
\end{proof}

リフトの存在について調べる。
まずは一般の写像のリフトではなく
パスのリフトに限って考えよう。
パスのリフトの存在に関する定理はすべて次の定理から導かれる。

\begin{theorem}[被覆ホモトピー定理]
    $p \colon E \to X$を被覆空間、
    $Y$を位相空間とする。
    このとき、図式
    \begin{equation}
        \begin{tikzcd}
            Y
                \ar{r}{\wt{f}}
                \ar{d}[swap]{y \mapsto (y, 0)}
                & E
                    \ar{d}{p} \\
            Y \times I
                \ar[dashed]{ru}{\wt{F}}
                \ar{r}[swap]{F}
                & X
        \end{tikzcd}
    \end{equation}
    を可換にする$\wt{F}$が一意に存在する。
\end{theorem}

\begin{proof}
    \TODO{}
\end{proof}

\begin{corollary}[パスのリフトの一意存在定理]
    $p \colon E \to X$を被覆空間、
    $\gamma \colon I \to X$をパス、
    $e \in p^{-1}(\gamma(0))$とする。
    このとき、$p$による$\gamma$のリフト
    $\wt{\gamma} \colon I \to E$であって
    $\wt{\gamma}(0) = e$を満たすものが一意に存在する。
\end{corollary}

\begin{proof}
    \TODO{}
\end{proof}

\begin{corollary}[モノドロミー定理]
    \termhidden[monodoromy theorem]{モノドロミー定理}[ものどろみーていり]
    $p \colon E \to X$を被覆空間、
    $f, g \colon I \to X$を
    点$x_0$から$x_1$へのパス、
    $\wt{x}_0 \in p^{-1}(x_0)$とする。
    \begin{enumerate}
        \item (パスのホモトピーのリフトの一意存在) \quad
            $H \colon f \sim g \; \rel \; \{ 0, 1 \}$ならば、
            $H$のリフト$\wt{H} \colon I \times I \to E$であって
            $\wt{H}(0, 0) = \wt{x}_0$を満たすものが一意に存在する。
        \item (モノドロミー定理) さらに
            $\wt{x}_0$を始点とする
            $f, g$のリフトをそれぞれ
            $\wt{f}, \wt{g}$とおくと、
            $\wt{f}$と$\wt{g}$は終点も一致し、
            さらに
            $\wt{H} \colon f \sim g \; \rel \; \{ 0, 1 \}$
            が成り立つ。
    \end{enumerate}
\end{corollary}

\begin{proof}
    \TODO{}
\end{proof}

モノドロミー定理を用いて
$S^1$は単連結でないことを示そう。

\begin{theorem}
    $S^1$は単連結でない。
\end{theorem}

\begin{proof}
    写像$p \colon \R \to S^1, \;
        t \mapsto \exp(2\pi i t)$
    は$S^1$の被覆空間である。
    ここで写像$\gamma, \wt{\gamma}$を
    \begin{alignat}{1}
        \gamma \colon I \to S^1, \quad & t \mapsto \exp(2\pi i t) \\
        \wt{\gamma} \colon I \to \R, \quad & t \mapsto t
    \end{alignat}
    で定めると、$\wt{\gamma}$は$\gamma$のリフトである。
    また、写像$\beta$を
    \begin{alignat}{1}
        \beta \colon I \to S^1, \quad & t \mapsto 1 \\
        \wt{\beta} \colon I \to \R, \quad & t \mapsto 0
    \end{alignat}
    で定めると、$\wt{\beta}$は$\beta$のリフトである。
    ここで$S^1$が単連結であると仮定すると、
    $\beta$と$\gamma$が共通の端点を持つことから
    \begin{equation}
        H \colon \beta \sim \gamma \quad \rel \quad \{ 0, 1 \}
    \end{equation}
    が成り立つ。
    $\wt{\beta}, \wt{\gamma}$は共通の始点を持つから
    モノドロミー定理より終点も共通である。
    ところが定義より$\wt{\beta}(1) = 0 \neq 1 = \wt{\gamma}(1)$だから矛盾。
    したがって$S^1$は単連結でない。
\end{proof}

次に写像のリフトの存在について考える。
一般の写像の定義域はパスの定義域$I$のように良い性質を持つとは限らない。
そこで定義域の連結性を限定したクラスのなかで存在条件を考えることになる。

\begin{theorem}[リフトの存在条件]
    $p \colon E \to X$を被覆空間、
    $Y$を\highlight{連結かつ局所弧状連結}な位相空間、
    $f \colon (Y, y_0) \to (X, x_0)$を連続写像、
    $\wt{x}_0 \in p^{-1}(x_0)$とする。
    このとき、次は同値である:
    \begin{enumerate}
        \item $f$のリフト$\wt{f} \colon (Y, y_0) \to (E, \wt{x}_0)$が一意に存在する。
        \item $f_* \pi_1(Y, y_0) \subset p_* \pi_1(E, \wt{x}_0)$である。
    \end{enumerate}
\end{theorem}

\begin{proof}
    $Y$は連結なので一意性は成り立つ。

    \TODO{}
\end{proof}



\subsection{ファイバーへの基本群の作用}

位相空間の基本群は、被覆空間のファイバーへの自然な群作用をもつ。

\TODO{}

%\begin{theorem}[モノドロミー作用の well-defined 性]
%    \label[theorem]{thm:monodromy-well-defined}
%    $p \colon E \to X$を被覆空間とし、$x \in X$とする。
%    このとき、基本群$\pi_1(X, x)$の$p^{-1}(x)$への右作用が
%    \begin{equation}
%        e \cdot [f] \coloneqq \tilde{f}_e(1) \quad (\forall e \in p^{-1}(x), [f] \in \pi_1(X, x))
%    \end{equation}
%    により定まる。
%    この作用を\term{モノドロミー作用}[monodromy action]{モノドロミー作用}という
%    (\cref{fig:monodromy-1})。
%\end{theorem}
%
%\begin{figure}[t]
%    \centering
%    \includegraphics[width=8cm]{\assetspath assets/monodromy-1.png}
%    \caption{モノドロミー作用}
%    \label[figure]{fig:monodromy-1}
%\end{figure}
%
%\begin{proof}
%    省略
%\end{proof}
%
%\begin{remark}[モノドロミー作用の不動点]
%    すぐわかるように、$f$を$X$内のパスとするとき、
%    $e \cdot [f] = e$であることと
%    $\tilde{f}_e$が$e$を基点とするループであることとは同値である。
%\end{remark}
%
%\begin{theorem}[弧状連結な被覆空間のモノドロミー作用]
%    \label[theorem]{thm:monodromy-path-connected}
%    \cref{thm:monodromy-well-defined}の状況でさらに$E$が弧状連結ならば、モノドロミー作用は推移的である。
%
%    \TODO{被覆空間は定義より弧状連結なのでは?}
%\end{theorem}
%
%\begin{proof}
%    省略
%\end{proof}
%
%\begin{theorem}[モノドロミー作用の固定部分群]
%    \label[theorem]{thm:monodromy-stabilizer}
%    $p \colon E \to X$を被覆空間、$x \in X$とし、$E$は弧状連結であるとする。
%    このとき、各$e \in p^{-1}(x)$に対し、
%    モノドロミー作用に関する$e$の固定部分群は$p_* (\pi_1(E, e)) \subseteq \pi_1(X, x)$で与えられる。
%\end{theorem}
%
%\begin{proof}
%    省略
%\end{proof}
%
%\TODO{普遍被覆を導入したあとで述べるべき?}
%
%\begin{corollary}[単連結な被覆空間のモノドロミー作用]
%    \label[theorem]{thm:monodromy-simply-connected}
%    $p \colon E \to X$を被覆空間とし、$E$は単連結であるとする。
%    このとき、各ファイバーへのモノドロミー作用は自由である。
%\end{corollary}
%
%\begin{proof}
%    $x \in X$を任意に固定する。
%    $\pi_1(X, x)$の$p^{-1}(x)$へのモノドロミー作用が自由であることは、
%    作用に関する固定部分群がすべて自明であることと同値であり、
%    さらに\cref{thm:monodromy-stabilizer}より、これはすべての$e \in p^{-1}(x)$に対し
%    $p_* (\pi_1(E, e))$が自明であることと同値である。
%    そこで、各$e \in p^{-1}(x)$に対し$p_* (\pi_1(E, e))$が自明であることを示す。
%    系の仮定より$E$は単連結だから、$\pi_1(E, e)$は自明である。
%    したがって$p_* (\pi_1(E, e))$も自明である。
%\end{proof}

\subsection{普遍被覆}

単連結な被覆空間を普遍被覆という。

\begin{definition}[普遍被覆]
    被覆空間$p \colon E \to X$が単連結であるとき、
    $p$を\term{普遍被覆}[universal covering]{普遍被覆}[ふへんひふく]という。
\end{definition}

普遍被覆は存在すれば次の定理の意味で一意である。
したがって$X$が単連結ならば
$X$自身が (自明な被覆として) 普遍被覆となる。

\begin{theorem}[普遍被覆の一意性]
    普遍被覆は被覆空間としての同型を除いて一意である。
    \TODO{}
\end{theorem}

\begin{proof}
    \TODO{}
\end{proof}

普遍被覆はつねに存在するとは限らない。
存在条件は次の定理で与えられる。

\begin{theorem}[普遍被覆の存在定理]
    半局所単連結であること
    \TODO{}
\end{theorem}

\begin{proof}
    \TODO{}
\end{proof}

\begin{proposition}[普遍被覆のファイバーと基本群]
    $X$を位相空間、$p \colon \wt{X} \to X$を普遍被覆とし、
    $p$は$\wt{x}_0 \in \wt{X}$を$x_0 \in X$に写すとする。
    このとき、写像
    \begin{equation}
        \pi_1(X, x_0) \to p^{-1}(x_0),
        \quad
        [\gamma] \mapsto \wt{\gamma}(1)
    \end{equation}
    は全単射である。ただし、$\wt{\gamma}$は$\wt{x}_0$を始点とする
    $\gamma$のリフトである。
    また、逆写像は$y \in p^{-1}(x_0)$に対し
    $\wt{x}_0$を$y$につなぐパス$\beta$をひとつ選んで
    $[p \circ \beta]$を対応付けることで得られる。
\end{proposition}

\begin{proof}
    \TODO{}
\end{proof}

\subsection{被覆変換群}

被覆空間の射と被覆変換群について述べる。

\begin{definition}[被覆空間の射]
    $p \colon E \to X, \; p' \colon E' \to X$を被覆空間とする。
    連続写像$f \colon E \to E'$であって
    \begin{equation}
        \begin{tikzcd}
            E \ar{rr}{f} \ar{rd}[swap]{p}
                && E' \ar{ld}{p'} \\
            & X
        \end{tikzcd}
    \end{equation}
    を可換にするものを
    \term{被覆空間の射}[morphism of covering spaces]
        {被覆空間の射}[ひふくくうかんのしゃ]
    という。
\end{definition}

\begin{definition}[被覆変換群]
    \idxsym{covering transformation group}{$\Deck(E)$}{被覆空間$E$の被覆変換群}
    被覆空間の自己同型射を
    \term{被覆変換}[covering transformation]
        {被覆変換}[ひふくへんかん]
    という。
    被覆空間$E$の被覆変換全体のなす群を
    $\Deck(E)$と書き、
    $E$の\term{被覆変換群}[covering transformation group]
        {被覆変換群}[ひふくへんかんぐん]
    という。
\end{definition}

普遍被覆の被覆変換群は基本群の計算に利用できる。

\begin{theorem}[普遍被覆の被覆変換群と基本群]
    $X$を局所弧状連結な位相空間、
    $p \colon \wt{X} \to X$を普遍被覆とする。
    このとき、任意の$x_0 \in X$に対し群の同型
    \begin{equation}
        \pi_1(X, x_0) \cong \Deck(\wt{X})
    \end{equation}
    が成り立つ。
\end{theorem}

\begin{proof}
    \TODO{}
\end{proof}


% ------------------------------------------------------------
%
% ------------------------------------------------------------
\section{
    \texorpdfstring{%
        $S^1$の基本群%
    }{%
        S1の基本群%
    }%
}

$S^1$の基本群を計算する。
$S^1$の基本群は最も身近にある非自明な例というばかりでなく、
ホモロジーの理論の基礎にもつながっている。
\TODO{どのように?}

\begin{lemma}[$S^1$のループのリフト]
    \label[lemma]{lem:s1-loop-lift}
    \cref{ex:covering-space}の被覆空間$p \colon \R \to S^1$を考える。
    ここだけの用語として、
    各$k \in \Z$に対し、$S^1$上を点$1$から反時計回りに$k$周するループ
    \begin{equation}
        \omega_k \colon I \to S^1, \quad t \mapsto e^{2\pi kit}
    \end{equation}
    を\emph{$k$-基本ループ ($k$-elementary loop)}と呼ぶことにする。
    \TODO{必要ある?}
    このとき、$0 \in \R$を始点とする$\omega_k$のリフト
    $(\tilde{\omega}_k)_0$は
    \begin{equation}
        (\tilde{\omega}_k)_0 \colon I \to \R, \quad t \mapsto kt
    \end{equation}
    で与えられる。
\end{lemma}

\begin{proof}
    補題の$(\tilde{\omega}_k)_0$は
    \begin{equation}
        \begin{tikzcd}
            & \R
                \ar{d}{p} \\
            I
                \ar{ur}{(\tilde{\omega}_k)_0}
                \ar{r}{\omega_k}
                & S^1
        \end{tikzcd}
    \end{equation}
    を可換にする連続写像だから$\omega_k$のリフトであり、
    また明らかに始点は$0$である。
\end{proof}

\begin{theorem}[$S^1$の基本群]
    $\pi_1(S^1) \cong \Z$である(\cref{fig:s1-monodromy-1})。
\end{theorem}

\begin{figure}[t]
    \centering
    \includegraphics[width=8cm]{\assetspath assets/s1-monodromy-1.png}
    \caption{$S^1$の基本群}
    \label[figure]{fig:s1-monodromy-1}
\end{figure}

\begin{proof}
    \cref{ex:covering-space}の被覆空間$p \colon \R \to S^1$を考える。
    $\R$は単連結だから、\cref{thm:monodromy-path-connected}および\cref{thm:monodromy-simply-connected}より
    $\pi_1(S^1, 1)$の$p^{-1}(1)$へのモノドロミー作用は推移的かつ自由である。
    したがって、写像$\Phi \colon \pi_1(S^1, 1) \to p^{-1}(1),$
    \begin{equation}
        [f] \mapsto 0 \cdot [f]
    \end{equation}
    は全単射である。
    \begin{innerproof}
        \uline{全射性}\; 作用が推移的であることの定義から明らか。\\
        \uline{単射性}\; $[f], [g] \in \pi_1(S^1, 1)$とする。
            \begin{alignat}{1}
                \Phi([f]) = \Phi([g]) 
                    &\Rightarrow 0 \cdot [f] = 0 \cdot [g] \\
                    &\Rightarrow 0 \cdot ([f] \cdot [g]^{-1}) = 0 \\
                    &\Rightarrow [f] \cdot [g]^{-1} = 1 \quad (\because \text{ 作用は自由}) \\
                    &\Rightarrow [f] = [g]
            \end{alignat}
            より、単射性がいえた。
    \end{innerproof}

    あとは$\Phi$が群準同型であることをいえばよい。
    そこで、まず$\pi_1(S^1, 1)$の具体的な形を考える。
    いま$p$の定め方から$p^{-1}(1) = \Z$であったから、
    $\pi_1(S^1, 1) = \Phi^{-1}(\Z)$である。
    $\Phi([\omega_k]) = 0 \cdot [\omega_k] = (\tilde{\omega}_k)_0(1) = k$ゆえに
    $\Phi^{-1}(k) = [\omega_k]$だから、
    \begin{equation}
        \pi_1(S^1, 1) = \Phi^{-1}(\Z) = \{ [\omega_k] \colon k \in \Z \}
    \end{equation}
    と表せることがわかる。

    $\Phi$が群準同型であることを示す。
    \begin{alignat}{1}
        \Phi([\omega_k] \cdot [\omega_j])
            &= 0 \cdot ([\omega_k] \cdot [\omega_j]) \\
            &= 0 \cdot [\omega_k * \omega_j] \\
            &= 0 \cdot [\omega_{k+j}] \\
            &= (\tilde{\omega}_{k+j})_0(1) \\
            &= k+j \quad (\cref{lem:s1-loop-lift}) \\
            &= (\tilde{\omega}_{k})_0(1) + (\tilde{\omega}_{j})_0(1) \quad (\cref{lem:s1-loop-lift}) \\
            &= 0 \cdot [\omega_{k}] + 0 \cdot [\omega_{j}] \\
            &= \Phi([\omega_k]) + \Phi([\omega_j])
    \end{alignat}
    だから、$\Phi$は$\Z$と$\pi_1(S^1, 1)$との群準同型であることがいえた。
    以上で$\pi_1(S^1) = \pi_1(S^1, 1) \cong \Z$がいえた。
\end{proof}

\begin{corollary}[トーラスの基本群]
    \label[corollary]{cor:torus-fundamental-group}
    $\pi_1(T^2) \cong \Z \times \Z$である。 
\end{corollary}

\begin{proof}
    \cref{ex:torus-fundamental-group}ですでに確かめた。
    また、\cref{problem:geometry2-4.2}で被覆変換群を用いる別解を与えた。
\end{proof}




% ------------------------------------------------------------
%
% ------------------------------------------------------------
\section{Seifert-Van Kampen の定理 \!${}^*$}

Seifert-Van Kampen の定理は、一定の条件下において、
位相空間の基本群が部分空間の基本群の融合積に分解できることを述べた定理である。
定理の系として、
wedge 和やCW複体の基本群が簡単に計算できるようになる。
ここでは wedge 和の例を計算する。


\begin{theorem}[Seifert-Van Kampen の定理]
    $X$を位相空間とする。$U, V \subset X$は開部分集合であって、
    $U \cup V = X$をみたし、$U, V, U \cap V$は弧状連結であるとする。
    $p \in U \cap V$とし、部分集合$C \subset \pi_1(U, p) * \pi_1(V, p)$を
    \begin{equation}
        C \coloneqq \{ (i_* \gamma)(j_* \gamma)^{-1} \colon \gamma \in \pi_1(U \cap V, p) \}
    \end{equation}
    で定義する。
    \TODO{}
\end{theorem}

\begin{proof}
    省略
\end{proof}

\begin{example}[円のブーケ]
    \TODO{}
\end{example}



% ------------------------------------------------------------
%
% ------------------------------------------------------------
\newpage
\section{演習問題}

\subsection{問題セット 2}

\begin{problem}[幾何学II 2.1]
    $\R^2 \setminus ((-\infty, 0] \times \{0\})$は可縮であることを示せ。
\end{problem}

\begin{answer}
    $X \coloneqq \R^2 \setminus ((-\infty, 0] \times \{0\})$とおく。
    $\id_X$が定値写像$c \colon X \to X, x \mapsto (1, 0)$に
    ホモトピックであることを示せばよいが、
    $X$は点$(1, 0)$に関し星型だから
    \begin{equation}
        H \colon X \times I \to X,
        \quad
        ((x, y), t) \mapsto ((1 - t) x + t, (1 - t) y)
    \end{equation}
    が求めるホモトピーを与える。
\end{answer}

\begin{problem}[幾何学II 2.2]
    球面$S^n$の北極を$N = (0, \dots, 0, 1)$とおくとき、
    集合$S^n \setminus \{N\}$は可縮であることを示せ。
\end{problem}

\begin{answer}
    $X \coloneqq S^n \setminus \{N\}$とおく。
    可縮性は位相不変だから
    $X$が可縮空間$\R^n$と同相であることをいえばよいが、
    立体射影$X \to \R^n$は同相写像だから$X \approx \R^n$である。
\end{answer}

\begin{problem}[幾何学II 2.3]
    次の集合は互いにホモトピー同値であることを示せ。
    \begin{enumerate}
        \item $S^1 \times S^1 \setminus \{ (-1, -1) \}$
        \item $(S^1 \times \{1\}) \cup (\{1\} \times S^1)$
    \end{enumerate}
\end{problem}

\begin{answer}
    \TODO{J. H. C. Whitehead を使うべし}
    %(1)は$S^1 \times S^1 \setminus \{(0, 0)\}$と同相だから、
    %(1)の代わりにこの集合 ((1)' とおく) を考えればよい。
    %まず$I \coloneqq [-1, 1]$とおき、
    %$I^2$の同値関係$\sim$を
    %\begin{equation}
    %    (a, b) \sim (a', b')
    %        \logeq ((a, b) = (a', b'))
    %            \vee (a = a' \wedge |b - b'| = 2)
    %            \vee (b = b' \wedge |a - a'| = 2)
    %\end{equation}
    %で定め、これにより定まる商写像$\colon I^2 \to I^2/\sim$を$\pi$とおく。
    %このとき、連続写像
    %\begin{equation}
    %    I^2 \to S^1 \times S^1, \quad
    %    (a, b) \mapsto (e^{\pi i a}, e^{\pi i b})
    %\end{equation}
    %により誘導される連続写像$I^2/\sim \to S^1 \times S^1$は同相写像となる。
    %そこで
    %\begin{alignat}{1}
    %    X &\coloneqq I^2 \setminus \{ (0, 0) \}, \\
    %    B &\coloneqq (I \times \{ -1, 1 \}) \cup (\{ -1, 1 \} \times I)
    %\end{alignat}
    %とおくと、上の同相により(1)', (2)はそれぞれ
    %$X/\sim, B/\sim$に対応する。
    %そこで、まず$X$と$B$のホモトピー同値を与える写像を構成し、
    %それらを商空間上に誘導することで
    %$X/\sim$と$B/\sim$の (したがって(1)'と(2)の) ホモトピー同値を示すことにする。
    %最初に、写像$r \colon X \to B$を次のように定める:
    %\begin{padding}
    %    各$x \in X$に対し、$(0, 0)$から$x$への半直線と
    %    $B$ (を$X$の部分空間とみなしたもの) との交点を$r(x)$と定める。
    %\end{padding}
    %これは明らかに連続写像である。
    %ここで包含写像$B \to X$を$\iota$とおくと、
    %$\iota \circ r$と$1_X$を結ぶホモトピー$H$を
    %\begin{equation}
    %    X \times I \to X, \quad
    %    (x, t) \mapsto (1 - t) \iota \circ r(x) + t 1_X(x)
    %\end{equation}
    %で定義できる。
    %また、$r$の定義から明らかに$r \circ \iota = 1_B$である。
    %したがって
    %\begin{equation}
    %    \begin{cases}
    %        \iota \circ r \simeq 1_X \\
    %        r \circ \iota \simeq 1_B
    %    \end{cases}
    %\end{equation}
    %が成り立ち、$X, B$はホモトピー同値である。
    %次に、これらの写像を商空間上に誘導することを考える。
    %$I^2 \times I$上の同値関係$\sim'$を
    %\begin{equation}
    %    ((a, b), t) \sim' ((a', b'), t')
    %        \logeq \begin{cases}
    %            (a, b) \sim (a', b') \\
    %            t = t'
    %        \end{cases}
    %\end{equation}
    %で定め、これにより定まる商写像$\colon I^2 \times I \to (I^2 \times I)/\sim'$を
    %$\pi'$とおく。このとき、商位相空間の普遍性から図式
    %\begin{equation}
    %    \begin{tikzcd}
    %        & I^2 \times I
    %            \ar{ld}[swap]{\pi'}
    %            \ar{rd}{\pi \times \id} \\
    %        (I^2 \times I)/\sim' \ar[dashed]{rr}[swap]{\varphi}
    %            && I^2/\sim \times I
    %    \end{tikzcd}
    %\end{equation}
    %を可換にする連続写像$\varphi$が誘導される。
    %すぐわかるように$\varphi$は全単射であり、
    %さらに$(I^2 \times I)/\sim'$のコンパクト性と$I^2 \times I$の Hausdorff 性から
    %$\varphi$は同相写像となる。
    %よって、図式
    %\begin{equation}
    %    \begin{tikzcd}
    %        X/\sim \times I
    %            \ar[dashed]{r}{\wt{H}}
    %            & X/\sim \\
    %        X \times I
    %            \ar{r}{H} \ar{u}{\pi \times \id}
    %            & X \ar{u}[swap]{\pi}
    %    \end{tikzcd}
    %\end{equation}
    %を可換にする連続写像$\wt{H}$が誘導される
    %(より詳しくいえば、$H$から$(X \times I)/\sim'$上に誘導された写像と$\varphi$を合成したものが
    %$\wt{H}$である)。
    %ここで、$\iota, r$によりそれぞれ$B/\sim, X/\sim$上に誘導される連続写像を
    %それぞれ$\wt{\iota}, \wt{r}$とおくと、
    %$\wt{H}$は$\wt{\iota} \circ \wt{r}$と
    %$1_{X/\sim}$を結ぶホモトピーである。
    %また、$r \circ \iota = 1_B$であることから明らかに
    %$\wt{r} \circ \wt{\iota} = 1_{B/\sim}$が成り立つ。
    %したがって
    %\begin{equation}
    %    \begin{cases}
    %        \wt{\iota} \circ \wt{r} \simeq 1_{X/\sim} \\
    %        \wt{r} \circ \wt{\iota} \simeq 1_{B/\sim}
    %    \end{cases}
    %\end{equation}
    %が成り立ち、$X/\sim, B/\sim$はホモトピー同値である。
    %よって(1)', (2)はホモトピー同値であり、題意の主張が示せた。
\end{answer}

\begin{problem}[幾何学II 2.4]
    次の集合が互いにホモトピー同値かどうか調べよ。
    \begin{enumerate}
        \item $\{ (x, y) \in \R^2 \colon x^2 + y^2 = 1 \}
            \cup ([-1, 1] \times \{ 0 \})$
        \item $\{ (x, y) \in \R^2 \colon (x + 1)^2 + y^2 = 1 \}
            \cup \{ (x, y) \in \R^2 \colon (x - 1)^2 + y^2 = 1 \}$
    \end{enumerate}
\end{problem}

\begin{answer}
    (1), (2)はホモトピー同値である。\TODO{}
\end{answer}

\begin{problem}[幾何学II 2.5]
    \label[problem]{problem:geometry2-2.5}
    位相空間$X, Y$に対して、
    $X$から$Y$への連続写像は$\pi_0(X)$から$\pi_0(Y)$への
    写像を誘導することを示せ。
    また、$X$から$Y$への互いにホモトピックな連続写像が誘導する写像は一致することを示せ。
\end{problem}

\begin{answer}
    まず図式
    \begin{equation}
        \begin{tikzcd}
            X \ar{r}{f} \ar{d}[swap]{\pi_0}
                & Y \ar{d}{\pi_0} \\
            \pi_0(X) \ar[dashed]{r}[swap]{\pi_0(f)}
                & \pi_0(Y)
        \end{tikzcd}
    \end{equation}
    を可換にする写像$\pi_0(f)$を構成する。そこで
    \begin{equation}
        \pi_0(f)(\pi_0(x)) \coloneqq \pi_0(f(x))
    \end{equation}
    と定義する。これは well-defined である。実際、
    $\pi_0(x) = \pi_0(x')$ならば
    $x, x'$をつなぐ$X$内のパス$\gamma \colon I \to X$が存在するから、
    合成$f \circ \gamma$が$f(x), f(x')$をつなぐ$Y$内のパスとなり、
    したがって$\pi_0(f(x)) = \pi_0(f(x'))$が成り立つ。

    次に連続写像$f, g \colon X \to Y$がホモトピックであるとし、
    $f$と$g$をつなぐホモトピーを$H \colon X \times I \to Y$とする。
    各$x \in X$に対し
    \begin{equation}
        H(x, \cdot) \colon I \to X, \quad
        t \mapsto H(x, t)
    \end{equation}
    は$f(x), g(x)$をつなぐ$Y$内のパスであるから、
    \begin{alignat}{2}
        && \pi_0(f(x)) &= \pi_0(g(x)) \\
        \therefore \quad && \pi_0(f)(\pi_0(x)) &= \pi_0(g)(\pi_0(x))
    \end{alignat}
    が成り立つ。したがって$\pi_0(f) = \pi_0(g)$である。
\end{answer}

\begin{problem}[幾何学II 2.6]
    位相空間$X$が連結かつ局所弧状連結ならば
    弧状連結であることを示せ。
\end{problem}

\begin{answer}
    \cref{prop:cnd-and-loc-path-cnd-implies-path-cnd} を参照。
\end{answer}

\begin{problem}[幾何学II 2.7]
    コンパクト空間から Hausdorff 空間への連続写像は
    閉写像であることを示せ。
\end{problem}

\begin{answer}
    \cref{thm:compact-to-Hausdorff} を参照。
\end{answer}

\begin{problem}[幾何学II 2.8]
    全射かつ連続な閉写像は商写像であることを示せ。
\end{problem}

\begin{answer}
    \cref{prop:surj-closed-cts-map-is-quotient-map} を参照。
\end{answer}

\begin{problem}[幾何学II 2.9]
    位相空間$X$の部分集合$A$の部分集合$S$が$A$の閉集合であるためには、
    $X$のある閉集合$F$に対して$S = A \cap F$となることが必要十分であることを示せ。
\end{problem}

\begin{answer}
    相対位相の定義に注意すれば
    \begin{alignat}{1}
        S \closedsubset A
            &\Leftrightarrow A \setminus S \opensubset A \\
            &\Leftrightarrow
                \exists U \opensubset X \quad \text{s.t.} \quad
                A \setminus S = A \cap U \\
            &\Leftrightarrow
                \exists U \opensubset X \quad \text{s.t.} \quad
                S = A \cap (X \setminus U) \\
            &\Leftrightarrow
                \exists F \closedsubset X \quad \text{s.t.} \quad
                S = A \cap F
    \end{alignat}
    より題意の主張が成り立つ。
\end{answer}

\begin{problem}[幾何学II 2.10]
    位相空間の部分集合は、開かつ閉かつ弧状連結ならば
    弧状連結成分であることを示せ。
\end{problem}

\begin{answer}
    \cref{prop:clopen-path-cnd-implies-path-cnd-component} を参照。
\end{answer}

\subsection{問題セット 3}

\begin{problem}[幾何学II 3.1]
    \label[problem]{problem:geometry2-3.1}
    球面$S^n$の懸垂$\Sigma S^n$は$S^{n + 1}$と同相であることを示せ。
\end{problem}

\begin{answer}
    $\Sigma S^n$はコンパクト空間$ZS^n$の連続像ゆえにコンパクトで
    $S^{n + 1}$は Hausdorff だから、
    $\Sigma S^n$と$S^{n + 1}$の同相を示すには
    連続全単射$\Sigma S^n \to S^{n + 1}$の存在をいえばよい。
    まず写像$f \colon ZS^n \to S^{n + 1}$を
    \begin{equation}
        (x, t) \mapsto (\sqrt{1 - (2t - 1)^2} x, 2t - 1)
    \end{equation}
    で定める (円柱の上下端を絞って球面にするイメージ)。明らかに$f$は連続である。
    全射性については、
    各$y = (y_1, \dots, y_{n + 2}) \in S^{n + 1}$に対し
    $(x, t) \in ZS^n$を
    \begin{equation}
        (x, t) \coloneqq \begin{cases}
            \left(
                \frac{1}{\sqrt{1 - {y_{n + 2}}^2}} (y_1, \dots, y_{n + 1}),
                \frac{y_{n + 2} + 1}{2}
            \right)
                & \text{if $y_{n + 2} \neq \pm 1$} \\
            (1, 0, \dots, 0, y_{n + 2})
                & \text{if $y_{n + 2} = \pm 1$}
        \end{cases}
    \end{equation}
    とおけば$f(x, t) = y$をみたすから$f$は全射である。
    また、標準射影$ZS^n \to \Sigma S^n$を$p$とおくと、
    $f$の定め方から明らかに
    \begin{equation}
        p(x, t) = p(x', t') \implies f(x, t) = f(x', t')
        \quad ((x, t), (x', t') \in ZS^n)
    \end{equation}
    が成り立つ。
    したがって、等化写像の普遍性により図式
    \begin{equation}
        \begin{tikzcd}
            ZS^n \ar{d} \ar{r}{f} & S^{n + 1} \\
            \Sigma S^n \ar[dashed]{ru}[swap]{\wb{f}}
        \end{tikzcd}
    \end{equation}
    を可換にする連続写像
    $\wb{f} \colon \Sigma S^n \to S^{n + 1}$が誘導される。
    このとき$f$の全射性より$\wb{f}$は全射である。
    あとは$\wb{f}$が単射であることを示せばよい。
    \begin{alignat}{1}
        \wb{f}([(x, t)]) = \wb{f}([(x, t')])
            &\implies f(x, t) = f(x', t') \\
            &\implies (\sqrt{1 - (2t - 1)^2} x, 2t - 1)
                = (\sqrt{1 - (2t' - 1)^2} x', 2t' - 1) \\
            &\implies \begin{cases}
                \sqrt{1 - (2t - 1)^2} x = \sqrt{1 - (2t' - 1)^2} x' \\
                t = t'
            \end{cases} \\
            &\implies \begin{cases}
                \sqrt{1 - (2t - 1)^2} (x - x') = 0 \\
                t = t'
            \end{cases} \\
            &\implies
                t = 0 \vee t = 1 \vee (x, t) = (x', t') \\
            &\implies [(x, t)] = [(x', t')]
    \end{alignat}
    より、$\wb{f}$は単射である。
\end{answer}

\begin{problem}[幾何学II 3.2]
    単位閉球体$D^n$において
    境界$\del D^n$を1点に縮めた空間$D^n / \del D^n$は
    球面$S^n$と同相であることを示せ。
\end{problem}

\begin{answer}
    \TODO{stereographic projection を経由する必要はない!}
    %$n \ge 1$とする。
    %$D^n$はコンパクトだから、その連続像$D^n / \del D^n$もコンパクトである。
    %一方、$S^n$は距離空間$\R^{n + 1}$の部分空間だから Hausdorff である。
    %よって、$D^n / \del D^n$と$S^n$の同相を示すには
    %連続全単射$D^n / \del D^n \to S^n$の存在をいえばよい。
    %準備として、
    %まず写像$\sigma \colon \R^n \to S^n \setminus \{(0, \dots, 0, 1)\}$を
    %\begin{equation}
    %    \sigma(y) \coloneqq \left(
    %        \frac{2}{\|y\|^2 + 1} y_1,
    %        \dots,
    %        \frac{2}{\|y\|^2 + 1} y_n,
    %        \frac{\|y\|^2 - 1}{\|y\|^2 + 1}
    %    \right)
    %\end{equation}
    %で定める。これは立体射影の逆写像だから同相写像である。
    %つぎに写像$\tau \colon B^n \to \R^n$を
    %\begin{equation}
    %    x \mapsto \left( \tan \frac{\pi}{2} \|x\| \right) \frac{1}{\|x\|} x
    %\end{equation}
    %で定める\TODO{このへんの写像を修正する}。これは連続逆写像
    %\begin{equation}
    %    y \mapsto \left( \frac{2}{\pi} \arctan \|y\| \right) \frac{1}{\|y\|} y
    %\end{equation}
    %を持つから同相写像である (ただし$\arctan$の値域は$(-\pi / 2, \pi / 2)$とする)。
    %さて、写像$f \colon D^n \to S^n$を
    %\begin{equation}
    %    f(x) \coloneqq \begin{cases}
    %        \sigma \circ \tau(x)
    %            & \text{if $\|x\| < 1$} \\
    %        (0, \dots, 0, 1)
    %            & \text{if $\|x\| = 1$}
    %    \end{cases}
    %\end{equation}
    %で定める。
    %$f$が連続であることを示したい。
    %$f$は$\|x\| < 1$なる点$x$においては明らかに連続であるから、
    %$\|x_0\| = 1$なる点$x_0$における連続性を示す。
    %そこで、点$f(x_0) = (0, \dots, 0, 1)$の開近傍$V \opensubset S^n$が
    %任意に与えられたとする。
    %示したいことは
    %\begin{equation}
    %    x_0 \in \exists U \opensubset D^n
    %    \quad \text{s.t.} \quad
    %    f(U) \subset V
    %\end{equation}
    %である。
    %いま$S^n$には$\R^{n + 1}$の部分空間としての相対位相が入っており、
    %また$\R^{n + 1}$の開矩形の全体からなる集合系は$\R^{n + 1}$の開基であるから、
    %或る$\eps > 0$が存在して
    %\begin{equation}
    %    f(x_0)
    %        \in \underbrace{(
    %            (-\eps, \eps)
    %            \times \dots
    %            \times (-\eps, \eps)
    %            \times (1 - \eps, 1 + \eps)
    %        ) \cap S^n}_{\eqqcolon V' \text{ とおく}}
    %        \subset V
    %\end{equation}
    %が成り立つ。
    %ここで$\sigma$の第$i$成分 ($i = 1, \dots, n$) について、
    %$\|y\| > 2 / \eps$なるすべての$y \in \R^n$に対し
    %\begin{alignat}{1}
    %    \left| \frac{2}{\|y\|^2 + 1} y_i \right|
    %        &\le \frac{2}{\|y\|^2 + 1} \|y\| \\
    %        &\le \frac{2}{\|y\|} \\
    %        &< \eps
    %\end{alignat}
    %が成り立つ。また$\sigma$の第$n + 1$成分について、
    %写像
    %\begin{equation}
    %    [0, \infty) \to \R,
    %    \quad
    %    t \mapsto \frac{t^2 - 1}{t^2 + 1} = 1 - \frac{2}{t^2 + 1}
    %\end{equation}
    %の単調増加性および$t \to \infty$の極限が$1$であることから、
    %或る$\delta > 0$が存在して、$\|y\| > \delta$なるすべての$y$に対し
    %\begin{equation}
    %    \left|\frac{\|y\|^2 - 1}{\|y\|^2 + 1} - 1\right|
    %        < \eps
    %\end{equation}
    %が成り立つ。よって、$\|y\| > 2 / \eps + \delta$なるすべての$y$に対し
    %\begin{equation}
    %    \label[equation]{eq:problem-3.2-1}
    %    \sigma(y) \in V' \subset V
    %\end{equation}
    %が成り立つ。また$\tau$について、各$x \in B^n$に対し
    %\begin{equation}
    %    \left\| \frac{1}{1 - \|x\|^2} x \right\|
    %        = \frac{\|x\|}{1 - \|x\|^2}
    %\end{equation}
    %より、或る$\rho > 0$が存在して、
    %$\|x\| > \rho$なるすべての$x \in B^n$に対し
    %\begin{equation}
    %    \label[equation]{eq:problem-3.2-2}
    %    \left\| \frac{1}{1 - \|x\|^2} x \right\|
    %        > 2 / \eps + \delta
    %\end{equation}
    %が成り立つ。そこで
    %\begin{equation}
    %    U \coloneqq \{ x \in D^n \colon \|x\| > \rho \}
    %\end{equation}
    %とおくと、各$x \in U$に対し
    %\begin{itemize}
    %    \item $\|x\| < 1$ならば
    %        \cref{eq:problem-3.2-2,eq:problem-3.2-1}より
    %        $f(x) \in V$
    %    \item $\|x\| = 1$ならば$f(x) = (0, \dots, 0, 1) \in V$
    %\end{itemize}
    %が成り立つ。したがってこの$U$が求めるものである。
    %よって$f$は連続である。
    %また、標準射影 $D^n \to D^n / \del D^n$を$p$とおくと、
    %$f$の定め方から明らかに
    %\begin{equation}
    %    p(x) = p(x') \implies f(x) = f(x')
    %    \quad (x, x' \in D^n)
    %\end{equation}
    %が成り立つ。
    %よって等化写像の普遍性により、図式
    %\begin{equation}
    %    \begin{tikzcd}
    %        D^n \ar{d} \ar{r}{f} & S^n \\
    %        D^n / \del D^n \ar[dashed]{ru}[swap]{\wt{f}}
    %    \end{tikzcd}
    %\end{equation}
    %を可換にする連続写像$\wt{f}$がただひとつ存在する。
    %$\wt{f}$が全単射であることを示したい。
    %$\sigma \circ \tau$は
    %$B^n \to S^n \setminus \{(0, \dots, 0, 1)\}$の
    %同相写像であることに注意する。
    %全射性について、
    %各$p \in S^n$に対し、$x \in D^n$を
    %\begin{itemize}
    %    \item $p \neq (0, \dots, 0, 1)$ならば
    %        \begin{equation}
    %            x \coloneqq (\sigma \circ \tau)^{-1} (p)
    %        \end{equation}
    %    \item $p = (0, \dots, 0, 1)$ならば
    %        \begin{equation}
    %            x \coloneqq (0, \dots, 0, 1)
    %        \end{equation}
    %\end{itemize}
    %とおけば$f(x) = p$が成り立つ。
    %よって$f$は全射であり、したがって$\wt{f}$は全射である。
    %単射性について、$x, x' \in D^n$に対し
    %$\wt{f}([x]) = \wt{f}([x']) \eqqcolon p$、
    %すなわち$f(x) = f(x') = p$とすると、
    %\begin{itemize}
    %    \item $p \neq (0, \dots, 0, 1)$ならば
    %        $\sigma \circ \tau$で引き戻して$x = x'$、したがって$[x] = [x']$となり、
    %    \item $p = (0, \dots, 0, 1)$ならば
    %        $\|x\| = \|x'\| = 1$だから$x, x' \in \del D^n$、
    %        したがってやはり$[x] = [x']$となる。
    %\end{itemize}
    %よって$\wt{f}$は単射である。
\end{answer}

\begin{problem}[幾何学II 3.3]
    位相空間$X$に対して、
    柱$ZX$は$X$にホモトピー同値であり、
    錐$CX$は可縮であることを示せ。
\end{problem}

\begin{answer}
    写像$f \colon X \to ZX,\; g \colon ZX \to X$をそれぞれ
    \begin{alignat}{1}
        f(x) &\coloneqq (x, 0) \\
        g(x, s) &\coloneqq x
    \end{alignat}
    で定める。これらは明らかに連続である。
    $g \circ f(x) = x \; (x \in X)$より$g \circ f \simeq \id_{X}$である。
    また、写像$H \colon ZX \times I \to ZX$を
    \begin{alignat}{1}
        H((x, s), t) \coloneqq (x, ts)
    \end{alignat}
    で定めると、これは明らかに連続であり、図式
    \begin{equation}
        \begin{tikzcd}
            ZX \ar{d}[swap]{\xi \mapsto (\xi, 0)} \ar{rd}{\id_{ZX}} \\
            ZX \times I \ar{r}{H} & ZX \\
            ZX \ar{u}{\xi \mapsto (\xi, 1)} \ar{ru}[swap]{f \circ g}
        \end{tikzcd}
    \end{equation}
    を可換にする。
    よって$f \circ g \simeq \id_{ZX}$である。
    したがって$f, g$をホモトピー同値写像として$X, ZX$はホモトピー同値である。
    つぎに標準射影$ZX \to CX$を$\pi$とおく。
    すると$(x, s), (x', s') \in ZX,\; t \in I$に対し
    \begin{alignat}{1}
        (\pi(x, s), t) = (\pi(x', s'), t)
            &\implies s = s' = 0 \vee (x, s) = (x', s') \\
            &\implies \pi(x, ts) = \pi(x', ts') \\
            &\implies \pi \circ H((x, s), t) = \pi \circ H((x', s'), t)
    \end{alignat}
    が成り立つ。
    よって J. H. C. Whitehead の補題より
    $\pi \times \id_I$は等化写像となるから、
    図式
    \begin{equation}
        \begin{tikzcd}
            ZX \times I \ar{d}[swap]{\pi \times \id} \ar{r}{H} & ZX \ar{d}{\pi} \\
            CX \times I \ar[dashed]{r}[swap]{\wb{H}} & CX
        \end{tikzcd}
    \end{equation}
    を可換にする連続写像$\wb{H}$が誘導される。
    図式より
    \begin{alignat}{1}
        \wb{H}(\pi(x, s), 0) &= \pi(x, 0) \quad (\text{$x$によらない定点}) \\
        \wb{H}(\pi(x, s), 1) &= \pi(x, s) = \id_{CX}(x, s)
    \end{alignat}
    が成り立つから、$CX$は可縮である。
\end{answer}

\begin{problem}[幾何学II 3.4]
    連続写像$f \colon X \to Y$に対して、
    写像柱$Z_f$は$Y$とホモトピー同値であることを示せ。
\end{problem}

\begin{answer}
    \TODO{書き直す}
    標準射影$ZX \sqcup Y \to Z_f$を$\pi$とおく。
    写像$g \colon ZX \sqcup Y \to Y$を
    \begin{equation}
        g(\xi) \coloneqq \begin{cases}
            f(x) & (\xi = (x, s) \in ZX) \\
            y & (\xi = y \in Y)
        \end{cases}
    \end{equation}
    と定義する。すると$g$は連続である。
    実際、各$V \opensubset Y$に対し
    \begin{equation}
        g^{-1}(V) \cap ZX = f^{-1}(V) \opensubset X,
        \quad
        g^{-1}(V) \cap Y = V \opensubset Y
    \end{equation}
    より$g^{-1}(V) \opensubset ZX \sqcup Y$である。
    また、各$\xi, \xi' \in ZX \sqcup Y$に対し、
    $\pi(\xi) = \pi(\xi')$を仮定すると、
    \begin{itemize}
        \item $\xi, \xi' \in ZX$あるいは$\xi, \xi' \in Y$ならば
            明らかに$g(\xi) = g(\xi')$である。
        \item $\xi = (x, s) \in ZX,\; \xi' = y' \in Y$ならば
            \begin{equation}
                \begin{cases}
                    (x, s) &= (x, 1) \\
                    y' &= f(x)
                \end{cases}
            \end{equation}
            だから$g(\xi) = g(\xi')$である。
            $\xi \in Y,\; \xi' \in ZX$の場合も同様である。
    \end{itemize}
    よって、等化写像の普遍性により
    連続写像$\wt{g} \colon Z_f \to Y$が誘導される。
    さらに連続写像$h \colon Y \to Z_f$を$h(y) \coloneqq \pi(y)$で定める。
    $\wt{g}, h$がホモトピー同値写像であることを示したい。
    まず$y \in Y$に対し
    \begin{alignat}{1}
        \wt{g} \circ h(y)
            &= \wt{g}(\pi(y)) \\
            &= g(y) \\
            &= y
    \end{alignat}
    より$\wt{g} \circ h \simeq \id_Y$である。
    次に$h \circ \wt{g} \simeq \id_{Z_f}$を示す。
    いま$\xi \in Z_f$に対し
    \begin{alignat}{1}
        h \circ \wt{g}(\pi(\xi))
            &= \begin{cases}
                \pi(f(x)) & (\xi = (x, s) \in ZX) \\
                \pi(y) & (\xi = y \in Y)
            \end{cases}
    \end{alignat}
    である。
    ホモトピーを構成するため、
    写像$H \colon (ZX \sqcup Y) \times I \to ZX \sqcup Y$を
    \begin{equation}
        H(\xi, t) \coloneqq \begin{cases}
            (x, t + (1 - t) s) & (\xi = (x, s) \in ZX) \\
            y & (\xi = y \in Y)
        \end{cases}
    \end{equation}
    で定義する。すると、ふたつの写像
    \begin{alignat}{1}
        ZX \times I \to ZX, &\quad ((x, s), t) \mapsto (x, t + (1 - t) s) \\
        Y \times I \to Y, &\quad (y, t) \mapsto y
    \end{alignat}
    の連続性より、$H$は連続である。
    また、各$\xi, \xi' \in ZX \sqcup Y,\; t \in I$に対し、
    $\pi(\xi) = \pi(\xi')$を仮定すると、
    \begin{itemize}
        \item $\xi, \xi' \in ZX$あるいは$\xi, \xi' \in Y$ならば
            明らかに$\pi \circ H(\xi, t) = \pi \circ H(\xi', t)$である。
        \item $\xi = (x, s) \in ZX,\; \xi' = y' \in Y$ならば
            \begin{equation}
                \begin{cases}
                    (x, s) &= (x, 1) \\
                    y' &= f(x)
                \end{cases}
            \end{equation}
            だから
            \begin{equation}
                \begin{cases}
                    \pi \circ H(\xi, t)
                        = \pi \circ H((x, 1), t)
                        = \pi(x, 1) \\
                    \pi \circ H(\xi', t)
                        = \pi \circ H(f(x), t)
                        = \pi(f(x))
                \end{cases}
            \end{equation}
            より$\pi \circ H(\xi, t) = \pi \circ H(\xi', t)$である。
            $\xi \in Y,\; \xi' \in ZX$の場合も同様である。
    \end{itemize}
    よって、J. H. C. Whitehead の補題の系より、
    \begin{equation}
        \begin{tikzcd}
            (ZX \sqcup Y) \times I
                \ar{d}[swap]{\pi \times \id}
                \ar{r}{H}
                & ZX \sqcup Y
                \ar{d}{\pi} \\
            Z_f \times I
                \ar[dashed]{r}[swap]{G}
                & Z_f
        \end{tikzcd}
    \end{equation}
    を可換にする連続写像$G$が存在する。
    $G$は各$\xi \in ZX \sqcup Y$に対し
    \begin{alignat}{1}
        G(\pi(\xi), 0)
            &= \pi \circ H(\xi, 0) \\
            &= \pi(\xi) \\
            &= \id_{Z_f}(\pi(\xi))
    \end{alignat}
    および
    \begin{alignat}{1}
        G(\pi(\xi), 1)
            &= \pi \circ H(\xi, 1) \\
            &= \begin{cases}
                \pi(x, 1) = \pi(f(x)) & (\xi = (x, s) \in ZX) \\
                \pi(y) & (\xi = y \in Y)
            \end{cases} \\
            &= h \circ \wt{g}(\pi(\xi))
    \end{alignat}
    をみたすから、$\id_{Z_f}$と$h \circ \wt{g}$をつなぐホモトピーである。
    よって$h \circ \wt{g} \simeq \id_{Z_f}$である。
    以上で$\wt{g}, h$がホモトピー同値写像であることが示された。
    したがって$Z_f$は$Y$とホモトピー同値である。
\end{answer}

\begin{problem}[幾何学II 3.5]
    連続写像$f \colon X \to Y$に対して、
    写像錐$C_f$への$Y$の自然な埋め込み$i \colon Y \to C$の
    写像錐$C_i$は、$X$の懸垂$\Sigma X$とホモトピー同値であることを示せ。
\end{problem}

\begin{answer}
    \TODO{}
\end{answer}

\begin{problem}[幾何学II 3.6]
    球面$S^n$から位相空間$X$への連続写像$f \colon S^n \to X$が
    定値写像にホモトープであるためには、
    単位閉球体$D^{n + 1}$からの連続写像$g \colon D^{n + 1} \to X$で
    $g|_{S^n} = f$をみたすものが存在することが必要十分であることを示せ。
\end{problem}

\begin{answer}
    $(\Leftarrow)$ \quad
    題意の連続写像$g$の存在を仮定する。
    写像$H \colon S^n \times I \to X$を
    \begin{equation}
        H(p, t) \coloneqq g(tp)
    \end{equation}
    で定めると、$H$は連続であり、図式
    \begin{equation}
        \begin{tikzcd}
            S^n \ar{d}[swap]{p \mapsto (p, 0)} \ar{rd}{p \mapsto g(0)} \\
            S^n \times I \ar{r}{H} & X \\
            S^n \ar{u}{p \mapsto (p, 1)} \ar{ru}[swap]{f}
        \end{tikzcd}
    \end{equation}
    を可換にする。
    よって、ホモトピー$H$により$f$は定値写像$p \mapsto g(0)$にホモトープである。

    $(\Rightarrow)$ \quad
    $f$が定値写像$S^n \to X,\; p \mapsto x_0$にホモトープであるとする。
    すると図式
    \begin{equation}
        \begin{tikzcd}
            S^n \ar{d}[swap]{p \mapsto (p, 0)} \ar{rd}{p \mapsto x_0} \\
            S^n \times I \ar{r}{H} & X \\
            S^n \ar{u}{p \mapsto (p, 1)} \ar{ru}[swap]{f}
        \end{tikzcd}
    \end{equation}
    を可換にするホモトピー$H$が存在する。
    そこで、写像$g \colon D^{n + 1} \to X$を
    \begin{equation}
        g(q) \coloneqq \begin{cases}
            H\left(\frac{1}{\|q\|} q, \|q\|\right) & (\|q\| \neq 0) \\
            x_0 & (\|q\| = 0)
        \end{cases}
    \end{equation}
    で定める。$g$の$\|q\| \neq 0$なる点$q \in D^{n + 1}$での連続性は明らか。
    また、$g$は点$q = 0$でも連続である。
    \begin{innerproof}
        \TODO{$\eps$-$\delta$論法でもっと楽に示せるか?}
        点$g(0) = x_0$の開近傍$V \opensubset X$が任意に与えられたとする。
        このとき各$p \in S^n$に対し、$x_0 = H(p, 0)$であることと
        $H$の連続性より
        \begin{equation}
            (p, 0) \in \exists U_p \opensubset S^n \times I
            \quad \text{s.t.} \quad
            H(U_p) \subset V
        \end{equation}
        が成り立つ。
        そこで$S^n \times I$の部分集合系$\scrU$を
        \begin{equation}
            \scrU \coloneqq \{ U_p \colon p \in S^n \} \cup \{ S^n \times (0, 1] \}
        \end{equation}
        で定めると、$\scrU$は open cover となる。
        $S^n \times I$はコンパクト距離空間だから、
        $\scrU$の Lebesgue 数$\rho > 0$が存在する。
        そこで$D^{n + 1}$の開部分集合$U$を
        \begin{equation}
            U \coloneqq \{ q \in D^{n + 1} \colon \|q\| < \rho / 2 \}
        \end{equation}
        で定める。
        $g(U) \subset V$であることを示したい。
        そこで$q \in U$とする。$q = 0$ならば$g(q) = x_0 \in V$が成り立つから、
        $q \neq 0$の場合を考える。
        Lebesgue 数の定義より
        \begin{equation}
            \exists W \in \scrU
            \quad \text{s.t.} \quad
            B\left(\frac{1}{\|q\|} q, \|q\|;\; \rho\right) \subset W
        \end{equation}
        が成り立つ (ただし$B(x; r)$は$D^{n + 1}$における開球)。
        いま$\|q\| < \rho / 2$ゆえに
        $B\left(\frac{1}{\|q\|} q, \|q\|;\; \rho\right)$は$S^n \times \{0\}$と
        交わりを持つから、$W = S^n \times (0, 1]$ではありえない。
        よって open cover $\scrU$の定め方から
        $W = U_{p_0}\; (\exists p_0 \in S^n)$
        である。したがって
        \begin{alignat}{1}
            & \left(\frac{1}{\|q\|} q, \|q\|\right) \in U_{p_0} \\
            \therefore \quad &g(q) =
                H \left(\frac{1}{\|q\|} q, \|q\|\right)
                \in H(U_{p_0}) \subset V
        \end{alignat}
        が成り立つ。
        これで$g(U) \subset V$がいえた。
        よって$g$は点$q = 0$において連続である。
    \end{innerproof}
    したがって$g$は$D^{n + 1}$上で連続である。
    $g$の定め方から明らかに$g|_{S^n} = f$も成り立つ。
\end{answer}

\subsection{問題セット 4}

\begin{problem}[幾何学II 4.1]
    次の写像は被覆空間を与えることを示せ。
    \begin{enumerate}
        \item $\pi_n \colon \C^* \to \C^*, \quad z \mapsto z^n$
        \item $\pi \colon \C \to \C^*, \quad z \mapsto \exp(2\pi \sqrt{-1} z)$
    \end{enumerate}
\end{problem}

\begin{answer}
    \uline{(1)} \quad
    位相群$S^1$が普通の積により$\C^*$に連続に作用していると考える。
    まず開集合$U_0 \opensubset \C^*, \; V_0 \opensubset \C^*$を
    \begin{alignat}{1}
        U_0 &\coloneqq \{ w \in \C^* \mid -\pi < \Arg w < \pi \} \\
        V_0 &\coloneqq \{ z \in \C^* \mid -\pi / n < \Arg z < \pi / n \}
    \end{alignat}
    で定める。ただし$\Arg$の値域は$(-\pi, \pi]$とする。
    $\pi_n$が被覆空間であることを示すため、
    $w \in \C^*$とし、$w$が$\pi_n$により自明に被覆されることを示す。
    \begin{alignat}{1}
        U &\coloneqq \frac{w}{|w|} \cdot U_0 \\
        V &\coloneqq \left(\frac{w}{|w|}\right)^{1/n} \cdot V_0
    \end{alignat}
    とおく。ただし$\left(\frac{w}{|w|}\right)^{1/n}
    \coloneqq \exp\left( i \frac{\Arg (w / |w|)}{n} \right)$である。
    このとき
    \begin{equation}
        \pi_n^{-1}(U) = \bigcup_{k = 0}^{n - 1} \alpha^k \cdot V
    \end{equation}
    と disjoint union に書ける。ただし$\alpha$は
    $1$の原始$n$乗根$e^{2\pi i / n}$である。
    \begin{innerproof}
        \TODO{}
    \end{innerproof}
    あとは各$k = 0, \dots, n - 1$に対し
    $\alpha^k \cdot V$が$\pi_n$の制限により$U$と同相であることを示せばよいが、
    位相群の連続作用の性質から
    \begin{equation}
        \alpha^k \cdot V \approx V
    \end{equation}
    なので、$V$が$\pi_n|_V$により$U$と同相となることを示せば十分である。
    さらに図式
    \begin{equation}
        \begin{tikzcd}[
            column sep=large,
            row sep=large,
            every label/.append style={font=\footnotesize}
        ]
            V_0
                \ar{d}[swap]{\left(\frac{w}{|w|}\right)^{1/n}}{\approx}
                \ar{r}{\pi_n|_{V_0}}
                & U_0
                \ar{d}{\frac{w}{|w|}}[swap]{\approx} \\
            V
                \ar{r}[swap]{\pi_n|_V}
                & U
        \end{tikzcd}
    \end{equation}
    が可換であることから、
    $V_0$が$\pi_n|_{V_0}$により$U_0$と同相であることを示せばよい。
    実際、$\pi_n|_{V_0}$は
    $V_0$から$U_0$への同相写像である。
    \begin{innerproof}
        $\pi_n|_{V_0} \colon V_0 \to U_0$が well-defined かつ
        連続全射であることは明らか。
        また、$\pi_n|_{V_0}$は$\C$の連結開集合$V_0$上の定数でない解析関数だから、
        開写像定理より$\C$への写像として open map であり、
        codomain を$U_0$に制限したものも open map である。
        最後に単射性について、
        \begin{equation}
            \pi_n|_{V_0}(z) = \pi_n|_{V_0}(z')
            \quad
            (z_0, z_1 \in V_0)
        \end{equation}
        とすると
        \begin{alignat}{2}
            z_0^n = z_1^n \quad
                &\therefore &\quad \left(\frac{z_0}{z_1}\right)^n &= 1 \\
                &\therefore &\quad \frac{z_0}{z_1} &= \alpha^k
                \quad (\exists k = 0, \dots, n - 1) \\
                &\therefore &\quad \Arg \frac{z_0}{z_1} &= 2\pi \frac{k}{n}
        \end{alignat}
        だから、$z_0, z_1 \in V_0$より
        $- \frac{2\pi}{n} < \Arg \frac{z_0}{z_1} < \frac{2\pi}{n}$
        であることとあわせて$k = 0$、
        よって$z_0 = z_1$である。
        これで単射性がいえた。
        したがって$\pi_n|_{V_0}$は$V_0$から$U_0$への同相写像である。
    \end{innerproof}

    \uline{(2)} \quad
    \TODO{}
\end{answer}

\begin{problem}[幾何学II 4.2]
    \label[problem]{problem:geometry2-4.2}
    トーラス$T^2 = S^1 \times S^1$の基本群を求めよ。
\end{problem}

\begin{proof}[解答1]
    基本群が直積を保つことを用いる。
    \cref{problem:geometry2-4.8}の証明と全く同様なので省略。
\end{proof}

\begin{proof}[解答2]
    \TODO{色々言葉が不足してる}
    $T^2 \approx \R^2 / \Z^2$だから、
    $\R^2 / \Z^2$の基本群を求めればよい。
   標準射影$\R^2 \to \R^2 / \Z^2$を$p$とおく。
    $p$が普遍被覆であることを示せば
    \begin{equation}
        \pi_1(\R^2 / \Z^2, \xi_0) \cong \Deck(\R^2)
        \quad
        (\forall \xi_0 \in \R^2 / \Z^2)
    \end{equation}
    が成り立つことを用いて基本群が求まる。
    実際、$p$は普遍被覆である。
    \begin{innerproof}
        $\xi_0 = p(x_0, y_0) \in \R^2 / \Z^2$とする。
        $p$の全射性より$\xi_0 = p(x_0, y_0)$なる$(x_0, y_0) \in \R^2$が存在する。
        そこで、中心$(x_0, y_0)$、半径$1/2$の$\R^2$内の開球
        $B((x_0, y_0), 1/2)$を$V$とおき、
        $U \coloneqq p(V)$とおく。
        いま$\R^2 / \Z^2$は位相群$\Z^2$の連続作用に関する商空間ゆえに
        $p$は open map だから、$U$は$\R^2 / \Z^2$の開集合である。
        $U$が$\xi_0$の自明化近傍となることを示せばよい。
        まず明らかに$p^{-1}(U)$は
        \begin{equation}
            p^{-1}(U)
                = \bigcup_{(m, n) \in \Z^2} B((x_0 + m, y_0 + n), 1/2)
                = \bigcup_{(m, n) \in \Z^2} (m, n) \cdot V
                \quad
                (\text{$\,\cdot\,$は$\Z^2$の作用})
        \end{equation}
        と disjoint union に書ける。
        あとは各$(m, n) \cdot V$が$p$の制限により$U$と同相となることを示せばよいが、
        位相群の連続作用の性質から
        \begin{equation}
            (m, n) \cdot V \approx V
        \end{equation}
        なので、$V$が$p|_V$により$U$と同相となることを示せば十分である。
        $p|_V \colon V \to U$が well-defined な連続全射であることは明らか。
        また、$p$は open map だから$p|_V \colon V \to U$も open map である。
        最後に単射性について、$(x, y), (x', y') \in V$に対し
        $p|_V(x, y) = p|_V(x', y')$を仮定すると
        $(x, y) - (x', y') = (m, n) \; (\exists m, n \in \Z)$が成り立つから
        \begin{equation}
            \|(x, y) - (x', y')\| = \sqrt{m^2 + n^2}
        \end{equation}
        である。したがって$V$の定義より$(m, n) = (0, 0)$でなければならず、
        $(x, y) = (x', y')$となる。
        よって$p|_V$は単射である。
        したがって$p|_V \colon V \to U$は同相写像であることがいえた。
        よって$U$は$\xi_0$の自明化近傍であり、$\xi_0$は$p$により自明に被覆される。
        $\xi_0 \in \R^2 / \Z^2$は任意であったから、
        $p$は$\R^2 / \Z^2$の普遍被覆である。
    \end{innerproof}
    また、$\R^2 / \Z^2$の局所弧状連結性は
    $\R^2$が局所弧状連結であることと$p$が局所同相写像であることから従う。
    また、$\Deck(\R^2)$は群として$\Z^2$と同型である。
    \begin{innerproof}
        写像$\Phi \colon \Z^2 \to \Deck(\R^2)$を、
        $(m, n) \in \Z^2$に対し
        $\Z^2$の連続作用のもとで$(m, n)$が定める写像
        \begin{equation}
            \mu_{(m, n)} \colon \R^2 \to \R^2,
            \quad
            (x, y) \mapsto (x + m, y + n)
        \end{equation}
        を対応付けるものと定める。
        明らかに$\Phi$は群準同型かつ単射である。
        全射性を示すため、$f \in \Deck(\R^2)$とする。
        $f$は図式
        \begin{equation}
            \begin{tikzcd}
                \R^2 \ar{rd}[swap]{p} \ar{rr}{f} && \R^2 \ar{ld}{p} \\
                & \R^2 / \Z^2
            \end{tikzcd}
        \end{equation}
        を可換にするから、各$(x, y) \in \R^2$に対し
        或る$(m_x, n_y) \in \Z^2$がただひとつ存在して
        \begin{equation}
            f(x, y) - (x, y) = (m_x, n_y)
        \end{equation}
        が成り立つ。
        この写像
        \begin{equation}
            (x, y) \mapsto f(x, y) - (x, y) = (m_x, n_y)
        \end{equation}
        は連結空間$\R^2$から離散空間$\Z^2$への連続写像だから定値である。
        よって$(m_x, n_y)$は$(x, y)$によらず決まる。
        したがって$f$は$\Z^2$の連続作用のもとで$(m, n)$により定まる写像、
        すなわち$f = \Phi(m, n)$である。
        これで$\Phi$の全射性がいえた。
        したがって、$\Phi$は群の同型$\Z^2 \cong \Deck(\R^2)$を与える。
    \end{innerproof}
    よって$\R^2 / \Z^2$の、したがって$T^2$の基本群は
    直積群$\Z^2$に同型である。
\end{proof}

\begin{problem}[幾何学II 4.3]
    実射影空間$\R P^n \; (n \ge 2)$の基本群の生成元を1つ与えよ。
\end{problem}

\begin{answer}
    $\R P^n = S^n / \{ \pm 1 \}$と考えることにする。
   標準射影$S^n \to \R P^n$を$p$とおくと、
    $p$は普遍被覆である。
    $p$の各ファイバーの濃度は$2$だから、
    $\R P^n$の基本群も濃度$2$であり、したがって$\Z / 2\Z$に群として同型である。
    ここで$x_0 \coloneqq (1, 0, \dots, 0) \in S^n$とおき、
    $x_0$と$-x_0$をつなぐ$S^n$内のパス$\gamma$を
    \begin{equation}
        \gamma(s) \coloneqq (\cos \pi s, \sin \pi s, 0, \dots, 0)
        \quad (s \in I)
    \end{equation}
    で定める。
    このとき$p \circ \gamma$は$\R P^n$内のループとなる。
    そこで$[p \circ \gamma]$が$\pi_1(\R P^n, p(x_0))$の単位元でないことを示せば、
    これが求める生成元のひとつである。
    実際、$[p \circ \gamma]$は単位元でない。
    \begin{innerproof}
        背理法のため$[p \circ \gamma]$が単位元であると仮定する。
        仮定より、ある$\xi_0 \in \R P^n$が存在して、
        あるホモトピー$H \colon I \times I \to \R P^n$により
        \begin{equation}
            p \circ \gamma \sim \xi_0 \quad \rel \quad \{ 0, 1 \}
        \end{equation}
        が成り立つ。
        また、$\gamma$は$p \circ \gamma$の、
        定値写像$x_0$は$\xi_0$のリフトであって
        共通の始点を持つ。
        したがって、端点を固定するホモトピーのリフトの一意存在定理より、
        $H$のリフト$\wt{H} \colon I \times I \to S^n$であって
        端点を固定して$\gamma$を$x_0$につなぐホモトピーであるものが存在する。
        このとき
        \begin{equation}
            -x_0 = \gamma(1) = \wt{H}(1, 0) = \wt{H}(1, 1) = x_0
        \end{equation}
        となり矛盾が従う。
        背理法より$[p \circ \gamma]$は単位元でない。
    \end{innerproof}
\end{answer}

\begin{problem}[幾何学II 4.4]
    基本群を利用して、座標空間$\R^n \; (n \ge 3)$は平面$\R^2$と同相でないことを示せ。
\end{problem}

\begin{answer}
    背理法のため$\R^n$と$\R^2$が同相であると仮定する。
    仮定より、$\R^2$から原点を、$\R^n$からそれに対応する1点をそれぞれ除いた空間も同相である。
    必要ならば平行移動を合成することで
    $\R^2 \setminus \{ 0 \} \approx \R^n \setminus \{ 0 \}$
    であるとしてよい。
    ここで、$S^1 \subset \R^2 \setminus \{ 0 \}$および
    $S^{n - 1} \subset \R^n \setminus \{ 0 \}$は、
    Euclid ノルムを$1$に正規化するレトラクションにより
    それぞれ変形レトラクトとなっている。
    したがって
    \begin{alignat}{1}
        \R^2 \setminus \{ 0 \} &\simeqhe S^1 \\
        \R^n \setminus \{ 0 \} &\simeqhe S^{n - 1}
    \end{alignat}
    である。
    よって、基本群のホモトピー不変性より
    $\pi_1(S^1) \cong \pi_1(S^{n - 1})$が成り立つ。
    左辺は非自明群で右辺は$n \ge 3$ゆえに自明群だから矛盾が従う。
    背理法より$\R^n$と$\R^2$は同相でない。
\end{answer}

\begin{problem}[幾何学II 4.5]
    基本群を利用して、$S^1$は$D^2$のレトラクトでないことを示せ。
\end{problem}

\begin{remark}
    一般に$n \in \Z_{\ge 0}$に対し$S^n$は$D^{n + 1}$のレトラクトでない。
    この事実はホモロジーを用いて示すことができる。
\end{remark}

\begin{answer}
    包含写像$S^1 \to D^2$を$i$とおく。
    レトラクション$r \colon D^2 \to S^1$が存在したとすると
    $r \circ i = \id_{S^1}$だから
    基本群に誘導される準同型
    $i_* \colon \pi_1(S^1, 1) \to \pi_1(D^2, 1)$
    は単射であるが、
    左辺は$\Z$、右辺は$0$だから単射ではありえない。
    背理法より$S^1$は$D^2$のレトラクトでない。
\end{answer}

\begin{problem}[幾何学II 4.6]
    \label[problem]{problem:geometry2-4.6}
    集合$\R^3 \setminus (S^1 \times \{ 0 \})$の基本群を求めよ。
\end{problem}

\begin{answer}
    所与の集合を$X \coloneqq \R^3 \setminus (S^1 \times \{ 0 \})$とおき、
    右半平面から1点$(1, 0)$を抜いた空間を
    \begin{equation}
        Y \coloneqq (\R_{\ge 0} \times \R) \setminus \{ (1, 0) \}
    \end{equation}
    とおく。
    写像$f \colon X \to Y$を
    \begin{equation}
        (x, y, z) \mapsto (\sqrt{x^2 + y^2}, z)
    \end{equation}
    で定める。これは明らかに連続である。
    \TODO{$f_*$の単射性をどう示す?}
\end{answer}

\begin{problem}[幾何学II 4.7]
    \label[problem]{problem:geometry2-4.7}
    位相空間$X$が単連結な開集合$U, V$で被覆され、
    共通部分$U \cap V$が弧状連結ならば、
    $X$は単連結であることを示せ。
\end{problem}

\begin{answer}
    \TODO{「基点は$U \setminus V$にあるとしてよい」からスタートしたほうがよいか?}
    $\gamma$を$X$内のループとし、
    $\gamma$が端点を固定して定値ループにホモトピックであることを示す。
    $\gamma(I) \subset U$または$\gamma(I) \subset V$の場合は、
    $U, V$の単連結性より$\gamma$は端点を固定して定値ループにホモトピックである。
    以下、$\gamma(I) \not\subset U$かつ$\gamma(I) \not\subset V$の場合を考える。
    このとき、$\gamma(I)$は$U \cap V$と交わりを持つ。
    \begin{innerproof}
        もし$\gamma(I)$が$U \cap V$と交わりを持たないとすると、
        \begin{alignat}{1}
            I &= \gamma^{-1}(X) \\
                &= \gamma^{-1}(U \cup V)
                \quad (\because \text{ $U, V$は$X$の被覆}) \\
                &= \gamma^{-1}(U) \cup \gamma^{-1}(V)
        \end{alignat}
        となる。
        $\gamma$の連続性より右辺は開集合の disjoint union だから、
        $I$の連結性より$\gamma^{-1}(U) = I$または$\gamma^{-1}(V) = I$となり、
        $\gamma(I) \subset U$または$\gamma(I) \subset V$となって矛盾が従う。
    \end{innerproof}
    そこで、必要ならばパラメータを取り替えて
    $\gamma$の基点は$U \cap V$上の点であるとしてよい。
    さて、$\{ \gamma^{-1}(U), \gamma^{-1}(V) \}$は
    コンパクト距離空間$I$の open cover だから、
    Lebesgue 数$\rho > 0$が存在する。
    さらに$1/N < \rho$なる$N \in \Z_{\ge 1}$が存在して、
    $1/N$も Lebesgue 数のひとつである。
    したがって$I$を$N$等分した小区間の列
    \begin{equation}
        [0, 1/N], [1/N, 2/N], \dots, [(N - 1) / N, 1]
    \end{equation}
    のそれぞれは$\gamma^{-1}(U)$または$\gamma^{-1}(V)$の少なくとも一方に含まれる。
    必要ならば隣り合う小区間を繋げて、$I$の小区間の列
    \begin{equation}
        [s_0, s_1], [s_1, s_2], \dots, [s_{N - 1}, s_N]
        \quad
        (s_0 = 0, \; s_N = 1)
    \end{equation}
    は次の条件をみたすとしてよい:
    \begin{enumerate}[label=(\roman*)]
        \item 各小区間は$\gamma^{-1}(U)$または$\gamma^{-1}(V)$の少なくとも一方に含まれる。
        \item 各$s_i \; (i = 0, \dots, N)$は
            $\gamma^{-1}(U) \cap \gamma^{-1}(V) = \gamma^{-1}(U \cap V)$に属する。
    \end{enumerate}
    \begin{innerproof}
        隣り合う小区間の繋げ方は、たとえば$[0, 1 / N]$が
        $\gamma^{-1}(U)$に含まれるなら、
        \begin{alignat}{1}
            [0, 1 / N] &\subset \gamma^{-1}(U) \\
            &\vdots \\
            [(k - 1) / N, k / N] &\subset \gamma^{-1}(U) \\
            [k / N, (k + 1) / N] &\subset \gamma^{-1}(V)
        \end{alignat}
        と初めて$\gamma^{-1}(V)$に含まれる小区間が現れたら
        $s_1 \coloneqq k / N$とおく操作を繰り返せばよい。
        こうして得られる小区間の列は明らかに(i)をみたす。
        また、上の例で$s_1 = k / N \in \gamma^{-1}(U) \cap \gamma^{-1}(V)$
        であることから(ii)も成り立つことがわかる。
    \end{innerproof}
    ここで、小区間$[s_i, s_{i + 1}] \; (i = 0, \dots, N - 1)$が
    $\gamma^{-1}(V)$に含まれるとする。
    小区間の条件(ii)より$\gamma(s_i), \gamma(s_{i + 1})$は
    $U \cap V$上の点だから、$U \cap V$の弧状連結性より
    $\gamma(s_i)$と$\gamma(s_{i + 1})$をつなぐ$U \cap V$内のパス$\beta_i$が存在する。
    このとき、パス$\gamma$の$\gamma(s_i)$から$\gamma(s_{i + 1})$までの部分は
    $\beta_i$と共通の端点をもつ$V$内のパスとなるから、
    $V$の単連結性より端点を固定して$\beta_i$にホモトピックである。
    同様の議論を$\gamma^{-1}(V)$に含まれるすべての小区間について行うことで、
    $\gamma$は$U$内の或るループ$\beta$に端点を固定してホモトピックであることがわかる。
    \begin{innerproof}
        $\gamma$の$\gamma(s_i)$から$\gamma(s_{i + 1})$までの部分とは、
        より正確には
        \begin{equation}
            \gamma_i(s) \coloneqq \gamma((1 - s) s_i + s s_{i + 1})
        \end{equation}
        で定まるパス$\gamma_i \colon I \to X$のことである。
        $\beta_i, \gamma_i$は共通の端点を持つ$V$内のパスであるから、
        $V$の単連結性より、
        端点を固定して$\beta_i$を$\gamma_i$につなぐホモトピー
        $H_i \colon I \times I \to V$が存在する。
        そこで、写像$H \colon I \times I \to X$を
        \begin{equation}
            H(s, t) \coloneqq \begin{cases}
                H_i \biggl( \frac{s - s_i}{s_{i + 1} - s_i}, t \biggr)
                & (s \in [s_i, s_{i + 1}] \subset \gamma^{-1}(V)) \\
                \gamma(s)
                & (s \in [s_i, s_{i + 1}] \subset \gamma^{-1}(U))
            \end{cases}
        \end{equation}
        で定める。各$H_i$は端点を固定するから、これは well-defined である。
        $H$の各閉集合$[s_i, s_{i + 1}] \times I$上への制限は連続だから、
        貼り合わせ補題より$H$も連続である。
        また、定め方から明らかに
        \begin{alignat}{2}
            H(s, 0) &= \gamma(s) && \quad (s \in I) \\
            H(s, 1) &\in U && \quad (s \in I) \\
            H(0, t) &= H(1, t) = \gamma(0) && \quad (t \in I)
        \end{alignat}
        が成り立つ。
        そこで$\beta \colon I \to U$を
        \begin{equation}
            \beta(s) \coloneqq H(s, 1)
        \end{equation}
        で定めれば、$\beta$は基点を$\gamma(0)$とする$U$内のループであり、
        $H$は端点を固定して$\gamma$を$\beta$につなぐホモトピーである。
    \end{innerproof}
    $U$の単連結性から$\beta$、したがって$\gamma$は定値ループにホモトピックである。
    以上で$X$の単連結性が示せた。
\end{answer}

\begin{problem}[幾何学II 4.8]
    \label[problem]{problem:geometry2-4.8}
    点付き空間$(X, x_0), (Y, y_0)$に対して、
    直積$(X \times Y, (x_0, y_0))$の基本群は
    群の直積$\pi_1(X, x_0) \times \pi_1(Y, y_0)$と同型であることを示せ。
\end{problem}

\begin{answer}
    $X \times Y$から$X, Y$への射影をそれぞれ$p, q$とおく。
    $p, q$から誘導される群準同型$p_*, q_*$の直積 (これも群準同型である)
    \begin{equation}
        p_* \times q_* \colon \pi_1(X \times Y, (x_0, y_0))
            \to \pi_1(X, x_0) \times \pi_1(Y, y_0)
    \end{equation}
    が同型を与えることを示せばよい。
    まず、$p_* \times q_*$は単射である。
    \begin{innerproof}
        $[\gamma] \in \pi_1(X \times Y, (x_0, y_0))$が
        $p_* \times q_* ([\gamma]) = 1$をみたすとすれば、
        \begin{alignat}{1}
            p_*([\gamma]) &= [p \circ \gamma] = 1 \\
            q_*([\gamma]) &= [q \circ \gamma] = 1
        \end{alignat}
        より、それぞれホモトピー$H, G$により
        \begin{alignat}{1}
            p \circ \gamma &\sim x_0 \quad \rel \quad \{ 0, 1 \} \\
            q \circ \gamma &\sim y_0 \quad \rel \quad \{ 0, 1 \}
        \end{alignat}
        となる。
        そこで$F \coloneqq H \times G \colon I \times I \to X \times Y$とおけば、
        $F$により
        \begin{equation}
            \gamma \sim (x_0, y_0) \quad \rel \quad \{ 0, 1 \}
        \end{equation}
        が成り立つ。
        よって$[\gamma] = 1$である。
        したがって$p_* \times q_*$は単射である。
    \end{innerproof}
    また、$p_* \times q_*$は全射である。
    \begin{innerproof}
        $([\alpha], [\beta]) \in \pi_1(X, x_0) \times \pi_1(Y, y_0)$とする。
        直積$\alpha \times \beta$は$X \times Y$内のパスであり
        \begin{equation}
            p_* \times q_*([\alpha \times \beta])
                = (p_*([\alpha \times \beta]), q_*([\alpha \times \beta]))
                = ([\alpha], [\beta])
        \end{equation}
        をみたす。
        よって$p_* \times q_*$は全射である。
    \end{innerproof}
    したがって$p_* \times q_*$は同型写像である。
\end{answer}

\subsection{幾何学II 練習問題}

\begin{problem}[幾何学II 練習問題25]
    実射影平面の被覆空間を分類せよ。
\end{problem}

\begin{proof}
    \TODO{}
\end{proof}


\end{document}

\newpage
\documentclass[report]{jlreq}
\usepackage{global}
\usepackage{../sub/local}
\subfiletrue
\def\assetspath{../}
\begin{document}

% ============================================================
%
% ============================================================
\chapter{指数型分布族}

% ------------------------------------------------------------
%
% ------------------------------------------------------------
\section{指数型分布族}

\begin{definition}[指数型分布族]
    \label[definition]{def:exponential-family}
    \idxsym{exponential family}{$(V, T, \mu)$}{指数型分布族}
    $\calX$を可測空間、
    $\emptyset \neq \calP \subset \calP(\calX)$とする。
    $\calP$が$\calX$上の
    \term{指数型分布族}[exponential family]
        {指数型分布族}[しすうがたぶんぷぞく]
    であるとは、次が成り立つことをいう:
    $\exists \; (V, T, \mu)$ s.t.
    \begin{description}
        \item[(E0)] $V$は有限次元$\R$-ベクトル空間である。
        \item[(E1)] $T \colon \calX \to V$は可測写像である。
        \item[(E2)] $\mu$は$\calX$上の$\sigma$-有限測度であり、
            $\forall \; p \in \calP$に対し$p \ll \mu$をみたす。
        \item[(E3)] $\forall \; p \in \calP$に対し、
            $\exists \; \theta \in V^\vee$ s.t.
            \begin{equation}
                \dd[p]{\mu}(x)
                    = \frac{
                        \exp \langle \theta, T(x) \rangle
                    }{
                        \int_\calX \exp \langle \theta, T(y) \rangle \, \mu(dy)
                    }
                    \quad
                    \text{$\mu$-a.e. $x \in \calX$}
            \end{equation}
            である。
            ただし$\langle \cdot, \cdot \rangle$は
            自然なペアリング$V^\vee \times V \to \R$である。
    \end{description}
    さらに次のように定める:
    \begin{itemize}
        \item $(V, T, \mu)$を$\calP$の
            \term{実現}[representation]
            {実現}[じつげん]
            という。
            \begin{itemize}
                \item $V$の次元を$(V, T, \mu)$の
                    \term{次元}[dimension]{次元}[じげん]
                    という。
                \item $T$を$(V, T, \mu)$の
                    \term{十分統計量}[sufficient statistic]
                    {十分統計量}[じゅうぶんとうけいりょう]
                    という。
                \item $\mu$を$(V, T, \mu)$の
                    \term{基底測度}[base measure]
                    {基底測度}[きていそくど]
                    という。
            \end{itemize}
        \item 集合$\Theta_{(V, T, \mu)}$
            \begin{equation}
                \Theta_{(V, T, \mu)}
                    \coloneqq \mybrace{
                        \theta \in V^\vee
                        \;\Big|\;
                        \int_\calX \exp \langle \theta, T(y) \rangle \, \mu(dy) < +\infty
                    }
            \end{equation}
            を$(V, T, \mu)$の
            \term{自然パラメータ空間}[natural parameter space]
            {自然パラメータ空間}[しぜんぱらめーたくうかん]
            という。
        \item 関数$\psi \colon \Theta_{(V, T, \mu)} \to \R,$
            \begin{equation}
                \psi(\theta)
                    \coloneqq
                    \log \int_\calX \exp \langle \theta, T(y) \rangle \, \mu(dy)
            \end{equation}
            を$(V, T, \mu)$の
            \term{対数分配関数}[log-partition function]
            {対数分配関数}[たいすうぶんぱいかんすう]
            という。
    \end{itemize}
\end{definition}

\begin{proposition}[自然パラメータ空間は凸集合]
    $\Theta_{(T, \mu)}$は$\R^m$の凸集合である。
    \TODO{$V$に修正}
\end{proposition}

\begin{proof}
    表記の簡略化のため$\Theta \coloneqq \Theta_{(T, \mu)}$とおく。
    $\theta, \theta' \in \Theta, \; t \in (0, 1)$とし、
    $(1 - t) \theta + t \theta' \in \Theta$を示せばよい。
    そこで$p \coloneqq \frac{1}{1 - t}, \; q \coloneqq \frac{1}{t}$とおくと、
    $p, q \in (1, +\infty)$であり、
    $\frac{1}{p} + \frac{1}{q} = (1 - t) + t = 1$であり、
    $e^{(1 - t)\langle \theta, T(x) \rangle} \in L^p(\calX, \mu)$かつ
    $e^{t \langle \theta', T(x) \rangle} \in L^q(\calX, \mu)$だから、
    H\"older の不等式より
    \begin{alignat}{1}
        \int_\calX e^{\langle (1 - t) \theta + t \theta', T(x) \rangle} \, \mu(dx)
            &= \int_\calX
                e^{(1 - t) \langle \theta, T(x) \rangle}
                e^{t \langle \theta', T(x) \rangle}
                \, \mu(dx) \\
            &\le \myparen{
                \int_\calX
                e^{(1 - t) \langle \theta, T(x) \rangle p}
                \, \mu(dx)
            }^{1 / p}
            \myparen{
                \int_\calX
                e^{t \langle \theta, T(x) \rangle q}
                \, \mu(dx)
            }^{1 / q} \\
            &= \myparen{
                \int_\calX
                e^{\langle \theta, T(x) \rangle}
                \, \mu(dx)
            }^{1 / p}
            \myparen{
                \int_\calX
                e^{\langle \theta, T(x) \rangle}
                \, \mu(dx)
            }^{1 / q} \\
            &< +\infty
    \end{alignat}
    が成り立つ。
    したがって$(1 - t) \theta + t \theta' \in \Theta$である。
\end{proof}

\begin{example}[有限集合上の確率分布]
    \label[example]{ex:finite-set}
    \TODO{$V$に修正}
    $\calX = \{ 1, \dots, n \}$、$\gamma$を$\calX$上の数え上げ測度とする。
    $\calX$上の確率分布全体の集合$\calP(\calX)$が
    $\calX$上の指数型分布族であることを確かめる。
    $\delta^j \; (j = 1, \dots, n)$を点$j$での Dirac 測度とおく。
    任意の$P \in \calP(\calX)$に対し、
    \begin{equation}
        P(dk)
            \coloneqq \sum_{j = 1}^n a_j \delta^j(dk),
            \quad
            a_1, \dots, a_n \in \R_{> 0},
            \quad
            \sum_{j = 1}^n a_j = 1
    \end{equation}
    が成り立つから、
    $\delta_{jk} \; (j, k = 1, \dots, n)$を
    Kronecker のデルタとして
    \begin{alignat}{1}
        P(dk)
            &= \exp\myparen{
                \sum_{j = 1}^n (\log a_j) \delta_{jk}
            } \, \gamma(dk) \\
            &= \exp\myparen{
                \sum_{j = 1}^n \theta_j \delta_{jk}
            } \, \gamma(dk)
    \end{alignat}
    (ただし$\theta_j \coloneqq \log a_j$)
    と表せる。
    したがって$T \colon \calX \to \R^n, \;
        k \mapsto \up{t}(\delta_{1k}, \dots, \delta_{nk})$
    とおけば、
    $(T, \gamma)$を実現として
    $\calP(\calX)$は指数型分布族となることがわかる。
\end{example}

\begin{example}[正規分布族]
    \TODO{$V$に修正}
    $\calX = \R$、
    $\lambda$を$\R$上の Lebesgue 測度とする。
    $\calX$上の確率分布の集合
    \begin{equation}
        \calP \coloneqq \mybrace{
            P_{(\mu, \sigma^2)}(dx)
                = \frac{1}{\sqrt{2\pi\sigma^2}} \exp\myparen{
                    -\frac{(x - \mu)^2}{2\sigma^2}
                } \lambda(dx)
            \;\Big|\;
            \mu \in \R, \; \sigma^2 > 0
        }
    \end{equation}
    を\term{正規分布族}[family of normal distributions]
        {正規分布族}[せいきぶんぷぞく]
    という。
    このとき$\calP$が$\calX$上の指数型分布族であることを確かめる。
    任意の$P_{(\mu, \sigma^2)} \in \calP$に対し
    \begin{alignat}{1}
        P_{(\mu, \sigma^2)}(dx)
            &= \frac{1}{\sqrt{2\pi\sigma^2}} \exp\myparen{
                -\frac{(x - \mu)^2}{2\sigma^2}
            } \lambda(dx) \\
            &= \exp\myparen{
                -\frac{1}{2\sigma^2} (x^2 - 2\mu x + \mu^2)
                -\frac{1}{2} \log 2\pi\sigma^2
            } \lambda(dx) \\
            &= \exp\myparen{
                \begin{bmatrix}
                    \frac{\mu}{\sigma^2} & -\frac{1}{2\sigma^2}
                \end{bmatrix}
                \begin{bmatrix}
                    x \\ x^2
                \end{bmatrix}
                - \frac{\mu^2}{2\sigma^2}
                - \frac{1}{2} \log 2\pi\sigma^2
            } \lambda(dx) \\
            &= \exp\myparen{
                \begin{bmatrix}
                    \theta_1 & \theta_2
                \end{bmatrix}
                \begin{bmatrix}
                    x \\ x^2
                \end{bmatrix}
                + \frac{\theta_1^2}{4\theta_2}
                - \frac{1}{2} \log\myparen{-\frac{\pi}{\theta_2}}
            } \lambda(dx)
    \end{alignat}
    (ただし$\theta_1 \coloneqq \frac{\mu}{\sigma^2}, \;
        \theta_2 \coloneqq -\frac{1}{2\sigma^2}$)
    が成り立つから、
    $T \colon \calX \to \R^2, x \mapsto \up{t}(x, x^2)$
    とおけば、
    $(T, \lambda)$を実現として
    $\calP$は指数型分布族となることがわかる。
\end{example}

\begin{example}[Poisson 分布族]
    \TODO{$V$に修正}
    $\calX = \N$、
    $\gamma$を$\N$上の数え上げ測度とする。
    $\calX$上の確率分布の集合
    \begin{equation}
        \calP \coloneqq \mybrace{
            P_\lambda(dk)
                = \frac{\lambda^k}{k!} e^{-\lambda} \, \gamma(dk)
            \;\Big|\;
            \lambda > 0
        }
    \end{equation}
    を$P_\lambda$を\term{Poisson 分布族}[family of Poisson distributions]
        {Poisson 分布族}[Poisson ぶんぷぞく]
    という。
    このとき$\calP$が$\calX$上の指数型分布族であることを確かめる。
    任意の$P_\lambda \in \calP$に対し
    \begin{alignat}{1}
        P_\lambda(dk)
            &= \frac{\lambda^k}{k!} e^{-\lambda} \, \gamma(dk) \\
            &= \exp\myparen{
                k \log\lambda - \lambda
            } \frac{1}{k!} \, \gamma(dk) \\
            &= \exp\myparen{
                \theta k - e^\theta
            } \frac{1}{k!} \, \gamma(dk)
    \end{alignat}
    (ただし$\theta \coloneqq \log \lambda$)
    が成り立つから、
    $T \colon \calX \to \R, k \mapsto k$
    とおけば、
    $\myparen{ T, \frac{1}{k!} \gamma(dk) }$を実現として
    $\calP$は指数型分布族となることがわかる。
\end{example}

% ------------------------------------------------------------
%
% ------------------------------------------------------------
\section{最小次元実現}

\TODO{節の内容を整理する}

\begin{definition}[最小次元実現]
    実現$(V, T, \mu)$が
    $\calP$の実現のうちで次元が最小のものであるとき、
    $(V, T, \mu)$を$\calP$の
    \term{最小次元実現}[minimal representation]
        {最小次元実現}[さいしょうじげんじつげん]という。
\end{definition}

\begin{theorem}[「$\theta$が一意の実現」の存在]
    \TODO{「単射性条件」の言葉に修正}
    $\calX$を可測空間、
    $\calP \subset \calP(\calX)$を
    $\calX$上の指数型分布族とする。
    このとき、$\calP$の「$\theta$が一意の実現」が存在する。
\end{theorem}

\begin{proof}
    $(V, T, \mu)$は$\calP$の実現のうちで次元が最小のものであるとする。
    $(V, T, \mu)$の次元 ($m$とおく) が$0$ならば
    $V^\vee$は1点集合だから証明は終わる。

    以下$m \ge 1$の場合を考え、
    $(V, T, \mu)$が「$\theta$が一意の実現」であることを示す。
    背理法のために$(V, T, \mu)$が「$\theta$が一意の実現」でないこと、
    すなわちある$p_0 \in \calP$および
    $\theta_0, \theta_0' \in V^\vee, \; \theta_0 \neq \theta_0'$が存在して
    \begin{equation}
        \locallabel{eq:assumption}
        \exp\myparen{\langle \theta_0, T(x) \rangle - \psi(\theta_0)}
            = \dd[p_0]{\mu}(x)
            = \exp\myparen{\langle \theta_0', T(x) \rangle - \psi(\theta_0')}
            \qquad
            \text{$\mu$-a.e. $x \in \calX$}
    \end{equation}
    が成り立つことを仮定する。
    証明の方針としては、
    次元$m - 1$の実現$(V', T', \mu)$を具体的に構成することにより、
    $(V, T, \mu)$の次元$m$が最小であることとの矛盾を導く。

    さて、式\localcref{eq:assumption}を整理して
    \begin{equation}
        \langle \theta_0 - \theta_0', T(x) \rangle
            = \psi(\theta_0) - \psi(\theta_0')
            \qquad
            \text{$\mu$-a.e. $x \in \calX$}
    \end{equation}
    を得る。
    表記の簡略化のために
    $\theta_1 \coloneqq \theta_0 - \theta_0' \in V^\vee, \;
        r \coloneqq \psi(\theta_0) - \psi(\theta_0') \in \R$
    とおけば
    \begin{equation}
        \locallabel{eq:costant-pairing}
        \langle \theta_1, T(x) \rangle
            = r
            \qquad
            \text{$\mu$-a.e. $x \in \calX$}
    \end{equation}
    を得る。
    ここで
    $V' \coloneqq (\R\theta)^\top = \{ v \in V \mid \langle \theta, v \rangle = 0 \}$
    とおき、
    次の claim を示す。
    \begin{description}
        \item[Claim] ある可測写像$T' \colon \calX \to V'$および
            $v_0 \in V$が存在して
            $T(x) = T'(x) + v_0 \; (\text{$\mu$-a.e.$x$})$
            が成り立つ。
    \end{description}
    \begin{innerproof}
        いま背理法の仮定より$\theta_1 \neq 0$であるから、
        $\theta_1$を延長した$V^\vee$の基底$\theta_1, \dots, \theta_m$が存在する。
        このとき、$\theta_1, \dots, \theta_m$を双対基底に持つ
        $V$の基底$v_1, \dots, v_m$が存在する。
        この基底$v_1, \dots, v_m$に関する
        $T$の成分表示を
        $T(x) = \sum_{i = 1}^m T^i(x) v_i, \;
            T^i \colon \calX \to \R$とおくと、
        \localcref{eq:costant-pairing}より
        $T^1(x) = \langle \theta_1, T(x) \rangle = r \; (\text{$\mu$-a.e.$x$})$
        が成り立つ。
        そこで$v_0 \coloneqq rv_1 \in V$とおくと
        $\langle \theta_1, T(x) - v_0 \rangle = 0 \; (\text{$\mu$-a.e.$x$})$
        が成り立つから、
        可測写像$T' \colon \calX \to V'$を
        \begin{equation}
            T'(x) \coloneqq \begin{cases}
                T(x) - v_0 & (\langle \theta_1, T(x) - v_0 \rangle = 0) \\
                0 & (\text{otherwise})
            \end{cases}
        \end{equation}
        と定めることができる。
        この$T, v_0$が求めるものである。
    \end{innerproof}
    $(V', T', \mu)$が$\calP$の実現であることを示す。
    \cref{def:exponential-family}の条件(E0)-(E2)は明らかに成立しているから、
    あとは条件(E3)を確認すればよい。
    そこで$p \in \calP$とする。
    いま$(V, T, \mu)$が$\calP$の実現であることより、
    ある$\theta \in V^\vee$が存在して
    \begin{equation}
        \dd[p]{\mu}(x)
            = \frac{
                \exp\langle \theta, T(x) \rangle
            }{
                \int_{\calX} \exp\langle \theta, T(y) \rangle \, \mu(dy)
            }
            \qquad
            \text{$\mu$-a.e. $x \in \calX$}
    \end{equation}
    が成り立つ。
    $T', v_0$を用いて式変形すると、$\mu$-a.e.$x$に対し
    \begin{alignat}{1}
        \dd[p]{\mu}(x)
            &= \frac{
                \exp\myparen{
                    \langle \theta, T(x) \rangle
                }
            }{
                \int_{\calX} \exp\myparen{
                    \langle \theta, T(x) \rangle
                } \, \mu(dy)
            } \\
            &= \frac{
                \exp\myparen{
                    \langle \theta, T'(x) \rangle
                    + \langle \theta, v_0 \rangle
                }
            }{
                \int_{\calX} \exp\myparen{
                    \langle \theta, T'(x) \rangle
                    + \langle \theta, v_0 \rangle
                } \, \mu(dy)
            } \\
            &= \frac{
                \exp\myparen{
                    \langle \theta, T'(x) \rangle
                }
            }{
                \int_{\calX} \exp\myparen{
                    \langle \theta, T'(x) \rangle
                } \, \mu(dy)
            }
    \end{alignat}
    が成り立つ。
    したがって$(V', T', \mu)$は条件(E3)も満たし、
    $\calP$の実現であることがいえた。
    $(V', T', \mu)$は次元$m - 1$だから
    $(V, T, \mu)$の次元$m$の最小性に矛盾する。
    背理法より$(V, T, \mu)$は$\calP$の「$\theta$が一意の実現」である。
\end{proof}

\begin{theorem}[極小実現の性質]
    \TODO{$V$に修正}
    $\calX$を可測空間、
    $\calP \subset \calP(\calX)$を
    $\calX$上の指数型分布族、
    $(T, \mu)$を$\calP$の次元$m$の実現とする。
    このとき、
    $(T, \mu)$が極小実現ならば、
    $\langle u, T(x) \rangle$が$\mu$-a.e. 定数であるような
    $u \in \R^m$は$u = 0$のみである。
\end{theorem}

\begin{proof}
    $(T, \mu)$を$\calP$の極小実現とする。
    背理法のため、ある$u \neq 0$が存在して
    $\langle u, T(x) \rangle$が$\calX$上$\mu$-a.e. 定数であると仮定しておく。
    $p \in \calP$とし、
    \cref{def:exponential-family}の条件(E3)の
    $\theta \in \R^m$をひとつ選ぶと、
    \begin{alignat}{1}
        \dd[p]{\mu}(x)
            &= \frac{
                e^{\langle \theta, T(x) \rangle}
            }{
                \int_{\calX} e^{\langle \theta, T(y) \rangle} \, \mu(dy)
            } \\
            &= \frac{
                e^{\langle \theta, T(x) \rangle}
            }{
                \int_{\calX} e^{\langle \theta, T(y) \rangle} \, \mu(dy)
            }
            \cdot \frac{
                e^{\langle u, T(x) \rangle}
            }{
                e^{\langle u, T(x) \rangle}
            } \\
            &= \frac{
                e^{\langle \theta + u, T(x) \rangle}
            }{
                \int_{\calX}
                e^{\langle \theta, T(y) \rangle}
                e^{\langle u, T(x) \rangle}
                \, \mu(dy)
            } \\
            &= \frac{
                e^{\langle \theta + u, T(x) \rangle}
            }{
                \int_{\calX}
                e^{\langle \theta, T(y) \rangle}
                e^{\langle u, T(y) \rangle}
                \, \mu(dy)
            } \\
            &= \frac{
                e^{\langle \theta + u, T(x) \rangle}
            }{
                \int_{\calX}
                e^{\langle \theta + u, T(y) \rangle}
                \, \mu(dy)
            }
    \end{alignat}
    を得る。
    したがって$\theta + u$も
    \cref{def:exponential-family}の条件(E3)を満たすが、
    いま$u \neq 0$より$\theta + u \neq \theta$だから、
    $(T, \mu)$が$\calP$の極小実現であることに反する。
    背理法より定理が示された。
\end{proof}

\begin{example}[有限集合上の確率分布族]
    \cref{ex:finite-set}の$(T, \gamma)$は
    $\calP(\calX)$の極小実現である。
    実際、任意の$P \in \calP(\calX)$に対し、
    $\theta_j$は
    $\theta_j = \log P(\{ j \}) \; (j = 1, \dots, n)$として
    一意に決まる。
\end{example}

\begin{proposition}
    \label[proposition]{prop:as-a}
    \TODO{上の命題とあわせる}
    $(V, T, \mu)$に関する次の条件は同値である:
    \begin{enumerate}
        \item $\langle \theta, T(x) \rangle$が
            $\calX$上$\mu$-a.e.定数であるような
            $\theta \in V^\vee$は$\theta = 0$のみである。
        \item 各$p \in \calP$に対し、
            \cref{def:exponential-family}の条件(E3)をみたす$\theta \in V^\vee$は
            ただひとつである。
    \end{enumerate}
\end{proposition}

\begin{proof}
    \uline{(2) \Rightarrow (1)} \quad
    前回示した。

    \uline{(1) \Rightarrow (2)} \quad
    $\theta, \theta' \in V^\vee$が
    \cref{def:exponential-family}の条件(E3)をみたすとすると、
    \begin{equation}
        e^{\langle \theta, T(x) \rangle - \psi(\theta)}
            = \dd[p]{\mu}(x)
            = e^{\langle \theta', T(x) \rangle - \psi(\theta')}
            \qquad
            \text{$\mu$-a.e.$x \in \calX$}
    \end{equation}
    が成り立つ。式を整理して
    \begin{equation}
        \langle \theta - \theta', T(x) \rangle
            = \psi(\theta) - \psi(\theta')
            \qquad
            \text{$\mu$-a.e.$x \in \calX$}
    \end{equation}
    が成り立つ。
    したがって(1)より$\theta = \theta'$である。
\end{proof}


本節の目標は、
最小次元実現の間のアファイン変換の一意存在を述べた
\cref{thm:transformation-between-representations}
の証明である。
本節では、
定理などのステートメントを簡潔にするために圏の言葉を用いる。

\begin{propdef}
    次のデータにより圏が定まる:
    \begin{itemize}
        \item 対象: $\calP$の実現$(V, T, \mu)$全体
        \item 射: $(V, T, \mu)$から$(V', T', \mu')$への射は、
            $V$から$V'$への全射アファイン写像
            $(L, b) \; (L \in \Lin(V, V'), \; b \in V')$
            であって
            $T'(x) = L(T(x)) + b \; \text{$\mu$-a.e.$x$}$をみたすもの
        \item 合成: アファイン写像の合成
            $(L, b) \circ (K, c) = (LK, Lc + b)$
    \end{itemize}
    この圏を$\bfC_\calP$と書く。
\end{propdef}

\begin{proof}
    示すべきことは、射の合成が射であること、恒等射の存在、結合律の3点である。
    射の合成が射であることは、
    全射と全射の合成が全射であることと、
    $\mu$と$\mu'$が互いに絶対連続であることから従う。
    また、$(V, T, \mu)$の恒等射は明らかに恒等写像$(\id_V, 0)$であり、
    結合律はアファイン写像の合成の結合律より従う。
\end{proof}

最小次元実現を特徴づける2つの条件を導入する。

\begin{propdef}[条件A]
    $\calP$の実現$(V, T, \mu)$に関する次の条件は同値である:
    \begin{enumerate}
        \item $P \colon \Theta \to \calP(\calX)$は単射である。
        \item $\forall \theta \in V^\vee$
            に対し
            「
                $\myangle{\theta}{T(x)} = \text{const. $\mu$-a.e.$x$}
                \implies
                \theta = 0$
            」
            が成り立つ。
        \item $V$の任意の真アファイン部分空間$W$に対し、
            「$T(x) \in W \; \text{$\mu$-a.e.$x$}$でない」
            が成り立つ。
    \end{enumerate}
    これらの条件が成り立つとき、
    $(V, T, \mu)$は\termsilent{条件A}をみたすという。
\end{propdef}

\begin{proof}
    (1) $\iff$ (2) は\url{0502_資料.pdf}の命題2.2で示した。
    (2) $\iff$ (3) は\url{0523_コメント.pdf}の命題0.4に記した。
\end{proof}

\begin{definition}[条件B]
    $\calP$の実現$(V, T, \mu)$に関する条件
    \begin{enumerate}
        \item $\Theta^\calP$は
            $V^\vee$を affine span する。
    \end{enumerate}
    が成り立つとき、
    $(V, T, \mu)$は\termsilent{条件B}をみたすという。
\end{definition}

条件Aは射の一意性を保証する。

\begin{proposition}[条件Aをみたす対象からの射の一意性]
    \label[proposition]{proposition:uniqueness-of-morphism}
    $(V, T, \mu), (V', T', \mu')$を$\bfC_\calP$の対象とする。
    このとき、
    $(V, T, \mu)$が条件Aをみたすならば、
    $(V, T, \mu)$から$(V', T', \mu')$への射は一意である。
\end{proposition}

\begin{proof}
    $(L, b), (K, c)$を$(V, T, \mu)$から$(V', T', \mu')$への射とする。
    射の定義より
    \begin{equation}
        \begin{cases}
            T'(x) = L(T(x)) + b & \text{$\mu$-a.e.$x$} \\
            T'(x) = K(T(x)) + c & \text{$\mu$-a.e.$x$}
        \end{cases}
    \end{equation}
    が成り立つから、2式を合わせて
    \begin{equation}
        (K - L)(T(x)) = b - c \qquad \text{$\mu$-a.e.$x$}
    \end{equation}
    となる。
    そこで基底を固定して
    成分ごとに$(V, T, \mu)$の条件A(2)を適用すれば、
    $K = L$を得る。
    よって
    上式で$K = L$として
    $b = c \; \text{$\mu$-a.e.}$
    したがって$b = c$を得る。
    以上より$(L, b) = (K, c)$である。
\end{proof}

射が存在するための十分条件を調べる。

\begin{proposition}[条件A, Bをみたす対象への射の存在]
    \label[proposition]{proposition:affine-map-between-representations}
    $(V, T, \mu)$を$\bfC_\calP$の対象とする。
    このとき、
    $(V, T, \mu)$が 条件Aと条件Bをみたすならば、
    任意の対象\xcancel{$(V', T', \mu)$}$(V', T', \mu')$から$(V, T, \mu)$への射が存在する。
\end{proposition}

この命題の証明には次の補題を用いる。

\begin{lemma}
    \label[lemma]{lemma:pairings_and_log_partition}
    $(V, T, \mu), (V', T', \mu')$を$\bfC_\calP$の対象とし、
    $\theta \colon \calP \to \Theta^\calP$
    および
    $\theta' \colon \calP \to \Theta'^\calP$
    を
    $P, P'$の右逆写像とする。
    このとき、
    任意の$p, q \in \calP$に対し、
    \begin{equation}
        \begin{alignedat}{1}
            &\phantom{=}
                \myangle{\theta(p) - \theta(q)}{T(x)}
                - \psi(\theta(p)) + \psi(\theta(q))
                \\
            &=
                \myangle{\theta'(p) - \theta'(q)}{T'(x)}
                - \psi'(\theta'(p)) + \psi'(\theta'(q))
        \end{alignedat}
        \qquad
        \text{$\mu$-a.e.$x$}
    \end{equation}
    が成り立つ。
\end{lemma}

\begin{proof}
    $p, q \in \calP$を任意とすると、
    指数型分布族の定義と
    $\mu, \mu'$が互いに絶対連続であることより、
    $\mu$-a.e.$x$に対し
    \begin{equation}
        \begin{alignedat}{2}
            \dd[p]{\mu}(x)
                &=
                    \exp(\myangle{\theta(p)}{T(x)} - \psi(\theta(p))),
                    \qquad
            &\dd[p]{\mu'}(x)
                &=
                    \exp(\myangle{\theta'(p)}{T'(x)} - \psi'(\theta'(p)))
                \\
            \dd[q]{\mu}(x)
                &=
                    \exp(\myangle{\theta(q)}{T(x)} - \psi(\theta(q))),
                    \qquad
            &\dd[q]{\mu'}(x)
                &=
                    \exp(\myangle{\theta'(q)}{T'(x)} - \psi'(\theta'(q)))
        \end{alignedat}
    \end{equation}
    が成り立つ。
    さらに$p, q$が互いに絶対連続であることから、
    $\mu$-a.e.$x$に対し
    \begin{alignat}{2}
        \dd[p]{q}(x)
            &=
                \dd[p]{\mu}(x) \Bigg/ \dd[q]{\mu}(x)
            &&=
                \exp\mybrace{
                    \myangle{\theta(p) - \theta(q)}{T(x)}
                    - \psi(\theta(p)) + \psi(\theta(q))
                }
                \\
        \dd[p]{q}(x)
            &=
                \dd[p]{\mu'}(x) \Bigg/ \dd[q]{\mu'}(x)
            &&=
                \exp\mybrace{
                    \myangle{\theta'(p) - \theta'(q)}{T'(x)}
                    - \psi'(\theta'(p)) + \psi'(\theta'(q))
                }
    \end{alignat}
    が成り立つ。
    $\log$をとって補題の主張の等式を得る。
\end{proof}

\begin{proof}[\cref{proposition:affine-map-between-representations}の証明]
    \uline{Step 0: $V, V^\vee$の基底を選ぶ} \quad
    $(V, T, \mu)$の条件Bより、
    $V^\vee$のあるアファイン基底
    $a^i \in \Theta^\calP \; (i = 0, \dots, m)$
    が存在する。
    そこで
    $e^i \coloneqq a^i - a^0 \in V^\vee \; (i = 1, \dots, m)$
    とおくとこれは$V^\vee$の基底である。
    さらに$e^i$の双対基底を$V$の元と同一視したものを
    $e_i \in V \; (i = 1, \dots, m)$とおいておく。

    \uline{Step 1: 射$(L, b)$の構成} \quad
    $P, P'$の右逆写像
    $\theta \colon \calP \to \Theta^\calP$
    および
    $\theta' \colon \calP \to \Theta'^\calP$
    をひとつずつ選んで
    $p^i \coloneqq P(a^i) \in \calP \; (i = 0, \dots, m)$とおき、
    $(L, b)$を次のように定める:
    \begin{alignat}{1}
        &L \colon V' \to V,
            \quad
            t' \mapsto
                \myangle{\theta'(p^i) - \theta'(p^0)}{t'} e_i
            \\
        &b \coloneqq
            \mybrace{
                \psi(\theta(p^i)) - \psi(\theta(p^0))
                - \psi'(\theta'(p^0)) + \psi'(\theta'(p^0))
            } e_i
            \in V
    \end{alignat}
    示すべきことは、
    \xcancel{すべての$p \in \calP$に対し}
    \begin{equation}
        T(x) = L(T'(x)) + b
            \quad
            \text{$\mu'$-a.e.$x$}
    \end{equation}
    が成り立つことと、
    $(L, b)$が全射となることである。

    \uline{Step 2: $T(x) = L(T'(x)) + b$の証明} \quad
    各$i = 1, \dots, m$に対し、
    \cref{lemma:pairings_and_log_partition}より
    \begin{equation}
        \begin{alignedat}{1}
            &\phantom{=}
                \myangle{\theta(p^i) - \theta(p^0)}{T(x)}
                - \psi(\theta(p^i))
                + \psi(\theta(p^0))
                \\
            &=
                \myangle{\theta'(p^i) - \theta'(p^0)}{T'(x)}
                - \psi'(\theta'(p^i))
                + \psi'(\theta'(p^0))
        \end{alignedat}
        \qquad
        \text{$\mu'$-a.e.$x$}
        \locallabel{eq:1}
    \end{equation}
    となる。
    ここで$(V, T, \mu)$の条件A (1)より
    $\theta(p^i) = a^i$
    が成り立つから、
    \localcref{eq:1}より
    \begin{equation}
        \begin{alignedat}{1}
            \myangle{a^i - a^0}{T(x)}
                &=
                    \myangle{\theta'(p^i) - \theta'(p^0)}{T'(x)}
                    \\
                &\qquad
                    + \psi(\theta(p^i))
                    - \psi(\theta(p^0))
                    - \psi'(\theta'(p^i))
                    + \psi'(\theta'(p^0))
                    \qquad
                    \text{$\mu'$-a.e.$x$}
        \end{alignedat}
    \end{equation}
    したがって
    \begin{equation}
        T(x) = L(T'(x)) + b
            \qquad
            \text{$\mu'$-a.e.$x$}
    \end{equation}
    が成り立つ。

    \uline{Step 3: $(L, b)$が全射であることの証明} \quad
    $L$が全射であることをいえばよい。
    もし$L$が全射でなかったとすると、
    $T(x) = L(T'(x)) + b \in \Im L + b$
    が
    $\mu'$-a.e.$x$
    したがって
    $\mu$-a.e.$x$
    に対し成り立つことになるが、
    $\Im L + b$は$V$の真アファイン部分空間だから
    $(V, T, \mu)$の条件A (3)に反する。
    したがって$L$は全射である。
\end{proof}

各条件をみたさない場合にも、射が存在する。

\begin{lemma}[条件Aをみたさない対象からの射の存在]
    \label[lemma]{lemma:morphism-existence-a}
    $(V, T, \mu)$を$\bfC_\calP$の対象とする。
    このとき、
    $(V, T, \mu)$が条件Aをみたさないならば、
    $(V, T, \mu)$よりも次元の小さい
    ある対象$(V', T', \mu')$への
    射$(V, T, \mu) \to (V', T', \mu')$が存在する。
\end{lemma}

\begin{proof}
    末尾の付録に記した。
\end{proof}

\begin{lemma}[条件Bをみたさない対象からの射の存在]
    \label[lemma]{lemma:morphism-existence-b}
    $(V, T, \mu)$を$\bfC_\calP$の対象とする。
    このとき、
    $(V, T, \mu)$が条件Bをみたさないならば、
    $(V, T, \mu)$よりも次元の小さい
    ある対象$(V', T', \mu')$への
    射$(V, T, \mu) \to (V', T', \mu')$が存在する。
\end{lemma}

\begin{proof}
    末尾の付録に記した。
\end{proof}

以上の補題を用いて
最小次元実現の特徴づけが得られる。

\begin{theorem}[最小次元実現の特徴づけ]
    \label[theorem]{thm:characterization-of-minimal-representation}
    $\calP$の実現$(V, T, \mu)$に関する次の条件は同値である:
    \begin{enumerate}
        \item $(V, T, \mu)$は$\calP$の最小次元実現である。
        \item $(V, T, \mu)$は条件Aと条件Bをみたす。
    \end{enumerate}
\end{theorem}

\begin{proof}
    \uline{(1) \Rightarrow (2)} \quad
    最小次元実現$(V, T, \mu)$が条件A, Bのいずれかをみたさなかったとすると、
    \cref{lemma:morphism-existence-a,lemma:morphism-existence-b}
    よりとくに$(V, T, \mu)$よりも次元の小さい実現が存在することになり、
    矛盾が従う。

    \uline{(2) \Rightarrow (1)} \quad
    $(V, T, \mu)$が条件Aと条件Bをみたすとする。
    $\calP$の任意の実現
    $(V', T', \mu')$に対し、
    \cref{proposition:affine-map-between-representations}より
    全射線型写像$L: V' \to V$が存在するから、
    $\dim V \le \dim V'$である。
    したがって$V$は$\calP$の最小次元実現である。
\end{proof}

\begin{example}[正規分布族の最小次元実現]
    \cref{thm:characterization-of-minimal-representation}により、
    \url{0425_資料.pdf}の例3.2でみた正規分布族の例は
    最小次元実現であることがわかる。
    実際、
    $T(x) = \up{t}(x, x^2)$の像は
    $\R^2$のいかなる真アファイン部分空間にも
    a.e.で含まれることはないから、
    条件A (3)が成り立つ。
    また、$\Theta^\calP = \R \times \R_{< 0}$となることから
    条件Bも成り立つ。
\end{example}

本節の目標の定理を示す。

\begin{theorem}[最小次元実現の間のアファイン変換]
    $(V, T, \mu), (V', T', \mu')$が
    ともに最小次元実現ならば、
    $(V, T, \mu)$から$(V', T', \mu')$への射$(L, b)$がただひとつ存在する。
    さらに、$L$は線型同型写像である。
\end{theorem}

\begin{proof}
    \cref{proposition:uniqueness-of-morphism,proposition:affine-map-between-representations}
    より、射$(L, b) \colon (V, T, \mu) \to (V', T', \mu')$はただひとつ存在する。
    また、
    \cref{proposition:affine-map-between-representations}
    より存在する射$(V', T', \mu') \to (V, T, \mu)$をひとつ選んで
    $(K, c)$とおくと、
    合成射$(K, c) \circ (L, b), \; (L, b) \circ (K, c)$は
    \cref{proposition:uniqueness-of-morphism}より
    恒等射$(\id_V, 0), \; (\id_{V'}, 0)$に一致する。
    したがって$L$は線型同型写像である。
\end{proof}

同じことを圏の言葉を使わずに言い換えると次のようになる。

\begin{theorem}[最小次元実現の間のアファイン変換]
    \label[theorem]{thm:transformation-between-representations}
    $(V, T, \mu), (V', T', \mu')$を$\bfC_\calP$の対象とする。
    このとき、
    $(V, T, \mu), (V', T', \mu')$が
    ともに最小次元実現ならば、
    全射線型写像$L: V \to V'$と
    ベクトル$b \in V'$であって
    \begin{equation}
        \xcancel{T(x) = L(T'(x)) + b} \quad
        T'(x) = L(T(x)) + b
            \qquad
            \text{$\mu$-a.e.$x$}
            \label{eq:transformation-of-T}
    \end{equation}
    をみたすものがただひとつ存在する。
    さらに、$L$は線型同型写像である。
    \qed
\end{theorem}

\begin{corollary}[自然パラメータの変換]
    上の定理の状況で、
    さらに$\theta^0 \in V^\vee$であって
    \begin{equation}
        \xcancel{\theta'(p)
            = \up{t}L(\theta(p)) + \theta^0} \quad
        \theta(p)
            = \up{t}L(\theta'(p)) + \theta^0
            \qquad
            (\forall p \in \calP)
            \label{eq:transformation-of-theta}
    \end{equation}
    をみたすものがただひとつ存在する。
    ただし
    写像$\theta \colon \calP \to \Theta^\calP$
    および
    $\theta' \colon \calP \to \Theta'^\calP$は
    $P, P'$の$\Theta^\calP, \Theta'^\calP$上への制限の逆写像である。
\end{corollary}

\begin{proof}
    \uline{Step 1: 一意性} \quad
    $\theta^0$が
    $(V, T, \mu), (V', T', \mu')$に対し一意であることは
    $L, \theta, \theta'$の一意性より明らかである。

    \uline{Step 2: 存在} \quad
    $q \in \calP$をひとつ選んで
    $\theta^0
        \coloneqq
            - \up{t}L(\theta(q)) + \theta'(q) \in V^\vee$
    と定め、
    この$\theta^0$が
    \cref{eq:transformation-of-theta}をみたすことを示せばよい。
    そこで$p \in \calP$を任意とすると、
    \cref{lemma:pairings_and_log_partition}より
    \begin{equation}
        \begin{alignedat}{1}
            &\phantom{=}
                \myangle{\theta(p) - \theta(q)}{T(x)}
                - \psi(\theta(p)) + \psi(\theta(q))
                \\
            &=
                \myangle{\theta'(p) - \theta'(q)}{T'(x)}
                - \psi'(\theta'(p)) + \psi'(\theta'(q))
        \end{alignedat}
        \qquad
        \text{$\mu$-a.e.$x$}
    \end{equation}
    が成り立ち、
    さらに\cref{eq:transformation-of-T}より
    \begin{equation}
        \begin{alignedat}{1}
            &\phantom{=}
                \myangle{\theta(p) - \theta(q)}{L(T(x)) + b}
                - \psi(\theta(p)) + \psi(\theta(q))
                \\
            &=
                \myangle{\theta'(p) - \theta'(q)}{T'(x)}
                - \psi'(\theta'(p)) + \psi'(\theta'(q))
        \end{alignedat}
        \qquad
        \text{$\mu$-a.e.$x$}
    \end{equation}
    が成り立つから、
    式を整理して
    \begin{equation}
        \begin{alignedat}{1}
            &\phantom{=}
                \myangle{
                    \up{t}L(\theta(p) - \theta(q))
                    - (\theta'(p) - \theta'(q))
                }{T'(x)}
                \\
            &=
                - \myangle{\theta(p) - \theta(q)}{b}
                + \psi(\theta(p)) - \psi(\theta(q))
                - \psi'(\theta'(p)) + \psi'(\theta'(q))
        \end{alignedat}
        \qquad
        \text{$\mu$-a.e.$x$}
    \end{equation}
    となる。
    この右辺は$x$によらないから、
    $(V', T', \mu')$の条件A (2)より
    \begin{alignat}{2}
        &\phantom{\therefore} \quad&
            \up{t}L(\theta(p) - \theta(q))
                - \theta'(p) - \theta'(q)
                &= 0
            \\
        &\therefore \quad&
            \up{t}L(\theta(p)) - \theta'(p)
                &=
                    \up{t}L(\theta(q)) - \theta'(q)
                =
                    - \theta^0
            \\
        &\therefore \quad&
            \up{t}L(\theta(p)) + \theta^0
                &=
                    \theta'(p)
    \end{alignat}
    が成り立つ。
    $p \in \calP$は任意であったから、
    \cref{eq:transformation-of-theta}の成立が示された。
\end{proof}

% ------------------------------------------------------------
%
% ------------------------------------------------------------
\section{対数分配関数}

\TODO{一般化した命題を使って証明を修正する}

本節では
$\calX$を可測空間、
$\calP \subset \calP(\calX)$を$\calX$上の指数型分布族、
$(V, T, \mu)$を$\calP$の次元$m$の実現、
$\Theta \subset V^\vee$を自然パラメータ空間、
$\psi \colon \Theta \to \R$を対数分配関数とする。
$V^\vee$における$\Theta$の内部を$\Theta^\circ$と書くことにする。
さらに
関数$h \colon \calX \times \Theta \to \R$および
$\lambda \colon \Theta \to \R$を
\begin{alignat}{2}
    h(x, \theta)
        &\coloneqq e^{\langle \theta, T(x) \rangle}
        &&\quad ((x, \theta) \in \calX \times \Theta) \\
    \lambda(\theta)
        &\coloneqq \int_\calX h(x, \theta) \, \mu(dx)
        &&\quad (\theta \in \Theta)
\end{alignat}
と定める (つまり$\psi(\theta) = \log \lambda(\theta)$である)。

本節の目標は次の定理を示すことである。

\begin{theorem}[$\lambda$と$\psi$の\smooth 性と積分記号下の微分]
    \label[theorem]{thm:smoothness_of_lambda}
    $\varphi = (\theta_1, \dots, \theta_m) \colon \Theta^\circ \to \R^m$
    を$\Theta^\circ$上のチャートとする。
    このとき、
    任意の$k \in \Z_{\ge 1}, \;
        i_1, \dots, i_k \in \{ 1, \dots, m \}$
    に対し、
    \begin{equation}
        \label{eq:smoothness_of_lambda_1}
        \del_{i_k} \cdots \del_{i_1} \lambda(\theta)
            = \int_\calX
                \del_{i_k} \cdots \del_{i_1} h(x, \theta)
                \, \mu(dx)
            \quad (\theta \in \Theta^\circ)
    \end{equation}
    が成り立つ
    ($\del_i$は$\deldel{\theta_i} \in \Gamma(T\Theta^\circ)$の略記)。
    ただし、
    左辺の微分可能性および
    右辺の可積分性も定理の主張に含まれる。
    とくに$\lambda$および$\psi$は$\Theta^\circ$上の\smooth 関数である。
\end{theorem}

\cref{thm:smoothness_of_lambda}の証明には次の事実を用いる。

\begin{fact}[積分記号下の微分]
    \label[fact]{fact:diff-under-integral}
    $\calY$を可測空間、
    $\nu$を$\calY$上の測度、
    $I \subset \R$を開区間、
    $f \colon \calY \times I \to \R$を
    \begin{enumerate}[label=(\roman*)]
        \item 各$t \in I$に対し$f(\cdot, t) \colon \calY \to \R$が可測
        \item 各$y \in \calY$に対し$f(y, \cdot) \colon I \to \R$が微分可能
    \end{enumerate}
    をみたす関数とする。
    このとき、$f$に関する条件
    \begin{enumerate}
        \item 各$t \in I$に対し
            $f(\cdot, t) \in L^1(\calY, \nu)$である。
        \item ある$\nu$-可積分関数
            $\Phi \colon \calY \to \R$が存在し、
            すべての$t' \in I$に対し
            $\myabs{
                \deldel[f]{t}(y, t')
            } \le \Phi(y) \; \text{a.e.$y$}$
            である。
    \end{enumerate}
    が成り立つならば、
    関数$I \to \R, \; t \mapsto \int_\calY f(y, t) \, \nu(dy)$は微分可能で、
    \begin{equation}
        \deldel{t} \int_\calY f(y, t) \, \nu(dy)
            = \int_\calY \deldel[f]{t}(y, t) \, \nu(dy)
    \end{equation}
    が成り立つ。
    \qed
\end{fact}

\cref{thm:smoothness_of_lambda}の証明において最も重要なステップは、
\cref{fact:diff-under-integral}の前提が満たされることの確認である。
そのための補題を次に示す。

\begin{lemma}[優関数の存在]
    \label[lemma]{lemma:existence_of_dominant_function}
    $e^i \; (i = 1, \dots, m)$を$V^\vee$の基底とし、
    この基底が定める$\Theta^\circ$上のチャートを
    $\varphi = (\theta_1, \dots, \theta_m) \colon \Theta^\circ \to \R^m$
    とおく。
    このとき、
    任意の$k \in \Z_{\ge 1}, \;
        i_1, \dots, i_k \in \{ 1, \dots, m \}$
    に対し、次が成り立つ:
    \begin{enumerate}
        \item 任意の$\theta \in \Theta^\circ$に対し、
            関数
            $\del_{i_k} \cdots \del_{i_1} h(\cdot, \theta)
                \colon \calX \to \R$
            は$L^1(\calX, \mu)$に属する。
        \item 任意の$\theta \in \Theta^\circ$に対し、
            $\Theta^\circ$における$\theta$のある近傍$U$と、
            ある$\mu$-可積分関数$\Phi \colon \calX \to \R$が存在し、
            すべての$\theta' \in U$に対し
            $\myabs{
                \del_{i_k} \cdots \del_{i_1} h(x, \theta')
            } \le \Phi(x) \; \text{a.e.$x$}$
            が成り立つ。
    \end{enumerate}
\end{lemma}

\begin{proof}
    (1)は(2)より直ちに従うから、(2)を示す。
    そこで$\theta \in \Theta^\circ$を任意とする。
    補題の主張は座標$\theta_1, \dots, \theta_m$を
    平行移動して考えても等価だから、
    点$\theta$の座標は
    $\varphi(\theta) = 0 \in \R^m$
    であるとしてよい。

    \uline{Step 1: $U$の構成} \quad
    $\eps > 0$を十分小さく選び、
    $\R^m$内の閉立方体
    \begin{alignat}{1}
        A_{2\eps}
            \coloneqq
            \prod_{i = 1}^m [- 2\eps, 2\eps]
        \quad
        A_{\eps}
            \coloneqq
            \prod_{i = 1}^m [- \eps, \eps]
    \end{alignat}
    が$\varphi(\Theta^\circ)$に含まれるようにしておく。
    すると
    $U \coloneqq \varphi^{-1}(\Int A_{\eps})
        \subset \varphi(\Theta^\circ)$は
    $\theta$の近傍となるが、
    これが求める$U$の条件を満たすことを示す。

    \uline{Step 2: $h$の座標表示} \quad
    まず具体的な計算のために
    $h$の座標表示を求める。
    いま各$\theta' \in U$に対し
    \begin{equation}
        h(x, \theta')
            = \exp\langle \theta', T(x) \rangle
            = \exp\langle \theta_i(\theta') e^i, T(x) \rangle
            = \exp\myparen{\theta_i(\theta') T^i(x)}
    \end{equation}
    が成り立っている。
    ただし
        $T^i \colon \calX \to \R, \;
        x \mapsto \langle e^i, T(x) \rangle \;
        (i = 1, \dots, m)$
    とおいた。
    したがって
    \begin{equation}
        \locallabel{eq:partial-derivative-of-h}
        \del_{i_k} \cdots \del_{i_1} h(x, \theta')
            = T^{i_1}(x) \cdots T^{i_k}(x)
                \exp\myparen{\theta_i(\theta') T^i(x)}
    \end{equation}
    と表せることがわかる。

    \uline{Step 3: $\Phi$の構成} \quad
    $\Phi$を構成するため、
    式\localcref{eq:partial-derivative-of-h}の絶対値を上から評価する。
    表記の簡略化のため
    $t' \coloneqq (t'_1, \dots, t'_m)
        \coloneqq \varphi(\theta')
        \in \R^m$
    とおいておく。
    まず$\frac{k + 1}{\eps} \frac{\eps}{k + 1} = 1$より
    \begin{alignat}{1}
        \myabs{
            T^{i_1}(x) \cdots T^{i_k}(x)
            \exp\myparen{
                \sum_{i = 1}^m
                t'_i T^i(x)
            }
        }
            &=
                \myparen{
                    \frac{k + 1}{\eps}
                }^k
                \myparen{
                    \prod_{\alpha = 1}^k
                        \frac{\eps}{k + 1}
                        |T^{i_\alpha}(x)|
                }
                \exp\myparen{
                    \sum_{i = 1}^m
                    t'_i T^i(x)
                } 
                \locallabel{eq:estimate}
    \end{alignat}
    であり、$\prod$の部分を評価すると
    \begin{alignat}{1}
        \prod_{\alpha = 1}^k
            \frac{\eps}{k + 1}
            |T^{i_\alpha}(x)|
            &\le \prod_{\alpha = 1}^k
                \myparen{
                    \exp\myparen{
                        \frac{\eps}{k + 1}
                        T^{i_\alpha}(x)
                    }
                    + \exp\myparen{
                        - \frac{\eps}{k + 1}
                        T^{i_\alpha}(x)
                    }
                }
                \quad
                (\because s \le e^s + e^{-s} \; (s \in \R))
                \\
            &= \sum_{\sigma \in \{ \pm 1 \}^k}
                \exp\myparen{
                    \sum_{\alpha = 1}^k
                        \frac{\eps}{k + 1}
                        \sigma_\alpha
                        T^{i_\alpha}(x)
                }
                \quad
                (\because \text{式の展開})
                \locallabel{eq:estimate-part}
    \end{alignat}
    (ただし$\sigma_\alpha$は$\sigma$の第$\alpha$成分)
    となるから、
    式\localcref{eq:estimate}と式\localcref{eq:estimate-part}を合わせて
    \begin{alignat}{1}
        \localcref{eq:estimate}
            &\le
                C
                \sum_{\sigma \in \{ \pm 1 \}^k}
                    \exp\myparen{
                        \sum_{\alpha = 1}^k
                            \frac{\eps}{k + 1}
                            \sigma_\alpha
                            T^{i_\alpha}(x)
                    }
                \exp\myparen{
                    \sum_{i = 1}^m
                    t'_i T^i(x)
                }
                \\
            &=
                C
                \sum_{\sigma \in \{ \pm 1 \}^k}
                    \exp\myparen{
                        \sum_{\alpha = 1}^k
                            \frac{\eps}{k + 1}
                            \sigma_\alpha
                            T^{i_\alpha}(x)
                        + \sum_{i = 1}^m
                            t'_i T^i(x)
                    }
                \locallabel{eq:estimate-2}
    \end{alignat}
    となる。
    ただし$C \coloneqq \myparen{\frac{k + 1}{\eps}}^k \in \R_{> 0}$とおいた。
    ここで最終行の$\exp$の中身について、
    各$i = 1, \dots, m$に対し
    $T^i(x)$の係数を評価することで、
    ある$t'' \in A_{2\eps}$が存在して
    \begin{equation}
        \localcref{eq:estimate-2}
            =
                C
                \sum_{\sigma \in \{ \pm 1 \}^k}
                    \exp\myparen{
                        \sum_{i = 1}^m
                            t''_i T^i(x)
                    }
            =
                2^k C
                    \exp\myparen{
                        \sum_{i = 1}^m
                            t''_i T^i(x)
                    }
                \locallabel{eq:estimate-3}
    \end{equation}
    と表せることがわかる。
    そこで
    $|t''_i| \le 2\eps \; (i = 1, \dots, m)$より
    \begin{alignat}{1}
        \localcref{eq:estimate-3}
            &\le
                2^k C
                \prod_{i = 1}^m
                    \myparen{
                        \exp\myparen{
                            2\eps
                            T^i(x)
                        }
                        + \exp\myparen{
                            - 2\eps
                            T^i(x)
                        }
                    }
                \\
            &=
                2^k C
                \sum_{\tau \in \{ \pm 1 \}^m}
                    \exp\myparen{
                        \sum_{i = 1}^m
                            2\eps
                            \tau_i
                            T^i(x)
                    }
    \end{alignat}
    を得る。
    この右辺は
    ($t'$によらないから) $\theta'$によらない$\calX$上の関数であり、
    また$\sum$の各項が
    $2\eps \tau \in A_{2\eps}$ゆえに$\mu$-可積分だから
    式全体も$\mu$-可積分である。
    したがってこれが求める優関数である。
\end{proof}

目標の\cref{thm:smoothness_of_lambda}を証明する。

\begin{proof}[\cref{thm:smoothness_of_lambda}の証明.]
    \cref{thm:smoothness_of_lambda}のステートメントで
    与えられているチャート$\varphi = (\theta_1, \dots, \theta_m)$は
    ($V^\vee$の基底が定めるものとは限らない)
    任意のものであるが、
    実は定理の主張を示すには、
    $V^\vee$の基底をひとつ選び、
    その基底が定めるチャート
    $\wt{\varphi} = (\wt{\theta}_1, \dots, \wt{\theta}_m)$
    に対して定理の主張を示せば十分である。
    その理由は次である:
    \begin{itemize}
        \item 式\cref{eq:smoothness_of_lambda_1}の左辺の微分可能性は、
            $\lambda$が$C^\infty$であればよいから、
            チャート$\wt{\varphi}$で考えれば十分。
        \item 式\cref{eq:smoothness_of_lambda_1}の右辺の可積分性および
            式\cref{eq:smoothness_of_lambda_1}の等号の成立については、
            Leibniz 則より、
            $\lambda$の$\wt{\theta}_1, \dots, \wt{\theta}_m$に関する
            $k$回偏導関数が、
            $\lambda$の$\theta_1, \dots, \theta_m$に関する
            $k$回以下の偏導関数たちの
            ($x$によらない) $C^\infty(\Theta^\circ)$-係数の
            線型結合に書けることから従う。
    \end{itemize}
    そこで、以降$\varphi$は
    $V^\vee$の基底が定めるチャートとする。

    \cref{lemma:existence_of_dominant_function} (1)より、
    式\cref{eq:smoothness_of_lambda_1}の右辺の可積分性はわかっている。
    よって、残りの示すべきことは
    \begin{enumerate}[label=(\roman*)]
        \item 式\cref{eq:smoothness_of_lambda_1}の左辺の微分可能性
        \item 式\cref{eq:smoothness_of_lambda_1}の等号の成立
    \end{enumerate}
    の2点である。

    まず$k = 1, i_k = 1$の場合に(i), (ii)が成り立つことを示す。
    そのためには、
    $t = (t_1, \dots, t_m) \in \varphi(\Theta^\circ)$を任意に固定したとき、
    $t_1$を含む$\R$の十分小さな開区間$I$が存在して、
    関数
    \begin{equation}
        \locallabel{eq:h_restriction}
        g \colon \calX \times I \to \R,
            \quad
            (x, s) \mapsto h(x, \varphi^{-1}(s, t_2, \dots, t_m))
    \end{equation}
    が\cref{fact:diff-under-integral}の仮定(1), (2)をみたすことをいえばよい。

    いま
    $\varphi^{-1}(t) \in \Theta^\circ$だから、
    \cref{lemma:existence_of_dominant_function}(2)のいう
    $\Theta^\circ$における$\varphi^{-1}(t)$の近傍$U$と
    $\mu$-可積分関数$\Phi \colon \calX \to \R$が存在する。
    このとき$\varphi(U)$は$\R^m$における$t$の近傍となるから、
    $t_1$を含む$\R$の十分小さな開区間$I$が存在して
    \begin{equation}
        I \times \{ t_2 \} \times \cdots \times \{ t_m \}
            \subset \varphi(U)
    \end{equation}
    が成り立つ。
    この$I$を用いて定まる関数$g$が
    \cref{fact:diff-under-integral}の仮定(1), (2)をみたすことを確認する。

    まず\cref{lemma:existence_of_dominant_function}の結果(1)より、
    $g$は\cref{fact:diff-under-integral}の仮定(1)をみたす。
    また\cref{lemma:existence_of_dominant_function}の結果(2)より、
    $g$は\cref{fact:diff-under-integral}の仮定(2)をみたす。
    したがって$k = 1, i_k = 1$の場合について(i),(ii)が示された。

    同様にして$i_k = 2, \dots, m$の場合についても示される。
    以降、$k$に関する帰納法で、すべての$k \in \Z_{\ge 1}$および
    $i_1, \dots, i_k \in \{ 1, \dots, m \}$に対して示される。
    これで定理の証明が完了した。
\end{proof}

\cref{thm:smoothness_of_lambda}から次の系が従う。

\begin{corollary}
    $\varphi = (\varphi_1, \dots, \varphi_m) \colon \Theta^\circ \to \R^m$を
    $V^\vee$の基底が定めるチャートとする。
    また、各$\theta \in \Theta$に対し、
    $\calX$上の確率測度$P_\theta$を
    $P_\theta(dx)
        = e^{\langle \theta, T(x) \rangle - \psi(\theta)} \, \mu(dx)$
    と定める。
    このとき、
    任意の$k \in \Z_{\ge 1}, \;
        i_1, \dots, i_k \in \{ 1, \dots, m \}$
    に対し、
    \begin{equation}
        E_{P_\theta}[T^{i_k}(x) \cdots T^{i_1}(x)]
            = \frac{
                \del_{i_k} \cdots \del_{i_1} \lambda(\theta)
            }{
                \lambda(\theta)
            }
            \quad
            (\theta \in \Theta^\circ)
    \end{equation}
    が成り立つ。
    ただし、左辺の期待値の存在も系の主張に含まれる。
    \qed
\end{corollary}

% ------------------------------------------------------------
%
% ------------------------------------------------------------
\section{Fisher 計量}

Fisher 計量を定義する。

\begin{propdef}[Fisher 計量]
    $\psi$を$\Theta^\circ$上の\smooth 関数とみなすと、
    各$\theta \in \Theta^\circ$に対し
    $(\Hess \psi)_\theta
        \in T^{(0, 2)}_\theta \Theta^\circ$
    は
    $\Var_{P_\theta}[T]$に一致する。
    さらに$(V, T, \mu)$が条件Aをみたすならば、
    $\Hess \psi$は正定値である。

    したがって
    $(V, T, \mu)$が条件Aをみたすとき、
    $\Hess\psi$は
    $\Theta^\circ$上の Riemann 計量となり、
    これを$\psi$の定める
    \term{Fisher 計量}[Fisher metric]{Fisher 計量}[Fisher けいりょう]
    という。
\end{propdef}

\begin{proof}
    まず
    $(\Hess \psi)_\theta = \Var_{P_\theta}[T] \;
        (\theta \in \Theta^\circ)$
    を示す。
    $\Theta^\circ$上の$D$-アファイン座標
    $\theta^i \; (i = 1, \dots, m)$をひとつ選ぶと、
    \cref{prop:hessian_components}より、
    座標$\theta^i$に関する$\Hess \psi$の成分表示は
    $\Hess\psi
        = \frac{\del^2 \psi}{\del \theta^i \del \theta^j}
        \, d\theta^i \otimes d\theta^j$
    となる。
    ここで前回 (\url{0516_資料.pdf}) の系2.4より
    \begin{alignat}{1}
        \frac{\del^2 \psi}{\del \theta^i \del \theta^j}(\theta)
            &=
                \del_i \del_j \log \lambda(\theta)
                \\
            &=
                \del_i \myparen{
                    \frac{\del_j \lambda(\theta)}{\lambda(\theta)}
                }
                \\
            &=
                \frac{
                    \del_i \del_j \lambda(\theta)
                }{
                    \lambda(\theta)
                }
                -
                \frac{
                    \del_i \lambda(\theta)
                    \del_j \lambda(\theta)
                }{
                    \lambda(\theta)^2
                }
                \\
            &=
                E[T^i(x) T^j(x)]
                -
                E[T^i(x)]
                E[T^j(x)]
                \\
            &=
                E[
                    (T^i(x) - E[T^i(x)])
                    (T^j(x) - E[T^j(x)])
                ]
    \end{alignat}
    を得る。
    ただし$E[\cdot]$は$P_\theta$に関する期待値$E_{P_\theta}[\cdot]$の略記である。
    したがって
    $\Hess_\theta \psi = \Var_{P_\theta}[T]$
    が成り立つ。

    次に、$(V, T, \mu)$が条件Aをみたすとし、
    $\Hess\psi$が正定値であることを示す。
    すなわち、
    各$\theta \in \Theta^\circ$に対し
    $(\Hess\psi)_\theta$が正定値であることを示す。
    そのためには各$u \in V^\vee$に対し
    「$(\Hess\psi)_\theta(u, u) = 0$ならば$u = 0$」
    を示せばよいが、
    上で示したことと
    \cref{prop:expectation-variance-pairing}より
    \begin{equation}
        (\Hess\psi)_\theta(u, u)
            = (\Var_{P_\theta}[T])(u, u)
            = \langle u \otimes u, \Var_{P_\theta}[T] \rangle
            = \Var_{P_\theta}[\langle u, T(x) \rangle]
    \end{equation}
    と式変形できるから、
    $(\Hess\psi)_\theta(u, u) = 0$ならば
    \cref{prop:zero_variance_condition}より
    $\langle u, T(x) \rangle$は$\text{a.e.}$定数であり、
    したがって条件Aより$u = 0$となる。
    よって$(\Hess\psi)_\theta$は正定値である。
    したがって$\Hess\psi$は正定値である。
\end{proof}


% ------------------------------------------------------------
%
% ------------------------------------------------------------
\section{Amari-Chentsov テンソルと$\alpha$-接続}


\subsection{多様体構造と平坦アファイン接続}

\begin{propdef}[$\calP$が開であること]
    指数型分布族$\calP$に関し、次は同値である:
    \begin{enumerate}
        \item ある最小次元実現$(V, T, \mu)$に対し、
            $\Theta^\calP_{(V, T, \mu)}$は$V^\vee$で開である。
        \item すべての最小次元実現$(V, T, \mu)$に対し、
            $\Theta^\calP_{(V, T, \mu)}$は$V^\vee$で開である。
    \end{enumerate}
    $\calP$がこれらの同値な2条件をみたすとき、
    $\calP$は\termsilent{開}[open]であるという。
\end{propdef}

\begin{proof}
    (1) $\Rightarrow$ (2)は、
    \url{0606_資料.pdf}系1.13より、
    最小次元実現の真パラメータ空間がアファイン変換で写り合うことから従う。
    (2) $\Rightarrow$ (1)は
    最小次元実現が存在することから従う。
\end{proof}

以降、本節では$\calP$は開とする。

\begin{propdef}[$\calP$の自然な多様体構造]
    $\calP$上の多様体構造$\calU$であって
    次をみたすものがただひとつ存在する:
    \begin{itemize}
        \item $\calP$の任意の最小次元実現$(V, T, \mu)$に対し、
            $\calU$は全単射$\theta_{(V, T, \mu)}$により
            $\Theta^\calP_{(V, T, \mu)}$から$\calP$上に誘導された多様体構造に一致する。
    \end{itemize}
    この$\calU$を$\calP$の
    \termsilent{自然な多様体構造}
    という。
\end{propdef}

\begin{proof}
    \uline{Step 1: $\calU$の一意性} \quad
    $\calU$の存在を仮定すれば、
    最小次元実現をひとつ選ぶことで
    $\calU$が決まるから、
    $\calU$は一意である。

    \uline{Step 2: $\calU$の存在} \quad
    最小次元実現$(V, T, \mu)$をひとつ選び、
    $\theta \coloneqq \theta_{(V, T, \mu)}$とおき、
    $\theta$により
    $\Theta^\calP_{(V, T, \mu)}$から$\calP$上に誘導された多様体構造を$\calU$とおく。
    この$\calU$が求めるものであることを示せばよい。
    示すべきことは、
    $(V', T', \mu')$を最小次元実現とし、
    $\theta' \coloneqq \theta_{(V', T', \mu')}$とおき、
    $\calU'$を
    $\theta'$により
    $\Theta^\calP_{(V', T', \mu')}$から
    $\calP$上に誘導された多様体構造とするとき、
    恒等写像$\id \colon (\calP, \calU) \to (\calP, \calU')$が
    微分同相となることである。
    これは図式
    \begin{equation}
        \begin{tikzcd}
            (\calP, \calU)
                \ar{d}[swap]{\theta}
                \ar{r}{\id}
                &(\calP, \calU')
                    \ar{d}{\theta'}
                \\
            \Theta^\calP_{(V, T, \mu)}
                \ar{r}[swap]{F}
                &\Theta^\calP_{(V', T', \mu')}
        \end{tikzcd}
    \end{equation}
    の可換性と、
    $\theta, \theta', F$が微分同相であることから従う。
    ただし$F$とは、
    \url{0606_資料.pdf}系1.13より一意に存在する
    アファイン変換$V^\vee \to V'^\vee$の制限である。
\end{proof}

以降、本節では
$\calP$に自然な多様体構造が定まっているものとする。

\begin{propdef}[$\calP$上の自然な平坦アファイン接続]
    \label[propdef]{propdef:natural-flat-connection}
    $\calP$上の平坦アファイン接続$\nabla$であって
    次をみたすものがただひとつ存在する:
    \begin{itemize}
        \item $\calP$の任意の最小次元実現$(V, T, \mu)$に対し、
            $\Theta^\calP_{(V, T, \mu)}$上の
            標準的な平坦アファイン接続を$\wt{\nabla}$とおくと、
            $\nabla$は$\nabla = \theta_{(V, T, \mu)}^* \wt{\nabla}$をみたす。
    \end{itemize}
    この$\nabla$を$\calP$上の
    \termsilent{自然な平坦アファイン接続}
    という。
\end{propdef}

証明には次の補題を用いる。

\begin{lemma}[アファイン変換によるアファイン接続の引き戻し]
    \label[lemma]{lemma:pullback-affine-connection}
    $V, V'$を有限次元$\R$-ベクトル空間、
    $F \colon V \to V'$をアファイン変換、
    $\nabla, \nabla'$をそれぞれ$V, V'$上の標準的な平坦アファイン接続とする。
    このとき
    $F^* \nabla' = \nabla$が成り立つ。
\end{lemma}

\begin{proof}
    資料末尾の付録に記した。
\end{proof}

\begin{proof}[\cref{propdef:natural-flat-connection}の証明]
    \uline{Step 1: $\nabla$の一意性} \quad
    $\nabla$の存在を仮定すれば、
    最小次元実現をひとつ選ぶことで$\nabla$が決まるから、
    $\nabla$は一意である。

    \uline{Step 2: $\nabla$の存在} \quad
    最小次元実現$(V, T, \mu)$をひとつ選び、
    $\theta \coloneqq \theta_{(V, T, \mu)}$、
    $\Theta^\calP_{(V, T, \mu)}$上の
    標準的な平坦アファイン接続を$\wt{\nabla}$、
    $\nabla \coloneqq \theta^* \wt{\nabla}$と定める。
    この$\nabla$が求めるものであることを示せばよい。
    示すべきことは、
    $(V', T', \mu')$を最小次元実現とし、
    $\theta' \coloneqq \theta_{(V', T', \mu')}$、
    $\Theta^\calP_{(V', T', \mu')}$上の
    標準的な平坦アファイン接続を$\wt{\nabla}'$とおくとき、
    $\theta^* \wt{\nabla} = \theta'^* \wt{\nabla}'$が成り立つことである。
    そこで、
    \url{0606_資料.pdf}系1.13より一意に存在する
    アファイン変換$V^\vee \to V'^\vee$を$F$とおくと、
    \begin{alignat}{1}
        \theta'^* \wt{\nabla}'
            &=
                \theta^* F^* \wt{\nabla}'
                \quad
                (\text{$F$と$\theta, \theta'$の関係})
                \\
            &=
                \theta^* \wt{\nabla}
                \quad
                (\text{\cref{lemma:pullback-affine-connection}})
    \end{alignat}
    が成り立つ。
    したがって$\theta^* \wt{\nabla} = \theta'^* \wt{\nabla}'$が示された。
    よって$\nabla$は命題-定義の主張の条件をみたす。
\end{proof}

以降、本節では
$\calP$に自然な平坦アファイン接続$\nabla$が定まっているものとする。

\subsection{Fisher 計量}

\begin{propdef}[$\calP$上の Fisher 計量]
    \label[propdef]{propdef:Fisher-metric}
    $\calP$上の Riemann 計量$g$であって
    次をみたすものがただひとつ存在する:
    \begin{itemize}
        \item $\calP$の任意の最小次元実現$(V, T, \mu)$に対し、
            $\Theta^\calP_{(V, T, \mu)}$上の Fisher 計量を$\wt{g}$とおくと、
            $g = \theta_{(V, T, \mu)}^* \wt{g}$が成り立つ。
    \end{itemize}
    これを$\calP$上の
    \termsilent{Fisher 計量}
    という。
\end{propdef}

証明には次の補題を用いる。

\begin{lemma}
    \label[lemma]{lemma:relationship-between-Fisher-metrics}
    $(V, T, \mu), (V', T', \mu')$を
    $\calP$の最小次元実現とし、
    $\theta \coloneqq \theta_{(V, T, \mu)}, \;
        \theta' \coloneqq \theta_{(V', T', \mu')}$
    とおき、
    $\Theta^\calP_{(V, T, \mu)}, \;
        \Theta^\calP_{(V', T', \mu')}$上の
    Fisher 計量をそれぞれ$g, g'$とおき、
    \url{0606_資料.pdf}定理1.12より一意に存在する
    線型同型写像$V \to V'$を$L$とおく。
    このとき、
    各$p \in \calP$に対し
    $g_{\theta(p)} = (L \otimes L)(g'_{\theta'(p)})$
    が成り立つ。
\end{lemma}

\begin{proof}
    $L$は
    $T'(x) = L(T(x)) + \text{const.} \; \text{$\mu$-a.e.$x$}$
    をみたし、
    また
    各$p \in \calP$に対し
    $g_{\theta(p)} = \Var_{p}[T], \;
        g'_{\theta'(p)} = \Var_{p}[T']$
    が成り立つから、
    期待値と分散のペアリングの命題
    (\url{0523_資料.pdf}命題1.1)
    と同様の議論により補題の主張の等式が成り立つ。
\end{proof}

\begin{proof}[\cref{propdef:Fisher-metric}の証明]
    \uline{Step 1: $g$の一意性} \quad
    $g$の存在を仮定すれば、
    最小次元実現をひとつ選ぶことで
    $g$が決まるから、
    $g$は一意である。

    \uline{Step 2: $g$の存在} \quad
    最小次元実現$(V, T, \mu)$をひとつ選び、
    $\theta \coloneqq \theta_{(V, T, \mu)}$、
    $\Theta^\calP_{(V, T, \mu)}$上の Fisher 計量を$\wt{g}$とおき、
    $g \coloneqq \theta^* \wt{g}$と定める。
    この$g$が求めるものであることを示せばよい。
    示すべきことは、
    $(V', T', \mu')$を最小次元実現とし、
    $\theta' \coloneqq \theta_{(V', T', \mu')}$、
    $\Theta^\calP_{(V', T', \mu')}$上の Fisher 計量を$\wt{g}'$とおいて、
    $\theta^* g = \theta'^* g'$が成り立つことである。
    そこで
    \url{0606_資料.pdf}定理1.12より一意に存在する
    線型同型写像$V \to V'$を$L$とおくと、
    各$p \in \calP, \; u, v \in T_p\calP$に対し
    \begin{alignat}{1}
        (\theta^* g)_p(u, v)
            &=
                g_{\theta(p)} (d\theta_p(u), d\theta_p(v))
                \\
            &=
                \myangle{
                    g_{\theta(p)}
                }{
                    d\theta_p(u) \otimes d\theta_p(v)
                }
                \\
            &=
                \myangle{
                    (L \otimes L) g'_{\theta'(p)}
                }{
                    d\theta_p(u) \otimes d\theta_p(v)
                }
                \quad
                (\cref{lemma:relationship-between-Fisher-metrics})
                \\
            &=
                \myangle{
                    g'_{\theta'(p)}
                }{
                    \up{t}L \circ d\theta_p(u) \otimes \up{t}L \circ d\theta_p(v)
                }
                \\
            &=
                \myangle{
                    g'_{\theta'(p)}
                }{
                    d(\up{t}L \circ \theta)_p (u) \otimes d(\up{t}L \circ \theta)_p (v)
                }
                \\
            &=
                \myangle{
                    g'_{\theta'(p)}
                }{
                    d\theta'_p (u) \otimes d\theta'_p (v)
                }
                \quad
                (\text{$L$と$\theta, \theta'$の関係})
                \\
            &=
                g'_p (d\theta'_p(u), d\theta'_p(v))
                \\
            &=
                (\theta'^* g')_p(u, v)
    \end{alignat}
    が成り立つ。
    したがって$\theta^* g = \theta'^* g'$が示された。
    よって$g$は命題-定義の主張の条件をみたす。
\end{proof}

以降、本節では
$\calP$に Fisher 計量$g$が定まっているものとする。

\subsection{Amari-Chentsov テンソルと$\alpha$-接続}

\begin{definition}[Amari-Chentsov テンソル]
    $\calP$上の$(0, 3)$-テンソル場$S$を
    $S \coloneqq \nabla g$で定め、
    これを$\calP$上の
    \term{Amari-Chentsov テンソル}[Amari-Chentsov tensor]
        {Amari-Chentsov テンソル}[Amari-Chentsov テンソル]
    という。
    また、
    $\calP$上の$(1, 2)$-テンソル場$A$を
    次の関係式により定める:
    \begin{equation}
        g(A(X, Y), Z)
            = S(X, Y, Z)
            \quad
            (\forall X, Y, Z \in \Gamma(T\calP))
    \end{equation}

    以降、「Amari-Chentsov テンソル」を
    「ACテンソル」と略記することがある。
\end{definition}

以降、本節では
$\calP$に Amari-Chentsov テンソル$S$が定まっているものとする。

\begin{proposition}[ACテンソルの成分]
    \label[proposition]{prop:components-of-AC-tensor}
    $(V, T, \mu)$を$\calP$の最小次元実現、
    $\Theta^\calP \coloneqq \Theta^\calP_{(V, T, \mu)}, \;
        \theta \coloneqq \theta_{(V, T, \mu)}$、
    $(V, T, \mu)$の対数分配関数を$\psi$とおく。
    このとき、
    $\calP$上の任意の
    $\nabla$-アファイン座標
    $x \coloneqq (x^1, \dots, x^m) \colon \calP \to \R^m$に対し、
    $\varphi \coloneqq (\varphi^1, \dots, \varphi^m)
        \coloneqq x \circ \theta^{-1} \colon \Theta^\calP \to \R^m$とおくと、
    $S$の成分は
    \begin{equation}
        S_{ijk}(p)
            = \frac{
                \del^3 \psi
            }{
                \del \varphi^i \del \varphi^j \del \varphi^k
            }(\theta(p))
            = E_p\mybracket{
                (T_i - E_p[T_i])
                (T_j - E_p[T_j])
                (T_k - E_p[T_k])
            }
    \end{equation}
    をみたす。
    ただし$T_i \; (i = 1, \dots, m)$とは、
    同一視$V = V^{\vee\vee} = T^\vee_{\theta(p)} \Theta^\calP$により
    $d\varphi^i \; (i = 1, \dots, m)$を$V$の基底とみなしたときの
    $T$の成分である。
\end{proposition}

\begin{proof}
    左側の等号と右側の等号についてそれぞれ示す。

    \uline{Step 1: 左側の等号} \quad
    $\Theta^\calP$上の標準的な平坦アファイン接続を$\wt{\nabla}$とおき、
    $\psi$の定める$\Theta^\calP$上の Fisher 計量を$\wt{g}$とおくと、
    \begin{alignat}{1}
        S\myparen{
            \deldel{x^i},
            \deldel{x^j},
            \deldel{x^k}
        }
            &=
                \myparen{
                    \nabla_{\deldel{x^i}}
                    g
                }
                \myparen{
                    \deldel{x^j},
                    \deldel{x^k}
                }
                \\
            &=
                \myparen{
                    \myparen{
                        \theta^* \wt{\nabla}
                    }_{\deldel{x^i}}
                    \myparen{
                        \theta^* \wt{g}
                    }
                }
                \myparen{
                    \deldel{x^j},
                    \deldel{x^k}
                }
                \\
            &=
                \myparen{
                    \theta^{-1}_*
                    \myparen{
                        \wt{\nabla}_{\theta_* \deldel{x^i}}
                        \wt{g}
                    }
                }
                \myparen{
                    \deldel{x^j},
                    \deldel{x^k}
                }
                \\
            &=
                \myparen{
                    \wt{\nabla}_{\theta_* \deldel{x^i}}
                    \wt{g}
                }
                \myparen{
                    d\theta\myparen{
                        \deldel{x^j}
                    },
                    d\theta\myparen{
                        \deldel{x^k}
                    }
                }
                \\
            &=
                \myparen{
                    \wt{\nabla}_{\deldel{\varphi^i}}
                    \wt{g}
                }
                \myparen{
                    \deldel{\varphi^j},
                    \deldel{\varphi^k}
                }
                \\
            &=
                \myparen{
                    \deldel{\varphi^i}
                    \myparen{
                        \frac{\del^2 \psi}{\del \varphi^l \del \varphi^n}
                    }
                    d\varphi^l d\varphi^n
                }
                \myparen{
                    \deldel{\varphi^j},
                    \deldel{\varphi^k}
                }
                \quad
                (\text{$\varphi$は$\wt{\nabla}$-アファイン座標})
                \\
            &=
                \frac{\del^3 \psi}{\del \varphi^i \del \varphi^j \del \varphi^k}
    \end{alignat}
    となるから、命題の主張の左側の等号が従う。

    \uline{Step 2: 右側の等号} \quad
    「$E_p$」の下付きの$p$を省略して書けば、
    直接計算より
    \begin{alignat}{1}
        &\phantom{=}
            E[(T_i - E[T_i])(T_j - E[T_j])(T_k - E[T_k])]
            \\
        &=
            E[T_i T_j T_k]
            - E[T_i] E[T_j T_k]
            - E[T_j] E[T_k T_i]
            - E[T_k] E[T_i T_j]
            + 2 E[T_i] E[T_j] E[T_k]
            \locallabel{eq:1}
    \end{alignat}
    が成り立つ。
    一方、
    $\lambda \coloneqq \exp \psi$とおき、
    $\deldel{\varphi^i}$を$\del_i$と略記すれば、
    直接計算より
    \begin{alignat}{1}
        \frac{\del^3 \psi}{\del \varphi^i \del \varphi^j \del \varphi^k}
            &=
                \del_i
                \del_j
                \del_k
                \log \lambda
                \\
            &=
                \frac{\del_i \del_j \del_k \lambda}{\lambda}
                - \frac{(\del_i \lambda) (\del_j \del_k \lambda)}{\lambda^2}
                - \frac{(\del_j \lambda) (\del_k \del_i \lambda)}{\lambda^2}
                - \frac{(\del_k \lambda) (\del_i \del_j \lambda)}{\lambda^2}
                + 2 \frac{(\del_i \lambda) (\del_j \lambda) (\del_k \lambda)}{\lambda^3}
    \end{alignat}
    が成り立つ。
    この右辺を
    \url{0516_資料.pdf}系2.4により期待値の形で表せば
    式\localcref{eq:1}に一致するから、
    命題の主張の右側の等号が従う。
\end{proof}

\begin{definition}[$\alpha$-接続]
    $\alpha \in \R$とする。
    $\calP$上のアファイン接続$\nabla^{(\alpha)}$を
    次の関係式により定める:
    \begin{equation}
        g(\nabla^{(\alpha)}_X Y, Z)
            =
                g(\nabla^{(g)}_X Y, Z)
                -
                \frac{\alpha}{2} S(X, Y, Z)
                \qquad
                (X, Y, Z \in \Gamma(T\calP))
    \end{equation}
    この$\nabla^{(\alpha)}$を
    $(g, S)$の定める
    \term{$\alpha$-接続}[$\alpha$-connection]
        {$\alpha$-接続}[alphaせつぞく]
    という。
    とくに$\alpha = 1, -1$の場合をそれぞれ
    \term{e-接続}[e-connection]
        {e-接続}[eせつぞく]、
    \term{m-接続}[m-connection]
        {m-接続}[mせつぞく]
    という。
\end{definition}

\begin{proposition}[$\nabla^{(g)}, \nabla^{(\alpha)}$のACテンソルによる表示]
    \label[proposition]{prop:connections_by_AC_tensor}
    $\calP$上の任意の$\nabla$-アファイン座標に関し、
    $\nabla^{(g)}$および$\nabla^{(\alpha)}$の
    接続係数は次をみたす:
    \begin{enumerate}
        \item
            \begin{equation}
                {\Gamma^{(g)}}_{ij}^k
                    = \frac{1}{2} A_{ij}^k,
                    \quad
                {\Gamma^{(g)}}_{ijk}
                    = \frac{1}{2} S_{ijk}
                \label{eq:Gamma_g}
            \end{equation}
        \item すべての$\alpha \in \R$に対し
            \begin{equation}
                {\Gamma^{(\alpha)}}_{ij}^k
                    = \frac{1 - \alpha}{2} A_{ij}^k,
                    \quad
                {\Gamma^{(\alpha)}}_{ijk}
                    = \frac{1 - \alpha}{2} S_{ijk}
                \label{eq:Gamma_alpha}
            \end{equation}
            とくに$\alpha = 1$のとき
            ${\Gamma^{(1)}}_{ij}^k = 0, \;
                {\Gamma^{(1)}}_{ijk} = 0$である。
    \end{enumerate}
\end{proposition}

\begin{proof}
    \uline{(1)} \quad
    \cref{eq:Gamma_g}の左側の等式は
    \begin{alignat}{1}
        {\Gamma^{(g)}}_{ij}^k
            &=
                \frac{1}{2} g^{kl}
                \myparen{
                    \del_i g_{jl}
                    + \del_j g_{li}
                    - \del_l g_{ij}
                }
                \\
            &=
                \frac{1}{2} g^{kl}
                \myparen{
                    S_{ijl}
                    + S_{jli}
                    - S_{lij}
                }
                \quad
                (\text{\cref{prop:components-of-AC-tensor}})
                \\
            &=
                \frac{1}{2} g^{kl} S_{ijl}
                \\
            &=
                \frac{1}{2} A_{ij}^k
    \end{alignat}
    より従う。
    $g$で添字を下げて\cref{eq:Gamma_g}の右側の等式も従う。

    \uline{(2)} \quad
    $\alpha$-接続の定義より
    ${\Gamma^{(\alpha)}}_{ijk}
        = {\Gamma^{(g)}}_{ijk}
            - \frac{\alpha}{2} S_{ijk}$
    だから、
    (1)とあわせて\cref{eq:Gamma_alpha}の左側の等式が従う。
    $g$で添字を下げて\cref{eq:Gamma_g}の右側の等式も従う。
\end{proof}

\begin{proposition}[捩率と曲率のACテンソルによる表示]
    $\calP$上の任意の$\nabla$-アファイン座標に関し、
    $\nabla^{(\alpha)}$の捩率テンソル$T^{(\alpha)}$
    および$(1, 3)$-曲率テンソル$R^{(\alpha)}$の成分表示は
    次をみたす:
    \begin{enumerate}
        \item すべての$\alpha \in \R$に対し
            \begin{equation}
                {T^{(\alpha)}}_{ij}^k
                    = 0
            \end{equation}
        \item すべての$\alpha \in \R$に対し
            \begin{equation}
                {R^{(\alpha)}}_{ijk}^l
                    = \frac{1 - \alpha}{2} \myparen{
                        \del_i A_{jk}^l
                        -
                        \del_j A_{ik}^l
                    }
                    + \myparen{\frac{1 - \alpha}{2}}^2
                    \myparen{
                        A_{jk}^m A_{im}^l
                        -
                        A_{ik}^m A_{jm}^l
                    }
            \end{equation}
            とくに$\alpha = 1$のとき
            ${R^{(1)}}_{ijk}^l = 0$である。
    \end{enumerate}
\end{proposition}

\begin{proof}
    \uline{(1)} \quad
    \begin{alignat}{1}
        {T^{(\alpha)}}_{ij}
            &=
                {\Gamma^{(\alpha)}}_{ij}^k
                - {\Gamma^{(\alpha)}}_{ji}^k
                \\
            &=
                \frac{1 - \alpha}{2} A_{ij}^k
                - \frac{1 - \alpha}{2} A_{ji}^k
                \quad
                (\cref{prop:connections_by_AC_tensor} (2))
                \\
            &=
                0
                \quad
                (\text{$A_{ij}^k = A_{ji}^k$})
    \end{alignat}
    より従う。

    \uline{(2)} \quad
    \begin{alignat}{1}
        {R^{(\alpha)}}_{ijk}^l
            &=
                \del_i {\Gamma^{(\alpha)}}_{jk}^l
                - \del_j {\Gamma^{(\alpha)}}_{ik}^l
                + {\Gamma^{(\alpha)}}_{jk}^m
                    {\Gamma^{(\alpha)}}_{im}^l
                - {\Gamma^{(\alpha)}}_{ik}^m
                    {\Gamma^{(\alpha)}}_{jm}^l
                \\
            &=
                \frac{1 - \alpha}{2}
                \myparen{
                    \del_i A_{jk}^l
                    - \del_j A_{ik}^l
                }
                + \myparen{\frac{1 - \alpha}{2}}^2
                \myparen{
                    A_{jk}^m A_{im}^l
                    - A_{ik}^m A_{jm}^l
                }
                \quad
                (\cref{prop:connections_by_AC_tensor} (2))
    \end{alignat}
    より従う。
\end{proof}


% ------------------------------------------------------------
%
% ------------------------------------------------------------
\section{指数型分布族の具体例}

% ------------------------------------------------------------
%
% ------------------------------------------------------------
\subsection{具体例: 有限集合上の full support な確率分布の族}

本節では、
有限集合上の full support な確率分布の族について、
$\alpha$-接続に関する測地線方程式を求めてみる。

\begin{settings}[有限集合上の full support な確率分布の族]
    $\calX \coloneqq \{ 1, \dots, n \} \; (n \in \Z_{\ge 1})$とし、
    \begin{equation}
        \calP \coloneqq \mybrace{
            \sum_{i = 1}^n p_i \delta^i
            \in \calP(\calX)
            \, \Big| \,
            0 < p_i < 1 \; (i = 1, \dots, n)
        }
    \end{equation}
    とおく。
    ただし$\delta^i$は
    1点$i \in \calX$での Dirac 測度である。
    これが$\calX$上の指数型分布族であることは
    \url{0425_資料.pdf}例3.1で確かめた。
\end{settings}

\begin{proposition}[最小次元実現の構成および$\calP$が開であることの確認]
    \label[proposition]{prop:minimal_representation}
    ~
    \begin{enumerate}
        \item $(V, T, \gamma)$を次のように定めると、
            これは$\calP$の実現となる:
            \begin{alignat}{1}
                &V \coloneqq \R^{n - 1}, \\
                &T \colon \calX \to V, \quad
                    k \mapsto \up{t}(\delta_{1k}, \dots, \delta_{(n - 1)k}), \\
                &\gamma \colon \text{数え上げ測度}
            \end{alignat}
        \item この実現の対数分配関数$\psi \colon \wt{\Theta} \to \R$は
            $\psi(\theta)
                =
                    \log\myparen{
                        1 + \sum_{i = 1}^{n - 1} \exp\theta^i
                    }$
            となる。
        \item 写像$P \coloneqq P_{(V, T, \gamma)} \colon \wt{\Theta} \to \calP(\calX)$は
            次をみたす:
            \begin{equation}
                P(\theta)
                    =
                        \frac{
                            1
                        }{
                            1 + \sum_{i = 1}^{n - 1} \exp\theta^i
                        }
                        \myparen{
                            \sum_{i = 1}^{n - 1}
                                (\exp\theta^i)
                                \delta^i
                                +
                                \delta^n
                        }
            \end{equation}
        \item $\Theta = \wt{\Theta} = V^\vee$が成り立つ。
        \item 次の写像$\theta \colon \calP \to \Theta$は$P$の逆写像である:
            \begin{alignat}{1}
                \theta
                    \colon
                        \calP \to \Theta,
                    \quad
                        \sum_{i = 1}^n p_i \delta^i
                        \mapsto
                        \myparen{
                            \log \frac{p_1}{p_n},
                            \dots,
                            \log \frac{p_{n - 1}}{p_n}
                        }
            \end{alignat}
        \item $(V, T, \gamma)$は最小次元実現である。
            とくに$\calP$は開である。
    \end{enumerate}
\end{proposition}

\begin{proof}
    \uline{(1)} \quad
    $(V, T, \gamma)$が
    実現であることは\url{0425_コメント.pdf}演習問題0.1に記した。

    \uline{(2)} \quad
    対数分配関数の定義より
    \begin{alignat}{1}
        \psi(\theta)
            &=
                \log \int_\calX
                    \exp \myangle{\theta}{T(k)}
                    \, \gamma(dk)
                \\
            &=
                \log \sum_{i = 1}^n
                    \exp \myparen{
                        \sum_{j = 1}^{n - 1}
                            \theta^j
                            \delta_{ji}
                    }
                \\
            &=
                \log \myparen{
                    \sum_{i = 1}^{n - 1}
                        \exp \theta^i
                    + 1
                }
    \end{alignat}
    である。

    \uline{(3)} \quad
    $P$の定義より
    \begin{alignat}{1}
        P(\theta)
            &=
                \exp(\myangle{\theta}{T(k)} - \psi(\theta)) \gamma
                \\
            &=
                \frac{
                    1
                }{
                    1 + \sum_{i = 1}^{n - 1} \exp\theta^i
                }
                \exp \myparen{
                    \sum_{i = 1}^{n - 1}
                        \theta^i
                        \delta_{ik}
                }
                \gamma
                \\
            &=
                \frac{
                    1
                }{
                    1 + \sum_{i = 1}^{n - 1} \exp\theta^i
                }
                \myparen{
                    \sum_{i = 1}^{n - 1}
                        (\exp\theta^i)
                        \delta^i
                        +
                        \delta^n
                }
    \end{alignat}
    である。

    \uline{(4)} \quad
    可積分性を考えると
    明らかに$\wt{\Theta} = V^\vee$である。
    また$P$が(3)のように表せることから
    $P(\wt{\Theta}) \subset \calP$がわかる。
    したがって$V^\vee = \wt{\Theta} \subset P^{-1}(\calP) = \Theta$である。
    よって$\Theta = \wt{\Theta} = V^\vee$である。

    \uline{(5)} \quad
    $P \circ \theta, \; \theta \circ P$を
    直接計算すれば確かめられる。

    \uline{(6)} \quad
    最小次元実現の特徴づけを確かめればよい。
    条件A(3)が成り立つことは、
    いま$V$の任意のアファイン部分空間に対し
    「$T(x) \in W \; \text{$\gamma$-a.e.$x$}$」
    と
    「$T(x) \in W \; \text{$\forall x$}$」
    が同値であることから明らか。
    条件Bが成り立つことは
    $\Theta = V^\vee$よりわかる。
\end{proof}

以降、
$\calP$には自然な位相および多様体構造が入っているものとして扱い、
$\calP$上の自然な平坦アファイン接続を$\nabla$、
Fisher 計量を$g$、
$(0, 3), (1, 2)$型の Amari-Chentsov テンソルを
それぞれ$S, A$とおく。
また、$\theta \colon \calP \to \Theta$は
多様体$\calP$の座標とみなす。

\begin{remark}[$\calP$の2通りの位相 \& 多様体構造]
    $\calP$上の位相 \& 多様体構造として、
    $\calX$上の符号付き測度全体のなす
    ベクトル空間$\calS(\calX) \cong \R^n$の部分多様体としてのものと、
    指数型分布族としての自然なものの2通りを考えられるが、
    これらは互いに一致する。
    なぜならば、
    いずれの位相 \& 多様体構造に関しても
    写像$\theta \colon \calP \to \Theta$は微分同相写像だからである。
\end{remark}

\begin{proposition}[Fisher 計量の成分]
    \label[proposition]{prop:fisher_metric_components}
    座標$\theta = (\theta^1, \dots, \theta^{n - 1})$に関する
    Fisher 計量$g$の成分は
    \begin{equation}
        g_{ij}(p)
            = \delta_{ij} p_i - p_i p_j
            \qquad
            (p \in \calP, \; i, j = 1, \ldots, n - 1)
    \end{equation}
    となる。
\end{proposition}

\begin{proof}
    微分同相写像$\theta$により
    $g$を$\Theta$上のテンソル場とみなして計算すれば、
    各$p \in \calP$に対し
    \begin{alignat}{1}
        g_{ij}(p)
            &=
                (\Var_p [T])(e^i, e^j)
                \\
            &=
                E_p[(T^i - E_p[T^i])(T^j - E_p[T^j])]
                \\
            &=
                \sum_{k = 1}^n
                    (\delta_{ik} - p_i)
                    (\delta_{jk} - p_j)
                    p_k
                \\
            &=
                \delta_{ij} p_i - p_i p_j
    \end{alignat}
    が成り立つ。
\end{proof}

\begin{proposition}[ACテンソルの成分]
    \label[proposition]{prop:ac_tensor_components}
    座標$\theta$に関する
    ACテンソル$S$の成分は
    \begin{equation}
        S_{ijk}(p)
            = p_i \delta_{ij} \delta_{jk}
                - p_i p_k \delta_{ij}
                - p_i p_j \delta_{jk}
                - p_j p_k \delta_{ik}
                + 2 p_i p_j p_k
            \qquad
            (p \in \calP, \; i, j, k = 1, \ldots, n - 1)
    \end{equation}
    となる。
\end{proposition}

\begin{proof}
    前回 (\url{0613_資料.pdf}) の命題1.9を用いると
    \begin{equation}
        S_{ijk}(p)
            = E_p [
                (T^i - E_p[T^i])
                (T^j - E_p[T^j])
                (T^k - E_p[T^k])
            ]
    \end{equation}
    となるから、
    \cref{prop:fisher_metric_components}と同様に直接計算して
    命題の主張の等式が得られる。
\end{proof}

以降、$n = 3$の場合を考える。

\begin{proposition}[$n = 3$での$g, S, A$の計算]
    座標$\theta$に関し、
    $g$の行列表示は
    \begin{equation}
        (g_{ij})_{i, j}
            = \begin{pmatrix}
                p_1 (1 - p_1) & - p_1 p_2 \\
                - p_1 p_2 & p_2 (1 - p_2)
            \end{pmatrix},
            \quad
        (g^{ij})_{i, j}
            = \frac{1}{p_3}
                \begin{pmatrix}
                    \frac{p_3}{p_1} + 1 & 1 \\
                    1 & \frac{p_3}{p_2} + 1
                \end{pmatrix}
    \end{equation}
    となる。
    $S$の成分は
    \begin{alignat}{1}
        S_{111}
            &= p_1 - 3 p_1^2 + 2 p_1^3, \\
        S_{112} = S_{121} = S_{211}
            &= - p_1 p_2 + 2 p_1^2 p_2, \\
        S_{122} = S_{212} = S_{221}
            &= - p_1 p_2 + 2 p_1 p_2^2, \\
        S_{222}
            &= p_2 - 3 p_2^2 + 2 p_2^3
    \end{alignat}
    となる。
    $A$の成分は
    \begin{alignat}{2}
        A_{11}^{\hphantom{11}1}
            &=
                1 - 2p_1,
                \qquad
        &A_{11}^{\hphantom{11}2}
            &=
                0
                \\
        A_{12}^{\hphantom{12}1}
            =
                A_{21}^{\hphantom{21}1}
            &=
                - p_2,
                \qquad
        &A_{12}^{\hphantom{12}2}
            =
                A_{21}^{\hphantom{21}2}
            &=
                - p_1
                \\
        A_{22}^{\hphantom{22}1}
            &=
                0,
                \qquad
        &A_{22}^{\hphantom{22}2}
            &=
                1 - 2p_2
    \end{alignat}
    となる。
\end{proposition}

\begin{proof}
    $g$の行列表示は
    \cref{prop:fisher_metric_components}よりわかる。
    その逆行列は直接計算よりわかる。
    $S$の成分は
    \cref{prop:ac_tensor_components}よりわかる。
    $A$の成分は「$A_{ij}^{\hphantom{ij}k} = g^{kl} S_{ijl}$」
    を用いて求める。
    具体的には以下の行列を直接計算すればわかる:
    \begin{equation}
        \begin{pmatrix}
            A_{11}^{\hphantom{11}1}
                & A_{12}^{\hphantom{12}1}
                & A_{22}^{\hphantom{22}1}
                \\
            A_{11}^{\hphantom{11}2}
                & A_{12}^{\hphantom{12}2}
                & A_{22}^{\hphantom{22}2}
        \end{pmatrix}
            =
                \frac{1}{p_3}
                \begin{pmatrix}
                    \frac{p_3}{p_1} + 1 & 1 \\
                    1 & \frac{p_3}{p_2} + 1
                \end{pmatrix}
                \begin{pmatrix}
                    S_{111}
                        & S_{121}
                        & S_{221}
                        \\
                    S_{112}
                        & S_{122}
                        & S_{222}
                \end{pmatrix}
    \end{equation}
\end{proof}

\begin{proposition}[$n = 3$での測地線方程式]
    各$\alpha \in \R$に対し、
    座標$\theta$に関する
    $\nabla^{(\alpha)}$-測地線の方程式は
    \begin{alignat}{1}
        \ddot{\theta^1}
            &=
                - \frac{1 - \alpha}{2}
                \myparen{
                    \myparen{
                        1 - \frac{2 \exp \theta^1}{1 + \exp \theta^1 + \exp \theta^2}
                    }
                    (\dot{\theta^1})^2
                    -
                    \frac{2 \exp \theta^2}{1 + \exp \theta^1 + \exp \theta^2}
                    \dot{\theta^1} \dot{\theta^2}
                }
                \\
        \ddot{\theta^2}
            &=
                - \frac{1 - \alpha}{2}
                \myparen{
                    -
                    \frac{2 \exp \theta^1}{1 + \exp \theta^1 + \exp \theta^2}
                    \dot{\theta^1} \dot{\theta^2}
                    + \myparen{
                        1 - \frac{2 \exp \theta^2}{1 + \exp \theta^1 + \exp \theta^2}
                    }
                    (\dot{\theta^2})^2
                }
    \end{alignat}
    となる。
    とくに$\alpha = 1$のとき
    \begin{equation}
        \ddot{\theta^1} = 0,
            \quad
            \ddot{\theta^2} = 0
    \end{equation}
    である。
\end{proposition}

\begin{proof}
    測地線の方程式
    \begin{equation}
        \ddot{\theta^k}
            = - \Gamma_{ij}^k \dot{\theta^i} \dot{\theta^j}
    \end{equation}
    に、前回 (\url{0613_資料.pdf}) の命題1.11の等式
    ${\Gamma^{(\alpha)}}_{ij}^k = \frac{1 - \alpha}{2} A_{ij}^{\hphantom{ij}k}$
    を代入して得られる。
\end{proof}

$\alpha \neq 1$の場合に
上の測地線方程式を解くのは難しいように思う。
数値計算の結果を資料末尾の付録に載せた。


% ------------------------------------------------------------
%
% ------------------------------------------------------------
\subsection{具体例: 正規分布族}

本節では、
正規分布族について、
$\alpha$-接続に関する測地線方程式を求めてみる。

\begin{settings}[正規分布族]
    $\calX \coloneqq \R$とし、
    \begin{equation}
        \calP \coloneqq \mybrace{
            \frac{1}{\sqrt{2 \pi \sigma^2}}
            \exp\myparen{
                - \frac{(x - \mu)^2}{2 \sigma^2}
            }
            \lambda(dx)
            \in \calP(\calX)
            \, \Big| \,
            (\mu, \sigma) \in \R \times \R_{> 0}
        }
    \end{equation}
    とおく。
    これが$\calX$上の指数型分布族であることは
    \url{0425_資料.pdf}例3.2で確かめた。
\end{settings}

以降、次の事実をしばしば用いる:

\begin{fact}
    次の2つの写像は互いに逆な{\smooth}写像である:
    \begin{alignat}{1}
        \R \times \R_{> 0} \to \R \times \R_{< 0},
            \qquad
            &(\mu, \sigma)
            \mapsto
            \myparen{
                \frac{\mu}{\sigma^2},
                - \frac{1}{2\sigma^2}
            },
            \\
        \R \times \R_{< 0} \to \R \times \R_{> 0},
            \qquad
            &(\theta^1, \theta^2)
            \mapsto
            \myparen{
                - \frac{\theta^1}{2\theta^2},
                \sqrt{- \frac{1}{2\theta^2}}
            }
    \end{alignat}
    \qed
\end{fact}

\begin{proposition}[最小次元実現の構成および$\calP$が開であることの確認]
    \label[proposition]{prop:minimal_representation_nd}
    ~
    \begin{enumerate}
        \item $(V, T, \lambda)$を次のように定めると、
            これは$\calP$の実現となる:
            \begin{alignat}{1}
                &V = \R^2, \\
                &T \colon \calX \to V, \quad
                    x \mapsto \up{t}(x, x^2), \\
                &\lambda \colon \text{Lebesgue 測度}.
            \end{alignat}
        \item この実現の対数分配関数$\psi \colon \wt{\Theta} \to \R$は
            $\psi(\theta)
                =
                    - \frac{(\theta^1)^2}{4 \theta^2}
                    - \frac{1}{2} \log (- \theta^2)
                    + \frac{1}{2} \log \pi
                $
            となる。
        \item $\Theta = \wt{\Theta} = \R \times \R_{< 0}$が成り立つ。
        \item 次の写像$\theta \colon \calP \to \Theta$は
            $P \coloneqq P_{(V, T, \lambda)}$の逆写像である:
            \begin{alignat}{1}
                \theta
                    \colon
                        \calP \to \Theta,
                    \quad
                        p
                        \mapsto
                        \myparen{
                            \frac{E_p[x]}{\Var_p[x]},
                            - \frac{1}{2 \Var_p[x]}
                        }
            \end{alignat}
        \item $(V, T, \lambda)$は最小次元実現である。
            とくに$\calP$は開である。
    \end{enumerate}
\end{proposition}

\begin{proof}
    \uline{(1)} \quad
    実現であることは\url{0425_資料.pdf}例3.2で確かめた。

    \uline{(2)} \quad
    対数分配関数の定義から直接計算よりわかる。

    \uline{(3)} \quad
    $\theta^2 \ge 0$だと
    $\exp\myparen{
        \theta^1 x
        + \theta^2 x^2
        - \psi(\theta)
    }$
    は積分可能でないから
    $\Theta \subset \wt{\Theta} \subset \R \times \R_{< 0}$
    である。
    逆に写像$P \coloneqq P_{(V, T, \lambda)}$について、
    すべての$p \in P(\R \times \R_{< 0})$は
    $p(dx) = \exp(
        \theta^1 x
        + \theta^2 x^2
        - \psi(\theta)
    ) \lambda(dx)
        \; (\exists (\theta^1, \theta^2) \in \R \times \R_{< 0})$
    と表せるから、
    $(\mu, \sigma) \coloneqq \myparen{
        - \frac{\theta^1}{2\theta^2}, \;
        \sqrt{- \frac{1}{2\theta^2}}
    } \in \R \times \R_{> 0}$
    とおけば
    $p(dx) = \frac{1}{\sqrt{2 \pi \sigma^2}}
        \exp\myparen{
            - \frac{(x - \mu)^2}{2 \sigma^2}
        } \lambda(dx)$
    と表せることになり$p \in \calP$がわかる。
    したがって
    $P(\R \times \R_{< 0}) \subset \calP$をみたすから
    $\R \times \R_{< 0} \subset P^{-1}(\calP) =  \Theta$である。
    よって
    $\Theta = \wt{\Theta} = \R \times \R_{< 0}$である。

    \uline{(4)} \quad
    $(\theta^1, \theta^2) \in \R \times \R_{< 0}$と
    $(\mu, \sigma) \in \R \times \R_{> 0}$の対応に注意すれば
    直接計算よりわかる。

    \uline{(5)} \quad
    最小次元実現の特徴づけの条件A(3)と条件Bが成り立つことから、
    最小次元実現であることがわかる。
\end{proof}

以降、
$\calP$には自然な位相および多様体構造が入っているものとして扱い、
$\calP$上の自然な平坦アファイン接続を$\nabla$、
Fisher 計量を$g$、
$(0, 3), (1, 2)$型の Amari-Chentsov テンソルを
それぞれ$S, A$とおく。
また、$\theta \colon \calP \to \Theta$は
多様体$\calP$の座標とみなす。

\begin{proposition}
    座標$(\mu, \sigma)$に関する$g$の行列表示は
    \begin{equation}
        (g_{ij})_{i, j}
            = \begin{pmatrix}
                \frac{1}{\sigma^2} & 0 \\
                0 & \frac{2}{\sigma^2}
            \end{pmatrix},
            \qquad
        (g^{ij})_{i, j}
            = \begin{pmatrix}
                \sigma^2 & 0 \\
                0 & \frac{\sigma^2}{2}
            \end{pmatrix}
    \end{equation}
    となる。
\end{proposition}

\begin{proof}
    微分同相写像$\theta$により
    $g$を$\Theta$上のテンソル場とみなして計算する。
    座標$(\theta^1, \theta^2)$と
    座標$(\mu, \sigma)$の間の座標変換が
    $\theta^1 = \frac{\mu}{\sigma^2}, \;
        \theta^2 = -\frac{1}{2 \sigma^2}$
    および
    $\mu = -\frac{\theta^1}{2\theta^2}, \;
        \sigma = \sqrt{-\frac{1}{2\theta^2}}$
    であることに注意すると
    \begin{alignat}{2}
        d\mu
            &=
                - \frac{1}{2\theta^2} d\theta^1
                + \frac{\theta^1}{2(\theta^2)^2} d\theta^2,
            &\qquad
        d\sigma
            &=
                \frac{1}{2\sqrt{2}} (-\theta^2)^{-3/2} d\theta^2,
                \\
        d\theta^1
            &=
                \frac{1}{\sigma^2} d\mu
                - \frac{2\mu}{\sigma^3} d\sigma,
            &\qquad
        d\theta^2
            &=
                \frac{1}{\sigma^3} d\sigma,
    \end{alignat}
    さらに
    \begin{alignat}{1}
        (d\theta^1)^2
            &=
                \frac{1}{\sigma^4} (d\mu)^2
                - \frac{\mu}{\sigma^5} d\mu d\sigma
                + \frac{4\mu^2}{\sigma^6} (d\sigma)^2,
                \\
        d\theta^1 d\theta^2
            &=
                \frac{1}{\sigma^5} d\mu d\sigma
                - \frac{2\mu}{\sigma^6} (d\sigma)^2,
                \\
        (d\theta^2)^2
            &=
                \frac{1}{\sigma^6} (d\sigma)^2
    \end{alignat}
    である。
    したがって、
    $\Theta$上の標準的な平坦アファイン接続を$D$とおくと
    \begin{alignat}{1}
        Dd\mu
            &=
                \frac{1}{(\theta^2)^2} d\theta^1 d\theta^2
                - \frac{\theta^1}{(\theta^2)^3} (d\theta^2)^2
            =
                \frac{4}{\sigma} d\mu d\sigma,
                \\
        Dd\sigma
            &=
                \frac{3}{4\sqrt{2}} (-\theta^2)^{-5/2} (d\theta^2)^2
            =
                \frac{3}{\sigma} (d\sigma)^2
    \end{alignat}
    である。
    よって
    \begin{alignat}{1}
        d\psi
            &=
                \frac{\mu}{\sigma^2}
                d\mu
                + \myparen{
                    - \frac{\mu^2}{\sigma^3}
                    + \frac{1}{\sigma}
                }
                d\sigma,
                \\
        \Hess\psi
            &=
                Dd\psi
                \\
            &=
                d\myparen{
                    \frac{\mu}{\sigma^2}
                }
                d\mu
                + \frac{\mu}{\sigma^2} Dd\mu
                + d\myparen{
                    - \frac{\mu^2}{\sigma^3}
                    + \frac{1}{\sigma}
                }
                d\sigma
                + \myparen{
                    - \frac{\mu^2}{\sigma^3}
                    + \frac{1}{\sigma}
                }
                Dd\sigma
                \\
            &=
                \frac{1}{\sigma^2} (d\mu)^2
                + \frac{2}{\sigma^2} (d\sigma)^2
    \end{alignat}
    である。
    これより命題の主張が従う。
\end{proof}

\begin{proposition}[ACテンソルの成分]
    座標$(\mu, \sigma)$に関するACテンソル$S$の成分は
    \begin{alignat}{1}
        S_{111}
            &=
                0
                \\
        S_{112} = S_{121} = S_{211}
            &=
                \frac{2}{\sigma^3}
                \\
        S_{122} = S_{212} = S_{221}
            &=
                0
                \\
        S_{222}
            &=
                \frac{8}{\sigma^3}
    \end{alignat}
    である。
    座標$(\mu, \sigma)$に関する$A$の成分は
    \begin{alignat}{2}
        A_{11}^{\hphantom{11}1}
            &=
                0,
                \qquad
        &A_{11}^{\hphantom{11}2}
            &=
                \frac{1}{\sigma},
                \\
        A_{12}^{\hphantom{12}1}
            &=
                A_{21}^{\hphantom{21}1}
            =
                \frac{2}{\sigma},
                \qquad
        &A_{12}^{\hphantom{12}2}
            &=
                A_{21}^{\hphantom{21}2}
            =
                0,
                \\
        A_{22}^{\hphantom{22}1}
            &=
                0,
                \qquad
        &A_{22}^{\hphantom{22}2}
            &=
                \frac{4}{\sigma}
    \end{alignat}
    である。
\end{proposition}

\begin{proof}
    微分同相写像$\theta$により
    $S, A$を$\Theta$上のテンソル場とみなして計算する。
    $\Theta$上の標準的な平坦アファイン接続を$D$とおくと
    \begin{alignat}{1}
        DDd\psi
            &=
                D \myparen{
                    \frac{1}{\sigma^2} (d\mu)^2
                    + \frac{2}{\sigma^2} (d\sigma)^2
                }
                \\
            &=
                - \frac{2}{\sigma^3} (d\mu)^2 d\sigma
                + \frac{1}{\sigma^2} D(d\mu)^2
                - \frac{4}{\sigma^3} (d\sigma)^3
                + \frac{2}{\sigma^2} D(d\sigma)^2
    \end{alignat}
    ここで
    \begin{alignat}{1}
        D(d\mu)^2
            &=
                2 d\mu Dd\mu
            =
                \frac{8}{\sigma} (d\mu)^2 d\sigma,
                \\
        D(d\sigma)^2
            &=
                2 d\sigma Dd\sigma
            =
                \frac{6}{\sigma} (d\sigma)^3
    \end{alignat}
    だから
    \begin{equation}
        DDd\psi
            =
                \frac{6}{\sigma^3} (d\mu)^2 d\sigma
                + \frac{8}{\sigma^3} (d\sigma)^3
    \end{equation}
    である。
    これより命題の主張の式が得られる。
    $A$の成分は
    「$A_{ij}^{\hphantom{ij}k} = g^{kl} S_{ijl}$」
    を用いて直接計算より得られる。
\end{proof}

\begin{proposition}[接続係数]
    ~
    \begin{enumerate}
        \item 座標$(\mu, \sigma)$に関する
            $\nabla^g$の接続係数は
            \begin{alignat}{2}
                {\Gamma^{g}}_{11}^1
                    = 0,
                    &\qquad
                        {\Gamma^{g}}_{12}^1
                            = {\Gamma^{g}}_{21}^1
                            = -\frac{1}{\sigma},
                    &&\qquad
                        {\Gamma^{g}}_{22}^1
                            = 0,
                    \\
                {\Gamma^{g}}_{11}^2
                    = \frac{1}{2\sigma},
                    &\qquad
                        {\Gamma^{g}}_{12}^2
                            = {\Gamma^{g}}_{21}^2
                            = 0,
                    &&\qquad
                        {\Gamma^{g}}_{22}^2
                            = -\frac{1}{\sigma}
            \end{alignat}
            である。
        \item 座標$(\mu, \sigma)$に関する
            $\nabla^{(\alpha)}$の接続係数は
            \begin{alignat}{2}
                {\Gamma^{(\alpha)}}_{11}^1
                    = 0,
                    &\qquad
                        {\Gamma^{(\alpha)}}_{12}^1
                            = {\Gamma^{(\alpha)}}_{21}^1
                            = - \frac{1 + \alpha}{\sigma},
                    &&\qquad
                        {\Gamma^{(\alpha)}}_{22}^1
                            = 0,
                    \\
                {\Gamma^{(\alpha)}}_{11}^2
                    = \frac{1 - \alpha}{2 \sigma},
                    &\qquad
                        {\Gamma^{(\alpha)}}_{12}^2
                            = {\Gamma^{(\alpha)}}_{21}^2
                            = 0,
                    &&\qquad
                        {\Gamma^{(\alpha)}}_{22}^2
                            = - \frac{1 + 2\alpha}{\sigma}
            \end{alignat}
            である。
    \end{enumerate}
\end{proposition}

\begin{proof}
    $\Gamma^g$は
    ${\Gamma^g}_{ij}^k
        = \frac{1}{2} g^{kl} \myparen{
            \partial_i g_{jl}
            + \partial_j g_{li}
            - \partial_l g_{ij}
        }$
    を直接計算することで得られる。
    $\Gamma^{(\alpha)}$は
    ${\Gamma^{(\alpha)}}_{ij}^k
        = {\Gamma^g}_{ij}^k - \frac{\alpha}{2} A_{ij}^{\hphantom{ij}k}$
    より得られる。
\end{proof}

\begin{proposition}[測地線方程式]
    $(\mu, \sigma)$座標に関する$\nabla^{(\alpha)}$-測地線の方程式は
    \begin{equation}
        \begin{cases}
            \ddot{\mu}
                - \frac{2 (1 + \alpha)}{\sigma} \dot{\mu} \dot{\sigma}
                = 0,
                \\
            \ddot{\sigma}
                + \frac{1 - \alpha}{2 \sigma} \dot{\mu}^2
                - \frac{1 + 2 \alpha}{\sigma} \dot{\sigma}^2
                = 0
        \end{cases}
    \end{equation}
    である。
    とくに$\alpha = 0$のとき
    \begin{equation}
        \begin{cases}
            \ddot{\mu}
                - \frac{2}{\sigma} \dot{\mu} \dot{\sigma}
                = 0,
                \\
            \ddot{\sigma}
                + \frac{1}{2 \sigma} \dot{\mu}^2
                - \frac{1}{\sigma} \dot{\sigma}^2
                = 0
        \end{cases}
    \end{equation}
    である。
\end{proposition}

\begin{proof}
    測地線の方程式
    「$\ddot{x^k} = - {\Gamma}_{ij}^k \dot{x^i} \dot{x^j}$」
    に接続係数を代入して得られる。
\end{proof}

\begin{proposition}
    $\nabla^g$-測地線の像は、
    楕円
    \begin{equation}
        \myparen{
            \frac{x - x_0}{\sqrt{2}}
        }^2
            + y^2 = r^2
            \qquad
            (x_0 \in \R, \; r \in \R_{> 0})
    \end{equation}
    の一部または
    $y$軸に平行な直線の一部である。
\end{proposition}

\begin{proof}[証明\footnote{
    証明の流れは\cite[Chap.3 14.4]{Tu17}を参考にした。
}]
    測地線の方程式
    \begin{alignat}{1}
        \ddot{\mu}
            - \frac{2}{\sigma} \dot{\mu} \dot{\sigma}
            &= 0,
            \locallabel{eq:1}
            \\
        \ddot{\sigma}
            + \frac{1}{2 \sigma} \dot{\mu}^2
            - \frac{1}{\sigma} \dot{\sigma}^2
            &= 0
            \locallabel{eq:2}
    \end{alignat}
    を変形していく。

    $\dot{\mu} = 0$の場合は
    $\mu = \text{const.}$ゆえに
    測地線は$y$軸に平行な直線の一部である。

    以下、$\dot{\mu} \neq 0$の場合を考える。
    \localcref{eq:1}の両辺を$\dot{\mu}$で割って
    \begin{equation}
        \frac{\ddot{\mu}}{\dot{\mu}}
            - 2\frac{\dot{\sigma}}{\sigma}
            = 0
    \end{equation}
    これより
    $\log \dot{\mu} = 2 \log \sigma + \text{const.}$
    したがって
    \begin{equation}
        \dot{\mu} = k \sigma^2
            \qquad
            (k \in \R)
            \locallabel{eq:3}
    \end{equation}
    である。
    一方、$\nabla^g$は$g$の Levi-Civita 接続であるから、
    測地線の速度ベクトルの$g$に関する大きさは一定、
    すなわち
    \begin{equation}
        \frac{\dot{\mu}^2 + 2 \dot{\sigma}^2}{\sigma^2}
            = r^2
            \qquad
            (a \in \R)
            \locallabel{eq:4}
    \end{equation}
    である。
    \localcref{eq:4}に\localcref{eq:3}を代入して
    \begin{alignat}{1}
        \frac{k^2 \sigma^4 + 2 \dot{\sigma}^2}{\sigma^2}
            &=
                a^2
                \\
        \dot{\sigma}
            &=
                \pm \sigma \sqrt{\frac{a^2 - k^2 \sigma^2}{2}}
    \end{alignat}
    を得る。
    これと\localcref{eq:3}より
    \begin{alignat}{1}
        \dd[\mu]{\sigma}
            =
                \frac{\dot{\mu}}{\dot{\sigma}}
            &=
                \frac{k \sigma^2}{\pm \sigma \sqrt{\frac{a^2 - k^2 \sigma^2}{2}}}
                \\
            &=
                \mp \frac{\sqrt{2} |a|}{k}
                \frac{
                    \myparen{
                        \frac{k}{a}
                    }^2
                    \sigma
                }{
                    \sqrt{
                        1 - \myparen{
                            \frac{k}{a}
                        }^2
                        \sigma^2
                    }
                }
                \\
        \therefore \mu
            &=
                \mp \frac{\sqrt{2} |a|}{k}
                \sqrt{
                    1 - \myparen{
                        \frac{k}{a}
                    }^2
                    \sigma^2
                }
                + \mu_0
                \qquad
                (\mu_0 \in \R)
    \end{alignat}
    を得る。
    よって
    \begin{equation}
        (\mu - \mu_0)^2
            =
                \frac{2 a^2}{k^2}
                - 2 \sigma^2
    \end{equation}
    $r \coloneqq \frac{a}{k}$とおいて整理すれば
    \begin{equation}
        \myparen{
            \frac{\mu - \mu_0}{\sqrt{2}}
        }^2
            + \sigma^2
            = r^2
    \end{equation}
    が得られる。
\end{proof}

% ------------------------------------------------------------
%
% ------------------------------------------------------------
\section{双対構造}


\begin{definition}[双対構造]
    $M$を多様体とする。
    $M$上の
    Riemann 計量$g$と
    アファイン接続$\nabla, \nabla^*$の組
    $(g, \nabla, \nabla^*)$
    が$M$上の
    \term{双対構造}[dualistic structure]
        {双対構造}[そうついこうぞう]
    であるとは、
    すべての$X, Y, Z \in \frakX(M)$に対し
    \begin{equation}
        X(g(Y, Z))
            =
                g(\nabla_X Y, Z) + g(Y, \nabla^*_X Z)
    \end{equation}
    が成り立つことをいう。
    このとき、
    $\nabla, \nabla^*$はそれぞれ$g$に関する$\nabla^*, \nabla$の
    \term{双対接続}[dual connection]
        {双対接続}[そうついせつぞく]
    であるという。

    さらに$\nabla, \nabla^*$がいずれも$M$上平坦であるとき、
    $(g, \nabla, \nabla^*)$は
    \term{双対平坦}[dually flat]
        {双対平坦}[そうついへいたん]
    であるという。
    双対平坦な双対構造を
    \term{双対平坦構造}[dually flat structure]
        {双対平坦構造}[そうついへいたんこうぞう]
    という。
\end{definition}

\begin{proposition}[双対接続の存在と一意性]
    \label[proposition]{prop:dual-connection-existence-uniqueness}
    $M$を多様体、
    $g$を$M$上のRiemann 計量、
    $\nabla$を$M$上のアファイン接続とする。
    このとき、
    $g$に関する$\nabla$の双対接続がただひとつ存在する。
\end{proposition}

\begin{proof}
    証明は付録に記した。
\end{proof}

指数型分布族の$\alpha$-接続について考える。
以降、$\calP$を可測空間$\calX$上の open な指数型分布族、
$\nabla$を$\calP$上の自然な平坦アファイン接続、
$g$を$\calP$上の Fisher 計量、
$S, A$をそれぞれ$(0, 3), (1, 2)$型の Amari-Chentsov テンソル、
$\nabla^{(\alpha)} \; (\alpha \in \R)$を$\alpha$-接続とする。

\begin{proposition}[曲率のACテンソルによる表示]
    \label[proposition]{prop:curvature-AC-tensor}
    $\alpha \in \R$、
    $R^{(\alpha)}$を$\nabla^{(\alpha)}$の
    $(1, 3)$-曲率テンソルとする。
    このとき、
    $\calP$の任意の$\nabla$-アファイン座標に関し、
    $R^{(\alpha)}$の成分は
    \begin{equation}
        {R^{(\alpha)}}_{ijk}^{\hphantom{ijk}l}
            =
                - \frac{1 - \alpha^2}{4}
                \myparen{
                    A_{jk}^{\hphantom{jk}m} A_{im}^{\hphantom{im}l}
                    - A_{ik}^{\hphantom{ik}m} A_{jm}^{\hphantom{jm}l}
                }
    \end{equation}
    となる。
\end{proposition}

\begin{proof}
    \url{0613_資料.pdf}命題1.12の式
    \begin{equation}
        {R^{(\alpha)}}_{ijk}^{\hphantom{ijk}l}
            = \frac{1 - \alpha}{2} \myparen{
                \del_i A_{jk}^{\hphantom{jk}l}
                -
                \del_j A_{ik}^{\hphantom{ik}l}
            }
            + \myparen{\frac{1 - \alpha}{2}}^2
            \myparen{
                A_{jk}^{\hphantom{jk}m} A_{im}^{\hphantom{im}l}
                -
                A_{ik}^{\hphantom{ik}m} A_{jm}^{\hphantom{jm}l}
            }
    \end{equation}
    を変形する。
    \begin{alignat}{1}
        \del_i A_{jk}^{\hphantom{jk}l}
        -
        \del_j A_{ik}^{\hphantom{ik}l}
            &=
                \del_i (
                    g^{la} S_{jka}
                )
                -
                \del_j (
                    g^{la} S_{ika}
                )
                \\
            &=
                \del_i (g^{la})
                S_{jka}
                +
                g^{la}
                \del_i S_{jka}
                -
                \del_j (g^{la})
                S_{ika}
                -
                g^{la}
                \del_j S_{ika}
                \\
            &=
                \del_i (g^{la})
                S_{jka}
                -
                \del_j (g^{la})
                S_{ika}
    \end{alignat}
    である。
    右辺第1項について、
    $0
        =
            \del_i \delta_m^l
        =
            \del_i (g^{la} g_{ma})
        =
            \del_i (g^{la}) g_{ma}
            +
            g^{lb} \del_i (g_{mb})$
    より
    $\del_i (g^{la})
        =
            - g^{ma} g^{lb} \del_i (g_{mb})$
    だから
    \begin{alignat}{1}
        \del_i (g^{la})
            S_{jka}
            &=
                -
                g^{ma}
                g^{lb}
                \del_i (g_{mb})
                S_{jka}
                \\
            &=
                -
                g^{ma}
                g^{lb}
                S_{imb}
                S_{jka}
                \\
            &=
                -
                A_{im}^{\hphantom{im}l}
                A_{jk}^{\hphantom{jk}m}
    \end{alignat}
    同様にして
    \begin{equation}
        \del_j (g^{la})
            S_{ika}
                =
                    -
                    A_{jm}^{\hphantom{jm}l}
                    A_{ik}^{\hphantom{ik}m}
    \end{equation}
    を得る。
    したがって
    $\del_i A_{jk}^{\hphantom{jk}l} - \del_j A_{ik}^{\hphantom{ik}l}
        =
            - A_{im}^{\hphantom{im}l} A_{jk}^{\hphantom{jk}m}
            + A_{jm}^{\hphantom{jm}l} A_{ik}^{\hphantom{ik}m}$
    だから
    \begin{equation}
        {R^{(\alpha)}}_{ijk}^{\hphantom{ijk}l}
            =
                \myparen{
                    - \frac{1 - \alpha}{2}
                    + \myparen{\frac{1 - \alpha}{2}}^2
                }
                \myparen{
                    A_{jk}^{\hphantom{jk}m} A_{im}^{\hphantom{im}l}
                    - A_{ik}^{\hphantom{ik}m} A_{jm}^{\hphantom{jm}l}
                }
            =
                -
                \frac{1 - \alpha^2}{4}
                \myparen{
                    A_{jk}^{\hphantom{jk}m} A_{im}^{\hphantom{im}l}
                    - A_{ik}^{\hphantom{ik}m} A_{jm}^{\hphantom{jm}l}
                }
    \end{equation}
    となる。
\end{proof}

\begin{corollary}
    \label[corollary]{corollary:flatness}
    ~
    \begin{enumerate}
        \item $\forall \alpha \in \R$に対し
            $R^{(\alpha)}
                =
                    (1 - \alpha^2)
                    R^{(0)}
                =
                    R^{(-\alpha)}$.
        \item 次は同値:
            \begin{enumerate}
                \item すべての$\alpha \in \R$に対し、
                    $\nabla^{(\alpha)}$は平坦である。
                \item ある$\alpha \neq \pm 1$が存在し、
                    $\nabla^{(\alpha)}$は平坦である。
            \end{enumerate}
    \end{enumerate}
\end{corollary}

\begin{proof}
    \uline{(1)} \quad
    \cref{prop:curvature-AC-tensor}
    より明らか。

    \uline{(2)} \quad
    まず(1)より次は同値である:
    \begin{enumerate}[label=(\alph*)']
        \item $\forall \alpha \in \R$に対し
            $R^{(\alpha)} = 0$.
        \item $\exists \alpha \neq \pm 1$
            \, s.t. \,
            $R^{(\alpha)} = 0$.
    \end{enumerate}
    さらに$\alpha$-接続はすべて torsion-free だから、
    曲率が$0$であることと平坦であることは同値である。
\end{proof}

\begin{theorem}[$\alpha$-接続による双対構造]
    任意の$\alpha \in \R$に対し、
    3つ組$(g, \nabla^{(\alpha)}, \nabla^{(-\alpha)})$は
    $\calP$上の双対構造となる。
    さらに、
    $\alpha = \pm 1$ならば
    $(g, \nabla^{(\alpha)}, \nabla^{(-\alpha)})$は
    双対平坦である。
\end{theorem}

\begin{proof}
    双対構造であることは、
    すべての$X, Y, Z \in \frakX(\calP)$に対し
    \begin{alignat}{1}
        g(\nabla^{(\alpha)}_X Y, Z)
            + g(Y, \nabla^{(-\alpha)}_X Z)
            &=
                g(\nabla^{g}_X Y, Z)
                - \frac{\alpha}{2} S(X, Y, Z)
                + g(Y, \nabla^{g}_X Z)
                + \frac{\alpha}{2} S(X, Z, Y)
                \\
            &=
                g(\nabla^{g}_X Y, Z)
                + g(Y, \nabla^{g}_X Z)
                \\
            &=
                X(g(Y, Z))
    \end{alignat}
    より従う。
    $\alpha = \pm 1$で双対平坦となることは
    \cref{corollary:flatness}
    よりわかる。
\end{proof}


% ------------------------------------------------------------
%
% ------------------------------------------------------------
\section{Legendre 変換}


\begin{definition}[Legendre 変換]
    $U \subset W$を開集合、
    $f \colon U \to \R$を$C^\infty$関数であって
    $\nabla f \colon U \to W^\vee$が単射であるものとする。
    関数
    \begin{equation}
        f^\vee \colon U' \to \R,
            \quad
            y
            \mapsto
            \myangle{(\nabla f)^{-1}(y)}{y} - f((\nabla f)^{-1}(y))
            \quad
            \text{where}
            \quad
            U' \coloneqq \nabla f(U)
    \end{equation}
    を$f$の
    \term{Legendre 変換}[Legendre transform]
        {Legendre 変換}[Legendre へんかん]
    という。
\end{definition}

\begin{example}[Legendre 変換の例]
    前回 (\url{0704_資料.pdf}) 扱った具体例について
    対数分配関数の Legendre 変換を計算してみる。
    \begin{itemize}
        \item \uline{Bernoulli 分布族 (i.e. 2元集合上の full support な確率分布の族):} \quad
            対数分配関数は
            $\psi \colon \R \to \R, \; \theta \mapsto \log (1 + \exp \theta)$
            であった。
            よって
            $\nabla \psi(\theta)
                =
                    \frac{\exp \theta}{1 + \exp \theta}$
            であり、
            $(\nabla \psi)^{-1}(\eta)
                =
                    \log \eta - \log (1 - \eta)$
            である。
            したがって
            $\psi^\vee(\eta)
                =
                    \eta \log \eta
                    + (1 - \eta) \log (1 - \eta)$
            である。
        \item \uline{正規分布族:} \quad
            対数分配関数は
            $\psi \colon \R \times \R_{< 0} \to \R, \;
                \theta
                \mapsto
                - \frac{(\theta^1)^2}{4 \theta^2}
                - \frac{1}{2} \log (- \theta^2)
                + \frac{1}{2} \log \pi$
            であった。
            よって
            $\nabla \psi(\theta)
                =
                    \begin{pmatrix}
                        - \frac{\theta^1}{2 \theta^2}
                        &
                        \frac{(\theta^1)^2}{4 (\theta^2)^2} - \frac{1}{2 \theta^2}
                    \end{pmatrix}$
            であり、
            $(\nabla \psi)^{-1}(\eta)
                =
                    \frac{1}{\eta_2 - (\eta_1)^2}
                    \begin{pmatrix}
                        \eta_1
                        \\
                        - 1/2
                    \end{pmatrix}$
            である。
            よって
            $\psi^\vee(\eta)
                =
                    - \frac{1}{2}
                    \myparen{
                        1 + \log 2\pi
                        + \log(\eta_2 - (\eta_1)^2)
                    }$
            である。
    \end{itemize}
\end{example}

本稿では、とくに次の状況を考えることになる。

\begin{proposition}
    \label[proposition]{prop:Legendre-transform-properties}
    $U \subset W$を凸開集合、
    $f \colon U \to \R$を$C^\infty$関数であって
    $\Hess f$が$U$上各点で (対称であることも含む意味で) 正定値であるものとする。
    このとき、次が成り立つ:
    \begin{enumerate}
        \item $\nabla f$は局所微分同相である。
            とくに$U' \coloneqq \nabla f(U)$は$W^\vee$の開集合である。
        \item $\nabla f \colon U \to U'$は微分同相である。
            とくに$\nabla f$は単射である。
    \end{enumerate}
    したがって$f^\vee$が定義でき、$f^\vee$は次をみたす:
    \begin{enumerate}
        \setcounter{enumi}{2}
        \item $f^\vee \colon U' \to \R$は$C^\infty$関数である。
        \item $\nabla f^\vee = (\nabla f)^{-1}$が成り立つ。
            とくに$\nabla f^\vee$は単射である。
        \item 各$y \in U'$に対し
            $(\Hess f^\vee)_y = ((\Hess f)_x)^{-1}$が成り立つ
            (ただし$x \coloneqq (\nabla f)^{-1}(y)$)。
            とくに$(\Hess f^\vee)_y$は正定値である。
    \end{enumerate}
\end{proposition}

\begin{proof}
    \uline{(1)} \quad
    命題の仮定より$\Hess f$は$U$上各点で正定値だから、
    $\nabla f$の微分は各点で線型同型である。
    したがって$\nabla f$は局所微分同相であり、
    とくに開写像である。
    よって$U' = \nabla f(U)$は$W^\vee$の開集合である。

    \uline{(2)} \quad
    $u, \wt{u} \in U, \; u \neq \wt{u}$を固定し、
    $[0, 1]$を含む$\R$の開区間$I$であって、
    すべての$t \in I$に対し
    $(1 - t)u + t\wt{u}$が
    $U$に属するようなものをひとつ選ぶ
    ($U$は$W$の凸開集合だからこれは可能)。
    さらに
    $\varphi \colon I \to U, \; t \mapsto f((1 - t)u + t\wt{u})$と定めると、
    平均値定理より、
    ある$\tau \in (0, 1)$が存在して
    \begin{alignat}{1}
        \myangle{
            \nabla f(\wt{u}) - \nabla f(u)
        }{
            \wt{u} - u
        }
            &=
                \varphi'(1) - \varphi'(0)
                \\
            &=
                \varphi''(\tau)
                \qquad
                (\text{平均値定理})
                \\
            &=
                \myangle{
                    (\Hess f)_{(1 - \tau)u + \tau\wt{u}}
                }{
                    (\wt{u} - u)^2
                }
                \\
            &>
                0
                \qquad
                (\text{$\Hess f$は正定値})
    \end{alignat}
    が成り立つ。
    よって$\nabla f(\wt{u}) \neq \nabla f(u)$である。
    したがって$\nabla f$は単射である。
    このことと (1) より
    $\nabla f \colon U \to U'$は微分同相である。

    \uline{(3)} \quad
    $\nabla f \colon U \to U'$が微分同相ゆえに
    $(\nabla f)^{-1} \colon U' \to U$は{\smooth}だから、
    $f^\vee$は{\smooth}関数である。

    \uline{(4)} \quad
    $f^\vee$の定義式を$\nabla$で微分すると、
    すべての$y \in U'$に対し
    \begin{alignat}{1}
        (\nabla f^\vee)(y)
            &=
                (\nabla f)^{-1}(y)
                + \myangle{
                    y
                }{
                    \nabla
                    (\nabla f)^{-1}
                    (y)
                }
                - \myangle{
                    (\nabla f)(
                        (\nabla f)^{-1}
                        (y)
                    )
                }{
                    \nabla
                    (\nabla f)^{-1}
                    (y)
                }
            =
                (\nabla f)^{-1}(y)
    \end{alignat}
    が成り立つ。
    よって$(\nabla f)^{-1} = \nabla f^\vee$である。

    \uline{(5)} \quad
    (4)より
    \begin{alignat}{1}
        (\Hess f^\vee)_y
            &=
                d(\nabla f^\vee)_y
                \\
            &=
                d((\nabla f)^{-1})_y
                \\
            &=
                (d(\nabla f)_x)^{-1}
                \\
            &=
                ((\Hess f)_x)^{-1}
    \end{alignat}
    となる。
\end{proof}

\begin{corollary}[Legendre 変換の対合性]
    $f^{\vee \vee} = f$.
\end{corollary}

\begin{proof}
    Legendre 変換の定義より、
    すべての$x \in U$に対し
    \begin{alignat}{1}
        f^{\vee\vee}(x)
            &=
                \myangle{
                    x
                }{
                    (\nabla f^\vee)^{-1}
                    (x)
                }
                - f^\vee(
                    (\nabla f^\vee)^{-1}
                    (x)
                )
                \\
            &=
                \myangle{
                    x
                }{
                    \nabla f
                    (x)
                }
                - f^\vee(
                    \nabla f
                    (x)
                )
                \qquad
                (\nabla f^\vee = (\nabla f)^{-1})
                \\
            &=
                \myangle{
                    x
                }{
                    \nabla f
                    (x)
                }
                - \myparen{
                    \myangle{
                        \nabla f
                        (x)
                    }{
                        (\nabla f)^{-1}(
                            \nabla f
                            (x)
                        )
                    }
                    - f(
                        (\nabla f)^{-1}(
                            \nabla f
                            (x)
                        )
                    )
                }
                \\
            &=
                f(x)
    \end{alignat}
    が成り立つ。
    よって$f^{\vee\vee} = f$である。
\end{proof}


% ------------------------------------------------------------
%
% ------------------------------------------------------------
\section{期待値パラメータ}


\begin{propdef}[期待値パラメータ空間]
    \label[propdef]{propdef:mean-parameter-space}
    集合
    \begin{equation}
        \calM
            \coloneqq
                \mybrace{
                    E_p[T] \in V
                    \mid
                    p \in \calP
                }
    \end{equation}
    は$V$の開部分多様体となり、
    写像$\eta \colon \calP \to \calM, \; p \mapsto E_p[T]$
    は微分同相写像となる。

    $\calM$を
    $(V, T, \mu)$に関する$\calP$の
    \term{期待値パラメータ空間}[mean parameter space]
        {期待値パラメータ空間}[きたいちぱらめーたくうかん]
    といい、
    $\eta$を
    $(V, T, \mu)$に関する$\calP$上の
    \term{期待値パラメータ座標}[mean parameter coordinates]
        {期待値パラメータ座標}[きたいちぱらめーたざひょう]
    という。
\end{propdef}

この証明には次の2つの事実を使う。

\begin{fact}[$\psi$の微分は十分統計量の期待値]
    \label[fact]{fact:mean-parameter-space}
    写像$\nabla \psi \colon \Theta \to V^{\vee\vee} = V$は
    \begin{equation}
        (\nabla \psi)(\theta(p))
            =
                \eta(p)
                \qquad
                (p \in \calP)
    \end{equation}
    をみたす。
    したがって
    $\calM = \nabla \psi(\Theta)$である。
    \qed
\end{fact}

\begin{fact}
    \label[fact]{fact:convexity-of-interior}
    位相ベクトル空間の凸集合の内部は凸集合である。
    \qed
\end{fact}

\begin{proof}[\cref{propdef:mean-parameter-space}の証明]
    まず$\calM$が$V$の開部分多様体となることを示す。
    $\psi$を$\Int \wt{\Theta}$上の関数とみなすと、
    \cref{fact:convexity-of-interior}とあわせて
    $\psi$は\cref{prop:Legendre-transform-properties}の前提をみたすから、
    \cref{prop:Legendre-transform-properties} (1)より
    $\nabla \psi \colon \Int \wt{\Theta} \to V^{\vee\vee} = V$は
    局所微分同相、とくに開写像でもある。
    したがって$\nabla \psi(\Int \wt{\Theta})$は$V$の開部分多様体となる。
    さらに$\Theta$は$\Int \wt{\Theta}$の開集合だから、
    $\nabla \psi(\Theta)$は$\nabla \psi(\Int \wt{\Theta})$の開部分多様体となる。
    このことと\cref{fact:mean-parameter-space}より、
    $\calM = \nabla \psi(\Theta)$は
    $\nabla \psi(\Int \wt{\Theta})$の開部分多様体となり、
    とくに$V$の開部分多様体となる。

    次に$\eta$が微分同相写像であることを示す。
    \cref{prop:Legendre-transform-properties} (2)より
    $\nabla \psi$は
    $\Int \wt{\Theta}$から$\nabla \psi(\Int \wt{\Theta})$への微分同相だから、
    部分多様体への制限により
    $\nabla \psi$は
    $\Theta$から$\calM$への微分同相を与える。
    したがって
    写像$\eta = (\nabla \psi) \circ \theta \colon \calP \to \calM$は
    微分同相である。
\end{proof}

以降、
$\psi|_{\Int \wt{\Theta}}$の Legendre 変換を
$\calM$上に制限したものを$\phi$と書くことにする。

\begin{theorem}[自然パラメータ座標と期待値パラメータ座標の関係]
    関数
    $\psi \colon \Theta \to \R$および
    $\phi \colon \calM \to \R$と、
    $\calP$上の
    自然パラメータ座標$\theta = (\theta^1, \dots, \theta^n)$および
    期待値パラメータ座標$\eta = (\eta_1, \dots, \eta_m)$
    に関し次が成り立つ:
    \begin{enumerate}
        \item
            \begin{equation}
                \deldel[\psi]{\theta^i}(\theta(p)) = \eta_i(p),
                    \qquad
                    \deldel[\phi]{\eta_i}(\eta(p)) = \theta^i(p)
                    \qquad
                    (p \in \calP).
            \end{equation}
        \item $g$の$\theta$-座標に関する成分は
            \begin{equation}
                g_{ij}(p)
                    =
                        \frac{
                            \del^2 \psi
                        }{
                            \del \theta^i
                            \del \theta^j
                        }
                        (\theta(p))
                    =
                        \deldel[\eta_j]{\theta^i}(p),
                        \qquad
                g^{ij}(p)
                    =
                        \frac{
                            \del^2 \phi
                        }{
                            \del \eta_i
                            \del \eta_j
                        }
                        (\eta(p))
                    =
                        \deldel[\theta^i]{\eta_j}(p)
                        \qquad
                        (p \in \calP)
            \end{equation}
            をみたす。
        \item $\delta_i^j$を Kronecker のデルタとして
            \begin{equation}
                g\myparen{
                    \deldel{\theta^i},
                    \deldel{\eta_j}
                }
                    =
                        \delta_i^j
            \end{equation}
            が成り立つ。
    \end{enumerate}
\end{theorem}

\begin{proof}
    \uline{(1)} \quad
    \cref{fact:mean-parameter-space}より
    $\nabla \psi \circ \theta = \eta$であることと、
    \cref{prop:Legendre-transform-properties} (4)より
    $\nabla \phi = (\nabla \psi)^{-1}$であることから従う。

    \uline{(2)} \quad
    $g$の定義および
    \cref{prop:Legendre-transform-properties} (5)より従う。

    \uline{(3)} \quad
    \begin{alignat}{1}
        g\myparen{
            \deldel{\theta^i},
            \deldel{\eta_j}
        }
            =
                g\myparen{
                    \deldel{\theta^i},
                    \deldel[\theta^k]{\eta_j}
                    \deldel{\theta^k}
                }
            =
                g_{ik} \deldel[\theta^k]{\eta_j}
            =
                g_{ik} g^{kj}
            =
                \delta_i^j.
    \end{alignat}
\end{proof}

\begin{theorem}
    期待値パラメータ座標は
    $\calP$上の$\nabla^{(-1)}$-アファイン座標である。
\end{theorem}

\begin{proof}
    $\del_i = \deldel{\theta^i}, \; \del^i = \deldel{\eta_i}$
    と略記すれば、
    上の定理の(3)より
    \begin{alignat}{1}
        0
            =
                \del^i \delta_k^j
            =
                g\myparen{
                    \nabla^{(1)}_{\del^i} \del_k,
                    \del^j
                }
                + g\myparen{
                    \del_k,
                    \nabla^{(1)}_{\del^i} \del^j
                }
    \end{alignat}
    だから
    \begin{alignat}{1}
        {\Gamma^{(-1)}}_k^{ij}
            &=
                g\myparen{
                    \del_k,
                    \nabla^{(-1)}_{\del^i} \del^j
                }
                \\
            &=
                -g\myparen{
                    \nabla^{(1)}_{\del^i} \del_k,
                    \del^j
                }
                \\
            &=
                - \deldel[\theta^l]{\eta_i}
                g\myparen{
                    \nabla^{(1)}_{\del_l} \del_k,
                    \del^j
                }
                \\
            &=
                - \deldel[\theta^l]{\eta_i}
                {\Gamma^{(1)}}_{lk}^j
                \\
            &=
                0
                \qquad
                ({\Gamma^{(1)}}_{lk}^j = 0)
    \end{alignat}
    となる。
\end{proof}


\end{document}

% ============================================================
%
% ============================================================
\newpage
\phantomsection
\addcontentsline{toc}{chapter}{演習問題の解答}
\chapter*{演習問題の解答}

\includecollection{answers}

% ============================================================
%
% ============================================================
\newpage
\phantomsection
\addcontentsline{toc}{chapter}{参考文献}
\renewcommand{\bibname}{参考文献}
\markboth{\bibname}{}
\bibliographystyle{amsalpha}
\bibliography{../mybibliography}

% ============================================================
%
% ============================================================
\newpage
\phantomsection
\addcontentsline{toc}{chapter}{記号一覧}
\printglossary[title={記号一覧}]

% ============================================================
%
% ============================================================
\newpage
\phantomsection
\addcontentsline{toc}{chapter}{索引}
\printindex

\end{document}