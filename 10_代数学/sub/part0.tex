\documentclass[report]{jlreq}
\usepackage{global}
\usepackage{./local}
\subfiletrue
%\makeindex
\begin{document}


% ============================================================
%
% ============================================================
\chapter{群}

群について述べる。

% ------------------------------------------------------------
%
% ------------------------------------------------------------
\section{群}

\begin{definition}[モノイド]
    $M$を集合、
    $e \in M$、
    $\cdot \colon M \times M \to M$を写像とし、
    各$x, y \in M$に対し$\cdot (x, y)$を
    $x \cdot y$や$xy$と書くことにする。
    組$(M, \cdot, e)$が
    \term{モノイド}[monoid]{モノイド}
    であるとは、次が成り立つことをいう:
    \begin{description}
        \item[(M1) 結合律]
            各$x, y, z \in M$に対して
            $(x \cdot y) \cdot z = x \cdot (y \cdot z)$
            が成り立つ。
        \item[(M2) 単位元]
            各$x \in M$に対して
            $x \cdot e = x = e \cdot x$
            が成り立つ。
    \end{description}
    組$(M, \cdot, e)$のことを
    記号の濫用で単に$(M, \cdot)$や$M$と書くことがある。
    さらに
    \begin{itemize}
        \item $e$を$M$の
            \term{単位元}[unit]{単位元}[たんいげん]という。
    \end{itemize}
\end{definition}

\begin{definition}[群]
    モノイド$(G, \cdot, e)$が
    \term{群}[group]{群}[ぐん]であるとは、
    次が成り立つことをいう:
    \begin{description}
        \item[(G1) 逆元]
            各$x \in G$に対して
            ある$y \in G$が存在して
            $x \cdot y = e = y \cdot x$
            が成り立つ。
    \end{description}
    さらに
    \begin{itemize}
        \item $y$を$x$の
            \term{逆元}[inverse]{逆元}[ぎゃくげん]といい、
            $x^{-1}$と書く。
    \end{itemize}
\end{definition}

\begin{definition}[アーベル群]
    群$(G, +, 0)$が
    \term{アーベル群}[abelian group]{アーベル群}[あーべるぐん]であるとは、
    次が成り立つことをいう:
    \begin{description}
        \item[(A1) 可換性]
            各$x, y \in G$に対して
            $x + y = y + x$
            が成り立つ。
    \end{description}
\end{definition}

\begin{definition}[群準同型]
    \TODO{}
\end{definition}



% ------------------------------------------------------------
%
% ------------------------------------------------------------
\section{部分群}

\begin{proposition}[部分群の特徴付け]
    \TODO{}
\end{proposition}

\begin{proof}
    \TODO{}
\end{proof}

\begin{definition}[生成された部分群]
    $G$を群、$S \subset G$とする。
    このとき、集合
    \begin{equation}
        \langle S \rangle
            \coloneqq \{
                g_1^{\eps_1} \cdots g_n^{\eps_n}
                \mid
                n \in \Z_{\ge 1}, \;
                g_i \in S, \;
                \eps_i \in \{ \pm 1 \}
            \}
    \end{equation}
    は定義から明らかに$G$の部分群となる。
    $\langle S \rangle$を
    \term{$S$により生成された$G$の部分群}[subgroup of $G$ generated by $S$]
        {生成された部分群}[せいせいされたぶぶんぐん]
    といい、
    $S$を$\langle S \rangle$の
    \term{生成系}[generating set]{生成系}[せいせいけい]
    という。

    $G$が有限集合$S$により生成されるとき、
    $G = \langle S \rangle$は
    \term{有限生成}[finitely generated]{有限生成}[ゆうげんせいせい]
    であるといい、
    さらに$S$が1点集合$S = \{ x \}$のとき
    波括弧を省略して$\langle x \rangle$と書き、
    $G = \langle x \rangle$はa
    \term{巡回群}[cyclic group]{巡回群}[じゅんかいぐん]
    であるという。
\end{definition}

\begin{proposition}[生成された部分群の特徴付け]
    $G$を群、$S \subset G$とする。
    このとき
    \begin{equation}
        \langle S \rangle
            = \bigcap_{\substack{
                G' \subset G \colon \text{部分群} \\
                G' \supset S
            }} G'
    \end{equation}
    が成り立つ。
\end{proposition}

\begin{proof}
    \TODO{}
\end{proof}



% ------------------------------------------------------------
%
% ------------------------------------------------------------
\section{群作用}

群の作用について述べる。

\begin{definition}[作用]
    $G$を群、$X$を集合とする。
    写像
    \begin{equation}
        G \times X \to X,
        \quad
        (g, x) \mapsto gx
    \end{equation}
    が与えられていて
    \begin{enumerate}
        \item 各$g_1, g_2 \in G, \; x \in X$に対して
            $(g_1 g_2) x = g_1 (g_2 x)$が成り立つ。
        \item 各$x \in X$に対して$e_G x = x$が成り立つ。
    \end{enumerate}
    をみたすとき、
    $G$は$X$に左から\term{作用}[act]{作用}[さよう]するという。
    $G$が左から作用している集合を
    \term{左$G$-集合}[left $G$-set]{$G$-集合}[Gしゅうごう]
    という。
    右からの作用も同様に定まる。
\end{definition}

\begin{definition}[軌道]
    $G$を群、$X$を左$G$-集合とする。
    $X$上の同値関係を
    \begin{equation}
        \text{$x$と$y$が同値}
        \quad \logeq \quad
        \exists g \in G \quad \text{s.t.} \quad gx = y
    \end{equation}
    で定めることができ、
    この同値関係に関する同値類を
    \term{軌道}[orbit]{軌道}[きどう]
    という。
\end{definition}

\begin{definition}[固定部分群]
    \idxsym{stabilizer}{$\Stab_G(x)$}{$x$の固定部分群}
    $G$を群、$X$を左$G$-集合とする。
    各$x \in X$に対し、$G$の部分群
    \begin{equation}
        \Stab_G(x) \coloneqq \{ g \in G \colon xg = x \}
    \end{equation}
    を$x$の
    \term{固定部分群}[stabilizer]{固定部分群}[こていぶぶんぐん]
    という。
\end{definition}

\begin{definition}[忠実作用]
    $G$を群、$X$を左$G$-集合とする。
    $G$の$X$への作用が
    \term{忠実}[faithful]{忠実}[ちゅうじつ]
    あるいは
    \term{効果的}[effective]{効果的}[こうかてき]
    であるとは、次が成り立つことをいう:
    \begin{itemize}
        \item すべての$x \in X$を
            固定する$g \in G$は単位元のみである。
    \end{itemize}
    定義から明らかに、作用が忠実であることは
    作用の定める表現$G \to \Aut(X)$が単射であることと同値である。
\end{definition}

\begin{definition}[自由作用]
    $G$を群、$X$を左$G$-集合とする。
    $G$の$X$への作用が
    \term{自由}[free]{自由}[じゆう]
    であるとは、
    単位元以外の$g \in G$はすべての$x \in X$を動かすように作用すること、すなわち
    \begin{equation}
        \forall g \in G \; (g \neq 1 \Rightarrow (\forall x \in X \; (xg \neq x)))
    \end{equation}
    が成り立つことをいう。
    これはすべての$x \in X$に対し
    $\Stab_G(x)$が自明群であることと同値である。
\end{definition}

\begin{definition}[推移的作用]
    $G$を群、$X$を左$G$-集合とする。
    各$x \in X$に対し$xG \coloneqq \{ xg \in X \colon g \in G \}$と書く。
    $G$の$X$への作用が
    \term{推移的}[transitive]{推移的}[すいいてき]
    であるとは、
    \begin{equation}
        X = xG \quad (\forall x \in X)
    \end{equation}
    が成り立つことをいう。これは次と同値である:
    \begin{itemize}
        \item $\forall x_0 \in X$を固定すると、
            $\forall y \in X$に対し$\exists g \in G$がとれて$y = x_0 g$が成り立つ。
    \end{itemize}
\end{definition}

\subsection{$G$-torsor}

\begin{definition}[$G$-torsor]
    $G$を群、$X$を非空な左$G$-集合とする。
    \term{shear map}{shear map}
    と呼ばれる写像
    \begin{equation}
        G \times X \to X \times X,
        \quad
        (g, x) \mapsto (gx, x)
    \end{equation}
    が全単射であるとき、
    $X$を\term{$G$-torsor}{$G$-torsor}[G-torsor]
    という。
\end{definition}

\begin{proposition}[$G$-torsor の特徴付け]
    $G$を群、$X$を左$G$-集合とする。
    このとき、次は同値である:
    \begin{enumerate}
        \item $X$は$G$-torsorである。
        \item $G$の$X$への作用は推移的かつ自由である。
        \item $G$の$X$への作用は推移的であり、さらに
            固定部分群が自明群であるような$x \in X$が存在する。
        \item $X$と$G$は左$G$-集合として同型である。
    \end{enumerate}
\end{proposition}

\begin{proof}
    \TODO{}
\end{proof}

\begin{theorem}[類等式]
    \TODO{}
\end{theorem}

\begin{proof}
    \TODO{}
\end{proof}

\begin{theorem}[Lagrange]
    \TODO{}
\end{theorem}

\begin{proof}
    \TODO{}
\end{proof}



% ------------------------------------------------------------
%
% ------------------------------------------------------------
\section{商群}



% ------------------------------------------------------------
%
% ------------------------------------------------------------
\section{準同型定理}

\begin{theorem}[準同型定理]
    \TODO{}
\end{theorem}

\begin{proof}
    \TODO{}
\end{proof}

\begin{theorem}[部分群の対応原理]
    \TODO{}
\end{theorem}

\begin{proof}
    \TODO{}
\end{proof}


% ------------------------------------------------------------
%
% ------------------------------------------------------------
\section{Sylow の定理}

\begin{theorem}[Sylow]
    \TODO{}
\end{theorem}

\begin{proof}
    \TODO{}
\end{proof}



% ------------------------------------------------------------
%
% ------------------------------------------------------------
\section{群の表現}
\label[section]{section:group-action}

\TODO{群の作用とはどう違う?}

\begin{definition}[群の表現]
    $G$を群、$\calC$を圏とする。
    $G$は、射を群の元とし単一の対象$*$からなる圏とみなせる。
    $\calC$における$G$の
    \term{表現}[representation]{表現}[ひょうげん]
    とは、圏$G$から$\calC$への関手のことである。
    $T \colon G \to \calC$を表現とするとき、
    各射$T(g)$は$\calC$の対象$X \coloneqq T(*)$上の自己同型射を与えるから、
    群準同型$G \to \Aut(X)$が定まる。
    この群準同型も\term{表現}[representation]{表現}[ひょうげん]と呼ぶ。
\end{definition}

\begin{remark}
    群の作用は
    集合の圏における群の表現
    (これを\term{置換表現}[permutation representation]{置換表現}[ちかんひょうげん]という)
    に他ならない。
\end{remark}

\begin{example}
    ~
    \begin{itemize}
        \item 有限群の表現
        \item 位相群の表現
        \item Lie 群の表現
        \item \TODO{}
    \end{itemize}
\end{example}

% ------------------------------------------------------------
%
% ------------------------------------------------------------
\section{自由群}

% ------------------------------------------------------------
%
% ------------------------------------------------------------
\section{自由積と融合積}

% ------------------------------------------------------------
%
% ------------------------------------------------------------
\section{アーベル化}

\begin{theorem}[アーベル化の普遍性]
    \TODO{}
\end{theorem}

\begin{proof}
    \TODO{}
\end{proof}

% ------------------------------------------------------------
%
% ------------------------------------------------------------
\section{可解群}




% ============================================================
%
% ============================================================
\chapter{基本的な群}

% ------------------------------------------------------------
%
% ------------------------------------------------------------
\section{対称群}

% ------------------------------------------------------------
%
% ------------------------------------------------------------
\section{2面体群}

% ------------------------------------------------------------
%
% ------------------------------------------------------------
\section{4元数群}

% ------------------------------------------------------------
%
% ------------------------------------------------------------
\section{一般線型群}




\end{document}