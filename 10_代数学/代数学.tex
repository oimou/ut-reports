\documentclass[report, notitlepage]{jlreq}
\usepackage{docmute}
\usepackage{global}
\usepackage{./sub/local}

\makeindex
\makeglossaries

\title{代数学}
\author{Yahata}
\date{}

\begin{document}

\maketitle
\begin{abstract}
    代数学とは…何?

    第1部では群について述べる。
    第2部では環と加群について整理する。
    第3部では環と加群に関する種々の主題として
    線型代数、外積代数、代数幾何学、有限群の表現論に触れる。
    第4部では体について述べる。
\end{abstract}

% ============================================================
%
% ============================================================
\tableofcontents
\markboth{\contentsname}{}

% ============================================================
%
% ============================================================
\part{群}
\documentclass[report]{jlreq}
\usepackage{global}
\usepackage{./local}
\subfiletrue
%\makeindex
\begin{document}


% ============================================================
%
% ============================================================
\chapter{群}

群について述べる。

% ------------------------------------------------------------
%
% ------------------------------------------------------------
\section{群}

\begin{definition}[モノイド]
    $M$を集合、
    $e \in M$、
    $\cdot \colon M \times M \to M$を写像とし、
    各$x, y \in M$に対し$\cdot (x, y)$を
    $x \cdot y$や$xy$と書くことにする。
    組$(M, \cdot, e)$が
    \term{モノイド}[monoid]{モノイド}
    であるとは、次が成り立つことをいう:
    \begin{description}
        \item[(M1) 結合律]
            各$x, y, z \in M$に対して
            $(x \cdot y) \cdot z = x \cdot (y \cdot z)$
            が成り立つ。
        \item[(M2) 単位元]
            各$x \in M$に対して
            $x \cdot e = x = e \cdot x$
            が成り立つ。
    \end{description}
    組$(M, \cdot, e)$のことを
    記号の濫用で単に$(M, \cdot)$や$M$と書くことがある。
    さらに
    \begin{itemize}
        \item $e$を$M$の
            \term{単位元}[unit]{単位元}[たんいげん]という。
    \end{itemize}
\end{definition}

\begin{definition}[群]
    モノイド$(G, \cdot, e)$が
    \term{群}[group]{群}[ぐん]であるとは、
    次が成り立つことをいう:
    \begin{description}
        \item[(G1) 逆元]
            各$x \in G$に対して
            ある$y \in G$が存在して
            $x \cdot y = e = y \cdot x$
            が成り立つ。
    \end{description}
    さらに
    \begin{itemize}
        \item $y$を$x$の
            \term{逆元}[inverse]{逆元}[ぎゃくげん]といい、
            $x^{-1}$と書く。
    \end{itemize}
\end{definition}

\begin{definition}[アーベル群]
    群$(G, +, 0)$が
    \term{アーベル群}[abelian group]{アーベル群}[あーべるぐん]であるとは、
    次が成り立つことをいう:
    \begin{description}
        \item[(A1) 可換性]
            各$x, y \in G$に対して
            $x + y = y + x$
            が成り立つ。
    \end{description}
\end{definition}

\begin{definition}[群準同型]
    \TODO{}
\end{definition}



% ------------------------------------------------------------
%
% ------------------------------------------------------------
\section{部分群}

\begin{proposition}[部分群の特徴付け]
    \TODO{}
\end{proposition}

\begin{proof}
    \TODO{}
\end{proof}

\begin{definition}[生成された部分群]
    $G$を群、$S \subset G$とする。
    このとき、集合
    \begin{equation}
        \langle S \rangle
            \coloneqq \{
                g_1^{\eps_1} \cdots g_n^{\eps_n}
                \mid
                n \in \Z_{\ge 1}, \;
                g_i \in S, \;
                \eps_i \in \{ \pm 1 \}
            \}
    \end{equation}
    は定義から明らかに$G$の部分群となる。
    $\langle S \rangle$を
    \term{$S$により生成された$G$の部分群}[subgroup of $G$ generated by $S$]
        {生成された部分群}[せいせいされたぶぶんぐん]
    といい、
    $S$を$\langle S \rangle$の
    \term{生成系}[generating set]{生成系}[せいせいけい]
    という。

    $G$が有限集合$S$により生成されるとき、
    $G = \langle S \rangle$は
    \term{有限生成}[finitely generated]{有限生成}[ゆうげんせいせい]
    であるといい、
    さらに$S$が1点集合$S = \{ x \}$のとき
    波括弧を省略して$\langle x \rangle$と書き、
    $G = \langle x \rangle$はa
    \term{巡回群}[cyclic group]{巡回群}[じゅんかいぐん]
    であるという。
\end{definition}

\begin{proposition}[生成された部分群の特徴付け]
    $G$を群、$S \subset G$とする。
    このとき
    \begin{equation}
        \langle S \rangle
            = \bigcap_{\substack{
                G' \subset G \colon \text{部分群} \\
                G' \supset S
            }} G'
    \end{equation}
    が成り立つ。
\end{proposition}

\begin{proof}
    \TODO{}
\end{proof}



% ------------------------------------------------------------
%
% ------------------------------------------------------------
\section{群作用}

群の作用について述べる。

\begin{definition}[作用]
    $G$を群、$X$を集合とする。
    写像
    \begin{equation}
        G \times X \to X,
        \quad
        (g, x) \mapsto gx
    \end{equation}
    が与えられていて
    \begin{enumerate}
        \item 各$g_1, g_2 \in G, \; x \in X$に対して
            $(g_1 g_2) x = g_1 (g_2 x)$が成り立つ。
        \item 各$x \in X$に対して$e_G x = x$が成り立つ。
    \end{enumerate}
    をみたすとき、
    $G$は$X$に左から\term{作用}[act]{作用}[さよう]するという。
    $G$が左から作用している集合を
    \term{左$G$-集合}[left $G$-set]{$G$-集合}[Gしゅうごう]
    という。
    右からの作用も同様に定まる。
\end{definition}

\begin{definition}[軌道]
    $G$を群、$X$を左$G$-集合とする。
    $X$上の同値関係を
    \begin{equation}
        \text{$x$と$y$が同値}
        \quad \logeq \quad
        \exists g \in G \quad \text{s.t.} \quad gx = y
    \end{equation}
    で定めることができ、
    この同値関係に関する同値類を
    \term{軌道}[orbit]{軌道}[きどう]
    という。
\end{definition}

\begin{definition}[固定部分群]
    \idxsym{stabilizer}{$\Stab_G(x)$}{$x$の固定部分群}
    $G$を群、$X$を左$G$-集合とする。
    各$x \in X$に対し、$G$の部分群
    \begin{equation}
        \Stab_G(x) \coloneqq \{ g \in G \colon xg = x \}
    \end{equation}
    を$x$の
    \term{固定部分群}[stabilizer]{固定部分群}[こていぶぶんぐん]
    という。
\end{definition}

\begin{definition}[忠実作用]
    $G$を群、$X$を左$G$-集合とする。
    $G$の$X$への作用が
    \term{忠実}[faithful]{忠実}[ちゅうじつ]
    あるいは
    \term{効果的}[effective]{効果的}[こうかてき]
    であるとは、次が成り立つことをいう:
    \begin{itemize}
        \item すべての$x \in X$を
            固定する$g \in G$は単位元のみである。
    \end{itemize}
    定義から明らかに、作用が忠実であることは
    作用の定める表現$G \to \Aut(X)$が単射であることと同値である。
\end{definition}

\begin{definition}[自由作用]
    $G$を群、$X$を左$G$-集合とする。
    $G$の$X$への作用が
    \term{自由}[free]{自由}[じゆう]
    であるとは、
    単位元以外の$g \in G$はすべての$x \in X$を動かすように作用すること、すなわち
    \begin{equation}
        \forall g \in G \; (g \neq 1 \Rightarrow (\forall x \in X \; (xg \neq x)))
    \end{equation}
    が成り立つことをいう。
    これはすべての$x \in X$に対し
    $\Stab_G(x)$が自明群であることと同値である。
\end{definition}

\begin{definition}[推移的作用]
    $G$を群、$X$を左$G$-集合とする。
    各$x \in X$に対し$xG \coloneqq \{ xg \in X \colon g \in G \}$と書く。
    $G$の$X$への作用が
    \term{推移的}[transitive]{推移的}[すいいてき]
    であるとは、
    \begin{equation}
        X = xG \quad (\forall x \in X)
    \end{equation}
    が成り立つことをいう。これは次と同値である:
    \begin{itemize}
        \item $\forall x_0 \in X$を固定すると、
            $\forall y \in X$に対し$\exists g \in G$がとれて$y = x_0 g$が成り立つ。
    \end{itemize}
\end{definition}

\subsection{$G$-torsor}

\begin{definition}[$G$-torsor]
    $G$を群、$X$を非空な左$G$-集合とする。
    \term{shear map}{shear map}
    と呼ばれる写像
    \begin{equation}
        G \times X \to X \times X,
        \quad
        (g, x) \mapsto (gx, x)
    \end{equation}
    が全単射であるとき、
    $X$を\term{$G$-torsor}{$G$-torsor}[G-torsor]
    という。
\end{definition}

\begin{proposition}[$G$-torsor の特徴付け]
    $G$を群、$X$を左$G$-集合とする。
    このとき、次は同値である:
    \begin{enumerate}
        \item $X$は$G$-torsorである。
        \item $G$の$X$への作用は推移的かつ自由である。
        \item $G$の$X$への作用は推移的であり、さらに
            固定部分群が自明群であるような$x \in X$が存在する。
        \item $X$と$G$は左$G$-集合として同型である。
    \end{enumerate}
\end{proposition}

\begin{proof}
    \TODO{}
\end{proof}

\begin{theorem}[類等式]
    \TODO{}
\end{theorem}

\begin{proof}
    \TODO{}
\end{proof}

\begin{theorem}[Lagrange]
    \TODO{}
\end{theorem}

\begin{proof}
    \TODO{}
\end{proof}



% ------------------------------------------------------------
%
% ------------------------------------------------------------
\section{商群}



% ------------------------------------------------------------
%
% ------------------------------------------------------------
\section{準同型定理}

\begin{theorem}[準同型定理]
    \TODO{}
\end{theorem}

\begin{proof}
    \TODO{}
\end{proof}

\begin{theorem}[部分群の対応原理]
    \TODO{}
\end{theorem}

\begin{proof}
    \TODO{}
\end{proof}


% ------------------------------------------------------------
%
% ------------------------------------------------------------
\section{Sylow の定理}

\begin{theorem}[Sylow]
    \TODO{}
\end{theorem}

\begin{proof}
    \TODO{}
\end{proof}



% ------------------------------------------------------------
%
% ------------------------------------------------------------
\section{群の表現}
\label[section]{section:group-action}

\TODO{群の作用とはどう違う?}

\begin{definition}[群の表現]
    $G$を群、$\calC$を圏とする。
    $G$は、射を群の元とし単一の対象$*$からなる圏とみなせる。
    $\calC$における$G$の
    \term{表現}[representation]{表現}[ひょうげん]
    とは、圏$G$から$\calC$への関手のことである。
    $T \colon G \to \calC$を表現とするとき、
    各射$T(g)$は$\calC$の対象$X \coloneqq T(*)$上の自己同型射を与えるから、
    群準同型$G \to \Aut(X)$が定まる。
    この群準同型も\term{表現}[representation]{表現}[ひょうげん]と呼ぶ。
\end{definition}

\begin{remark}
    群の作用は
    集合の圏における群の表現
    (これを\term{置換表現}[permutation representation]{置換表現}[ちかんひょうげん]という)
    に他ならない。
\end{remark}

\begin{example}
    ~
    \begin{itemize}
        \item 有限群の表現
        \item 位相群の表現
        \item Lie 群の表現
        \item \TODO{}
    \end{itemize}
\end{example}

% ------------------------------------------------------------
%
% ------------------------------------------------------------
\section{自由群}

% ------------------------------------------------------------
%
% ------------------------------------------------------------
\section{自由積と融合積}

% ------------------------------------------------------------
%
% ------------------------------------------------------------
\section{アーベル化}

\begin{theorem}[アーベル化の普遍性]
    \TODO{}
\end{theorem}

\begin{proof}
    \TODO{}
\end{proof}

% ------------------------------------------------------------
%
% ------------------------------------------------------------
\section{可解群}




% ============================================================
%
% ============================================================
\chapter{基本的な群}

% ------------------------------------------------------------
%
% ------------------------------------------------------------
\section{対称群}

% ------------------------------------------------------------
%
% ------------------------------------------------------------
\section{2面体群}

% ------------------------------------------------------------
%
% ------------------------------------------------------------
\section{4元数群}

% ------------------------------------------------------------
%
% ------------------------------------------------------------
\section{一般線型群}




\end{document}

\part{環と加群}
\documentclass[report]{jlreq}
\usepackage{global}
\usepackage{./local}
\subfiletrue
%\makeindex
\begin{document}

% ============================================================
%
% ============================================================
\chapter{環}

環の基礎事項について述べる。

% ------------------------------------------------------------
%
% ------------------------------------------------------------
\section{環}

\begin{definition}[環]
    \idxsym{A}{$A$}{一般の環}
    \idxsym{R}{$R$}{可換環}
    組$(A, +, \cdot, 0, 1)$が\term{環}[ring]{環}[かん]であるとは、
    次が成り立つことをいう\footnote{
        文献によっては環の定義に単位元の存在を仮定しない立場もある。
    }:
    \begin{description}
        \item[(R1)] $(A, +, 0)$がアーベル群
        \item[(R2)] $(A, \cdot, 1)$がモノイド
        \item[(R3) 分配律]
            $\forall x, y, z \in A$に対し次が成り立つ:
            \begin{equation}
                \begin{cases}
                    x \cdot (y + z) = x \cdot y + x \cdot z \\
                    (x + y) \cdot z = x \cdot z + y \cdot z
                \end{cases}
            \end{equation}
    \end{description}
    さらに
    \begin{itemize}
        \item $0$を$A$の\term{零元}[zero element]{零元}[れいげん]といい、
            $0_A$とも書く。
        \item $1$を$A$の\term{単位元}[unit element]{単位元}[たんいげん]といい、
            $1_A$とも書く。
    \end{itemize}
\end{definition}

\begin{remark}
    環の元が乗法に関する左逆元を持つとしても
    右逆元を持つとは限らない。
\end{remark}

\begin{example}[可換環の例]
    ~
    \begin{itemize}
        \item (零環) 1点集合$\{0\}$には環構造が一意に定まる。これを
            \term{零環}[zero ring]{零環}[れいかん]という。
            環$A$が零環であることと$0_A = 1_A$であることとは同値である。
        \item (整数環) 有理整数環$\Z$や Gauss 整数環$\Z[i]$は可換環である。
        \item (関数環) 位相空間$X$上の$\C$値連続関数全体のなす環$C(X)$は
            点ごとの和と積を演算として可換環となる。
    \end{itemize}
\end{example}

\begin{example}[非可換環の例]
    \idxsym{ring of endomorphisms}{$\End(A)$}{アーベル群$A$の自己準同型環}
    ~
    \begin{itemize}
        \item (全行列環) 環$A$の全行列環$M_n(A)$は一般に非可換である。
        \item (自己準同型環) アーベル群$A$ (より一般に環上の加群) の自己準同型環$\End(A)$は
            点ごとの和と写像の合成を演算として環となる。
            これは一般に非可換である。
    \end{itemize}
\end{example}

\begin{remark}
    一般の環を$A$、可換環を$R$で書くことが多い。
\end{remark}

\begin{example}[測度論との関連]
    \TODO{ring of sets などの話をしたい}
\end{example}

\subsection{反転環}

反転環を定義する。

\begin{definition}[反転環]
    \idxsym{Opposite Ring}{$A^\OP$}{反転環}
    $(A, +, \cdot)$を環とする。
    集合$A^\OP$を$A^\OP \coloneqq A$とおき、
    アーベル群$(A^\OP, +)$に乗法$\cdot'$を
    \begin{equation}
        a \cdot' b \coloneqq b \cdot a
        \quad (a, b \in A^\OP)
    \end{equation}
    で定めると、$(A^\OP, +, \cdot')$は環をなす。
    この環を$A$の\term{反転環}[opposite ring]{反転環}[はんてんかん]という。
\end{definition}

\begin{example}[反転環の例]
    ~
    \begin{itemize}
        \item $R$を可換環とすると、写像
            \begin{equation}
                M_n(R) \to M_n(R)^\OP,
                \quad X \mapsto {}^t\!X
            \end{equation}
            は環同型となる。
    \end{itemize}
\end{example}

\subsection{零因子と整域}

零因子と、零因子を用いて定義される整域の概念を導入する。

\begin{definition}[零因子と整域]
    $A$を環とする。
    $a \in A, a \neq 0$が次をみたすとき、
    $a$は$A$の\term{零因子}[zero divisor]{零因子}[れいいんし]であるという:
    \begin{equation}
        \exists x \in A, x \neq 0
        \quad \text{s.t.} \quad
        ax = 0 \text{ or } xa = 0
    \end{equation}
    また、$A$が次のすべてをみたすとき、
    $A$を\term{整域}[integral domain]{整域}[せいいき]という:
    \begin{description}
        \item[(I1)] $A$は零環でない。
        \item[(I2)] $A$は可換環である。
        \item[(I3)] 零因子を持たない。
    \end{description}
\end{definition}

\begin{example}[零因子と整域の例]
    ~
    \begin{itemize}
        \item $\Z$は整域である。
        \item $\Z \times \Z$は整域でない ((I1), (I2) をみたすが (I3) をみたさない)。
            零因子の例のひとつは
            \begin{equation}
                (1, 0) \cdot (0, 1) = (0, 0)
            \end{equation}
            である。
        \item Hamilton の四元数環$\H$は整域でない ((I1), (I3) をみたすが (I2) をみたさない)。
    \end{itemize}
\end{example}

\subsection{可除環と体}

可除環を定義する。

\begin{definition}[乗法群]
    \TODO{}
\end{definition}

\begin{definition}[可除環]
    $A$を環とする。
    $A^\times = A \setminus \{0\}$であるとき、
    $A$を\term{可除環}[division ring]{可除環}[かじょかん]あるいは
    \term{斜体}[skew field]{斜体}[しゃたい]という。
    可換な可除環を
    \term{体}[field]{体}[たい]という。
\end{definition}

\begin{example}[可除環の例]
    ~
    \begin{itemize}
        \item 零環$A = \{ 0 \}$は
            $A^\times = \{ 0 \} \neq \emptyset = A \setminus \{ 0 \}$
            より可除環ではない。
        \item Hamilton の四元数環$\H$は可除環である。
            しかし非可換なので体でも整域でもない。
        \item $\R$や$\C$は可除環である。さらに可換なので体でもある。
    \end{itemize}
\end{example}

可除環に対する次の性質は基本的である。

\begin{proposition}
    可除環は零因子を持たない。
\end{proposition}

\begin{proof}
    可除環$A$が零因子$x \neq 0$を持ったとすると、
    ある$y \in A - \{ 0 \}$が存在して$xy = 0$が成り立つ。
    一方$A$は可換環だからある$x' \in A$が存在して$x'x = 1$が成り立つ。
    よって$y = 1y = x'xy = x'0 = 0$となり
    $y \neq 0$に矛盾。
    したがって$A$は零因子を持たない。
\end{proof}

\subsection{冪等元}

\begin{definition}[冪等元]
    $A$を環とする。
    $e \in A$が$e^2 = e$を満たすとき、
    $e$を$A$の\term{冪等元}[idempotent]{冪等元}[べきとうげん]という。
    $A$の中心に属する冪等元をとくに
    \term{中心冪等元}[central idempotent]{中心冪等元}[ちゅうしんべきとうげん]という。
\end{definition}

\begin{proposition}[中心冪等元により生成される環]
    $A$を環とする。
    $e \in A$が$A$の中心冪等元であるとき、
    $Ae$は$A$の両側イデアルとなり、さらに$e$を単位元とする環となる。
\end{proposition}

\begin{proof}
    省略
\end{proof}

\begin{definition}[環準同型]
    \TODO{}
\end{definition}



% ------------------------------------------------------------
%
% ------------------------------------------------------------
\section{イデアルと商環}

イデアルの概念を導入する。

\begin{definition}[可換環のイデアル]
    $R$を可換環とする。
    $I \subset R$が$R$の\term{イデアル}[ideal]{イデアル}[いである]であるとは、
    \begin{description}
        \item[(I1)] $I$は$R$の加法部分群
        \item[(I2)] $a \in R,\; b \in I$ならば$ab \in I$
    \end{description}
    をみたすことをいう。
\end{definition}

一般の環においては
イデアルに左/右/両側の区別がある。

\begin{definition}[一般の環のイデアル]
    $A$を環とする。
    $I \subset R$が$R$の
    \term{左イデアル}[left ideal]{左イデアル}[ひだりいである]であるとは、
    上の (I1) と
    \begin{description}
        \item[(LI2)] $a \in A,\; b \in I$ならば$ab \in I$
    \end{description}
    をみたすことをいう。
    \term{右イデアル}[right ideal]{右イデアル}[みぎいである]も
    同様に定義される。
    $A$の左かつ右イデアルを
    \term{両側イデアル}[two-sided ideal]{両側イデアル}[りょうがわいである]
    という。
    明らかに可換環のイデアルは左かつ右かつ両側イデアルである。
\end{definition}

\begin{definition}[固有イデアル]
    $A$を環、
    $I \subset A$を左/右/両側イデアルとする。
    $I$が\term{固有}[proper]{固有}[こゆう]であるとは、
    $I \neq A$であることをいう。
\end{definition}

環準同型とイデアルの間には次の関係がある。

\begin{theorem}[環準同型とイデアル]
    \label[theorem]{thm:ring-hom-and-ideals}
    $A, B$を環、
    $f \colon A \to B$を環準同型とする。
    \begin{enumerate}
        \item $J$が$B$の両側イデアルならば、
            $f^{-1}(J)$は$A$の両側イデアルである。
        \item $f$が全射で$I$が$A$の両側イデアルならば、
            $f(I)$は$B$の両側イデアルである。
        \item $\Ker f$は$A$の両側イデアルである。
        \item $f$が全射ならば、$\Im f$は$B$の両側イデアルである。
        \item $\Im f$は$B$の部分環である。
    \end{enumerate}
\end{theorem}

\begin{proof}
    \uline{(1), (2)} \quad
    \TODO{}

    \uline{(3), (4)} \quad
    (1), (2) の特別な場合である。

    \uline{(5)} \quad
    \TODO{}
\end{proof}

環を両側イデアルで割った商は環をなす。

\begin{definition}[商環]
    \TODO{}
\end{definition}

環の両側イデアルは商環の両側イデアルと次の定理のように対応する。
\TODO{束の同型?}

\begin{theorem}[両側イデアルの対応原理]
    \label[theorem]{thm:ideal-correspondence-principle}
    $A$を環、$I \subset A$を両側イデアルとする。
    \begin{alignat}{1}
        \scrI_I(A) &\coloneqq \{
            J \colon \text{$J$は$I$を含む$A$の両側イデアル}
        \} \\
        \scrI(A/I) &\coloneqq \{
            J \colon \text{$J$は$A/I$の両側イデアル}
        \}
    \end{alignat}
    とおくと、
    \begin{equation}
        \wt{p} \colon \scrI_I(A) \to \scrI(A/I),
        \quad J \mapsto p(J)
    \end{equation}
    は包含関係を保つ全単射であり、
    $\wt{p}$の逆写像$q$は
    \begin{equation}
        q \colon \scrI(A / I) \to \scrI_I(A),
        \quad J' \mapsto p^{-1}(J')
    \end{equation}
    で与えられる。
\end{theorem}

\begin{remark}
    定理より、$A / I$の両側イデアルは
    $I$を含む$A$の両側イデアル$J$を用いて
    $J / I$の形に一意的に書けて、
    さらに$A$の両側イデアル$J, J'$に関し
    \begin{equation}
        J' \subset J
        \quad \iff \quad
        J' / I \subset J / I
    \end{equation}
    が成り立つことがわかる。
    さらに、包含関係を保つことと$JJ' / I = (J / I)(J' / I)$より、
    素イデアルは素イデアルと、極大イデアルは極大イデアルと
    それぞれ対応することもわかる。
\end{remark}

\begin{proof}
    $q$が well-defined であることを示す。
    $A / I$の両側イデアル$J'$に対し
    $p^{-1}(J')$が$A$の両側イデアルであることは
    \cref{thm:ring-hom-and-ideals}より成り立ち、
    また$I = p^{-1}(0) \subset p^{-1}(J')$も成り立つ。
    よって$q$は well-defined である。

    $q$が$\wt{p}$の逆写像であることを示す。
    $J' \in \scrI(A / I)$に対し
    $p(p^{-1}(J')) = J'$であることは
    $p$の全射性より従う。
    $J \in \scrI_I(A)$に対し
    $p^{-1}(p(J)) = J$であることを示す。
    "$\supset$"は集合の一般論より成り立つ。
    逆に$x \in p^{-1}(p(J))$とすると、
    ある$j \in J$が存在して$p(x) = p(j)$となる。
    よって$p(x - j) = 0$だから
    $x - j \in p^{-1}(0) = I \subset J$である。
    したがって$x = (x - j) + j \in J$が成り立つから
    "$\subset$"もいえた。
    よって$q$は$\wt{p}$の逆写像である。

    $\wt{p}$が包含を保つことは
    写像による部分集合の像と逆像が包含を保つことから従う。
    以上で定理の主張が示された。
\end{proof}

イデアルの演算について述べる。

\begin{definition}[加法部分群の和と積]
    \idxsym{I+J}{$I + J$}{加法部分群の和}
    \idxsym{IJ}{$IJ$}{加法部分群の積}
    $A$を環、
    $I, J \subset A$を加法部分群とする。
    \begin{alignat}{1}
        I + J \coloneqq \{ a + b
            \mid
            a \in I, \;
            b \in J
        \} \\
        IJ \coloneqq \left\{
            \sum_{i=1}^n a_i b_i
            \mid
            n \in \Z_{\ge 1}, \;
            a_i \in I, \;
            b_i \in J
        \right\}
    \end{alignat}
    と書く。
\end{definition}

\begin{proposition}[イデアルの和と積]
    $A$を環とする。
    \begin{enumerate}
        \item $A$の任意の左 (resp. 右, 両側) イデアル$I, J \subset A$に対し、
            $I + J$は$A$の左 (resp. 右, 両側) イデアルである。
        \item $A$の任意の両側イデアル$I, J \subset A$に対し、
            $I + J$は$A$の両側イデアルである。
    \end{enumerate}
\end{proposition}

\begin{proof}
    \TODO{}
\end{proof}

\begin{proposition}
    $A$を環、
    $I \subset J$を$A$の両側イデアルとする。
    このとき、環の同型
    \begin{equation}
        \frac{A / I}{J / I} \cong \frac{A}{J}
    \end{equation}
    が成り立つ。
\end{proposition}

\begin{proof}
    図式
    \begin{equation}
        \begin{tikzcd}
            A
                \ar{r}{\pi}
                \ar{d}[swap]{p}
                & A / I
                    \ar{d}{q} \\
            A / J
                \ar[dashed]{r}[swap]{f}
                & \frac{A / I}{J / I}
        \end{tikzcd}
    \end{equation}
    を可換にする環の同型が誘導されることを示せばよく、
    そのためには$J = \Ker q \circ \pi$をいえばよい。
    $j \in J$とすると
    $j + I \in J + I$だから
    $q \circ \pi(j) = q(j + I) = 0$である。
    よって$J \subset \Ker q \circ \pi$である。
    逆に$q \circ \pi(a) = 0 \; (a \in A)$とすると
    $\pi(a) \in J / I$だから、
    両側イデアルの対応原理 (\cref{thm:ideal-correspondence-principle})
    より$a \in \pi^{-1}(J / I) = J$である。
    したがって$J = \Ker q \circ \pi$がいえた。
    準同型定理より上の図式を可換にする環の同型$f$が誘導されて
    証明が完成した。
\end{proof}

\begin{proposition}[直積環のイデアル]
    $A, B$を環とする。
    $A \times B$の任意の左 (resp. 右/両側) イデアル$J$は
    $A, B$のある左 (resp. 右/両側) イデアル$J_A, J_B$を用いて
    $J = J_A \times J_B$の形に書ける。
\end{proposition}

\begin{proof}
    \TODO{}
\end{proof}

部分集合により生成されるイデアルについて述べる。
\TODO{$Ax$とかの記法は?}

\begin{definition}[生成されたイデアル]
    $R$を可換環、
    $n \in \Z_{\ge 1}$とする。
    \begin{itemize}
        \item $a_1, \dots, a_n \in A$に対し
            \begin{equation}
                (a_1, \dots, a_n) \coloneqq \{
                    b_1 a_1 + \dots + b_n a_n
                    \colon b_1, \dots, b_n \in R
                \}
            \end{equation}
            は$R$のイデアルである。これを$a_1, \dots, a_n$で
            \term{生成されたイデアル}{生成されたイデアル}[せいせいされたいである]という。
        \item イデアル$I \subset R$が
            有限個の元により生成されたイデアルであるとき、
            $I$は\term{有限生成}[finitely generated]
            {有限生成!イデアルとして---}[ゆうげんせいせい]であるという。
            とくに$I$が1個の元により生成されるとき、
            $I$を\term{単項イデアル}[principal ideal]{単項イデアル}[たんこういである]という。
    \end{itemize}
\end{definition}

\begin{proposition}[生成されたイデアルの特徴付け]
    \label[proposition]{prop:generated-ideal-characterization}
    \TODO{}
\end{proposition}

\begin{proof}
    \TODO{}
\end{proof}



% ------------------------------------------------------------
%
% ------------------------------------------------------------
\section{中国剰余定理}

中国剰余定理はイデアルの性質に関する定理であり、
環論において最も重要かつ基本的な定理のひとつである。

\begin{theorem}[中国剰余定理]
    \label[theorem]{thm:chinese-remainder-theorem}
    $A$を環、
    $I_1, \dots, I_n$を$A$の両側イデアルとする。
    $I_i$らは\term{互いに素}{互いに素}[たがいにそ]、
    すなわち$I_i + I_j = A \; (I \neq j)$をみたすとする。
    このとき、準同型定理によって図式
    \begin{equation}
        \begin{tikzcd}
            A \ar{d} \ar{r} & A / I_1 \times \cdots \times A / I_n \\
            A / (I_1 \cap \cdots \cap I_n) \ar[dashed]{ru}
        \end{tikzcd}
    \end{equation}
    の破線部に誘導される環準同型は環の同型を与える。
\end{theorem}

\begin{proof}
    次のように写像に名前をつける:
    \begin{equation}
        \begin{tikzcd}
            A \ar{d} \ar{r}{p = p_1 \times \dots \times p_n}
                & A / I_1 \times \cdots \times A / I_n \\
            A / (I_1 \cap \cdots \cap I_n) \ar[dashed]{ru}[swap]{\wb{p}}
        \end{tikzcd}
    \end{equation}
    $\Ker \wb{p} = I_1 \cap \cdots \cap I_n$より
    $\wb{p}$の単射性は明らか。
    $\wb{p}$の全射性を$n$に関する帰納法によって示す。
    $n = 2$の場合を考える。
    \begin{align}
        e_1 &\coloneqq (1, 0) \in A / I_1 \times A / I_2 \\
        e_2 &\coloneqq (0, 1) \in A / I_1 \times A / I_2
    \end{align}
    とおき、$e_1, e_2 \in \Im \wb{p}$をいえばよい。
    $I_1, I_2$は互いに素ゆえに
    $x + y = 1$なる$x \in I_1, \; y \in I_2$が存在する。
    このとき
    \begin{alignat}{1}
        p_1(x) &= 0 \\
        p_2(x) &= p_2(1 - y) = p_2(1) = 1
    \end{alignat}
    よって
    \begin{equation}
        p(x) = (p_1(x), p_2(x)) = (0, 1) = e_2 \in \Im \wb{p}
    \end{equation}
    が成り立つ。
    同様に$e_1 \in \Im \wb{p}$も成り立つ。
    \TODO{}
\end{proof}

さらに環が可換の場合は次が成り立つ。

\begin{theorem}
    $R$を可換環とし、$I_1, \dots, I_n$をイデアルとする。
    $I_i$らは互いに素、すなわち$I_i + I_j = A \; (I \neq j)$をみたすとする。
    このとき、
    \begin{equation}
        I_1 \cap \cdots \cap I_n = I_1 \cdots I_n
    \end{equation}
    が成り立つ。
\end{theorem}

\begin{proof}
    cf. \cref{problem:algebra-2.31}
\end{proof}



% ------------------------------------------------------------
%
% ------------------------------------------------------------
\section{極大イデアル}

極大イデアルを定義する。

\begin{definition}[極大イデアル]
    $A$を環、$I \subset A$を左イデアルとする。
    $I$が
    \term{極大左イデアル}[maximal left ideal]{極大左イデアル}[きょくだいひだりいである]
    であるとは、
    $I$が$A$の固有左イデアルのうち包含関係に関し極大であることをいう。
    極大右イデアルおよび極大両側イデアルも同様に定義される。
    左/右/両側が文脈から明らかな場合は省略して
    \term{極大イデアル}[maximal ideal]{極大イデアル}[きょくだいいである]
    ということがある。
\end{definition}

\begin{theorem}[Krull の定理]
    \label[theorem]{thm:krull}
    $A \neq 0$を環、
    $I \subset A$を固有両側 (resp. 左, 右) イデアルとする。
    このとき、$I$を含む極大両側 (resp. 左, 右) イデアルが存在する。
\end{theorem}

\begin{proof}
    Zorn の補題を用いる。
    \TODO{}
\end{proof}

極大イデアルによって定義される環のクラスのうち
最も素朴なものが単純環である。
単純環については
\cref{chapter:semisimple-module-and-semisimple-ring}
でより詳しく調べる。

\begin{definition}[単純環]
    $(0)$が極大両側イデアルとなる環を
    \term{単純環}[simple ring]{単純環}[たんじゅうかん]という。
\end{definition}

\begin{example}[単純環の例]
    \label[example]{example:simple-ring}
    ~
    \begin{itemize}
        \item division ring は単純環である。
        \item 単純環の部分環は単純であるとは限らない。
            例えば、$\Z \subset \Q$は単純環でないし、
            $\Z[i, j, k] \subset \H$も単純環でない。
    \end{itemize}
\end{example}

\TODO{森田同値の現れ?}

\begin{theorem}[単純環の全行列環は単純環]
    $A$が単純環のとき、$M_n(A)$も単純環である。
\end{theorem}

\begin{proof}
    \TODO{}
\end{proof}

極大イデアルは商環により特徴付けられる。

\begin{theorem}[極大イデアルと商環]
    $A$を環とする。
    このとき、$A$の両側イデアル$I$に関し
    $I$が極大両側イデアルであることと、
    $A/I$が単純環であることとは同値である。
\end{theorem}

\begin{proof}
    両側イデアルの対応原理 (\cref{thm:ideal-correspondence-principle})
    より明らか。
\end{proof}

極大両側イデアルは極大左/右イデアルとは限らないが、
商環が可除となる場合には次のように特徴づけることができる。

\begin{theorem}[極大イデアルと可除環]
    $A$を環とする。
    $A$の両側イデアル$I$に関して次は同値である:
    \begin{enumerate}
        \item $I$は極大左イデアルである。
        \item $I$は極大右イデアルである。
        \item $A/I$は可除環である。
    \end{enumerate}
\end{theorem}

\begin{proof}
    \uline{(1) \Rightarrow (3)} \quad
    $A$の両側イデアル$I$が
    極大左イデアルであるとする。
    $A / I$の零でない元は$a + I \; (a \in A - I)$と表せて、
    $A$の左イデアル$Aa + I$は極大左イデアル$I$を真に含むから
    $Aa + I = A$である。
    よって$(b + I)(a + I) = ba + I = 1 + I$なるある$b \in A - I$が存在する。
    $A$の左イデアル$Ab + I$は極大左イデアル$I$を真に含むから
    $Ab + I = A$である。
    よって$(c + I)(b + I) = cb + I = 1 + I$なるある$c \in A - I$が存在する。
    したがって$b + I \in (A / I)^\times$であり、
    とくに$a + I$は$b + I$の逆元となるから
    $a + I \in (A / I)^\times$が従う。
    いま$a + I$は$A / I$の零でない任意の元であったから、
    $A / I$は可除環であることがいえた。

    \uline{(3) \Rightarrow (1)} \quad
    $A / I$を可除環とする。
    $J \supset I$を左イデアルとする。
    $I \subsetneq J$よりある$j \in J - I$が存在する。
    いま$A / I$は可除環だから
    $1 + I = (a + I)(j + I) = aj + I$なるある$a \in A$が存在する。
    よって$1 - aj = i$なるある$i \in I \subset J$が存在する。
    $J$が左イデアルゆえに$aj \in J$であることとあわせて
    $1 = aj + i \in J$が従う。
    よって$J = A$となり、
    $I$が極大左イデアルであることがいえた。

    \uline{(2) \Leftrightarrow (3)} \quad
    (1) \Leftrightarrow (3) の議論と同様。
\end{proof}

\begin{corollary}
    可換環$R$と$R$のイデアル$I$について、
    $I$が極大イデアルであることと、
    $R/I$が体であることとは同値である。
    \qed
\end{corollary}






% ============================================================
%
% ============================================================
\chapter{代数}

可換環上の代数の基礎事項について述べる。

% ------------------------------------------------------------
%
% ------------------------------------------------------------
\section{代数}

代数とは、
和と積とスカラー倍について閉じている代数系のことである。
代数の定義にはいくつかのやり方があるが、
ここでは特別な環としての定義を採用する。

\begin{definition}[代数]
    \idxsym{$R$-alg homomorphisms}{$\Hom_R^\al(A, B)$}
        {$R$-代数準同型$A \to B$全体の集合}
    $A$を環、
    $R$を可換環とする。
    \begin{itemize}
        \item 環$A$と環準同型$\varphi \colon R \to Z(A)$の組
            $(A, \varphi)$を
            \term{$R$-代数}[$R$-algebra]{代数}[だいすう]
            あるいは
            \term{$R$-多元環}[$R$-algebra]{多元環}[たげんかん]という\footnote{
                文献によっては代数にスカラー倍の結合性を仮定しない立場もある。
                翻って本稿における代数の定義では結合性が自動で導かれる。
                結合性を仮定するという立場を明確にするために
                \term{結合的代数}[associative algebra]{結合的代数}[けつごうてきだいすう]
                と呼ばれることもある。
            }。
            記号の濫用で
            $A$における$x \in R$の像も$x$と書くことがある。
        \item $(A, \varphi), (B, \psi)$を$R$-代数とする。
            環準同型$f \colon A \to B$が図式
            \begin{equation}
                \begin{tikzcd}
                    & R \ar{ld}[swap]{\phi} \ar{rd}{\psi} \\
                    A \ar{rr}[swap]{f} && B
                \end{tikzcd}
            \end{equation}
            を可換にするとき、$f$は
            \term{$R$-代数準同型}[$R$-algebra homomorphism]
            {代数準同型}[だいすうじゅんどうけい]
            であるという。
        \item $A \to B$なる$R$-代数準同型全体の集合を
            $\Hom_R^{\mathrm{al}} (A, B)$と書く。
    \end{itemize}
\end{definition}

\begin{example}[代数の例]
    ~
    \begin{itemize}
        \item $A$を環とする。
            $A$は環準同型
            \begin{equation}
                \Z \to A,
                \quad
                n \mapsto \underbrace{1 + \cdots + 1}_{n \text{ times}}
            \end{equation}
            により$\Z$-代数となる。
        \item 可換環$R$上の$n$次正方行列の全体$M_n(R)$は、
            環準同型$R \to Z(M_n(R)),\lambda \mapsto \lambda I_n$により
            $R$-代数となる。
            これは非可換な$R$-代数の例となっている。
        \item 環$\C$は、環準同型$\R \to \C, x \mapsto x$により$\R$-代数となる。
        \item 環$\C \times \C$は、
            環準同型$\C \to \C \times \C, z \mapsto (z, z)$により$\C$-代数となる。
        \item 位相空間$X$上の$\C$値連続関数全体のなす環$C(X)$は、環準同型$\C \to C(X),$
            \begin{equation}
                \lambda \mapsto (x \mapsto \lambda)
            \end{equation}
            により$\C$-代数となる。
    \end{itemize}
\end{example}

\begin{definition}[$R$-部分代数]
    $R$を可換環、
    $(A, \varphi)$を$R$-代数とする。
    $A$の部分環$B$が$\varphi$により$R$-代数となるとき、
    $B$を$A$の\term{$R$-部分代数}[$R$-subalgebra]{$R$-部分代数}[ぶぶんだいすう]
    という。
\end{definition}

体上の代数には体が埋め込まれているとみなせる。

\begin{proposition}[代数への体の埋め込み]
    \label[proposition]{prop:embedding-of-field-into-algebra}
    $K$を体とする。
    このとき、$0$でない任意の$K$-代数$(A, \varphi)$に対し
    $\varphi$は単射である。
\end{proposition}

\begin{proof}
    $\varphi$は環準同型$K \to Z(A)$だから、
    $K$が体であることより
    $\Ker \varphi = K$または$\Ker \varphi = 0$である。
    $\Ker \varphi = K$であったとすると
    $1_A = \varphi(1_K) = 0_A$より
    $A = 0$となり矛盾。
    したがって$\varphi = 0$、すなわち$\varphi$は単射である。
\end{proof}



% ------------------------------------------------------------
%
% ------------------------------------------------------------
\section{モノイド代数と群環}

モノイド代数と群環を定義する。
モノイド代数は後で定義する多項式環の一般化である。

\begin{definition}[モノイド代数]
    \idxsym{RM}{$R[M]$}{$R$上の$M$のモノイド代数}
    \idxsym{finite sum}{$\fsum$}{形式的実質的有限和}
    $M$をモノイド、
    $R \neq 0$を可換環とする。
    集合$R[M]$を
    \begin{equation}
        R[M] \coloneqq \left\{
            \fsum_{m \in M} a_m \cdot m
            \; \bigm\vert \;
            a_m \in R
        \right\}
    \end{equation}
    とおく。ただし$\fsum$は
    \term{形式的実質的有限和}[formal essential finite sum]
    {形式的実質的有限和}[けいしきてきじっしつてきゆうげんわ]
    といい、有限個の$m$を除いて$a_m = 0$となる和である。
    $R[M]$に加法と乗法を
    \begin{alignat}{1}
        \left( \fsum_{m \in M} a_m \cdot m \right)
            + \left( \fsum_{m \in M} b_m \cdot m \right)
            &\coloneqq \fsum_{m \in M} (a_m + b_m) \cdot m \\
        \left( \fsum_{m \in M} a_m \cdot m \right)
            \cdot \left( \fsum_{m \in M} b_m \cdot m \right)
            &\coloneqq \fsum_{m \in M} \left(
                \fsum_{\substack{x, y \in M \\ xy = m}} a_x b_y
            \right) \cdot m
    \end{alignat}
    で定めると、
    $\left( R[M], +, \cdot, \fsum_{m \in M} 0 \cdot m, 1 \cdot 1_M \right)$
    は環となる (このあと示す)。
    さらに環準同型
    \begin{equation}
        R \to R[M],
        \quad
        r \mapsto r \cdot 1_M
    \end{equation}
    により$R$-代数の構造が入る (このあと示す)。
    $R$-代数$R[M]$を$R$上の$M$の
    \term{モノイド代数}[monoid algebra]{モノイド代数}[ものいどだいすう]
    という。
\end{definition}

\begin{proof}
    \TODO{}
\end{proof}

\begin{definition}[群代数]
    \idxsym{RG}{$R[G]$}{$R$上の$G$の群環}
    $G$を群、
    $R \neq 0$を可換環とする。
    モノイド代数$R[G]$を
    $R$上の$G$の
    \term{群代数}[group algebra]{群代数}[ぐんだいすう]
    あるいは
    \term{群環}[group ring]{群環}[ぐんかん]
    という。
\end{definition}

$R, M$から$R[M]$への自然な埋め込みが次のように定まる。

\begin{proposition}[モノイド代数への埋め込み]
    $R \neq 0$を可換環、
    $M$をモノイドとする。
    このとき、写像
    \begin{alignat}{1}
        M \to R[M],
            \quad
            &x \mapsto 1_R \cdot x \\
        R \to R[M],
            \quad
            &r \mapsto r \cdot 1_M \\
    \end{alignat}
    はそれぞれ (乗法的) モノイド準同型、$R$-代数準同型となる。
\end{proposition}

\begin{proof}
    省略
\end{proof}

\begin{proposition}[モノイド代数の加群構造]
    $R \neq 0$を可換環、
    $M$をモノイドとする。
    このとき、
    $R[M]$は$M$を基底とする$R$上の自由$R$-加群である。
\end{proposition}

\begin{proof}
    \TODO{}
\end{proof}

モノイド代数は次の普遍性を持つ。

\begin{theorem}[モノイド代数の普遍性]
    $R \neq 0$を可換環、
    $M$をモノイドとする。
    このとき次が成り立つ:
    \begin{alignat}{1}
        &\forall \; A
            \colon \text{ $R$-代数} \\
        &\forall \; \varphi \colon M \to A
            \colon \text{ (乗法的) モノイド準同型} \\
        &\exists! \; \wb{\varphi} \colon R[M] \to A
            \colon \text{ $R$-代数準同型}
            \quad \text{s.t.} \quad \\
        &\quad
            \begin{tikzcd}[ampersand replacement=\&]
                R[M]
                    \ar[dashed]{r}{\varphi}
                    \& A \\
                M
                    \ar{u}
                    \ar{ur}[swap]{\wb{\varphi}}
            \end{tikzcd}
    \end{alignat}
\end{theorem}

\begin{proof}
    $\varphi$がモノイド準同型であることより
    $\wb{\varphi}$は
    \begin{equation}
        \wb{\varphi}\left(
            \fsum_{m \in M} a_m \cdot m
        \right)
            = \fsum_{m \in M} a_m \cdot \varphi(m)
            \quad
            (a_m \in R)
    \end{equation}
    をみたさなければならないが、
    上の命題より$M$は$R[M]$の$R$-加群としての基底だから、
    このような$R$-加群準同型$\wb{\varphi}$は一意に存在する。
    あとは$\wb{\varphi}$が$R$-代数準同型であることを示せばよい。
    \TODO{cf. \cite[p.5]{Pie82}}
\end{proof}

\begin{corollary}[群代数の普遍性]
    $R \neq 0$を可換環、
    $G, G'$を群、
    $\varphi \colon G \to G'$を群準同型とする。
    このとき、$R$-代数準同型$h \colon R[G] \to R[G']$であって
    次をみたすものが一意に存在する:
    \begin{alignat}{1}
        h(x) = \varphi(x) \quad (\forall x \in G) \\
        h(a) = a \quad (\forall a \in R)
    \end{alignat}
    \qed
\end{corollary}




% ------------------------------------------------------------
%
% ------------------------------------------------------------
\section{多項式環}

多項式環を定義する。
多項式環は可換環の重要な例のひとつである。

\begin{definition}[多項式環]
    \idxsym{Polynomial ring}{$R[X_1, \dots, X_n]$}{$R$係数多項式環}
    $R$を可換環、
    $X_1, \dots, X_n$を形式的記号とする。
    形式的に
    \begin{equation}
        X_1^{k_1} \dots X_n^{k_n}
        \quad
        ((k_1, \dots, k_n) \in \Z_{\ge 0}^n)
    \end{equation}
    というものを考え、
    これを\term{単項式}[monomial]{単項式}[たんこうしき]と呼ぶ。
    ここで集合
    \begin{equation}
        M_n \coloneqq \{
            X_1^{k_1} \dots X_n^{k_n}
            \colon
            (k_1, \dots, k_n) \in \Z_{\ge 0}^n
        \}
    \end{equation}
    を定め、普通の方法で積を入れて可換モノイドにする。
    モノイド代数$R[M_n]$を
    $R[X_1, \dots, X_n]$と書き、
    \term{$R$-係数$n$変数多項式環}[polynomial ring]{多項式環}[たこうしきかん]
    と呼ぶ。
\end{definition}

\begin{definition}[多項式関数]
    \TODO{}
\end{definition}

\begin{proposition}[多項式の表示の一意性]
    $R$を可換環とする。
    $R[X_1, \dots, X_n]$の元の
    \begin{equation}
        \fsum_{k_1, \dots, k_n \ge 0} a_{k_1, \dots, k_n} X_1^{k_1} \dots X_n^{k_n}
    \end{equation}
    の形での表示は一意である。
\end{proposition}

\begin{proof}
    \TODO{形式的実質的有限和を写像$M \to R$とみれば一意性は明らか?}
\end{proof}

\begin{theorem}[除法定理]
    \label[theorem]{thm:division-theorem-of-polynomial-ring}
    $R$を可換環とする。
    このとき、
    任意の$f \in R[X]$および
    最高次係数が単元であるような
    任意の$g \in R[X]$に対し
    \begin{equation}
        \exists!\; q, r \in R[X]
        \quad \text{s.t.} \quad
        f = g q + r,
        \; \deg r < \deg g
    \end{equation}
    が成り立つ。
\end{theorem}

\begin{proof}
    $m \coloneqq \deg f,\; n \coloneqq \deg g \ge 0$とおく。
    題意の$q, r \in R[X]$の存在を$m$に関する帰納法で示す。
    $m < n$ならば$q(X) = 0, r(X) = f(X)$とおけばよい。
    $m \ge n$とし、$f, g$の最高次係数をそれぞれ$a_m, b_n$とおく。
    ここで
    \begin{equation}
        h(X) \coloneqq f(X) - g(X) a_m b_n^{-1} X^{m - n}
    \end{equation}
    とおくと$\deg h < \deg f$であるから、
    帰納法の仮定より
    \begin{equation}
        \exists q_1, r_1 \in R[X]
        \quad \text{s.t.} \quad
        h = gq_1 + r_1,
        \; \deg r_1 < \deg g
    \end{equation}
    が成り立つ。
    そこで
    \begin{equation}
        q(X) \coloneqq q_1(X) + a_m b_n^{-1} X^{m - n},
        \quad
        r(X) \coloneqq r_1(X)
    \end{equation}
    とおけばよい。
    つぎに一意性を示す。$q^*, r^* \in R[X]$が
    \begin{equation}
        f = g q^* + r^*,
        \; \deg r^* < \deg g
    \end{equation}
    をみたすとする。$f = gq + r$と差をとって
    \begin{equation}
        (q^* - q) g = r^* - r
    \end{equation}
    が成り立つ。よって、もし$q^* - q \neq 0$ならば
    \begin{alignat}{1}
        \deg ((q^* - q) g)
            &= \deg (q^* - q) + \deg g \quad (\because \text{ $g$の最高次係数は可逆元}) \\
            &\ge \deg g
    \end{alignat}
    が成り立つが、これは
    \begin{alignat}{1}
        \deg (r^* - r)
            &\le \max \{ \deg r^*, \deg r \} \\
            &< \deg g
    \end{alignat}
    に矛盾する。
    よって$q^* = q$、したがって$r^* = r$である。
    これで一意性がいえた。
\end{proof}

\begin{remark}[除法定理が成り立たない例]
    最高次係数が可逆元でない例を考える。
    $R = \Z$のとき、$X, 2X \in \Z[X]$に対し
    \begin{equation}
        X = 2X \cdot q(X) + r(X),\;
        \deg r < \deg (2X) = 1
    \end{equation}
    なる$q, r \in \Z[X]$は存在しない。
\end{remark}

\TODO{「割り切る」の概念が未定義}

\begin{corollary}[剰余定理]
    $R$を可換環とし、$\alpha \in R$とする。
    このとき
    \begin{equation}
        \exists! \; q \in R[X]
        \quad \text{s.t.} \quad
        f(X) = (X - \alpha) q(X) + f(\alpha)
    \end{equation}
    が成り立つ。とくに
    \begin{equation}
        \text{$X - \alpha$が$f$を割り切る}
        \quad \iff \quad
        f(\alpha) = 0
    \end{equation}
    が成り立つ。
\end{corollary}

\begin{proof}
    省略
\end{proof}

1変数多項式環においては、次の意味で
代入が定義できる。

\begin{theorem}[1変数多項式環の普遍性]
    $R$を可換環、
    $A$を$R$-代数とする。
    このとき、任意の$a \in A$に対し
    $R$-代数準同型$\ev_a \colon R[X] \to A$であって
    \begin{equation}
        \ev_a(X) = a
    \end{equation}
    をみたすものがただひとつ存在する。
\end{theorem}

\begin{proof}
    \TODO{}
\end{proof}

多変数多項式環でも代入を定義できるが、
多変数の場合は可換性が必要である。

\begin{theorem}[多変数多項式環の普遍性]
    \label[theorem]{thm:multivariate-polynomial-ring-universality}
    $R$を可換環、
    $A$を\highlight{可換}$R$-代数とする。
    このとき、任意の$a_1, \dots, a_n \in A$に対し
    $R$-代数準同型
    $\ev_{(a_1, \dots, a_n)} \colon R[X_1, \dots, X_n] \to A$
    であって
    \begin{equation}
        \ev_{(a_1, \dots, a_n)}(X_i) = a_i \quad (i = 1, \dots, n)
    \end{equation}
    をみたすものがただひとつ存在する。
\end{theorem}

\begin{proof}
    \TODO{cf. [雪江 p.16]}
\end{proof}

\begin{theorem}[多項式環の特徴付け]
    \TODO{}
\end{theorem}

\begin{proof}
    \TODO{}
\end{proof}

\begin{corollary}[多変数多項式環の自然な同型]
    \label[corollary]{corollary:polynomial-ring-isomorphism}
    \begin{equation}
        \ev_{(X_1, \dots, X_n)} \colon
            R[X_1, \dots, X_n] \to (R[X_1, \dots, X_{n - 1}])[X_n]
    \end{equation}
    は$R$-代数の同型である。
    \TODO{}
\end{corollary}

\begin{definition}[次数]
    \TODO{}
\end{definition}

多項式環からその係数環への
評価準同型は簡単な形の$\Ker$を持っている。

\begin{proposition}[多項式環の評価準同型の核]
    \label[proposition]{prop:ev-kernel}
    $R$を可換環、
    $\alpha_1, \dots, \alpha_n \in R$とする。
    このとき、評価準同型
    $\ev_{(\alpha_1, \dots, \alpha_n)} \colon R[X_1, \dots, X_n] \to R$
    の核は
    \begin{equation}
        \Ker(\ev_{(\alpha_1, \dots, \alpha_n)})
            = (X_1 - \alpha_1, \dots, X_n - \alpha_n)
    \end{equation}
    の形である。
\end{proposition}

\begin{proof}
    "$\supset$"は明らかに成り立つ。
    "$\subset$"を$n$についての帰納法で示す。
    $n = 1$のときは剰余定理
    (\cref{thm:division-theorem-of-polynomial-ring})
    からただちに従う。
    $n \in \Z_{\ge 2}$とし、
    $n - 1$で成立を仮定して$n$での成立を示す。
    そこで
    $f(X_1, \dots, X_n) \in \Ker(\ev_{(\alpha_1, \dots, \alpha_n)})$
    とする。
    \cref{corollary:polynomial-ring-isomorphism}より
    $R[X_1, \dots, X_n] \cong (R[X_1, \dots, X_{n - 1}])[X_n]$だから
    \begin{align}
        f(X_1, \dots, X_n)
            = \sum_{i = 0}^k f_i(X_1, \dots, X_{n - 1}) X_n^i \\
        \qquad (
            f_i \in R[X_1, \dots, X_{n - 1}], \;
            f_k \neq 0_{R[X_1, \dots, X_{n - 1}]}
        )
    \end{align}
    の形に一意に表せる。
    ここで各$0 \le i \le k$に対し
    \begin{equation}
        h_i(X_1, \dots, X_{n - 1})
            \coloneqq f_i(X_1, \dots, X_{n - 1})
            - f_i(\alpha_1, \dots, \alpha_{n - 1})
    \end{equation}
    とおくと、
    定め方から$h_i(\alpha_1, \dots, \alpha_{n - 1}) = 0$だから、
    帰納法の仮定より
    \begin{alignat}{1}
        h_i(X_1, \dots, X_{n - 1})
            &\in \Ker(\ev_{(\alpha_1, \dots, \alpha_{n - 1})}) \\
            &= (X_1 - \alpha_1, \dots, X_{n - 1} - \alpha_{n - 1}) \\
            &\subset
                (X_1 - \alpha_1, \dots, X_{n - 1} - \alpha_{n - 1}, X_n - \alpha_n)
    \end{alignat}
    となる。
    また、
    $\sum_{i = 0}^k
        f_i(\alpha_1, \dots, \alpha_{n - 1})
        \alpha_n^i
        = f(\alpha_1, \dots, \alpha_n)
        = 0$
    と剰余定理
    (\cref{thm:division-theorem-of-polynomial-ring})
    から
    $\sum_{i = 0}^k
        f_i(\alpha_1, \dots, \alpha_{n - 1})
        X_n^i
        \in (X_n - \alpha_n)$
    が成り立つ。
    よって
    \begin{alignat}{1}
        f(X_1, \dots, X_n)
            &= \sum_{i = 0}^k
                \bigl(
                    h_i(X_1, \dots, X_{n - 1})
                    + f_i(\alpha_1, \dots, \alpha_{n - 1})
                \bigr)
                X_n^i \\
            &= \underbrace{
                    \sum_{i = 0}^k
                    h_i(X_1, \dots, X_{n - 1})
                    X_n^i
                }_{\in (X_1 - \alpha_1, \dots, X_n - \alpha_n)}
                + \underbrace{
                    \sum_{i = 0}^k
                    f_i(\alpha_1, \dots, \alpha_{n - 1})
                    X_n^i
                }_{\in (X_n - \alpha_n)} \\
            &\in (X_1 - \alpha_1, \dots, X_n - \alpha_n)
    \end{alignat}
    となり、$n$での成立がいえた。
    帰納法より命題の主張が示せた。
\end{proof}



% ------------------------------------------------------------
%
% ------------------------------------------------------------
\section{自由代数}

自由代数を定義する。

\begin{definition}[自由$R$-代数]
    \idxsym{W(S)}{$W(S)$}{word 全体の集合}
    \idxsym{R[W(S)]}{$R[W(S)]$}{自由$R$代数}

    $S$を集合、
    $R \neq 0$を可換環とする。
    $S$の元$s_1, \dots, s_n$を形式的に$s_1 \dots s_n$と並べた
    \term{語}[word]{語}[ご] の全体を
    \begin{equation}
        W(S) \coloneqq \{ \text{$S$の元からなる語} \}
            \cup \{ \emptyset \}
    \end{equation}
    と定める。$\emptyset$を$1$と書き、乗法を
    \begin{alignat}{1}
        (s_1 \dots s_n) (s_1' \dots s_m') &= s_1 \dots s_n s_1' \dots s_m' \\
        1 (s_1 \dots s_n) &= (s_1 \dots s_n) 1 = s_1 \dots s_n
    \end{alignat}
    で定めてモノイド構造を入れる。
    モノイド代数$R[W(S)]$を
    $S$により生成される
    \term{自由$R$-代数}[free $R$-algebra]{自由代数}[じゆうだいすう]
    という。
\end{definition}

\begin{proposition}[自由代数の普遍性]
    \TODO{}
\end{proposition}

\begin{proof}
    \TODO{}
\end{proof}



% ------------------------------------------------------------
%
% ------------------------------------------------------------
\section{生成された部分代数}

部分集合によって部分代数を生成することができる。

\begin{definition}[部分集合により生成された部分代数]
    \idxsym{RS}{$R \langle S \rangle$}{集合$S$により生成された$R$-部分代数}
    $R \neq 0$を可換環、
    $A$を$R$-代数、
    $S \subset A$とする。
    このとき、標準包含$\iota \colon S \hookrightarrow A$
    により誘導される$R$-代数準同型$\wb{\iota} \colon R[W(S)] \to A$
    の像$\Im \wb{\iota}$を$R\langle S \rangle$と書き、
    \term{$S$で生成された$A$の$R$-部分代数}[$R$-subalgebra generated by $S$]
        {生成された部分代数}[せいせいされたぶぶんだいすう]
    という。
    とくに$S$が有限集合ならば、$A$は$R$-代数として
    \term{有限生成}[finitely generated]{有限生成!代数として---}[ゆうげんせいせい]
    であるという。
\end{definition}

%生成された代数は次の意味で普遍性を持つ。
%標語的に言えば「準同型が生成元で決まる」ということである。
%
%\begin{theorem}[生成された代数の普遍性]
%    \TODO{これは誤り?}
%    $R \neq 0$を可換環、
%    $A$を$R$-代数、
%    $S \subset A$とする。
%    このとき、
%    $S$により生成される$A$の$R$-部分代数$R\langle S \rangle$に対し
%    次が成り立つ:
%    \begin{alignat}{1}
%        &\forall \; B
%            \colon \text{ $R$-代数} \\
%        &\forall \; \varphi \colon S \to B
%            \colon \text{ 写像} \\
%        &\exists! \; \wb{\varphi} \colon R\langle S \rangle \to B
%            \colon \text{ $R$-代数準同型}
%            \quad \text{s.t.} \quad \\
%        &\quad
%            \begin{tikzcd}[ampersand replacement=\&, row sep=large]
%                R\langle S \rangle
%                    \ar[dashed]{rr}{\wb{\varphi}}
%                    \& \& B \\
%                \& S
%                    \ar{ul}
%                    \ar{ur}[swap]{\varphi}
%            \end{tikzcd}
%    \end{alignat}
%\end{theorem}
%
%\begin{proof}
%    \TODO{}
%\end{proof}

生成された代数は自由代数と同様の普遍性を持つわけではないことに注意すべきである。

\begin{remark}
    $\Z$上$S \coloneqq \{ 1/2 \}$により生成された
    $\Q$の$\Z$-部分代数$\Z\langle S \rangle$
    を考える。
    $\Z$上$S$により生成された$\Z$-部分代数が
    自由代数の場合と同様の "普遍性" を持ったとすると、
    写像$f \colon S \to \Z[X], \; f(1/2) \coloneqq X$に対し
    \begin{equation}
        \begin{tikzcd}
            \Z\langle S \rangle
                \ar[dashed]{r}{g}
                & \Z[X] \\
            S
                \ar[hook]{u}
                \ar{ru}[swap]{f}
        \end{tikzcd}
    \end{equation}
    を可換にする$\Z$-代数準同型$g$が一意に存在する。
    図式の可換性より
    $g(1/2) = X$だから
    $g(1) = g(2 \cdot 1/2) = 2g(1/2) = 2X$であるが、
    一方$g$は環準同型だから$g(1) = 1 \neq 2X$であり矛盾を得る。
\end{remark}

\begin{example}[多項式環は有限生成代数]
    $R \neq 0$を可換環とする。
    $R$-係数多項式環$R[X_1, \dots, X_n]$は
    有限集合$\{ X_1, \dots, X_n \} \subset R[X_1, \dots, X_n]$により
    生成される$R$-代数だから、$R$-代数として有限生成である。
\end{example}

生成された部分代数は次のように特徴付けられる。
これは生成されたイデアルの特徴付け
(\cref{prop:generated-ideal-characterization})
の類似である。

\begin{proposition}[生成された部分代数の特徴付け]
    $R \neq 0$を可換環、
    $A$を$R$-代数、
    $S \subset A$とする。
    このとき
    \begin{equation}
        R\langle S \rangle
            = \bigcap_{\substack{
                B \subset A \text{ : $R$-部分代数} \\
                B \supset S
            }} B
    \end{equation}
    が成り立つ。
\end{proposition}

\begin{proof}
    $R\langle S \rangle$は$S$を含む$A$の$R$-部分代数ゆえに
    右辺の項として現れるから "$\supset$" が成り立つ。

    "$\subset$" を示す。
    そこで$B \subset A$を$S$を含む$A$の$R$-部分代数とする。
    また
    $\Phi \colon S \to R[W(S)]$を標準射、
    \begin{equation}
        \iota_S^A \colon S \to A, \quad
        \iota_S^B \colon S \to B, \quad
        \iota_B^A \colon B \to A
    \end{equation}
    をそれぞれ標準包含とする。
    すると$R$上$S$により生成された自由代数の普遍性より
    図式
    \begin{equation}
        \begin{tikzcd}
            S
                \ar{d}[swap]{\Phi}
                \ar[hook]{r}{\iota_S^A}
                & A \\
            R[W(S)]
                \ar[dashed]{ru}[swap]{\wb{\iota}_S^A}
        \end{tikzcd}
        \qquad
        \begin{tikzcd}
            S
                \ar{d}[swap]{\Phi}
                \ar[hook]{rd}{\iota_S^B} \\
            R[W(S)]
                \ar[dashed]{r}[swap]{\wb{\iota}_S^B}
                & B
        \end{tikzcd}
    \end{equation}
    を可換にする$R$-代数準同型
    $\wb{\iota}_S^A, \; \wb{\iota}_S^B$が
    一意に存在する。
    ここで各$s \in S$に対し
    $\iota_B^A \circ \wb{\iota}_S^B \circ \Phi(s)
        = \iota_B^A \circ \iota_S^B(s)
        = \iota_S^A(s)
        = \wb{\iota}_S^A \circ \Phi(s)$
    が成り立つから、一意性より
    $\iota_B^A \circ \wb{\iota}_S^B = \wb{\iota}_S^A$である。
    したがって
    $R\langle S \rangle
        = \Im \wb{\iota}_S^A
        = \Im \iota_B^A \circ \wb{\iota}_S^B
        = \iota_B^A \circ \wb{\iota}_S^B (R[W(S)])
        \subset \iota_B^A(B)
        = B$
    である。
    よって "$\subset$" が示せた。
\end{proof}

有限生成可換$R$-代数は
次のように特徴付けることができる。

\begin{proposition}[有限生成可換$R$-代数の特徴付け]
    $R \neq 0$を可換環、
    $A$を可換$R$-代数、
    $S \subset A$とする。
    このとき、次は同値である:
    \begin{enumerate}
        \item $A$は有限生成$R$-代数である。
        \item ある$n \in \Z_{\ge 1}$と
            全射$R$-代数準同型$f \colon R[X_1, \dots, X_n] \to A$
            が存在する。
    \end{enumerate}
\end{proposition}

\begin{proof}
    \TODO{}
\end{proof}




% ============================================================
%
% ============================================================
\chapter{可換環}

可換環についてより詳しく調べる。

% ------------------------------------------------------------
%
% ------------------------------------------------------------
\section{素イデアル}

素イデアルを定義する。
この章では非可換環の素イデアルを扱うことはないが、
議論のまとまりのために素イデアルの定義は非可換環の場合も含めて与えておく。

\begin{definition}[素イデアル]
    $A$を環、
    $P$を$A$の固有両側イデアルとする。
    \begin{enumerate}
        \item $P$が$A$の
            \term{素イデアル}[prime ideal]{素イデアル}[そいである]
            であるとは、
            $A$の任意の固有両側イデアル$I, J$であって
            $IJ \subset P$をみたすものに対して
            $I \subset P$または$J \subset P$が成り立つことをいう。
        \item $P$が$A$の
            \term{完全素イデアル}[completely prime ideal]{完全素イデアル}[かんぜんそいである]
            であるとは、
            任意の$x, y \in A$であって$xy \in P$をみたすものに対して
            $x \in P$または$y \in P$が成り立つことをいう。
    \end{enumerate}
\end{definition}

\begin{definition}[素イデアル全体の集合]
    \idxsym{Spec}{$\Spec$}{素イデアル全体の集合}
    $R$を可換環とする。
    $R$の素イデアル全体の集合を
    \begin{equation}
        \Spec(R) \coloneqq \{
            \frakp \subset R \colon \text{$\frakp$は素イデアル}
        \}
    \end{equation}
    と書く。
\end{definition}

可換環における
素イデアルの特徴付けを与える。

\begin{proposition}[素イデアルの特徴付け]
    $R$を可換環とする。
    $R$の固有イデアル$\frakp \subset R$に関し
    次は同値である:
    \begin{enumerate}
        \item $\frakp$は$R$の素イデアルである。
        \item $\frakp$は$R$の完全素イデアルである。
        \item $R / \frakp$は整域である。
    \end{enumerate}
\end{proposition}

\begin{proof}
    \uline{(2) \Rightarrow (1)} \quad
    cf. \cref{problem:algebra2-2.36}

    \TODO{}
\end{proof}

\begin{example}[素イデアルの例]
    ~
    \begin{itemize}
        \item 素イデアルは極大イデアルとは限らない。
            実際、$\Z \cong \Z / (0)$は整域だが体でないので、
            $(0)$は$\Z$の素イデアルだが極大イデアルではない。
            しかし、可換アルティン環においては素イデアルは極大イデアルとなる
            (\cref{problem:algebra2-5.75})。
    \end{itemize}
\end{example}

根基と準素イデアルを定義する。
これらの概念は冪零元と深い関わりを持つ。

\begin{definition}[根基]
    \idxsym{radical}{$\sqrt{I}$}{$I$の根基}
    $R$を可換環、
    $I$を$R$の固有イデアルとする。
    このとき
    \begin{equation}
        \sqrt{I} \coloneqq \{
            r \in R
            \mid
            r^n \in I \; (\exists n \in \Z_{\ge 1})
        \}
    \end{equation}
    は$R$の固有イデアルとなり (このあと示す)、
    $\sqrt{I}$を$I$の
    \term{根基}[radical]{根基}[こんき]という。
    とくに$(0)$の根基$\sqrt{(0)}$を
    $R$の\term{冪零根基}[nilradical]{冪零根基}[べきれいこんき]という。
\end{definition}

\begin{proof}
    cf. \cref{problem:algebra2-2.29}
\end{proof}

\begin{proposition}[根基の特徴付け]
    $R$を可換環、$I$を$R$の固有イデアルとする。
    このとき、$I = \sqrt{I}$となることは
    $R / I$の$0$でない冪零元が存在しないための必要十分条件である。
\end{proposition}

\begin{proof}
    cf. \cref{problem:algebra2-5.65}
\end{proof}

\begin{definition}[準素イデアル]
    $R$を可換環とする。
    $R$の固有イデアル$I$が
    \term{準素イデアル}[primary ideal]{準素イデアル}[じゅんそいである]
    であるとは、
    $x, y \in R$に関し
    \begin{equation}
        (xy \in I \land x \not\in I)
        \quad \implies \quad
        y \in \sqrt{I}
    \end{equation}
    が成り立つことをいう。
\end{definition}

\begin{proposition}[準素イデアルの特徴付け]
    \label[proposition]{prop:primary-ideal-characterization}
    $R$を可換環、$I$を$R$の固有イデアルとする。
    このとき、$I$が準素イデアルであることは
    $R / I$の零因子がすべて冪零元になるための
    必要十分条件である。
\end{proposition}

\begin{proof}
    cf. \cref{problem:algebra2-5.66}
\end{proof}

\begin{proposition}[準素イデアルの根基]
    \label[proposition]{prop:primary-ideal-radical}
    $R$を可換環、$I$を$R$の準素イデアルとする。
    このとき$\sqrt{I}$は$I$を含む最小の素イデアルである。
\end{proposition}

\begin{proof}
    cf. \cref{problem:algebra2-6.78}
\end{proof}


% ------------------------------------------------------------
%
% ------------------------------------------------------------
\section{素元と既約元}

倍元と約元の概念を定義する。

\begin{definition}[倍元と約元]
    $R$を可換環、
    $a, b \in R$とする。
    $a$が$b$の\term{倍元}[multiple]{倍元}[ばいげん]、
    あるいは$b$が$a$の\term{約元}[divisor]{約元}[やくげん]であるとは、
    ある$r \in R$が存在して$a = rb$が成り立つことをいい、
    このことを$b \mid a$と書いて表す。
    $b \mid a$であるとき
    \term{$b$は$a$を割り切る}[$b$ divides $a$]{割り切る}[わりきる]、
    あるいは
    \term{$a$は$b$で割り切れる}[$a$ is divisible by $b$]{割り切れる}[わりきれる]
    という。
\end{definition}

\begin{definition}[同伴元]
    $R$を可換環、
    $a \in R$とする。
    $b \in R$が$a | b$かつ$b | a$をみたすとき、
    $a$は$b$の\term{同伴元}[associate]{同伴元}[どうはんげん]であるという。
    明らかにこのとき$b$は$a$の同伴元である。
\end{definition}

\begin{theorem}[整域における同伴元の特徴付け]
    $R$を整域とする。
    $a, b \in R$に関し、
    $a, b$が互いに同伴元であるための必要十分条件は、
    ある$u \in R^\times$が存在して
    $a = ub$が成り立つことである。
\end{theorem}

\begin{proof}
    十分性は明らか。

    \TODO{}
\end{proof}

最大公約元の概念を定義する。

\begin{definition}[最大公約元]
    $R$を可換環、
    $a_1, \dots, a_n \in R, \; g \in R$とする。
    $g$が$a_1, \dots, a_n$の
    \term{最大公約元}[greatest common divisor]{最大公約元}[さいだいこうやくげん]
    であるとは、
    $g$が次をみたすことをいう:
    \begin{enumerate}
        \item $g$は$a_1, \dots, a_n$を割り切る。
        \item $a_1, \dots, a_n$を割り切る任意の$g' \in R$に対し
            $g'$は$g$を割り切る。
    \end{enumerate}
\end{definition}

素元と既約元を定義する。
既約元は非自明な分解を持たない元のことである。

\begin{definition}[素元と既約元]
    $R$を可換環とする。
    \begin{itemize}
        \item $a \in R - \{0\}$が
            \term{素元}[prime element]{素元}[そげん]であるとは、
            $(a) \in \Spec(R)$であることをいう。
        \item $a \in R - \{0\}$が
            \term{既約元}[irreducible element]{既約元}[きやくげん]であるとは、
            \begin{enumerate}
                \item $a \not\in R^\times$
                \item $\forall a, b \in R$に対し、
                    \begin{equation}
                        a = bc
                        \quad \implies \quad
                        (b \in R^\times \lor c \in R^\times)
                    \end{equation}
            \end{enumerate}
            をみたすことをいう。
    \end{itemize}
\end{definition}

\begin{example}[既約元は素元とは限らない]
    cf. \cref{problem:algebra2-3.38}
    \TODO{}
    $\Z[\sqrt{-5}]$において$6 = 2 \cdot 3 = (q + \sqrt{-5})(1 - \sqrt{-5})$
    を考える。

    ただし、後でみるようにUFDでは既約元が素元となる。
\end{example}

\begin{example}[素元は既約元とは限らない]
    $K$を体とし、可換環$K \times K$を考える。
    $(K \times K) / ((1, 0)) \cong K$ゆえに
    $(1, 0)$は素元である。
    一方、$(1, 0) = (1, 0) (1, 0)$と可逆でない2元の積に書けることから
    $(1, 0)$は既約元ではない。
\end{example}

整域においては素元は既約元でもある。
既約元の判定はしばしば難しく、
素元の判定は比較的簡単なことがあるため、
この定理は既約元の判定の足がかりとなる。

\begin{theorem}[整域では素元は既約元]
    $R$を整域とする。
    $R$の素元は既約元である。
\end{theorem}

\begin{proof}
    $x \in R - \{ 0 \}$を素元とする。
    $x = ab, \; a, b \in R$とすると
    $x$が素元ゆえに$(x)$が素イデアルであることから
    $a \in (x)$または$b \in (x)$である。
    $a \in (x)$の場合を考える。
    $a = rx \; (r \in R)$と表せるから
    よって$x = rxb = rbx$である。
    いま$R$は整域だから$1 = rb$が成り立つ。
    したがって$b \in R^\times$である。
    同様に$b \in (x)$ならば$a \in R^\times$である。
    したがって$x$は$R$の既約元である。
\end{proof}



% ------------------------------------------------------------
%
% ------------------------------------------------------------
\section{UFD}

UFDを定義する。
UFDは既約元分解が次の意味で一意的に存在する整域である。

\begin{definition}[UFD]
    整域$R$が
    \term{一意分解整域}[unique factorization domain]{一意分解整域}[いちいぶんかいせいいき]、
    あるいは略して\term{UFD}{UFD}であるとは、
    $R$が次をみたすことをいう:
    \begin{enumerate}
        \item (既約元分解の存在)
            $0$でも単元でもない$r \in R$は
            既約元の積$r = p_1 \dots p_m$の形に表せる。
            各$p_i$を$r$の
            \term{素因子}[prime factor]{素因子}[そいんし]
            という。
        \item (既約元分解の一意性)
            既約元$p_1, \dots, p_m, q_1, \dots, q_n \in R$が
            $p_1 \dots p_m = q_1 \dots q_n$をみたすならば、
            $m = n$が成り立ち、かつ
            ある置換$\sigma \in S_n$が存在して
            $p_i$と$q_{\sigma(i)}$は互いに同伴元となる。
    \end{enumerate}
\end{definition}

異なる概念として定義された素元と既約元だが、
整域においては素元は既約元となるのであった。
さらにUFDではこの逆も成り立つ。
したがってUFDでは既約元分解は素元分解と考えても同じことである。

\begin{proposition}[UFDの既約元は素元]
    UFDの既約元は素元である。
    したがってUFDの零でない元が
    既約元であることと素元であることは同値である。
\end{proposition}

\begin{proof}[証明\footnote{
    PIDにおける別証明は\cref{problem:algebra2-3.39}を参照。
}.]
    $R$をUFD、$p \in R - \{ 0 \}$を既約元とする。
    $(p)$が素イデアルとなることを示せばよい。
    $ab \in (p)$とすると
    $ab = rp \; (r \in R)$と表せる。
    $p$は既約元だから、左辺の既約元分解には$p$の同伴元が含まれる。
    したがって既約元分解の一意性より、$a, b$の少なくとも一方の既約元分解に
    $p$の同伴元が含まれる。
    よって$a \in (p)$または$b \in (p)$が成り立つ。
    したがって$(p)$は素イデアルである。
    よって$p$は$R$の素元である。
\end{proof}

UFDは最大公約元を持つ。

\begin{proposition}[UFDは最大公約元を持つ]
    $R$をUFDとする。
    任意の$a_1, \dots, a_n \in R$に対し$a_1, \dots, a_n$の最大公約元が存在する。
\end{proposition}

\begin{proof}
    \TODO{cf. \cite[p.107]{Rot15}}
\end{proof}

上の命題により次の定義が可能となる。

\begin{definition}[互いに素]
    $R$をUFD、
    $a_1, \dots, a_n \in R$とする。
    $a_1, \dots, a_n$が
    \term{互いに素}[relatively prime]{互いに素}[たがいにそ]
    であるとは、
    $a_1, \dots, a_n$の最大公約元が単元のみであることをいう。
\end{definition}

\begin{proposition}[互いに素な元と互いに素なイデアル]
    \begin{equation}
        (a) + (b) = R
    \end{equation}
    \TODO{}
\end{proposition}

\begin{proof}
    \TODO{素イデアルが極大イデアルになることや
        同伴元と単項イデアルの関係を使って示すべき?}
\end{proof}

体上の多項式環はUFDである。

\begin{theorem}
    体上の多項式環はUFDである。
\end{theorem}

\begin{proof}
    \cite[p.111]{Rot15}
\end{proof}

\begin{definition}[重複度]
    $f \in K[X]$が
    $f(X) = (X - \alpha)^k g(X) \;
        (\text{$g$は$\alpha$を根に持たない})$
    の形に表せるとき、
    $k$を$\alpha$の
    \term{重複度}[multiplicity]{重複度}[ちょうふくど]
    という。
\end{definition}

% ------------------------------------------------------------
%
% ------------------------------------------------------------
\section{PID}

PIDについて述べる。
PIDの概念は\cref{chapter:linear-algebra}で述べる単因子論の基礎となる。

\begin{definition}[PID]
    任意のイデアルが単項イデアルとなる整域を
    \term{単項イデアル整域}[principal ideal domain]{単項イデアル整域}[たんこういであるせいいき]
    、あるいは略して\term{PID}{PID}という。
\end{definition}

単項イデアルの生成元は次の意味で一意である。

\begin{theorem}[単項イデアルの生成元の一意性]
    $R$を整域とする。
    $a, b \in R$に対し次は同値である:
    \begin{enumerate}
        \item $(a) = (b)$
        \item $\exists r \in A^\times \quad \text{s.t.} \quad ra = b$
    \end{enumerate}
\end{theorem}

\begin{proof}
    \TODO{}
\end{proof}

\begin{theorem}[PIDの$0$でない素イデアルは極大イデアル]
    PIDの$0$でない素イデアルは極大イデアルである。
\end{theorem}

\begin{proof}
    $(x) \neq 0$を素イデアルとし、
    $(y) \supsetneq (x)$をイデアルとする。
    $x \in (y)$だから$x = yz$と書ける。
    よって$yz \in (x)$である。
    したがって$y \in (x) \lor z \in (x)$だが、
    いま$(y) \supsetneq (x)$だから$y \notin (x)$、
    したがって$z \in (x)$である。
    よって$z = wx$と書ける。
    したがって$x = yz = ywx$である。
    よって$1 = yw$ゆえに$y$は単元だから$(y) = (1)$である。
\end{proof}

\begin{proposition}
    PIDにおいて、既約元の生成する単項イデアルは極大イデアルである。
\end{proposition}

\begin{proof}
    \TODO{}
\end{proof}

\begin{theorem}
    \label[theorem]{thm:PID-is-UFD}
    $R$をPIDとする。
    \begin{enumerate}
        \item $R$はUFDである。
        \item $a \in R - R^\times$に対し
            $\bigcap_{n \ge 1} (a^n) = 0$が成り立つ。
    \end{enumerate}
\end{theorem}

\begin{proof}
    \uline{(1)} \quad
    \TODO{}

    \uline{(2)} \quad
    $a = 0$の場合は明らかに成り立つ。
    $a \neq 0$とすると、$R$がUFDであることより
    $a$は単元と$k_a \in \Z_{\ge 1}$個の既約元の積に分解できる。
    $x \in \bigcap_{n \ge 1} (a^n)$とする。
    $x \in R^\times$であったとすると
    $x \in (a)$より$a$も単元となり矛盾。
    したがって$x \notin R^\times$である。
    $x \neq 0$と仮定し矛盾を導く。
    $x \notin R^\times, \; x \neq 0$より
    $x$は単元と$k_x \in \Z_{\ge 1}$個の既約元の積に分解できる。
    $x \in \bigcap_{n \ge 1} (a^n)$より
    すべての$n \in \Z_{\ge 1}$に対し
    $x \in (a^n)$だから
    $x = r_n a^n \; (r_n \in R)$と表せるが、
    両辺の既約元分解に現れる既約元の個数は
    左辺にちょうど$k_x$個、右辺に$nk_a$個以上だから、
    十分大きな$n$に対しては等しくなりえず、矛盾が従う。
    よって$x = 0$であり、(2)の主張が示された。
\end{proof}

% ------------------------------------------------------------
%
% ------------------------------------------------------------
\section{Euclid 整域}

Euclid 整域について述べる。

\begin{definition}[Euclid 整域]
    $R$を整域とする。
    $R$が\term{Euclid 整域}[Euclidean domain]{Euclid 整域}[Euclid せいいき]であるとは、
    次をみたす写像$\delta \colon R \to \Z_{\ge 0} \cup \{-\infty\}$が
    存在することをいう:
    \begin{itemize}
        \item $\delta(R - \{0\}) \subset \Z_{\ge 0}$
        \item $\delta(0) = -\infty$
        \item (割り算原理)
            $\forall a \in R$と$\forall h \in R - \{0\}$に対し
            \begin{equation}
                \exists q, r \in R \quad \text{s.t.} \quad
                a = hq + r \quad \text{and} \quad
                \delta(r) < \delta(h)
            \end{equation}
    \end{itemize}
\end{definition}

Euclid 整域は PID である。

\begin{theorem}
    Euclid 整域は PID である。
\end{theorem}

\begin{proof}
    \TODO{}
\end{proof}

体上の1変数多項式環は Euclid 整域となる。

\begin{theorem}
    体上の1変数多項式環は Euclid 整域である。
\end{theorem}

\begin{proof}
    cf. \cref{problem:algebra2-2.20}
\end{proof}


% ------------------------------------------------------------
%
% ------------------------------------------------------------
\section{局所環}

極大イデアルによって定義される可換環のクラスのうち
最も重要なもののひとつが局所環である。

\begin{definition}[局所環]
    極大イデアルをただひとつ持つ可換環を
    \term{局所環}[local ring]{局所環}[きょくしょかん]という。
\end{definition}

\begin{example}[単純環と局所環の例]
    ~
    \begin{itemize}
        \item 可換な単純環は$(0)$を唯一の極大イデアルとする局所環である。
        \item 体は単純環かつ局所環である。
    \end{itemize}
\end{example}

局所環$R$の極大イデアル$\frakm$は具体的に表せる。
すなわち$\frakm$は$R$の非単元全体の集合である。

\begin{theorem}[局所環の乗法群による特徴付け]
    $R$を可換環とする。
    次は同値である:
    \begin{enumerate}
        \item $R$は局所環である。
        \item $R - R^\times$は$R$の極大イデアルである。
    \end{enumerate}
\end{theorem}

\begin{proof}
    (2) \Rightarrow (1) は明らかだから
    (1) \Rightarrow (2) を示す。
    $\frakm$を$R$の唯一の極大イデアルとする。
    $R^\times \cap \frakm = \emptyset$であることは
    $\frakm$が固有イデアルであることから明らか。
    $x \in R - R^\times$とすると
    $(x)$は固有イデアルだから
    \cref{thm:krull} より$(x)$を含む極大イデアルが存在するが、
    いま$R$は局所環だからそれは$\frakm$である。
    したがって$x \in \frakm$が成り立つ。
\end{proof}

\begin{corollary}
    可換な単純環は体である。
    \qed
\end{corollary}



% ------------------------------------------------------------
%
% ------------------------------------------------------------
\section{局所化と商体}

局所化について述べる。

\begin{definition}[局所化]
    \idxsym{localization}{$S^{-1}R$}{$R$の$S$による局所化}
    $R$を可換環とする。
    \begin{itemize}
        \item $R$の乗法に関する部分モノイドを
            $R$の\term{積閉集合}[multiplicative set]{積閉集合}[せきへいしゅうごう]
            という。
        \item $S \subset R$を$R$の積閉集合とする。
            $R \times S$上の同値関係$\sim$を
            \begin{equation}
                (r_1, s_1) \sim (r_2, s_2)
                    \quad \iff \quad
                    \exists \; s \in S
                    \; \text{s.t.} \;
                    (r_1 s_2 - r_2 s_1) s = 0
            \end{equation}
            で定める (ことができる)。
            この同値関係による商集合を
            $S^{-1}R \coloneqq (R \times S) / \sim$とおき、
            $(r, s)$の属する類を$\frac{r}{s}$と書く。
            $S^{-1}R$には自然な加法と乗法が入り、
            $\frac{0}{1}$を零元、$\frac{1}{1}$を単位元として環となる。
            $S^{-1}R$を$R$の$S$による
            \term{局所化}[localization]{局所化}[きょくしょか]という。
    \end{itemize}
\end{definition}

\begin{remark}[局所化は局所環とは限らない]
    局所化は局所環とは限らない。
\end{remark}

局所化は次の普遍性を持つ。

\begin{theorem}[局所化の普遍性]
    $R$を可換環、
    $S \subset R$を$R$の積閉集合、
    標準射$R \to S^{-1}R$を$f$とおく。
    このとき、$S$の元を可換環$B$の単元に写すような
    任意の環準同型$g \colon R \to B$に対し、
    ある環準同型$h \colon S^{-1}R \to B$であって
    \begin{equation}
        \begin{tikzcd}
            S^{-1}R
                \ar[dashed]{rr}{h}
                && B \\
            & R
                \ar{lu}{f}
                \ar{ru}[swap]{g}
        \end{tikzcd}
    \end{equation}
    を可換にするものが一意に存在する。
\end{theorem}

\begin{proof}
    \TODO{}
\end{proof}

局所化は次の性質を持つ。
局所化によって$S$の元は分数の分母のところに置けて
単元になるというイメージである。

\begin{proposition}[局所化の性質]
    \begin{enumerate}
        \item $s \in S$に対し$f(s)$は$S^{-1}R$の単元である。
        \item $f(r) = 0$ならばある$s \in S$が存在して$rs = 0$である。
        \item $S^{-1}R$の任意の元は
            ある$r \in R$と$s \in S$により
            $f(r)f(s)^{-1}$の形に表せる。
    \end{enumerate}
    \TODO{}
\end{proposition}

\begin{proof}
    \TODO{}
\end{proof}

\begin{definition}[saturation]
    \TODO{}
\end{definition}

\begin{definition}[extension]
    \TODO{}
\end{definition}

\begin{proposition}[局所化のイデアル]
    \TODO{}
\end{proposition}

\begin{proof}
    \TODO{}
\end{proof}

\begin{theorem}[局所化の素イデアルの対応原理]
    \TODO{}
\end{theorem}

\begin{proof}
    \TODO{}
\end{proof}



% ============================================================
%
% ============================================================
\chapter{基本的な環}

% ------------------------------------------------------------
%
% ------------------------------------------------------------
\section{整数}

\TODO{初等整数論を展開する}

\begin{theorem}[Euclid の互除法]
    \TODO{}
\end{theorem}

\begin{proof}
    \TODO{}
\end{proof}



% ------------------------------------------------------------
%
% ------------------------------------------------------------
\section{有理数}

\begin{lemma}[有理数の表示]
    任意の$q \in \Q$は
    $q = k / l, \; k \in \Z, \; l \in \Z_{\ge 1}, \; \gcd(k, l) = 1$
    の形に一意的に表せる。
\end{lemma}

\begin{proof}
    \TODO{}
\end{proof}

% ------------------------------------------------------------
%
% ------------------------------------------------------------
\section{全行列環}

% ------------------------------------------------------------
%
% ------------------------------------------------------------
\section{多項式環}

% ------------------------------------------------------------
%
% ------------------------------------------------------------
\section{形式的冪級数環}

\begin{definition}
    \TODO{$R[[X]]$}
\end{definition}

\begin{proposition}
    $K$を体とする。
    $K[[X]]$のイデアルは
    \begin{equation}
        (0), (X^d)\; (d \in \Z_{\ge 0})
    \end{equation}
    で尽くされる。
    とくに$K[[X]]$は局所環かつPIDである。
\end{proposition}

\begin{proof}
    cf. \cref{problem:algebra2-2.30}
\end{proof}



% ------------------------------------------------------------
%
% ------------------------------------------------------------
\section{Weyl 代数}

\begin{definition}[Weyl 代数]
    \idxsym{Weyl algebra}{$\C[x; \del]$}{Weyl 代数}
    商$\C$-代数
    \begin{equation}
        \C[x; \del] \coloneqq \C[W(\{ \wt{x}, \wt{\del} \})]
            \; \big/ \; (\wt{\del} \wt{x} - \wt{x} \wt{\del} - 1)
    \end{equation}
    を\term{Weyl 代数}[Weyl algebra]{Weyl 代数}[Weylだいすう]という。
    $\wt{x}, \wt{\del}$の像をそれぞれ$x, \del$と書く。
    より一般に
    \begin{equation}
        \C[x_1, \dots, x_n; \; \del_1, \dots, \del_n]
            \coloneqq \C[
                W(\{ \wt{x}_1, \dots, \wt{x}_n, \wt{\del}_1, \dots, \wt{\del}_n \})
            ] \; \big/ \; I
    \end{equation}
    も Weyl 代数と呼ぶ。ただし、$I$は次の元たちから生成されるイデアルである:
    \begin{equation}
        \begin{cases}
            \wt{\del}_i \wt{x}_j - \wt{x}_j \wt{\del}_i - \delta_{ij} \\
            \wt{x}_i \wt{\del}_j - \wt{\del}_j \wt{x}_i \\
            \wt{\del}_i \wt{\del}_j - \wt{\del}_j \wt{\del}_i
        \end{cases}
    \end{equation}
\end{definition}

\begin{definition}[標準基底と標準形]
    cf. \cref{problem:algebra2-3.49}
    \TODO{}
\end{definition}

\begin{proposition}[次数]
    \TODO{}
\end{proposition}

\begin{proposition}
    Weyl 代数$\C[x : \del]$は単純環である。
\end{proposition}

\begin{proof}
    cf. \cref{problem:algebra2-3.50}
\end{proof}




% ------------------------------------------------------------
%
% ------------------------------------------------------------
\newpage
\section{演習問題}

\subsection{Problem set 1}

\begin{problem}[代数学II 1.1]
    $\End(\Z)$を求めよ。
\end{problem}

\begin{answer}
    \TODO{}
\end{answer}

\begin{problem}[代数学II 1.2]
    $\mu_2$を2次巡回群とする。
    群環$\C[\mu_2]$は$\C \times \C$と
    $\C$-algebra として同型であることを示せ。
\end{problem}

\begin{answer}
    まず、$\C[\mu_2]$および$\C \times \C$はそれぞれ環準同型
    \begin{alignat}{1}
        \varphi \colon \C \to \C[\mu_2], &\quad z \mapsto z \wb{0} \\
        \psi \colon \C \to \C \times \C, &\quad z \mapsto (z, z)
    \end{alignat}
    により$\C$-alg となっている。
    写像$f \colon \C \times \C \to \C[\mu_2]$を
    \begin{equation}
        (a, b) \mapsto \frac{a + b}{2} \wb{0} + \frac{a - b}{2} \wb{1}
    \end{equation}
    で定める。$f$が$\C$-alg 準同型であることを示す。
    加法について
    \begin{alignat}{1}
        f((a, b) + (a', b'))
            &= f(a + a', b + b') \\
            &= \frac{a + a' + b + b'}{2} \wb{0} + \frac{a + a' - b - b'}{2} \wb{1} \\
            &= \frac{a + b}{2} \wb{0} + \frac{a - b}{2} \wb{1}
                + \frac{a' + b'}{2} \wb{0} + \frac{a' - b'}{2} \wb{1} \\
            &= f(a, b) + f(a', b')
    \end{alignat}
    乗法について
    \begin{alignat}{1}
        f(a, b) \cdot f(a', b')
            &= \left(
                    \frac{a + b}{2} \wb{0} + \frac{a - b}{2} \wb{1}
                \right)
                \cdot
                \left(
                    \frac{a' + b'}{2} \wb{0} + \frac{a' - b'}{2} \wb{1}
                \right) \\
            &= \frac{1}{4}
                \left((a + b)(a' + b') + (a - b)(a' - b')\right) \wb{0} \\
            &\quad +
                \frac{1}{4}
                \left((a - b)(a' + b') + (a + b)(a' - b')\right) \wb{1} \\
            &= \frac{aa' + bb'}{2} \wb{0} + \frac{aa' - bb'}{2} \wb{1} \\
            &= f(aa', bb') \\
            &= f((a, b) \cdot (a', b'))
    \end{alignat}
    単位元について
    \begin{equation}
        f(1, 1)
            = \frac{1 + 1}{2} \wb{0} + \frac{1 - 1}{2} \wb{1}
            = 1 \wb{0}
    \end{equation}
    が成り立つから、$f$は環準同型である。
    また、図式
    \begin{equation}
        \begin{tikzcd}
            \C \times \C \ar{rr}{f} && \C[\mu_2] \\
            & \C \ar{ul}{\psi} \ar{ur}[swap]{\varphi}
        \end{tikzcd}
    \end{equation}
    が可換となることは
    \begin{equation}
        f \circ \psi(z)
            = f(z, z)
            = z \wb{0}
            = \varphi(z)
    \end{equation}
    よりわかる。
    したがって$f$は$\C$-alg 準同型である。
    $f$の定義より明らかに$\Ker f = 0$だから、$f$は単射である。
    また、再び$f$の定義から明らかに$f$は全射である。
    よって$f$は全単射、したがって$\C$-alg 同型である。
\end{answer}

\begin{problem}[代数学II 1.3]
    $A, B$を零環でない環、$f \colon A \to B$を単射とする。
    さらに任意の$x, y \in A$に対して
    \begin{equation}
        f(x + y) = f(x) + f(y), \quad f(xy) = f(x) f(y)
    \end{equation}
    が成り立つとする。
    このとき$f$は環準同型となるか?
\end{problem}

\begin{answer}
    反例を挙げる。
    $A \coloneqq \left\{
        \begin{bmatrix}
            a & 0 \\
            0 & a
        \end{bmatrix} \in M_2(\Z)
        \mid
        a \in \Z
    \right\}, \; B \coloneqq M_2(\Z)$とおき、
    写像$f \colon A \to B$を
    \begin{equation}
        f \colon \begin{bmatrix}
            a & 0 \\
            0 & a
        \end{bmatrix}
            \mapsto
            \begin{bmatrix}
                a & 0 \\
                0 & 0
            \end{bmatrix}
    \end{equation}
    で定める。
    $f$は明らかに単射で、また行列の演算の性質から
    $f(x + y) = f(x) + f(y), \; f(xy) = f(x) f(y)$も成り立つ。
    しかし
    \begin{equation}
        f \colon 1_A = \begin{bmatrix}
            1 & 0 \\
            0 & 1
        \end{bmatrix}
            \mapsto
            \begin{bmatrix}
                1 & 0 \\
                0 & 0
            \end{bmatrix}
            \neq 1_B
    \end{equation}
    だから$f$は環準同型ではない。
\end{answer}

\begin{problem}[代数学II 1.4]
    可換環の2つの冪零元の和は冪零元になることを示せ。
    また非可換環の場合は同じことが成り立つか?
\end{problem}

\begin{answer}
    $a, b$が冪零元であるとし、
    $a^m = 0, b^n = 0\; (m, n \in \Z_{>0})$とする。
    $l = m + n$とおくと、
    \begin{equation}
        (a + b)^l = \sum_{k = 0}^l \binom{l}{k} a^{l - k} b^k
            = 0
    \end{equation}
    が成り立つ。
    ただし、
    最初の等号では$a, b$が可換であることを用い、
    最後の等号では
    \begin{itemize}
        \item $k \ge m$のとき$a^k = 0$
        \item $k < m$のとき$l - k > l - m = n$より$b^{l - k} = 0$
    \end{itemize}
    であることを用いた。
    したがって$a + b$も冪零元である。

    非可換環の場合の反例として
    \begin{equation}
        A = \begin{bmatrix}
            0 & 1 \\
            0 & 0
        \end{bmatrix},\;
        B = \begin{bmatrix}
            0 & 0 \\
            1 & 0
        \end{bmatrix}
        \in M_2(\Z)
    \end{equation}
    を考える。
    \begin{equation}
        A^2 = 0, B^2 = 0
    \end{equation}
    だからこれらは冪零元である。一方、
    \begin{equation}
        A + B = \begin{bmatrix}
            0 & 1 \\
            1 & 0
        \end{bmatrix}
    \end{equation}
    ゆえに
    \begin{equation}
        (A + B)^2 = I_2
    \end{equation}
    だから、$(A + B)^n = 0$なる正整数$n$があったとすると
    \begin{equation}
        I_2
            = (A + B)^{2n}
            = 0
    \end{equation}
    となり矛盾。
\end{answer}


\begin{problem}[代数学II 1.5]
    $A$を環、$u \in A^\times$とし、$n \in A$を冪零元として
    $un = nu$であるとする。
    このとき$u + n \in A^\times$を示せ。
\end{problem}

\begin{answer}
    $n^{k - 1} \neq 0, n^k = 0, k \in \Z_{>0}$とする。
    $u + n$の逆元を発見する手立てとして、等比数列の公式を思い出して形式的に
    \begin{equation}
        (u + n)^{-1}
            \stackrel{?}{=} \frac{1}{u + n}
            \stackrel{?}{=} u^{-1} \frac{1}{1 + nu^{-1}}
            \stackrel{?}{=} u^{-1} \sum_{i = 0}^{\infty} (-nu^{-1})^i
    \end{equation}
    と書いてみると、
    \begin{equation}
        u^{-1} \sum_{i = 0}^{k - 1} (-1)^i u^{-i} n^{i}
    \end{equation}
    が$u + n$の逆元になりそうだと気づく。そして実際、
    \begin{alignat}{1}
        (u + n) u^{-1} \sum_{i = 0}^{k - 1} (-1)^i u^{-i} n^{i}
            &= (u + n) \sum_{i = 0}^{k - 1} (-1)^i u^{-i - 1} n^{i} \\
            &= u \sum_{i = 0}^{k - 1} (-1)^i u^{-i - 1} n^{i}
                + n \sum_{i = 0}^{k - 1} (-1)^i u^{-i - 1} n^{i} \\
            \intertext{$un = nu$より$u^{-1}n = nu^{-1}$であることに注意して}
            &= \sum_{i = 0}^{k - 1} (-1)^i u^{-i} n^{i}
                + \sum_{i = 0}^{k - 1} (-1)^i u^{-i - 1} n^{i + 1} \\
            &= \sum_{i = 0}^{k - 1} (-1)^i u^{-i} n^{i}
                + \sum_{i = 0}^{k - 2} (-1)^i u^{-i - 1} n^{i + 1} \\
            &= \sum_{i = 0}^{k - 1} (-1)^i u^{-i} n^{i}
                + \sum_{i = 1}^{k - 1} (-1)^{i - 1} u^{-i} n^{i} \\
            &= 1
    \end{alignat}
    となる。
    したがって$u + n \in A^\times$である。
\end{answer}


\begin{problem}[代数学II 1.6]
    $S$を index set とし、$\{A_i\}_{i \in S}$を環の族とする。
    $(B, \{ q_i \colon B \to A_i \}_{i \in S})$が
    $\{ A_i \}_{i \in S}$の圏論的直積であるとは、
    $\{ q_i \colon B \to A_i \}_{i \in S}$は環準同型の族であって
    任意の環準同型の族$\{ f_i \colon C \to A_i \}_{i \in S}$に対して
    図式
    \begin{equation}
        \begin{tikzcd}
            C \ar{rd}[swap]{f_i} \ar{rr}{F} && B \ar{ld}{q_i} \\
            & A_i
        \end{tikzcd}
    \end{equation}
    を可換にするような環準同型$F \colon C \to B$が一意に存在することをいう。
    \begin{enumerate}[label=(\alph*)]
        \item 直積環$(\prod_{i \in S} A_i,
            \{ p_i \colon \prod_{i \in S} A_i \to A_i \}_{i \in S})$は
            $\{ A_i \}_{i \in S}$の圏論的直積であることを示せ。
            ここで$p_i$は標準射影である。
        \item 任意の$\{ A_i \}_{i \in S}$の圏論的直積
            $(B, \{ q_i \colon B \to A_i \}_{i \in S})$
            に対して、
            \begin{equation}
                \begin{tikzcd}
                    \prod_{i \in S} A_i
                        \ar{rd}[swap]{p_i} \ar{rr}{\Psi}
                        && B \ar{ld}{q_i} \\
                    & A_i
                \end{tikzcd}
            \end{equation}
            を可換にするような環の同型写像
            $\Psi \colon \prod_{i \in S} A_i \to B$が一意に存在することを示せ。
    \end{enumerate}
\end{problem}

\begin{answer}
    \TODO{}
\end{answer}


\begin{problem}[代数学II 1.7]
    $R$を環とする。全行列環$M_n(R)$の中心$Z(M_n(R))$を求めよ。
\end{problem}

\begin{answer}
    $A = (a_{ij}) \in M_n(R)$とする。
    $(i, j)$成分が$1$の行列単位を$E_{ij}$と書くことにする。
    まず
    \begin{equation}
        \begin{bmatrix}
            0 & \dots & 0 \\
            \vdots & & \vdots \\
            a_{i1} & \dots & a_{in} \\
            \vdots & & \vdots \\
            0 & \dots & 0
        \end{bmatrix}
            = E_{ii} A
            = A E_{ii}
            = \begin{bmatrix}
                0 & \dots & a_{1i} & \dots & 0 \\
                \vdots & & \vdots & & \vdots \\
                0 & \dots & a_{ni} & \dots & 0
            \end{bmatrix}
    \end{equation}
    より$A$は対角行列である。
    つぎに
    \begin{equation}
        a_{jj} E_{ij} = E_{ij} A = A E_{ij} = a_{ii} E_{ij}
    \end{equation}
    より$a_{jj} = a_{ii}$であるから、$A$はスカラー行列である。
    そこで$A = a I_n, a \in R$とおく。
    任意の$b \in R$に対し
    \begin{equation}
        ab I_n = A (b I_n) = (b I_n) A = ba I_n
    \end{equation}
    したがって$ab = ba$が成り立つ。
    よって$a \in Z(R)$である。
    逆に対角成分が$Z(R)$の元であるようなスカラー行列は明らかに$Z(M_n(R))$に属する。
    したがって$Z(M_n(R))$は
    対角成分が$Z(R)$の元であるようなスカラー行列の全体である。
\end{answer}


\begin{problem}[代数学II 1.8]
    $(A, +, \cdot, 0)$が
    \term{nonunital ring}{nonunital ring} であるとは、
    \begin{itemize}
        \item $(A, +, 0)$がアーベル群かつ
        \item $(A, \cdot)$が半群で
        \item 分配法則をみたすもの
    \end{itemize}
    とする。$f \colon A \to B$が nonunital ring の準同型であるとは、
    \begin{itemize}
        \item 任意の$x, y \in A$に対して
            $f(x + y) = f(x) + f(y)$かつ
            $f(xy) = f(x) f(y)$が成り立つこと
    \end{itemize}
    とする。
    さて、任意の nonunital ring $A$に対して、
    環$A_1$と nonunital ring の準同型$\iota \colon A \to A_1$であって
    次の条件を満たすものが存在することを示せ:
    \begin{description}
        \item[(条件)]
            任意の環$B$と nonunital ring の準同型$f \colon A \to B$に対して
            図式
            \begin{equation}
                \begin{tikzcd}[row sep=large]
                    A \ar{d}[swap]{f} \ar{r}{\iota}
                        & A_1 \ar[dashed]{ld}{f_1} \\
                    B
                \end{tikzcd}
            \end{equation}
            を可換にするような環準同型$f_1 \colon A_1 \to B$が
            一意に存在する。
    \end{description}
\end{problem}

\begin{answer}
    \TODO{}
\end{answer}


\begin{problem}[代数学II 1.9]
    \label[problem]{problem:algebra2-1.9}
    $C(\R)$を$\R$上の$\C$値連続関数全体のなす$\C$-alg とする。
    $C(\R)$の零因子を求めよ。
\end{problem}

\begin{answer}
    $f \in C(\R)$が$C(\R)$の零因子であることが次と同値であることを示す:
    \begin{equation}
        f \neq 0
        \quad \text{and} \quad
        \exists U \opensubset \R
        \quad \text{s.t.} \quad 
        \forall x \in U
        \quad \text{に対し} \quad 
        f(x) = 0
    \end{equation}

    ($\Leftarrow$) \quad
    $U$は$\R$の開集合だから、
    或る開区間$(x_0 - \eps, x_0 + \eps),\; x_0 \in \R, \eps > 0$を含む。
    そこで
    \begin{equation}
        g(x) \coloneqq \begin{cases}
            x - (x_0 - \eps / 2) & x \in (x_0 - \eps / 2, x_0) \\
            - x + (x_0 + \eps / 2) & x \in (x_0, x_0 + \eps / 2) \\
            0 & \text{otherwise}
        \end{cases}
    \end{equation}
    と定めれば$g \in C(\R), g \neq 0$であり、
    $f|_U = 0$の仮定から$f \cdot g = 0$が従う。
    $f \neq 0, g \neq 0$だから$f$は$C(\R)$の零因子である。

    ($\Rightarrow$) \quad
    $f$は$C(\R)$の零因子であるとする。
    零因子の定義から$f \neq 0$である。
    背理法のため
    \begin{equation}
        \forall U \opensubset \R
        \quad \text{に対し} \quad
        \exists x \in U
        \quad \text{s.t.} \quad
        f(x) \neq 0
    \end{equation}
    を仮定する。
    いま$f$が$C(\R)$の零因子であることから
    \begin{equation}
        \exists g \in C(\R), g \neq 0
        \quad \text{s.t.} \quad
        f \cdot g = 0
    \end{equation}
    である。
    このとき、$g \neq 0$より
    \begin{equation}
        \exists x_0 \in \R
        \quad \text{s.t.} \quad
        g(x_0) \neq 0
    \end{equation}
    である。
    このことと$g$の点$x_0$における連続性より
    \begin{equation}
        \exists U \colon \text{$x_0$の開近傍}
        \quad \text{s.t.} \quad
        \forall x \in U
        \quad \text{に対し} \quad
        g(x) \neq 0
    \end{equation}
    が成り立つ。
    ここで、この$U$に対して背理法の仮定を用いると
    \begin{equation}
        \exists x_1 \in U
        \quad \text{s.t.} \quad
        f(x_1) \neq 0
    \end{equation}
    である。よって
    \begin{equation}
        (f \cdot g)(x_1) = f(x_1) g(x_1) \neq 0
    \end{equation}
    が成り立つ。
    これは$f \cdot g = 0$に矛盾。
\end{answer}


\begin{problem}[代数学II 1.10]
    $A$を$1 < \dim_\C A < \infty$なる$\C$-alg とする。
    このとき$A$は零因子を持つことを示せ。
\end{problem}

\begin{answer}
    $\dim_\C A > 1$だから$\C$-線型独立な$v_1, v_2 \in A$がとれる。
    $A$が零因子を持たないとして矛盾を導く。
    $a \in A \setminus \{0\}$とし、2つの写像
    \begin{equation}
        L_a \colon A \to A, \quad x \mapsto ax, \quad
        R_a \colon A \to A, \quad x \mapsto xa
    \end{equation}
    を考える。これらは明らかに$\C$-線型であり、
    $A$が零因子を持たないという仮定から$\Ker$は自明、したがって単射である。
    $\dim_\C < \infty$であることとあわせて、
    $L_a, R_a$の全射性が従う。
    よって
    \begin{equation}
        \begin{cases}
            \exists x \in A \quad \text{s.t.} \quad ax = 1 \\
            \exists y \in A \quad \text{s.t.} \quad ya = 1
        \end{cases}
    \end{equation}
    であり、逆元の一意性から$x = y$が従う。
    よって$a$は$A$の可逆元である。
    さて、$a \in A \setminus \{0\}$は任意であったから、
    とくに$v_1, v_2$も$A$の可逆元である。
    そこで$w \coloneqq v_2 v_1^{-1}\; (\neq 0)$とおき、
    $\C$-線型写像$L_w$の特性多項式\footnotemark を$\Phi(X)$とおく。
    Cayley-Hamilton の定理より
    $\Phi(L_w) = 0$が成り立つから、
    \begin{equation}
        0 = (\Phi(L_w))(v_1)
            = \Phi(w) v_1
    \end{equation}
    であり、$A$が零因子を持たないという仮定から
    $\Phi(w) = 0$が従う。
    ここで、$\C$は代数的閉体だから$\Phi(X)$は1次式の積に分解し
    \begin{equation}
        \Phi(X) = (X - \mu_1) (X - \mu_2) \cdots (X - \mu_n)
    \end{equation}
    の形に書ける。ただし$\mu_i$らは$L_w$の固有値である。
    したがって、$\Phi(w) = 0$であることと、
    $A$が零因子を持たないという仮定をあわせて
    \begin{equation}
        w - \mu_k \cdot 1_A = 0 \quad (\exists k = 1, \dots, n)
    \end{equation}
    が成り立ち、$w$の定義とあわせて
    \begin{equation}
        v_2 = w v_1 = \mu_k \cdot v_1
    \end{equation}
    が従う。これは$v_1, v_2$の$\C$-線型独立性に矛盾する。
    よって$A$が零因子を持たないとした仮定は偽で、題意の主張が示せた。
\end{answer}

\footnotetext{
    有限次元線型空間の自己準同型$f$に対し、
    適当な基底による行列表現$B$の特性多項式$\det(XI_n - B)$を
    $f$の\term{特性多項式}[characteristic polynomial]{特性多項式}[とくせいたこうしき]という。
}


\begin{problem}[代数学II 1.11]
    $A$を環とする。
    ある零環でない環$B, C$が存在して$A \cong B \times C$となるための必要十分条件は
    $0, 1 \neq e$なる冪等元$e \in Z(A)$が存在することであることを示せ。
\end{problem}

\begin{answer}
    ($\Rightarrow$) \quad
    $B \times C$において
    $(1_B, 0) \neq 1_{B \times C}, 0_{B \times C}$は
    \begin{equation}
        (1_B, 0)^2 = (1_B, 0)
    \end{equation}
    をみたすから冪等元であり、また明らかに$B \times C$の中心に属する。
    そこで、これを同型によって$A$に写したものが求める$e$となる。

    ($\Leftarrow$) \quad
    $e, 1 - e$は$A$の中心冪等元だから、
    \begin{equation}
        B \coloneqq A e, \quad
        C \coloneqq A (1 - e)
    \end{equation}
    はそれぞれ$e, 1 - e$を単位元として環をなす。
    そこで写像$A \to B \times C$を
    \begin{equation}
        x \mapsto (xe, x(1 - e))
    \end{equation}
    で定めれば、これが環同型$A \cong B \times C$を与える。
    定義より$B, C$は零環でないから、これらが求めるものである。
\end{answer}


\begin{problem}[代数学II 1.12 Hamilton's Quaternions]
    $M_2(\C)$を実ベクトル空間と考え、
    \begin{equation}
        \begin{bmatrix}
            1 & 0 \\
            0 & 1
        \end{bmatrix},
        \begin{bmatrix}
            i & 0 \\
            0 & -i
        \end{bmatrix},
        \begin{bmatrix}
            0 & 1 \\
            -1 & 0
        \end{bmatrix},
        \begin{bmatrix}
            0 & i \\
            i & 0
        \end{bmatrix}
    \end{equation}
    で生成される$4$次元実部分空間を$\H$とおく。
    $\H$は$M_2(\C)$の部分環となり、さらに division algebra となることを示せ。
\end{problem}

\begin{answer}
    所与の行列を左から順に$1, I, J, K$と書くことにする。
    乗積表は
    \begin{center}
        \begin{tabular}{R|RRRR}
              & 1 & I & J & K \\ \hline
            1 & 1 & I & J & K \\
            I & I & -1 & K & -J \\
            J & J & -K & -1 & I \\
            K & K & J & -I & -1
        \end{tabular}
    \end{center}
    となる。
    よって
    \begin{alignat*}{1}
        (a1 + bI + cJ + dK)(a'1 + b'I + c'J + d'K)
            &= aa'1 + bb'I + cc'J + dd'K \\
            &+ ba'I - bb'1 + bc'K - bd'J \\
            &+ ca'J - cb'K - cc'1 + cd'I \\
            &+ da'K + db'J - dc'I - dd'1
    \end{alignat*}
    である。そこで
    \begin{equation}
        a' = a,\quad
        b' = -b,\quad
        c' = -c,\quad
        d' = -d
    \end{equation}
    とおけば
    \begin{equation}
        (a1 + bI + cJ + dK)(a1 - bI - cJ - dK)
            = (a^2 + b^2 + c^2 + d^2) 1
    \end{equation}
    となる。
    したがって$a^2 + b^2 + c^2 + d^2 \neq 0$のとき、
    すなわち$a1 + bI + cJ + dK \in \H \setminus \{0\}$のとき
    $a1 + bI + cJ + dK$の乗法逆元が
    \begin{equation}
        \frac{1}{a^2 + b^2 + c^2 + d^2}
            (a1 - bI - cJ - dK)
    \end{equation}
    で与えられることがわかる。
    よって$\H \setminus \{0\} \subset \H^\times$である。
    逆向きの包含も明らかに成り立つ。
    よって$\H \setminus \{0\} = \H^\times$、
    したがって$\H$は division algebra である。
\end{answer}


\begin{problem}[代数学II 1.13]
    \label[problem]{problem:algebra-1.13}
    $\{x \in \H \mid x^2 = -1\}$は無限集合であることを示せ。
\end{problem}

\begin{answer}
    $x = a + bi + cj + dk \in \H$とおくと
    \begin{alignat}{1}
        x^2
            &= (a + bi + cj + dk)^2 \\
            &= aa + abi + acj + adk \\
            &+ bai -bb + bck - bdj \\
            &+ caj -cbk - cc + cdi \\
            &+ dak + dbj - dci - dd \\
            &= a^2 - b^2 - c^2 - d^2
                + 2abi + 2acj + 2adk
    \end{alignat}
    だから、これが$-1$に一致する条件は
    \begin{equation}
        \begin{cases}
            a^2 - b^2 - c^2 - d^2 = -1 \\
            ab = ac = ad = 0
        \end{cases}
    \end{equation}
    すなわち
    \begin{equation}
        a = 0 \wedge b^2 + c^2 + d^2 = 1
    \end{equation}
    である。
    よって
    \begin{alignat}{1}
        \{ x \in \H \mid x^2 = -1 \}
            &= \{ bi + cj + dk \mid b^2 + c^2 + d^2 = 1 \} \\
            &= \{
                (\sin\theta_1 \sin\theta_2) i
                + (\sin\theta_1 \cos\theta_2) j
                + (\cos\theta_1) k
                \mid 
                \theta_1, \theta_2 \in \R
            \}
    \end{alignat}
    が成り立つ。右辺は無限集合だから、左辺もそうであり、
    したがって題意の主張が示せた。
\end{answer}


\begin{problem}[代数学II 1.14]
    非可換$3$次元$\C$-alg は$0$でない冪零元をもつことを示せ。
\end{problem}

\begin{answer}
    \TODO{1.10を使えば零因子の存在はいえるが…?}
    cf. \url{http://doi.org/10.5169/seals-46956}
\end{answer}


\begin{problem}[代数学II 1.15]
    $M_n(\C)$の$\C$-alg としての自己同型写像をすべて求めよ。
\end{problem}

\begin{answer}
    \TODO{}
\end{answer}


\begin{problem}[代数学II 1.16]
    任意の巡回群を考えその演算を加法とみなす。
    すると環の構造を与える乗法が一意に定まることを示せ。
\end{problem}

\begin{answer}
    \TODO{環同型を除いて?}
    巡回群は$\Z$あるいは$\Z/n\Z$に群同型だから、
    $\Z, \Z/n\Z$について考えれば十分。
    そこで、まず$\Z$について考える。
    $\Z$の通常の加法、乗法をそれぞれ$+, \times$で表すことにし、
    さらに$\Z$に乗法$\odot$が与えられたとする。
    乗法$\odot$に関する単位元を$e$とおく。
    このとき$e = 0$なら
    \begin{alignat}{1}
        1
            &= e \odot 1 \\
            &= 0 \odot 1 \\
            &= 0
    \end{alignat}
    となり矛盾だから、$e \neq 0$である。
    $e > 0$の場合
    \begin{alignat}{1}
        1
            &= e \odot 1 \\
            &= (\underbrace{1 + \dots + 1}_{\text{$e$ times}}) \odot 1 \\
            &= \underbrace{1 \odot 1 + \dots + 1 \odot 1}_{\text{$e$ times}} \\
            &= e \times (1 \odot 1)
    \end{alignat}
    ゆえに$e$は乗法$\times$に関し可逆だから$e = \pm 1$である。
    いま$e > 0$であったから$e = 1$である。
    すると$1 \odot 1 = e \odot e = e = 1$だから、
    各$a, b \in \Z$に対し
    \begin{alignat}{1}
        a \odot b
            &= a \times b \times (1 \odot 1) \\
            &= a \times b \times 1 \\
            &= a \times b
    \end{alignat}
    が成り立つ。
\end{answer}


\begin{problem}[代数学II 1.17]
    $2$次元$\C$-alg を同型を除いて分類せよ。
\end{problem}

\begin{answer}
    \TODO{}
\end{answer}


\begin{problem}[代数学II 1.18]
    $\Z[\sqrt{2}] = \{ n + \sqrt{2}m \mid n, m \in \Z \}$は
    $\C$の部分環になることを示せ。
    また次を示し$\Z[\sqrt{2}]^\times$が無限群であることを示せ:
    \begin{equation}
        \Z[\sqrt{2}]^\times
            = \{ n + \sqrt{2}m \mid n, m \in \Z, n^2 - 2m^2 = \pm 1 \}
    \end{equation}
\end{problem}

\begin{answer}
    $\Z[\sqrt{2}]$が$\C$の加法部分群であることと$1$を含むことは明らか。
    乗法について閉じていることは
    \begin{alignat}{1}
        (n + \sqrt{2} m) (a + \sqrt{2} b)
            &= na + 2mb + \sqrt{2} (nb + ma)
    \end{alignat}
    よりわかる。
    したがって$\Z[\sqrt{2}]$は$\C$の部分環である。
    乗法群$\Z[\sqrt{2}]^\times$が題意のように表されることを示す。
    "$\subset$"はモノイド準同型
    \begin{equation}
        N \colon \Z[\sqrt{2}]^\times \to \Z,
        \quad n + \sqrt{2} m \mapsto n^2 - 2m^2
    \end{equation}
    を用いて示せる。
    \begin{innerproof}
        $(n + \sqrt{2} m) (a + \sqrt{2} b) = 1$であるとすれば、
        両辺を$N$で写して
        \begin{equation}
            (n^2 - 2m^2) (a^2 - 2b^2) = 1
        \end{equation}
        よって$n^2 - 2m^2 = \pm 1$を得る。
    \end{innerproof}
    "$\supset$"は、$n^2 - 2m^2 = \pm 1$のとき
    $n + \sqrt{2} m$の逆元が$\pm (n - \sqrt{2} m)$となることから明らか。
    最後に$\Z[\sqrt{2}]^\times$が無限群であることは、
    ひとつの解$n + \sqrt{2} m$から
    新たな解として$(n^2 + 2m^2) + 2 \sqrt{2} nm$を構成できることからわかる。
    \begin{innerproof}
        実際、
        \begin{alignat}{1}
            (n^2 + 2m^2)^2 - 2 (2nm)^2
                &= (n^2 - 2m^2)^2 \\
                &= 1
        \end{alignat}
        である。
        新たな解の実部$n^2 + 2m^2$はもとの解より大きいから、
        この構成で得られる無限個の解たちはすべて相異なる。
    \end{innerproof}
\end{answer}

\subsection{Problem set 2}

\begin{problem}[代数学II 2.19]
    次数が$k$である$n$変数単項式の個数を$d_n(k)$とおく。
    \begin{equation}
        \frac{1}{(1 - t)^n}
            = \sum_{k = 0}^\infty d_n(k) t^k
            \quad (|t| < 1)
    \end{equation}
    を示せ。また$d_n(k)$を求めよ。
\end{problem}

\begin{answer}
    \TODO{}
    まず$n \in \Z_{\ge 1}, k \in \Z_{\ge 0}$に対し
    \begin{equation}
        d_n(k) = \binom{k + n - 1}{n - 1}
    \end{equation}
    である (区別のある$n$個の箱に区別のない$k$個の玉を入れることを考える)。
    つぎに、$n \in \Z_{\ge 0}$に対し
    \begin{alignat}{1}
        \frac{d^n}{dt^n} \frac{t^n}{1 - t}
            &= \frac{d^n}{dt^n} \sum_{k = 0}^\infty t^{k + n} \\
            &= \sum_{k = 0}^\infty \frac{d^n}{dt^n} t^{k + n} \\
            &= \sum_{k = 0}^\infty (k + n) \dots (k + 1) t^k
    \end{alignat}
    が成り立つ。一方、
    \begin{alignat}{1}
        \frac{d^n}{dt^n} \frac{t^n}{1 - t}
            &= \frac{d^n}{dt^n} \left(
                - (t^{n-1} + \dots + t + 1) + \frac{1}{1 - t}
            \right) \\
            &= \frac{d^n}{dt^n} \frac{1}{1 - t} \\
            &= n! \frac{1}{(1 - t)^{n+1}}
    \end{alignat}
    が成り立つ。
    したがって
    \begin{alignat}{1}
        \frac{1}{(1 - t)^{n+1}}
            &= \frac{1}{n!} \frac{d^n}{dt^n} \frac{t^n}{1 - t} \\
            &= \frac{1}{n!} \sum_{k = 0}^\infty (k + n) \dots (k + 1) t^k \\
            &= \sum_{k = 0}^\infty \binom{k + n}{n}  t^k \\
            &= \sum_{k = 0}^\infty d_{n + 1}(k)  t^k
    \end{alignat}
    である。
\end{answer}


\begin{problem}[代数学II 2.20]
    \label[problem]{problem:algebra2-2.20}
    体$K$上の1変数多項式環$K[X]$は Euclid 整域であることを示せ。
\end{problem}

\begin{answer}
    $K[X]$が Euclid 整域であることを示す。
    まず$K[X]$が整域であることは、任意の$f, g \in K[X] - \{0\}$について
    \begin{alignat}{1}
        f(X) &= \sum_{i=0}^{\deg(f)} a_i X^i, \quad a_{\deg(f)} \neq 0 \\
        g(X) &= \sum_{i=0}^{\deg(g)} b_i X^i, \quad b_{\deg(g)} \neq 0
    \end{alignat}
    と表したときに、これらの積
    \begin{equation}
        f(X) \cdot g(X)
            = \sum_{i=0}^{\deg(f) + \deg(g)}
                \sum_{j=0}^i a_j b_{i-j} X^i
    \end{equation}
    の最高次係数$a_{\deg(f)} b_{\deg(g)}$が$0$でない
    ($\because$ $K$は整域) ことから従う。
    さらに写像$\deg \colon K[X] \to \Z_{\ge 0} \cup \{-\infty\}$は
    次をみたす:
    \begin{enumerate}
        \item $\deg(K[X] - \{0\}) \subset \Z_{\ge 0}$
        \item $\deg(0) = -\infty$
        \item $\forall f \in K[X]$と$\forall g \in K[X] - \{0\}$に対し、
            $K$が体ゆえに$g$の最高次係数は単元だから、
            多項式環の除法定理
            (\cref{thm:division-theorem-of-polynomial-ring})
            より$\exists q, r \in K[X]$が存在して
            \begin{equation}
                f = g \cdot q + r,\quad \deg(r) < \deg(g)
            \end{equation}
            が成り立つ。
    \end{enumerate}
    したがって$K[X]$は Euclid 整域である。
\end{answer}


\begin{problem}[代数学II 2.21]
    \label[problem]{problem:algebra-2.21}
    $K$を体、$f \in K[X], f \neq 0$で$f$の次数を$n$とする。
    このとき$\{ a \in K \colon f(a) = 0 \}$の濃度は
    $n$以下であることを示せ。
\end{problem}

\begin{remark}
    この問題の主張は$K$が整域ならば成り立つが、
    $K$が division algebra の場合は成り立たない (\cref{problem:algebra-1.13})。
    一般的に、可換性を要する命題の証明では、その過程で根の個数の不等式を利用することがよくある。
\end{remark}

\begin{answer}
    $n$に関する帰納法で示す。
    $n = 0$の場合は$f$の根は存在しないから主張が成り立つ。
    $n \ge 1$とし、すべての$k = 0, 1, \dots, n - 1$に対し
    主張の成立を仮定する。
    $f$が根を持たなければただちに主張が成り立つから、
    $f$は少なくとも1つの根$\alpha_0 \in K$を持つとする。
    剰余定理より
    \begin{equation}
        \exists g \in K[X]
        \quad \text{s.t.} \quad
        f(X) = (X - \alpha_0) g(X)
    \end{equation}
    が成り立つ。いま$K$は体、とくに整域だから
    $\alpha_0$以外の$f$の根は$g$の根でもある。
    さらに$\deg g = \deg f - 1 = n - 1$であることとあわせて、
    帰納法の仮定より
    \begin{alignat}{1}
        |\{ a \in K \colon f(a) = 0 \}|
            &\le |\{ a \in K \colon g(a) = 0 \} \cup \{ \alpha_0 \}| \\
            &\le |\{ a \in K \colon g(a) = 0 \}| + 1 \\
            &\le n
    \end{alignat}
    が成り立つ。
\end{answer}


\begin{problem}[代数学II 2.22]
    $K$を体とする。
    乗法群$K^\times$の任意の有限部分群は巡回群になることを示せ。
\end{problem}

\begin{answer}
    \TODO{トーシェント関数についてどこかに書きたい}

    Euler のトーシェント関数を$\varphi$とおく。
    すなわち
    \begin{equation}
        \varphi(n) = (\text{$n$と互いに素な$n$以下の正整数の個数})
        \quad (n \in \Z_{\ge 1})
    \end{equation}
    とおく。
    一般に、任意の巡回群$G$に対し
    \begin{equation}
        \label[equation]{eq:algebra-2.22-1}
        \sharp \{ x \in G \colon \text{$x$は$G$の生成元} \}
        = \varphi(|G|)
    \end{equation}
    が成り立つ。
    \begin{innerproof}
        $d = |G|$とおき、$G$の生成元$g_0$をひとつ固定する。
        $g \in G$を$G$の生成元とすると、
        ある$1 \le d' \le d$がただひとつ存在して
        $g = g_0^{d'}$が成り立つ。
        このとき$\gcd(d', d) = 1$である。
        実際、$g$が生成元であることより、ある$1 \le k \le d$が存在して
        \begin{equation}
            g_0 = g^k = g_0^{d'k}
        \end{equation}
        が成り立つ。
        よって、ある$l \in \Z$が存在して
        \begin{equation}
            1 = d'k + dl
        \end{equation}
        が成り立つ。よって$\gcd(d', d) = 1$である。
        逆に$1 \le d'' \le d$が$\gcd(d'', d) = 1$をみたすとすれば
        ある$k', l' \in \Z$が存在して
        \begin{equation}
            1 = d''k' + dl'
        \end{equation}
        が成り立つから、
        \begin{equation}
            g_0 = g_0^{d''k'}
        \end{equation}
        となり、したがって$g_0^{d''}$は$G$の生成元である。
        以上より全単射
        \begin{equation}
            \{ x \in G \colon \text{$x$は$G$の生成元} \}
            \leftrightarrow
            \{ k \in \{ 1, \dots, d \} \colon \gcd(k, d) = 1 \}
        \end{equation}
        が存在するから、求める主張が従う。
    \end{innerproof}
    また、一般に任意の正整数$n$に対し
    \begin{equation}
        \label[equation]{eq:algebra-2.22-2}
        n = \sum_{d | n} \varphi(d)
    \end{equation}
    が成り立つ。
    \begin{innerproof}
        巡回群$\Z / n\Z$を考える。
        各$d \in \Z_{\ge 1},\; d | n$に対し、
        $\Z / n\Z$の位数$d$の巡回部分群$H_d$はただひとつ存在する
        (ちなみにそれは$H_d = \biggl\langle \frac{n}{d} + n\Z \biggr\rangle$である)
        から、
        \begin{alignat}{1}
            \Z / n\Z
                &= \bigsqcup_{d | n} \{
                    x \in \Z / n\Z \colon \text{$x$の位数は$d$}
                \} \\
                &= \bigsqcup_{d | n} \{
                    x \in \Z / n\Z \colon \text{$x$は$H_d$の生成元}
                \}
        \end{alignat}
        が成り立つ。
        よって
        \begin{alignat}{1}
            n
                &= \sum_{d | n} \sharp \{
                    x \in \Z / n\Z \colon \text{$x$は$H_d$の生成元}
                \} \\
                &= \sum_{d | n} \varphi(d)
        \end{alignat}
        である。
    \end{innerproof}
    さて、$H \subset K^\times$を有限部分群とし、$H$が巡回群であることを示す。
    $n \coloneqq |H|$とおくと
    \begin{equation}
        H = \bigsqcup_{d | n} \{
            x \in H \colon \text{$x$の位数は$d$}
        \}
    \end{equation}
    が成り立つ。
    $d \in \Z_{\ge 1},\; d | n$とし、
    $H$の位数$d$の元$x_0$が存在したとする。
    すると
    \begin{equation}
        \langle x_0 \rangle
            \subset \{ x \in H \colon x^d = 1 \}
    \end{equation}
    が成り立つが、いま$H$は体$K$の部分集合であったから
    右辺の集合の濃度は$d$以下である (\cref{problem:algebra-2.21})。
    このことと、左辺の集合の濃度が$d$であることをあわせて
    \begin{equation}
        \langle x_0 \rangle
            = \{ x \in H \colon x^d = 1 \}
    \end{equation}
    が成り立つ。
    よって$x_0$は巡回群$\{ x \in H \colon x^d = 1 \}$の生成元である。
    したがって、各$d \in \Z_{\ge 1},\; d | n$に対し
    \begin{equation}
        H_d \coloneqq \{ x \in H \colon x^d = 1 \}
    \end{equation}
    とおけば
    \begin{equation}
        \{
            x \in H \colon \text{$x$の位数は$d$}
        \}
            = \begin{cases}
                \{
                    x \in H \colon \text{$x$は$H_d$の生成元}
                \} \quad & (\text{$H$が位数$d$の元をもつ}) \\
                \emptyset \quad & (\text{otherwise})
            \end{cases}
    \end{equation}
    が成り立つ。
    したがって
    \begin{alignat}{1}
        n
            &= |H| \\
            &= \sum_{d | n} \sharp \{
                x \in H \colon \text{$x$の位数は$d$}
            \} \\
            &\le \sum_{d | n} \sharp \{
                x \in H \colon \text{$x$は$H_d$の生成元}
            \} \\
            &= \sum_{d | n} \varphi(d)
            \quad (\because \cref{eq:algebra-2.22-1}) \\
            &= n
            \quad (\because \cref{eq:algebra-2.22-2})
    \end{alignat}
    が成り立つ。
    よってとくに集合$\{ x \in H \colon \text{$x$の位数は$d$} \}$は空でなく、
    $H$は位数$n$の元をもつ。
    したがって$H$は巡回群である。
\end{answer}


\begin{problem}[代数学II 2.23]
    Gauss 整数環$\Z[i] = \{ m + ni \colon m, n \in \Z \}$は
    Euclid 整域であることを示せ。
\end{problem}

\begin{answer}
    写像$N \colon \Z[i] \to \Z$を
    \begin{equation}
        N(m + ni) \coloneqq m^2 + n^2
    \end{equation}
    で定める。$\C$における絶対値の性質から明らかに
    \begin{equation}
        N(\alpha \beta) = N(\alpha) N(\beta)
        \quad (\forall \alpha, \beta \in \Z[i])
    \end{equation}
    が成り立つ。さて、$\alpha, \beta \in \Z[i],\; \beta \neq 0$に対し
    \begin{equation}
        \exists \gamma, \delta \in \Z[i]
        \quad \text{s.t.} \quad
        \alpha = \beta \gamma + \delta,\quad N(\delta) < N(\beta)
    \end{equation}
    を示す。目標の式から逆算して形式的に変形してみると
    \begin{equation}
        \frac{\alpha}{\beta} - \gamma = \frac{\delta}{\beta}
    \end{equation}
    を得る。右辺の絶対値をできるだけ小さくすればうまくいきそうである。
    そこで$\alpha / \beta$に最も近い Gauss 整数のひとつを$\gamma \in \Z[i]$とおき、
    $\delta \coloneqq \alpha - \beta \gamma \in \Z[i]$とおく。
    あとは$N(\delta) < N(\beta)$を示せばよい。
    $\gamma$の定め方から明らかに
    \begin{equation}
        \left| \gamma - \frac{\alpha}{\beta} \right| < 1
    \end{equation}
    なので
    \begin{alignat}{1}
        1 > N\left(\frac{\delta}{\beta}\right)
            = \frac{N(\delta)}{N(\beta)}
    \end{alignat}
    よって$N(\delta) < N(\beta)$が成り立つ。
    そこで写像$N' \colon \Z[i] \to \Z_{\ge 0} \cup \{ -\infty \}$を
    \begin{equation}
        N'(\alpha) \coloneqq \begin{cases}
            N(\alpha) - 1 & (\alpha \neq 0) \\
            -\infty & (\alpha = 0)
        \end{cases}
    \end{equation}
    と定めれば$N'$は
    \begin{enumerate}
        \item $N'(\Z[i] - \{0\}) \subset \Z_{\ge 0}$
        \item $N'(0) = -\infty$
        \item $\forall \alpha \in \Z[i]$と$\beta \in \Z[i] - \{0\}$に対し、
            $\exists \gamma, \delta \in \Z[i]$が存在して
            \begin{equation}
                \alpha = \beta \gamma + \delta,\quad N'(r) < N'(g)
            \end{equation}
            が成り立つ (除法定理)。
    \end{enumerate}
    をみたすから、$\Z[i]$は Euclid 整域である。
\end{answer}


\begin{problem}[代数学II 2.24]
    \label[problem]{problem:algebra-2.24}
    $K$を無限個の元を持つ体とする。
    このとき$\mathrm{Map}(K^n, K)$で
    $K^n$から$K$への写像全体のなす集合とする。
    $K$上の$n$変数多項式環$K[X_1, \dots, X_n]$から$\mathrm{Map}(K^n, K)$への写像$F$を
    $K[X_1, \dots, X_n]$の元を対応する多項式関数に写すことで与える。
    このとき$F$は単射であることを示せ。
\end{problem}

\begin{answer}
    各$n \in \Z_{\ge 1}$に対し$F$を$F_n$と書くことにし、
    $\mathrm{Map}(K^n, K)$には$K$上の和と積から自然に環構造を入れる。
    すると各$F_n$は環準同型である。
    よって、$F_n$の単射性を示すには$\Ker F_n = \{ 0_{K[X_1, \dots, K_n]} \}$を
    示せばよい。
    これを$n$に関する数学的帰納法で示す。

    \uline{[1]} \quad
    $\Ker F_1 \subset \{ 0_{K[X_1]} \}$を示す。
    そこで$f \in K[X_1]$について$f \neq 0_{K[X_1]}$とすると、
    $K$が体であることから$f$の根は有限個である。
    一方、問題の仮定より$K$は無限個の元を持つから、
    或る$b_1 \in K$が存在して$f(b_1) \neq 0$が成り立つ。
    よって$F_1(f) \neq 0_{\mathrm{Map}(K^1, K)}$である。
    したがって$\Ker F_1 \subset \{ 0_{K[X_1]} \}$である。
    逆の包含は明らかだから$\Ker F_1 = \{ 0_{K[X_1]} \}$が成り立つ。

    \uline{[2]} \quad
    $n - 1$のとき成立を仮定し、
    $\Ker F_n \subset \{ 0_{K[X_1, \dots, X_n]} \}$を示す。
    そこで$f \in K[X_1, \dots, X_n]$とする。
    $f$は
    \begin{equation}
        f = \sum_{\substack{0 \le i_1 \le d_1 \\ \cdots \\ 0 \le i_n \le d_n}}
            a_{i_1 \dots i_n} X_1^{i_1} \cdots X_n^{i_n}
            \quad (a_{i_1 \dots i_n} \in K)
    \end{equation}
    と表せる。
    ここで$f \neq 0_{K[X_1, \dots, X_n]}$とすると、
    或る$0 \le \exists k_j \le d_j \; (j = 1, \dots, n)$が存在して
    \begin{equation}
        a_{k_1 \dots k_n} \neq 0
    \end{equation}
    が成り立つ。
    さて、$F_n(f) \neq 0_{\mathrm{Map}(K^n, K)}$を示したい。
    ここで$(K[X_1, \dots, X_{n - 1}])[X_n]$の元
    \begin{equation}
        \sum_{i_n} \left(
            \sum_{i_1, \dots, i_{n - 1}}
                a_{i_1 \dots i_n} X_1^{i_1} \cdots X_{n - 1}^{i_{n - 1}}
        \right) X_n^{i_n}
    \end{equation}
    を考えると、
    $a_{k_1 \dots k_n} \neq 0$より
    $k_n$次の係数は
    \begin{equation}
        \sum_{i_1, \dots, i_{n - 1}}
            a_{i_1 \dots i_{n - 1} k_n} X_1^{i_1} \cdots X_{n - 1}^{i_{n - 1}}
            \neq 0_{K[X_1, \dots, X_{n - 1}]}
    \end{equation}
    をみたす。したがって帰納法の仮定より
    \begin{equation}
        F_{n - 1} \left(
            \sum_{i_1, \dots, i_{n - 1}}
            a_{i_1 \dots i_{n - 1} k_n} X_1^{i_1} \cdots X_{n - 1}^{i_{n - 1}}
        \right) \neq 0_{\mathrm{Map}(K^{n - 1}, K)}
    \end{equation}
    である。よって、或る$(b_1, \dots, b_{n - 1}) \in K^{n - 1}$が存在して
    \begin{equation}
        \sum_{i_1, \dots, i_{n - 1}}
            a_{i_1 \dots i_{n - 1} k_n} b_1^{i_1} \cdots b_{n - 1}^{i_{n - 1}}
            \neq 0
    \end{equation}
    が成り立つ。
    そこで$K[X_n]$の元
    \begin{equation}
        \sum_{i_n} \left(
            \sum_{i_1, \dots, i_{n - 1}}
                a_{i_1 \dots i_n} b_1^{i_1} \cdots b_{n - 1}^{i_{n - 1}}
        \right) X_n^{i_n}
    \end{equation}
    を考えると、これは$k_n$次の係数が$0$でないから
    $\neq 0_{K[X_n]}$となる。
    よって、$n = 1$の場合と同様の議論により、
    或る$b_n \in K$が存在して
    \begin{equation}
        \sum_{i_n} \left(
            \sum_{i_1, \dots, i_{n - 1}}
                a_{i_1 \dots i_n} b_1^{i_1} \cdots b_{n - 1}^{i_{n - 1}}
        \right) b_n^{i_n}
        \neq 0
    \end{equation}
    が成り立つ。
    よって
    \begin{alignat}{1}
        F_n(f) (b_1, \dots, b_n)
            &= \sum_{i_1, \dots, i_n}
                a_{i_1 \dots i_n} b_1^{i_1} \cdots b_n^{i_n} \\
            &= \sum_{i_n} \left(
                \sum_{i_1, \dots, i_{n - 1}}
                    a_{i_1 \dots i_n} b_1^{i_1} \cdots b_{n - 1}^{i_{n - 1}}
            \right) b_n^{i_n} \\
            &\neq 0
    \end{alignat}
    である。したがって
    \begin{equation}
        F_n(f) \neq 0_{\mathrm{Map}(K^n, K)}
    \end{equation}
    である。
    これで$\Ker F_n \subset \{ 0_{K[X_1, \dots, X_n]} \}$がいえた。
    逆の包含は明らかだから$\Ker F_n = \{ 0_{K[X_1, \dots, X_n]} \}$、
    したがって$F_n$は単射である。
    帰納法より、すべての$n \in \Z_{\ge 1}$に対して
    $F_n$が単射であることが示せた。
\end{answer}


\begin{problem}[代数学II 2.25]
    $K$を有限個の元からなる体とする。
    \cref{problem:algebra-2.24}のような写像
    $F \colon K[X_1, \dots, X_n] \to \mathrm{Map}(K^n, K)$を考えると
    $F$は全射であることを示せ。
\end{problem}

\begin{answer}
    各$(a_1, \dots, a_n) \in K^n$に対し、
    多項式$\delta_{a_1 \dots a_n}$を
    \begin{equation}
        \delta_{a_1 \dots a_n}(X_1, \dots, X_n)
            \coloneqq \prod_{i = 1}^n \prod_{\substack{t \in K \\ t \neq a_i}}
            (a_i - t)^{-1} (X_i - t)
    \end{equation}
    で定めると、
    $K$が有限個の元からなることから右辺は有限積となり
    $\delta_{a_1 \dots a_n}$は well-defined である。
    $\delta_{a_1 \dots a_n}$は$(t_1, \dots, t_n) \in K^n$に対し
    \begin{equation}
        \delta_{a_1 \dots a_n}(t_1, \dots, t_n)
            = \begin{cases}
                1 & (t_1, \dots, t_n) = (a_1, \dots, a_n) \\
                0 & \text{otherwise}
            \end{cases}
    \end{equation}
    をみたす。
    そこで$f \in \mathrm{Map}(K^n, K)$に対し
    多項式$g \in K[X_1, \dots, X_n]$を
    \begin{equation}
        g(X_1, \dots, X_n) \coloneqq \sum_{(a_1, \dots, a_n) \in K^n}
            f(a_1, \dots, a_n) \delta_{a_1 \dots a_n}(X_1, \dots, X_n)
    \end{equation}
    で定めると、
    $K$、したがって$K^n$が有限個の元からなることから
    右辺は有限和となり$g$は well-defined である。
    $g$は$(t_1, \dots, t_n) \in K^n$に対し
    \begin{equation}
        F(g)(t_1, \dots, t_n) = f(t_1, \dots, t_n)
    \end{equation}
    をみたす。よって$F$は全射である。
\end{answer}


\begin{problem}[代数学II 2.26]
    加法群が巡回群になるような環を
    同型を除いてすべて決定せよ。
\end{problem}

\begin{answer}
    \TODO{}
\end{answer}


\begin{problem}[代数学II 2.27]
    1つの元で$\C$上生成される$\C$-代数で零因子を持たないものを
    同型を除いてすべて決定せよ。
\end{problem}

\begin{answer}
    $A$を1つの元$s \in A$で$\C$上生成される$\C$-代数とする。
    このとき評価準同型$\ev_s \colon \C[X] \to A$は全射だから、
    準同型定理より$A \cong \C[X] / \Ker \ev_s$が成り立つ。
    ここで$\C[X]$のイデアルは$(0), (1), (X^m) \; (m \in \Z_{\ge 1})$で尽くされるから
    $\Ker \ev_s$はこれらのいずれかに一致する。
    $\Ker \ev_s = (0)$なら$A \cong \C[X] / (0) \cong \C[X]$であり、
    これは整域だから零因子をもたない。
    $\Ker \ev_s = (1)$なら$A \cong \C[X] / (1) = 0$であり、
    これは零因子をもたない。
    $\Ker \ev_s = (X)$なら$A \cong \C[X] / (X)$であり、
    $(X)$は素イデアルだから$\C[X] / (X)$は整域で零因子をもたない。
    $\Ker \ev_s = (X^m) \; (m \in \Z_{\ge 2})$なら
    $A \cong \C[X] / (X^m)$は$s \neq 0, \; s^m = 0$となり
    $s$が零因子となる。
    よって$A$は
    $(0), \C[X], \C[X] / (X)$のいずれかであり、
    逆にこれらはそれぞれ$0, X, X + (X)$により
    $\C$上生成される$\C$-代数で零因子を持たない。
    したがって求めるものは
    $(0), \C[X], \C[X] / (X)$である。
\end{answer}


\begin{problem}[代数学II 2.28]
    $\C[X]^\times \cong \C^\times$および
    $(\C[X] / (X^{n + 1}))^\times \cong \C^\times \times \C^n$を示せ。
    ただし$\C^n$は$\C$を加法群とみたものの$n$個のコピーの直積である。
\end{problem}

\begin{answer}
    \TODO{}
\end{answer}


\begin{problem}[代数学II 2.29]
    \label[problem]{problem:algebra2-2.29}
    $R$を可換環、$I$を$R$の固有イデアルとする。
    このとき
    \begin{equation}
        \sqrt{I} \coloneqq \{
            a \in R \colon a^n \in I \; (\exists n \in \Z_{\ge 1})
        \}
    \end{equation}
    とおくと、$\sqrt{I}$は$R$の固有イデアルとなることを示せ。
\end{problem}

\begin{answer}
    $\sqrt{I}$が$R$のイデアルであるとは
    \begin{enumerate}
        \item $\sqrt{I}$は$R$の加法部分群である。
        \item $p \in R, \; q \in \sqrt{I}$ならば$pq \in \sqrt{I}$である。
    \end{enumerate}
    が成り立つことであった。

    まず(1)を示す。
    $a, b \in \sqrt{I}$をとる。
    $a + b \in \sqrt{I}$を示す。
    このとき
    \begin{equation}
        \exists \; m, n \ge 1
        \quad \text{s.t.} \quad
        a^m \in I, \; b^n \in I
    \end{equation}
    が成り立つ。
    このとき
    \begin{equation}
        (a + b)^{m + n}
            = \underbrace{a^{m + n}}_{\in I}
            + \binom{m + n}{1} \underbrace{a^{m + n - 1} b}_{\in I}
            + \dots
            + \binom{m + n}{m + n - 1} \underbrace{a b^{m + n - 1}}_{\in I}
            + \underbrace{b^{m + n}}_{\in I}
    \end{equation}
    となる。
    よって$a + b \in \sqrt{I}$となる。
    よって$\sqrt{I}$は加法について閉じている。
    また、$\sqrt{I}$は加法の単位元$0$を含む。
    なぜなら$0 = 0^1 \in I$だからである。
    つぎに$a \in \sqrt{I}$をとる。
    $-a \in \sqrt{I}$を示す。
    $a \in \sqrt{I}$より
    \begin{equation}
        \exists \; m \ge 1
        \quad \text{s.t.} \quad
        a^m \in I
    \end{equation}
    である。
    このとき$(-1)^m a^m \in I$なので、
    $R$が可換であることより$(-a)^m \in I$である。
    したがって$-a \in \sqrt{I}$となる。
    以上より$\sqrt{I}$は$R$の加法部分群である。

    (2)を示す。
    $p \in R, \; q \in \sqrt{I}$をとる。
    $pq \in \sqrt{I}$を示す。
    $q \in \sqrt{I}$より
    \begin{equation}
        \exists \; m \ge 1
        \quad \text{s.t.} \quad
        q^m \in I
    \end{equation}
    となる。
    これと$p^m \in R$より$p^m q^m \in I$となる。
    $R$が可換であることより$(pq)^m \in I$である。
    よって$pq \in \sqrt{I}$となる。
\end{answer}


\begin{problem}[代数学II 2.30]
    \label[problem]{problem:algebra2-2.30}
    体$K$上の1変数形式的べき級数環$K[[X]]$のイデアルをすべて求め
    $K[[X]]$が局所環かつPIDであることを示せ。
\end{problem}

\begin{answer}
    $K[[X]]$のイデアルが
    \begin{equation}
        (0), (X^d)\; (d \in \Z_{\ge 0})
    \end{equation}
    で尽くされることを示せばよく、
    このときイデアルの形から明らかに$K[[X]]$は局所環かつPIDである。
    各$f = \sum_{i \in \Z_{\ge 0}} a_i X^i \in K[[X]]$に対し、
    係数が単元である次数の最小値を$\deg^-(f)$と表すことにする。すなわち、
    \begin{equation}
        \deg^-(f) \coloneqq \min (\{
            i \in \Z_{\ge 0} \colon a_i \in K^\times
        \} \cup \{ \infty \})
    \end{equation}
    とおく。
    $I \subset K[[X]]$を$I \neq (0)$なるイデアルとする。ここで
    \begin{equation}
        d \coloneqq \min \{ \deg^-(f) \colon f \in I \}
    \end{equation}
    とおくと、$I \neq (0)$より$d \in \Z_{\ge 0}$である。
    $I = (X^d)$を示せばよい。
    ここで、$d$の定め方から
    $\deg^-(f_0) = d$なる$f_0 \in I$が存在し、
    $\deg^-$の定義から
    \begin{equation}
        \exists g = \sum_{i \in \Z_{\ge 0}} a_i X^i \in K[[X]]
        \quad \text{s.t.} \quad
        a_0 \in K^\times,\;
        f_0(X) = g(X) X^d
    \end{equation}
    が成り立つ。
    このとき、$a_0 \in K^\times$より$g \in K[[X]]^\times$である。
    \begin{innerproof}
        $a_0 \in K^\times$より
        \begin{alignat}{1}
            g(X) = a_0 \bigg(
                1 - \underbrace{
                    \bigg(- \sum_{i \in \Z_{\ge 0}} a_0^{-1} a_{i + 1} X^i \bigg)
                }_{\eqqcolon h(X)}
                X
            \bigg)
                = a_0 (1 - h(X) X)
        \end{alignat}
        であり、
        \begin{equation}
            (1 - h(X) X) \sum_{i \in \Z_{\ge 0}} h(X)^i X^i = 1
        \end{equation}
        より
        \begin{equation}
            g(X) a_0^{-1} \sum_{i \in \Z_{\ge 0}} h(X)^i X^i = 1
        \end{equation}
        が成り立つ。よって$g \in K[[X]]^\times$である。
    \end{innerproof}
    したがって
    \begin{equation}
        X^d = g(X)^{-1} f_0(X) \in I
    \end{equation}
    より$(X^d) \subset I$である。
    また、$d$の定め方から逆向きの包含も成り立つ。
    したがって$I = (X^d)$である。
    これが示したいことであった。
\end{answer}


\begin{problem}[代数学II 2.31]
    \label[problem]{problem:algebra-2.31}
    (\term{中国剰余定理}[Chinese Remainder Theorem]{中国剰余定理}[ちゅうごくじょうよていり])
    $R$を可換環とし、$I_1, \dots, I_n$をその固有イデアルとする。
    さらに$i \neq j$なる$1 \le i, j \le n$に対し$I_i + I_j = R$をみたすとする。
    このとき、
    \begin{equation}
        I_1 \cap \dots \cap I_n = I_1 \cdots I_n
    \end{equation}
    となることを示せ。
\end{problem}

\begin{answer}
    $(\supset)$は明らかだから$(\subset)$を示す。
    $n$に関する帰納法で示す。$n = 1$のときは明らかだから、$n \ge 2$とし、
    $a \in I_1 \cap \dots \cap I_n$とする。
    ここで、各$i = 1, \dots, n$に対し
    或る$c_i \in I_i$と$d_i \in \bigcap_{j \neq i} I_j$が存在して
    \begin{equation}
        c_i + d_i = 1
    \end{equation}
    が成り立つ。
    \begin{innerproof}
        $R = I_i + \bigcap_{j \neq i} I_j$を示せばよい。
        各$j \neq i$に対し、
        問題の仮定$I_i + I_j = R$より
        或る$a_j \in I_i$と$b_j \in I_j$が存在して
        \begin{equation}
            a_j + b_j = 1
        \end{equation}
        が成り立つ。このとき$\prod_{j \neq i} (a_j + b_j) = 1$であり、
        左辺を展開すると$a_j \in I_i \; (j \neq i)$を含む項と
        $\prod_{j \neq i} b_j \in \bigcap_{j \neq i} I_j$との和の形になるから、
        整理して
        \begin{equation}
            1 - \prod_{j \neq i} b_j \in I_i
        \end{equation}
        を得る。よって
        \begin{alignat}{1}
            1 &= \bigg( 1 - \prod_{j \neq i} b_j \bigg) + \prod_{j \neq i} b_j \\
                &\in I_i + \bigcap_{j \neq i} I_j
        \end{alignat}
        が成り立つ。したがって$R = I_i + \bigcap_{j \neq i} I_j$がいえた。
    \end{innerproof}
    このとき
    \begin{equation}
        \prod_{i = 1}^n (c_i + d_i) = 1
        \quad \therefore \quad
        a \prod_{i = 1}^n (c_i + d_i) = a
    \end{equation}
    が成り立つ。左辺は
    \begin{equation}
        ad_i \in I_i \bigcap_{j \neq i} I_j
            \overbrace{=}^{
                \substack{\text{帰納法} \\ \text{の仮定}}
            } I_1 \cdots I_n
        \quad (i = 1, \dots, n)
    \end{equation}
    を含む項と
    \begin{equation}
        a \prod_{i = 1}^n c_i \in I_1 \cdots I_n
    \end{equation}
    との和の形になるから$a \in I_1 \cdots I_n$である。
    よって$I_1 \cap \dots \cap I_n = I_1 \cdots I_n$がいえた。
\end{answer}


\begin{problem}[代数学II 2.32]
    (\term{Frobenius 準同型}{Frobenius 準同型}[Frobenius じゅんどうけい])
    $R$を可換環、$p$を素数とする。
    さらに$\iota \colon \Z \to R$を一意的にきまる環準同型とし
    $\Ker(\iota) = (p)$であると仮定する。
    このとき写像$F \colon R \to R$を$F(x) = x^p$で定めると
    これは環準同型になることを示せ。
\end{problem}

\begin{answer}
    二項係数$\binom{p}{i} \; (1 \le \forall i \le p - 1)$は$p$で割り切れるから、
    $x, y \in R$に対し
    \begin{alignat}{1}
        F(x + y)
            &= (x + y)^p \\
            &= \sum_{i = 0}^{p} \binom{p}{i} x^i y^{p - i}
                \quad (\because \text{ $R$は可換環}) \\
            &= x^p + y^p + \sum_{i = 1}^{p - 1} \binom{p}{i} x^i y^{p - i} \\
            &= x^p + y^p + \sum_{i = 1}^{p - 1}
                \iota\left(\binom{p}{i}\right) x^i y^{p - i} \\
            &= x^p + y^p
                \quad (\because \text{ $\Ker(\iota) = (p)$}) \\
            &= F(x) + F(y)
    \end{alignat}
    が成り立つ。
    また
    \begin{alignat}{1}
        F(xy) 
            &= (xy)^p \\
            &= x^p y^p \quad (\because \text{ $R$は可換環}) \\
            &= F(x) F(y)
    \end{alignat}
    と
    \begin{equation}
        F(1_R) = 1_R^p = 1_R
    \end{equation}
    も成り立つ。よって$F$は環準同型である。
\end{answer}


\begin{problem}[代数学II 2.33]
    $\Z$を加法群とみたときの群環$\C[\Z]$のイデアル
    および素イデアルをすべて求めよ。
\end{problem}

\begin{answer}
    \TODO{}
\end{answer}


\begin{problem}[代数学II 2.34]
    $p$を素数とするとき
    $\Z / (p^{k + 1})$の乗法群を求めよ。
\end{problem}

\begin{answer}
    \TODO{}
\end{answer}

\begin{problem}[代数学II 2.35]
    極大両側イデアルは素イデアルになることを示せ。
\end{problem}

\begin{answer}
    \TODO{}
\end{answer}


\begin{problem}[代数学II 2.36]
    \label[problem]{problem:algebra2-2.36}
    完全素イデアルは素イデアルになることを示せ。
    また、逆は成り立つか?
\end{problem}

\begin{answer}
    $A$を環とし、$P \subset A$を完全素イデアルとする。
    $A$の固有両側イデアル$I, J$が$IJ \subset P$をみたすとし、
    さらに$I \not\subset P$を仮定する。
    仮定より或る$i_0 \in I \setminus P$が存在する。
    このとき、任意の$j \in J$に対し$j \in P$が成り立つ。
    実際、$i_0 j \in IJ \subset P$だから
    $P$が完全素イデアルであることより$i_0 \in P$または$j \in P$であるが、
    いま$i_0 \not\in P$であったから$j \in P$である。
    したがって$J \subset P$である。
    よって$P$は素イデアルである。

    逆が成り立たないことを反例によって示す。
    $\R$上の$2$次の全行列環$M_2(\R)$と、その零イデアル$(0)$を考える。
    まず、$(0)$は素イデアルであることを示したい。
    そのために$M_2(\R)$の両側イデアルが自明なものしかないことを示しておく。
    $I \subset M_2(\R)$を$I \neq (0)$なる両側イデアルとする。
    $I \neq (0)$より或る$B_0 \in I, B_0 \neq 0$が存在する。
    このとき$\rank B_0 > 0$だから、或る正則行列$Q, R \in M_2(\R)$が存在して
    \begin{equation}
        QB_0R = \begin{bmatrix}
            1 & 0 \\
            0 & *
        \end{bmatrix}
    \end{equation}
    が成り立つ。
    $I$は両側イデアルであったから
    \begin{equation}
        \begin{bmatrix}
            1 & 0 \\
            0 & *
        \end{bmatrix} \in I
    \end{equation}
    が成り立つ。
    さらに
    \begin{alignat}{1}
        \begin{bmatrix}
            1 & 0 \\
            0 & 0
        \end{bmatrix}
        =
        \begin{bmatrix}
            1 & 0 \\
            0 & *
        \end{bmatrix}
        \begin{bmatrix}
            1 & 0 \\
            0 & 0
        \end{bmatrix}
        \in I,
        \quad
        \begin{bmatrix}
            0 & 0 \\
            0 & 1
        \end{bmatrix}
        =
        \begin{bmatrix}
            0 & 0 \\
            1 & 0
        \end{bmatrix}
        \begin{bmatrix}
            1 & 0 \\
            0 & *
        \end{bmatrix}
        \begin{bmatrix}
            0 & 1 \\
            0 & 0
        \end{bmatrix}
        \in I
    \end{alignat}
    も成り立つ。
    よって
    \begin{equation}
        1_{M_2(\R)}
            = \begin{bmatrix}
                1 & 0 \\
                0 & 0
            \end{bmatrix}
            + \begin{bmatrix}
                0 & 0 \\
                0 & 1
            \end{bmatrix}
            \in I
    \end{equation}
    である。したがって$I = M_2(\R)$となる。
    よって$M_2(\R)$の両側イデアルは自明なものしかないことがいえた。
    よって、明らかに$(0)$は素イデアルである。
    ところが、$(0)$は完全素イデアルではない。
    実際、$M_2(\R) / (0) \cong M_2(\R)$は零因子を持つ。
\end{answer}


\begin{problem}[代数学II 期末問題候補]
    $R$を可換環、$A$を$R$-alg とし、$S \subset A$とする。
    \begin{equation}
        \calS_S
            \coloneqq \{ B \subset A \colon \text{$B$は$R$-subalg かつ$S \subset B$} \}
    \end{equation}
    とおくとき
    \begin{equation}
        R \langle S \rangle = \bigcap_{B \in \calS_S} B
    \end{equation}
    となることを示せ。
\end{problem}

\begin{answer}
    \TODO{}
\end{answer}

\subsection{Problem set 3}


\begin{problem}[代数学II 3.37]
    \TODO{}
\end{problem}

\begin{answer}
    \TODO{}
\end{answer}


\begin{problem}[代数学II 3.38]
    \label[problem]{problem:algebra2-3.38}
    $k$を正整数、$X, Y$を不定元として
    $A = \C[X, Y] / (X^2 - Y^{2k + 1})$とおく。
    $x \in A$を標準射影による$X$の像としたとき、
    $x$は既約元であるが素元ではないことを示せ。
\end{problem}

\begin{answer}
    $Z$を不定元とする多項式環$\C[Z]$を考える。
    部分集合$\{ Z^{2k + 1}, Z^2 \} \subset \C[Z]$
    により$\C$上生成される$\C[Z]$の部分$\C$-代数
    $\C\langle \{ Z^{2k + 1}, Z^2 \} \rangle$
    を$B$とおく。
    $A$は$B$と$\C$-代数として同型であることを示す。
    多項式環の普遍性より、
    $\C$-代数準同型$\varphi \colon \C[X, Y] \to \C[Z]$であって
    \begin{equation}
        X \mapsto Z^{2k + 1}, \quad Y \mapsto Z^2
    \end{equation}
    をみたすものが存在する。
    このとき、$\Im \varphi = B$である。
    \begin{innerproof}
        両辺とも$Z^{2k + 1}, Z^2 \in \C[Z]$の
        和、積および$\C$の元によるスカラー倍の有限回の組み合わせで
        表せる元全体の集合だから、たしかに一致する。
    \end{innerproof}
    また、$\Ker \varphi = (X^2 - Y^{2k + 1})$である。
    \begin{innerproof}
        ($\supset$) は$\varphi$の定義より明らかだから、
        ($\subset$) を示す。
        $f \in \Ker \varphi$とする。
        $f$と$X^2 - Y^{2k + 1}$を ($\C$-代数としての) 同型$\C[X, Y] \cong (\C[Y])[X]$により
        $(\C[Y])[X]$の元とみなすと、
        $X^2 - Y^{2k + 1}$の最高次係数$1$は$\C[X, Y]$の単元だから、
        除法定理より
        或る$g(X, Y) \in  (\C[Y])[X]$と$r_1, r_0 \in \C[Y]$が存在して
        \begin{equation}
            f(X, Y) = (X^2 - Y^{2k + 1}) g(X, Y)
                + r_1(Y) X + r_0(Y)
        \end{equation}
        が成り立つ。
        いま$f \in \Ker \varphi$ゆえに$f(Z^{2k + 1}, Z^2) = 0$だから
        \begin{equation}
            r_1(Z^2) Z^{2k + 1} + r_0(Z^2) = 0
        \end{equation}
        である。左辺の第1項は奇数次の項しかなく、第2項は偶数次の項しかないから、
        右辺と係数を比較して$r_1 = 0, r_0 = 0$となる。
        よって
        \begin{equation}
            f(X, Y) = (X^2 - Y^{2k + 1}) g(X, Y)
                \in (X^2 - Y^{2k + 1})
        \end{equation}
        が成り立つから$\Ker \varphi \subset (X^2 - Y^{2k + 1})$である。
        したがって$\Ker \varphi = (X^2 - Y^{2k + 1})$である。
    \end{innerproof}
    したがって、準同型定理より$\C$-代数の同型
    $\wb{\varphi} \colon B \to A$が誘導される。
    $\wb{\varphi}$により$x \in A$と対応する元は$Z^{2k + 1} \in B$だから、
    $Z^{2k + 1}$が$B$の既約元であって素元でないことを示せばよい。
    まず、$Z^{2k + 1}$は$B$の既約元である。
    \begin{innerproof}
        背理法を用いる。すなわち$Z^{2k + 1}$が$B$の既約元でないと仮定して矛盾を導く。
        まず$\deg Z^{2k + 1} \neq 0$より$Z^{2k + 1}$は$B$の単元でない。
        よって、背理法の仮定から$Z^{2k + 1}$は$B$の単元でない2元$f, g$の積で
        $Z^{2k + 1} = f(Z) g(Z)$と表せる。
        すると$\C[Z]$が整域であることとあわせて
        \begin{equation}
            \deg f \ge 1, \quad \deg g \ge 1,
            \quad
            \deg f + \deg g = 2k + 1
        \end{equation}
        が成り立ち、とくに
        \begin{equation}
            1 \le \deg f \le 2k, \quad 1 \le \deg g \le 2k
        \end{equation}
        が成り立つ。
        このことと$f, g \in B$であることから、
        $f, g$は$Z^2$の和、積および$\C$の元によるスカラー倍の
        有限回の組み合わせで表せる。
        よって$f, g$の次数は偶数であり、$\deg f + \deg g = 2k + 1$に矛盾する。
        背理法より$Z^{2k + 1}$は$B$の既約元である。
    \end{innerproof}
    さらに、$Z^{2k + 1}$は$B$の素元でない。
    \begin{innerproof}
        $Z^{2k + 1}$により生成される$B$の単項イデアルを$J$とおく
        ($\C[Z]$でなく$B$のイデアルだから、たとえば$Z^{2k + 2}$は$J$には属さない)。
        すると
        \begin{equation}
            Z^{4k} Z^2 = (Z^{2k + 1})^2 \in J,
            \quad Z^{4k}, Z^2 \not\in J
        \end{equation}
        だから$J$は$B$の素イデアルでない。
        よって$Z^{2k + 1}$は$B$の素元でない。
    \end{innerproof}
    以上より、$x$は$A$の既約元だが素元ではないことが示せた。
\end{answer}


\begin{problem}[代数学II 3.39]
    \label[problem]{problem:algebra2-3.39}
    PIDにおいて既約元は素元になることを示せ。
\end{problem}

\begin{answer}
    $R$をPIDとし、$r \in R$が既約元であるとする。
    まず$(r)$は極大イデアルである。
    \begin{innerproof}
        $(r)$を含む任意の固有イデアルは、
        $R$がPIDであることからある$a \in R$を用いて$(a)$の形に書ける。
        すると$r \in (r) \subset (a)$より
        $r = ab \; (\exists b \in R)$と書ける。
        いま$(a)$は固有イデアルゆえに$a$は単元でないから、
        $r$が既約元であることとあわせて$b$は単元となる。
        よって$a = rb^{-1} \in (r)$だから$(a) \subset (r)$である。
        よって$(r) = (a)$である。
        したがって$(r)$は極大イデアルである。
    \end{innerproof}
    極大イデアルは素イデアルだから$(r)$は素イデアルである。
    よって$r$は素元である。
\end{answer}


\begin{problem}[代数学II 3.40]
    \label[problem]{problem:algebra2-3.40}
    $A$を可換$\C$-代数で$d = \dim_\C A$としたとき$0 < d < \infty$であるとする。
    このとき$A$は高々$d$個しか極大イデアルを持たないことを示せ。
\end{problem}

\begin{answer}
    \TODO{可換性いつ使う?}
    $A$の相異なる$r > d$個の極大イデアル
    $\frakm_1, \dots, \frakm_r$がとれたとして矛盾を導く。
    CRT より
    \begin{equation}
        \begin{tikzcd}
            A \ar{d} \ar{r}
                & A / \frakm_1 \times \cdots \times A / \frakm_r
                = A / \frakm_1 \oplus \cdots \oplus A / \frakm_r \\
            A / (\frakm_1 \cap \cdots \cap \frakm_r)
                \ar[dashed, end anchor=south west]{ru}[swap]{\cong}
        \end{tikzcd}
    \end{equation}
    を可換にする環の同型が誘導されるから、
    図式の上側の射 (これは$\C$-代数準同型である) は全射である。
    よって
    \begin{equation}
        \dim_C A
            \ge \dim_C (
                A / \frakm_1 \oplus \cdots \oplus A / \frakm_r
            )
            \ge r
            > d
    \end{equation}
    となり矛盾が従う。
    \begin{innerproof}
        各$A / \frakm_i$は$0$でない$\C$-代数だから
        $\C$上の次元が$1$以上の$\C$-ベクトル空間である。
        よって
        \begin{equation}
            \dim_C (
                A / \frakm_1 \oplus \cdots \oplus A / \frakm_r
            ) \ge r
        \end{equation}
        である。
    \end{innerproof}
    よって$A$の極大イデアルは$d$個以下である。
\end{answer}


\begin{problem}[代数学II 3.41]
    $A$を可換$\C$-代数で$d = \dim_\C A$としたとき$0 < d < \infty$であるとする。
    $A$が$0$でない冪零元を持たないならば、
    $A$は$d$個の複素数体の直積と同型になることを示せ。
\end{problem}

\begin{answer}
    \cref{problem:algebra2-3.40}より
    $A$の極大イデアルは$d$個以下である。
    そこで$A$の相異なる極大イデアルのすべてを
    \begin{equation}
        \frakm_1, \dots, \frakm_{r}
        \quad (1 \le r \le d)
    \end{equation}
    とおく。
    すると CRT より
    \begin{equation}
        \begin{tikzcd}
            A \ar{d} \ar{r}
                & A / \frakm_1 \times \cdots \times A / \frakm_r
                = A / \frakm_1 \oplus \cdots \oplus A / \frakm_r \\
            A / (\frakm_1 \cap \cdots \cap \frakm_r)
                \ar[dashed, end anchor=south west]{ru}[swap]{\cong}
        \end{tikzcd}
    \end{equation}
    を可換にする環の同型が誘導される。
    ここで各$A / \frakm_i$は
    極大イデアルによる商だから体であり、
    また$\C$上の次元は$d$以下である。
    よって Dixmier の定理より$A / \frakm_i \cong \C$が成り立ち、
    上の図式より
    \begin{equation}
        \begin{tikzcd}
            A \ar{d} \ar{r}
                & \C^r \\
            A / (\frakm_1 \cap \cdots \cap \frakm_r)
                \ar[end anchor=south west]{ru}[swap]{\cong}
        \end{tikzcd}
    \end{equation}
    が成り立つ。
    さて、$A$は$0$でない冪零元を持たないとする。
    いま$A$は可換アルティン環だから、
    \TODO{アルティンとは仮定されてない}
    $A$の素イデアルは極大イデアルでもある。
    よって$\frakm_1 \cap \cdots \cap \frakm_r$は
    $A$の素イデアル全部の交わり、
    すなわち$A$の冪零根基である。
    いま$A$は$0$でない冪零元を持たないから
    $\frakm_1 \cap \cdots \cap \frakm_r = 0$である。
    よって図式から
    $A \cong \C^r$が従う。
    $\C$上の次元を比較して$r = d$でなければならず、
    $A \cong \C^d$がいえた。
    \TODO{初等的に示せるか?}
\end{answer}


\begin{definition}[derivation]
    $K$を体、$A$を$K$-代数とする。
    $K$-線型写像$D \colon A \to A$が
    $A$上の\term{derivation}{derivation}であるとは、
    任意の$a, b \in A$に対して
    \begin{equation}
        D(ab) = D(a) b + a D(b)
    \end{equation}
    が成り立つことをいう。
\end{definition}

\begin{problem}[代数学II 3.43]
    複素数体上の$n$変数多項式環$\C[X_1, \dots, X_n]$上の
    derivation をすべて求めよ。
\end{problem}

\begin{answer}
    \TODO{monomial について議論すればよい}
\end{answer}


\begin{definition}[$G$-gradation]
    $G$を加法群、$K$を体、$A$を$K$-代数とする。
    $A$の\term{$G$-gradation}{$G$-gradation}[G-gradation]とは、
    $G$を添字集合とする$A$の$K$-部分ベクトル空間の族による
    $A$の$K$-ベクトル空間としての直和分解
    \begin{equation}
        A = \bigoplus_{g \in G} A(g)
    \end{equation}
    であって、任意の$x, y \in G, \; a \in A(x), \; b \in A(y)$に対して
    $ab \in A(x + y)$をみたすものをいう。
\end{definition}

\begin{problem}[代数学II 3.44]
    $K$を体、$A$を$K$-代数とする。
    $K$を加法群とみなしたときの$A$の$K$-gradation
    \begin{equation}
        A = \bigoplus_{z \in K} A(z)
    \end{equation}
    が与えられていたとする。
    このとき$K$-線型写像$D \colon A \to A$を
    $z \in K, \; a \in A(z)$に対して$D(a) = za$となるように定める。
    このとき$D$は derivation になることを示せ。
\end{problem}

\begin{answer}
    $a, b \in A$とする。
    $A$は所与の$K$-gradation のような直和分解を持つから、
    或る$x, y \in K$が存在して
    \begin{equation}
        a \in A(x), \quad b \in A(y)
    \end{equation}
    が成り立つ。
    このとき、$K$-gradation の定義より
    $ab \in A(x + y)$だから
    $D(ab) = (x + y) ab$である。
    よって
    \begin{alignat}{1}
        D(a) b + a D(b)
            &= (xa) b + a (yb) \\
            &= xab + yab \\
            &= (x + y) ab \\
            &= D(ab)
    \end{alignat}
    が成り立つ。
    したがって$D$は$A$上の derivation である。
\end{answer}


\begin{problem}[代数学II 3.46]
    単純環の中心は体になることを示せ。
\end{problem}

\begin{remark}
    単純環の部分環が単純であるとは限らないことに注意せよ (\cref{example:simple-ring})。
\end{remark}

\begin{answer}
    $A$を単純環とし、$Z(A)$が体であることを示す。
    中心の定義から$Z(A)$が可換環であることは明らかだから、
    あとは$Z(A)$の$0$でない任意の元が$Z(A)$に逆元を持つことを示せばよい。
    $x \in Z(A), \; x \neq 0$とすると、
    $Ax$は$A$の左イデアルであり、$(0)$を真に含む。
    いま$A$は単純環ゆえに$(0)$は$A$の極大イデアルなので、
    $A = Ax$である。
    よって、或る$x' \in A$が存在して
    \begin{alignat}{1}
        1 &= x' x \quad (\because A = Ax) \\
            &= x x' \quad (\because x \in Z(A))
    \end{alignat}
    が成り立つ。このとき
    \begin{equation}
        x' y = x' y x x' = x' x y x' = y x'
        \quad (\forall y \in A)
    \end{equation}
    が成り立つから$x' \in Z(A)$である。
    よって、$x$は$Z(A)$の可逆元である。
    したがって$Z(A)$は体である。
\end{answer}

\begin{problem}[代数学II 3.47]
    有限体の元の個数はある素数の冪になることを示せ。
\end{problem}

\begin{proof}
    cf. \cref{thm:cardinality-of-finite-field}
\end{proof}

\begin{problem}[代数学II 3.49]
    \label[problem]{problem:algebra2-3.49}
    Weyl 代数$\C[x : \del]$は$\C$-ベクトル空間としての基底
    $\{ x^i \del^j \mid i, j \in \Z_{\ge 0}$
    をもつことを示せ。
\end{problem}

\begin{proof}
    \TODO{}
\end{proof}

\begin{problem}[代数学II 3.50]
    \label[problem]{problem:algebra2-3.50}
    Weyl 代数$\C[x : \del]$は単純環であることを示せ。
\end{problem}

\begin{proof}
    \TODO{}
\end{proof}

\subsection{Problem set 5}

\begin{problem}[代数学II 5.64]
    \label[problem]{problem:algebra2-5.64}
    $R$を可換環、$n$を正整数、$\frakm$を$R$の極大イデアルとする。
    このとき$R / \frakm^n$は局所環であることを示せ。
\end{problem}

\begin{answer}
    イデアルの対応原理より、
    $\frakm / \frakm^n$は
    $R / \frakm^n$の極大イデアルである。
    これが唯一の極大イデアルであることを示せばよい。
    そこで$J / \frakm^n$を$R / \frakm^n$の極大イデアルとする。
    すると$J$は$R$の極大イデアルであって$\frakm^n$を含む。
    ここで$J$は$R$の素イデアルでもあるから、
    $J$は$\frakm$を含む。
    よって、$\frakm$が$R$の極大イデアルであることより
    $J = \frakm$である。
    したがって$J / \frakm^n = \frakm / \frakm^n$である。
    これで唯一性がいえた。
\end{answer}


\begin{problem}[代数学II 5.65]
    \label[problem]{problem:algebra2-5.65}
    $R$を可換環、$I$を$R$の固有イデアルとする。
    このとき$I = \sqrt{I}$となることは
    $R / I$の$0$でない冪零元が存在しないための必要十分条件であることを示せ。
\end{problem}

\begin{answer}
    $R / I$の$0$でない冪零元$a + I \; (a \in R - I)$が存在したとすると、
    ある$n \in \Z_{\ge 1}$が存在して
    $a^n + I = (a + I)^n = 0 + I$より$a^n \in I$、
    したがって$a \in \sqrt{I}$が成り立つ。
    よって$I \neq \sqrt{I}$である。
    逆に$I \neq \sqrt{I}$とすると$I \subsetneq \sqrt{I}$であるから、
    ある$b \in \sqrt{I} - I$がとれる。
    このとき
    \begin{itemize}
        \item $b \in \sqrt{I}$よりある$n \in \Z_{\ge 1}$が存在して
            $(b + I)^n = b^n + I = 0 + I$
        \item $b \not\in I$より$b + I \neq 0 + I$
    \end{itemize}
    したがって$b + I$は$R / I$の$0$でない冪零元である。
\end{answer}


\begin{problem}[代数学II 5.66]
    \label[problem]{problem:algebra2-5.66}
    $R$を可換環、$I$を$R$の固有イデアルとする。
    このとき$I$が準素イデアルであることは
    $R / I$の零因子がすべて冪零元になるための
    必要十分条件であることを示せ。
\end{problem}

\begin{answer}
    $I$を準素イデアルであるとする。
    $r + I \in R / I$を零因子とすると
    \begin{equation}
        (s + I)(r + I) = 0 + I \quad (\exists s \in R - I)
    \end{equation}
    が成り立つ。よって
    \begin{equation}
        sr \in I, \quad s \not\in I
    \end{equation}
    だから、$I$が準素イデアルであることより
    $r \in \sqrt{I}$である。
    よって$r + I$は$R / I$の冪零元である。
    逆に、$R / I$の零因子はすべて冪零元であるとする。
    $x, y \in R$について、$xy \in I$かつ$x \neq I$であるとする。
    $y \in I$の場合は$y \in I \subset \sqrt{I}$である。
    $y \not\in I$の場合は
    \begin{align}
        (x + I)(y + I) = xy + I = 0 + I \\
        x + I \neq 0 + I, \quad y + I \neq 0 + I
    \end{align}
    より$y + I$は零因子、したがって冪零元だから
    $y \in \sqrt{I}$が成り立つ。
    よって$I$は準素イデアルである。
\end{answer}

\begin{problem}[代数学II 5.67]
    $R$を可換環、$\frakp$を$R$の素イデアル、
    $n$を正整数とする。
    このとき$\frakp^n$は準素イデアルであることを示せ。
    また、任意の準素イデアルはこのような形に書けるか?
    正しければ証明を、誤りならば反例を与えよ。
\end{problem}

\begin{answer}
    \TODO{問題の前半は誤植。反例がある。\url{https://math.stackexchange.com/questions/93478/is-each-power-of-a-prime-ideal-a-primary-ideal}}

    $\C[X, Y]$のイデアル$I = (X, Y^2)$を考える。
    準素イデアルの特徴付け (\cref{prop:primary-ideal-characterization})
    を用いて$I$が$\C[X, Y]$の準素イデアルであることを示す。
    ここで$\C[X, Y] / I \cong \C[Y] / (Y^2)$である。
    $f + (Y^2), \; g + (Y^2) \neq 0 + (Y^2), \;
        fg + (Y^2) = 0 + (Y^2)$とすると、
    $f, g$は$Y$の倍元でなければならないから
    $f^2 + (Y^2) = 0 + (Y^2), \; g^2 + (Y^2) = 0 + (Y^2)$である。
    したがって$\C[X, Y] / I \cong \C[Y] / (Y^2)$の
    零因子はすべて冪零元である。
    特徴付けより$I$は準素イデアルである。

    一方、$I$は素イデアルの冪でないことを示す。
    $I$は準素イデアルだから、
    \cref{prop:primary-ideal-radical}より
    $\sqrt{I}$は$I$を含む最小の素イデアルである。
    ここで$I$が素イデアルの冪$I = \frakp^n$の形に表せたとする。
    すると$I = \frakp^n \subset \frakp$だから
    $\sqrt{I}$の最小性より
    $\frakp^n = I \subset \sqrt{I} \subset \frakp$
    が成り立つ。
    さらに$\sqrt{I}$は素イデアルだから
    $\frakp^n \subset \sqrt{I}$より
    $\frakp \subset \sqrt{I}$が成り立つ。
    よって$\frakp \subset \sqrt{I} \subset \frakp$、
    したがって$\sqrt{I} = \frakp$が従う。
    $\sqrt{I}$を具体的に求める。
    同型$\C[X, Y] / I \cong \C[Y] / (Y^2)$を再び用いる。
    $\C[Y]$がPIDゆえに$\C[Y]$の素イデアルは極大イデアルでもあることに注意すると、
    \cref{problem:algebra2-5.64}より
    $\C[Y] / (Y^2)$の素イデアルは$(Y) / (Y^2)$ただひとつであるから、
    同型で写して$\C[X, Y] / I$の素イデアルは
    $(X, Y) / I$ただひとつである。
    したがって$\C[X, Y]$の$I$を含む素イデアルは
    $(X, Y)$ただひとつであり、$\frakp = \sqrt{I} = (X, Y)$が従う。
    すると
    $\frakp^2 = (X, Y)^2 \subsetneq I = (X, Y^2) \subsetneq (X, Y) = \frakp$
    が成り立つ
    (左の不等号は$X \notin (X, Y)^2$より従い、
    右の不等号は$Y \notin (X, Y^2)$より従う)。
    これは$I = \frakp^n$に矛盾。
    したがって$I$は素イデアルの冪でない。
\end{answer}

\subsection{Problem set 6}

\begin{problem}[代数学II 6.78]
    \label[problem]{problem:algebra2-6.78}
    $R$を可換環、$I$を$R$の準素イデアルとすると
    $\sqrt{I}$は$I$を含む最小の素イデアルであることを示せ。
\end{problem}

\begin{answer}
    $\sqrt{I}$が素イデアルであることを示す。
    $ab \in \sqrt{I}, \; (a, b \in R), \;
        a \notin \sqrt{I}$
    とし、$b \in \sqrt{I}$を示す。
    $ab \in \sqrt{I}$より、ある$n \in \Z_{\ge 1}$が存在して
    $a^n b^n \in I$が成り立つ。
    このとき$a \notin \sqrt{I}$より$a^n \notin I$だから、
    $I$が準素イデアルであることより
    ある$m \in \Z_{\ge 1}$が存在して$b^{nm} \in I$が成り立つ。
    したがって$b \in \sqrt{I}$である。
    よって$\sqrt{I}$は素イデアルである。
    \TODO{冪零根基の特徴付けを使うべき?}

    $\frakp$を$I$を含む素イデアルとし、
    $\sqrt{I} \subset \frakp$を示す。
    $x \in \sqrt{I}$とすると、
    ある$n \in \Z_{\ge 1}$が存在して$x^n \in I$が成り立つ。
    このとき$x^n \in I \subset \frakp$だから、
    $\frakp$が素イデアルであることより
    $x \in \frakp$となる。
    したがって$\sqrt{I} \subset \frakp$である。
\end{answer}

\subsection{Problem set 12}

\begin{problem}[代数学II 12.151]
    $\C[X, Y, Z] / (XYZ - 1)$はPIDでないことを示せ。
\end{problem}

\begin{answer}
    $R \coloneqq \C[X, Y]$とおき、
    $R$の部分集合$S$を
    $S \coloneqq \{ X^k Y^k \mid k \in \Z_{\ge 0} \}$で定めると
    $S$は$R$の積閉集合であり、
    環の同型$\C[X, Y, Z] / (XYZ - 1) \cong S^{-1}R$が成り立つ。
    \begin{innerproof}
        \TODO{}
    \end{innerproof}
    そこで$S^{-1}R$がPIDでないことを示せばよい。
    $S^{-1}R$のイデアル$\left(\frac{X - 1}{1}, \frac{Y - 1}{1}\right)$を$J$とおく。
    $J$が単項イデアルでないことを示す。
    背理法のため、ある$f \in R$が存在して
    $\left(\frac{f}{1}\right) = J$
    と表せたとする。
    このとき$J = S^{-1}R$である。
    \begin{innerproof}
        $f/1 \in J$より
        ある$g, h \in R$および$m, n \in \Z_{\ge 0}$が存在して
        \begin{equation}
            \begin{cases}
                \frac{X - 1}{1} &= \frac{g}{X^m Y^m} \frac{f}{1} \vspace{1em} \\
                \frac{Y - 1}{1} &= \frac{h}{X^n Y^n} \frac{f}{1}
            \end{cases}
        \end{equation}
        が成り立つ。必要ならば$g, h$をそれぞれ
        $g X^n Y^n, h X^m Y^m$に置き換えることで
        $m = n$であるとしてよい。
        $R$は整域かつ$S \not\ni 0$だから通分して
        \begin{equation}
            \label[equation]{eq:12.151.1}
            \begin{cases}
                X^m Y^m (X - 1) &= fg \\
                X^m Y^m (Y - 1) &= fh
            \end{cases}
        \end{equation}
        が成り立つ。
        よって$X^m Y^m (X - 1) h = fgh = X^m Y^m (Y - 1) g$であり、
        $X^m Y^m$を払って
        $(X - 1) h = (Y - 1) g$が成り立つ。
        $X - 1, Y - 1$は互いに素だから、
        \TODO{説明不足}
        $R$が体上の多項式環ゆえにUFDであることより
        ある$g' \in R$が存在して
        $g = g' (X - 1)$と表せる。
        したがって\cref{eq:12.151.1}の第1式より
        \begin{equation}
            X^m Y^m (X - 1) = fg' (X - 1)
            \quad
            \therefore
            \quad
            X^m Y^m = fg'
        \end{equation}
        を得る。
        よって
        $\frac{1}{1}
            = \frac{g'}{X^m Y^m} \frac{f}{1}
            \in \left(\frac{f}{1}\right) = J$
        だから$J = S^{-1}R$がいえた。
    \end{innerproof}
    よって$(X - 1, Y - 1) \cap S \neq \emptyset$である。
    \begin{innerproof}
        $I \coloneqq (X - 1, Y - 1)$とおき、
        標準射$R \to S^{-1}R, \; r \mapsto r / 1$を$\varphi$とおく。
        明らかに$J = \left(\frac{X - 1}{1}, \frac{Y - 1}{1}\right)
            = S^{-1} \varphi(I)$
        であることに注意する。
        いま$J = S^{-1}R$だから$\varphi^{-1}(J) = \varphi^{-1}(S^{-1}R) = R$である。
        よって$1 \in \varphi^{-1}(J) = \varphi^{-1}(S^{-1} \varphi(I))$だから、
        ある$i \in I, \; s \in S$が存在して
        $\frac{1}{1} = \frac{i}{s}$と表せる。
        通分して$s = i$だから$I \cap S \neq \emptyset$がいえた。
    \end{innerproof}
    よってある$m \in \Z_{\ge 0}$が存在して
    $X^m Y^m \in (X - 1, Y - 1)$、
    したがって
    \begin{equation}
        X^m Y^m = u (X - 1) + v (Y - 1)
            \quad
            (\exists u, v \in R)
    \end{equation}
    となるが、両辺に$X = 1, Y = 1$を代入すると
    $1 = 0$となり矛盾が従う。
    背理法より$J$は単項イデアルでない。
    したがって$S^{-1}R$、ひいては$\C[X, Y, Z] / (XYZ - 1)$がPIDでないことが示せた。
\end{answer}


\end{document}
\documentclass[report]{jlreq}
\usepackage{global}
\usepackage{./local}
\subfiletrue
%\makeindex
\begin{document}

% ============================================================
%
% ============================================================
\chapter{加群}

群が集合の置換群としての表現を持つように、
環はアーベル群の自己準同型環としての表現を持っている。
したがってそのような表現を通して
環の性質を調べることができそうである。
加群とは、このような表現によって環が作用しているアーベル群のことである。
また別の見方では、加群とは加法とスカラー倍を備えた代数系のことであり、
アーベル群やベクトル空間の一般化である。

% ------------------------------------------------------------
%
% ------------------------------------------------------------
\section{加群}

\begin{definition}[加群]
    $A$を環とする。
    集合$M$が\term{左$A$-加群}[left $A$-module]{加群}[かぐん]
    あるいは\emph{$A$上の左加群}であるとは、
    次が成り立つことをいう:
    \begin{description}
        \item[(M1)] $V$はアーベル群である。
        \item[(M2)]
            \term{スカラー倍}[scalar multiplication]{スカラー倍}[すからーばい]と呼ばれる
            写像$R \times V \to V, \; (r, v) \mapsto rv$が定義されている。
        \item[(M3)] $a, b \in R, \; x \in M$に対し
            \begin{equation}
                1x = x, \quad (ab) x = a (bx)
            \end{equation}
            が成り立つ。
        \item[(M4)] $a, b \in R, \; x, y \in M$に対し
            \begin{alignat}{1}
                (a + b) x &= ax + bx \\
                a (x + y) &= ax + ay
            \end{alignat}
            が成り立つ。
    \end{description}
    \emph{右$A$-加群 (right $A$-module)}も同様に定義される。
    本稿では左$A$-加群を単に\emph{$A$-加群 ($A$-module)}や加群と呼ぶことにする。
\end{definition}

\begin{definition}[両側加群]
    $A, B$を環とする。
    集合$M$が
    \term{$(A, B)$-両側加群}[$(A, B)$-bimodule]{両側加群}[りょうがわかぐん]
    であるとは、
    次が成り立つことをいう:
    \begin{description}
        \item[(BM1)] $M$は左$A$-加群かつ右$B$-加群である。
        \item[(BM2)] $a \in A, \; b \in B, \; x \in M$に対し
            \begin{equation}
                (ax)b = a(xb)
            \end{equation}
            が成り立つ。
    \end{description}
\end{definition}

\begin{example}[加群の例]
    \idxsym{regular modules}{$\down{A}A, A_A$}{左/右正則加群}
    $A$を環とする。
    \begin{itemize}
        \item 任意のアーベル群は$\Z$-加群である。
        \item $A$の左イデアル、とくに$A$自身は左からの積で$A$-加群となる。
            このように環$A$自身を左$A$-加群とみなしたものを
            $\down{A}A$と書き、
            \term{左正則加群}[left regular module]
                {左正則加群}[ひだりせいそくかぐん]と呼ぶ。
            同様に環$A$自身を
            右からの積で右$A$-加群とみなしたものを
            $A_A$と書き、
            \term{右正則加群}[left regular module]
                {右正則加群}[みぎせいそくかぐん]と呼ぶ。
        \item 自明群$0$は任意の環上の加群である。
    \end{itemize}
\end{example}

既存の加群の係数を制限することで
新たな加群を構成することができる。
これを\term{係数の制限}[restriction of scalars]{係数制限}[けいすうせいげん]
といい、詳しくは
\cref{section:restriction-and-extension-of-scalars}
で調べる。

\begin{example}[係数の制限]
    \label[example]{example:restriction-of-scalars}
    $A, B$を環、
    $M$を$B$-加群、
    $\phi \colon A \to B$を環準同型とする。
    このとき
    $ax \coloneqq \phi(a)x$でスカラー倍を定めることで
    $M$に$A$-加群の構造が入る。
    \begin{itemize}
        \item $A \subset B$が部分環なら、
            標準包含により$B$は$A$-加群となる。
        \item $R$が可換環なら、
            標準包含により$R[X_1, \dots, X_n]$は$A$-加群である。
        \item $R$を可換環とし、$H \subset G$を部分群とすると、
            \cref{prop:group-ring-homomorphism}より
            $R[H]$は$R[G]$の部分環である。
            よって$R[G]$は$R[H]$-加群である。
        \item $V \coloneqq \F_2^3$は$\F_2$上のベクトル空間(とくに加群)である。
            $\Z$から$\F_2$への自然な環準同型により
            $V$は$\Z$-加群となる。
            また、$\Z/4\Z$から$\F_2$への自然な環準同型により
            $V$は$\Z/4\Z$-加群にもなる。
        \item $\C[x, y]/(x, y^2)$は$\C$-加群だから、
            evaluation homomorphism $\C[x, y] \to \C$により
            $\C[x, y]$-加群にもなる。
    \end{itemize}
\end{example}

\begin{definition}[加群の準同型]
    $A$を環、
    $V_1, V_2$を$A$-加群とする。
    写像$\varphi \colon V_1 \to V_2$が
    \term{$A$-加群準同型}[$A$-module homomorphism]
        {加群準同型}[かぐんじゅんどうけい]
    であるとは、
    \begin{enumerate}
        \item $\varphi$は群準同型
        \item $\varphi(av) = a\varphi(v) \quad (a \in R, v \in V_1)$
    \end{enumerate}
    が成り立つことをいう。
\end{definition}

\begin{example}[加群準同型の例]
    \label[example]{ex:module-homomorphism}
    $A \coloneqq \C[x, y], I \coloneqq (x, y) \subset A$とする。
    写像$\phi \colon A^2 \to I,$
    \begin{equation}
        [f_1, f_2] \mapsto f_1 x + f_2 y
    \end{equation}
    を考える。$\phi$は定義から明らかに全射であり、
    $I$を$A$-加群とみれば$\phi$は$A$-加群準同型である。
    $\Ker(\phi)$を求める。
    $[f_1, f_2] \in \Ker(\phi)$ならば$f_1 x + f_2 y = 0$である。
    よって$f_1 x = -f_2 y$となるが、$A$はUFDで$x, y$は互いに素だから、
    素元分解の一意性より$f_1 = yg_1, f_2 = xg_2 \; (g_1, g_2 \in A)$と表せる。
    よって$g_1 xy = - g_2 xy$であり、$A$は整域だから
    $g_1 = -g_2$、したがって$[f_1, f_2] = g_1 \cdot [y, -x]$である。
    よって$\Ker(\phi) \subset A \cdot [y, -x]$である。
    逆の包含も明らか。したがって$\Ker(\phi) = A \cdot [y, -x]$である。
\end{example}



% ------------------------------------------------------------
%
% ------------------------------------------------------------
\section{部分加群}

\begin{definition}[部分加群]
    $A$を環とし、$M$を$A$-加群とする。
    部分集合$N \subset M$が$M$の和とスカラー倍により加群となるとき、
    $N$を$M$の\term{部分$A$-加群}[$A$-submodule]{部分加群}[ぶぶんかぐん]という。
    $N$が$M$の部分加群であることは次と同値である:
    \begin{enumerate}
        \item $N$は$M$の加法部分群であり、
        \item $a \in A, n \in N \Rightarrow an \in N$が成り立つ。
    \end{enumerate}
\end{definition}

\begin{definition}[イデアル上の線型結合からなる部分加群]
    \idxsym{submodule with ideal coefficients}{$IM$}
        {左イデアル$I$上の線型結合全体からなる部分加群}
    $A$を環、$M$を$A$-加群、$I \subset A$を左イデアルとする。
    $M$の元の$I$-線型結合全体の集合を
    \begin{equation}
        IM \coloneqq \{
            a_1 v_1 + \cdots + a_n v_n \mid
            n \in \Z_{\ge 0}, \; a_i \in I, \; v_i \in M
        \}
    \end{equation}
    とおく。
    $IM$は$M$の部分$A$-加群となる。
\end{definition}

\begin{example}[部分加群の例]
    ~
    \begin{itemize}
        \item 正則左$R$-加群の部分$R$-加群は
            $R$の左イデアルに他ならない。
            右/双加群が右/両側イデアルに対応することも同様である。
    \end{itemize}
\end{example}

% ------------------------------------------------------------
%
% ------------------------------------------------------------
\section{生成された加群}

\TODO{$Rv$とかの記法は?それは環のイデアル?}

\begin{definition}[部分集合により生成された加群]
    \idxsym{submodule generated by a subset}
        {$\langle S \rangle, \; AS$}{$S$により生成された部分加群}
    \idxsym{submodule generated by an element}
        {$\langle v \rangle, \; Av$}{$v$により生成された部分加群}
    $A$を環、$M$を$A$-加群とする。
    \begin{itemize}
        \item 部分集合$S \subset M$に対し、
            $S$を含む$M$の最小の部分加群を
            $S$により\term{生成された部分加群}[generated submodule]
            {生成された部分加群}[せいせいされたかぐん]と呼び、
            $\langle S \rangle$や$AS$と書く。
        \item $S$が1元集合$S = \{ v \}$の場合、
            $\langle S \rangle$を
            $v$により$A$上生成された
            \term{巡回加群}[cyclic submodule]{巡回加群}[じゅんかいかぐん]と呼び、
            波括弧を省略して
            $\langle v \rangle$や$Av$とも書く。
    \end{itemize}
\end{definition}

有限集合で生成される加群は特に重要であるが、
ここでは定義と簡単な例を述べるにとどめ、
詳しく調べるのは\cref{chapter:finiteness}にまわす。

\begin{definition}[有限生成加群]
    $A$を環、$M$を$A$-加群とする。
    $M$が有限集合で生成されるとき、$M$は
    $A$-加群として\term{有限生成}[finitely generated]
    {有限生成!加群として---}[ゆうげんせいせい]であるという。
\end{definition}

\begin{example}[有限生成加群とそうでない加群の例]
    ~
    \begin{itemize}
        \item $A$を環とする。
            $n \in \Z_{\ge 1}$に対し$A^n$は$A$上の有限生成加群である。
            生成元は$(0, \dots, \overset{\stackrel{j}{\smile}}{1}, \dots, 0)
                \; (j = 1, \dots, n)$
            である。
        \item $A$を可換環とし、$B \coloneqq A[X]$とすると、
            $B$は$A$-加群として有限生成\highlight{ではない}。
            しかし$A$-代数としては$x$により生成されるから有限生成である。
        \item \cref{section:noetherian-modules-and-artinian-modules}
            で述べるネーター性は有限生成性を強化した性質である。
    \end{itemize}
\end{example}

$\Q$が$\Z$上有限生成でないことは
簡単な反例を作るときに役立つかもしれない。

\begin{lemma}
    \label[lemma]{lemma:Q-is-not-finitely-generated-over-Z}
    $\Q$は$\Z$上有限生成でない。
\end{lemma}

\begin{proof}
    $\Q$が有限個の元$q_1, \dots, q_n$により$\Z$上生成されたとする。
    $q_i = k_i / l_i, \;
        k_i \in \Z, \;
        l_i \in \Z_{\ge 1}, \;
        \gcd(k_i, l_i) = 1$と表す。
    $l_1, \dots, l_n$の素因数分解に現れない素数$p$をひとつ選ぶ。
    $1 / p = \sum_{i = 1}^n n_i k_i / l_i$と表せる。
    よって
    $l_1 \dots l_n = p \sum_{i = 1}^n n_i k_i \prod_{j \neq i} l_j$
    が成り立つ。
    $p$の選び方より
    左辺は$p$で割り切れないが、
    右辺は$p$で割り切れるから矛盾。
    よって$\Q$は$\Z$上有限生成でない。
\end{proof}

% ------------------------------------------------------------
%
% ------------------------------------------------------------
\section{商加群}

\begin{definition}[商加群]
    \TODO{}
\end{definition}


% ------------------------------------------------------------
%
% ------------------------------------------------------------
\section{準同型定理}

\begin{proposition}[加群準同型の像と核]
    $A$を環、
    $f \colon M \to N$を$A$-加群準同型とする。
    このとき$\Ker f, \; \Im f$はそれぞれ
    $M, N$の$A$-部分加群である。
\end{proposition}

\begin{proof}
    \TODO{}
\end{proof}

\begin{theorem}[準同型定理]
    \TODO{}
\end{theorem}

\begin{proof}
    \TODO{}
\end{proof}

\begin{example}[準同型定理の例]
    \cref{ex:module-homomorphism}の具体例を考える。
    $A = \C[x]$とし、$M = A^2$とする。
    $N = A \cdot [1, x]$とすると
    $N$は$M$の部分$A$-加群である。
    $\phi \colon M \to A, [a, b] \mapsto b - ax$は$A$-加群準同型である。
    $\phi([0, b]) = b$なので、$\phi$は全射である。
    さらに$\Ker(\phi) = A \cdot [1, x] = N$
    より$M/N \cong A$である。
    \begin{equation}
        \begin{tikzcd}
            & A \\
            M \ar{ru}{\phi} \ar[two heads]{r} & M/N \ar[dashed]{u}
        \end{tikzcd}
    \end{equation}
\end{example}

\begin{theorem}[第2同型定理, 菱形同型定理]
    $A$を環、
    $M$を$A$-加群、
    $H, K \subset M$を$A$-部分加群とする。
    このとき$(H + K) / K \cong H / (H \cap K)$が成り立つ。
    \begin{equation}
        \begin{tikzcd}
            & H + K \\
            H \ar[dashed, no head]{ru}
                && K \ar[dash]{lu} \\
            & H \cap K \ar[dash]{lu} \ar[dashed, no head]{ru}
        \end{tikzcd}
    \end{equation}
\end{theorem}

\begin{proof}
    準同型$H \to (H + K) / K, \; h \mapsto h + K$に
    準同型定理を用いればよい。
\end{proof}

\begin{theorem}[第3同型定理]
    $A$を環、
    $K \subset L \subset M$を$A$-部分加群の列とする。
    このとき
    \begin{equation}
        \frac{M / K}{L / K} \cong \frac{M}{L}
    \end{equation}
    が成り立つ。
\end{theorem}

\begin{proof}
    準同型$M / K \to M / L, \; m + K \to m + L$に
    準同型定理を用いればよい。
    cf. \cref{problem:algebra2-5.63}
\end{proof}

加群に対しても
環の両側イデアルの対応原理
(\cref{thm:ideal-correspondence-principle})
と類似の主張が成り立つ。
\TODO{束の同型?}
この定理は、商加群のイデアル全体の集合にある種の下界を与える。
したがって、たとえば\cref{chapter:finiteness}で述べる
イデアルの降鎖条件の確認に使うことができ、
アルティン環の例を考えるためにも役立つ。

\begin{theorem}[部分加群の対応原理]
    \label[theorem]{thm:supmodule-correspondence-principle}
    $A$を環、
    $M$を$A$-加群、
    $N \subset M$を$A$-部分加群とする。
    \begin{alignat}{1}
        \scrI_N(M) &\coloneqq \{
            K \colon \text{$K$は$N$を含む$M$の$A$-部分加群}
        \} \\
        \scrI(M / N) &\coloneqq \{
            K \colon \text{$K$は$M / N$の$A$-部分加群}
        \}
    \end{alignat}
    とおくと、
    \begin{equation}
        \wt{p} \colon \scrI_N(M) \to \scrI(M / N),
        \quad K \mapsto p(K)
    \end{equation}
    は包含関係を保つ全単射であり、
    $\wt{p}$の逆写像$q$は
    \begin{equation}
        q \colon \scrI(M / N) \to \scrI_N(M),
        \quad K' \mapsto p^{-1}(K')
    \end{equation}
    で与えられる。
\end{theorem}

\begin{proof}
    \TODO{}
%    $\wt{p}$の逆写像$q$が
%    \begin{equation}
%        q \colon \scrI(A / I) \to \scrI_I(A),
%        \quad J' \mapsto p^{-1}(J')
%    \end{equation}
%    で与えられることを示す。
%    $A / I$の両側イデアル$J'$に対し
%    $p^{-1}(J')$が$A$の両側イデアルであることは
%    \cref{thm:ring-hom-and-ideals}より成り立ち、
%    また$I = p^{-1}(0) \subset p^{-1}(J')$も成り立つ。
%    よって$q$は well-defined である。
%
%    $q$が$\wt{p}$の逆写像であることを示す。
%    $J' \in \scrI(A / I)$に対し
%    $p(p^{-1}(J')) = J'$であることは
%    $p$の全射性より従う。
%    $J \in \scrI_I(A)$に対し
%    $p^{-1}(p(J)) = J$であることを示す。
%    "$\supset$"は集合の一般論より成り立つ。
%    逆に$x \in p^{-1}(p(J))$とすると、
%    ある$j \in J$が存在して$p(x) = p(j)$となる。
%    よって$p(x - j) = 0$だから
%    $x - j \in p^{-1}(0) = I \subset J$である。
%    したがって$x = (x - j) + j \in J$が成り立つから
%    "$\subset$"もいえた。
%    よって$q$は$\wt{p}$の逆写像である。
%
%    $\wt{p}$が包含を保つことは
%    写像による部分集合の像と逆像が包含を保つことから従う。
%    以上で定理の主張が示された。
\end{proof}


% ------------------------------------------------------------
%
% ------------------------------------------------------------
\section{自己準同型環}

加群準同型全体の集合には次のように加群の構造が入る。

\begin{definition}[加群準同型全体の集合]
    \label[definition]{definition:set-of-module-homomorphisms}
    \idxsym{Module homomorphism}{$\Hom_A(V_1, V_2)$}
        {$A$-加群準同型$V_1 \to V_2$全体のなす集合}
    \begin{description}
        \item[($\Hom$の$\Z$-加群構造)]
            $A$を環、
            $V_1, V_2$を$A$-加群とする。
            集合
            \begin{equation}
                \Hom_A(V_1, V_2) \coloneqq \{
                    \varphi \colon V_1 \to V_2 \mid \varphi \text{ は$A$-加群準同型}
                \}
            \end{equation}
            に対し、
            加法、零元を
            \begin{alignat}{1}
                (\varphi + \psi)(v)
                    &\coloneqq \varphi(v) + \psi(v) \\
                (\varphi + 0)(v)
                    &= \varphi(v)
                    = (0 + \varphi)(v)
            \end{alignat}
            として$\Z$-加群の構造が入る。
        \item[(環上の加群)]
            $A, B$を環、
            $V_1$を$A$-加群、
            $V_2$を$B$-加群とする。
            $\Z$-加群$\Hom_{\Z}(V_1, V_2)$に対し、
            スカラー倍を
            \begin{equation}
                (a\varphi)(m) \coloneqq \varphi(am)
            \end{equation}
            として$A$-加群の構造が入る。
        \item[(環上の両側加群)]
            $A, B$を環、
            $V_1$を$(B, A)$-両側加群、
            $V_2$を$B$-加群とする。
            $B$-加群$\Hom_B(V_1, V_2)$に対し、
            スカラー倍を
            \begin{equation}
                (a\varphi)(m) \coloneqq \varphi(ma)
            \end{equation}
            として$A$-加群の構造が入る。
        \item[(代数上の加群)]
            $R$を可換環、
            $A$を$R$-代数とする。
            $\Z$-加群$\Hom_A(V_1, V_2)$に対し、
            スカラー倍を
            \begin{equation}
                (r \varphi)(v) \coloneqq r \varphi(v)
            \end{equation}
            として$R$-加群の構造が入る\footnote{
                $A$が可換環でないときは、
                $\Hom_A(V_1, V_2)$に$A$-加群の構造が入るとは限らない。
                実際、$a, b \in A, \; ab \neq ba$をとり
                $a\id \in \Hom_A(V_1, V_1)$を仮定すると、
                $x \in V_1 \setminus \{ 0 \}$
                に対し
                $bax = (a\id)(bx) = abx$
                より$ab = ba$となり矛盾する。
            }。
    \end{description}
\end{definition}

とくに自己準同型全体の加群には次のように環や代数の構造が入る。

\begin{definition}[自己準同型環]
    \idxsym{Module endomorphism}{$\End_A(V)$}
        {$A$-加群自己準同型$V \to V$全体のなす集合}
    \begin{description}
        \item[(環上の加群)]
            $A$を環、
            $V$を$A$-加群とする。
            $A$-加群$\Hom_A(V, V)$を$\End_A(V)$と書き、
            $\End_A(V)$に対し、
            乗法、単位元を
            \begin{alignat}{1}
                (\varphi \cdot \psi)(v)
                    &\coloneqq (\varphi \circ \psi)(v) \\
                (\id_V \cdot \varphi)(v)
                    &= \varphi(v)
                    = (\varphi \cdot \id_V)(v)
            \end{alignat}
            として環の構造が入る。
            $\End_A(V)$を$V$の
            \term{自己準同型環}[endmorphism ring]{自己準同型環}[じこじゅんどうけいかん]
            という。
        \item[(代数上の加群)]
            $R$を可換環、
            $A$を$R$-代数、
            $V$を$A$-加群とする。
            環$\End_A(V)$に対し、
            環準同型
            \begin{equation}
                R \to Z(\End_A(V)),
                \quad
                r \mapsto (v \mapsto rv)
            \end{equation}
            により$R$-代数の構造が入る。
    \end{description}
\end{definition}

加群の自己準同型をひとつ固定すると、
次の例のように係数環を多項式環上まで拡張できる。
この構成は\cref{chapter:linear-algebra}で重要となる。

\begin{example}[多項式環上の加群]
    $R$を可換環、$M$を$R$-加群、$\varphi \in \End_R(M)$とする。
    このとき、$M$は写像
    \begin{equation}
        R[X] \times M \to M,
        \quad
        (f, v) \mapsto f(\varphi)(v)
    \end{equation}
    をスカラー乗法として$R[X]$-加群となる。
    $M$が$R$上有限生成ならば、
    明らかに$R[X]$上でも有限生成である。
\end{example}



% ------------------------------------------------------------
%
% ------------------------------------------------------------
\section{直積と直和}

加群の直積と直和を定義する。
まず圏論的直積を考える。

\begin{definition}[圏論的直積]
    $A$を環、
    $S$を集合、
    $\{ V_i \}_{i \in S}$を$A$-加群の族とする。
    $A$-加群$W$と
    $A$-加群準同型の族$\{ q_i \colon W \to V_i \}_{i \in S}$の対
    $(W, \{ q_i \}_i)$が
    \term{圏論的直積}[categorical direct product]{圏論的直積}[けんろんてきちょくせき]
    であるとは、
    \begin{alignat}{1}
        &\forall \; \{ f_i \colon U \to V_i \}_{i \in S}
            \colon \text{ $A$-加群準同型の族} \\
        &\exists! \; F \colon U \to W
            \colon \text{ $A$-加群準同型}
            \quad \text{s.t.} \quad \\
        &\quad
            \begin{tikzcd}[ampersand replacement=\&]
                U \ar{rd}[swap]{f_i} \ar[dashed]{rr}{F} \&\& W \ar{ld}{q_i} \\
                \& V_i
            \end{tikzcd}
    \end{alignat}
    が成り立つことをいう\footnote{
        つまり、直積写像が一意に存在するということである。
    }。
\end{definition}

圏論的直積の具体的な構成を与えよう。

\begin{definition}[直積]
    \idxsym{direct product module}{$\prod_{i \in S} V_i$}{加群の直積}
    $A$を環、
    $S$を集合、
    $\{ V_i \}_{i \in S}$を$A$-加群の族とする。
    直積集合$\prod_{i \in S} V_i$に加法とスカラー倍を
    \begin{alignat}{1}
        (v_i)_i + (w_i)_i &\coloneqq (v_i + w_i)_i \\
        a \cdot (v_i)_i &\coloneqq (a \cdot v_i)_i
    \end{alignat}
    で定め、零元を$(0)_i$として$A$-加群の構造を入れたものを
    加群の\term{直積}[direct product]{直積}[ちょくせき]という。
    直積加群は標準射影の族
    \begin{equation}
        p_k \colon \prod_{i \in S} V_i \to V_k,
        \quad
        (v_i)_i \mapsto v_k
    \end{equation}
    とあわせて考える。
\end{definition}

\begin{proposition}[直積は圏論的直積]
    $A$を環、
    $S$を集合、
    $\{ V_i \}_{i \in S}$を$A$-加群の族とする。
    このとき、直積加群$\prod_{i \in S} V_i$とその標準射影の族$\{ p_i \}_i$
    の対は圏論的直積である。
\end{proposition}

\begin{proof}
    $F(u) \coloneqq (f_i(u))_i$と定めればよい。
\end{proof}

つぎに圏論的直和を考える。

\begin{definition}[圏論的直和]
    $A$を環、
    $S$を集合、
    $\{ V_i \}_{i \in S}$を$A$-加群の族とする。
    $A$-加群$W$と
    $A$-加群準同型の族$\{ \iota_i \colon V_i \to W \}_{i \in S}$の対
    $(W, \{ \iota_i \}_i)$が
    \term{圏論的直和}[categorical direct sum]{圏論的直和}[けんろんてきちょくわ]
    であるとは、
    \begin{alignat}{1}
        &\forall \; \{ f_i \colon V_i \to U \}_{i \in S}
            \colon \text{ $A$-加群準同型の族} \\
        &\exists! \; F \colon W \to U
            \colon \text{ $A$-加群準同型}
            \quad \text{s.t.} \quad \\
        &\quad
            \begin{tikzcd}[ampersand replacement=\&]
                W \ar[dashed]{rr}{F} \&\& U \\
                \& V_i \ar{lu}{\iota_i} \ar{ru}[swap]{f_i}
            \end{tikzcd}
    \end{alignat}
    が成り立つことをいう\footnote{
        つまり、直和写像が一意に存在するということである。
    }。
\end{definition}

圏論的直和の具体的な構成を与える。

\begin{definition}[外部直和]
    \idxsym{external direct sum module}{$\bigoplus_{i \in S} V_i$}{加群の外部直和}
    $A$を環、
    $S$を集合、
    $\{ V_i \}_{i \in S}$を$A$-加群の族とする。
    $\prod_{i \in S} V_i$の部分$A$-加群
    \begin{equation}
        \bigoplus_{i \in S} V_i
            \coloneqq \biggl\{
                (v_i)_i \in \prod_{i \in S} V_i
                \; \bigg| \;
                \text{有限個の$i \in S$を除いて$v_i = 0$}
            \biggr\}
    \end{equation}
    を加群の\term{外部直和}[external direct sum]{外部直和}[がいぶちょくわ]という。
    外部直和は標準射の族
    \begin{equation}
        \iota_k \colon V_k \to \bigoplus_{i \in S} V_i,
        \quad
        v \mapsto (v_i)_i
        \quad \text{with} \quad
        v_i = \begin{cases}
            0 & i \neq k \\
            v & i = k
        \end{cases}
    \end{equation}
    とあわせて考える。
\end{definition}

\begin{remark}
    $S$が有限集合ならば、
    定義から明らかに
    直積$\prod_{i \in S} V_i$と
    (外部)直和$\bigoplus_{i \in S} V_i$は一致する。
\end{remark}

\begin{definition}[内部直和]
    \TODO{}
\end{definition}

圏論的直和の特徴付けを与える。
この特徴付けは
\cref{section:additive-functors}
で加法的関手を調べる際に役立つ。

\begin{proposition}[直和の特徴付け]
    $A$を環、
    $\Lambda$を集合、
    $\{ M_\lambda \}_{\lambda \in \Lambda}$
    を$A$-加群の族、
    $\{ j_\lambda \colon M_\lambda \to M \}_{\lambda \in \Lambda}$
    を$A$-加群準同型の族とする。
    このとき、次は同値である:
    \begin{enumerate}
        \item $(M, \{ j_\lambda \}_{\lambda \in \Lambda})$は圏論的直和である。
        \item ある$A$-加群準同型の族
            $\{ q_\lambda \colon M \to M_\lambda \}_{\lambda \in \Lambda}$
            が存在して次をみたす:
            \begin{enumerate}
                \item 各$\lambda, \mu \in \Lambda$に対し
                    $q_\mu \circ j_\lambda = \delta_{\lambda \mu} \id_{M_\lambda}$
                    である。
                \item 各$x \in M$に対し、
                    有限個の$\lambda \in \Lambda$を除いて
                    $q_\lambda(x) = 0$
                    である。
                \item 各$x \in M$に対し
                    $\sum_{\lambda \in \Lambda} j_\lambda \circ q_\lambda(x) = x$
                    である。
            \end{enumerate}
    \end{enumerate}
\end{proposition}

\begin{proof}
    \uline{(1) \Rightarrow (2)} \quad
    \TODO{}

    \uline{(2) \Rightarrow (1)} \quad
    \begin{equation}
        f(x) \coloneqq
            \sum_{\lambda \in \Lambda}
            f_\lambda \circ q_\lambda(x)
    \end{equation}
    \TODO{}
\end{proof}

有限直和の場合は明らかに$(2b)$の条件は不要である。

\begin{corollary}[有限直和の特徴付け]
    $A$を環、
    $M_1, \dots, M_n \; (n \in \Z_{\ge 2})$
    を$A$-加群、
    $j_k \colon M_k \to M \; (k = 1, \dots, n)$
    を$A$-加群準同型とする。
    このとき、次は同値である:
    \begin{enumerate}
        \item $(M, (j_1, \dots, j_n))$は圏論的直和である。
        \item ある$A$-加群準同型
            $q_k \colon M \to M_k \; (k = 1, \dots, n)$
            が存在して次をみたす:
            \begin{enumerate}
                \item 各$1 \le k, l \le n$に対し
                    $q_l \circ j_k = \delta_{kl} \id_{M_k}$
                    である。
                \item $\sum_{k = 1}^n j_k \circ q_k = \id_M$
                    である。
            \end{enumerate}
    \end{enumerate}
    \qed
\end{corollary}

2つの直和の場合はさらに簡単になり、
$q_l \circ j_k = 0 \; (l \neq k)$の条件を除くことができる。

\begin{corollary}
    $A$を環、
    $M_1, M_2$を$A$-加群、
    $j_k \colon M_k \to M \; (k = 1, 2)$
    を$A$-加群準同型とする。
    このとき、次は同値である:
    \begin{enumerate}
        \item $(M, (j_1, j_2))$は圏論的直和である。
        \item ある$A$-加群準同型
            $q_k \colon M \to M_k \; (k = 1, 2)$
            が存在して次をみたす:
            \begin{enumerate}
                \item $q_1 \circ j_1 = \id_{M_1}, \;
                    q_2 \circ j_2 = \id_{M_2}$
                    である。
                \item $j_1 \circ q_1 + j_2 \circ q_2 = \id_M$
                    である。
            \end{enumerate}
    \end{enumerate}
\end{corollary}

\begin{remark}
    3個以上の直和の場合は
    $q_l \circ j_k = 0 \; (l \neq k)$
    の条件を除くことはできない。
    実際、$M = M_1 = M_2 = M_3 = \Z/2\Z$として
    $j_k \colon M_k \to M, \; x \mapsto x$の場合を考えると、
    $q_k \colon M \to M_k, \; x \mapsto x$は
    $q_k \circ j_k = \id_{M_k}, \;
        j_1 \circ q_1 + j_2 \circ q_2 + j_3 \circ q_3 = \id_M$
    をみたすが、明らかに$M \not\cong M_1 \oplus M_2 \oplus M_3$である。
\end{remark}

\begin{proof}
    条件(2)から$q_1 \circ j_2 = 0, \; q_2 \circ j_1 = 0$が従うことをいえばよい。
    \begin{alignat}{1}
        q_1
            &= q_1 \circ (j_1 \circ q_1 + j_2 \circ q_2)
                \quad (\text{条件(2b)}) \\
            &= q_1 \circ j_1 \circ q_1 + q_1 \circ j_2 \circ q_2 \\
            &= q_1 + q_1 \circ j_2 \circ q_2
                \quad (\text{条件(2a)})
    \end{alignat}
    より$q_1 \circ j_2 \circ q_2 = 0 = 0 \circ q_2$であるが、
    いま$q_2 \circ j_2$が恒等写像ゆえに$q_2$は全射だから
    両辺の$q_2$を打ち消して
    $q_1 \circ j_2 = 0$を得る。
    同様に$q_2 \circ j_1 = 0$も得られる。
\end{proof}



% ------------------------------------------------------------
%
% ------------------------------------------------------------
\section{完全系列}

直和の概念は
系列の分裂という概念につながる。

\begin{definition}[完全系列]
    $A$を環とする。
    $\lMod{A}$の系列
    \begin{equation}
        \begin{tikzcd}
            M_1
                \ar{r}{f_1}
                & M_2
                    \ar{r}{f_2}
                & \cdots
                    \ar{r}{f_{n-1}}
                & M_n
        \end{tikzcd}
    \end{equation}
    が$\Im f_i = \Ker f_{i + 1} \; (i = 1, \dots, n - 2)$をみたすとき、
    この系列は\term{完全}[exact]{完全}[かんぜん]であるという。

    また、
    \begin{equation}
        \begin{tikzcd}
            0
                \ar{r}
                & L
                    \ar{r}{\phi}
                & M
                    \ar{r}{\psi}
                & N
                    \ar{r}
                & 0
        \end{tikzcd}
    \end{equation}
    の形の完全系列を
    \term{短完全系列}[short exact sequence]{短完全系列}[たんかんぜんけいれつ]という。
\end{definition}

\begin{example}[短完全系列の例]
    $M$を加群、$N \subset M$を部分加群とする。
    $\pi \colon M \to M/N$を自然な準同型とすると、列
    \begin{equation}
        \begin{tikzcd}
            0 \ar{r} & N \ar[hook]{r} & M \ar{r}{\pi} & M/N \ar{r} & 0
        \end{tikzcd}
    \end{equation}
    は短完全系列である。
\end{example}

\begin{definition}[分裂]
    $A$を環とする。
    $A$-加群の短完全系列
    \begin{equation}
        \begin{tikzcd}
            0 \ar{r}
                & L \ar{r}{i}
                & M \ar{r}{p}
                & N \ar{r}
                & 0
        \end{tikzcd}
    \end{equation}
    が\term{分裂}[split]{分裂}[ぶんれつ]するとは、
    次の同値な条件のどれかひとつ (よって全て) が成り立つことをいう:
    \begin{enumerate}
        \item $p \circ g = \id_N$なる$A$-加群準同型$g \colon N \to M$が存在する。
            $g$を\term{right splitting}{right splitting}という。
        \item $f \circ i = \id_L$なる$A$-加群準同型$f \colon M \to L$が存在する。
            $f$を\term{left splitting}{left splitting}という。
        \item $M$の$A$-部分加群$M'$が存在して
            $M = \Im i \oplus M'$が成り立つ。
    \end{enumerate}
\end{definition}

\begin{remark}
    left splitting の定義域は
    $\Im i$でなく$M$全体であることに注意。
    同様に right splitting の定義域は
    $\Im p$でなく$N$全体であることに注意。
\end{remark}

\begin{proof}
    \TODO{}
\end{proof}

\begin{example}[分裂しない短完全系列の例]
    $\pi \colon \Z \to \Z/2\Z$を自然な準同型とすると、列
    \begin{equation}
        \begin{tikzcd}
            0 \ar{r}
                & \Z \ar{r}{2 \times}
                & \Z \ar{r}{\pi}
                & \Z/2\Z \ar{r}
                & 0
        \end{tikzcd}
    \end{equation}
    は短完全系列である。
    この短完全系列は分裂しない。
\end{example}

\begin{proposition}[分裂すれば直和で書ける]
    $A$を環とする。
    $A$-加群の短完全系列
    \begin{equation}
        \begin{tikzcd}
            0 \ar{r}
                & L \ar{r}{i}
                & M \ar{r}{p}
                & N \ar{r}
                & 0
        \end{tikzcd}
    \end{equation}
    が分裂するならば
    外部直和との同型$M \cong L \oplus N$が成り立つ。
    より詳しく、
    この短完全系列の right splitting $j$に対し
    内部直和$M = \Im i \oplus \Im j$が成り立つ。
\end{proposition}

\begin{proof}
    $j$を right splitting として
    $M = \Im i \oplus \Im j$を示す。
    $m \in M$とすると
    $m - jp(m) \in \Ker p = \Im i$だから
    $m = m - jp(m) + jp(m) \in \Im i + \Im j$である。
    つぎに$\Im i \cap \Im j = 0$を示す。
    $x \in \Im i \cap \Im j$が$0$であることを示せばよいが、
    これは次の図式の diagram chasing により明らか:
    \begin{equation}
        \begin{tikzcd}
            &&& C \ar{ld}[swap]{j} \\
            0 \ar{r}
                & A \ar{r}[swap]{i}
                & B \ar{r}[swap]{p}
                & C \ar[equal]{u} \ar{r}
                & 0
        \end{tikzcd}
    \end{equation}
    よって$M = \Im i \oplus \Im j \cong L \oplus N$がいえた。
    ただし、$i, j$が単射ゆえに
    $L \cong \Im i, \; N \cong \Im j$であることを用いた。
\end{proof}

\begin{example}[直和で書けても分裂するとは限らない]
    上の命題の逆は一般には成り立たない。
    \TODO{}
\end{example}




% ------------------------------------------------------------
%
% ------------------------------------------------------------
\section{自由加群}

加群に対しても、ベクトル空間の場合と同様に基底の概念が定義できる。

\begin{definition}[線型独立]
    $A$を環、$M$を$A$-加群とする。
    $B \subset M$が$A$上
    \term{線型独立}[linearly independent]{線型独立}[せんけいどくりつ]
    であるとは、
    $B$の任意の元$v_1, \dots, v_k \in B \; (k \in \Z_{\ge 1})$と
    $A$の任意の元$a_1, \dots, a_k \in A$に対し
    「$a_1 v_1 + \cdots + a_k v_k = 0 \implies a_1 = \cdots = a_k = 0$」
    が成り立つことをいう。
\end{definition}

\begin{definition}[自由加群]
    $A$を環、$M$を$A$-加群とする。
    \begin{itemize}
        \item 部分集合$B \subset M$が線型独立かつ$\langle B \rangle = M$をみたすとき、
            $B$を$M$の\term{基底}[basis]{基底}[きてい]という。
        \item $M$が$M = \{0\}$であるかまたは基底を持つとき、
            $M$は\term{自由}[free]{自由}[じゆう]であるという。
    \end{itemize}
\end{definition}

\begin{example}[自由加群の例]
    ~
    \begin{itemize}
        \item $\Z$-加群$\Z/m\Z$は基底をもたない (よって自由加群でない)。
            実際、任意の有限部分集合$\{ a_1, \dots, a_k \} \subset \Z/m\Z$に対し
            $m a_1 + \dots + m a_k = 0$である。
    \end{itemize}
\end{example}

自由加群は係数環の直和で書けるという特徴付けを持つ。

\begin{proposition}[自由加群の特徴付け]
    $A$を環、
    $V$を$A$-加群とする。
    $V$が自由加群であることと、
    ある集合$S$が存在して$A$-加群の同型
    $V \cong A^{\oplus S}$が成り立つこととは同値である。
\end{proposition}

\begin{proof}
    \TODO{}
\end{proof}

\begin{proposition}[自由加群の有限直和]
    $A$を環とする。
    自由$A$-加群の有限個の直和も自由$A$-加群である。
\end{proposition}

\begin{proof}
    \TODO{}
\end{proof}

\begin{theorem}[自由加群の普遍性]
    $A$を環とし、$V$を自由$A$-加群、$B \subset V$を基底とする。
    このとき次が成り立つ:
    \begin{alignat}{1}
        &\forall \; W
            \colon \text{ $A$-加群} \\
        &\forall \; \varphi \colon B \to W 
            \colon \text{ 写像} \\
        &\exists! \; \wt{\varphi} \colon V \to W \colon \text{ $A$-加群準同型}
            \quad \text{s.t.} \quad \\
        &\quad \begin{tikzcd}[ampersand replacement=\&]
            B \ar[hook]{d} \ar{r}{\varphi} \& W \\
            V \ar[dashed]{ru}[swap]{\wt{\varphi}}
        \end{tikzcd}
    \end{alignat}
\end{theorem}

\begin{proof}
    \begin{equation}
        \wt{\varphi} \biggl(
            \fsum_{b \in B} a_b b
        \biggr)
            \coloneqq \fsum_{b \in B} a_b \varphi(b)
    \end{equation}
    と定めればよい。
\end{proof}

\begin{theorem}[体上の加群の性質]
    $K$を体とする。
    \begin{enumerate}
        \item 任意の$K$-加群は自由加群である。
        \item 集合$S_1, S_2$に関し
            \begin{equation}
                \down{K} K^{\oplus S_1} \cong \down{K} K^{\oplus S_2}
                \quad \iff \quad
                \sharp S_1 = \sharp S_2
            \end{equation}
            が成り立つ。
    \end{enumerate}
\end{theorem}

\begin{proof}
    \TODO{}
\end{proof}

\begin{theorem}[可換環上の加群の性質]
    $R$を可換環とする。
    集合$S_1, S_2$に関し
    \begin{equation}
        \down{R} R^{\oplus S_1} \cong \down{R} R^{\oplus S_2}
        \quad \iff \quad
        \sharp S_1 = \sharp S_2
    \end{equation}
    が成り立つ。
\end{theorem}

\begin{proof}
    \TODO{}
\end{proof}

% ------------------------------------------------------------
%
% ------------------------------------------------------------
\section{可換環上の自由加群}

ベクトル空間における次元と類似の概念として、
可換環上の自由加群のランクが定義できる。
とくに体$K$上の自由加群とは$K$-ベクトル空間に他ならない。
さらにこのとき$M$の$K$上のランクとは
$K$-ベクトル空間としての次元$\dim_K M$に他ならない。

\begin{definition}[自由加群のランク]
    \TODO{ねじれがある場合は?}
    \idxsym{rank of free module}{$\rk$}{自由加群のランク}
    $R$を可換環、
    $M$を$R$上の自由加群とする。
    このとき、$M$の基底はすべて同じ濃度を持ち、
    $M$の任意の生成系の濃度は基底の濃度以上である(このあと示す)。
    そこで、$M$の$K$上の\term{ランク}[rank]{ランク} $\rk(M)$を
    \begin{itemize}
        \item $M \neq \{0\}$なら基底の濃度
        \item $M = \{0\}$なら$0$
    \end{itemize}
    と定める。
    ランクが有限の自由加群は
    \term{有限ランク自由加群}[free module of finite rank]
    {有限ランク自由加群}[ゆうげんらんくじゆうかぐん]
    あるいは形容詞で
    \term{free of finite rank}{free of finite rank}
    であるという。
\end{definition}

\begin{proof}
    \TODO{}
\end{proof}

\begin{corollary}
    $R$を可換環、
    $A$を$R$-代数とする。
    $A$が$R$-加群として有限ランクの自由加群であるとき、
    集合$S_1, S_2$に関し
    \begin{equation}
        \down{R} R^{\oplus S_1} \cong \down{R} R^{\oplus S_2}
        \quad \implies \quad
        \sharp S_1 = \sharp S_2
    \end{equation}
    が成り立つ。
\end{corollary}

\begin{proof}
    \TODO{}
\end{proof}

有限ランク自由加群と有限生成加群の間には次の関係がある。

\begin{theorem}[有限ランク自由加群と有限生成加群の関係]
    $R$を可換環とする。
    このとき、$R$-加群$M$に関し次は同値である:
    \begin{enumerate}
        \item ある$n \in \Z_{\ge 0}$が存在して全射$R$-加群準同型$R^n \to M$が存在する。
            すなわち、$M$はある有限ランク自由加群の商加群である。
        \item $M$は$R$-加群として有限生成である。
    \end{enumerate}
\end{theorem}

\begin{proof}
    \uline{(1) \Rightarrow (2)} \quad
    題意の全射を$f$とおくと、$M$は明らかに$R$上
    \begin{equation}
        \{ f(1, 0, \dots, 0), f(0, 1, \dots, 0), \dots, f(0, 0, \dots, 1) \}
    \end{equation}
    により生成される。

    \uline{(2) \Rightarrow (1)} \quad
    $M$を生成する有限部分集合$S = \{ s_1, \dots, s_n \} \subset M$をひとつ選べば、
    写像$(r_1, \dots, r_n) \mapsto r_1 s_1 + \cdots + r_n s_n$が
    求める全射となる。
\end{proof}



% ------------------------------------------------------------
%
% ------------------------------------------------------------
\section{帰納極限と射影極限}

\begin{definition}[帰納極限]
    ~
    \begin{itemize}
        \item 半順序集合$(I, \le)$が
            \term{有向的}[directed]{有向的}[ゆうこうてき]であるとは、
            任意の$x, y \in I$に対して
            ある$z \in I$が存在して
            $x \le z$かつ$y \le z$が成り立つことをいう。
        \item $A$を環とし、$(I, \le)$を有向的半順序集合とする。
            $A$-加群の
            \term{帰納系}[inductive system]{帰納系}[きのうけい]あるいは
            \term{有向系}[direct system]{有向系}[ゆうこうけい]とは、
            組$(\{ M_i \}_{i \in I}, \{ \varphi_{ij} \}_{i \le j})$であって
            次をみたすものをいう:
            \begin{enumerate}
                \item $M_i \; (i \in I)$は$A$-加群である。
                \item $i \le j$なる$i, j \in I$に対して
                    $\varphi_{ij} \colon M_i \to M_j$は
                    $A$-加群の準同型である。
                \item $i \le j \le k$に対し
                    $\varphi_{jk} \circ \varphi_{ij} = \varphi_{ik}$が成り立つ。
                    \begin{equation}
                        \begin{tikzcd}
                            M_i \ar{r}{\varphi_{ij}}
                                \ar[bend right=60]{rr}[swap]{\varphi_{ik}}
                                & M_j \ar{r}{\varphi_{jk}}
                                & M_k
                        \end{tikzcd}
                    \end{equation}
                \item $\varphi_{ii} = \id_{M_i} \; (i \in I)$である。
            \end{enumerate}
        \item $A$を環、$(I, \le)$を有向的半順序集合とし、
            $(\{ M_i \}_{i \in I}, \{ \varphi_{ij} \}_{i \le j})$を
            $A$-加群の有向系とする。
            組$(L, \{ \phi_i \}_{i \in I})$が
            $(\{ M_i \}_{i \in I}, \{ \varphi_{ij} \}_{i \le j})$の
            \term{帰納極限}[inductive limit]{帰納極限}[きのうきょくげん]
            であるとは、次が成り立つことをいう:
            \begin{enumerate}
                \item $L$は$A$-加群である。
                \item $\phi_i \colon M_i \to L \; (i \in I)$は$A$-加群準同型である。
                \item $i \le j$なる$\forall i, j \in I$に対して
                    $\phi_j \circ \varphi_{ij} = \phi_i$が成り立つ。
                    \begin{equation}
                        \begin{tikzcd}
                            M_i \ar{rr}{\varphi_{ij}}
                                \ar{rd}[swap]{\phi_i}
                                && M_j \ar{ld}{\phi_j} \\
                                & L
                        \end{tikzcd}
                    \end{equation}
                \item (帰納極限の普遍性) 次の条件をみたす:
                    \begin{alignat}{1}
                        &\forall \; N
                            \colon \text{ $A$-加群} \\
                        &\forall \; \{ \xi_i \colon M_i \to N \}_{i \in I}
                            \colon \text{ $A$-加群準同型の族} \\
                        &\qquad \text{with} \quad
                            \text{
                                $i \le j$なる$i, j \in I$に対し
                                $\xi_j \circ \varphi_{ij} = \xi_i$
                            } \\
                        &\exists! \; \eta \colon L \to N
                            \colon \text{ $A$-加群準同型}
                            \quad \text{s.t.} \quad \\
                        &\forall \; i \in I
                            \quad \text{に対し} \quad \\
                        &\qquad \begin{tikzcd}[ampersand replacement=\&]
                            \& M_i
                                \ar{ld}[swap]{\phi_i}
                                \ar{rd}{\xi_i} \\
                            L \ar[dashed]{rr}[swap]{\eta}
                                \& \& N
                        \end{tikzcd}
                    \end{alignat}
            \end{enumerate}
        \item 上の定義で「$A$-加群」の部分を
            「$R$-代数」に置き換えることで、
            $R$-代数の有向系およびその帰納極限も同様に定義される。
    \end{itemize}
\end{definition}

\begin{remark}
    帰納極限の具体的な構成は
    \cref{problem:algebra2-7-95}を参照せよ。
\end{remark}

\begin{definition}[射影極限]
    ~
    \begin{itemize}
        \item $A$を環とし、$(I, \le)$を有向的半順序集合とする。
            $A$-加群の
            \term{射影系}[projective system]{射影系}[しゃえいけい]あるいは
            \term{逆向系}[inverse system]{逆向系}[ぎゃっこうけい]とは、
            組$(\{ M_i \}_{i \in I}, \{ \varphi_{ij} \}_{i \ge j})$であって
            次をみたすものをいう:
            \begin{enumerate}
                \item $M_i \; (i \in I)$は$A$-加群である。
                \item $i \ge j$なる$i, j \in I$に対して
                    $\varphi_{ij} \colon M_i \to M_j$は
                    $A$-加群の準同型である。
                \item $i \ge j \ge k$に対し
                    $\varphi_{jk} \circ \varphi_{ij} = \varphi_{ik}$が成り立つ。
                    \begin{equation}
                        \begin{tikzcd}
                            M_i \ar{r}{\varphi_{ij}}
                                \ar[bend right=60]{rr}[swap]{\varphi_{ik}}
                                & M_j \ar{r}{\varphi_{jk}}
                                & M_k
                        \end{tikzcd}
                    \end{equation}
                \item $\varphi_{ii} = \id_{M_i} \; (i \in I)$である。
            \end{enumerate}
        \item $A$を環、$(I, \le)$を有向的半順序集合とし、
            $(\{ M_i \}_{i \in I}, \{ \varphi_{ij} \}_{i \ge j})$を
            $A$-加群の射影系とする。
            組$(L, \{ \phi_i \}_{i \in I})$が
            $(\{ M_i \}_{i \in I}, \{ \varphi_{ij} \}_{i \ge j})$の
            \term{射影極限}[projective limit]{射影極限}[しゃえいきょくげん]
            であるとは、次が成り立つことをいう:
            \begin{enumerate}
                \item $L$は$A$-加群である。
                \item $\phi_i \colon L \to M_i \; (i \in I)$は$A$-加群準同型である。
                \item $i \ge j$なる$\forall i, j \in I$に対して
                    $\phi_j = \varphi_{ij} \circ \phi_i$が成り立つ。
                    \begin{equation}
                        \begin{tikzcd}
                            & L
                                \ar{ld}[swap]{\phi_i}
                                \ar{rd}{\phi_j} \\
                            M_i \ar{rr}[swap]{\varphi_{ij}}
                                & & M_j
                        \end{tikzcd}
                    \end{equation}
                \item (射影極限の普遍性) 次の条件をみたす:
                    \begin{alignat}{1}
                        &\forall \; N
                            \colon \text{ $A$-加群} \\
                        &\forall \; \{ \xi_i \colon N \to M_i \}_{i \in I}
                            \colon \text{ $A$-加群準同型の族} \\
                        &\qquad \text{with} \quad
                            \text{
                                $i \ge j$なる$i, j \in I$に対し
                                $\xi_j = \varphi_{ij} \circ \xi_i$
                            } \\
                        &\exists! \; \eta \colon N \to L
                            \colon \text{ $A$-加群準同型}
                            \quad \text{s.t.} \quad \\
                        &\forall \; i \in I
                            \quad \text{に対し} \quad \\
                        &\qquad \begin{tikzcd}[ampersand replacement=\&]
                            N \ar[dashed]{rr}{\eta}
                                \ar{rd}[swap]{\xi_i}
                                \& \& L
                                \ar{ld}{\phi_i} \\
                                \& M_i
                        \end{tikzcd}
                    \end{alignat}
            \end{enumerate}
        \item 上の定義で「$A$-加群」の部分を
            「$R$-代数」に置き換えることで、
            $R$-代数の射影系およびその射影極限も同様に定義される。
    \end{itemize} 
\end{definition}






% ------------------------------------------------------------
%
% ------------------------------------------------------------
\newpage
\section{演習問題}

\subsection{Problem set 4}

\begin{problem}[代数学II 4.51]
    \label[problem]{problem:algebra2-4.51}
    $A$を環としたとき$\End_A(\down{A}A) \cong A^\OP$を示せ。
\end{problem}

\begin{answer}
    写像$\Phi \colon \End_A(\down{A}A) \to A^\OP$を
    \begin{equation}
        \Phi(f) = f(1)
    \end{equation}
    で定める。
    $\Phi$は環準同型である。
    \begin{innerproof}
        $\Phi(0) = 0, \; \Phi(f + g) = \Phi(f) + \Phi(g)$は明らか。
        積を保つことは、$A^\OP$の積を$*$と書けば
        \begin{alignat}{1}
            \Phi(f \circ g)
                &= f \circ g (1) \\
                &= f(g(1)) \\
                &= f(g(1) \cdot 1) \\
                &= g(1) f(1) \\
                &= f(1) * g(1) \\
                &= \Phi(f) * \Phi(g)
        \end{alignat}
        より成り立つ。
    \end{innerproof}
    逆写像$\Psi \colon A^\OP \to \End_A(\down{A}A)$は
    \begin{equation}
        \Psi(a) = (x \mapsto x \cdot a)
    \end{equation}
    で定まる。
    \begin{innerproof}
        逆写像であることは
        \begin{alignat}{1}
            \Phi \circ \Psi(a)
                &= \Phi(x \mapsto x \cdot a) \\
                &= a
        \end{alignat}
        および
        \begin{alignat}{1}
            \Psi \circ \Phi(f)
                &= (x \mapsto x \cdot f(1)) \\
                &= (x \mapsto f(x)) \\
                &= f
        \end{alignat}
        よりわかる。
    \end{innerproof}
    よって$\Phi$は環の同型$\End_A(\down{A}A) \cong A^\OP$を与える。
\end{answer}

\begin{problem}[代数学II 4.56]
    $\down{A}A \cong \down{A}A \oplus \down{A}A$なる環$A \neq 0$の例を挙げよ。
\end{problem}

\begin{answer}
    \TODO{c.f. \url{https://mathlog.info/articles/619}}
\end{answer}

\begin{problem}[代数学II 4.57]
    $K$を体、$A$を$K$-代数、$V$を既約$A$-加群であって
    $\dim_K(V) < \infty$なるものとする。
    このとき、$\dim_K(V) < n$ならば
    $V^{\oplus n}$は巡回$A$-加群とならないことを示せ。
\end{problem}

\begin{answer}
    \TODO{$V$が既約であることはいつ使う?}
    $d \coloneqq \dim_K V \in \Z_{\ge 1}$とおく。
    $V$の$K$-ベクトル空間としての基底$e_1, \dots, e_d$をひとつ選ぶ。
    $n \in \Z_{\ge 0}$とする。
    $n = 0$の場合は示したい含意の前提が偽なので成り立つ。
    以下、$n \ge 1$の場合を考える。
    対偶を示すため、$V^{\oplus n}$は巡回$A$-加群であると仮定する。
    仮定より、$V^{\oplus n}$の$A$-加群としての生成元
    $v_1, \dots, v_n$が存在する。
    このとき$v_1, \dots, v_n$は$V$において$K$上1次独立である。
    \begin{innerproof}
        $\mu_1, \dots, \mu_n \in K$に対し線型関係
        \begin{equation}
            \mu_1 v_1 + \dots + \mu_n v_n = 0
        \end{equation}
        を仮定する。
        いま$V^{\oplus n}$は$A$上$v_1, \dots, v_n$で生成されるのであったから、
        とくにある$a_i \in A \; (i = 1, \dots, n)$が存在して
        \begin{equation}
            a_i \cdot (v_1, \dots, v_n) \quad
                \bigl(= (a_i \cdot v_1, \dots, a_i \cdot v_n) \bigr) \quad
                = (0, \dots, \overset{\stackrel{i}{\smile}}{e_1}, \dots, 0)
        \end{equation}
        が成り立つ。
        そこで、上の線型関係の式の両辺に左から$a_i$を掛けて
        \begin{alignat}{2}
            && a_i \cdot (\mu_1 v_1 + \dots + \mu_n v_n) &= 0 \\
            &\therefore& \mu_i e_1 &= 0 \\
            &\therefore& \mu_i &= 0
        \end{alignat}
        をすべての$i = 1, \dots, n$に対し得る。
        よって$v_1, \dots, v_n$は$K$上1次独立である。
    \end{innerproof}
    したがって$\dim_K(V) \ge n$であり、対偶が示せた。
\end{answer}

\begin{problem}[代数学II 4.58]
    $A$を環、$V, W$を互いに同型でない既約$A$-加群とする。
    このとき$V \oplus W$は巡回$A$-加群であることを示せ。
\end{problem}

\begin{remark}
    「互いに同型でない」という条件は必須である。
    実際、$V = W = \Z / 2\Z$は既約$\Z$-加群であるが、
    $V \oplus W = \Z / 2\Z \oplus \Z / 2\Z$は巡回$\Z$-加群でない。
\end{remark}

\begin{answer}
    中国剰余定理の証明と同様の流れで示す。
    $V, W$は既約だから、$A$のある極大左イデアル$I, J$が存在して
    $V \cong \down{A}A / I, \; W \cong \down{A}A / J$が成り立つ。
    ここで準同型定理より図式
    \begin{equation}
        \begin{tikzcd}
            \down{A}A \ar{d} \ar{r}{p = p_1 \times p_2}
                & \down{A}A / I \times \down{A}A / J
                = \down{A}A / I \oplus \down{A}A / J
                = V \oplus W \\
            \down{A}A / (I \cap J)
                \ar[dashed, end anchor=south west]{ur}[swap]{\wb{p}}
        \end{tikzcd}
    \end{equation}
    の破線部に$A$-加群準同型$\wb{p}$が誘導される。
    $\down{A}A$が巡回$A$-加群ゆえに
    $\down{A}A / (I \cap J)$も巡回$A$-加群なので、
    $V \oplus W$が巡回$A$-加群であることを示すには
    $\wb{p}$が同型であることを示せば十分である。
    そのためには全単射をいえばよい。
    $\wb{p}$が単射であることは
    $\Ker \wb{p} = I \cap J$より明らか。
    $\wb{p}$が全射であることを示す。
    そのためには
    \begin{align}
        e_1 &\coloneqq (1, 0) \in \down{A}A / I \times A / J \\
        e_2 &\coloneqq (0, 1) \in \down{A}A / I \times A / J
    \end{align}
    とおき、$e_1, e_2 \in \Im \wb{p}$をいえばよい。
    いま$V, W$は同型でないから$I \neq J$、
    したがって$I, J$が極大であることとあわせて$I + J = A$である。
    よって$x + y = 1$なる$x \in I, \; y \in J$が存在する。
    このとき
    \begin{alignat}{1}
        p_1(x) &= 0 \\
        p_2(x) &= p_2(1 - y) = p_2(1) = 1
    \end{alignat}
    よって
    \begin{equation}
        p(x) = (p_1(x), p_2(x)) = (0, 1) = e_2 \in \Im \wb{p}
    \end{equation}
    が成り立つ。
    同様に$e_1 \in \Im \wb{p}$も成り立つ。
    よって$\wb{p}$は同型である。
    したがって$V \oplus W$は巡回$A$-加群である。
\end{answer}

\begin{problem}[代数学II 4.59]
    $D$を division algebra, $D^n$を$n$次元縦ベクトルの空間とし
    これを$D^n$の左からの積で$D$-加群とみなす。
    このとき$\End_D(D^n) \cong M_n(D)^{\OP}$を示せ。
\end{problem}

\begin{answer}
    \TODO{division algebra であることはいつ使う?}

    $(D^n)^n$は$D^n$の元である縦ベクトルを横に$n$個並べたもの全部の空間とする。
    $(D^n)^n$には
    左$\End_D(D^n)$-加群の構造
    \begin{equation}
        \varphi . [v_1, \dots, v_n]
            \coloneqq [\varphi(v_1), \dots, \varphi(v_n)]
            \quad
            (\varphi \in \End_D(D^n), \; v_i \in D^n)
    \end{equation}
    が入り (ただし「$.$」で$\End_D(D^n)$の作用を表す)、
    また右からの積で右$M_n(D)$-加群の構造が入る。
    さらに各$\varphi \in \End_D(D^n), \; v_i \in D^n, \; A = (a_{ij}) \in M_n(D)$に対し
    \begin{alignat}{1}
        (\varphi . [v_1, \dots, v_n]) A
            &= [\varphi(v_1), \dots, \varphi(v_n)] A \\
            &= \left[
                \sum_{j = 1}^n a_{j1} \varphi(v_j), \dots,
                \sum_{j = 1}^n a_{jn} \varphi(v_j)
            \right] \\
            &= \varphi . \left[
                \sum_{j = 1}^n a_{j1} v_j, \dots,
                \sum_{j = 1}^n a_{jn} v_j
            \right] \\
            &= \varphi . ([v_1, \dots, v_n] A)
    \end{alignat}
    が成り立つから、
    $(D^n)^n$は$(\End_D(D^n), M_n(D))$-両側加群である。
    ここで$D^n$の標準基底を$e_1, \dots, e_n$とおき、
    \begin{equation}
        E \coloneqq [e_1, \dots, e_n] \in (D^n)^n
    \end{equation}
    とおくと、$e_1, \dots, e_n$が$D^n$の基底であることから
    各$f \in \End_D(D^n)$に対し
    $f . E = E \up{t}A$なる$A \in M_n(D)$がただひとつ定まる。
    逆に各$A \in M_n(D)$に対し、自由加群の普遍性より
    $f . E = E \up{t}A$なる$f \in \End_D(D^n)$がただひとつ定まる。
    したがって、このように$f$を$A$に写す写像を
    $\Phi \colon \End_D(D^n) \to M_n(D)^{\OP}$とおけば$\Phi$は全単射である。
    あとは$\Phi$が$D$-代数の準同型であることを示せばよい。
    各$f, g \in \End_D(D^n), \; a \in D$に対し
    \begin{alignat}{1}
        (f + g) . E
            &= f.E + g.E \\
            &= E \up{t}\Phi(f) + E \up{t}\Phi(g) \\
        (f \circ g) . E
            &= f . (g . E) \\
            &= f . (E \up{t}\Phi(g)) \\
            &= (f . E) \up{t}\Phi(g) \\
            &= E \up{t}\Phi(f) \up{t}\Phi(g) \\
            &= E \up{t}(\Phi(g) \Phi(f)) \\
        (\id) . E
            &= E = E \up{t} I_n \\
        (af) . E
            &= a . (f . E) \\
            &= a . (E \up{t}\Phi(f)) \\
            &= aE \up{t}\Phi(f) \\
            &= E \up{t} (a \Phi(f))
    \end{alignat}
    が成り立つことから$\Phi$は$D$-代数の準同型、
    したがって同型である。
    よって$\End_D(D^n) \cong M_n(D)^{\OP}$である。
\end{answer}

\subsection{Problem set 5}

\begin{problem}[代数学II 5.63]
    \label[problem]{problem:algebra2-5.63}
    $A$を環、$M$を$A$-加群とし、
    $N$を$M$の部分$A$-加群、
    $L$を$N$の部分$A$-加群とする。
    このとき$N / L$は$M / L$の部分$A$-加群とみなせて、
    \begin{equation}
        M / N \cong (M / L) / (N / L)
    \end{equation}
    が成り立つことを示せ。
\end{problem}

\begin{answer}
    \TODO{長過ぎる?}
    いま図式
    \begin{equation}
        \begin{tikzcd}
            0 \ar{r} & N \ar{r} & M \ar{r} & M / N \ar{r} & 0
        \end{tikzcd}
    \end{equation}
    は exact である。
    また、
    $\begin{tikzcd}
        N \ar{d} \ar{r} & M \ar{d} \\
        N / L \ar[dashed]{r} & M / L
    \end{tikzcd}$
    について、
    $n \in N$が
    $\begin{tikzcd}
        N \ar{d} \\
        N / L
    \end{tikzcd}$
    で$0$に写ることは$n \in L$と同値で、
    さらにこれは
    $n$が
    $\begin{tikzcd}
        N \ar{r} & M \ar{d} \\
        & M / L
    \end{tikzcd}$
    で$0$に写ることと同値である。
    よって単射
    $\begin{tikzcd}
        N / L \ar{r} & M / L
    \end{tikzcd}$
    が誘導される。
    これにより$N / L$を$M / L$の部分$A$-加群とみなすことができる。
    したがって上下の行が exact な可換図式
    \begin{equation}
        \begin{tikzcd}
            0 \ar{r}
                & N \ar{d} \ar{r}
                & M \ar{d} \ar{r}
                & M / N \ar{r}
                & 0 \\
            0 \ar{r}
                & N / L \ar{r}
                & M / L \ar{r}
                & (M / L) / (N / L) \ar{r}
                & 0
        \end{tikzcd}
    \end{equation}
    を得る。
    このとき
    $\begin{tikzcd}
        M \ar{d} \ar{r} & M / N \ar[dashed]{d} \\
        M / L \ar{r} & (M / L) / (N / L)
    \end{tikzcd}$
    を可換にする射
    $\begin{tikzcd}
        M / N \ar{d} \\
        (M / L) / (N / L)
    \end{tikzcd}$
    が誘導される。なぜならば、
    $m \in M$が
    $\begin{tikzcd}
        M \ar{r} & M / N
    \end{tikzcd}$
    で$0$に写るとすれば、上の行の exact 性により
    $m$はある$n \in N$の
    $\begin{tikzcd}
        N \ar{r} & M
    \end{tikzcd}$
    による像となり、
    $m$の
    $\begin{tikzcd}
        M \ar{d} \\
        M / L \ar{r} & (M / L) / (N / L)
    \end{tikzcd}$
    による像は
    $n$の
    \begin{equation}
        \begin{tikzcd}
            N \ar{r} & M \ar{d} \\
            & M / L \ar{r} & (M / L) / (N / L)
        \end{tikzcd}
        =
        \begin{tikzcd}
            N \ar{d} \\
            N / L \ar{r} & M / L \ar{r} & (M / L) / (N / L)
        \end{tikzcd}
    \end{equation}
    による像$0$に一致するからである。
    ただし、像が$0$であることは
    下の行の exact 性による。
    さらに
    $\begin{tikzcd}
        M \ar{d} \\
        M / L \ar{r} & (M / L) / (N / L)
    \end{tikzcd}$
    の全射性より
    $\begin{tikzcd}
        M / N \ar{d} \\
        (M / L) / (N / L)
    \end{tikzcd}$
    は全射である。
    あとは
    $\begin{tikzcd}
        M / N \ar{d} \\
        (M / L) / (N / L)
    \end{tikzcd}$
    の単射性を示せばよい。
    $m + N \in M / N$が
    $\begin{tikzcd}
        M / N \ar{d} \\
        (M / L) / (N / L)
    \end{tikzcd}$
    で$0$に写るとする。
    そこで$m + N$の
    $\begin{tikzcd}
        M \ar{r} & M / N
    \end{tikzcd}$
    による逆像のひとつ$m'$を選ぶと、
    $m'$は
    $\begin{tikzcd}
        M \ar{d} \\
        M / L \ar{r} & (M / L) / (N / L)
    \end{tikzcd}$
    により$0$に写り、したがって
    $m' + L$は
    $\begin{tikzcd}
        M / L \ar{r} & (M / L) / (N / L)
    \end{tikzcd}$
    により$0$に写る。
    下の行の exact 性により、
    $m' + L$は
    $\begin{tikzcd}
        N / L \ar{r} & M / L
    \end{tikzcd}$
    の像に含まれる。
    よって、$m'$はある$n \in N$と$l \in L$により$m' = n + l$と表せる。
    したがって
    \begin{equation}
        m + N = m' + N = (n + l) + N = 0
    \end{equation}
    となり、
    $\begin{tikzcd}
        M / N \ar{d} \\
        (M / L) / (N / L)
    \end{tikzcd}$
    の単射性が示せた。
\end{answer}







% ============================================================
%
% ============================================================
\chapter{既約加群}

既約加群について述べる。

% ------------------------------------------------------------
%
% ------------------------------------------------------------
\section{既約加群}

加群の既約の概念を定義する。

\begin{definition}[既約加群]
    \label[definition]{def:irreducible-module}
    $A$を環とする。
    $A$-加群$M$が
    \term{単純}[simple]{単純}[たんじゅん]あるいは
    \term{既約}[irreducible]{既約}[きやく]
    であるとは、
    $M \neq 0$であって
    $0$と$M$以外の部分加群を持たないことをいう
    \footnote{
        表現論では「既約」、環論では「単純」ということが多いらしい。
        本稿では主に前者を用いる。
    }
    。
\end{definition}

既約加群は次のように特徴付けられる。

\begin{theorem}[既約加群の特徴付け]
    $A$を環、$U$を$A$-加群とする。
    このとき次は同値である:
    \begin{enumerate}
        \item $U$は既約である。
        \item 任意の$v \in U, \; v \neq 0$は$U$を$A$上生成する。
    \end{enumerate}
\end{theorem}

\begin{proof}
    \uline{(1) \Rightarrow (2)} \quad
    $U = 0$のときは明らかだから、$U \neq 0$のときを考える。
    $v \in U - \{ 0 \}$とする。
    $\langle v \rangle \neq 0$だから、
    $U$が既約であることより$\langle v \rangle = U$である。
    よって$U$は$A$上$v$で生成される巡回加群である。

    \uline{(2) \Rightarrow (1)} \quad
    \TODO{}
\end{proof}

既約加群は次の意味で完全列を分裂させる。

\begin{theorem}
    $A$を環、$U$を既約$A$-加群とする。
    このとき$\lMod{A}$の任意の完全列
    \begin{equation}
        \begin{tikzcd}
            0
                \ar{r}
                & X
                    \ar{r}{f}
                & U
                    \ar{r}{g}
                & Y
                    \ar{r}
                & 0
        \end{tikzcd}
    \end{equation}
    は分裂する。
\end{theorem}

\begin{proof}
    $U$が既約であることより
    $\Ker g$は$0$または$U$である。
    $\Ker g = 0$の場合、
    $g$は単射だから完全列より全単射となる。
    したがって$g$は right splitting となり
    所与の完全列は分裂する。
    $\Ker g = U$の場合、
    $\Im f = \Ker g = U$より
    $f$は全射だから完全列より全単射となる。
    したがって$f$は left splitting となり
    所与の完全列は分裂する。
\end{proof}

\begin{corollary}
    \label[corollary]{corollary:homomorphic-image-of-irreducible-module}
    既約加群の準同型像は$0$または既約である。
\end{corollary}

\begin{proof}
    $A$を環、
    $Y$を$A$-加群、
    $U$を既約$A$-加群、
    $f \colon U \to Y$を$A$-加群準同型とする。
    $\Im f \neq 0$と仮定して$\Im f$が既約であることを示せばよい。
    ここで、定理より完全列
    \begin{equation}
        \begin{tikzcd}
            0
                \ar{r}
                & \Ker f
                    \ar[hook]{r}
                & U
                    \ar{r}{f}
                & \Im f
                    \ar{r}
                & 0
        \end{tikzcd}
    \end{equation}
    は分裂するから$U \cong \Ker f \oplus \Im f$である。
    いま$\Im f \neq 0$だから、
    $U$が既約であることより$\Ker f = 0$でなければならない。
    したがって$U \cong \Im f$となり
    $\Im f$は既約となる。
\end{proof}



% ------------------------------------------------------------
%
% ------------------------------------------------------------
\section{ねじれと annihilator}

Annihilator の概念を定義する。
Annihilator は群論でいう元の位数 (order) や群の冪数 (exponent) の
加群論における一般化である。

\begin{definition}[ねじれ]
    \idxsym{tortion subset}{$M_\tor$}{$M$のねじれ元全体と$0$からなる部分集合}
    $A$を環、
    $M$を$A$-加群とする。
    \begin{itemize}
        \item $v \in M - \{ 0 \}$が
            ある$r \in A - \{ 0 \}$に対し
            $rv = 0$をみたすとき、
            $v$は$M$の
            \term{ねじれ元}[tortion element]{ねじれ元}[ねじれげん]
            であるという。
        \item $M$がねじれ元を持たないとき、
            $M$は\term{ねじれなし}[tortion-free]{ねじれなし}であるという。
        \item $M$のすべての元がねじれ元であるとき、
            $M$は\term{ねじれ加群}[tortion module]{ねじれ加群}[ねじれかぐん]
            であるという。
        \item $M$のすべてのねじれ元と$0$からなる\highlight{集合}を$M_\tor$と書く。
    \end{itemize}
\end{definition}

一般に$M_\tor$は$M$の部分加群であるとは限らないが、
$M$が整域上の加群ならば$M_\tor$は部分加群となる。

\begin{proposition}[ねじれ部分加群]
    $R$を可換環、
    $M$を$R$-加群とする。
    $M_\tor$は$M$の部分加群であり、
    $M / M_\tor$はねじれなしである。
\end{proposition}

\begin{proof}
    \TODO{}
\end{proof}

\begin{definition}[Annihilator]
    \idxsym{annihilator of an element}{$\ann_A(x)$}{$A$-加群の元の annihilator}
    \idxsym{annihilator of a module}{$\Ann_A(V)$}{$A$-加群の annihilator}
    $A \neq 0$を環、$V$を$A$-加群とする。
    \begin{itemize}
        \item 各$x \in V$に対し
            \begin{equation}
                \ann_A(x) \coloneqq \{ a \in A \mid ax = 0 \}
            \end{equation}
            を$x$の$A$における \term{annihilator}{annihilator} という。
            これは$A$の左イデアルである。
        \item
            \begin{alignat}{1}
                \Ann_A(V)
                    &\coloneqq \{
                        a \in A \mid \forall x \in V \text{ に対し } ax = 0
                    \} \\
                    &= \bigcap_{x \in V} \ann_A(x)
            \end{alignat}
            を$V$の \term{annihilator}{annihilator} という。
            これは$A$の両側イデアルである。
        \item $\Ann_A(V) = 0$のとき、
            $V$は\term{忠実}[faithful]{忠実}[ちゅうじつ]であるという。
    \end{itemize}
\end{definition}

既約加群の annihilator と係数環の極大左イデアルは
次のように対応する。

\begin{theorem}[既約加群の annihilator と極大左イデアルの対応]
    \label[theorem]{thm:ann-and-maximal-ideal-of-irreducible-module}
    $A \neq 0$を環とする。
    \begin{enumerate}
        \item $I \subset A$を極大左イデアルとすると、
            $\down{A}A / I$は既約$A$-加群である。
        \item $V$を既約$A$-加群、
            $v \in V - \{ 0 \}$とする。
            このとき$\ann_A(v)$は$A$の極大左イデアルである。
        \item $R$を可換環とする。
            次の全単射が成り立つ:
            \begin{alignat}{2}
                \{
                    \text{既約$R$-加群の同型類}
                \}
                    &\quad \leftrightarrow \quad
                    &&\Max(R) \\
                R / \frakm
                    &\quad \mapsfrom \quad
                    &&\frakm \\
                V
                    &\quad \mapsto \quad
                    &&\Ann_R(V)
            \end{alignat}
    \end{enumerate}
\end{theorem}

\begin{remark}
    この定理によれば、環$A$の左イデアル$I$に関し、
    商加群$\down{A}A / I$が既約であることと
    $I$が極大左イデアルであることは同値である。
\end{remark}

\begin{proof}
    \uline{(1)} \quad
    部分加群の対応原理 (\cref{thm:supmodule-correspondence-principle})
    より明らか。

    \uline{(2)} \quad
    $\ann_A(v)$の定義より、
    $A$-加群準同型
    $\varphi \colon \down{A}A \to V, \; a \mapsto av$
    は$\Ker(\varphi) = \ann_A(v)$をみたす。
    したがって
    \begin{equation}
        \begin{tikzcd}
            \down{A}A
                \ar{r}{\varphi}
                \ar[twoheadrightarrow]{d}
                & V \\
            A / \ann_A(v)
                \ar[dashed]{ru}[swap]{\wb{\varphi}}
        \end{tikzcd}
    \end{equation}
    を可換にする$A$-加群の同型$\wb{\varphi}$が誘導される。
    $A / \ann_A(v) \cong V$は
    既約ゆえに非自明な部分加群を持たないから、
    部分加群の対応原理 (\cref{thm:supmodule-correspondence-principle})
    より$\ann_A(v)$は$A$の極大左イデアルである。

    \uline{(3)} \quad
    \TODO{}
\end{proof}



% ------------------------------------------------------------
%
% ------------------------------------------------------------
\section{Schur の補題}

Schur の補題とその系について述べる。

\begin{definition}[代数的閉体]
    $K$を体とする。
    $K$が\term{代数的閉体}[algebraically closed field; ACF]{代数的閉体}[だいすうてきへいたい]
    であるとは、
    任意の$f \in K[X], \; n = \deg f \ge 1$が
    \begin{equation}
        f(X) = a(X - \alpha_1) \cdots (X - \alpha_n)
            \quad
            (a \neq 0, \; \alpha_i \in K)
    \end{equation}
    と表せることをいう。
\end{definition}

\cref{prop:embedding-of-field-into-algebra}より、
体上の代数には体が埋め込まれているとみなせるのであった。
このとき次が成り立つ。

\begin{lemma}[Dixmier の補題]
    \label[lemma]{lemma:dixmier}
    $K$を代数的閉体、
    $D$を$\dim_K D < \sharp K$なる可除$K$-代数とする。
    このとき$D = K$が成り立つ\footnote{
        厳密には、$D$が環準同型$\varphi$により$K$-代数になっているとして
        $K$-代数の同型$(D, \varphi) \cong (K, \id_K)$が成り立つということである。
    }。
\end{lemma}

\begin{proof}
    $K \subset D$であることはよい。
    $K \subsetneq D$であったと仮定して矛盾を導く。
    仮定よりある$\gamma \in D - K$が存在する。
    このとき、評価準同型$\ev_\gamma \colon K[X] \to D$は単射である。
    \begin{innerproof}
        $\ev_\gamma$が単射でないと仮定し矛盾を導く。
        仮定よりある$0 \neq f \in \Ker \ev_\gamma$が存在する。
        明らかに$f \notin K$だから$n \coloneqq \deg f \ge 1$である。
        そこで$K$が代数的閉体であることより
        \begin{equation}
            f(X) = a(X - \alpha_1) \cdots (X - \alpha_n)
                \quad
                (a \neq 0, \; \alpha_i \in K)
        \end{equation}
        と表せる。よって
        $0 = f(\gamma) = a(\gamma - \alpha_1) \cdots (\gamma - \alpha_n)$
        が成り立つ。
        $\gamma \notin K$ゆえに各$\gamma - \alpha_i$は$0$でなく、
        また$a$も$0$でないから、
        とくに$a$は$D$の零因子である。
        これは$D$が可除ゆえに零因子を持たないことに反する。
        背理法より$\ev_\gamma$は単射である。
    \end{innerproof}
    各$\alpha \in K$に対し、
    $\gamma \in D - K$ゆえに
    $\gamma - \alpha \neq 0$だから、
    $D$が可除であることより
    逆元$(\gamma - \alpha)^{-1} \in D$が存在する。
    そこで
    $B \coloneqq
        \{
            (\gamma - \alpha)^{-1} \in D \mid \alpha \in K
        \}$
    とおくと、$B$は$K$上1次独立である。
    \begin{innerproof}
        背理法のために、
        $B$が$K$上1次独立でないとする。
        すなわち、相異なるある$\alpha_1, \dots, \alpha_n \in K$と
        ある$a_1, \dots, a_n \in K^\times$が存在して
        \begin{equation}
            \sum_{i = 1}^n a_i (\gamma - \alpha_i)^{-1} = 0
        \end{equation}
        が成り立つと仮定する。
        いま$D$は可除ゆえに零因子を持たないから$n \ge 2$である。
        よって
        \begin{equation}
            a_1 (\gamma - \alpha_1)^{-1}
                + \sum_{i = 2}^n a_i (\gamma - \alpha_i)^{-1}
                = 0
        \end{equation}
        である。両辺に$(\gamma - \alpha_1) \cdots (\gamma - \alpha_n)$をかけて
        \begin{equation}
            a_1 (\gamma - \alpha_2) \cdots (\gamma - \alpha_n)
                + \sum_{i = 2}^n a_i \prod_{k \neq i} (\gamma - \alpha_k)
                = 0
        \end{equation}
        を得る。
        このとき$\ev_\gamma \colon K[X] \to D$が単射であることより、
        $K[X]$において
        \begin{equation}
            a_1 (X - \alpha_2) \cdots (X - \alpha_n)
                + \sum_{i = 2}^n a_i \prod_{k \neq i} (X - \alpha_k)
                = 0_{K[X]}
        \end{equation}
        が成り立つ。
        そこで$X$に$\alpha_1$を代入して
        $a_1 (\alpha_1 - \alpha_2) \cdots (\alpha_1 - \alpha_n) = 0_K$
        を得る。
        各$\alpha_i$は相異なるから各$\alpha_1 - \alpha_i$は$0_K$でなく、
        また$a_1$も$0_K$でないから、
        とくに$a_1$は$K$の零因子である。
        これは$K$が体ゆえに零因子を持たないことに反する。
        背理法より$B$は$K$上1次独立である。
    \end{innerproof}
    よって
    $\sharp K \le \sharp B \le \sharp \dim_K D < \sharp K$
    となり矛盾が従う。
    背理法より$K = D$である。
\end{proof}

次に述べる Schur の補題は
加群準同型全体のなす加群の構造に関する主張であり、
既約加群の性質からほとんど直ちに導かれるものであるが、
いくつかの有用な系が従う。

\begin{theorem}[Schur の補題]
    \label[theorem]{thm:Schur-lemma}
    \termhidden{Schur の補題}[Schur のほだい]
    $A$を環、$U_1, U_2$を既約$A$-加群とする。
    このとき、
    $\Hom_A(U_1, U_2)$の$0$でない元はすべて
    $A$-加群の同型
    $U_1 \xrightarrow{\sim} U_2$
    を与える。
    とくに
    $\Hom_A(U_1, U_2) \neq 0 \iff U_1 \cong U_2$である。
\end{theorem}

\begin{proof}
    $0 \neq f \in \Hom_A(U_1, U_2)$とする。
    $U_1, U_2$は既約$A$-加群だから
    $\Ker f, \; \Im f$は自明な部分加群であるが、
    いま$f \neq 0$より$\Ker f \neq U_1, \; \Im f \neq 0$だから
    $\Ker f = 0, \; \Im f = U_2$である。
    よって$f$は全単射、したがって$A$-加群の同型である。
\end{proof}

\begin{corollary}
    \label[corollary]{corollary:homomorphic-image-of-irreducible-module-is-isomorphic}
    既約加群$U$の準同型像は$0$でなければ$U$と同型である。
\end{corollary}

\begin{proof}
    Schur の補題と
    \cref{corollary:homomorphic-image-of-irreducible-module}
    より従う。
\end{proof}

\begin{corollary}
    \label[corollary]{corollary:Schur-lemma-corollary-1}
    $A$を環、
    $U$を既約$A$-加群とする。
    このとき、次が成り立つ:
    \begin{enumerate}
        \item $D \coloneqq \End_A(U)$は可除環である。
        \item 各$n \in \Z_{\ge 1}$に対し環の同型
            \begin{equation}
                \End_A(U^{\oplus n})
                    \cong
                    M_n(D)
            \end{equation}
            が成り立つ\TODO{$A$が可換環なら$A$-代数としての同型?}。
    \end{enumerate}
\end{corollary}

\begin{proof}
    \uline{(1)} \quad
    \cref{thm:Schur-lemma}より明らか。

    \uline{(2)} \quad
    \begin{alignat}{1}
        \End_A(U^{\oplus n})
            &= \Hom_A(U_1 \oplus \cdots \oplus U_n, U_1 \oplus \cdots \oplus U_n) \\
            &= \bigoplus_{i, j} \Hom_A(U_j, U_i) \\
            &= \bigoplus_{i, j} D_{ij}
    \end{alignat}
    \TODO{}
\end{proof}

\begin{example}
    上の系について、$A$が体$K$の場合を考えてみよう。
    $K$は非自明なイデアルを持たないから、
    $K$自身を$K$-加群とみなすと既約である。
    ここで$f \in \End_K(K)$を$f(1) \in K$に写す写像によって
    $K$-代数としての同型$\End_K(K) \cong K$が成り立つことに注意すれば、
    上の系は$\End_K(K^n) \cong M_n(K)$が成り立つことを主張している。
    このことは、有限次元ベクトル空間$K^n$の自己準同型が
    $K$上の行列と対応するというよく知られた線型代数学の結果に他ならない。
\end{example}

\begin{corollary}[Schur-Dixmier の補題]
    \termhidden{Schur-Dixmier の補題}[Schur-Dixmier のほだい]
    $K$を代数的閉体、
    $A$を$\dim_K A < \sharp K$なる$K$-代数とする。
    このとき、任意の既約$A$-加群$U$に対し
    \begin{equation}
        \End_A(U) = K
    \end{equation}
    が成り立つ。
\end{corollary}

\begin{remark}
    $K = \C, \; \dim_K A = \aleph_0$の場合などがよくある。
\end{remark}

\begin{proof}
    $D \coloneqq \End_A(U)$とおく。
    $A$は$K$-代数だから$D$も$K$-代数であり、
    \cref{corollary:Schur-lemma-corollary-1}より
    $D$は可除$K$-代数となる。
    よって$\dim_K D < \sharp K$を示せば
    Dixmier の補題 (\cref{lemma:dixmier}) より
    $\End_A(U) = D = K$が従う。
    $U$は既約$A$-加群だから、
    \cref{thm:ann-and-maximal-ideal-of-irreducible-module}より
    $A$のある極大イデアルによる商と$A$-加群として同型である。
    したがって
    $\dim_K U \le \dim_K A < \sharp K$である。
    一方、$x \in U - \{ 0 \}$をひとつ選んで
    $F \colon D \to U, \; \varphi \mapsto \varphi(x)$とおけば
    $F$は$K$-線型写像である。
    さらに$F$は単射である。
    実際、$\varphi \in D, \; F(\varphi) = \varphi(x) = 0$とすると、
    もし$\varphi \neq 0$なら
    Schur の補題 (\cref{thm:Schur-lemma}) より
    $\varphi$は同型だから$x = 0$となり$x \neq 0$に矛盾する。
    よって$\varphi = 0$、したがって$F$は単射である。
    よって$\dim_K D \le \dim_K U$だから
    $\dim_K D < \sharp K$が成り立つ。
    これが示したいことであった。
\end{proof}

% ------------------------------------------------------------
%
% ------------------------------------------------------------
\section{直既約加群}

直既約加群の概念を定義する。

\begin{definition}[直既約]
    $A$を環とする。
    $A$-加群$M$が
    \term{直既約}[indecomposable]{直既約}[ちょくきやく]
    であるとは、
    $M \neq 0$であって
    $0$と$M$以外の直和成分を持たないことをいう。
\end{definition}

\begin{remark}
    直既約加群の概念は
    その名の通り既約加群の定義とよく似ている。
    既約加群は非自明な部分加群を持たないから
    もちろん非自明な直和成分を持たず、したがって直既約である。
    逆に直既約加群は非自明な部分加群を持ちうるから、
    既約加群であるとは限らない。
\end{remark}

\begin{theorem}
    $R$をPID、
    $x \in R^\times$を素元とする。
    このとき
    任意の$n \in \Z_{\ge 1}$に対し
    $R / (x^n)$は局所環であり、
    また$R$-加群としては直既約となる。
\end{theorem}

\begin{proof}
    $A \coloneqq R / (x^n)$とおく。
    $x$の$A$における像を$\wb{x}$と書く。
    このとき$(\wb{x})$は$A$の極大イデアルである。
    $\frakm$を$A$の任意の極大イデアルとすると
    $0 = \wb{x}^n \in \frakm$ゆえに
    $\wb{x} \in \frakm$である。
    したがって$(\wb{x}) = \frakm$だから
    $A$は局所環である。

    $A$が$R$-加群として直既約であることを示す。
    $A = M_1 \oplus M_2$と
    $0$でない$R$-部分加群の直和に分解できたとすると、
    ある$e_i \in M_i \; (i = 1, 2)$が存在して
    $1 = e_1 + e_2$が成り立つ。
    このとき$e_1 e_2 \in M_1 \cap M_2 = 0$
    だから$e_1, e_2$は単元ではなく、
    したがって$e_1, e_2 \in (\wb{x})$となる。
    よって$1 = e_1 + e_2 \in (\wb{x})$となり
    $(\wb{x})$が$A$の極大イデアルであることに矛盾する。
    したがって$A$は直既約である。
\end{proof}






% ============================================================
%
% ============================================================
\chapter{有限生成性}
\label[chapter]{chapter:finiteness}

この章では加群と環の有限生成性について述べる。
まず Jacobson 根基の概念を導入し、
それを用いて有限生成加群に関する最も重要な定理のひとつである
Nakayama の補題を示す。
次に加群の有限生成性を強化した概念であるネーター加群とアルティン加群を導入し、
さらに強い概念として組成列を持つ加群を導入する。
最後に加群から環の話題へ移ってネーター環を定義する。

% ------------------------------------------------------------
%
% ------------------------------------------------------------
\section{Jacobson 根基と Nakayama の補題}

この節では、
有限生成加群に関する最も重要な定理のひとつである
Nakayama の補題について述べる。

$0$でない有限生成加群は既約な商加群をもつ。

\begin{proposition}[既約商加群の存在]
    \label[proposition]{prop:irreducible-quotient-module}
    $A$を環、
    $V$を有限生成$A$-加群とする。
    このとき、任意の$A$-部分加群$V' \subsetneq V$に対し、
    $V' \subset W \subsetneq V$
    なるある$A$-部分加群$W$であって
    $V / W$が既約となるものが存在する。
\end{proposition}

\begin{proof}
    \TODO{Zorn の補題を使う}
\end{proof}

Jacobson 根基を定義する。
環$A$の Jacobson 根基の元は、あらゆる既約$A$-加群に$0$として作用するものである。

\begin{definition}
    \idxsym{left primitive ideal}{$\Prim(A)$}{$A$の左原始イデアル}
    $A$を環とする。
    \begin{enumerate}
        \item $A$の両側イデアル$I$が
            \term{左原始イデアル}[left primitive ideal]{左原始イデアル}[ひだりげんしいである]
            であるとは、
            ある既約$A$-加群$U$が存在して
            $I = \Ann_A(U)$が成り立つことをいう。
        \item $A$の左原始イデアルの全体を
            \begin{equation}
                \Prim(A) \coloneqq \{
                    \text{$A$の左原始イデアル}
                \}
            \end{equation}
            と書く。
        \item $A$の左原始イデアル全部の共通部分を
            \begin{equation}
                J(A) \coloneqq \bigcap_{I \in \Prim(A)} I
            \end{equation}
            と書き、これを$A$の
            \term{Jacobson 根基}[Jacobson radical]{Jacobson 根基}[Jacobson こんき]
            という。
    \end{enumerate}
\end{definition}

Jacobson 根基は次のように特徴付けることができる。

\begin{theorem}[Jacobson 根基の特徴付け]
    $A$を環とする。
    $x \in A$に関し次は同値である:
    \begin{enumerate}
        \item $x \in J(A)$
        \item 任意の既約$A$-加群$U$に対し$xU = 0$となる。
        \item 任意の極大左イデアル$I \subset A$に対し$x(A / I) = 0$となる。
        \item $x \in \bigcap_{I \colon \text{$A$の極大左イデアル}} I$
        \item 任意の$a \in A$に対し$1 - ax$が左逆元を持つ。
    \end{enumerate}
    \TODO{}
\end{theorem}

\begin{proof}
    \uline{(1) \Leftrightarrow (2)} \quad
    定義より明らか。

    \uline{(1) \Rightarrow (2)} \quad
    \TODO{}

    \uline{(4) \Rightarrow (5)} \quad
    $x \in \bigcap_{I \colon \text{$A$の極大左イデアル}} I$とする。
    $a \in A$とすると
    $1 = ax + (1 - ax)$である。
    $1 - ax$がある極大左イデアル$I$に属したとすると、
    $x$したがって$ax$も$I$に属するから$1 \in I$となり
    $I$が極大左イデアルであることに矛盾する。
    よって$1 - ax$はいかなる極大左イデアルにも属さず、
    したがって$A (1 - ax) = A$である。
    よって$1 - ax$は左逆元を持つ。
\end{proof}

Jacobson 根基は右イデアルを用いて特徴付けることもできる。

\begin{proposition}[Jacobson 根基の右イデアルによる特徴付け]
    \TODO{}
\end{proposition}

\begin{proof}
    \TODO{}
\end{proof}

\begin{lemma}
    $A$を環、$V$を$A$-加群、$V' \subsetneq V$を部分$A$-加群とする。
    $V / V'$が既約$A$-加群ならば
    \begin{equation}
        J(A) V \subset V'
    \end{equation}
    が成り立つ。
\end{lemma}

\begin{proof}
    $V / V'$は既約$A$-加群だから
    Jacobson 根基の定義より$J(A) (V / V') = 0$であり、
    したがって$J(A) V \subset V'$である。
\end{proof}

Nakayama の補題を示す。

\begin{theorem}[Nakayama の補題]
    \termhidden{Nakayama の補題}[Nakayama のほだい]
    $A$を環、$V$を有限生成$A$-加群、
    $V' \subset V$を部分$A$-加群とする。
    このとき$V' + J(A) V = V$ならば
    $V' = V$である。
\end{theorem}

\TODO{局所環の場合、有限次元ベクトル空間に帰着させるための橋渡しとなる? cf. [Reid]}

\TODO{Cayley-Hamilton から示すこともできる? cf. \cref{problem:algebra2-6.86}}

\begin{proof}
    $V' \subsetneq V$と仮定して矛盾を導く。
    $V$は有限生成$A$-加群だから、
    $V' \subsetneq V$の仮定と
    \cref{prop:irreducible-quotient-module}より
    $V' \subset W \subsetneq V$なる
    ある$A$-部分加群$W$が存在して$V / W$は既約となる。
    したがって上の補題より
    $J(A)V \subset W$が成り立つ。
    よって$V = V' + J(A)V \subset W \subsetneq V$
    となり矛盾が従う。
\end{proof}

\begin{corollary}
    $A$を環、$V$を有限生成$A$-加群とする。
    このとき$J(A) V = V$ならば$V = 0$である。
    \qed
\end{corollary}

\begin{corollary}
    $A$を環、$V$を有限生成$A$-加群、
    $v_1, \dots, v_n$を$V / J(A)V$の$A / J(A)$上の生成元、
    $p \colon V \to V / J(A)V$を標準射影とする。
    このとき、$p(\wt{v}_i) = v_i$なる任意の$\wt{v}_i$に対し
    $V$は$A$上$\wt{v}_1, \dots, \wt{v}_n$により生成される。
\end{corollary}

\begin{proof}
    \TODO{}
\end{proof}

\begin{corollary}
    $A$を可換局所環
    \TODO{}
\end{corollary}

\begin{proof}
    \TODO{}
\end{proof}



% ------------------------------------------------------------
%
% ------------------------------------------------------------
\section{ネーター加群とアルティン加群}
\label[section]{section:noetherian-modules-and-artinian-modules}

ネーター加群とアルティン加群を定義する。

\begin{definition}[昇鎖条件と降鎖条件]
    $A$を環、
    $V$を$A$-加群とする。
    \begin{itemize}
        \item $V$の$A$-部分加群の増大列
            \begin{equation}
                V_1 \subset V_2 \subset \cdots
            \end{equation}
            あるいは減少列
            \begin{equation}
                V_1 \supset V_2 \supset \cdots
            \end{equation}
            が\term{停留的}[starionary]{停留的}[ていりゅうてき]であるとは、
            ある$N \in \Z_{\ge 1}$が存在して
            \begin{equation}
                V_N = V_{N + 1} = \cdots
            \end{equation}
            をみたすことをいう。
        \item $V$の任意の$A$-部分加群の増大列が停留的であるとき、
            $V$は
            \term{昇鎖条件}[ascending chain condition; ACC]{昇鎖条件}[しょうさじょうけん]
            をみたすという。
        \item $V$の任意の$A$-部分加群の増大列が停留的であるとき、
            $V$は
            \term{降鎖条件}[descending chain condition; DCC]{降鎖条件}[こうさじょうけん]
            をみたすという。
    \end{itemize}
\end{definition}

ネーター加群は、有限生成加群の有限生成性を強化したものとみなせる。

\TODO{有限余生成はコンパクト性のFIPによる特徴付けと似ている?}

\begin{definition}[ネーター加群とアルティン加群]
    $A$を環、
    $V$を$A$-加群とする。
    $V$が
    \term{ネーター加群}[noetherian module]{ネーター加群}[ねーたーかぐん]
    であるとは、$V$が次の互いに同値な条件のうち
    少なくとも1つ (よってすべて) をみたすことをいう (同値性はこのあと示す):
    \begin{description}
        \item[(N1)] $V$は昇鎖条件をみたす。
        \item[(N2)] $V$の任意の$A$-部分加群は有限生成である。
        \item[(N3)] $V$の$A$-部分加群からなる任意の集合は空でない限り極大元を持つ。
    \end{description}

    $V$が
    \term{アルティン加群}[artinian module]{アルティン加群}[あるてぃんかぐん]
    であるとは、$V$が次の互いに同値な条件のうち
    少なくとも1つ (よってすべて) をみたすことをいう (同値性はこのあと示す):
    \begin{description}
        \item[(A1)] $V$は降鎖条件をみたす。
        %\item $V$の任意の$A$-部分加群は有限余生成である。
        \item[(A3)] $V$の$A$-部分加群からなる任意の集合は空でない限り極小元を持つ。
    \end{description}
\end{definition}

\begin{proof}[同値性の証明.]
    アルティン性の同値性の証明はネーター性の場合と同様だから、
    ネーター性の同値性のみ示す。

    \uline{(1) \Rightarrow (2)} \quad
    \begin{equation}
        0
            \subsetneq \langle v_1 \rangle
            \subsetneq \langle v_1, v_2 \rangle
            \subsetneq \dots
            \subsetneq \langle v_1, \dots, v_{n - 1} \rangle
            \subsetneq W
    \end{equation}
    \TODO{}

    \uline{(2) \Rightarrow (1)} \quad
    \TODO{}
\end{proof}

\begin{example}[ネーター環とアルティン環の例]
    ~
    \begin{itemize}
        \item $A$を環とする。既約$A$-加群は、定義から明らかにネーターかつアルティンである。
        \item $K$を体、$A$を$K$-代数とする。
            $A$-加群$V$が$K$上有限次元ならば、
            $V$はネーターかつアルティンである。
    \end{itemize}
\end{example}

\begin{example}[ネーター/アルティン加群と有限生成加群の関係]
    ~
    \begin{itemize}
        \item ネーター加群はアルティン加群であるとは限らない。
            実際、$\down{\Z}\Z$は明らかにネーター加群であるが、
            $2\Z \supsetneq 4\Z \supsetneq \cdots$
            は停留的でないから$\down{\Z}\Z$はアルティン加群でない。
        \item アルティン加群はネーター加群であるとは限らない
            (cf. \cref{problem:algebra2-5.68})。
        \item アルティン加群は有限生成加群であるとは限らない。
            \TODO{例?}
        \item 有限生成加群はネーター加群であるとは限らない。
            実際、無限変数多項式環$R \coloneqq \Q[X_0, X_1, \dots]$の左正則加群$\down{R}R$は
            $R$上$1$により生成される有限生成加群だが、
            $R$-部分加群$\langle X_0, X_1, \dots \rangle$は
            $R$上有限生成でないから$\down{R}R$はネーター加群でない。
    \end{itemize}
\end{example}

次の定理により、ネーター性/アルティン性は完全系列を介して "伝播" することがわかる。
とくにネーター性/アルティン性は部分加群や商加群に遺伝する。

\begin{theorem}[完全系列と有限性]
    \label[theorem]{thm:exact-sequence-and-finiteness}
    $A$を環とする。
    $\lMod{A}$の完全列
    \begin{equation}
        \begin{tikzcd}
            0
                \ar{r}
                & X
                    \ar{r}
                & Y
                    \ar{r}
                & Z
                    \ar{r}
                & 0
        \end{tikzcd}
    \end{equation}
    に対し次は同値である:
    \begin{enumerate}
        \item $Y$はネーター (resp. アルティン) である。
        \item $X, Z$はネーター (resp. アルティン) である。
    \end{enumerate}
\end{theorem}

\begin{proof}
    \TODO{}
\end{proof}

$0$でないアルティン加群は既約な部分加群をもつ。

\begin{theorem}[既約部分加群の存在]
    $A$を環とする。
    $0$でないアルティン$A$-加群は
    既約部分加群をもつ。
\end{theorem}

\begin{proof}
    $V$を$0$でないアルティン$A$-加群とすると、
    $V$のアルティン性より
    $V$の$0$でない$A$-部分加群全体の集合は
    極小元$V_0$をもつ。
    このとき$V_0 \neq 0$と極小性より$V_0$は既約である。
\end{proof}



% ------------------------------------------------------------
%
% ------------------------------------------------------------
\section{組成列}

組成列について述べる。
まずフィルトレーションの概念を定義する。

\begin{definition}[フィルトレーション]
    $A$を環、
    $V$を$A$-加群とする。
    $A$-部分加群の減少列
    \begin{equation}
        V = V_0 \supsetneq V_1 \supsetneq \cdots \supsetneq V_n = 0
    \end{equation}
    を$V$の\term{フィルトレーション}[filtration]{フィルトレーション}
    といい、
    $n$をフィルトレーションの
    \term{長さ}[length]{長さ}[ながさ]
    という。
\end{definition}

\begin{definition}
    $A$を環とする。
    \begin{itemize}
        \item $A$-加群$V$が
            \term{有限の長さを持つ}[of finite length]{有限の長さを持つ}[ゆうげんのながさをもつ]
            とは、
            $V$の長さ$n \in \Z_{\ge 0}$の
            フィルトレーション$\{ V_i \}_{i = 0}^n$であって、
            各$V_i / V_{i + 1} \; (0 \le i \le n - 1)$が既約であるようなものが
            存在することをいう。
            このとき、$\{ V_i \}_i$を$V$の
            \term{組成列}{組成列}[そせいれつ]
            といい、
            各$V_i / V_{i + 1}$を$\{ V_i \}$の
            \term{既約成分}{既約成分}[きやくせいぶん]という。
        \item $n$を$\{ V_i \}$の
            \term{長さ}[length]{長さ}[ながさ]
            という。
        \item 各既約$A$-加群$U$に対して、
            $U \cong V_i / V_{i + 1}$となる
            $0 \le i \le n$の個数を$U$の$\{ V_i \}$における
            \term{重複度}{重複度}[ちょうふくど]
            という。
        \item $V$のすべての組成列の長さの最小値を$l(V)$で表す。
            ただし$l(0) = 0$と定める。
        \item ふたつの組成列$\{ V_i \}$と$\{ V'_i \}$が
            \term{同値}{同値!組成列の---}[どうち]であるとは、
            長さが一致し、任意の既約$A$-加群の
            重複度が一致することをいう。
    \end{itemize}
\end{definition}

有限の長さを持つ加群は
非常に強い有限性を持っている。

\begin{theorem}
    $A$を環とする。
    $A$-加群$V$に関し次は同値である:
    \begin{enumerate}
        \item $V$は有限の長さを持つ。
        \item $V$はネーターかつアルティンである。
    \end{enumerate}
\end{theorem}

\begin{proof}
    \uline{(1) \Rightarrow (2)} \quad
    $V = V_0 \supsetneq V_1 \supsetneq \cdots \supsetneq V_n = 0$を
    $V$の長さ$n$の組成列とする。
    系列
    \begin{equation}
        \begin{tikzcd}
            0
                \ar{r}
                & V_n
                    \ar{r}
                & V_{n - 1}
                    \ar{r}
                & V_{n - 1} / V_n
                    \ar{r}
                & 0
        \end{tikzcd}
    \end{equation}
    は完全列であり、
    $V_n = 0$および既約$A$-加群$V_{n - 1} / V_n$は
    ネーターかつアルティンだから、
    \cref{thm:exact-sequence-and-finiteness}より
    $V_{n - 1}$はネーターかつアルティンである。
    帰納的に$V_0 = V$がネーターかつアルティンであることがわかる。

    \uline{(2) \Rightarrow (1)} \quad

    \TODO{}
\end{proof}

\begin{corollary}
    $A$を環とする。
    $\lMod{A}$の完全列
    \begin{equation}
        \begin{tikzcd}
            0
                \ar{r}
                & V_1
                    \ar{r}
                & V_2
                    \ar{r}
                & V_3
                    \ar{r}
                & 0
        \end{tikzcd}
    \end{equation}
    に対し次は同値である:
    \begin{enumerate}
        \item $V_2$は有限の長さを持つ。
        \item $V_1, V_3$は有限の長さを持つ。
    \end{enumerate}
\end{corollary}

\begin{proof}
    \cref{thm:exact-sequence-and-finiteness}
    より明らか。
\end{proof}

組成列は本質的に一意的である。
これにより加群の次元のようなものを一義的に定義することができ、
これはベクトル空間の次元のように振る舞う。

\TODO{どういうこと?}

\begin{theorem}[Jordan-H\"{o}lder の定理]
    \label[theorem]{thm:Jordan-Holder}
    $A$を環とする。
    $A$-加群$V$が有限の長さを持つとき、
    $V$の任意の組成列は互いに同値となる。
\end{theorem}

\begin{proof}
    \TODO{}
\end{proof}



% ------------------------------------------------------------
%
% ------------------------------------------------------------
\section{ネーター環}

ネーター環を定義する。


\begin{definition}[ネーター環]
    $A$を環とする。
    $\down{A}A$がネーター加群のとき
    $A$を
    \term{左ネーター環}[left noetherian ring]{ネーター環}[ねーたーかん]
    という。
    右も同様に定義する。
    $A$が可換環のときは単に
    \term{ネーター環}[noetherian ring]{ネーター環}[ねーたーかん]
    という。
\end{definition}

環のネーター性がその上の加群のネーター性をもたらすことを確認しよう。
左ネーター環上の有限生成加群はネーター加群となる。

\begin{theorem}
    \label[theorem]{thm:fg-module-over-noetherian-ring-is-noetherian}
    $A$を左ネーター環とすると、
    任意の有限生成$A$-加群はネーター加群である。
\end{theorem}

\begin{proof}
    まず$A^{\oplus n}$がネーター加群であることを
    $n$に関する帰納法によって示す。
    $A = A^{\oplus n} / A^{\oplus (n - 1)}$ゆえに
    完全列
    \begin{equation}
        \begin{tikzcd}
            0
                \ar{r}
                & A^{\oplus (n - 1)}
                    \ar{r}
                & A^{\oplus n}
                    \ar{r}
                & A
                    \ar{r}
                & 0
        \end{tikzcd}
    \end{equation}
    を得る。
    いま$A$は左ネーター環ゆえにネーター加群で、
    また帰納法の仮定より$A^{\oplus (n - 1)}$もネーター加群なので、
    \cref{thm:exact-sequence-and-finiteness}より
    $A^{\oplus n}$もネーター加群である。

    つぎに$V$を有限生成$A$-加群とする。
    $V = \langle v_1, \dots, v_n \rangle$と表せて、
    $v_1, \dots, v_n$により定まる全射
    $p \colon A^{\oplus n} \to V$により
    完全列
    \begin{equation}
        \begin{tikzcd}
            0
                \ar{r}
                & \Ker p
                    \ar{r}
                & A^{\oplus n}
                    \ar{r}{p}
                & V
                    \ar{r}
                & 0
        \end{tikzcd}
    \end{equation}
    を得る。
    $A^{\oplus n}$はネーター加群なので
    \cref{thm:exact-sequence-and-finiteness}より
    $V$もネーター加群である。
\end{proof}

驚くべきことに、
ネーター環がさらに可換であれば、
その上の有限生成な加群のみならず
有限生成な代数までもネーター性をもつ。

\begin{theorem}[Hilbert の基底定理\footnote{
    Hilbert の時代には生成系のことを基底 (basis) と呼んでいたため
    このような名前になっている \cite{Rei95}。
}]
    \termhidden{Hilbert の基底定理}[Hilbert のきていていり]
    $R$を可換ネーター環とすると、
    $R$上有限生成な可換$R$-代数$A$は
    ネーター環である。
\end{theorem}

\begin{proof}
    $R[X_1, \dots, X_n]$がネーター環であることを示せばよく、
    さらに\cref{corollary:polynomial-ring-isomorphism}より
    $R[X]$がネーター環であることを示せばよい。
    $I \subset R[X]$を任意のイデアルとし、
    $I$が$R$-加群として有限生成であることを示す。
    \TODO{}
\end{proof}

\begin{remark}
    可換ネーター環の部分環はネーター環であるとは限らない
    (cf. \cref{problem:algebra2-5.74})。
\end{remark}

\begin{proposition}
    PIDはネーター環である。
\end{proposition}

\begin{proof}
    \TODO{生成元の既約元分解を考える}
\end{proof}



% ============================================================
%
% ============================================================
\chapter{半単純加群と半単純環}
\label[chapter]{chapter:semisimple-module-and-semisimple-ring}

\TODO{アルティン単純環は半単純加群/環とどういう関係?}

この章では半単純加群について述べた後、
アルティン単純環と半単純環の構造について詳しく調べる。
この章の目標は、代数学における最も重要な定理のひとつである
Wedderburn-Artin の構造定理を示すことである。

% ------------------------------------------------------------
%
% ------------------------------------------------------------
\section{半単純加群}

半単純加群の概念を定義する。

\begin{definition}[半単純加群]
    $A$を環、$V$を$A$-加群とする。
    $V$が\term{半単純}[semisimple]{半単純}[はんたんじゅん]
    あるいは
    \term{完全可約}[completely reducible]{完全可約}[かんぜんかやく]
    であるとは、
    $V$の既約部分$A$-加群の族$\{ V_i \}_{i \in I}$が存在して
    $V = \bigoplus_{i \in I} V_i$が成り立つことをいう。
\end{definition}

\begin{example}[半単純加群の例]
    ~
    \begin{itemize}
        \item $K$を体とする。有限次元$K$-ベクトル空間は
            $K$の有限個の直和に同型だから、
            $K$上の半単純加群である。
    \end{itemize}
\end{example}

加群が半単純であることを定義に沿って示すには
直和分解の存在を示さなければならないが、
実はもう少し簡単な条件を確認すればよい。
すなわち、加群が既約部分加群による "被覆" を持つとき、
その "部分被覆" によって直和分解ができる。

\begin{theorem}
    \label[theorem]{thm:semisimple-cover}
    $A$を環、$V$を$A$-加群、
    $\{ V_i \}_{i \in I}$を
    $V$の既約部分$A$-加群の族とする。
    このとき、
    $V = \sum_{i \in I} V_i$が成り立つならば、
    ある$J \subset I$が存在して
    $V = \bigoplus_{i \in J} V_i$が成り立つ。
    とくに$V$は半単純である。
\end{theorem}

\begin{proof}
    $\calS \coloneqq
        \left\{
            J \subset I
            \; \middle| \;
            \sum_{j \in J} V_j = \bigoplus_{j \in J} V_j
        \right\}$
    とおく。
    いま$V = \sum_{i \in I} V_i$ゆえに
    $I \neq \emptyset$だから$I$の1点からなる部分集合が存在して、
    それは明らかに$\calS$に属する。
    よって$\calS \neq \emptyset$である。
    $\calS$が帰納的半順序集合であることを示す。
    そこで$\calI \subset \calS$を任意の全順序部分集合とする。
    ここで$J_0 \coloneqq \bigcup_{J \in \calI} J$とおくと
    $J_0$は$\calS$における$\calI$の上界である。
    \begin{innerproof}
        \TODO{}
    \end{innerproof}
    したがって Zorn の補題より
    $\calS$は極大元$J_1$を持つ。
    そこで$V' \coloneqq \bigoplus_{j \in J_1} V_j$とおく。
    ここで$V' \subsetneq V$であったとすると、
    $V = \sum_{i \in I} V_i$の仮定とあわせて、
    ある$k \in I - J_1$が存在して
    $V_k \not\subset V'$が成り立つ。
    いま$V_k$は既約ゆえに$V_k \cap V' = 0$だから
    $\sum_{j \in J_1 \cup \{ k \}} V_j
        = \left( \bigoplus_{j \in J_1} V_j \right) \oplus V_k$
    が成り立つ。
    よって$J_1 \cup \{ k \} \in \calS$となり、
    $J_1$の極大性に矛盾する。
    したがって$V' = V$である。
\end{proof}

半単純加群の商加群は半単純である。

\begin{corollary}[半単純加群の商は半単純]
    \label[corollary]{corollary:quotinet-of-semisimple-module}
    $A$を環、
    $V$を半単純$A$-加群、
    $W \subset V$を$A$-部分加群とする。
    このとき$V / W$は半単純$A$-加群である。
    詳しくいえば、
    $p \colon V \to V / W$を標準射影とするとき、
    $V$の既約部分加群への直和分解
    $V = \bigoplus_{i \in I} V_i$に対し、
    ある$J \subset I$が存在して
    $V / W = \bigoplus_{i \in J} p(V_i)$が
    既約部分加群への直和分解となる。
\end{corollary}

\begin{proof}
    $V$の直和分解を$p$で写して
    $V / W
        = p\left( \bigoplus_{i \in I} V_i \right)
        = \sum_{i \in I} p\left( V_i \right)$
    が成り立つ。
    このとき、$V_i$が既約であることから
    $p(V_i)$は$0$または既約である。
    そこで$I$から$p(V_i) = 0$なる$i$をすべて除いたものを$I'$とおけば、
    $V / W = \sum_{i \in I'} p(V_i)$
    と既約部分加群の和で表せる。
    したがって上の定理より
    ある$J \subset I' \subset I$が存在して
    $V / W = \bigoplus_{j \in J} p(V_j)$
    と既約部分加群への直和分解が成り立つ。
    よって$V / W$は半単純である。
\end{proof}

Socle とは、加群の既約部分加群すべての和である。
実は最大の半単純部分加群にもなっている。

\begin{corollary}
    $A$を環、
    $V$を$A$-加群とする。
    このとき
    \begin{equation}
        \soc(V) \coloneqq
            \sum_{\substack{
                V_0 \subset V \colon
                    \text{既約$A$-部分加群}
            }} V_0
    \end{equation}
    とおくと、$\soc(V)$は$V$の最大の半単純部分加群である。
\end{corollary}

\begin{proof}
    \TODO{}
\end{proof}

半単純加群の部分加群は半単純となり、さらに直和因子となる。

\begin{theorem}
    \label[theorem]{thm:submodule-of-semisimple-module}
    $A$を環、
    $V$を半単純$A$-加群、
    $W$を$V$の$A$-部分加群とする。
    このとき次が成り立つ:
    \begin{enumerate}
        \item $W$は半単純である。
        \item ある$V$の$A$-部分加群$W'$が存在して
            $V = W \oplus W'$となる。
    \end{enumerate}
\end{theorem}

\begin{proof}
    まず(2)を示す。
    $p \colon V \to V / W$を標準射影とし、短完全列
    \begin{equation}
        \begin{tikzcd}
            0
                \ar{r}
                & W
                    \ar[hook]{r}
                & V
                    \ar{r}{p}
                & V / W
                    \ar{r}
                & 0
        \end{tikzcd}
    \end{equation}
    を考える。
    \cref{corollary:quotinet-of-semisimple-module}より
    ある$J \subset I$が存在して
    $V / W = \bigoplus_{i \in J} p(V_i)$が
    既約部分加群への直和分解となる。
    ここで Schur の補題 (\cref{thm:Schur-lemma}) より
    各$p|_{V_i} \colon V_i \to p(V_i)$は同型となるから、
    $A$-加群準同型
    \begin{equation}
        s \colon
            V / W = \bigoplus_{i \in J} p(V_i)
            \overset{\prod_{i \in J} (p|_{V_i})^{-1}}{\to}
            \bigoplus_{i \in J} V_i
            \hookrightarrow
            V
    \end{equation}
    が上の短完全列の right splitting を与える。
    よって$V = W \oplus s(V / W)$が成り立ち、(2)が従う。

    つぎに(1)を示す。
    $J' \coloneqq I - J$とおくと
    $V
        = \left(
            \bigoplus_{i \in J'} V_i
        \right)
            \oplus \left( \bigoplus_{i \in J} V_i \right)
        = \left(
            \bigoplus_{i \in J'} V_i
        \right) \oplus s(V / W)$
    となるから、標準射影に対して準同型定理を適用して
    $W \cong V / s(V / W) \cong \bigoplus_{i \in J'} V_i$が成り立つ。
    いま各$V_i$は既約だから
    $W$は半単純である。
\end{proof}

\begin{corollary}
    \label[corollary]{corollary:finite-irreducible-decomposition}
    $A$を環、
    $V$を半単純ネーター (またはアルティン) $A$-加群とする。
    このとき$V$の有限個の既約$A$-部分加群
    $V_1, \dots, V_n$が存在して
    $V = V_1 \oplus \cdots \oplus V_n$が成り立つ。
\end{corollary}

\begin{proof}
    \TODO{系なのか?}
    $V$は半単純だから、既約部分加群への直和分解
    $V = \bigoplus_{i \in I} V_i$が存在する。
    そこで$I$が有限集合であることをいえばよい。
    背理法のために$I$が無限集合であると仮定する。
    すると可算無限部分集合$J = \{ i_0, i_1, \dots \} \subset I$が存在する。
    いま各$V_i$は既約ゆえに$V_i \neq 0$だから、
    \begin{itemize}
        \item $V$がネーターの場合は部分加群の増大列
            \begin{equation}
                V_{i_0}
                    \subsetneq V_{i_0} \oplus V_{i_1}
                    \subsetneq \cdots
            \end{equation}
            が停留的でないため矛盾が従い、
        \item $V$がアルティンの場合は部分加群の減少列
            \begin{equation}
                \bigoplus_{i \in J} V_i
                    \supsetneq \bigoplus_{i \in J - \{ i_0 \}} V_i
                    \supsetneq \bigoplus_{i \in J - \{ i_0, i_1 \}} V_i
                    \supsetneq \cdots
            \end{equation}
            が停留的でないため矛盾が従う。
    \end{itemize}
    したがって$I$は有限集合である。
\end{proof}

半単純加群を短完全列の分裂によって特徴付けることができる。

\begin{theorem}[半単純加群の特徴付け]
    $A$を環とする。
    $A$-加群$M$に関し、次は同値である:
    \begin{enumerate}
        \item $M$は半単純加群である。
        \item $\lMod{A}$の任意の完全列
            \begin{equation}
                \begin{tikzcd}
                    0 \ar{r}
                        & V_1 \ar{r}
                        & M \ar{r}
                        & V_2 \ar{r}
                        & 0
                        & (\text{exact})
                \end{tikzcd}
            \end{equation}
            は分裂する。
    \end{enumerate}
\end{theorem}

\begin{proof}
    \uline{(1) \Rightarrow (2)} \quad
    \cref{thm:submodule-of-semisimple-module}より従う。

    \uline{(2) \Rightarrow (1)} \quad
    \TODO{}
\end{proof}



% ------------------------------------------------------------
%
% ------------------------------------------------------------
\section{アルティン環}

この節では、次節で述べるアルティン単純環の準備としてアルティン環を定義する。
アルティン環は見かけ上はネーター環と対になる概念であるが、
後で\cref{prop:commutative-artinian-ring-is-noetherian}
で述べるように、
可換環においてはアルティン環はネーター環でもある。
また、ある意味ではアルティン環は体の次に簡単な種類の環である。
\TODO{どういう意味?}

\begin{definition}[アルティン環]
    $A$を環とする。
    $\down{A}A$がアルティン加群のとき
    $A$を
    \term{左アルティン環}[left artinian ring]{アルティン環}[あるてぃんかん]
    という。
    右も同様に定義する。
    $A$が可換環のときは単に
    \term{アルティン環}[artinian ring]{アルティン環}[あるてぃんかん]
    という。
\end{definition}

可換なアルティン環はいくつかの著しい性質を持つ。

\begin{proposition}
    可換アルティン環の素イデアルは極大イデアルである。
\end{proposition}

\begin{proof}
    cf. \cref{problem:algebra2-5.75}
\end{proof}

\begin{proposition}
    可換アルティン環は極大イデアルを高々有限個しか持たない。
\end{proposition}

\begin{proof}
    cf. \cref{problem:algebra2-5.76}
\end{proof}

\begin{proposition}
    \label[proposition]{prop:commutative-artinian-ring-is-noetherian}
    可換アルティン環はネーター環である。
\end{proposition}

\begin{answer}
    cf. \cref{problem:algebra2-6.84}
\end{answer}



% ------------------------------------------------------------
%
% ------------------------------------------------------------
\section{アルティン単純環}

アルティン単純環、すなわち単純なアルティン環について調べる。

\begin{proposition}
    \label[proposition]{prop:end-and-opposite-ring}
    $A$を環とする。
    $\End_A(\down{A}A) \cong A^\OP$が成り立つ。
\end{proposition}

\begin{proof}
    cf. \cref{problem:algebra2-4.51}
\end{proof}

\begin{proposition}
    $A$を左アルティン単純環とする。
    \begin{enumerate}
        \item 既約$A$-加群$U$が同型を除いて一意に存在する。
        \item $D \coloneqq \End_A(U)$とおく。
            ある$n \in \Z_{\ge 1}$が存在して
            $A \cong M_n(D^\OP)$が成り立つ。
    \end{enumerate}
\end{proposition}

\begin{proof}
    $A$は左アルティン環だから、
    $A$の$0$でない左イデアルのなかで極小なもの$U \subset A$が存在する。
    このとき$U \neq 0$と極小性より$U$は$A$-加群として既約である。
    $U$の一意性の証明は最後にまわす。
    ここで任意の$a \in A$に対し
    $Ua \subset A$だから、
    $U$の既約性と
    \cref{corollary:homomorphic-image-of-irreducible-module-is-isomorphic}
    より$Ua = 0$または$Ua \cong U$である。
    さて、$\sum_{a \in A} Ua$は$A$の$0$でない両側イデアルだから、
    $A$が単純環であることとあわせて
    $\sum_{a \in A} Ua = A$が成り立つ。
    したがって、\cref{thm:semisimple-cover}と
    \cref{corollary:finite-irreducible-decomposition}
    よりある$a_1, \dots, a_n \in A$が存在して
    $A = \bigoplus_{i = 1}^n Ua_i, \; Ua_i \cong U$が成り立つ。
    よって$\End_A(\down{A}A) \cong M_n(D)$となり、
    \cref{prop:end-and-opposite-ring}
    より
    $A \cong (A^\OP)^\OP
        \cong \End_A(\down{A}A)^\OP
        \cong M_n(D)^\OP
        \cong M_n(D^\OP)$
    が成り立つ。
    これで(2)がいえた。

    最後に$U$の一意性を示す。
    $A = \bigoplus_{i = 1}^n Ua_i, \; Ua_i \cong U$より、
    $\down{A}A$は既約成分がすべて$U$に同型な組成列を持つ。
    一方、$V$を既約$A$-加群とすると
    \cref{thm:ann-and-maximal-ideal-of-irreducible-module}
    より$A$のある極大左イデアル$I$が存在して
    $A$-加群の同型$V \cong A / I$が成り立つ。
    このことと$A$の左アルティン性より、
    極大$A$-部分加群を順次とることで得られる列
    $\down{A}A \supsetneq I \supsetneq \cdots$
    は$V$と同型な既約成分をもつ$\down{A}A$の組成列となる。
    したがって、
    Jordan-H\"{o}lder の定理
    (\cref{thm:Jordan-Holder})
    より$U \cong V$が成り立つ。
    これで$U$の一意性がいえて(1)が示せた。
\end{proof}

\begin{corollary}
    $K$を代数的閉体、
    $A$を$K$上有限次元な単純$K$-代数とする。
    このとき、ある$n \in \Z_{\ge 1}$が存在して
    $A \cong M_n(K)$が成り立つ。
\end{corollary}

\begin{proof}
    \TODO{}
\end{proof}

\begin{corollary}
    左アルティン単純環は
    左/右ネーター、右アルティンになる。
\end{corollary}

\begin{proof}
    \TODO{}
\end{proof}

\begin{theorem}
    $D$を可除環とする。
    \begin{enumerate}
        \item $\down{D}D, \; D_D$は既約である。
        \item \TODO{}
            \begin{equation}
                \Hom_{D^\OP}(D^m, D^n) \cong M_{n, m}(D)
            \end{equation}
        \item $m \neq n$ならば$D^m \not\cong D^n$である。
    \end{enumerate}
\end{theorem}

\begin{proof}
    \TODO{}
\end{proof}

\begin{definition}[右ベクトル空間の次元]
    $D$を可除環とする。
    右$D$-ベクトル空間が有限生成ならば
    次元が well-defined に定まる。
    \TODO{}
\end{definition}

\begin{theorem}
    $D$を可除環、
    $n \in \Z_{\ge 1}$とする。
    $A \in M_n(D)$に関し次は同値である:
    \begin{enumerate}
        \item $A$は可逆である。
        \item $A$はいくつかの基本行列の積である。
        \item 右1次独立なある$v_1, \dots, v_n \in D^n$が存在して
            $A = \begin{pmatrix}
                v_1 & \cdots & v_n
            \end{pmatrix}$が成り立つ。
    \end{enumerate}
\end{theorem}

\begin{proof}
    \TODO{}
\end{proof}

\begin{corollary}
    \label[corollary]{corollary:e1-matrix}
    任意の$0 \neq v \in D^n$に対し、
    $Ae_1 = v$となる$A \in \GL_n(D)$が存在する。
\end{corollary}

\begin{corollary}
    $D^n$は既約$M_n(D)$-加群である。
\end{corollary}

\TODO{ベクトル空間の自己同型写像が行列と対応することと関係ある?}

\begin{lemma}
    $D$を可除環、
    $n \in \Z_{\ge 1}$とする。
    このとき
    $\End_{M_n(D)}(D^n) \cong D^\OP$が成り立つ。
\end{lemma}

\begin{proof}
    $D^n$の標準基底を$e_1, \dots, e_n$とおく。
    $B \subset M_n(D)$を
    \begin{equation}
        B \coloneqq \left\{
            \left[\begin{smallmatrix}
                1 \\
                0 \\
                \vdots & \quad \text{\large *} \quad \\[1ex]
                0
            \end{smallmatrix}\right]
            \in M_n(D)
        \right\}
    \end{equation}
    で定めると、$B$は$M_n(D)$の部分環となり、
    $Be_1 = \{ e_1 \}$が成り立つ。
    さて、$\varphi \in \End_{M_n(D)}(D^n), \;
        v \in D^n - \{ 0 \}$
    とする。
    すると\cref{corollary:e1-matrix}より
    ある$g \in \GL_n(D)$が存在して
    $ge_1 = v$が成り立つ。
    ここで、すべての$A \in B$に対し
    $\varphi(e_1) = \varphi(Ae_1) = A\varphi(e_1)$
    が成り立つから、
    $B$の元の形から明らかに、ある$d \in D$が一意に存在して
    $\varphi(e_1) = e_1 d$が成り立つ。
    したがって
    $\varphi(v) = \varphi(ge_1) = g\varphi(e_1) = ge_1d = vd$
    となる。
    以上より
    写像$\End_{M_n(D)}(D^n) \to D^\OP, \; \varphi \mapsto d$
    が得られた。
    \TODO{}
\end{proof}

\begin{theorem}
    $D_1, D_2$を可除環とする。
    このとき$M_m(D_1) \cong M_n(D_2)$ならば
    $D_1 \cong D_2$かつ$m = n$が成り立つ。
\end{theorem}

\begin{proof}
    \TODO{}
\end{proof}



% ------------------------------------------------------------
%
% ------------------------------------------------------------
\section{半単純環}

\begin{theorem}
    $A$を左アルティン環、
    $\{ I_j \}_{j \in J}$を$A$の左イデアルの族とする。
    このとき$J$の有限部分集合$J_0$が存在して
    \begin{equation}
        \bigcap_{j \in J} I_j = \bigcap_{j \in J_0} I_j
    \end{equation}
    が成り立つ。
\end{theorem}

\begin{proof}
    \TODO{}
\end{proof}

\begin{definition}[半単純環]
    $A$を環とする。
    $\down{A}A$が半単純加群であるとき、
    $A$を\term{半単純環}[semisimple ring]{半単純環}[はんたんじゅんかん]という。
\end{definition}

\begin{remark}
    単純環は半単純環であるとは限らない。
\end{remark}

\begin{theorem}
    $A$を半単純環とする。
    \begin{enumerate}
        \item $\down{A}A$は有限個の既約部分加群の直和になる。
            したがって$A$は左ネーターかつ左アルティンである。
        \item $A$上の既約加群は同型を除いて有限個であり、
            これらと同型な$\down{A}A$の部分加群が存在する。
    \end{enumerate}
\end{theorem}

\begin{proof}
    \TODO{}
\end{proof}

\TODO{Socle とはちょっと違う?}

\begin{definition}
    $A$を半単純環、
    $U$を既約$A$-加群とする。
    \begin{equation}
        A_U \coloneqq \sum_{\substack{
            U_0 \subset \down{A}A \colon
                \text{$A$-部分加群} \\
            U_0 \cong U
        }} U_0
    \end{equation}
    とおく。
\end{definition}

\begin{lemma}
    $V$を$A_U$の既約部分加群とすると
    $V \cong U$が成り立つ。
\end{lemma}

\begin{proof}
    \TODO{}
\end{proof}

\begin{theorem}[Wedderburn]
    $A$を半単純環、
    $U_1, \dots, U_l$を$A$の既約加群の同型類の完全代表系とする。
    \begin{enumerate}
        \item $A_{U_i}$は$A$の両側イデアルである。
        \item ある$n_i \in \Z_{\ge 1}$が存在して
            $A_{U_i} \cong U_i^{\oplus n_i}$が成り立つ。
        \item $A = A_{U_1} \oplus \cdots \oplus A_{U_l}$
        \item $A_{U_i} A_{U_j} = 0 \; (i \neq j)$
    \end{enumerate}
\end{theorem}

\begin{proof}
    \TODO{}
\end{proof}

Wedderburn-Artin の定理を述べる。
これは代数学における最も重要な定理のひとつである\cite[p.153]{AF92}。

\begin{theorem}[Wedderburn-Artin]
    環$A$に関し次は同値である:
    \begin{enumerate}
        \item $A$は半単純環である。
        \item $A^\OP$は半単純環である。
        \item 任意の$A$-加群は半単純である。
        \item ある可除代数$D_1, \dots, D_l$と
            $n_1, \dots, n_l \in \Z_{\ge 1}$が存在して
            \highlight{環の同型}
            $A \cong M_{n_1}(D_1) \oplus \cdots \oplus M_{n_l}(D_l)$が成り立つ。
    \end{enumerate}
\end{theorem}

\begin{proof}
    \TODO{}
\end{proof}

\begin{theorem}
    $A$を環とする。
    \begin{enumerate}
        \item $A$が左アルティン環ならば、
            $A / J(A)$は半単純環である。
        \item $A$が単純環ならば$J(A) = 0$である。
    \end{enumerate}
\end{theorem}

\begin{proof}
    \TODO{}
\end{proof}





% ------------------------------------------------------------
%
% ------------------------------------------------------------
\newpage
\section{演習問題}

\subsection{Problem set 5}

\begin{problem}[代数学II 5.68]
    \label[problem]{problem:algebra2-5.68}
    アルティン加群であるがネーター加群でないような例を挙げよ。
\end{problem}

\begin{answer}
    $p$を素数とする。
    \term{Pr\"{u}fer $p$群}[Pr\"{u}fer $p$-group]{Pr\"{u}fer 群}[Pr\"{u}fer ぐん]
    $M \coloneqq \Z[1/p] / \Z$が
    問題の条件をみたす例であることを示す\footnote{
        より強く、$M$は$\Z$上有限生成でもない。
        \TODO{証明?}
    }。
    ただし$\Z[1/p]$は$1/p$により$\Z$上生成された$\Q$の$\Z$-部分代数を
    $\Z$-加群とみなしたものである。

    まず次の claim を示しておく。
    \begin{itemize}
        \item $M$の任意の$\Z$-真部分加群$N$に対し
            ある$k \in \Z_{\ge 0}$が存在して
            $N = \langle 1/p^k \rangle / \Z$が成り立つ。
    \end{itemize}
    ただし各$q \in \Z[1/p]$に対し$\langle q \rangle$は
    $\Z$上$q$により生成された$\Z[1/p]$の巡回部分加群を表す。
    \begin{innerproof}
        各$x \in M$に対し
        $k_x \coloneqq \min\{
                k \in \Z_{\ge 0}
                \mid
                x \in \langle 1/p^k \rangle / \Z
            \}$
        とおくと
        \begin{equation}
            x = \frac{n}{p^{k_x}} + \Z
            \quad
            (n \in \Z, \; \gcd(n, p^{k_x}) = 1)
        \end{equation}
        と表せる。
        このとき$\gcd(n, p^{k_x}) = 1$より
        ある$l, m \in \Z$が存在して
        $nl + p^{k_x}m = 1$が成り立つから
        \begin{equation}
            N \ni lx = \frac{1 - p^{k_x}m}{p^{k_x}} + \Z
                = \frac{1}{p^{k_x}} + \Z
        \end{equation}
        が成り立つ。
        よって
        $\langle 1/p^{k_x} \rangle / \Z \subset N$が成り立つ。
        ここで、
        $N$が$M$の真部分加群であることから
        集合$\{ k_x \mid x \in N \}$には
        最大値$k_0 \in \Z_{\ge 0}$が存在する。
        $N \subset \bigcup_{x \in N} (\langle 1/p^{k_x} \rangle / \Z)
            \subset \langle 1/p^{k_0} \rangle / \Z$だから
        $N = \langle 1/p^{k_0} \rangle / \Z$となる。
    \end{innerproof}

    $M$がアルティンであることを示す。
    各$x \in M$に対し$\langle x \rangle_M$で
    $\Z$上$x$により生成された$\Z[1/p]$の巡回部分加群を表すことにする。
    $M$の$\Z$-真部分加群の任意の減少列
    \begin{equation}
        \left\langle \frac{1}{p^{n_0}} \right\rangle_M
            \supset \left\langle \frac{1}{p^{n_1}} \right\rangle_M
            \supset \cdots
            \quad
            (n_i \in \Z_{\ge 0})
    \end{equation}
    は$n_i$が非負整数の減少列であることから停留的である。
    したがって$M$はアルティンである。
    $M$がネーターでないことは、
    $M$の$\Z$-真部分加群の増大列
    \begin{equation}
        \left\langle \frac{1}{p} \right\rangle_M
            \subset \left\langle \frac{1}{p^2} \right\rangle_M
            \subset \cdots
    \end{equation}
    が停留的でないことから従う。
    以上より$M$はアルティン加群だがネーター加群でないことが示せた。
\end{answer}

\begin{problem}[代数学II 5.71]
    可換環$R$上の1変数多項式環$R[X]$が
    ネーター環ならば、$R$はネーター環か?
\end{problem}

\begin{answer}
    $R$がネーター環となることの対偶を示す。
    $R$がネーター環でないとすると、
    昇鎖条件をみたさない$R$のイデアルの増大列
    \begin{equation}
        I_1 \subset I_2 \subset \cdots
    \end{equation}
    がとれる。
    そこで
    \begin{equation}
        J_n \coloneqq \left\{
            \sum_{i = 0}^r a_i X^i \in R[X] \mid a_i \in I_n
        \right\}
        \quad (n = 1, 2, \dots)
    \end{equation}
    とおけば、
    \begin{equation}
        J_1 \subset J_2 \subset \cdots
    \end{equation}
    は昇鎖条件をみたさない$R[X]$のイデアルの増大列となる。
    よって$R[X]$はネーター環でない。
    これで対偶がいえた。
\end{answer}

\begin{problem}[代数学II 5.74]
    \label[problem]{problem:algebra2-5.74}
    可換ネーター環の部分環は常にネーターか?
    正しければ証明を、誤りなら反例を与えよ。
\end{problem}

\begin{answer}[解法1.]
可換ネーター環の部分環はネーターとは限らない。
実際、$\C[X, Y]$は
$\C$が体ゆえにネーター環であることと
Hilbert の基底定理よりネーター環であるが、
部分環
\begin{equation}
    \C\langle \{ XY, XY^2, XY^3, \dots \} \rangle
\end{equation}
は昇鎖条件をみたさないイデアルの増大列
\begin{equation}
    (XY) \subsetneq (XY, XY^2) \subsetneq (XY, XY^2, XY^3) \subsetneq \cdots
\end{equation}
をもつからネーター環ではない。
\end{answer}

\begin{answer}[解法2.]
    無限変数多項式環$\C[X_1, X_2, \dots]$は
    ネーター環ではないが、
    これは整域だから商体に埋め込める。
    商体は体ゆえにネーター環だからこれが反例になっている。
\end{answer}

\begin{problem}[代数学II 5.75]
    \label[problem]{problem:algebra2-5.75}
    可換アルティン環において素イデアルは極大イデアルになることを示せ。
\end{problem}

\begin{answer}
$R$を可換アルティン環、
$P$を$R$の素イデアルとする。
$B \coloneqq A / P$とおくと、
\cref{thm:exact-sequence-and-finiteness}
より$B$はアルティン環であり、
さらに$P$: 素イデアルより$B$は整域でもある。
$P$が極大イデアルであることをいうには、
$B$が体であることを示せばよい。
そこで$x \in B - \{ 0 \}$とする。
降鎖条件より、ある$n \in \Z_{\ge 0}$が存在して
$(x^n) = (x^{n + 1})$が成り立つ。
したがって、ある$y \in B$が存在して
$x^n = yx^{n + 1}$となる。
いま$B$は整域だから$x^n$を打ち消して
$1 = xy$が成り立つ。
よって$x \in B^\times$となる。
$B$は零環でないから
これで$B^\times = B - \{ 0 \}$がいえた。
よって$B = A / P$は体であり、
したがって$P$は極大イデアルである。
\end{answer}

\begin{problem}[代数学II 5.76]
    \label[problem]{problem:algebra2-5.76}
    可換アルティン環は極大イデアルを高々有限個しか持たないことを示せ。
\end{problem}

\begin{answer}
    \TODO{}
    cf. \cite[p.89]{AM69}
\end{answer}

\subsection{Problem set 6}

\begin{problem}[代数学II 6.79]
    $A, B$を左ネーター環とするとき
    直積環$A \times B$も左ネーター環になることを示せ。
\end{problem}

\begin{answer}
    $A \times B$の左イデアルは
    $A, B$の左イデアルの直積の形に書けることに注意する。
    $A \times B$の左イデアルの昇鎖
    \begin{equation}
        I_1 \times J_1 \subset I_2 \times J_2 \subset \cdots
    \end{equation}
    が与えられたとする。このとき、とくに
    \begin{equation}
        \begin{cases}
            I_1 \subset I_2 \subset \cdots \\
            J_1 \subset J_2 \subset \cdots
        \end{cases}
    \end{equation}
    が成り立つから、$A, B$の左ネーター性より
    \begin{align}
        &\exists m \in \Z_{\ge 1}
            \quad \text{s.t.} \quad
            I_m = I_{m + 1} = \cdots \\
        &\exists n \in \Z_{\ge 1}
            \quad \text{s.t.} \quad
            J_n = J_{n + 1} = \cdots
    \end{align}
    が成り立つ。よって$k \coloneqq \max\{m, n\}$とおけば
    \begin{equation}
        I_k \times J_k = I_{k + 1} \times J_{k + 1} = \cdots
    \end{equation}
    となる。したがって$A \times B$は左ネーター環である。
\end{answer}

\begin{problem}[代数学II 6.82]
    $A$を左アルティン環とすると、
    ある正整数$n$が存在して$J(A)^n = 0$となることを示せ。
\end{problem}

\begin{answer}
    \TODO{本当に正しい?}

    \cite[p.172]{AF92}

    $A$は左アルティン環だから、降鎖
    $J(A) \supset J(A)^2 \supset \dots$
    は停留的である。すなわちある正整数$n$が存在して
    $J(A)^n = J(A)^{n + 1} = \dots$となる。
    この$n$に対し$J(A)^n = 0$となることを示せばよい。
    $J(A)^n \neq 0$であると仮定して矛盾を導く。
    集合$P$を
    \begin{equation}
        P \coloneqq \{
            I \subset A \colon \text{左イデアル}
            \mid
            I \subset J(A)^n, \; IJ(A)^n \neq 0
        \}
    \end{equation}
    と定める。
    $J(A)^n \in P$より$P \neq \emptyset$だから、
    $A$のアルティン性より$P$は包含に関する極小元$I_0$をもつ。

    ここで$I_0$は単項イデアルであることを示す。
    $I_0 J(A)^n \neq 0$より
    ある$x_0 \in I_0 - \{ 0 \}$であって
    $x_0 J(A)^n \neq 0$なるものが存在する。
    このとき$(x_0) J(A)^n \neq 0$である。
    $(x_0) \subset I_0 \subset J(A)^n$であることとあわせて
    $(x_0) \in P$だから、$I_0$の極小性より
    $(x_0) = I_0$が従い、
    $I_0$は単項イデアルであることがいえた。

    さて、
    $I_0 J(A)^n \subset J(A)^n$かつ
    $I_0 J(A)^n J(A)^n = I_0 J(A)^n \neq 0$より
    $I_0 J(A)^n \in P$であり、
    また$I_0 J(A)^n \subset I_0$だから、
    $I_0$の極小性より
    $I_0 J(A)^n = I_0$である。
    したがって
    $I_0 J(A)^n = I_0 = (x_0)$は
    $x_0$で生成される (有限生成) $A$-加群であって
    $I_0 J(A)^n J(A) = I_0 J(A)^n$をみたすから、
    Nakayama の補題より$I_0 J(A)^n = 0$である。
    これは$I_0 J(A)^n = I_0 \neq 0$に矛盾。
    背理法より$J(A)^n = 0$である。
\end{answer}

\begin{problem}[代数学II 6.84]
    \label[problem]{problem:algebra2-6.84}
    可換アルティン環はネーター環になることを示せ。
\end{problem}

\begin{answer}
    \TODO{cf. \cite[p.90]{AM69}}
\end{answer}

\begin{problem}[代数学II 6.86](行列式の技巧\footnote{
    cf. \cite{松村00}
})
    \label[problem]{problem:algebra2-6.86}
    $R$を可換環とする。$n$を正整数として
    $M$を$n$個の元$v_1, \dots, v_n$で生成される$R$-加群とする。
    $\varphi \in \End_R(M)$とし、
    各$1 \le i, j \in n$に対して$a_{i,j} \in R$を
    \begin{equation}
        \varphi(v_i) = \sum_{j = 1}^n a_{i,j} v_j
    \end{equation}
    をみたすようにとる。
    $X$を不定元、$\delta_{i, j}$を Kronecker のデルタとして
    行列$L = (\delta_{i, j}X - a_{i,j})_{1 \le i, j\ \le n} \in M_n(R[X])$
    を考える。そこで$d(X) = \det L \in R[X]$とすると
    任意の$v \in M$に対して$d(\varphi) v = 0$となることを示せ。
\end{problem}

\begin{answer}
    $v_i \; (1 \le i \le n)$は$M$を$R$上生成するから
    $d(\varphi) v_i = 0 \; (1 \le i \le n)$をいえばよい。
    $M$に$X$を$\varphi$として作用させることで$R[X]$-加群の構造を入れる。
    これにより$M^n$に$M_n(R[X])$-加群の構造が入り、
    \begin{equation}
        v \coloneqq \begin{bmatrix}
            v_1 \\
            \vdots \\
            v_n
        \end{bmatrix}
        \in M^n
    \end{equation}
    とおくと
    \begin{equation}
        Lv = 0
    \end{equation}
    が成り立つ。
    左から$L$の余因子行列を掛けて
    \begin{equation}
        \det(L) I_n v = 0
    \end{equation}
    したがって$d(\varphi) v_i = d(X) v_i = 0 \; (1 \le i \le n)$が成り立つ。
\end{answer}

\begin{problem}[代数学II 6.87]
    \label[problem]{problem:algebra2-6.87}
    $R$を可換環、$I \subset R$をイデアル、$M$を有限生成$R$-加群とする。
    $IM = M$が成り立つならば
    ある$x \in I$であって$1 + x \in \Ann_R(M)$をみたすものが存在することを示せ。
\end{problem}

\begin{answer}
    \cref{problem:algebra2-6.86}の記号を引き続き用いて
    $\varphi = \id_M$の場合を考える。
    $IM = M$の仮定から
    $a_{i, j}$らは$I$の元にとれる。
    ここで行列式の定義より
    \begin{equation}
        d(X) = \det(L) = \sum_{\sigma \in \calS_n}
            \sgn(\sigma)
            (\delta_{\sigma(1) 1} X - a_{\sigma(1) 1})
    \end{equation}
    だから、右辺の形より$d(X)$の最高次以外の係数はすべて$I$に属する。
    したがって
    \begin{equation}
        d(X) = X^n + b_1 X^{n - 1} + \cdots + b_{n - 1} X + b_n
            \quad
            (b_i \in I, \; 1 \le i \le n)
    \end{equation}
    と表せる。
    このことと$d(\id_M)v = d(X)v = 0 \; (v \in M)$より
    \begin{equation}
        (1 + b_1 + \dots + b_n) v = 0
            \quad
            (v \in M)
    \end{equation}
    が成り立つ。
    したがって$x \coloneqq b_1 + \dots + b_n$とおけば
    $x \in I, \; 1 + x \in \Ann_R(M)$が成り立つ。
\end{answer}

\begin{problem}[代数学II 6.88]
    $R$を可換環、$M$を有限生成$R$-加群とする。
    $f \in \End_R(M)$が全射ならば$M$の自己同型になることを示せ。
\end{problem}

\begin{answer}
    $f$の単射性をいえばよい。
    $M$に$X$を$f$として作用させることで$R[X]$-加群の構造を入れる。
    $M$は$R$-加群として有限生成だから$R[X]$-加群としても有限生成である。
    また、$f$の全射性より$R[X]$のイデアル$(X)$は
    $(X)M = M$をみたす。
    そこで\cref{problem:algebra2-6.87}より、
    ある$F \in R[X]$であって$1 + F(X)X \in \Ann_{R[X]}(M)$をみたすものが存在する。
    したがって
    \begin{align}
        \id_M(v) + F(f) \circ f(v) &= 0
            \quad
            (v \in M) \\
        \therefore
            - F(f) \circ f &= \id_M
    \end{align}
    が成り立つから、$f$は左逆写像$-F(f)$をもつ。
    よって$f$は単射であり、
    したがって$f$は自己同型である。
\end{answer}

\begin{problem}[代数学II 6.89]
    極大両側イデアルは左原始イデアルになることを示せ。
\end{problem}

\begin{answer}
    \TODO{}
\end{answer}


\end{document}
\documentclass[report]{jlreq}
\usepackage{global}
\usepackage{./local}
\subfiletrue
%\makeindex
\begin{document}



% ============================================================
%
% ============================================================
\chapter{加群のテンソル積}

% ------------------------------------------------------------
%
% ------------------------------------------------------------
\section{可換環上の加群のテンソル積}

テンソル積を定義する。まずは可換環上の加群に限って考える。

\begin{definition}[双線型写像]
    $A$を環、$M, N, L$を$A$-加群とする。
    写像$f \colon M \times N \to L$が
    \term{$A$-双線型写像}[$A$-bilinear map]
    {双線型写像}[そうせんけいしゃぞう]
    であるとは、
    各$x_1, x_2 \in M, \; y_1, y_2 \in N, \; a_1, a_2 \in A$に対し
    \begin{alignat}{1}
        f(x_1, a_1 y_1 + a_2 y_2) &= a_1 f(x_1, y_1) + a_2 f(x_1, y_2) \\
        f(a_1 x_1 + a_2 x_2, y_1) &= a_1 f(x_1, y_1) + a_2 f(x_2, y_1)
    \end{alignat}
    が成り立つことをいう。
\end{definition}

\begin{definition}[圏論的テンソル積]
    $R$を可換環、$M, N$を$R$-加群とする。
    組$(Z, \varphi)$が$M, N$の
    \term{圏論的テンソル積}[categorical tensor product]
    {圏論的テンソル積}[けんろんてきてんそるせき]
    であるとは、次が成り立つことをいう:
    \begin{description}
        \item[(T1)] $Z$は$R$-加群である。
        \item[(T2)] $\varphi$は$R$-双線型写像$M \times N \to Z$である。
        \item[(T3)] (普遍性) 次が成り立つ:
            \begin{alignat}{1}
                &\forall \; L \colon \text{ $R$-加群} \\
                &\forall \; f \colon M \times N \to L
                    \colon \text{ $R$-双線型写像} \\
                &\exists! \; \wb{f} \colon Z \to L
                    \colon \text{ $R$-加群準同型}
                    \quad \text{s.t.} \quad \\
                &\quad \begin{tikzcd}[ampersand replacement=\&]
                    \& M \times N
                        \ar{ld}[swap]{\varphi}
                        \ar{rd}{f} \\
                    Z
                        \ar[dashed]{rr}[swap]{\wb{f}}
                        \& \& L
                \end{tikzcd}
            \end{alignat}
    \end{description}
\end{definition}

\begin{remark}
    \TODO{誘導された準同型の単射性の確認に関する注意を述べたい}
\end{remark}

\begin{theorem}[圏論的テンソル積の一意性]
    $R$を可換環、
    $M, N$を$R$-加群、
    $(Z, \varphi), (Z', \varphi')$を$M, N$の圏論的テンソル積とする。
    このとき、次の$\lMod{R}$の図式を可換にする
    $R$-加群準同型$i$が一意に存在する:
    \begin{equation}
        \begin{tikzcd}
            & M \times N
                \ar{ld}[swap]{\varphi}
                \ar{rd}{\varphi'} \\
            Z
                \ar[dashed]{rr}[swap]{i}
                & & Z'
        \end{tikzcd}
    \end{equation}
\end{theorem}

\begin{proof}
    \TODO{}
\end{proof}

可換環上の加群のテンソル積を具体的に構成する。

\begin{definition}[可換環上の加群のテンソル積の構成]
    \idxsym{tensor product}{$M \otimes_R N$}{可換環上の加群のテンソル積}
    $R$を可換環、$M, N$を$R$-加群とする。

    \TODO{}

    商加群
    \begin{equation}
        M \otimes_R N \coloneqq [M \times N] / Bl
    \end{equation}
    を$M$と$N$の$R$上の
    \term{テンソル積}[tensor product]{テンソル積}[てんそるせき]
    といい、
    写像
    \begin{equation}
        \otimes \colon M \times N \to M \otimes_R N,
        \quad
        (m, n) \mapsto p(m, n)
    \end{equation}
    を$M \otimes_R N$の
    \term{標準射影}{標準射影!テンソル積の---}[ひょうじゅんしゃえい]という。
    $(M \otimes_R N, \otimes)$は
    $M, N$の圏論的テンソル積になっている (このあと示す)。
\end{definition}

\begin{proof}
    \TODO{}
\end{proof}

\begin{theorem}[有限生成加群のテンソル積の生成系]
    $R$を可換環、$M, N$を$R$-加群、
    $S \subset M, \; T \subset N$を部分$R$-加群、
    $M = \langle S \rangle, \; N = \langle T \rangle$
    とする。
    このとき
    \begin{equation}
        S \otimes T \coloneqq \{
            s \otimes t \in M \otimes_R N
            \mid
            s \in S, t \in T
        \}
    \end{equation}
    とおくと$\langle S \otimes T \rangle = M \otimes_R N$が成り立つ。
    とくに$M, N$が有限生成ならば$M \otimes_R N$も有限生成である。
\end{theorem}

\begin{proof}
    テンソル積の定義より、$M \otimes_R N$の元は
    \begin{equation}
        \sum_{i = 1}^n m_i \otimes n_i
            \quad
            (m_i \in M, \; n_i \in N)
    \end{equation}
    の形に書けるが、
    いま$M = \langle S \rangle, \; N = \langle T \rangle$だから、これは
    \begin{equation}
        \sum_{i = 1}^n
            \left(
                \sum_{j = 1}^k r_{j} s_{j}
            \right)
            \otimes
            \left(
                \sum_{j' = 1}^{k'} r_{j'} t_{j'}
            \right)
            \quad
            (r_j, r_{j'} \in R, \; s_j \in S, \; t_{j'} \in T)
    \end{equation}
    の形に書ける。右辺を整理して
    \begin{equation}
        \sum_{i, j, j'} r_{j} r_{j'} s_{j} \otimes t_{j'}
            \in \langle S \otimes T \rangle
    \end{equation}
    を得る。
\end{proof}

\begin{theorem}[自由加群のテンソル積の基底]
    \label[theorem]{thm:basis-of-tensor-product-of-free-modules}
    $R$を可換環、$M, N$を自由$R$-加群、
    $\{ v_i \}_{i \in I}$を$M$の基底、
    $\{ w_j \}_{j \in J}$を$N$の基底とする。
    このとき
    \begin{equation}
        B \coloneqq \{ v_i \otimes w_j \mid i \in I, j \in J \}
    \end{equation}
    は$M \otimes_R N$の$R$上の基底である。
\end{theorem}

\begin{proof}
    $\langle B \rangle = M \otimes_R N$となるのは上の定理よりわかる。
    あとは$B$が$R$上1次独立であることをいえばよく、
    そのためには$B$を何らかの$R$-加群準同型で写した像が
    $R$上1次独立であることをいえばよい。
    そこで$R$-加群準同型
    $\Psi \colon M \otimes_R N \to R^{\otimes (I \times J)}$
    を
    \begin{equation}
        \Psi\left(
            \left(
                \fsum_{i \in I} a_i v_i
            \right)
            \otimes
            \left(
                \fsum_{j \in J} b_j w_j
            \right)
        \right)
            \coloneqq (a_i b_j)_{(i, j) \in I \times J}
    \end{equation}
    で定める。
    ただし、右辺が有限項を除いて$0$であることは左辺が有限和であることから明らかで、
    また$R$-双線型性も明らか。
    すると
    \begin{equation}
        \Psi(v_i \otimes w_j) = (c_{pq})_{(p, q) \in I \times J},
            \quad
            c_{pq} = \begin{cases}
                1 & (p, q) = (i, j) \\
                0 & \text{otherwise}
            \end{cases}
    \end{equation}
    より$\{ \Psi(v_i \otimes w_j) \mid i \in I, j \in J \}$
    は$R$上1次独立である。
    よって$B$も$R$上1次独立である。
\end{proof}

\begin{corollary}[テンソル積の可換性]
    \TODO{}
\end{corollary}

\begin{corollary}[テンソル積の結合性]
    \TODO{}
\end{corollary}

\begin{definition}[代数のテンソル積]
    $R$を可換環、$A, B$を$R$-代数とする。
    $A, B$は$R$-加群とみなせるから、
    テンソル積加群$A \otimes_R B$が考えられる。

    \TODO{乗法を定める}
\end{definition}

% ------------------------------------------------------------
%
% ------------------------------------------------------------
\section{非可換環上の加群のテンソル積}

テンソル積の概念を非可換環上の加群まで一般化しよう。

\begin{definition}[$A$-平衡$R$-双線型写像]
    $R$を可換環、$A$を$R$-代数、
    $M$を右$A$-加群、$N$を左$A$-加群、$L$を$R$-加群とする。
    写像$f \colon M \times N \to L$が
    \term{$A$-平衡$R$-双線型写像}[$A$-balanced $R$-bilinear map]
    {平衡双線型写像}[へいこうそうせんけいしゃぞう]
    であるとは、次が成り立つことをいう:
    \begin{enumerate}
        \item $f$は$R$-双線型写像である。
        \item (平衡性) $m \in M, n \in N, a \in A$に対し
            \begin{equation}
                f(ma, n) = f(m, an)
            \end{equation}
            が成り立つ。
    \end{enumerate}
\end{definition}

非可換環上の加群のテンソル積を具体的に構成する。

\begin{definition}[非可換環上の加群のテンソル積の構成]
    \TODO{}
\end{definition}

上の構成は次の意味での普遍性をみたすが、
実はもう少し広い意味での普遍性が成り立つことを後で示す。

\begin{theorem}[非可換環上の加群のテンソル積の普遍性]
    $R$を可換環、$A$を$R$-代数、
    $M$を右$A$-加群、$N$を左$A$-加群とする。
    このとき次が成り立つ:
    \begin{alignat}{1}
        &\forall \; L \colon \text{ $R$-加群} \\
        &\forall \; f \colon M \times N \to L
            \colon \text{ $A$-平衡$R$-双線型写像} \\
        &\exists! \; \wb{f} \colon M \otimes_A N \to L
            \colon \text{ $R$-加群準同型}
            \quad \text{s.t.} \quad \\
        &\quad \begin{tikzcd}[ampersand replacement=\&]
            \& M \times N
                \ar{ld}[swap]{\otimes}
                \ar{rd}{f} \\
            M \otimes_A N
                \ar[dashed]{rr}[swap]{\wb{f}}
                \& \& L
        \end{tikzcd}
    \end{alignat}
\end{theorem}

\begin{proof}
    \TODO{}
\end{proof}

\begin{definition}[左$B$-線型$A$-平衡$\Z$-双線型写像]
    $A, B$を環、
    $M$を$(B, A)$-両側加群、
    $N$を左$A$-加群、$L$を左$B$-加群とする。
    写像$f \colon M \times N \to L$が
    \term{左$B$-線型$A$-平衡$\Z$-双線型写像}[left $B$-linear $A$-balanced $\Z$-bilinear map]
    {左線型平衡$\Z$-双線型写像}[ひだりせんけいへいこうZそうせんけいしゃぞう]
    であるとは、次が成り立つことをいう:
    \begin{enumerate}
        \item $f$は$A$-平衡$\Z$-双線型写像である。
        \item (左$B$-線型性) $m \in M, n \in N, b \in B$に対し
            \begin{equation}
                f(bm, n) = bf(m, n)
            \end{equation}
            が成り立つ。
    \end{enumerate}
\end{definition}

\begin{definition}[テンソル積への左作用]
    $A, B$を環、
    $M$を$(B, A)$-両側加群、
    $N$を左$A$-加群とする。
    このとき、$M \otimes_A N$に
    左$B$-加群の構造を
    \begin{equation}
        b (m \otimes n) \coloneqq (bm) \otimes n
        \quad
        (b \in B)
    \end{equation}
    で定めることができる。
\end{definition}

\begin{proof}
    \TODO{}
\end{proof}

\begin{theorem}[$\Z$の場合さえ考えればよいということ]
    $R$を可換環、$A$を$R$-代数、
    $M$を右$A$-加群、$N$を左$A$-加群とし、
    \begin{itemize}
        \item $M \otimes_A^1 N$: $A$を$R$-代数とみたときのテンソル積
        \item $M \otimes_A^2 N$: $A$を$\Z$-代数とみたときのテンソル積
    \end{itemize}
    とおく。
    このとき次が成り立つ:
    \begin{alignat}{1}
        &\exists! \; \iota
            \colon M \otimes_A^1 N \to M \otimes_A^2 N \colon \text{ $\Z$-加群の同型}
            \quad \text{s.t.} \quad \\
        &\quad \begin{tikzcd}[ampersand replacement=\&]
            \& M \times N
                \ar{ld}[swap]{\otimes^1}
                \ar{rd}{\otimes^2} \\
            M \otimes_A^1 N
                \ar[dashed]{rr}[swap]{\iota}{\cong}
                \& \& M \otimes_A^2 N
        \end{tikzcd}
    \end{alignat}
    ただし$\otimes^1, \otimes^2$は標準射影である。
\end{theorem}

\begin{proof}
    $\iota$の逆写像にあたるものを考える。
    $\otimes^1$は$A$-平衡$R$-双線型写像だから、
    とくに$A$-平衡$\Z$-双線型写像でもある。
    したがってテンソル積$M \otimes_A^2 N$の普遍性より
    \begin{equation}
        \begin{tikzcd}[ampersand replacement=\&]
            \& M \times N
                \ar{ld}[swap]{\otimes^2}
                \ar{rd}{\otimes^1} \\
            M \otimes_A^2 N
                \ar[dashed]{rr}[swap]{\wb{\otimes^1}}
                \& \& M \otimes_A^1 N
        \end{tikzcd}
    \end{equation}
    を可換にする$\Z$-加群準同型$\wb{\otimes^1}$が
    ただひとつ存在する。

    $A$の$R$-代数としての構造を定める環準同型を
    $\varphi \colon R \to Z(A)$とおく。
    このとき、$M$に$(R, A)$-両側加群の構造を
    \begin{equation}
        rm \coloneqq m \varphi(r)
        \quad
        (m \in M, \; r \in R)
    \end{equation}
    で定義できる。
    これによりテンソル積$M \otimes_A^2 N$への$R$の左作用を定めて
    左$R$-加群の構造を入れる\TODO{どういうこと?}。

    \TODO{}
\end{proof}

\begin{theorem}[テンソル積の普遍性 (最終形)]
    $A, B$を環、
    $M$を$(B, A)$-両側加群、
    $N$を左$A$-加群とする。
    このとき次が成り立つ:
    \begin{alignat}{1}
        &\forall \; L \colon \text{ 左$B$-加群} \\
        &\forall \; f \colon M \times N \to L
            \colon \text{ 左$B$-線型$A$-平衡$\Z$-双線型写像} \\
        &\exists! \; \wb{f} \colon M \otimes_A N \to L
            \colon \text{ $B$-加群準同型}
            \quad \text{s.t.} \quad \\
        &\quad \begin{tikzcd}[ampersand replacement=\&]
            \& M \times N
                \ar{ld}[swap]{\otimes}
                \ar{rd}{f} \\
            M \otimes_A N
                \ar[dashed]{rr}[swap]{\wb{f}}
                \& \& L
        \end{tikzcd}
    \end{alignat}
\end{theorem}

\begin{proof}
    \TODO{}
\end{proof}

テンソル積は直和との間の分配律をみたす。

\begin{theorem}[テンソル積の分配律]
    \label[theorem]{thm:distribution-law-of-tensor-product}
    $A, B$を環、
    \begin{enumerate}
        \item $\{ M_i \}_{i \in I}$を$(B, A)$-両側加群の族、
            $N$を$A$-加群、
            $\iota_i \colon M_i \hookrightarrow \bigoplus_{j \in I} M_j$
            を標準包含とする。
            このとき
            \begin{equation}
                \bigoplus_{i \in I} (\iota_i \otimes \id_N)
                    \colon
                    \bigoplus_{i \in I} (M_i \otimes_A N)
                    \overset{\sim}{\to}
                    \left(\bigoplus_{i \in I} M_i\right) \otimes_A N
            \end{equation}
            は$B$-加群の同型となる。
        \item $M$を$(B, A)$-両側加群、
            $\{ N_i \}_{i \in I}$を$A$-加群の族、
            $\iota_i \colon N_i \hookrightarrow \bigoplus_{j \in I} N_j$
            を標準包含とする。
            このとき
            \begin{equation}
                \bigoplus_{i \in I} (\id_M \otimes \iota_i)
                    \colon
                    \bigoplus_{i \in I} (M \otimes_A N_i)
                    \overset{\sim}{\to}
                    M \otimes_A \left(\bigoplus_{i \in I} N_i\right)
            \end{equation}
            は$B$-加群の同型となる。
    \end{enumerate}
\end{theorem}

\begin{proof}
    (1)についてのみ示す。(2)も同様にして示せる。
    左$B$-線型$A$-平衡$\Z$-双線型写像
    $\Phi \colon \left(\bigoplus_{i \in I} M_i\right) \times N
        \to \bigoplus_{i \in I} (M_i \otimes_A N)$
    を
    \begin{equation}
        \Phi((x_i)_{i \in I}, y)
            \coloneqq (x_i \otimes y)_{i \in I}
    \end{equation}
    で定めることができる。
    よって、$B$-線型写像
    \begin{equation}
        \wb{\Phi}
            \colon
            \left(\bigoplus_{i \in I} M_i\right) \otimes_A N
            \to
            \bigoplus_{i \in I} (M_i \otimes_A N),
            \quad
            (x_i)_{i \in I} \otimes y
            \mapsto
            (x_i \otimes y)_{i \in I}
    \end{equation}
    が誘導される。
    $\wb{\Phi}$が
    $\bigoplus_{i \in I} (\iota_i \otimes \id_N)$の逆写像であることを示す。
    右逆写像であることは
    \begin{alignat}{1}
        \left(
            \bigoplus_{i \in I} (\iota_i \otimes \id_N)
        \right)
            \circ \wb{\Phi}
            ((x_i)_{i \in I} \otimes y)
            &= \left(
                \bigoplus_{i \in I} (\iota_i \otimes \id_N)
            \right)
                ((x_i \otimes y)_{i \in I}) \\
            &= \fsum_{i \in I}
                (\iota_i \otimes \id_N)(x_i \otimes y) \\
            &= \fsum_{i \in I}
                ((z_{i; j})_{j \in I} \otimes y)
                \quad
                \text{ただし}
                \quad
                z_{i; j} \coloneqq \begin{cases}
                    x_i & (j = i) \\
                    0 & (j \neq i)
                \end{cases} \\
            &= \left(
                \fsum_{i \in I} (z_{i; j})_{j \in I}
            \right) \otimes y \\
            &= (x_i)_{i \in I} \otimes y
    \end{alignat}
    より従う。
    左逆写像であることも同様にしてわかる。
    よって$\bigoplus_{i \in I} (\iota_i \otimes \id_N)$は
    $B$-加群の同型である。
\end{proof}



% ============================================================
%
% ============================================================
\chapter{加群の圏}

この章では加群自体というより加群の圏について考える。


% ------------------------------------------------------------
%
% ------------------------------------------------------------
\section{加群の圏と関手}

加群の圏とそれにまつわる用語を導入する。

\begin{definition}[加群の圏]
    \TODO{}
\end{definition}

\begin{definition}[共変関手]
    $A, B$を環とする。
    $T \colon \lMod{A} \to \lMod{B}$が
    \term{共変関手}[covariant functor]{共変関手}[きょうへんかんしゅ]
    であるとは、
    \begin{itemize}
        \item 写像$T \colon \Ob(\lMod{A}) \to \Ob(\lMod{B})$
        \item 写像$T \colon \Ar(\lMod{A}) \to \Ar(\lMod{B})$
    \end{itemize}
    が定まっていて
    \begin{enumerate}
        \item 各$M, N \in \Ob(\lMod{A})$に対し
            \begin{equation}
                T(\Hom_A(M, N)) \subset \Hom_B(T(M), T(N))
            \end{equation}
        \item $T(\id_M) = \id_{T(M)} \quad (M \in \Ob(\lMod{A}))$
        \item 各$M, N, L \in \Ob(\lMod{A})$と
            $f \in \Hom_A(M, N), \; g \in \Hom_A(N, L)$に対し
            \begin{equation}
                T(g) \circ T(f) = T(g \circ f)
            \end{equation}
    \end{enumerate}
    が成り立つことをいう。
\end{definition}

\begin{definition}[反変関手]
    $A, B$を環とする。
    $T \colon \lMod{A} \to \lMod{B}$が
    \term{反変関手}[contravariant functor]{反変関手}[はんぺんかんしゅ]
    であるとは、
    \begin{itemize}
        \item 写像$T \colon \Ob(\lMod{A}) \to \Ob(\lMod{B})$
        \item 写像$T \colon \Ar(\lMod{A}) \to \Ar(\lMod{B})$
    \end{itemize}
    が定まっていて
    \begin{enumerate}
        \item 各$M, N \in \Ob(\lMod{A})$に対し
            \begin{equation}
                T(\Hom_A(M, N)) \subset \Hom_B(T(N), T(M))
            \end{equation}
        \item $T(\id_M) = \id_{T(M)} \quad (M \in \Ob(\lMod{A}))$
        \item 各$M, N, L \in \Ob(\lMod{A})$と
            $f \in \Hom_A(M, N), \; g \in \Hom_A(N, L)$に対し
            \begin{equation}
                T(f) \circ T(g) = T(g \circ f)
            \end{equation}
    \end{enumerate}
    が成り立つことをいう。
\end{definition}

\begin{definition}[押し出しと引き戻し]
    $R$を可換環、$A$を$R$-代数、
    $M, N, L$を$A$-加群とする。
    \begin{enumerate}
        \item $f \in \Hom_A(N, L)$とする。
            $R$-代数準同型$f_\sharp$を
            \begin{equation}
                f_\sharp \colon \Hom_A(M, N) \to \Hom_A(M, L),
                \quad
                \varphi \mapsto f \circ \varphi
            \end{equation}
            で定める。
            $f_\sharp$を
            $f$による\term{押し出し}[pushout]{押し出し}[おしだし]という。
        \item $h \in \Hom_A(L, M)$とする。
            $R$-代数準同型$h^\sharp$を
            \begin{equation}
                h^\sharp \colon \Hom_A(M, N) \to \Hom_A(L, N),
                \quad
                \varphi \mapsto \varphi \circ h
            \end{equation}
            で定める。
            $h^\sharp$を
            $h$による\term{引き戻し}[pullback]{引き戻し}[ひきもどし]という。
    \end{enumerate}
\end{definition}

テンソル積や$\Hom$をとる操作は共変/反変関手の一例である。

\begin{theorem}[テンソル関手]
    $A, B$を環、$M$を$(B, A)$-両側加群とする。
    このとき、関手$M \otimes_A \Box$を
    \begin{equation}
        \begin{tikzcd}
            \lMod{A} \ar{r}
                & \lMod{B}
                & \Hom_A(X, Y) \ar{r}
                & \Hom_B(M \otimes_A X, M \otimes_A Y) \\
            X \ar[mapsto]{r}
                & M \otimes_A X
                & f \ar[mapsto]{r}
                & \id_M \otimes f
        \end{tikzcd}
    \end{equation}
    で定めることができる。
\end{theorem}

\begin{proof}
    \TODO{}
\end{proof}

\cref{definition:set-of-module-homomorphisms}でみたように
加群準同型全体の集合には加群の構造が入るのであった。
このことを利用して次のような関手を定めることができる。

\begin{theorem}[共変ホム関手]
    $R$を可換環、
    $A, B$を$R$-代数、$M$を$(A, B)$-両側加群とする。
    このとき、関手$\Hom_A(M, \Box)$を
    \begin{equation}
        \begin{tikzcd}
            \lMod{A} \ar{r}
                & \lMod{B}
                & \Hom_A(X, Y) \ar{r}
                & \Hom_B(\Hom_A(M, X), \Hom_A(M, Y)) \\
            X \ar[mapsto]{r}
                & \Hom_A(M, X)
                & f \ar[mapsto]{r}
                & f_\sharp
        \end{tikzcd}
    \end{equation}
    で定めることができる。
\end{theorem}

\begin{remark}
    $M$が単に左$A$-加群の場合は、
    $M$を$(A, \Z)$-両側加群とみなせば定理を適用できる。
\end{remark}

\begin{proof}
    \TODO{}
\end{proof}

\begin{theorem}[反変ホム関手]
    $A, B$を環、$M$を$(A, B)$-両側加群とする。
    このとき、反変関手$\Hom_A(\Box, M)$を
    \begin{equation}
        \begin{tikzcd}
            \lMod{A} \ar{r}
                & \lMod{B}
                & \Hom_A(X, Y) \ar{r}
                & \Hom_B(\Hom_A(Y, M), \Hom_A(X, M)) \\
            X \ar[mapsto]{r}
                & \Hom_A(X, M)
                & f \ar[mapsto]{r}
                & f^\sharp
        \end{tikzcd}
    \end{equation}
    で定めることができる。
\end{theorem}

\begin{proof}
    \TODO{}
\end{proof}



% ------------------------------------------------------------
%
% ------------------------------------------------------------
\section{自然変換}

\begin{definition}[自然変換]
    \TODO{}
\end{definition}

\begin{definition}[関手の同型]
    \idxsym{isomorphic covariant functors}{$T \cong S$}{共変関手$T, S$の同型}
    $A, B$を環、
    $T, S \colon \lMod{A} \to \lMod{B}$を共変関手とする。
    $T, S$が\term{同型}[isomorphic]{同型}[どうけい]であるとは、
    \begin{enumerate}
        \item 任意の$A$-加群$M$に対し
            $B$-加群の同型
            $\tau_M \colon T(M) \overset{\sim}{\to} S(M)$が定まっている。
        \item 任意の$A$-加群$M, N$と
            $\varphi \in \Hom_A(M, N)$に対し図式
            \begin{equation}
                \begin{tikzcd}
                    T(M) \ar{d}[swap]{\tau_M}{\sim} \ar{r}{T(\varphi)}
                        & T(N) \ar{d}{\tau_N}[swap]{\sim} \\
                    S(M) \ar{r}[swap]{S(\varphi)}
                        & S(N)
                \end{tikzcd}
            \end{equation}
            が可換となる。
    \end{enumerate}
    が成り立つことをいう。
    このとき$T \cong S$と書く。
    \TODO{随伴関手のとき$\Box$に対象を代入するとそのまま同型を表しているように読める!}
\end{definition}

\begin{remark}
    反変関手についても同様の定義ができる。
\end{remark}

\begin{definition}[関手的]
    $A, B, C_1, C_2$を環、
    $T_i \colon \lMod{A} \to \lMod{C_i}, \;
    S_i \colon \lMod{B} \to \lMod{C_i}$
    ($i = 1, 2$)を共変関手とする。
    $M \in \lMod{A}$と$N \in \lMod{B}$でパラメータ付けられた
    $\Z$-加群同型の族
    \begin{equation}
        \tau_{M, N} \colon
            \Hom_{C_1}(T_1(M), T_1(N))
            \to
            \Hom_{C_2}(S_2(M), S_2(N))
    \end{equation}
    が$M, N$に関し\term{関手的}{関手的}[かんしゅてき]であるとは、
    \begin{enumerate}
        \item 任意の$A$-加群$M, M'$、$B$-加群$N$および
            $f \in \Hom_A(M, M')$に対し図式
            \begin{equation}
                \begin{tikzcd}
                    \Hom_{C_1}(T_1(M), S_1(N))
                        \ar{d}[swap]{\tau_{M, N}}{\sim}
                        & \Hom_{C_1}(T_1(M'), S_1(N))
                        \ar{d}{\tau_{M', N}}[swap]{\sim}
                        \ar{l}[swap]{f^\sharp} \\
                    \Hom_{C_2}(T_2(M), S_2(N))
                        & \Hom_{C_2}(T_2(M'), S_2(N))
                        \ar{l}{f^\sharp}
                \end{tikzcd}
            \end{equation}
            が可換となる。
            \TODO{$f^\sharp$は$T_1(f)$や$T_2(f)$を合成している?}
        \item 任意の$A$-加群$M$、$B$-加群$N, N'$および
            $g \in \Hom_B(N, N')$に対し図式
            \begin{equation}
                \begin{tikzcd}
                    \Hom_{C_1}(T_1(M), S_1(N))
                        \ar{d}[swap]{\tau_{M, N}}{\sim}
                        \ar{r}{g_\sharp}
                        & \Hom_{C_1}(T_1(M), S_1(N'))
                        \ar{d}{\tau_{M, N'}}[swap]{\sim} \\
                    \Hom_{C_2}(T_2(M), S_2(N))
                        \ar{r}[swap]{g_\sharp}
                        & \Hom_{C_2}(T_2(M), S_2(N'))
                \end{tikzcd}
            \end{equation}
            が可換となる。
    \end{enumerate}
    が成り立つことをいう。
\end{definition}

\begin{theorem}[米田の補題の1つの型]
    $A, B$を環、
    $T, S \colon \lMod{A} \to \lMod{B}$を共変関手とする。
    このとき次は同値である:
    \begin{enumerate}
        \item $T \cong S$
        \item 各$A$-加群$M$と$B$-加群$N$に対し、
            $M, N$に関して関手的な
            $\Z$-加群の同型
            \begin{equation}
                \Psi_{M, N} \colon
                    \Hom_B(T(M), N)
                    \to
                    \Hom_B(S(M), N)
            \end{equation}
            が存在する。
        \item 各$A$-加群$M$と$B$-加群$N$に対し、
            $M, N$に関して関手的な
            $\Z$-加群の同型
            \begin{equation}
                \Phi_{N, M} \colon
                    \Hom_B(N, T(M))
                    \to
                    \Hom_B(N, S(M))
            \end{equation}
            が存在する。
    \end{enumerate}
\end{theorem}

\begin{proof}
    \TODO{}
\end{proof}

\begin{definition}[随伴関手]
    $A, B$を環、
    $T \colon \lMod{A} \to \lMod{B}, \;
    S \colon \lMod{B} \to \lMod{A}$
    を共変関手とする。
    $T$が$S$の\term{左随伴関手}[left adjoint functor]
    {左随伴関手}[ひだりずいはんかんしゅ]、あるいは
    $S$が$T$の\term{右随伴関手}[right adjoint functor]
    {右随伴関手}[みぎずいはんかんしゅ]であるとは、
    各$A$-加群$M$、$B$-加群$N$に対し、
    $M, N$に関し関手的な$\Z$-加群の同型
    \begin{equation}
        \varphi_{M, N} \colon
            \Hom_B(T(M), N)
            \to
            \Hom_A(M, S(N))
    \end{equation}
    が存在することをいう。
\end{definition}

\begin{theorem}[随伴の一意性]
    \TODO{}
\end{theorem}

\begin{proof}
    \TODO{}
\end{proof}

\begin{theorem}[テンソル関手とホム関手の随伴性]
    $A, B$を環、
    $M$を$(B, A)$-両側加群とする。
    このとき
    関手$M \otimes_A \Box \colon \lMod{A} \to \lMod{B}$は
    $\Hom_B(M, \Box) \colon \lMod{B} \to \lMod{A}$の
    左随伴関手である。
    すなわち
    $M \otimes_A \Box \dashv \Hom_B(M, \Box)$が成り立つ。
\end{theorem}

\begin{proof}
    $X \in \lMod{A}, \; Y \in \lMod{B}$に関し関手的な
    $\Z$-加群同型の族
    \begin{equation}
        \varphi_{X, Y} \colon
            \Hom_B(M \otimes_A X, Y)
            \to
            \Hom_A(X, \Hom_B(M, Y))
    \end{equation}
    を構成する。
    \TODO{}
\end{proof}

% ------------------------------------------------------------
%
% ------------------------------------------------------------
\section{係数制限と係数拡大}
\label[section]{section:restriction-and-extension-of-scalars}

\cref{example:restriction-of-scalars}で触れた係数制限と、
その随伴的な操作である係数拡大について述べる。

\begin{definition}[係数制限]
    \idxsym{restriction}{$\Res_A^B M, \; M|_B$}{$A$-加群$M$の$B$への制限}
    $A, B$を環、
    $\varphi \colon B \to A$を環準同型、
    $M$を$A$-加群とする。
    $B$-加群$\Res_A^B M = M|_B$を、
    $M$への$B$の作用を
    \begin{equation}
        B \times M \to B,
        \quad
        (b, m) \mapsto \varphi(b) m
    \end{equation}
    で定めたものとし、
    これを$M$の$B$への
    \term{係数の制限}[restriction of scalars]{係数制限}[けいすうせいげん]
    という。
    断らない限り$\varphi$として包含写像を用いる。
    $\Res_A^B \colon \lMod{A} \to \lMod{B}$は共変関手となる。
\end{definition}

\begin{definition}[係数拡大]
    $A, B$を環、$\varphi \colon B \to A$を環準同型とする。
    $A$に次のように$(A, B)$-両側加群の構造を入れる:
    \begin{equation}
        axb \coloneqq ax\varphi(b)
            \quad
            (x \in A, \; a \in A, \; b \in B)
    \end{equation}
    $B$-加群$M$に対し、
    \begin{equation}
        \Ind_B^A M \coloneqq A \otimes_B M
    \end{equation}
    を$M$の$A$への
    \term{係数の拡大}[extension of scalars]{係数の拡大}[けいすうのかくだい]
    という。
    $\Ind_B^A \colon \lMod{B} \to \lMod{A}$は共変関手となる。
\end{definition}

\begin{theorem}[係数制限と係数拡大の随伴性]
    $A, B$を環、
    $\varphi \colon B \to A$を環準同型とする。
    このとき、係数の拡大$\Ind_B^A$は係数の制限$\Res_A^B$の
    左随伴関手である。
\end{theorem}

\begin{proof}
    \TODO{}
\end{proof}

\begin{corollary}[Induction By Stage]
    \begin{equation}
        \Ind_C^A \cong \Ind_B^A \circ \Ind_C^B
    \end{equation}
    \TODO{}
\end{corollary}

\begin{proof}
    \TODO{}
\end{proof}

\begin{definition}[Production]
    $A, B$を環、
    $\varphi \colon B \to A$を環準同型とし、
    $A$に
    \begin{equation}
        bxa \coloneqq \varphi(b) xa
    \end{equation}
    により$(B, A)$-両側加群の構造を入れる。
    このとき、$B$-加群$M$に対し
    \begin{equation}
        \Pro_B^A(M) \coloneqq \Hom_B(A, M)
    \end{equation}
    と定めると、共変関手
    $\Pro_B^A \colon \lMod{B} \to \lMod{A}$
    が定まる。
\end{definition}

\begin{proposition}
    $\Pro_B^A$は$\Res_A^B$の右随伴関手である。
\end{proposition}

\begin{proof}
    \TODO{}
\end{proof}


% ------------------------------------------------------------
%
% ------------------------------------------------------------
\section{加法的関手と完全性}
\label[section]{section:additive-functors}

$A$を環とする。
\cref{definition:set-of-module-homomorphisms}
でみたように、すべての$A$-加群$M, N$に対し
$\Hom_A(M, N)$は$\Z$-加群となるのであった。
このような性質は加群の圏には欠かせないものであり、
加群の圏の間の関手を調べる際には
この$\Z$-加群構造を保つものが特に重要といえる。
そこで、この節では加法的関手とその完全性の概念を定義する。

\begin{definition}[加法的関手]
    $A, B$を環とする。
    共変関手$T \colon \lMod{A} \to \lMod{B}$が
    \term{加法的}[additive]{加法的}[かほうてき]であるとは、
    \begin{enumerate}
        \item $T(0) = 0$
        \item 各$M, N \in \Ob(\lMod{A})$に対し
            $T \colon \Hom_A(M, N) \to \Hom_B(T(M), T(N))$が
            $\Z$-加群の準同型となる。
    \end{enumerate}
    が成り立つことをいう。
\end{definition}

\begin{example}[加法的関手の例]
    ~
    \begin{itemize}
        \item $A, B$を環、$M$を$(B, A)$-両側加群とする。
            このとき、テンソル関手
            $M \otimes_A \Box \colon \lMod{A} \to \lMod{B}$は加法的である。
        \item $R$を可換環、$A, B$を$R$-代数、$M$を$(A, B)$-両側加群とする。
            このとき、共変ホム関手
            $\Hom_A(M, \Box) \colon \lMod{A} \to \lMod{B}$は加法的である。
    \end{itemize}
\end{example}

加法的関手は有限直和を保つ。

\begin{theorem}[加法的関手は有限直和を保つ]
    $A, B$を環、
    $T \colon \lMod{A} \to \lMod{B}$を加法的関手とする。
    任意の$A$-加群$M_1, M_2$と
    直和の標準射影
    $\iota_i \colon M_i \hookrightarrow M_1 \oplus M_2 \; (i = 1, 2)$
    に対し
    \begin{equation}
        T(\iota_1) \oplus T(\iota_2) \colon
            T(M_1) \oplus T(M_2) \to T(M_1 \oplus M_2)
    \end{equation}
    は$B$-加群の同型を与える。
    逆写像は、
    $p_i \colon M_1 \oplus M_2 \to M_i \; (i = 1, 2)$
    を直積の標準射影として
    \begin{equation}
        T(p_1) \times T(p_2) \colon
            T(M_1 \oplus M_2) \to T(M_1) \oplus T(M_2)
    \end{equation}
    で与えられる。
\end{theorem}

\begin{proof}
    \TODO{}
\end{proof}

\begin{theorem}[加法的関手は分裂短完全列を保つ]
    $A, B$を環、
    \begin{equation}
        \begin{tikzcd}
            0 \ar{r}
                & M_1 \ar{r}
                & M_2 \ar{r}
                & M_3 \ar{r}
                & 0
                & (\text{exact})
        \end{tikzcd}
    \end{equation}
    を$\lMod{A}$の分裂短完全系列とする。
    関手$T \colon \lMod{A} \to \lMod{B}$が加法的ならば、
    $T$はこの系列を分裂短完全系列に写す。
\end{theorem}

\begin{proof}
    \TODO{}
\end{proof}

左完全関手を定義する。

\begin{definition}[左完全関手\footnote{
    $T$で写した系列の左端に射$0 \to \bullet$が現れることから「左」完全と呼ばれる。
}]
    $A, B$を環とする。
    加法的共変関手$T \colon \lMod{A} \to \lMod{B}$が
    \term{左完全}[left exact]{左完全}[ひだりかんぜん]であるとは、
    $\lMod{A}$の任意の完全系列
    \begin{equation}
        \begin{tikzcd}
            0 \ar{r}
                & U \ar{r}{i}
                & V \ar{r}{p}
                & W
                & (\text{exact})
        \end{tikzcd}
    \end{equation}
    に対し、$\lMod{B}$の列
    \begin{equation}
        \begin{tikzcd}
            0 \ar{r}
                & T(U) \ar{r}{T(i)}
                & T(V) \ar{r}{T(p)}
                & T(W)
        \end{tikzcd}
    \end{equation}
    が完全系列となることをいう。

    加法的反変関手$T \colon \lMod{A} \to \lMod{B}$が左完全であるとは、
    $\lMod{A}$の任意の完全系列
    \begin{equation}
        \begin{tikzcd}
            U \ar{r}{i}
                & V \ar{r}{p}
                & W \ar{r}
                & 0
                & (\text{exact})
        \end{tikzcd}
    \end{equation}
    に対し、$\lMod{B}$の列
    \begin{equation}
        \begin{tikzcd}
            0 \ar{r}
                & T(W) \ar{r}{T(p)}
                & T(V) \ar{r}{T(i)}
                & T(U)
        \end{tikzcd}
    \end{equation}
    が完全系列となることをいう。
\end{definition}

共変/反変ホム関手は左完全である。

\begin{theorem}[共変ホム関手の左完全性]
    \label[theorem]{thm:hom-left-exactness}
    $A, B$を環、
    $X$を$(A, B)$-両側加群とする。
    このとき、共変ホム関手
    $\Hom_A(X, \Box) \colon \lMod{A} \to \lMod{B}$は左完全である。
\end{theorem}

\begin{proof}
    \TODO{$\Ab$でなく$\lMod{B}$に修正したい}

    $\lMod{A}$の任意の完全系列
    \begin{equation}
        \begin{tikzcd}
            0 \ar{r}
                & U \ar{r}{i}
                & V \ar{r}{p}
                & W
                & (\text{exact})
        \end{tikzcd}
    \end{equation}
    に対し、$\Ab$の列
    \begin{equation}
        \begin{tikzcd}
            0 \ar{r} & \Hom_A(X, U) \ar{r}{i_*}
                & \Hom_A(X, V) \ar{r}{p_*}
                & \Hom_A(X, W)
        \end{tikzcd}
    \end{equation}
    が完全系列となることを示す。

    \uline{$\Ker i_* = 0$であること} \quad
    $i$の単射性から明らか。

    \uline{$\Im i_* \subset \Ker p_*$であること} \quad
    $\Im i \subset \Ker p$より明らか。

    \uline{$\Ker p_* \subset \Im i_*$であること} \quad
    $g \in \Ker p_*$とすると、
    \begin{equation}
        g(x) \in \Ker p = \Im i \quad (\forall x \in X)
    \end{equation}
    が成り立つ。
    よって
    \begin{equation}
        g(x) = i(u_x) \quad (\exists u_x \in U)
    \end{equation}
    が成り立ち、$i$の単射性より$u_x$は$x$に対し一意に定まる。
    よって写像$f \colon X \to U, \; x \mapsto u_x$は well-defined である。
    さらに、直接計算により$f \in \Hom_A(X, U)$であることもわかる。
    よって$g = i_* f \in \Im i_*$である。
\end{proof}

\begin{theorem}[反変ホム関手の左完全性]
    $A, B$を環、
    $X$を$(A, B^\OP)$-両側加群とする。
    このとき、反変ホム関手
    $\Hom_A(\Box, X) \colon \lMod{A} \to \lMod{B}$は左完全である。
    \TODO{終域あってる?}
\end{theorem}

\begin{proof}
    \TODO{}
\end{proof}

右完全関手を定義する。

\begin{definition}[右完全関手]
    $A, B$を環とする。
    加法的共変関手$T \colon \lMod{A} \to \lMod{B}$が
    \term{右完全}[right exact]{右完全}[みぎかんぜん]であるとは、
    $\lMod{A}$の任意の完全系列
    \begin{equation}
        \begin{tikzcd}
            B' \ar{r}{i} & B \ar{r}{p} & B'' \ar{r} & 0
        \end{tikzcd}
    \end{equation}
    に対し、$\lMod{B}$の列
    \begin{equation}
        \begin{tikzcd}
            T(B') \ar{r}{T(i)} & T(B) \ar{r}{T(p)} & T(B'') \ar{r} & 0
        \end{tikzcd}
    \end{equation}
    が完全系列となることをいう。
\end{definition}

テンソル関手は右完全である。
すなわち全射を保つ。

\begin{theorem}[テンソル関手の右完全性]
    $A, B$を環、$M$を$(B, A)$-両側加群とする。
    このとき、テンソル関手$M \otimes_A \Box$は右完全である。
\end{theorem}

\begin{proof}
    \TODO{$\Ab$でなく$\lMod{B}$に修正したい}

    $\lMod{A}$の任意の完全系列
    \begin{equation}
        \begin{tikzcd}
            B' \ar{r}{i} & B \ar{r}{p} & B'' \ar{r} & 0
        \end{tikzcd}
    \end{equation}
    に対し、$\Ab$の列
    \begin{equation}
        \begin{tikzcd}
            M \otimes_A B' \ar{r}{1 \otimes i}
                & M \otimes_A B \ar{r}{1 \otimes p}
                & M \otimes_A B'' \ar{r}
                & 0
        \end{tikzcd}
    \end{equation}
    が完全系列となることを示す。

    \uline{$\Im(1 \otimes i) \subset \Ker(1 \otimes p)$であること} \quad
    \begin{equation}
        (1 \otimes p) \circ (1 \otimes i)
            = 1 \otimes (p \circ i)
            = 1 \otimes 0
            = 0
    \end{equation}
    より明らか。

    \uline{$\Ker(1 \otimes p) \subset \Im(1 \otimes i)$であること} \quad
    $E \coloneqq \Im(1 \otimes i)$とおく。
    上で示した$E \subset \Ker(1 \otimes p)$より、図式
    \begin{equation}
        \begin{tikzcd}
            M \otimes_A B \ar{rr}{\pi} \ar{rd}[swap]{1 \otimes p}
                && (M \otimes_A B) / E \ar[dashed]{ld}{\what{p}} \\
            & M \otimes_A B''
        \end{tikzcd}
    \end{equation}
    を可換にする準同型$\what{p}$が誘導される。
    ここで、もし$\what{p}$が同型であることを示せたならば、
    \begin{alignat}{1}
        \Ker (1 \otimes p)
            &= \Ker (\what{p} \circ \pi) \\
            &= \Ker \pi \quad (\because \text{ $\what{p}$は同型}) \\
            &= E \\
            &= \Im(1 \otimes i)
    \end{alignat}
    より示したいことが従う。
    そこで、$\what{p}$の逆写像$M \otimes_A B'' \to (M \otimes_A B)/E$を構成する。
    写像$f \colon M \times B'' \to (M \otimes_A B)/E$を次のように定める。
    すなわち、各$(a, b'') \in M \times B''$に対し、
    $p$の全射性より$p(b) = b''$なる$b \in B$がとれるから、
    $f(a, b'') \coloneqq a \otimes b + E$と定める。
    well-defined 性と$A$-双線型性は直接計算によりわかる。
    よって準同型$\what{f} \colon M \otimes_A B'' \to (M \otimes_A B)/E$が誘導され、
    $\what{f}$は$\what{p}$の逆写像となる。
    したがって$\what{p}$が同型であることがいえた。

    \uline{$1 \otimes p$が全射であること:}
    $p$の全射性より明らか。
\end{proof}

一方、テンソル関手は単射を保つとは限らない。

\begin{example}[テンソル関手が単射を保たない例]
    \label[example]{example:tensor-need-not-preserve-injectivity}
    $\Z$-加群の完全系列
    \begin{equation}
        \begin{tikzcd}
            0 \ar{r} & \Z \ar{r}{i} & \Q \ar{r} & \Q/\Z \ar{r} & 0
        \end{tikzcd}
    \end{equation}
    を考える。
    テンソル関手
    $\Z/2\Z \otimes_\Z \Box \colon \lMod{\Z} \to \lMod{\Z}$
    の右完全性より
    \begin{equation}
        \begin{tikzcd}
            \Z/2\Z \otimes \Z \ar{r}{1 \otimes i}
                & \Z/2\Z \otimes \Q \ar{r}
                & \Z/2\Z \otimes \Q/\Z \ar{r}
                & 0
        \end{tikzcd}
    \end{equation}
    は完全系列である。
    この列の左端は$\Z/2\Z \otimes \Z \cong \Z$である。
    一方、各$a \otimes q \in \Z/2\Z \otimes \Q$に対し
    \begin{alignat}{1}
        a \otimes q
            &= 2a \otimes (q/2) \\
            &= 0 \otimes (q/2) \\
            &= 0
    \end{alignat}
    である。よって$\Z/2\Z \otimes \Q = 0$である。
    したがって$1 \otimes i$は単射ではありえず、
    関手$\Z/2\Z \otimes_\Z \Box$は完全関手でないことがわかる。
\end{example}

左右の完全性を兼ね備えたものが完全関手である。

\begin{definition}[完全関手]
    $A, B$を環とする。
    加法的共変関手$T \colon \lMod{A} \to \lMod{B}$が
    \term{完全}[exact]{完全}[かんぜん]であるとは、
    $\lMod{A}$の任意の完全系列
    \begin{equation}
        \begin{tikzcd}
            0 \ar{r}
                & U \ar{r}{i}
                & V \ar{r}{p}
                & W \ar{r}
                & 0
                & (\text{exact})
        \end{tikzcd}
    \end{equation}
    に対し、$\lMod{B}$の列
    \begin{equation}
        \begin{tikzcd}
            0 \ar{r}
                & T(U) \ar{r}{T(i)}
                & T(V) \ar{r}{T(p)}
                & T(W) \ar{r}
                & 0
        \end{tikzcd}
    \end{equation}
    が完全系列となることをいう。

    加法的反変関手$T \colon \lMod{A} \to \lMod{B}$が完全であるとは、
    $\lMod{A}$の任意の完全系列
    \begin{equation}
        \begin{tikzcd}
            0 \ar{r}
                & U \ar{r}{i}
                & V \ar{r}{p}
                & W \ar{r}
                & 0
                & (\text{exact})
        \end{tikzcd}
    \end{equation}
    に対し、$\lMod{B}$の列
    \begin{equation}
        \begin{tikzcd}
            0 \ar{r}
                & T(W) \ar{r}{T(p)}
                & T(V) \ar{r}{T(i)}
                & T(U) \ar{r}
                & 0
        \end{tikzcd}
    \end{equation}
    が完全系列となることをいう。
\end{definition}

\begin{theorem}[完全関手は完全系列を完全系列に写す]
    \TODO{}
\end{theorem}

\begin{proof}
    \TODO{}
\end{proof}

\begin{corollary}
    加法的関手が左完全かつ右完全であることと、
    完全であることとは同値である。
\end{corollary}

\begin{proof}
    \TODO{}
\end{proof}

% ------------------------------------------------------------
%
% ------------------------------------------------------------
\section{射影加群}

共変ホム関手$\Hom_A(P, \Box)$を完全にするのが射影加群である。

\begin{definition}[射影加群]
    $A$を環とし、$P$を$A$-加群とする。
    $P$が次の同値な条件のうちのひとつ(したがってすべて)をみたすとき、
    $P$を\term{射影加群}[projective module]{射影加群}[しゃえいかぐん]という。
    \begin{enumerate}
        \item (リフトの存在)
            任意の$A$-加群準同型$f \colon P \to M''$と
            全射$A$-加群準同型$g \colon M \to M''$に対し、
            $A$-加群準同型$h \colon P \to M$であって
            \begin{equation}
                \begin{tikzcd}
                    & P \ar[dashed]{ld}[swap]{h} \ar{d}{f} \\
                    M \ar{r}[swap]{g}
                        & M'' \ar{r}
                        & 0
                        & (\text{exact})
                \end{tikzcd}
            \end{equation}
            を可換にするものが存在する。
        \item $\lMod{A}$のすべての完全列$0 \to M' \to M'' \to P \to 0$が分裂する。
        \item $P$はある自由$A$-加群の直和因子である。
            すなわち、ある$A$-加群$M$が存在して$P \oplus M$は自由となる。
        \item 函手$\Hom_A(P, \Box) \colon \lMod{A} \to \lMod{\Z}$は完全である。
    \end{enumerate}
\end{definition}

\begin{proof}[同値性の証明.]
    \uline{(1) $\Rightarrow$ (4)} \quad
    $P$が (1) をみたすとし、
    $\lMod{A}$の任意の完全系列
    \begin{equation}
        \begin{tikzcd}
            0 \ar{r}
                & U \ar{r}{i}
                & V \ar{r}{p}
                & W \ar{r}
                & 0
                & (\text{exact})
        \end{tikzcd}
    \end{equation}
    に対し、$\Ab$の列
    \begin{equation}
        \begin{tikzcd}
            0 \ar{r}
                & \Hom_A(P, U) \ar{r}{i_*}
                & \Hom_A(P, V) \ar{r}{p_*}
                & \Hom_A(P, W) \ar{r}
                & 0
        \end{tikzcd}
    \end{equation}
    が完全系列となることを示す。
    \cref{thm:hom-left-exactness} より
    $\Hom_A(P, \Box)$の左完全性はわかっているから、あとは
    \begin{equation}
        \begin{tikzcd}
            \Hom_A(P, V) \ar{r}{p_*}
                & \Hom_A(P, W) \ar{r}
                & 0
        \end{tikzcd}
    \end{equation}
    が完全系列であること、すなわち$p_*$の全射性をいえばよい。
    そのためには$\lMod{A}$の任意の射$h \colon P \to W$に対し
    \begin{equation}
        \begin{tikzcd}
            & P \ar[dashed]{ld}[swap]{g} \ar{d}{h} \\
            V \ar{r}[swap]{p}
                & W \ar{r}
                & 0
        \end{tikzcd}
    \end{equation}
    を可換にする射$g$が存在することをいえばよいが、
    $p$が全射であることと
    $P$が (1) をみたすことからこのような$g$は存在する。
    よって (4) がいえた。

    \uline{(4) $\Rightarrow$ (1)} \quad
    $\Hom_A(P, \Box)$は完全であるとする。
    準同型$f \colon P \to M''$と全射準同型$g \colon M \to M''$が任意に与えられたとする。
    このとき、列
    \begin{equation}
        \begin{tikzcd}
            M \ar{r}{g}
                & M'' \ar{r}
                & 0
                & (\text{exact})
        \end{tikzcd}
    \end{equation}
    は完全系列である。
    したがって、$\Hom_A(P, \Box)$の完全性より
    \begin{equation}
        \begin{tikzcd}
            \Hom_A(P, M) \ar{r}{g_*}
                & \Hom_A(P, M'') \ar{r}
                & 0
                & (\text{exact})
        \end{tikzcd}
    \end{equation}
    は完全系列、すなわち$g_*$は全射である。
    よって、図式
    \begin{equation}
        \begin{tikzcd}
            M \ar{r}{g}
                & M'' \ar{r}
                & 0
                & (\text{exact}) \\
            & P \ar[dashed]{lu}{h} \ar{u}[swap]{f}
        \end{tikzcd}
    \end{equation}
    を可換にする$h \in \Hom_A(P, M)$が存在する。
    よって (1) がいえた。

    \TODO{cf. [Lang] p.137}
\end{proof}

自由加群は、定義から明らかな射影加群の例のひとつである。

\begin{proposition}
    \label[proposition]{prop:free-module-is-projective}
    自由加群は射影加群である。
\end{proposition}

\begin{proof}
    射影加群の定義の条件(3)より従う。
\end{proof}

\begin{example}[自由でない射影加群の例]
    $\Z/6\Z = \Z/2\Z \oplus \Z/3\Z$は$\Z/6\Z$-加群として自由だから、
    直和因子$\Z/2\Z, \Z/3\Z$は射影$\Z/6\Z$-加群である。
    一方、$\Z/3\Z$は$\Z/6\Z$-加群として自由ではない。
    実際、もし自由加群ならばその濃度は$6$のべきか無限となるはずである。
    同様に$\Z/2\Z$も自由でない。
\end{example}

直和により射影加群の例を増やすことができる。

\begin{proposition}[射影的な直和加群]
    $A$を環とする。
    $\{ M_i \}_{i \in I}$を$A$-加群の族とするとき、
    直和$M \coloneqq \bigoplus_{i \in I} M_i$が射影的であることと
    各$M_i$が射影的であることとは同値である。
\end{proposition}

\begin{proof}
    \TODO{}
\end{proof}

% ------------------------------------------------------------
%
% ------------------------------------------------------------
\section{入射加群}

反変ホム関手$\Hom_A(\Box, Q)$を完全にするのが入射加群である。

\begin{definition}[入射加群]
    $A$を環とし、$Q$を$A$-加群とする。
    $Q$が次の同値な条件のうちのひとつ(したがってすべて)をみたすとき、
    $Q$を\term{入射加群}[injective module]{入射加群}[にゅうしゃかぐん]という。
    \begin{enumerate}
        \item 任意の$A$-加群$M$とその$A$-部分加群$M'$、
            および$A$-加群準同型$f \colon M' \to Q$に対し、
            $A$-加群準同型$h \colon M \to Q$であって
            \begin{equation}
                \begin{tikzcd}
                    0 \ar{r} & M' \ar{d}[swap]{f} \ar[hook]{r} & M \ar[dashed]{ld}{h} \\
                    & Q
                \end{tikzcd}
            \end{equation}
            を可換にするものが存在する。
        \item 函手$\Hom_A(\Box, Q) \colon \lMod{A}^\OP \to \lMod{\Z}$は完全である。
        \item すべての完全列$0 \to Q \to M \to M'' \to 0$が分裂する。
    \end{enumerate}
\end{definition}

\begin{proof}
    cf. \cite[p.782]{Lan02}
\end{proof}

\begin{theorem}
    \label[theorem]{thm:Q-over-Z-is-injective}
    $\Q / \Z$は入射$\Z$-加群である。
\end{theorem}

\begin{proof}
    $\varphi \colon X \to Y$を単射な$\Z$-加群準同型、
    $h \colon X \to \Q / \Z$を$\Z$-加群準同型とする。
    図式
    \begin{equation}
        \begin{tikzcd}
            0 \ar{r}
                & X \ar[tail]{r}{\varphi} \ar{d}[swap]{h}
                & Y \ar[dashed]{ld}{\psi}
                & (\text{exact}) \\
            & \Q / \Z
        \end{tikzcd}
    \end{equation}
    を可換にする$\Z$-加群準同型$\psi$の存在を示す。
    ここで
    \begin{equation}
        \calQ \coloneqq \{
            (Z, \xi) \mid
            \text{$Z$は$\Z$-加群}, \; X \subset Z \subset Y, \;
            \text{$\xi \colon Z \to \Q / \Z$は$\Z$-加群準同型}, \;
            \xi = h \; \text{ on } \; X
        \}
    \end{equation}
    とおく。
    $\calQ$上の関係$\le$を
    \begin{equation}
        (Z, \xi) \le (Z', \xi')
            \quad \iff \quad
            Z \subset Z'
            \quad \text{かつ} \quad
            \xi = \xi' \; \text{ on } \; Z
    \end{equation}
    で定めると、$\le$は$\calQ$上の半順序となり、
    $\calQ$は帰納的半順序集合となる。
    Zorn の補題より$\calQ$は極大元$(W, \psi)$をもつ。
    \TODO{$W \subsetneq Y$として矛盾を導く}
\end{proof}

\begin{theorem}[完全関手の随伴]
    \label[theorem]{thm:adjoint-of-exact-functor}
    $A, B$を環、
    $T \colon \lMod{A} \to \lMod{B}$を関手とする。
    このとき次が成り立つ:
    \begin{enumerate}
        \item $T$がある完全関手の左随伴ならば、
            $T$は射影加群を射影加群に写す。
        \item $T$がある完全関手の右随伴ならば、
            $T$は入射加群を入射加群に写す。
    \end{enumerate}
\end{theorem}

\begin{proof}
    (1) を示す。(2) も同様である。
    $T$は完全関手$S \colon \lMod{B} \to \lMod{A}$の左随伴であるとし、
    $M$を射影$A$-加群とする。
    各$B$-加群$X$に対し、
    随伴性より$\Hom_B(T(M), X) \cong \Hom_A(M, S(X))$だから、
    共変ホム関手$\Hom_B(T(M), \Box)$は
    $X \mapsto \Hom_B(T(M), X) \cong \Hom_A(M, S(X))$と写す。
    右辺は完全関手の合成$\Hom_A(M, S(\Box))$だから完全である。
    したがって$\Hom_B(T(M), \Box)$は完全である。
    よって$T(M)$は射影加群である。
\end{proof}

\begin{definition}[可除加群]
    $R$を可換環、$M$を$R$-加群とする。
    $M$が\term{可除}[divisible]{可除}[かじょ]であるとは、
    $R$の$0$でも零因子でもない任意の元$a \in R$に対し、
    $a$倍写像$M \ni m \mapsto am \in M$が全射になることをいう。
\end{definition}

\begin{proposition}
    可換環上の入射加群は可除加群である。
\end{proposition}

\begin{proof}
    \cref{problem:algebra2-9.114} を参照。
\end{proof}



% ------------------------------------------------------------
%
% ------------------------------------------------------------
\section{平坦加群}

テンソル関手を完全にするのが平坦加群である。
ここでは平坦加群の定義として右側加群のものを与えるが、
明らかに左側加群についても同様に定義される。

\begin{definition}[平坦加群]
    $A$を環、$F$を右$A$-加群とする。
    $F$が次の同値な条件のうちのひとつ(したがってすべて)をみたすとき、
    $F$を\term{平坦加群}[flat module]{平坦加群}[へいたんかぐん]という:
    \begin{enumerate}
        \item $\lMod{A}$の任意の完全系列
            \begin{equation}
                \begin{tikzcd}
                    E' \ar{r}
                        & E \ar{r}
                        & E''
                        & (\text{exact})
                \end{tikzcd}
            \end{equation}
            に対し$\lMod{\Z}$の系列
            \begin{equation}
                \begin{tikzcd}
                    F \otimes_A E' \ar{r}
                        & F \otimes_A E \ar{r}
                        & F \otimes_A E''
                \end{tikzcd}
            \end{equation}
            は完全となる。
        \item $\lMod{A}$の任意の完全系列
            \begin{equation}
                \begin{tikzcd}
                    0 \ar{r}
                        & E' \ar{r}
                        & E \ar{r}
                        & E'' \ar{r}
                        & 0
                        & (\text{exact})
                \end{tikzcd}
            \end{equation}
            に対し$\lMod{\Z}$の系列
            \begin{equation}
                \begin{tikzcd}
                    0 \ar{r}
                        & F \otimes_A E' \ar{r}
                        & F \otimes_A E \ar{r}
                        & F \otimes_A E'' \ar{r}
                        & 0
                \end{tikzcd}
            \end{equation}
            は完全となる。
        \item (単射を保つこと) $\lMod{A}$の任意の完全系列
            \begin{equation}
                \begin{tikzcd}
                    0 \ar{r}
                        & E' \ar{r}
                        & E
                        & (\text{exact})
                \end{tikzcd}
            \end{equation}
            に対し$\lMod{\Z}$の系列
            \begin{equation}
                \begin{tikzcd}
                    0 \ar{r}
                        & F \otimes_A E' \ar{r}
                        & F \otimes_A E
                \end{tikzcd}
            \end{equation}
            は完全となる。
    \end{enumerate}
\end{definition}

\begin{proof}
    \TODO{}
\end{proof}

平坦でない加群の例のひとつは、
\cref{example:tensor-need-not-preserve-injectivity}
で紹介した ($\Z$-加群としての) $\Z / 2\Z$である。

いくつかの基本的な平坦加群のクラスを挙げる。

\begin{lemma}
    \label[lemma]{lemma:module-tensor-isomorphism}
    $A$を環とする。
    \begin{enumerate}
        \item 任意の$A$-加群$N$に対し、
            $A$-加群準同型$\mu \colon A \otimes_A N \to N$であって
            \begin{equation}
                \mu(a \otimes n) = an
                \quad
                \mu^{-1}(n) = 1_A \otimes n
            \end{equation}
            なるものが存在する。
        \item 任意の右$A$-加群$M$に対し、
            右$A$-加群準同型$\eta \colon M \otimes_A A \to M$であって
            \begin{equation}
                \eta(m \otimes a) = ma,
                \quad
                \eta^{-1}(m) = m \otimes 1_A
            \end{equation}
            なるものが存在する。
    \end{enumerate}
\end{lemma}

\begin{proof}
    \TODO{}
\end{proof}

\begin{theorem}[基本的な平坦加群]
    $A$を環とする。
    \begin{enumerate}
        \item $A_A$ (resp. $\down{A}A$) は平坦な右 (resp. 左) $A$-加群である。
        \item $\{ F_i \}_{i \in I}$を右 (resp. 左) $A$-加群の族とするとき、
            直和$F \coloneqq \oplus F_i$が平坦な右 (resp. 左) $A$-加群であることと
            各$F_i$が平坦な右 (resp. 左) $A$-加群であることとは同値である。
        \item 射影加群は平坦な左$A$-加群である。
    \end{enumerate}
\end{theorem}

\begin{proof}
    \uline{(1)} \quad
    右$A$-加群の場合のみ示す。
    $f \colon X \to Y$を単射$A$-加群準同型とする。
    上の補題より、図式
    \begin{equation}
        \begin{tikzcd}
            0
                \ar{r}
                & X
                    \ar{r}{f}
                & Y \\
            & A \otimes_A X
                \ar{u}{\mu}
                \ar{r}[swap]{\id_A \otimes f}
                & A \otimes_A Y
                    \ar{u}[swap]{\mu'}
        \end{tikzcd}
    \end{equation}
    を可換にする$A$-加群の同型$\mu, \mu'$が存在する
    (図式の可換性は$a \otimes x$の形の元の行き先を追跡すればよい)。
    $f$の単射性より$\id_A \otimes f$は単射である。
    したがって$A_A$は平坦である。

    \uline{(2)} \quad
    \TODO{}

    \uline{(3)} \quad
    \TODO{(1), (2)を使う}
\end{proof}

平坦加群の条件(2)の逆も成り立つものは忠実平坦と呼ばれる。

\begin{definition}[忠実平坦加群]
    $A$を環、$F$を右$A$-加群とする。
    $F$が
    \term{忠実平坦加群}[faithfully flat module]{忠実平坦加群}[ちゅうじつへいたんかぐん]
    であるとは、次が成り立つことをいう:
    \begin{itemize}
        \item $\lMod{A}$の系列
            \begin{equation}
                \begin{tikzcd}
                    0 \ar{r}
                        & E' \ar{r}
                        & E \ar{r}
                        & E'' \ar{r}
                        & 0
                \end{tikzcd}
                \label[equation]{eq:faithfully-flat-1}
            \end{equation}
            および
            $\lMod{\Z}$の系列
            \begin{equation}
                \begin{tikzcd}
                    0 \ar{r}
                        & F \otimes_A E' \ar{r}
                        & F \otimes_A E \ar{r}
                        & F \otimes_A E'' \ar{r}
                        & 0
                \end{tikzcd}
                \label[equation]{eq:faithfully-flat-2}
            \end{equation}
            に関し、
            系列\cref{eq:faithfully-flat-1}が完全であることと
            系列\cref{eq:faithfully-flat-2}が完全であることとは同値である。
    \end{itemize}
\end{definition}

忠実平坦加群の例のひとつは自由加群である。

\begin{proposition}[自由加群は忠実平坦]
    $0$でない自由加群は忠実平坦である。
    $0$でない射影加群は忠実平坦とは限らない。
\end{proposition}

\begin{proof}
    cf. \cref{problem:algebra2-9.119}
\end{proof}

平坦性を用いて
入射加群のひとつの例が得られる。

\begin{proposition}
    $A$を環、$M$を平坦右$A$-加群とする。
    $M$を$(\Z, A)$-両側加群とみて、
    $\Hom_\Z(M, \Q / \Z)$は
    入射$A$-加群である。
\end{proposition}

\begin{proof}
    $M$が平坦であることより$M \otimes_A \Box$は完全関手だから、
    \cref{thm:adjoint-of-exact-functor}
    より$M \otimes_A \Box$の右随伴$\Hom_\Z(M, \Box)$は入射性を保つ。
    \cref{thm:Q-over-Z-is-injective}より
    $\Q / \Z$は入射$\Z$-加群であるから、
    $\Hom_\Z(M, \Q / \Z)$は入射$A$-加群である。
\end{proof}

\begin{theorem}[入射加群への埋め込み]
    $A$を環、$M$を$A$-加群とする。
    このとき、ある入射$A$-加群$I$と
    単射$A$-加群準同型$\varphi \colon M \to I$が存在する。
\end{theorem}

\begin{proof}
    \TODO{}
\end{proof}

\begin{proposition}[局所化と平坦性]
    \label[proposition]{prop:localization-and-flatness}
    $R$を可換環とする。
    \begin{enumerate}
        \item $S \subset R$を$R$の積閉集合とするとき、
            $S^{-1}R$は$R$-加群として平坦である。
        \item \TODO{}
    \end{enumerate}
\end{proposition}

\begin{proof}
    \TODO{}
\end{proof}

$\Q$は$\Z$の商体だから、
上で示した局所化の平坦性より$\Q$は$\Z$上平坦であることがわかる。
一方、この事実は別の方法で示すこともできる。

\begin{proposition}
    \label[proposition]{prop:Q-is-flat-over-Z}
    $\Q$は$\Z$上平坦である。
\end{proposition}

\begin{proof}
    \TODO{これであってる?}
    まず、$\Q$の任意の有限生成$\Z$-部分加群は
    自由$\Z$-加群である。
    \begin{innerproof}
        $S$を$\Q$の有限生成$\Z$-部分加群とする。
        $S = \emptyset$の場合は明らかだから$S \neq \emptyset$とする。
        $S$は有限生成だから
        \begin{equation}
            S = \langle \{ q_1, \dots, q_n \} \rangle,
            \quad
            q_1, \dots, q_n \in \Q
        \end{equation}
        と表せる。
        ここで、必要ならば$q_1, \dots, q_n$の
        添字の小さい方から順に$\Z$上の1次独立性を保つ元のみを選び出して、
        $q_1, \dots, q_n$は$\Z$上1次独立であるとしてよい。
        このとき$\Z$-加群準同型
        \begin{equation}
            f \colon \Z^n \to S,
            \quad
            (m_1, \dots, m_n) \mapsto m_1 q_1 + \cdots + m_n q_n
        \end{equation}
        に対し準同型定理を用いて
        同型$\Z^n \cong S$を得る。
    \end{innerproof}
    $M, N$を$\Z$-加群とし、
    $f \colon M \to N$を単射$\Z$-加群準同型とする。
    $\id_\Q \otimes f \colon \Q \otimes_\Z M \to \Q \otimes_\Z N$が
    単射であることを示す。
    $\xi \in \Q \otimes_\Z M$が
    $(\id_\Q \otimes f)(\xi) = 0$をみたすとし、$\xi = 0$を示す。
    $\xi$は
    \begin{equation}
        \xi = \sum_{j = 1}^n q_j \otimes v_j
            \quad
            (q_j \in \Q, \; v_j \in M)
    \end{equation}
    の形に表せる。
    そこで$\Q$のイデアル、すなわち$\Z$-部分加群$I$を
    $I \coloneqq \langle \{ q_1, \dots, q_j \} \rangle$
    とおき、標準射影$I \to \Q$を$j$とおく。
    このとき
    \begin{equation}
        \begin{tikzcd}
            \Q \otimes_Z M
                \ar{r}{\id_\Q \otimes f}
                & \Q \otimes_Z N \\
            I \otimes_Z M
                \ar{u}{j \otimes \id_M}
                \ar{r}[swap]{\id_I \otimes f}
                & I \otimes_Z N
                \ar{u}[swap]{j \otimes \id_N}
        \end{tikzcd}
    \end{equation}
    は可換となる\TODO{なぜ?}。
    冒頭で示したように$I$は自由$\Z$-加群、したがって平坦だから
    $\id_I \otimes f$は単射である。
    \TODO{}
\end{proof}

平坦加群の応用のひとつが、
線型代数学で学んだ Rank-Nullity Theorem の
加群バージョンである。

\begin{theorem}[有限ランク自由$\Z$-加群の Rank-Nullity Theorem]
    $M, N$を有限ランク自由$\Z$-加群、
    $f \colon M \to N$を$\Z$-加群準同型とする。
    このとき
    \begin{equation}
        \rk M = \rk\Ker f + \rk\Im f
    \end{equation}
    が成り立つ。
\end{theorem}

\begin{proof}
    \cref{prop:localization-and-flatness}
    あるいは
    \cref{prop:Q-is-flat-over-Z}
    より、$\Q$は$\Z$-加群として平坦である。
    したがって$\Q \otimes_\Z \Box \colon \lMod{\Z} \to \lMod{\Q}$は完全関手である。
    いま
    \begin{equation}
        \begin{tikzcd}
            0 \ar{r}
                & \Ker f \ar{r}
                & M \ar{r}{f}
                & \Im f \ar{r}
                & 0
                & (\text{exact})
        \end{tikzcd}
    \end{equation}
    は完全系列だから
    \begin{equation}
        \begin{tikzcd}
            0 \ar{r}
                & \Q \otimes_\Z \Ker f \ar{r}
                & \Q \otimes_\Z M \ar{r}{\id_\Q \otimes f}
                & \Q \otimes_\Z \Im f \ar{r}
                & 0
        \end{tikzcd}
    \end{equation}
    も完全系列となる。
    よって線型代数学の Rank-Nullity Theorem より
    \begin{equation}
        \rk M = \rk(\Q \otimes_\Z \Ker f) + \rk(\Q \otimes_\Z \Im f)
    \end{equation}
    が成り立つ。
    いま\cref{thm:basis-of-tensor-product-of-free-modules}より
    \begin{alignat}{1}
        \rk M &= \dim_\Q (\Q \otimes_\Z M) \\
        \rk\Ker f &= \dim_\Q (\Q \otimes_\Z \Ker f) \\
        \rk\Im f &= \dim_\Q (\Q \otimes_\Z \Im f)
    \end{alignat}
    だから
    \begin{equation}
        \rk M = \rk\Ker f + \rk\Im f
    \end{equation}
    を得る。
\end{proof}


% ------------------------------------------------------------
%
% ------------------------------------------------------------
\newpage
\section{演習問題}

\subsection{Problem set 7}

\begin{problem}[代数学II 7.93]
    $A$を環とし、$x, y \in A$が$xy = 1$をみたすとする。
    このとき常に$x, y \in A^\times$となるか?
    正しければ証明を、誤りならば反例を与えよ。
\end{problem}

\begin{answer}
    \TODO{}
    \url{https://math.stackexchange.com/questions/1702297/is-there-a-ring-which-satisfies-xy-1-and-yx-neq-1}
\end{answer}

\begin{problem}[代数学II 7.95]
    \label[problem]{problem:algebra2-7-95}
    $A$を環、$R$を可換環、$(I, \le)$を有向的半順序集合、
    $(\{ M_i \}_{i \in I}, \{ \varphi_{ij} \}_{i \le j})$を
    $A$-加群 ($R$-代数) の有向系とする。
    $x \in M_i, \; y \in M_j$に対しある$k \in I$が存在して
    \begin{equation}
        \begin{cases}
            i, j \le k \\
            \varphi_{ik}(x) = \varphi_{jk}(y)
        \end{cases}
    \end{equation}
    をみたすとき$x \sim y$と書くことにすれば、
    これは disjoint union $\bigsqcup_{i \in I} M_i$上の
    同値関係を定める。このとき
    \begin{equation}
        \varinjlim_{i \in I} M_i
            \coloneqq \bigsqcup_{i \in I} M_i \bigg/ \sim
    \end{equation}
    とおくと$A$-加群 ($R$-代数) の構造が自然に入り、
    $\iota_i \colon M_i \to \varinjlim_{i \in I} M_i$を
    包含写像$M_i \hookrightarrow \bigsqcup_{i \in I} M_i$から
    誘導された写像とすると
    $A$-加群 ($R$-代数) の準同型になり、
    組$(\varinjlim_{i \in I} M_i, \{ \iota_i \}_{i \in I})$は
    $(\{ M_i \}_{i \in I}, \{ \varphi_{ij} \}_{i \le j})$の
    帰納極限になることを示せ。
\end{problem}

\begin{answer}
    $I$は少なくともひとつの元をもつとしてよい。
    \begin{innerproof}
        もし$I = \emptyset$なら
        $\varinjlim_{i \in I} M_i = \emptyset$となってしまい
        $A$-加群の構造が入らない。
    \end{innerproof}
    $\varinjlim_{i \in I} M_i$に$A$-加群の構造が入ることを確かめる。
    $\bigsqcup_{i \in I} M_i$の元を$(i, x)$の形に書くことにすれば、
    $\varinjlim_{i \in I} M_i$の元は$[(i, x)]$の形に書ける。
    まず各$[(i, x)], [(j, y)] \in \varinjlim_{i \in I} M_i$に対し
    和を次のように定義する:
    \begin{equation}
        [(i, x)] + [(j, y)]
            \coloneqq [(k, \varphi_{ik}(x) + \varphi_{jk}(y))]
    \end{equation}
    ただし$k \in I$は、
    $k \ge i, j$なる$k$をひとつ選ぶとする
    (このような$k$は$I$が有向系であることにより確かに存在する)。
    この和は well-defined に定まり、
    可換性および結合律をみたす。
    \begin{innerproof}
        可換性および結合律は
        各$M_i$の加法の可換性および結合律より明らかだから、
        well-defined 性のみ示す。
        \begin{equation}
            \label[equation]{eq:problem-7-95-1}
            \begin{cases}
                (i, x) \sim (i', x') \\
                (j, y) \sim (j', y')
            \end{cases}
        \end{equation}
        とする。
        $k \ge i, j$なる$k \in I$と
        $k' \ge i', j'$なる$k' \in I$をひとつずつ選び、
        \begin{equation}
            (k, \varphi_{ik}(x) + \varphi_{jk}(y))
                \sim (k', \varphi_{i'k'}(x') + \varphi_{j'k'}(y'))
        \end{equation}
        が成り立つことを示せばよい。
        まず (\cref{eq:problem-7-95-1}) より
        ある$s, t \in I$が存在して
        \begin{equation}
            \begin{cases}
                i, i' \le s \\
                \varphi_{is}(x) = \varphi_{i's}(x') \\
            \end{cases}
            \quad \text{かつ} \quad
            \begin{cases}
                j, j' \le t \\
                \varphi_{jt}(y) = \varphi_{j't}(y')
            \end{cases}
        \end{equation}
        が成り立つ。
        さらに$I$が有向系であることを用いて
        次の図の下から順に$k, k', m \in I$を選んでいく:
        \begin{equation}
            \begin{tikzcd}[column sep=small]
                &&& m \\
                & \cdot \ar{rru} &&&& \cdot \ar{llu} \\
                s \ar{ru}
                    && k \ar{lu}
                    && k' \ar{ru}
                    && t \ar{lu} \\
                i \ar{u} \ar{rru}
                    && i' \ar{llu} \ar{rru}
                    && j \ar{llu} \ar{rru}
                    && j' \ar{llu} \ar{u}
            \end{tikzcd}
        \end{equation}
        すると
        \begin{alignat}{1}
            \varphi_{km} (\varphi_{ik}(x) + \varphi_{jk}(y))
                &= \varphi_{km} (\varphi_{ik} (x)) + \varphi_{km} (\varphi_{jk} (y))
                    \quad (\text{準同型}) \\
                &= \varphi_{im} (x) + \varphi_{jm} (y)
                    \quad (\text{有向系の定義}) \\
                &= \varphi_{sm} (\varphi_{is} (x)) + \varphi_{tm} (\varphi_{jt} (y))
                    \quad (\text{有向系の定義}) \\
                &= \varphi_{sm} (\varphi_{i's} (x')) + \varphi_{tm} (\varphi_{j't} (y'))
                    \quad (\text{$s, t$のとり方}) \\
                &= \varphi_{i'm} (x') + \varphi_{j'm} (y')
                    \quad (\text{有向系の定義}) \\
                &= \varphi_{k'm} (\varphi_{i'k'} (x')) + \varphi_{k'm} (\varphi_{j'k'} (y'))
                    \quad (\text{有向系の定義}) \\
                &= \varphi_{km} (\varphi_{ik}(x') + \varphi_{jk}(y'))
                    \quad (\text{準同型})
        \end{alignat}
        が成り立つ。よって
        \begin{equation}
            (k, \varphi_{ik}(x) + \varphi_{jk}(y))
                \sim (k', \varphi_{i'k'}(x') + \varphi_{j'k'}(y'))
        \end{equation}
        が示せた。
    \end{innerproof}
    いま$I$は元をもつとしていたからある$i_0 \in I$がとれる。
    このとき$[(i_0, 0)]$はすべての$[(i, x)] \in \varinjlim M_i$に対し
    \begin{equation}
        [(i_0, 0)] + [(i, x)] = [(i, x)]
    \end{equation}
    をみたす。
    \begin{innerproof}
        $k \ge i_0, i$なる$k \in I$をひとつ選ぶ。
        \begin{alignat}{1}
            [(i_0, 0)] + [(i, x)]
                &= [(k, \varphi_{i_0k}(0) + \varphi_{ik}(x))]
                    \quad (\text{和の定義}) \\
                &= [(k, \varphi_{ik}(x))]
                    \quad (\text{準同型}) \\
                &= [(i, x)]
        \end{alignat}
        となる。
        ただし最後の等号が成り立つのは、いま$i, k \le k$であり、また
        $\varphi_{kk} = \id_{M_k}$より
        \begin{equation}
            \varphi_{ik}(x) = \varphi_{kk} (\varphi_{ik}(x))
        \end{equation}
        したがって$(i, x) \sim (k, \varphi_{ik}(x)) $だからである。
    \end{innerproof}
    加法逆元は
    \begin{equation}
        - [(i, x)] \coloneqq [(i, -x)]
    \end{equation}
    により定まる。
    したがって$\varinjlim M_i$はアーベル群となる。
    つぎに各$[(i, x)] \in \varinjlim_{i \in I} M_i, \; a \in A$に対し
    スカラー倍を次のように定義する:
    \begin{equation}
        a [(i, x)] \coloneqq [(i, ax)]
    \end{equation}
    このスカラー倍は well-defined に定まり、
    $1 \in A$は自明に作用し、結合律および分配律が成り立つ。
    \begin{innerproof}
        $1 \in A$が自明に作用することと結合律および分配律が成り立つことは
        各$M_i$のスカラー乗法に対するそれらの性質より明らかだから、
        well-defined 性のみ示す。
        $(i, x) \sim (i', x')$とすると
        ある$s \in I$が存在して
        \begin{equation}
            \begin{cases}
                i, i' \le s \\
                \varphi_{is}(x) = \varphi_{i's}(x')
            \end{cases}
        \end{equation}
        が成り立つ。$\varphi_{is}, \varphi_{i's}$が
        $A$-加群準同型であることより
        \begin{equation}
            \varphi_{is}(ax) = \varphi_{i's}(ax')
        \end{equation}
        が成り立つ。したがって
        $(i, ax) \sim (i', ax')$である。
    \end{innerproof}
    したがって$\varinjlim M_i$は$A$-加群となる。
    つぎに各$\iota_i \colon M_i \to \varinjlim M_i$が
    $A$-代数準同型となることを示す。
    $\iota_i$の定義より
    \begin{equation}
        \iota_i(x) = [(i, x)]
        \quad (x \in M_i)
    \end{equation}
    であることに注意すれば、
    $\varinjlim M_i$への加法とスカラー乗法の定め方から
    明らかに$\iota_i$は$A$-代数準同型である。
    最後に$(\varinjlim M_i, \{ \iota_i \}_{i \in I})$が
    $(\{ M_i \}_{i \in I}, \{ \varphi_{ij} \}_{i \le j})$の
    帰納極限となることを示す。
    そこで$A$-加群と$A$-加群準同型の族の組
    $(N, \{ \xi_i \colon M_i \to N \}_{i \in I})$であって
    $\xi_i = \xi_j \circ \varphi_{ij}$をみたすものが与えられたとする。
    図式
    \begin{equation}
        \begin{tikzcd}
            & \varinjlim M_k \ar[dashed]{dd}{\eta} \\
            M_i \ar{ru}{\iota_{i}} \ar{rd}[swap]{\xi_i} \\
            & N
        \end{tikzcd}
    \end{equation}
    を可換にする$A$-加群準同型$\eta \colon \varinjlim M_k \to N$を構成する。
    そこで写像$\eta \colon \varinjlim M_k \to N$を
    \begin{equation}
        \eta([(i, x)]) \coloneqq \xi_i(x)
    \end{equation}
    と定める。
    \TODO{}
\end{answer}

\begin{problem}[代数学7.96]
    $\R$における$0$の開近傍全体のなす集合を$\calN$とおく。
    $U, V \in \calN$に対し
    $U \le V \logeq U \supset V$と定めると、
    これは$\calN$上の有向的半順序を与える。
    各$U, V \in \calN, \; U \le V$に対し
    $r_{UV} \colon \smooth(U) \to \smooth(V)$を制限写像とすると
    $(\{ \smooth(U) \}_{U \in \calN}, \{ r_{UV} \}_{U \le V})$
    は$\C$-代数の有向系となる。
    ここで
    \begin{equation}
        C_0^\infty \coloneqq \varinjlim_{U \in \calN} \smooth(U)
    \end{equation}
    とおく。$C_0^\infty$は局所環となることを示し、
    極大イデアルと異なる素イデアルを持つことを示せ。
\end{problem}

\begin{answer}
    帰納極限の構成から明らかに$C_0^\infty$は可換$\C$-代数である。
    $C_0^\infty$の元は
    $0$の十分近くで一致する関数を同一視した類になっている。
    \begin{equation}
        I \coloneqq \{
            [(U, f)] \in C_0^\infty
            \mid
            f(0) = 0
        \}
    \end{equation}
    とおくと、$I$は$C_0^\infty$の極大イデアルである。
    \begin{innerproof}
        \TODO{}
    \end{innerproof}
    $f(0) \neq 0$なる元$[(U, f)]$を含むイデアルは
    $C_0^\infty$に一致するから、
    $C_0^\infty$の任意の極大イデアル$I'$はそのような元は含まない。
    よって$I$の定義から$I' \subset I$であり、
    $I'$が極大イデアルであることから$I' = I$である。
    したがって$C_0^\infty$の極大イデアルは$I$のみである。
    よって$C_0^\infty$は局所環である。
    つぎに
    \begin{equation}
        J \coloneqq \{ [(\R, 0)] \}
    \end{equation}
    とおけば$J$は$C_0^\infty$の零イデアルであるが、
    これは素イデアルでもある。
    \begin{innerproof}
        \TODO{}
        (\cref{problem:algebra2-1.9})
    \end{innerproof}
    $I$はたとえば$[(\R, x^2)]$を含むから零イデアルではないことに注意すれば、
    $J$が求める素イデアル、すなわち$I$と異なる素イデアルである。
\end{answer}

\subsection{Problem set 8}

\begin{problem}[代数学II 8.108]
    $A, B$を環、
    $T \colon \lMod{A} \to \lMod{B}$を加法的共変関手とする。
    $A$-加群の任意の完全系列
    \begin{equation}
        \begin{tikzcd}
            0 \ar{r}
                & M_1 \ar{r}{\alpha}
                & M_2 \ar{r}{\beta}
                & M_3 \ar{r}
                & 0
                & (\text{exact})
        \end{tikzcd}
    \end{equation}
    に対し、
    \begin{equation}
        \begin{tikzcd}
            T(M_1) \ar{r}{T(\alpha)}
                & T(M_2) \ar{r}{T(\beta)}
                & T(M_3) \ar{r}
                & 0
                & (\text{exact})
        \end{tikzcd}
    \end{equation}
    は完全となるとする。
    このとき$T$は右完全関手であることを示せ。
\end{problem}

\begin{proof}
    \TODO{加法的であることはいつ使う?}
    $A$-加群の完全系列
    \begin{equation}
        \begin{tikzcd}
            M_1 \ar{r}{\alpha}
                & M_2 \ar{r}{\beta}
                & M_3 \ar{r}
                & 0
                & (\text{exact})
        \end{tikzcd}
    \end{equation}
    が任意に与えられたとし、
    \begin{equation}
        \begin{tikzcd}
            T(M_1) \ar{r}{T(\alpha)}
                & T(M_2) \ar{r}{T(\beta)}
                & T(M_3) \ar{r}
                & 0
        \end{tikzcd}
    \end{equation}
    が完全系列であることを示す。
    準同型定理より図式
    \begin{equation}
        \begin{tikzcd}
            M_1 \ar{d}[swap]{\pi} \ar{r}{\alpha}
                & M_2 \\
            M_1 / \Ker\alpha \ar[dashed]{ru}[swap]{\wb{\alpha}}
        \end{tikzcd}
        \label[equation]{eq:algebra2-8.108-1}
    \end{equation}
    を可換にする単射な$A$-加群準同型$\wb{\alpha}$が誘導される。
    このとき$\Im\alpha = \Im\wb{\alpha}$も成り立つから
    \begin{equation}
        \begin{tikzcd}
            0 \ar{r}
                & M_1 / \Ker\alpha \ar{r}{\wb{\alpha}}
                & M_2 \ar{r}{\beta}
                & M_3 \ar{r}
                & 0
                & (\text{exact})
        \end{tikzcd}
    \end{equation}
    は完全系列である。
    そこで問題の仮定より
    \begin{equation}
        \begin{tikzcd}
            T(M_1 / \Ker\alpha) \ar{r}{T(\wb{\alpha})}
                & T(M_2) \ar{r}{T(\beta)}
                & T(M_3) \ar{r}
                & 0
                & (\text{exact})
        \end{tikzcd}
        \label[equation]{eq:algebra2-8.108-2}
    \end{equation}
    は完全系列である。
    $T$は共変だから、
    \cref{eq:algebra2-8.108-1}より図式
    \begin{equation}
        \begin{tikzcd}
            T(M_1) \ar{d}[swap]{T(\pi)} \ar{r}{T(\alpha)}
                & T(M_2) \\
            T(M_1 / \Ker\alpha) \ar{ru}[swap]{T(\wb{\alpha})}
        \end{tikzcd}
    \end{equation}
    は可換である。
    いま\cref{eq:algebra2-8.108-2}が完全系列であることより
    $\Im T(\wb{\alpha}) = \Ker T(\beta)$だから、
    あとは$\Im T(\alpha) = \Im T(\wb{\alpha})$を示せばよい。
    そのためには$T(\pi)$の全射性をいえばよいが、
    \begin{equation}
        \begin{tikzcd}
            0 \ar{r}
                & \Ker\alpha \ar{r}
                & M_1 \ar{r}{\pi}
                & M_1 / \Ker\alpha \ar{r} 
                & 0
                & (\text{exact})
        \end{tikzcd}
    \end{equation}
    が完全系列であることと問題の仮定より
    \begin{equation}
        \begin{tikzcd}
            T(\Ker\alpha) \ar{r}
                & T(M_1) \ar{r}{T(\pi)}
                & T(M_1 / \Ker\alpha) \ar{r} 
                & 0
                & (\text{exact})
        \end{tikzcd}
    \end{equation}
    は完全系列だから、とくに$T(\pi)$は全射である。
    したがって$\Im T(\alpha) = \Im T(\wb{\alpha}) = \Ker T(\beta)$である。
    よって
    \begin{equation}
        \begin{tikzcd}
            T(M_1) \ar{r}{T(\alpha)}
                & T(M_2) \ar{r}{T(\beta)}
                & T(M_3) \ar{r}
                & 0
        \end{tikzcd}
    \end{equation}
    は完全系列である。
\end{proof}

\subsection{Problem set 9}

\begin{problem}[代数学II 9.114]
    \label[problem]{problem:algebra2-9.114}
    $R$を可換環、$M$を$R$-加群とする。
    このとき$M$が入射的ならば可除になることを示せ。
\end{problem}

\begin{proof}
    $a \in M, \; a \neq 0$を零因子でないものとする。
    $R$-加群準同型$a \times \colon M \to M, \; x \mapsto ax$は単射となるから、
    いま$M$が入射的であることから図式
    \begin{equation}
        \begin{tikzcd}
            M \ar[tail]{r}{a \times} \ar{d}[swap]{\id}
                & M \ar[dashed]{ld}{f} \\
            M
        \end{tikzcd}
    \end{equation}
    を可換にする$R$-加群準同型$f$が存在する。
    各$y \in M$に対し$y = f(ay) = af(y)$が成り立つから
    $a \times$は全射である。
    したがって$M$は可除である。
\end{proof}

\begin{problem}[代数学II 9.117]
    $A$を環とする。
    左$A$-加群$M$が
    \term{平坦}[flat]{平坦}[へいたん]であるとは
    共変関手$\Box \otimes_A M \colon \rMod{A} \to \lMod{\Z}$が
    完全関手になることとする。
    このとき、左$A$-加群$M$が平坦であることの必要十分条件は
    $M$を右$A^\OP$-加群とみなしたときに
    $M$が平坦であることであることを示せ。
\end{problem}

\begin{proof}
    $X$を右$A$-加群とし、$a.x \coloneqq xa$により左$A^\OP$-加群ともみなす。
    このとき写像
    $M \times X \to X \otimes_A M, \; (m, x) \mapsto x \otimes m$は
    左$\Z$-線型$A^\OP$-平衡$\Z$-双線型写像であるから、
    $\Z$-線型写像$M \otimes_{A^\OP} X \to X \otimes_A M$が誘導される。
    この逆写像は
    $X \times M \to M \otimes_{A^\OP} X, \; (x, m) \mapsto m \otimes x$
    により誘導される。
    したがって$X \otimes_A M \cong M \otimes_{A^\OP} X$であるから、
    $\rMod{A}$の完全列
    \begin{equation}
        \begin{tikzcd}
            0
                \ar{r}
                & X
                    \ar{r}
                & Y
        \end{tikzcd}
    \end{equation}
    (これは$\lMod{A^\OP}$の完全列でもある) に対し
    \begin{equation}
        \begin{tikzcd}
            0
                \ar{r}
                & M \otimes_{A^\OP} X
                    \ar{r}
                & M \otimes_{A^\OP} Y
        \end{tikzcd}
    \end{equation}
    が$\lMod{\Z}$の完全列であることと
    \begin{equation}
        \begin{tikzcd}
            0
                \ar{r}
                & X \otimes_A M
                    \ar{r}
                & Y \otimes_A M
        \end{tikzcd}
    \end{equation}
    が$\lMod{\Z}$の完全列であることは同値である。
    よって問題の主張が示せた。
\end{proof}

\begin{problem}[代数学II 9.119]
    \label[problem]{problem:algebra2-9.119}
    $0$でない自由加群は忠実平坦であることを示せ。
    また$0$でない射影加群はどうか?
\end{problem}

\begin{proof}
    $A$を環とし、
    $M$を$0$でない自由右$A$-加群とする。
    このときとくに$A$は零環でない。
    $M$が忠実平坦であることを示す ($M$が左$A$-加群の場合も同様である)。
    $M$は自由ゆえに平坦だから、
    $M$が忠実平坦であることをいうには、
    $f \colon X \to Y$を任意の$A$-加群準同型として、
    $\Z$-加群準同型
    $\id_M \otimes f \colon M \otimes_A X \to M \otimes_A Y$
    が単射であるとき
    $f$が単射であることを示せばよい。
    いま$M$は自由右$A$-加群だから、$M$の基底をひとつ固定すれば
    右$A$-加群の同型
    $\mu \colon M \overset{\sim}{\to} \bigoplus_{i \in I} A_A$が存在する。
    このとき図式
    \begin{equation}
        \begin{tikzcd}[column sep=huge]
            M \otimes_A X
                \ar{r}{\id_M \otimes f}
                \ar{d}[swap]{\mu}
                & M \otimes_A Y
                    \ar{d}{\mu} \\
            \left(
                \bigoplus_{i \in I} A
            \right) \otimes_A X
                \ar{r}[swap]{\id_{\bigoplus A} \otimes f}
                & \left(
                    \bigoplus_{i \in I} A
                \right) \otimes_A Y
        \end{tikzcd}
    \end{equation}
    は可換だから、
    $\id_M \otimes f$の単射性より
    $\id_{\bigoplus A} \otimes f$の単射性が従う。
    ここで任意の$A$-加群$Z$に対し
    \begin{equation}
        \begin{tikzcd}
            \left(
                \bigoplus_{i \in I} A
            \right) \otimes_A Z
                \ar{r}{\sim}
                & \bigoplus_{i \in I}
                    (A \otimes_A Z)
                    \ar{r}{\sim}
                & \bigoplus_{i \in I} Z \\
            (a_i)_{i \in I} \oplus z
                \ar[mapsto]{r}
                & (a_i \otimes z)_{i \in I}
                    \ar[mapsto]{r}
                & (a_i z)_{i \in I}
        \end{tikzcd}
    \end{equation}
    は$A$-加群の同型である
    (\cref{thm:distribution-law-of-tensor-product},
    \cref{lemma:module-tensor-isomorphism})。
    よって
    \begin{equation}
        \begin{tikzcd}[row sep=large]
            \left(
                \bigoplus_{i \in I} A
            \right) \otimes_A X
                \ar{r}{\sim}
                \ar{d}[swap]{\id_{\bigoplus A} \otimes f}
                & \bigoplus_{i \in I}
                    (A \otimes_A X)
                    \ar{r}{\sim}
                & \bigoplus_{i \in I} X
                    \ar[dashed]{d}{g} \\
            \left(
                \bigoplus_{i \in I} A
            \right) \otimes_A Y
                \ar{r}{\sim}
                & \bigoplus_{i \in I}
                    (A \otimes_A Y)
                    \ar{r}{\sim}
                & \bigoplus_{i \in I} Y
        \end{tikzcd}
    \end{equation}
    の右端に誘導される$A$-加群準同型
    $g((a_i x)_{i \in I}) = (a_i f(x))_{i \in I}$
    は単射である。
    $f$が単射であることを示す。
    いま$\bigoplus_{i \in I} A \cong M \neq 0$だから
    ある$i_0 \in I$が存在する。
    $x \in X$とし、$f(x) = 0_Y$と仮定すると
    $g((\delta_{i_0 i} x)_{i \in I})
        = (\delta_{i_0 i} f(x))_{i \in I}
        = 0_{\bigoplus Y}$
    である。ただし$\delta_{i_0 i}$は
    \begin{equation}
        \delta_{i_0 i} \coloneqq \begin{cases}
            1_A & (i = i_0) \\
            0_A & (i \neq i_0)
        \end{cases}
    \end{equation}
    と定義した。
    $g$の単射性より$(\delta_{i_0 i} x)_{i \in I} = 0_{\bigoplus X}$だから
    $x = \delta_{i_0 i_0} x = 0_X$である。
    よって$f$は単射である。
    したがって$M$は忠実平坦である。

    一方、$0$でない射影加群が忠実平坦とは限らない例を挙げる。
    $R \coloneqq \Z / 6\Z$とし、
    $R$の元を$n \in \Z$に対し$\wb{n}$と書くことにする。
    $R$のイデアル$M, N$を
    \begin{equation}
        M \coloneqq (\wb{3}) = \{ \wb{0}, \wb{3} \},
        \qquad
        N \coloneqq (\wb{2}) = \{ \wb{0}, \wb{2}, \wb{4} \}
    \end{equation}
    で定める。
    $R$を$R$-加群、$M, N$を$R$の$R$-部分加群とみなすと
    $R = M + N, \; M \cap N = 0$より
    $R = M \oplus N$が成り立つ。
    $R$は自由$R$-加群だから、とくに$M$は ($0$でない) 射影$R$-加群である。
    そこで$M$が忠実平坦でないことを示せばよい。
    標準射影$R \to M$を$p$とおく。
    直和分解$R = M \oplus N$に沿って
    $\wb{0} = \wb{0} + \wb{0}, \;
        \wb{2} = \wb{0} + \wb{2}$
    と表せるから$p(\wb{0}) = \wb{0} = p(\wb{2})$であり、
    したがって$p$は単射でない。
    一方$\id_M \otimes p \colon M \otimes_R R \to M \otimes_R M$は単射であることを示す。
    $R$-加群準同型$f \colon M \to M \otimes_R M, \;
        m \mapsto m \otimes \wb{3}$
    を考える。
    $R$-加群の同型
    $\mu \colon M \overset{\sim}{\to} M \otimes_R R, \;
        m \mapsto m \otimes \wb{1}$
    は図式
    \begin{equation}
        \begin{tikzcd}[column sep=large]
            M
                \ar{rd}{f}
                \ar{d}{\sim}[swap]{\mu} \\
            M \otimes_R R
                \ar{r}[swap]{\id_M \otimes p}
                & M \otimes_R M
        \end{tikzcd}
    \end{equation}
    を可換にするから、$\id_M \otimes p$が単射であることをいうには
    $f$が単射であることをいえばよい。
    ここで$f$の右逆写像は、
    環$R$における積を$R$のイデアル$M$に制限した演算から誘導される
    $g \colon M \otimes_R M \to M, \;
        m \otimes m' \mapsto mm'$
    により与えられる。
    実際、各$m \in M$に対し
    $m \overset{f}{\mapsto} m \otimes \wb{3}
        \overset{g}{\mapsto} m \wb{3}$
    だから
    $m = \wb{0}$なら$m \wb{3} = \wb{0}$、
    $m = \wb{3}$なら$m \wb{3} = \wb{9} = \wb{3}$
    となり、$g$はたしかに$f$の右逆写像である。
    したがって$f$、ひいては$\id_M \otimes p$の単射性がいえた。
    よって$M$は$R$-加群として忠実平坦でない。
\end{proof}

\begin{problem}[代数学II 9.121]
    $R$を可換環、$M, N$を平坦$R$-加群とする。
    このとき$M \otimes_R N$は平坦であることを示せ。
\end{problem}

\begin{answer}
    $\lMod{R}$の完全列
    \begin{equation}
        \begin{tikzcd}
            0
                \ar{r}
                & X
                    \ar{r}
                & Y
        \end{tikzcd}
    \end{equation}
    に対し、$N$の平坦性より
    \begin{equation}
        \begin{tikzcd}
            0
                \ar{r}
                & N \otimes_R X
                    \ar{r}
                & N \otimes_R Y
        \end{tikzcd}
    \end{equation}
    は$\lMod{R}$の完全列だから、
    $M$の平坦性より
    \begin{equation}
        \begin{tikzcd}
            0
                \ar{r}
                & M \otimes_R N \otimes_R X
                    \ar{r}
                & M \otimes_R N \otimes_R Y
        \end{tikzcd}
    \end{equation}
    は$\lMod{R}$の完全列である。
    したがって$M \otimes_R N$は平坦である。
\end{answer}


\end{document}

\documentclass[report]{jlreq}
\usepackage{../../global}
\usepackage{./local}
\subfiletrue
%\makeindex
\begin{document}

\TODO{接続とは一体何なのか?}

この部では接続について論じる。
接続とは、ベクトル場を方向微分して
新たなベクトル場を作る手続きのようなものである。
接束の接続はアファイン接続と呼ばれ、とくに重要である。


% ============================================================
%
% ============================================================
\newpage
\chapter{ベクトル値微分形式}

% ------------------------------------------------------------
%
% ------------------------------------------------------------
\section{ベクトル値微分形式}
\label[section]{sec:vector-valued-forms}

微分形式の概念をベクトル束に値をもつように一般化する。
これは後に主ファイバー束の接続を定義するために用いる。

\begin{definition}[ベクトル束に値をもつ微分形式]
    $M$を多様体、$E \to M$をベクトル束とし、$p \in \Z_{\ge 0}$とする。
    ベクトル束$\bigwedge^p T^*M \otimes E$の切断を
    \term{$E$に値をもつ$p$-形式}
    {ベクトル束に値をもつ微分形式}[べくとるそくにあたいをもつびぶんけいしき]
    あるいは
    \term{$E$-値$p$-形式}[$E$-valued $p$-form]
    {ベクトル束に値をもつ微分形式}[べくとるそくにあたいをもつびぶんけいしき]
    という。
    $E$-値$p$-形式全体のなす集合を
    \begin{equation}
        A^p(E) \coloneqq \Gamma\Bigl(
            \Bigl(\bigwedge^p T^*M\Bigr) \otimes E
        \Bigr)
    \end{equation}
    と書く。
    $E$-値$p$-形式は
    $\theta \otimes \xi \; (\theta \in A^p(M), \; \xi \in A^0(E))$の形
    の元の和に (一意ではないが) 書ける。
\end{definition}

\begin{remark}
    \TODO{どういうこと?}
    ベクトル空間の同型
    \begin{equation}
        \Hom(\Lambda^k T_xM, V)
            \cong (\Lambda^k T_xM)^* \otimes V
            \cong (\Lambda^k T_x^*M) \otimes V
    \end{equation}
    に注意すれば、$V$に値をもつ$k$-形式の値は、
    確かに$\Lambda^k T_xM \to V$の$\R$線型写像とみなせることがわかる。
\end{remark}

\begin{remark}
    テキストでは$\theta$と$\xi$の順序が逆になったりしているが、
    ここでは$\theta \otimes \xi$の順序に統一する。
\end{remark}

ベクトル値形式は
従来の意味での微分形式ではなく、
したがって外積は定義されていないが、
通常の外積から自然に定義が拡張される。

\begin{definition}[ベクトル値形式の外積]
    $M$を多様体、$E \to M$をベクトル束、
    $p, q \in \Z_{\ge 0}$とする。
    $\wedge \colon A^p(M) \times A^q(M) \to A^{p + q}(M)$を
    通常の外積とし、
    その一般化として
    $\wedge \colon A^p(M) \times A^q(E) \to A^{p + q}(E)$を
    \begin{alignat}{1}
        (\omega, \xi)
            = \left(
                \omega,
                \sum_{i} \alpha_i \otimes \xi_i
            \right)
            \mapsto
            \omega \wedge \xi
            &\coloneqq
            \sum_{i} \omega \wedge \alpha_i \otimes \xi_i \\
        &\qquad \quad
            (\alpha_i \in A^q(M), \; \xi_i \in A^0(E))
    \end{alignat}
    と定める。
    これは明らかに$\xi$の表し方によらず well-defined に定まる。
\end{definition}

\begin{definition}[ベクトル値形式の内積]
    $M$を多様体、
    $E \to M, \; F \to M$をベクトル束、
    $g \colon A^0(E) \times A^0(F) \to A^0(M)$を
    $\smooth(M)$-双線型写像とする。
    $g$の一般化として、同じ記号で写像
    $g \colon A^p(E) \times A^q(F) \to A^{p + q}(M)$を
    \begin{alignat}{1}
        (\omega, \xi)
            = \left(
                \sum_{i} \alpha_i \otimes \omega_i,
                \sum_{j} \beta_j \otimes \xi_j
            \right)
            &\mapsto
            g(\omega, \xi)
            \coloneqq
            \sum_{i, j}
            g(\omega_i, \xi_j)
            \alpha_i \wedge \beta_j \\
        &
            (
                \alpha_i \in A^p(M), \; \beta_j \in A^q(M), \;
                \omega_i \in A^0(E), \; \xi_j \in A^0(F)
            )
    \end{alignat}
    と定める。
    これは$\omega, \xi$の表し方によらず well-defined に定まり (証明略)、
    また$\smooth(M)$-双線型写像である。
\end{definition}

\begin{remark}
    上の定義の双線型写像$g \colon A^0(E) \times A^0(F) \to A^0(M)$の例としては、
    \begin{itemize}
        \item 双対の定める内積
            $\langle , \rangle \colon A^0(E^*) \times A^0(E) \to A^0(M)$
        \item 計量
            $g \colon A^0(E) \times A^0(E) \to A^0(M)$
    \end{itemize}
    などがある。
\end{remark}



% ============================================================
%
% ============================================================
\newpage
\chapter{主ファイバー束}

主ファイバー束は、多様体$M$上局所自明な群の族である。
ここで主ファイバー束という概念を持ち出す理由はベクトル束を調べるためであるが、
実際ベクトル束と主ファイバー束の間には良い関係がある。
というのも、ベクトル束はフレーム束と呼ばれる主ファイバー束と対応し、
逆に主ファイバー束はその構造群の表現を通してベクトル束と対応する。
したがって、あるベクトル束について調べたいときに
代わりに主ファイバー束を考えることで議論の見通しがよくなることがある。
そこで、この章では主ファイバー束とベクトル束の基本的な関係を調べることにする。

% ------------------------------------------------------------
%
% ------------------------------------------------------------
\section{ファイバー束}

ファイバー束を定義する。

\TODO{ファイバー束は構造群付きを基本として、
    修飾しない場合は自明な構造群を持つものと定義したい}

\begin{definition}[ファイバー束]
    $M, F$を多様体とする。
    多様体$E$が
    \term{ファイバー束}[fiber bundle]{ファイバー束}[ふぁいばーそく]
    であるとは、
    $E$が次をみたすことである:
    \begin{enumerate}
        \item 全射な{\smooth}写像$\pi \colon E \to M$が与えられている。
        \item \TODO{局所自明性}
    \end{enumerate}
\end{definition}

\TODO{主ファイバー束をファイバー束の特別な場合として定義したい}

\begin{definition}[主ファイバー束]
    $M$を多様体、
    $G$を Lie 群とする。
    多様体$P$が
    $G$を\term{構造群}[structure group]{構造群}[こうぞうぐん]とする$M$上の
    \term{主ファイバー束}[principal fiber bundle]{主ファイバー束}[しゅふぁいばーそく]、
    あるいは\term{主$G$束}[principal $G$-bundle]{主$G$束}[しゅGそく]であるとは、
    $P$が次をみたすことである:
    \begin{enumerate}
        \item 全射な{\smooth}写像$p \colon P \to M$が与えられている。
        \item $G$は$P$に右から{\smooth}に作用しており、
            さらに次をみたす:
            \begin{enumerate}[label=(\arabic{enumi}-\alph*)]
                \item 作用はファイバーを保つ。
                \item 作用はファイバー上単純推移的\footnote{
                        作用が\term{単純推移的}[simply transitive]
                        {単純推移的}[たんじゅんすいいてき]
                        であるとは、自由かつ推移的であることをいう。
                    }である。
            \end{enumerate}
        \item $M$のある開被覆$\{ U_\alpha \}_{\alpha \in A}$が存在して、
            各$U_\alpha$上に
            次をみたす写像
            $\sigma_\alpha \colon U_\alpha \to p^{-1}(U_\alpha)$
            が存在する:
            \begin{enumerate}[label=(\arabic{enumi}-\alph*)]
                \item $\sigma_\alpha$は{\smooth}であって
                    $p \circ \sigma_\alpha = \id_{U_\alpha}$をみたす。
                    すなわち$\sigma_\alpha$は$U_\alpha$上の
                    $P$の切断である。
                \item (局所自明性) 写像
                    \begin{equation}
                        \varphi_\alpha \colon p^{-1}(U_\alpha) \to U_\alpha \times G,
                        \quad
                        \underbrace{\sigma_\alpha(x) . s}_{
                            \mathclap{\text{群作用を「$.$」で書く。}}
                        } \mapsto (x, s)
                    \end{equation}
                    が diffeo である
                    (写像として well-defined に定まることはすぐ後で確かめる)\footnote{
                        このように定めた写像$\varphi_\alpha$が diffeo かどうか
                        (とくに{\smooth}かどうか) は
                        他の条件からはおそらく導かれない気がするので (\TODO{本当に?})、
                        独立な条件として与えておくことにする。
                        \TODO{cf. \url{https://math.stackexchange.com/questions/2930299/trivialization-from-a-smooth-frame}}
                    }。
            \end{enumerate}
    \end{enumerate}
    ここで
    \begin{itemize}
        \item $\varphi_\alpha$を$U_\alpha$上の$P$の
            \term{局所自明化}[local trivialization]{局所自明化}[きょくしょじめいか]
            という。
    \end{itemize}
\end{definition}

\begin{lemma}[$G$-torsor の特徴付け]
    $G$を群、
    $X$を空でない集合とし、
    $G$は$X$に右から作用しているとする。
    このとき次は同値である:
    \begin{enumerate}
        \item $G$の作用が単純推移的である。
        \item 写像
            \begin{equation}
                \theta \colon X \times G \to X \times X,
                \quad
                (x, g) \mapsto (x.g, x)
            \end{equation}
            が全単射である\footnote{
                写像$\theta$を shear map といい、
                shear map が全単射のとき$X$を$G$-torsor という。
            }。
    \end{enumerate}
    したがって、とくに上の定義の$\varphi_\alpha$が確かに写像として定まる。
\end{lemma}

\begin{proof}
    \begin{alignat}{1}
        \theta \colon \text{ 全射}
            &\iff \forall x, y \in X \; \exists g \in G \; [x.g = y] \\
            &\iff \text{$G$の作用が推移的} \\
        \theta \colon \text{ 単射}
            &\iff \forall x \in X \;
                \forall g, g' \in G \;
                [x.g = x.g' \implies g = g'] \\
            &\iff \forall x \in X \;
                \forall g, g' \in G \;
                [x = x.g'g^{-1} \implies g'g^{-1} = 1] \\
            &\iff \forall x \in X \;
                \forall g \in G \;
                [x = x.g \implies g = 1] \\
            &\iff \text{$G$の作用が自由}
    \end{alignat}
\end{proof}

\begin{definition}[変換関数]
    $M$を多様体、
    $p \colon P \to M$を主$G$束とすると、
    主$G$束の定義より、$M$の open cover $\{U_\alpha\}_{\alpha \in A}$であって
    各$U_\alpha$上に切断
    $\sigma_\alpha \colon U_\alpha \to p^{-1}(U_\alpha)$
    を持つものがとれる。
    各$\alpha, \beta \in A, \; U_\alpha \cap U_\beta \neq \emptyset$
    に対し、
    写像$\psi_{\alpha\beta} \colon U_\alpha \cap U_\beta \to G$を
    $x \in U_\alpha \cap U_\beta$を
    $\sigma_\beta(x) = \sigma_\alpha(x) . s$なる$s \in G$
    に写す写像、すなわち
    \begin{equation}
        x \overset{\sigma_\beta}{\mapsto} \sigma_\beta(x) = \sigma_\alpha(x) . s
            \overset{
                \substack{\sigma_\alpha \text{ より定まる} \\ \text{局所自明化}}
            }{\mapsto} (x, s)
            \overset{\mathrm{pr}_2}{\mapsto} s
    \end{equation}
    で定めると、これは{\smooth}である。
    {\smooth}写像の族$\{ \psi_{\alpha\beta} \}$を、
    切断の族$\{ \sigma_\alpha \}$から定まる
    $P$の\term{変換関数}[transition function]{変換関数}[へんかんかんすう]という。
\end{definition}

% ------------------------------------------------------------
%
% ------------------------------------------------------------
\section{ベクトル束と主ファイバー束の同伴}

\subsection{ベクトル束から主ファイバー束へ}

多様体上のランク$r$ベクトル束が与えられると、
フレーム束とよばれる主$\GL(r, \R)$束を構成できる。

\TODO{フレーム束はフレーム多様体をファイバーとする主$\GL(r, \R)$束?}

\TODO{フレーム束の主ファイバー束構造は全単射により誘導する?}

\begin{definition}[フレーム束]
    $M$を$n$次元多様体、
    $E \to M$をランク$r$ベクトル束とする。
    $M$の atlas
    $\{ (U_\alpha, \psi_\alpha) \}_{\alpha \in A}$であって、
    各$\alpha$に対して$U_\alpha$上の$E$の局所自明化$\rho_\alpha$が存在するものがとれる。
    \begin{innerproof}
        各$x \in M$に対し、
        多様体の定義とベクトル束の定義より、
        $x$の$M$における開近傍$V_x, W_x$であって
        $V_x$を定義域とするチャートが存在し、
        かつ$W_x$上の$E$の局所自明化が存在するようなものがとれる。
        そこで$U_x \coloneqq V_x \cap W_x$とおけば
        $\{ U_x \}_{x \in M}$が求める atlas となる。
    \end{innerproof}
    $E$の局所自明化の族$\{ \rho_\alpha \}$により定まる
    $E$の変換関数を$\{ \rho_{\alpha\beta} \}$とおく。
    $E$の\term{フレーム束}[frame bundle]{フレーム束}[ふれーむそく]
    とよばれる主$\GL(r, \R)$束
    $p \colon P \to M$を次のように構成する:
    \begin{enumerate}
        \item 各$x \in M$に対し、集合$P_x$を
            \begin{equation}
                P_x \coloneqq \{
                    u \colon \R^r \to E_x
                    \mid
                    \text{$u$は線型同型}
                \}
            \end{equation}
            で定める。
            $P_x$は$E_x$の基底全体の集合とみなせる。
        \item $P_x$らの disjoint union を
            \begin{equation}
                P \coloneqq \coprod_{x \in M} P_x
            \end{equation}
            とおく。
        \item 射影$p \colon P \to M$を
            \begin{equation}
                p((x, u)) \coloneqq x 
            \end{equation}
            で定義する。
        \item $\GL(r, \R)$の$P$への右作用$\beta$を
            次のように定める:
            \begin{equation}
                \beta \colon P \times \GL(r, \R) \to P,
                \quad
                ((x, u), s) \mapsto (x, u \circ s)
            \end{equation}
        \item 各$\alpha \in A$に対し、
            $U_\alpha$上の$E$の局所自明化$\rho_\alpha$をひとつ選び、
            それにより定まる$E$のフレームを
            $e_1^{(\alpha)}, \dots, e_r^{(\alpha)}$とおく。
            写像$\sigma_\alpha \colon U_\alpha \to p^{-1}(U_\alpha)$を
            次のように定める:
            \begin{itemize}
                \item 各$x \in U_\alpha$に対し、
                    $E_x$の基底$e_1^{(\alpha)}(x), \dots, e_r^{(\alpha)}(x)$により
                    定まる線型同型$\R^r \to E_x$を
                    一時的な記号で$\sigma_\alpha(x)_2$と書く。
                \item $\sigma_\alpha(x) \coloneqq (x, \sigma_\alpha(x)_2)$と定める。
                    記号の濫用で$\sigma_\alpha(x)_2$も$\sigma_\alpha(x)$と書く。
            \end{itemize}
        \item 写像$\varphi_\alpha$を
            \begin{equation}
                \varphi_\alpha
                    \colon p^{-1}(U_\alpha) \to U_\alpha \times \GL(r, \R),
                    \quad
                    (x, \sigma_\alpha(x) \circ s) \mapsto (x, s)
            \end{equation}
            と定める。
            ただし、$(x, \sigma_\alpha(x) \circ s)$から
            $s$が一意に定まることは
            $s = \sigma_\alpha(x)^{-1} \circ \sigma_\alpha(x) \circ s$
            と表せることよりわかる。
            また、$\varphi_\alpha$は明らかに可逆である。
        \item 写像族$\{ \varphi_\alpha \}$を用いて
            $P$に多様体構造が入る (このあとすぐ示す)。
        \item $p \colon P \to M$は、
            $\{ \sigma_\alpha \colon U_\alpha \to p^{-1}(U_\alpha) \}$
            を切断の族、
            これにより定まる変換関数を$\{ \rho_{\alpha\beta} \}$として
            $M$上の主$\GL(r, \R)$束となる (このあとすぐ示す)。
    \end{enumerate}
    $P$は$E$に\term{同伴する}[associated]{同伴する}[どうはんする]
    主ファイバー束と呼ばれる。
\end{definition}

\begin{proof}
    $\GL(r, \R) = \R^{r^2}$と同一視する。
    まず$P$に多様体構造が入ることを示す。
    $M$の atlas $\{ (U_\alpha, \psi_\alpha) \}$は、
    小さい範囲に制限した chart、すなわち
    \begin{equation}
        (U'_\alpha, \psi_\alpha|_{U'_\alpha})
        \quad
        (\alpha \in A, \; U'_\alpha \opensubset U_\alpha)
    \end{equation}
    をすべて含むとしてよい。
    写像族$\{ \Phi_\alpha \colon p^{-1}(U_\alpha) \to \R^{n + r^2} \}$を
    \begin{equation}
        \begin{tikzcd}
            p^{-1}(U_\alpha)
                \ar{r}{\varphi_\alpha}
                \ar[bend right=30, end anchor=south west]{rr}[swap]{\Phi_\alpha}
                & U_\alpha \times \GL(r, \R)
                \ar{r}{\psi_\alpha \times \id}
                & \psi_\alpha(U_\alpha) \times \R^{r^2}
                \subset \R^{n + r^2}
        \end{tikzcd}
    \end{equation}
    を可換にするものとして定める。
    $P$に$\{ \Phi_\alpha \}$を atlas とする多様体構造が入ることを示すため、
    Smooth Manifold Chart Lemma (\cref{lemma:smooth-manifold-chart-lemma})
    の条件を確認する。
    $\varphi_\alpha$が可逆であることと
    $\psi_\alpha$が$M$の chart であることから、
    $\Phi_\alpha$は$\R^{n + r^2}$の開部分集合
    $\psi_\alpha(U_\alpha) \times \R^{r^2}$への全単射である。
    よって (i) が満たされる。

    各$\alpha, \beta \in A$に対し
    $\psi_\alpha, \psi_\beta$が$M$の chart であることから
    \begin{align}
        \Phi_\alpha(p^{-1}(U_\alpha) \cap p^{-1}(U_\beta))
            = \psi_\alpha(U_\alpha \cap U_\beta) \times \R^{r^2} \\
        \Phi_\beta(p^{-1}(U_\alpha) \cap p^{-1}(U_\beta))
            = \psi_\beta(U_\alpha \cap U_\beta) \times \R^{r^2}
    \end{align}
    はいずれも$\R^{n + r^2}$の開部分集合である。
    よって (ii) が満たされる。

    各$\alpha, \beta \in A$に対し
    合成写像$\varphi_\beta \circ \varphi_\alpha^{-1}$は
    \begin{equation}
        \begin{tikzcd}
            (U_\alpha \cap U_\beta) \times \GL(r, \R)
                \ar{r}{\varphi_\alpha^{-1}}
                & \pi^{-1}(U_\alpha \cap U_\beta)
                \ar{r}{\varphi_\beta}
                & (U_\alpha \cap U_\beta) \times \GL(r, \R) \\[-1em]
            (x, s)
                \ar[mapsto]{r}
                & (x, \sigma_\alpha(x) \circ s)
                \ar[mapsto]{r}
                & (x, \sigma_\beta(x)^{-1} \circ \sigma_\alpha(x) \circ s)
        \end{tikzcd}
    \end{equation}
    という対応を与えるが、
    ここで$\sigma_\beta(x)^{-1} \circ (\sigma_\alpha(x)) \circ s$は
    $(x, s)$に関し{\smooth}である。
    \begin{innerproof}
        $s$を右から合成する演算は
        Lie 群$\GL(r, \R)$における積なので{\smooth}である。
        そこで$\sigma_\beta(x)^{-1} \circ \sigma_\alpha(x)$について考える。
        いま各$x \in U_\alpha \cap U_\beta$に対し
        \begin{equation}
            \begin{tikzcd}
                \R^r \ar{rr}{\sigma_\beta(x)^{-1} \circ \sigma_\alpha(x)}
                    \ar{dr}[swap]{\sigma_\beta(x)}
                    & & \R^r \ar{dl}{\sigma_\alpha(x)} \\
                & E_x
            \end{tikzcd}
        \end{equation}
        は可換であるが、
        $\sigma_\alpha, \sigma_\beta$は定め方から
        $E$の局所自明化の$E_x$への制限$\rho_\alpha(x), \rho_\beta(x)$の逆写像である。
        よって写像
        \begin{equation}
            U_\alpha \cap U_\beta \to \GL(r, \R),
            \quad
            x \mapsto \sigma_\beta(x)^{-1} \circ \sigma_\alpha(x)
        \end{equation}
        は$E$の変換関数$\rho_{\beta\alpha}$に他ならず、
        したがってこれは{\smooth}である。
        よって、$\sigma_\beta(x)^{-1} \circ (\sigma_\alpha(x)) \circ s$は
        $(x, s)$に関し{\smooth}である。
    \end{innerproof}
    したがって
    \begin{equation}
        \Phi_\beta \circ \Phi_\alpha^{-1}
            = (\psi_\beta \times \id) \circ \varphi_\beta
                \circ \varphi_\alpha^{-1}
                \circ (\psi_\alpha \times \id)^{-1}
    \end{equation}
    は$\Phi_\alpha(p^{-1}(U_\alpha) \cap p^{-1}(U_\beta))$上{\smooth}である。
    よって (iii) が満たされる。

    $\{ (U_\alpha, \psi_\alpha) \}$は
    小さい範囲に制限した chart をすべて含むことから
    明らかに (iv) が満たされる。

    以上で Smooth Manifold Chart Lemma の条件が確認できた。
    したがって$P$は
    $\{ (p^{-1}(U_\alpha), \Phi_\alpha) \}$を atlas として多様体となる。

    つぎに、$P$は
    $\{ \sigma_\alpha \colon U_\alpha \to p^{-1}(U_\alpha) \}$
    を切断の族として
    $M$上の主$\GL(r, \R)$束となることを示す。
    そのためには次を示せばよい:
    \begin{enumerate}
        \item $p$が{\smooth}であること
        \item 作用$\beta$がファイバーを保つこと
        \item 作用$\beta$がファイバー上単純推移的であること
        \item 作用$\beta$が{\smooth}であること
        \item $\sigma_\alpha$が$U_\alpha$上の$P$の切断となること
        \item 主ファイバー束の定義の局所自明性が満たされること
        \item $\{ \sigma_\alpha \}$により定まる$P$の変換関数が
            $\{ \rho_{\alpha\beta} \}$であること
    \end{enumerate}
    ここで、$\varphi_\alpha$らは diffeo である。実際、図式
    \begin{equation}
        \begin{tikzcd}
            p^{-1}(U_\alpha)
                \ar{r}{\varphi_\alpha}
                \ar[bend right=30, end anchor=south west]{rr}[swap]{\Phi_\alpha}
                & U_\alpha \times \GL(r, \R)
                \ar{r}{\psi_\alpha \times \id}
                & \psi_\alpha(U_\alpha) \times \R^{r^2}
                \subset \R^{n + r^2}
        \end{tikzcd}
    \end{equation}
    が可換であることと$\psi_\alpha \times \id, \; \Phi_\alpha$が
    diffeo であることから従う。

    $p$が{\smooth}であることは
    各点の近傍での{\smooth}性を示せばよいが、これは
    各$(x, u) \in P$に対し$p^{-1}(U_\alpha)$が開近傍となるような
    $\alpha \in A$がとれて
    \begin{equation}
        \begin{tikzcd}
            p^{-1}(U_\alpha)
                \ar{rd}[swap]{p}
                \ar{r}{\Phi_\alpha}
                & p^{-1}(U_\alpha) \times \R^{n + r^2}
                \ar{d}{\mathrm{pr}_1} \\
            & P
        \end{tikzcd}
    \end{equation}
    が可換となることから従う。

    $\GL(r, \R)$の$P$への作用
    \begin{equation}
        \beta((x, u), s)
            = (x, u \circ s)
    \end{equation}
    がファイバーを保つことは定義から明らか。

    $\beta$がファイバー$P_x = p^{-1}(x) \; (x \in M)$上単純推移的であることは、
    shear map
    \begin{equation}
        P_x \times \GL(r, \R) \to P_x \times P_x,
        \quad
        ((x, u), s) \mapsto ((x, u \circ s), (x, u))
    \end{equation}
    が逆写像
    \begin{equation}
        P_x \times P_x \to P_x \times \GL(r, \R),
        \quad
        ((x, t), (x, u)) \mapsto ((x, u), u^{-1} \circ t)
    \end{equation}
    を持つことから従う。

    $\beta$が{\smooth}であることを示す。
    $(x, u) \in P$の近傍$U_\alpha$上で
    \begin{equation}
        (x, u) = (x, \sigma_\alpha(x) \circ t)
        \quad
        (t \in \GL(r, \R))
    \end{equation}
    の形に書けることに注意すれば、
    \begin{alignat}{1}
            &((x, u), s) \in p^{-1}(U_\alpha) \times \GL(r, \R) \\
        \overset{
            \mathclap{\id \times (\mathrm{pr}_2 \circ \varphi_\alpha)}
        }{\mapsto} \qquad
            &((x, u), s, t)
            \in p^{-1}(U_\alpha) \times \GL(r, \R) \times \GL(r, \R) \\
        \overset{\mathclap{\text{$\GL(r, \R)$での積}}}{\mapsto} \qquad
            &((x, u), ts)
            \in p^{-1}(U_\alpha) \times \GL(r, \R) \\
        \overset{p}{\mapsto} \qquad
            &(x, ts)
            \in U_\alpha \times \GL(r, \R) \\
        \overset{\mathclap{\varphi_\alpha^{-1}}}{\mapsto} \qquad
            &(x, \sigma_\alpha(x) \circ ts)
            = (x, u \circ s)
            \in p^{-1}(U_\alpha)
    \end{alignat}
    の各写像が{\smooth}であることから、
    $\beta$は$U_\alpha$上{\smooth}であることがわかる。
    したがって$\beta$は{\smooth}である。

    $\sigma_\alpha$が$U_\alpha$上の$P$の切断となることを示す。
    $p \circ \sigma_\alpha(x) = x$となることは定義から明らか。
    {\smooth}性は
    \begin{equation}
        \sigma_\alpha(x)
            = \varphi_\alpha^{-1}(x, 1)
    \end{equation}
    よりわかる。
    したがって$\sigma_\alpha$は$U_\alpha$上の$P$の切断である。
    さらに$\varphi_\alpha$の定義と$\varphi_\alpha$が diffeo であることから
    主ファイバー束の定義の局所自明性も満たされる。

    最後に、$x \in U_\alpha \cap U_\beta, \; \alpha, \beta \in A$に対し
    \begin{equation}
        \sigma_\beta(x)
            = \sigma_\alpha(x) \circ \sigma_\alpha^{-1} \circ \sigma_\beta(x)
            = \sigma_\alpha(x) \circ \rho_{\alpha\beta}(x)
    \end{equation}
    が成り立つことから、
    $\{ \sigma_\alpha \}$により定まる$P$の変換関数は
    $\{ \rho_{\alpha\beta} \}$である。

    以上で$P$は
    $\{ \sigma_\alpha \}$を切断の族とし、
    これにより定まる$P$の変換関数を
    $\{ \rho_{\alpha\beta} \}$として
    $M$上の主$\GL(r, \R)$束となることが示せた。
\end{proof}

\begin{example}[構造群の縮小]
    $E$をベクトル束、
    $g$を$E$の内積とする。
    フレーム束の定義の$P_x$を
    \begin{equation}
        Q_x \coloneqq \{ u \colon \R^r \to E_x
            \mid u \text{ は線型同型かつ内積を保つ}
        \}
    \end{equation}
    に置き換えると、$Q$は
    直交群$O(r)$を構造群とする$M$上の主束となる。
    このとき$Q$は$P$の部分束であり、
    $Q$は$P$の構造群$\GL(r, \R)$を$O(r)$に
    \term{縮小}[reduction]{縮小}[しゅくしょう]
    して得られたという。
\end{example}

\subsection{主ファイバー束からベクトル束へ}
\label[subsection]{subsec:principal-fiber-bundle-to-vector-bundle}

逆に主$G$束$P$と
表現$\rho \colon G \to \GL(r, \R)$が与えられると、
ランク$r$ベクトル束$E$が構成できる。

\begin{definition}[同伴するベクトル束]
    $M$を多様体、$P \to M$を主$G$束、
    $\rho \colon G \to \GL(r, \R)$を Lie 群の表現とする。
    直積多様体$P \times \R^r$への
    $G$の{\smooth}右作用を
    \begin{equation}
        (P \times \R^r) \times G \to P \times \R^r,
        \quad
        ((u, y), s) \mapsto (u.s, \rho(s)^{-1} y)
    \end{equation}
    で定め、軌道空間$(P \times \R^r) / G$を
    \begin{equation}
        P \times_\rho \R^r
    \end{equation}
    と書く。
    このとき、$P \times_\rho \R^r$は
    $M$上のベクトル束となり、
    $P$のある変換関数$\{ \psi_{\alpha\beta} \}$に対し
    $\{ \rho \circ \psi_{\alpha\beta} \}$が
    $P \times_\rho \R^r$の変換関数のひとつとなる
    (このあとすぐ示す)。
    これを$P$に
    \term{同伴する}[associated]{同伴する}[どうはんする]
    ベクトル束という。
\end{definition}

\begin{proof}
    $P \times_\rho \R^r$が$M$上のベクトル束になることを、
    Vector Bundle Chart Lemma を用いて示す。
   標準射影$P \to M$および
    $P \times \R^r \to P \times_\rho \R^r$を
    それぞれ$p, q$とおく。

    まず射影を構成する。図式
    \begin{equation}
        \begin{tikzcd}
            P \times \R^r
                \ar{d}[swap]{\mathrm{pr}_1}
                \ar{r}{q}
                & P \times_\rho \R^r
                \ar[dashed]{d}{\pi} \\
            P \ar{r}[swap]{p}
                & M
        \end{tikzcd}
    \end{equation}
    において、$p \circ \mathrm{pr}_1$は$q$のファイバー上定値である。
    \begin{innerproof}
        $u \in P_x, \; u' \in P_{x'} \; (x, x' \in M),
        \; y, y' \in \R^r$について
        $q(u, y) = q(u', y')$ならば、
        $q$の定義から
        ある$s \in G$が存在して
        $(u, y) = (u' . s, \rho(s)^{-1} y')$が成り立ち、
        とくに$u = u' . s$だが、
        $G$の$P$への作用がファイバーを保つことから
        $x = x'$が成り立つ。
    \end{innerproof}
    したがって
    写像$\pi \colon P \times_\rho \R^r \to M$が誘導される。
    このとき$p \circ \mathrm{pr}_1$が全射であることより
    $\pi$も全射である。

    つぎに$P \times_\rho \R^r$の局所自明化を構成する。
    $P$の切断の族$\{ \sigma_\alpha \colon U_\alpha \to P \}_{\alpha \in A}$
    であって$\bigcup U_\alpha = P$なるものをひとつ選ぶ。
    これにより定まる$P$の局所自明化の族を$\{ \varphi_\alpha \}$とおき、
    さらにこれにより定まる$P$の変換関数を$\{ \psi_{\alpha\beta} \}$とおく。
    このとき、各$\alpha \in A$に対し図式
    \begin{equation}
        \begin{tikzcd}[column sep=large]
            U_\alpha  \times \R^r
                \ar[dashed]{drr}
                \ar{r}{\substack{(x, y) \\ \; \mapsto (x, 1, y)}}
                & U_\alpha \times G \times \R^r
                \ar{r}{\varphi_\alpha^{-1} \times \id}
                & p^{-1}(U_\alpha) \times \R^r
                \ar{d}{q} \\
            && p^{-1}(U_\alpha) \times_\rho \R^r
                = \pi^{-1}(U_\alpha)
        \end{tikzcd}
    \end{equation}
    の破線部の写像は全単射である。
    \begin{innerproof}
        $(u, y), (u', y') \in U_\alpha \times \R^r$について
        \begin{alignat}{1}
                &q(\varphi_\alpha^{-1}(u, 1), y)
                    = q(\varphi_\alpha^{-1}(u', 1), y') \\
            \iff
                &\exists s \in G
                \quad \text{s.t.} \quad
                \begin{cases}
                    \varphi_\alpha^{-1}(u, 1) = \varphi_\alpha^{-1}(u', 1) . s \\
                    y = \rho(s)^{-1} y'
                \end{cases} \\
            \iff
                &\exists s \in G
                \quad \text{s.t.} \quad
                \begin{cases}
                    \varphi_\alpha^{-1}(u, 1) = \varphi_\alpha^{-1}(u', s) \\
                    y = \rho(s)^{-1} y'
                \end{cases} \\
            \iff
                &\exists s \in G
                \quad \text{s.t.} \quad
                \begin{cases}
                    (u, 1) = (u', s) \\
                    y = \rho(s)^{-1} y'
                \end{cases} \\
            \iff
                &\begin{cases}
                    u = u' \\
                    y = y'
                \end{cases}
        \end{alignat}
    \end{innerproof}
    ただし、図式の右下が
    $p^{-1}(U_\alpha) \times_\rho \R^r = \pi^{-1}(U_\alpha)$であることは
    次のようにしてわかる。
    \begin{innerproof}
        $(\subset)$ \quad
        \begin{align}
            \pi(p^{-1}(U_\alpha) \times_\rho \R^r)
                &= \pi \circ q(p^{-1}(U_\alpha) \times \R^r) \\
                &= p \circ \mathrm{pr}_1 (p^{-1}(U_\alpha) \times \R^r) \\
                &= p \circ p^{-1}(U_\alpha) \\
                &\subset U_\alpha
        \end{align}
        より$p^{-1}(U_\alpha) \times_\rho \R^r \subset \pi^{-1}(U_\alpha)$である。

        \noindent
        $(\supset)$ \quad
        $(u, y) \in p^{-1}(U_\alpha) \times \R^r$について
        $\pi(q(u, y)) \in U_\alpha$ならば
        \begin{equation}
            p(u) = p \circ \mathrm{pr}_1(u, y) \in U_\alpha
        \end{equation}
        だから$(u, y) \in p^{-1}(U_\alpha) \times \R^r$、
        したがって$q(u, y) \in p^{-1}(U_\alpha) \times_\rho \R^r$である。
    \end{innerproof}
    そこで、破線矢印の逆向きの写像$\pi^{-1}(U_\alpha) \to U_\alpha \times \R^r$を
    $\Phi_\alpha$とおく。
    各$x \in M$に対し、
    $x \in U_\alpha$なる$\alpha \in A$をひとつ選べば、
    $\Phi_\alpha(x) \colon \pi^{-1}(x) \to \{ x \} \times \R^r = \R^r$
    は可逆である。
    実際、
    \begin{equation}
        \{ x \} \times \R^r \to \pi^{-1}(x),
        \quad
        (x, y) \mapsto q(\varphi^{-1}(x, 1), y)
    \end{equation}
    が逆写像を与える。
    そこで、この 1:1 対応により$\pi^{-1}(x)$に
    $r$次元$\R$-ベクトル空間の構造を入れる。

    最後に、$U_\alpha \cap U_\beta \neq \emptyset$なる$\alpha, \beta \in A$と
    $(x, y) \in (U_\alpha \cap U_\beta) \times \R^r$に対し
    \begin{alignat}{1}
        \Phi_\alpha \circ \Phi_\beta^{-1} (x, y)
            = (x, \rho \circ \psi_{\alpha\beta} y)
    \end{alignat}
    が成り立つ。
    \begin{innerproof}
        まず
        \begin{alignat}{1}
            \Phi_\alpha \circ \Phi_\beta^{-1} (x, y)
                &= \Phi_\alpha(q(\varphi_\beta^{-1}(x, 1), y)) \\
                &= \Phi_\alpha(q(\varphi_\beta(x)^{-1}(1), y))
        \end{alignat}
        である。このとき
        \begin{align}
            (\varphi_\beta(x)^{-1}(1), y)
            &= (\sigma_\beta(x), y) \\
            &= (\sigma_\alpha(x) . \psi_{\alpha\beta}(x), y)
        \end{align}
        が成り立つから
        \begin{align}
            \Phi_\alpha(q(\varphi_\beta(x)^{-1}(1), y))
                &= \Phi_\alpha(q(\sigma_\alpha(x) . \psi_{\alpha\beta}(x), y)) \\
                &= \Phi_\alpha(
                    \sigma_\alpha(x),
                    \rho(\psi_{\alpha\beta}(x)^{-1})^{-1} y
                ) \\
                &= \Phi_\alpha(
                    \varphi_\alpha(x)^{-1}(1),
                    \rho(\psi_{\alpha\beta}(x)) y
                ) \\
                &= \Phi_\alpha \circ \Phi_\alpha^{-1}(
                    x,
                    \rho(\psi_{\alpha\beta}(x)) y
                ) \\
                &= (x, \rho \circ \psi_{\alpha\beta} y)
        \end{align}
        となる。
    \end{innerproof}
    $\rho, \psi_{\alpha\beta}$はいずれも{\smooth}だから
    $\rho \circ \psi_{\alpha\beta} \colon U_\alpha \cap U_\beta \to \GL(r, \R)$
    も{\smooth}である。

    以上で Vector Bundle Chart Lemma の条件が確認できた。
    したがって$P \times_\rho \R^r$は$M$上のベクトル束となり、
    $\{ \Phi_\alpha \}$は$P \times_\rho \R^r$の局所自明化の族となり、
    これにより定まる$P \times_\rho \R^r$の変換関数は
    $\{ \rho \circ \psi_{\alpha\beta} \}$である。
\end{proof}

\begin{example}[ベクトル束のフレーム束に同伴するベクトル束]
    $E \to M$をランク$r$ベクトル束、
    $\{ \psi_{\alpha\beta} \}$を$E$の変換関数、
    $P$を$E$から構成されたフレーム束とする。
    フレーム束の定義より、
    $\{ \psi_{\alpha\beta} \}$も$P$の変換関数であった。
    よって表現$\rho \colon \GL(r, \R) \to \GL(r, \R)$を
    恒等写像とすれば、
    $P \times_\rho \R^r$の変換関数は
    $\{ \rho \circ \psi_{\alpha\beta} = \psi_{\alpha\beta} \}$となり、
    $P \times_\rho \R^r$が$E$に一致することがわかる。
\end{example}

\begin{example}[直和束]
    \TODO{}
\end{example}

\begin{example}[テンソル積束]
    \TODO{$\rho(s) = s \otimes s$}
\end{example}

\begin{example}[双対束]
    \TODO{$\rho(s) = \up{t}s^{-1}$}
\end{example}


% ============================================================
%
% ============================================================
\chapter{アファイン接続}

接続の概念に慣れるため、
まずは多様体の接束の接続であるアファイン接続からはじめる。

\section{アファイン接続}
\label[section]{sec:affine-connection}

\begin{definition}[ベクトル束の接続]
    $M$を多様体とする。
    $M$の\term{アファイン接続}[affine connection]{アファイン接続}[あふぁいんせつぞく]とは、
    $\R$-線型写像$A^0(TM) \to A^1(TM)$であって、
    Leibniz の公式
    \begin{equation}
        \nabla(fY) = df \otimes Y + f \nabla Y
            \quad (f \in A^0(M),\; Y \in A^0(TM))
    \end{equation}
    をみたすものである。
    各$Y \in A^0(M),\; X \in \frakX(M)$に対し、
    $\nabla Y (X) \in A^0(TM)$を$\nabla_X Y$とも書き、
    $Y$の$X$方向の\term{共変微分}[covariant derivative]{共変微分}[きょうへんびぶん]と呼ぶ。
\end{definition}

\begin{example}[アファイン接続の例]
    ~
    \begin{itemize}
        \item \TODO{座標を明示せよ} $\R^n$のアファイン接続$\wb{\nabla}$を
            \begin{equation}
                \wb{\nabla}_X Y
                    \coloneqq X(Y^1) \deldel{x^1} + \dots + X(Y^n) \deldel{x^n}
                    = X(Y^i) \deldel{x^i}
            \end{equation}
            で定めることができる。
            $\wb{\nabla}$を
            \term{Euclid 接続}[Euclidean connection]{Euclid 接続}[Euclid せつぞく]
            という。
            $X$を書かずに表せば
            \begin{equation}
                \wb{\nabla} Y
                    = dY^i \deldel{x^i}
            \end{equation}
            となる。
            \cref{def:vector-bundle-connection}の Leibniz の公式の成立を確かめると、
            \begin{alignat}{1}
                \wb{\nabla}(fY)
                    &= d(fY^i) \deldel{x^i} \\
                    &= d(fY^i) \otimes \deldel{x^i}
                        \quad (\text{$TM$-値微分形式の同一視}) \\
                    &= (Y^i \,df + f \,dY^i) \otimes \deldel{x^i} \\
                    &= df \otimes Y + f \cdot \wb{\nabla} Y
            \end{alignat}
            より確かに成り立つ。
            \TODO{接続係数が0であることを述べる}
        \item \TODO{tangential connection}
    \end{itemize}
\end{example}

次に定義する接続形式とは、
局所フレームに関する接続の行列表示のようなものである。

\begin{definition}[接続形式]
    $M$を多様体、$U \opensubset M$、
    $(E_i)$を$U$上の$TM$の局所フレームとする。
    \begin{itemize}
        \item 各$j$に対し、
            \begin{equation}
                \nabla E_j = \omega^k_j E_k
            \end{equation}
            と表したときの$1$-形式$\omega^k_j$らの族$\omega \coloneqq (\omega^k_j)$を
            $\nabla$の\term{接続形式}[connection form]{接続形式}[せつぞくけいしき]という。
        \item {\smooth}関数$\Gamma_{ij}^k \colon U \to \R,$
            \begin{equation}
                \Gamma^k_{ij} \coloneqq \omega^k_j(E_i)
            \end{equation}
            を$\nabla$の
            \term{接続係数}[connection coefficient]{接続係数}[せつぞくけいすう]
            という。
            定義から明らかに、接続係数は
            \begin{equation}
                \nabla_{E_i} E_j = \Gamma^k_{ij} E_k
            \end{equation}
            をみたす。
    \end{itemize}
\end{definition}


% ------------------------------------------------------------
%
% ------------------------------------------------------------
\section{捩率と曲率}

この節では捩率テンソルと曲率テンソルを定義する。

\begin{definition}[捩率テンソル]
    $M$を多様体、
    $\nabla$を$M$のアファイン接続とする。
    このとき
    \begin{equation}
        T \colon \frakX(M) \times \frakX(M) \to \frakX(M),
        \quad
        (X, Y) \mapsto \frac{1}{2} (\nabla_X Y - \nabla_Y X - [X, Y])
    \end{equation}
    と定義すると
    $T$は交代$\smooth(M)$-双線型写像となる (このあと示す)。
    そこで$T$は$M$上の$TM$に値をもつ2次形式とみなせて、
    これを接続$\nabla$の
    \term{捩率テンソル}[torsion tensor]{捩率テンソル}[れいりつてんそる]
    という。
    捩率の値が$M$上恒等的に$0$であるとき、
    接続$\nabla$は
    \term{捩れなし}[torsion-free]{捩れなし}[ねじれなし]
    あるいは
    \term{対称}[symmetric]{対称}[たいしょう]であるという\footnote{
        「対称」という語は、
        接続が対称であるための必要十分条件
        $\Gamma^k_{ij} = \Gamma^k_{ji}$からきている
        \cite[p.121]{Lee18}。
    }。
\end{definition}

\begin{proof}
    \TODO{}
\end{proof}

\begin{definition}[曲率テンソル]
    $M$を多様体、
    $\nabla$を$M$のアファイン接続とする。
    このとき、
    \begin{equation}
        R \colon
            \frakX(M) \times \frakX(M) \times \frakX(M)
            \to \frakX(M),
        \quad
        (X, Y, Z)
            \mapsto \nabla_X \nabla_Y Z - \nabla_Y \nabla_X Z - \nabla_{[X, Y]} Z
    \end{equation}
    は$M$上の$(1, 3)$-テンソル場となり、これを
    接続$\nabla$の
    \term{曲率テンソル}[curvature tensor]{曲率テンソル}[きょくりつてんそる]
    という。
\end{definition}

捩率テンソルは、局所的には接続形式を用いて表せる。

\begin{proposition}[第1構造方程式]
    $e_1, \dots, e_n$を$TM$の局所フレーム、
    $\theta^1, \dots, \theta^n$をその双対フレームとする。
    捩率$T$は$A^2(TM)$の元だから
    \begin{equation}
        T = \sum_{i = 1}^n \Theta^i \otimes e_i
            \quad (\Theta^i \in A^2(M))
    \end{equation}
    と表せる。
    すると
    \begin{equation}
        \Theta^i(e_k, e_l) = d\theta^i(e_k, e_l) + \omega^i_j \wedge \theta^j (e_k, e_l)
    \end{equation}
    すなわち
    \begin{equation}
        \Theta^i = d\theta^i + \omega^i_j \wedge \theta^j
    \end{equation}
    が成り立つ。これをアファイン接続$\nabla$の
    \term{第1構造方程式}[first structure equation]{第1構造方程式}[だい1こうぞうほうていしき]
    という。
\end{proposition}

\begin{proof}
    \TODO{}
\end{proof}

Bianchi の第1恒等式は、
曲率テンソルの非対称性を捩率を用いて表すものである。

\begin{proposition}[Bianchi の第1恒等式]
    \begin{equation}
        R(X, Y)Z + R(Y, Z)X + R(Z, X)Y = 2(DT)(X, Y, Z)
    \end{equation}
    接続形式で書けば
    \begin{equation}
        \Omega^i_j \wedge \theta^j = d\Theta^i + \omega^i_j \wedge \Theta^j
    \end{equation}
    \TODO{}
\end{proposition}

\begin{proof}
    \TODO{}
\end{proof}

次に定義する曲率形式とは、
局所フレームに関する曲率テンソルの行列表示のようなものである。

\begin{definition}[曲率形式]
    $M$を多様体、$U \opensubset M$とし、
    $(E_i)$を$U$上の$TM$の局所フレームとする。
    各$j$に対し、
    \begin{equation}
        R(X, Y)(E_j) = \Omega^k_j(X, Y) E_k
    \end{equation}
    と表したときの$2$-形式$\Omega^k_j$らの族$\Omega \coloneqq (\Omega^k_j)$を
    $\nabla$の\term{曲率形式}[curvature form]{曲率形式}[きょくりつけいしき]という。
\end{definition}

曲率は、局所的には接続形式を用いて表せる。

\begin{proposition}[第2構造方程式]
    \label[proposition]{prop:second-structure-equation}
    $M$を多様体、$U \opensubset M$とし、
    $(E_i)$を$U$上の$TM$の局所フレームとする。
    このとき、曲率形式に関する方程式
    \begin{equation}
        \Omega^k_j
            = d\omega^k_j + \omega^k_i \wedge \omega^i_j
    \end{equation}
    が成り立つ。これを接続$\nabla$の
    \term{第2構造方程式}[second structure equation]{第2構造方程式}[だい2こうぞうほうていしき]
    という。
\end{proposition}

\begin{proof}
    \TODO{}
\end{proof}

Bianchi の第2恒等式は、
曲率テンソルの共変外微分が消えることを表す。

\begin{proposition}[Bianchi の第2恒等式]
    \begin{equation}
        DR = 0
    \end{equation}
    接続形式で書けば
    \begin{equation}
        d\Omega^\mu_\lambda
            - \Omega^\mu_\nu \wedge \omega^\nu_\lambda
            + \omega^\mu_\nu \wedge \Omega^\nu_\lambda
            = 0
    \end{equation}
    \TODO{}
\end{proposition}

\begin{proof}
    \TODO{}
\end{proof}

Ricci の恒等式は、
共変微分の非可換性を表すものである。

\begin{proposition}[Ricci の恒等式]
    \begin{equation}
        \frac{1}{2}(\nabla^2 K(X, Y) - \nabla^2 K(Y, X))
            = -R(X, Y) K + \nabla_{T(X, Y)} K
    \end{equation}
    \TODO{}
\end{proposition}

\begin{proof}
    \TODO{}
\end{proof}


% ------------------------------------------------------------
%
% ------------------------------------------------------------
\section{平行移動}

この節では測地線について述べた後、その一般化として平行の概念を導入する。

\begin{definition}[測地線]
    $M$を多様体とし、$\nabla$を$M$のアファイン接続とする。
    $M$上の曲線$\gamma$が$M$の\term{測地線}[geodesic]{測地線}[そくちせん]であるとは、
    $\gamma'$方向の$\gamma'$の共変微分が恒等的に$0$であること、すなわち
    \begin{equation}
        \nabla_{\gamma'} \gamma' \equiv 0
    \end{equation}
    が成り立つことをいう。
\end{definition}

\begin{example}[測地線の例]
    \TODO{}
\end{example}

\begin{definition}[平行]
    $M$を多様体とし、$\nabla$を$M$のアファイン接続とする。
    $\gamma$を$M$上の曲線とする。
    $\gamma$に沿ったテンソル場$V$が
    $\gamma$に沿って\term{平行}[parallel]{平行}[へいこう]であるとは、
    $\gamma'$方向の$V$の共変微分が恒等的に$0$であること、すなわち
    \begin{equation}
        \nabla_{\gamma'} V \equiv 0
    \end{equation}
    が成り立つことをいう。
\end{definition}

\begin{example}[平行なテンソル場の例]
    \TODO{}
    ~
    \begin{itemize}
        \item $\gamma$が測地線であるとは、
            その速度ベクトル場が$\gamma$自身に沿って平行であることと同値である。
    \end{itemize}
\end{example}

\begin{definition}[平行移動]
    $M$を多様体とし、$\nabla$を$M$のアファイン接続とする。
    さらに、$\gamma \colon I \to M$を$M$上の曲線とし、
    $t_0 \in I,\; v \in T_{\gamma'(t_0)} M$とする。
    このとき、$\gamma$に沿って平行なベクトル場$V$であって
    $V(t_0) = v$をみたすものがただひとつ存在し、
    このような$V$を$\gamma$に沿った$v$の
    \term{平行移動}[parallel transport]{平行移動}[へいこういどう]という。
\end{definition}




% ============================================================
%
% ============================================================
\chapter{ベクトル束の接続}

ベクトル束の接続について考える。

% ------------------------------------------------------------
%
% ------------------------------------------------------------
\section{ベクトル束の接続}

\begin{definition}[ベクトル束の接続]
    \label[definition]{def:vector-bundle-connection}
    $M$を多様体、
    $\pi \colon E \to M$をベクトル束とする。
    $E$の\term{接続}[connection]{接続}[せつぞく]とは、
    $\R$-線型写像$\nabla \colon A^0(E) \to A^1(E)$であって、
    Leibniz の公式
    \begin{equation}
        \nabla(f\xi) = df \otimes \xi + f \nabla\xi
            \quad (f \in A^0(M),\; \xi \in A^0(E))
    \end{equation}
    をみたすものである。
    各$\xi \in A^0(E),\; X \in \frakX(M)$に対し、
    $\nabla\xi(X) \in A^0(E)$を$\nabla_X\xi$とも書き、
    $\xi$の$X$方向の\term{共変微分}[covariant derivative]{共変微分}[きょうへんびぶん]と呼ぶ。
\end{definition}

定義からわかるように、
接続$\nabla$は$\smooth(M)$-線型ではないが、
2つの接続$\nabla, \nabla'$の差$\nabla - \nabla'$は$\smooth(M)$-線型である。
この事実はたとえば情報幾何学において、
$m$-接続と$e$-接続の差によって Amari-Chentsov テンソルを定義する際に用いられる。

\begin{proposition}[接続の差は$\smooth(M)$-線型]
    $M$を多様体、
    $E$を$M$上のベクトル束、
    $\nabla, \nabla'$を$E$の接続とする。
    このとき
    $\nabla - \nabla' \colon A^0(E) \to A^1(E), \;
        \xi \mapsto \nabla\xi - \nabla'\xi$
    は$\smooth(M)$-線型である。
\end{proposition}

\begin{proof}
    \begin{alignat}{1}
        (\nabla - \nabla')(f\xi)
            &= \nabla(f\xi) - \nabla'(f\xi) \\
            &= df \otimes \xi + f \nabla\xi - df \otimes \xi - f \nabla'\xi \\
            &= f (\nabla - \nabla')\xi
    \end{alignat}
    より従う。
\end{proof}

% ------------------------------------------------------------
%
% ------------------------------------------------------------
\section{接続形式}

接続形式を導入する。接続形式は、接続の座標表示にあたるものである。

\begin{definition}[接続形式]
    $M$を多様体、
    $E \to M$をランク$r$のベクトル束、
    $\nabla$を$E$の接続とする。
    さらに$U \opensubset M$、
    $\calE \coloneqq (e_1, \dots, e_r)$を$U$上の$E$のフレームとする。
    このとき、
    $U$上の$1$-形式の族$\omega = (\omega_\lambda^\mu)_{\lambda, \mu}$により
    \begin{equation}
        \nabla e_\lambda
            = \sum_{\mu} \omega_\lambda^\mu \otimes e_\mu
            \quad (\lambda = 1, \dots, r)
    \end{equation}
    と書ける。
    $\omega$をフレーム$\calE$に関する$\nabla$の
    \term{接続形式}[connection form]{接続形式}[せつぞくけいしき]という。
\end{definition}

もうひとつのフレームに関する接続形式を考えると、
ふたつの接続形式の間の変換規則が立ち現れる。

\begin{proposition}[接続形式の変換規則]
    上の定義の状況で、
    さらに$\calE' \coloneqq (e'_1, \dots, e'_r)$も$U$上の$E$のフレームとし、
    $\calE'$に関する$\nabla$の接続形式を$\omega'$とする。
    フレームの取り替えの行列$(a_\lambda^\mu)$は
    \begin{equation}
        e'_\lambda = \sum_{\mu} a_\lambda^\mu e_\mu
        \quad (a_\lambda^\mu \in A^0(U))
    \end{equation}
    とおく。
    このとき、接続形式の変換規則は
    \begin{equation}
        \omega' = a^{-1} \omega a + a^{-1} da
    \end{equation}
    となる。
\end{proposition}

\begin{proof}
    \TODO{}
\end{proof}

逆に、$\mathfrak{gl}(r, \R)$に値をもつ1-形式の族から接続を構成できる。

\begin{proposition}[接続形式から定まる接続]
    \label[proposition]{prop:connection-from-1-forms}
    $M$を多様体、
    $E \to M$をランク$r$のベクトル束とする。
    $\{ U_\alpha \}_{\alpha \in A}$を$M$の open cover であって
    各$U_\alpha$上で$g$に関するフレーム
    $\calE_\alpha = (e^{(\alpha)}_1, \dots, e^{(\alpha)}_r)$
    を持つものとする。
    さらに、各$U_\alpha$上の局所自明化$\varphi_\alpha$を
    $\calE_\alpha$から定め、
    変換関数を$\{ \psi_{\alpha\beta} \}$とおく。
    このとき、$\mathfrak{gl}(r, \R)$に値をもつ$1$-形式の族
    \begin{equation}
        \omega = \{ \omega_\alpha \}_{\alpha \in A}
    \end{equation}
    であって、変換規則
    \begin{equation}
        \omega_\beta
            = \psi_{\alpha\beta}^{-1} \omega_\alpha \psi_{\alpha\beta}
            + \psi_{\alpha\beta}^{-1} \, d \psi_{\alpha\beta}
            \quad
            \text{on $U_\alpha \cap U_\beta$}
    \end{equation}
    をみたすものが与えられたならば、
    次をみたす$E$の接続が構成できる:
    \begin{enumerate}
        \item 各フレーム$\calE_\alpha$に関する
            $\nabla$の接続形式は$\omega_\alpha$である。
    \end{enumerate}
\end{proposition}

\begin{proof}
    \TODO{}
\end{proof}

\TODO{大域的な与え方と接続形式による与え方}

\begin{definition}[直和束の接続]
    \TODO{}
\end{definition}

\begin{definition}[テンソル積束の接続]
    \TODO{}
\end{definition}

\begin{definition}[双対束の接続]
    \TODO{}
\end{definition}

\begin{definition}[引き戻し束の接続]
    \TODO{}
\end{definition}

% ------------------------------------------------------------
%
% ------------------------------------------------------------
\section{ベクトル束の共変外微分と曲率}

微分形式に対する外微分を一般化し、
ベクトル束に値をもつ微分形式に対し共変外微分とよばれる演算を定義する。
さらに共変外微分から曲率を定義する。

\begin{definition}[共変外微分]
    $M$を多様体、
    $E \to M$をベクトル束、
    $\nabla$を$E$の接続、
    $p \in \Z_{\ge 0}$とする。
    $\R$-線型写像$D \colon A^p(E) \to A^{p + 1}(E)$を
    \begin{equation}
        D(\theta \otimes \xi)
            \coloneqq d\theta \otimes \xi + \theta \wedge \nabla\xi
            \quad (\theta \in A^p(M), \; \xi \in A^0(E))
    \end{equation}
    で定め、$D$を
    \term{共変外微分}[covariant exterior derivative]{共変外微分}[きょうへんがいびぶん]
    という。
\end{definition}

\begin{remark}
    \label[remark]{remark:covariant-exterior-derivative-and-vector-fields}
    とくに$p = 1$のとき、
    $\varphi \in A^1(E)$に対し
    \begin{equation}
        (D\varphi)(X, Y)
            = \nabla_X (\varphi(Y))
            - \nabla_Y (\varphi(X))
            - \varphi([X, Y])
            \quad
            (X, Y \in \Gamma(TM))
    \end{equation}
    が成り立つ。
    これは通常の外微分の公式\cref{remark:exterior-derivative-and-vector-fields}
    の拡張になっている。
\end{remark}

共変外微分は次の性質をみたす。

\begin{proposition}[共変外微分の anti-derivation 性 (外積に関して)]
    $M$を多様体、
    $E \to M$をベクトル束、
    $\nabla$を$E$の接続、
    $D$を$\nabla$から定まる共変外微分とする。
    $p, q \in \Z_{\ge 0}$に対し
    \begin{equation}
        D(\theta \wedge \varphi)
            = d\theta \wedge \varphi + (-1)^p \theta \wedge D\varphi
            \quad
            (\theta \in A^p(M), \; \varphi \in A^q(E))
    \end{equation}
    が成り立つ\footnote{
        [小林]では$E$-値形式を表すときの$\xi$と$\theta$の順序が逆なので
        \begin{equation}
            D(\varphi \wedge \theta)
                = D\varphi \wedge \theta + (-1)^p \varphi \wedge d\theta
        \end{equation}
        という形になっている。
    }。
\end{proposition}

\begin{proof}
    \TODO{}
\end{proof}

ある性質を満たす双線型写像に対し、
共変外微分は anti-derivation 性をみたす。

\begin{proposition}[共変外微分の anti-derivation 性 (双線型写像に関して)]
    \label[proposition]{prop:signed-leibniz-rule}
    $M$を多様体、
    $E \to M, \; E' \to M$をベクトル束、
    $\nabla, \nabla'$をそれぞれ$E, E'$の接続、
    $D, D'$をそれぞれ$\nabla, \nabla'$から定まる共変外微分、
    $g \colon A^0(E) \times A^0(E') \to A^0(M)$を
    $\smooth(M)$-双線型写像とする。
    $D, D'$が条件
    \begin{equation}
        d(g(\xi, \eta)) = g(\nabla\xi, \eta) + g(\xi, \nabla'\eta)
            \quad
            (\xi \in A^0(E), \; \eta \in A^0(E'))
    \end{equation}
    をみたすならば\footnote{
        とくに$g$が$M$の Riemann 計量で
        $E = E' = TM$の状況でこの条件が成り立っているならば、
        $\nabla'$は$g$に関する$\nabla$の
        \term{双対接続}[dual connection]{双対接続}[そうついせつぞく]
        であるという。
    }、
    $p, q \in \Z_{\ge 0}$に対し
    \begin{equation}
        d(g(\xi, \eta))
            = g(D\xi, \eta) + (-1)^p g(\xi, D'\eta)
            \quad
            (\xi \in A^p(E), \; \eta \in A^q(E'))
    \end{equation}
    が成り立つ。
\end{proposition}

\begin{proof}
    Einstein の記法を用いる。
    $A^p(E), A^q(E')$の元はそれぞれ
    \begin{align}
        &\alpha \otimes \xi \in A^p(E)
            \quad (\alpha \in A^p(M), \; \xi \in A^0(E)) \\
        &\beta \otimes \eta \in A^q(E')
            \quad (\beta \in A^q(M), \; \eta \in A^0(E'))
    \end{align}
    の形の元の有限和で書けるから、
    このような形の元について示せば十分である。
    \begin{align}
        \nabla \xi = \alpha^i \otimes \xi_i,
            \quad
            (\alpha^i \in A^1(M), \; \xi_i \in A^0(E)) \\
        \nabla \eta = \beta^j \otimes \eta_j
            \quad
            (\beta^j \in A^1(M), \; \eta_j \in A^0(E'))
    \end{align}
    とおいておく (ただし、Einstein の記法を使うために共変・反変による
    添字の上下の慣例を一時的に無視している)。
    まず
    \begin{alignat}{1}
        &\quad d(g(\alpha \otimes \xi, \beta \otimes \eta)) \\
        &= d(g(\xi, \eta) \alpha \wedge \beta)
            \quad (\text{$\because$ ベクトル束値形式の内積の定義}) \\
        &= d(g(\xi, \eta)) \alpha \wedge \beta
            + g(\xi, \eta) d\alpha \wedge \beta
            + (-1)^p g(\xi, \eta) \alpha \wedge d\beta \\
        &= d(g(\xi, \eta)) \alpha \wedge \beta
            + g(\xi \otimes d\alpha, \eta \otimes \beta)
            + (-1)^p g(\xi \otimes \alpha, \eta \otimes d\beta)
            \label[equation]{eq:signed-leibniz-rule-1}
    \end{alignat}
    となる。ここで、第1項は
    \begin{alignat}{1}
        &\quad d(g(\xi, \eta)) \alpha \wedge \beta \\
        &= g(\nabla \xi, \eta) \alpha \wedge \beta
            + g(\xi, \nabla' \eta) \alpha \wedge \beta
            \quad (\text{$\because$ 命題の仮定}) \\
        &= g(\xi_i, \eta) \alpha^i \wedge \alpha \wedge \beta
            + g(\xi, \eta_j) \beta^j \wedge \alpha \wedge \beta \\
        &= g(\xi_i, \eta) \alpha^i \wedge \alpha \wedge \beta
            + (-1)^p g(\xi, \eta_j) \alpha \wedge \beta^j \wedge \beta \\
        &= g(\xi_i \otimes \alpha^i \wedge \alpha, \eta \otimes \beta)
            + (-1)^p g(\xi \otimes \alpha, \eta_j \beta^j \wedge \beta) \\
        &= g(\nabla \xi \wedge \alpha, \eta \otimes \beta)
            + (-1)^p g(\xi \otimes \alpha, \nabla' \eta \wedge \beta)
    \end{alignat}
    となる。
    したがって、\cref{eq:signed-leibniz-rule-1}より
    \begin{alignat}{1}
        d(g(\alpha \otimes \xi, \beta \otimes \eta))
            &= g(\nabla \xi \wedge \alpha, \eta \otimes \beta)
                + g(\xi \otimes d\alpha, \eta \otimes \beta) \\
            &\qquad \quad + (-1)^p g(\xi \otimes \alpha, \nabla' \eta \wedge \beta)
                + (-1)^p g(\xi \otimes \alpha, \eta \otimes d\beta) \\
            &= g(D(\xi \wedge \alpha), \eta \otimes \beta)
                + (-1)^p g(\xi \otimes \alpha, D'(\eta \wedge \beta))
    \end{alignat}
    が成り立つ。
\end{proof}

曲率を定義する。

\begin{definition}[曲率]
    $M$を多様体、
    $E \to M$をベクトル束、
    $\nabla$を$E$の接続、
    $D$を$\nabla$により定まる共変外微分とする。
    $R \coloneqq D^2$とおき、
    $R$を$\nabla$の
    \term{曲率}[curvature]{曲率}[きょくりつ]
    という。
\end{definition}

\begin{proposition}
    写像$R = D^2 \colon A^0(E) \to A^2(E)$は
    $A^2(\End E)$の元ともみなせる。
    \TODO{why?}
\end{proposition}

\begin{proof}
    \TODO{}
\end{proof}

曲率は、局所的には接続形式を用いて表せる。
ここで登場する構造方程式は、アファイン接続の場合の第2構造方程式
(\cref{prop:second-structure-equation})
に他ならない。
それでは第1構造方程式はどこにいったのかと気になるが、
一般の接続では捩率が定義できないから第1構造方程式にあたるものは登場しない。
Bianchi の恒等式に関しても同様である。

\begin{proposition}[構造方程式]
    \label[proposition]{prop:structure-equation}
    \begin{equation}
        \Omega^\mu_\lambda = d\omega^\mu_\lambda + \omega^\mu_\nu \wedge \omega^\nu_\lambda
    \end{equation}
    \TODO{}
\end{proposition}

\begin{proof}
    \TODO{}
\end{proof}

\begin{proposition}[Bianchi の恒等式]
    \begin{equation}
        DR = 0
    \end{equation}
    接続形式で書けば
    \begin{equation}
        d\Omega^\mu_\lambda
            - \Omega^\mu_\nu \wedge \omega^\nu_\lambda
            + \omega^\mu_\nu \wedge \Omega^\nu_\lambda
            = 0
    \end{equation}
    \TODO{}
\end{proposition}

\begin{proof}
    \TODO{}
\end{proof}

\begin{proposition}[Ricci の恒等式]
    共変外微分の公式\cref{remark:covariant-exterior-derivative-and-vector-fields}で
    $\varphi = D\xi, \; \xi \in A^0(E)$とおいて計算すると
    \begin{equation}
        R(X, Y) \xi
            = D(D\xi)(X, Y)
            = (\nabla_X \nabla_Y - \nabla_Y \nabla_X - \nabla_{[X, Y]}) \xi
    \end{equation}
    すなわち
    \begin{equation}
        R(X, Y) = \nabla_X \nabla_Y - \nabla_Y \nabla_X - \nabla_{[X, Y]}
    \end{equation}
    を得る。
    これを\term{Ricci の恒等式}[Ricci's identity]{Ricci の恒等式}[Ricci のこうとうしき]
    という。
\end{proposition}

\begin{proof}
    \TODO{}
\end{proof}

\begin{definition}[直和束の曲率]
    \TODO{}
\end{definition}

\begin{definition}[テンソル積束の曲率]
    \TODO{}
\end{definition}

\begin{definition}[双対束の曲率]
    \TODO{}
\end{definition}

\begin{definition}[引き戻し束の曲率]
    \TODO{}
\end{definition}



% ============================================================
%
% ============================================================
\chapter{主ファイバー束の接続}

前章ではベクトル束の接続について考えた。
この章ではまず主ファイバー束の接続について2通りの定義を述べた後、
主ファイバー束の接続とベクトル束の接続との対応について調べる。

% ------------------------------------------------------------
%
% ------------------------------------------------------------
\section{主ファイバー束の接続 (微分形式)}

主$G$束$P$の接続の定義の方法は2つあり、
\begin{enumerate}
    \item 1つ目は$P$上の$\frakg$値1形式としての定義である。
    \item 2つ目は$T_u P$から垂直部分空間への射影としての定義である。
        こちらは幾何学的な様子がわかりやすいという利点がある。
\end{enumerate}
この節ではまずは$\frakg$値1形式としての定義で接続を導入する。

\subsection{主ファイバー束の接続形式}

\cref{prop:connection-from-1-forms} で見たように、
ベクトル束の接続は
$\GL(r; \R)$値の変換関数と
$\mathfrak{gl}(r; \R)$値の1形式により定めることができた。
そこで、主$G$束でも同様の方法により
$\frakg$値1形式として接続を定義する。

\TODO{こちらはむしろ特徴付けにするべきでは?
    ゲージポテンシャルを主役とみる立場ならこちらが定義?}

\begin{definition}[主ファイバー束の接続形式]
    $M$を多様体、
    $G$を Lie 群、$\frakg$を$G$の Lie 代数、
    $p \colon P \to M$を主$G$束とする。
    $\{ (U_\alpha, \varphi_\alpha) \}_{\alpha \in A}$
    を$\bigcup U_\alpha = P$なる$P$の局所自明化の族とし、
    これにより定まる切断の族を$\{ \sigma_\alpha \}$とおく。
    さらに各$\alpha$に対し、
    $\omega_\alpha$を$U_\alpha$上の
    $\frakg$値$1$-形式であって関係式
    \begin{equation}
        \omega_\beta = \varphi_{\alpha\beta}^{-1} \omega_\alpha \varphi_{\alpha\beta}
            + \varphi_{\alpha\beta}^{-1} d\varphi_{\alpha\beta}
            \quad \text{on} \quad
            U_\alpha \cap U_\beta
    \end{equation}
    をみたすものとする。
    このとき、$P$上の$\frakg$値$1$-形式$\omega$を
    各$p^{-1}(U_\alpha) \subset P$上で
    \begin{equation}
        \omega \coloneqq
            s_\alpha^{-1} (\pi^* \omega_\alpha) s_\alpha
            + s_{\alpha}^{-1} ds_\alpha
    \end{equation}
    と定めることができる (このあとすぐ示す)。
    ただし右辺の積は$TG$における積であり、
    $s_\alpha$は
    \begin{equation}
        s_\alpha \coloneqq \mathrm{pr}_2 \circ \varphi_\alpha
        \colon \pi^{-1}(U_\alpha) \to G,
        \quad
        (x, \sigma_\alpha(x) . s) \mapsto s
    \end{equation}
    と定めた。
    $\omega$を$P$の
    \term{接続形式}[connection form]{接続形式!主ファイバー束の---}[せつぞくけいしき]
    という。
\end{definition}

\begin{proof}
    \uline{($\Rightarrow$)} \quad
    $\{ (U_\alpha, \varphi_\alpha) \}_{\alpha \in A}$
    を$\bigcup U_\alpha = P$なる$P$の局所自明化の族とし、
    これにより定まる切断の族を$\{ \sigma_\alpha \}$とおき、
    $\omega$はこれにより定まる$P$の接続形式であるとする。
    \TODO{}
\end{proof}

\begin{remark}
    $\sigma_\alpha^* \omega = \omega_\alpha$が成り立つ\footnote{
        $\sigma_\alpha^* \omega$は物理学では
        \term{ゲージポテンシャル}[gauge potential]{ゲージポテンシャル}
        と呼ばれる。
    }。
    \TODO{}
\end{remark}

\begin{theorem}[主ファイバー束の接続形式の特徴付け]
    $M$を多様体、
    $G$を Lie 群、$\frakg$を$G$の Lie 代数、
    $p \colon P \to M$を主$G$束とする。
    $\frakg$に値をもつ$P$上の$1$-形式$\omega$に関し、
    $\omega$が$P$の接続形式であることと
    $\omega$がつぎの条件をみたすこととは同値である:
    \begin{enumerate}
        \item ($G$-同変性) $R_a^* \omega = (\Ad g^{-1}) \omega \quad (a \in G)$
        \item $\omega(A^*) = A \quad (A \in \frakg)$
    \end{enumerate}
\end{theorem}

\begin{proof}
    \TODO{}
\end{proof}

\begin{definition}[Ehresmann 接続]
    $M$を多様体、$p \colon P \to M$を主$G$束とする。
    上の定理の条件 (1), (2) をみたす
    $P$上の$\frakg$値1形式$\omega$を
    $P$上の\term{Ehresmann 接続}[Ehresmann connection]{Ehresmann 接続}[Ehresmann せつぞく]
    あるいは単に
    \term{接続}[connection]{接続!主ファイバー束の---}[せつぞく]
    という。
\end{definition}

主ファイバー束の接続に対し、
Lie 群の構造方程式
(\cref{def:lie-group-structure-equation})
と類似の方程式が成り立つ。
したがって
\begin{itemize}
    \item $P$の接続は$G$の接続の一般化
    \item $P$の接続の構造方程式は$G$の構造方程式の一般化
\end{itemize}
とみることができる。\TODO{どういう意味?}

\begin{proposition}[接続の構造方程式]
    \begin{equation}
        d\omega = - [\omega, \omega]
    \end{equation}
    \TODO{}
\end{proposition}

\begin{proof}
    \TODO{}
\end{proof}

\subsection{主ファイバー束の曲率}

主ファイバー束の曲率を定義する。
ベクトル束の接続の曲率は構造方程式 (\cref{prop:structure-equation})
をみたすのであった。
そこで、主ファイバー束の接続の曲率は逆にこの方程式によって定義する。

\begin{definition}[曲率形式]
    \TODO{}
\end{definition}

% ------------------------------------------------------------
%
% ------------------------------------------------------------
\section{主ファイバー束の接続 (水平部分空間の方法)}

前節では主ファイバー束の接続を微分形式として定義した。
この節では水平部分空間の方法を用いて接続を定義する。

\subsection{接分布}

まず基本的な概念を導入しておく。

\begin{definition}[接分布]
    $M$を多様体とする。
    $D \subset TM$が$M$上の
    \term{接分布}[tangent distribution]{接分布}[せつぶんぷ]
    であるとは、
    $D$が$TM$の部分ベクトル束であることをいう。
\end{definition}

\begin{definition}[積分多様体]
    $M$を多様体、$D \subset TM$を$M$上の接分布とする。
    部分多様体$N \subset M$が$D$の
    \term{積分多様体}[integral manifold]{積分多様体}[せきぶんたようたい]
    であるとは、
    \begin{equation}
        T_xN = D_x
            \quad
            (\forall x \in N)
    \end{equation}
    が成り立つことをいう。

    各$x \in M$に対し
    $D$のある積分多様体$N \subset M$が存在して
    $x \in N$となるとき、
    $D$は\term{積分可能}[integrable]{積分可能}[せきぶんかのう]
    であるという。
\end{definition}

\begin{definition}[包合的]
    $M$を多様体、$D \subset TM$を$M$上の接分布とする。
    $D$が\term{包合的}[involutive]{包合的}[ほうごうてき]であるとは、
    $D$の任意の局所切断$X, Y$に対し
    $[X, Y]$も$D$の局所切断となることをいう。
\end{definition}

\begin{theorem}[Frobenius]
    \TODO{}
\end{theorem}

\subsection{垂直接分布と水平接分布}

垂直接分布を定義する。
垂直接分布は接続とは関係なく主ファイバー束の構造のみによって決まる。

\begin{definition}[垂直接分布]
    $M$を多様体、$p \colon P \to M$を主$G$束とする。
    各$u \in P$に対し、$\R$-部分ベクトル空間
    \begin{equation}
        V_u \coloneqq \Ker p_*
    \end{equation}
    を$T_uP$の\term{垂直部分空間}[vertical subspace]{垂直部分空間}[すいちょくぶぶんくうかん]
    という。
    さらに$\coprod_{u \in P} V_u$は$TP$の部分ベクトル束となり、
    これを\term{垂直接分布}[vertical distribution]{垂直接分布}[すいちょくせつぶんぷ]という。

    垂直接分布に属する元は
    \term{垂直}[vertical]{垂直}[すいちょく]であるという。
\end{definition}

垂直部分空間は次のように表せる。
これにより$V_u$と$\frakg$を同一視すれば、
主ファイバー束の接続形式の条件$\omega(A^*) = A$とは
$\omega$が$V_u$上恒等写像であるという条件に他ならない。

\begin{proposition}
    $V_u = \{ A^*_u \mid A \in \frakg \}$
    \TODO{}
\end{proposition}

\begin{proof}
    \TODO{}
\end{proof}

\begin{theorem}[垂直接分布は積分可能]
    \TODO{}
\end{theorem}

\begin{proof}
    \TODO{}
\end{proof}

次に水平部分空間を定義する。
水平部分空間とは$T_u P = V_u \oplus H_u$なる部分空間$H_u$のことであるが、
$H_u$は主ファイバー束の構造のみからは決定されない。
後で詳しく見るが、主ファイバー束に接続を与えることは、
本質的には右不変な水平部分空間を選ぶのと同じことである。

\TODO{あとで接続から水平部分空間が定まることをいうのだから、
    水平部分空間の定義には接続を含むべきでないのでは?}

\begin{definition}[水平部分空間]
    $M$を多様体、$p \colon P \to M$を主$G$束、
    $\omega$を$P$の接続形式とする。
    各$u \in P$に対し、$\R$-部分ベクトル空間
    \begin{equation}
        H_u \coloneqq \Ker \omega_u
    \end{equation}
    を$T_uP$の\term{水平部分空間}[horizontal subspace]{水平部分空間}[すいへいぶぶんくうかん]
    という。
\end{definition}

\subsection{水平接分布と接続}

水平接分布により主ファイバー束の接続を特徴付ける。

\begin{theorem}[水平接分布から接続へ]
    $p \colon P \to M$を主$G$束、
    $H$を右不変な水平接分布とする。
    このとき、$P$上の$\frakg$値1形式$\omega$を
    \begin{equation}
        \omega_u \colon T_u P \to V_u \to \frakg
    \end{equation}
    により定めると、$\omega$は$P$上の接続となる。
\end{theorem}

\begin{proof}
    \TODO{cf. [Tu] p.255}
\end{proof}

\begin{theorem}[接続から水平接分布へ]
    $p \colon P \to M$を主$G$束、
    $\omega$を$P$上の接続とする。
    このとき、$H_u \coloneqq \Ker \omega_u$は
    $P$の右不変な水平部分空間である。
\end{theorem}

\begin{proof}
    \TODO{cf. [Tu] p.257}
\end{proof}

主ファイバー束の曲率形式も
水平接分布を用いて表すことができる。

\begin{proposition}
    接続形式$\omega$の曲率形式$\Omega$は
    \begin{equation}
        \Omega(X, Y) = d\omega(X^H, Y^H)
            \quad
            (X, Y \in T_uP)
    \end{equation}
    をみたす。
\end{proposition}

\begin{proof}
    \TODO{}
\end{proof}

水平接分布の積分可能性と曲率には密接な関係がある。

\begin{theorem}[水平接分布の積分可能性]
    $P$の水平接分布$\coprod_{u \in P} H_u$に関し次は同値である:
    \begin{enumerate}
        \item $\coprod H_u$は積分可能である。
        \item $P$の曲率は$0$である。
    \end{enumerate}
\end{theorem}

\begin{proof}
    \TODO{}
\end{proof}


% ------------------------------------------------------------
%
% ------------------------------------------------------------
\section{同伴ベクトル束の接続}

同伴ベクトル束を思い出そう。
\cref{subsec:principal-fiber-bundle-to-vector-bundle}
で見たように、主ファイバー束$P$と表現$\rho$から
同伴ベクトル束$E = P \times_\rho \R^r$が構成できるのであった。
このとき、$P$の共変外微分から$E$に共変外微分が誘導される。
とくに$P$の接続形式から$E$の接続形式が定まる。

\begin{theorem}[微分形式の対応]
    $P$上の$\R^r$値$p$-形式$\widetilde{\xi}$で
    \begin{enumerate}
        \item $R_a^* \widetilde{\xi} = \rho(a)^{-1} \widetilde{\xi} \quad (a \in G)$
        \item ある$i$で$X_i$が垂直ならば
            $\widetilde{\xi}(X_1, \dots, X_p) = 0$
    \end{enumerate}
    をみたすもの全体の空間を$\widetilde{A}^p(P)$とおく。
    $A^p(E)$と$\widetilde{A}^p(P)$は
    次の対応により1:1に対応する。
    \begin{equation}
        \widetilde{\xi}(X_1, \dots, X_p)
            = u^{-1} (\xi (\pi_* X_1, \dots, \pi_* X_p))
            \quad
            (X_1, \dots, X_p \in T_u P)
    \end{equation}
    \TODO{}
\end{theorem}

\begin{proof}
    \TODO{}
\end{proof}

\begin{proposition}
    上の定理の対応は
    外積代数$A(E)$から$\widetilde{A}(P)$への
    $A(M)$-加群同型である。
    \TODO{}
\end{proposition}

\begin{proof}
    \TODO{}
\end{proof}

$\widetilde{A}(P)$に共変外微分を定義し、
上の同型により$A(E)$に共変外微分を誘導する。

\begin{definition}[$\widetilde{A}(P)$の共変外微分]
    \begin{equation}
        D\widetilde{\xi}(X_1, \dots, X_{p + 1})
            = d\widetilde{\xi}(X^H_1, \dots, X^H_{p + 1})
            \quad
            (X_1, \dots, X_{p + 1} \in T_u P)
    \end{equation}
    \TODO{}
\end{definition}

$D\widetilde{\xi}$は次のように書くこともできる。

\begin{proposition}
    \begin{equation}
        D\widetilde{\xi} = d\widetilde{\xi} + \rho(\omega) \wedge \widetilde{\xi}
    \end{equation}
    \TODO{}
\end{proposition}

\begin{proof}
    \TODO{}
\end{proof}

% ------------------------------------------------------------
%
% ------------------------------------------------------------
\section{平行移動とホロノミー}

この節では、平行移動とホロノミーについて述べる。

\subsection{ベクトル束の平行移動とホロノミー}

さて、ここで曲線$\gamma$に沿う$\xi$の共変微分
「$\nabla_{\dot{\gamma}(t)} \xi$」を定義したい。
ところが、ややこしいことに「曲線$\gamma$に沿う$E$の切断」は「$E$の切断」ではないため、
「$\nabla_{\dot{\gamma}(t)} \xi$」という文字列に
正確な意味を与えるにはさらなる定義が必要となる。

\TODO{引き戻しを用いて定義したほうがよさそう cf. [Tu] p. 262}

\begin{definition}[曲線に沿う切断の拡張可能性]
    $M$を多様体、
    $\pi \colon E \to M$をベクトル束、
    $J$を$\R$の区間、
    $\gamma \colon J \to M$を{\smooth}曲線とする。
    曲線$\gamma$に沿う$E$の切断$\xi \colon J \to E$が
    \term{拡張可能}[extendible]{拡張可能}[かくちょうかのう]
    であるとは、
    $\gamma$の像$\gamma(J)$を含む$U \opensubset M$と
    $U$上の$E$の切断$\widetilde{\xi}$が存在して
    \begin{equation}
        \widetilde{\xi}_{\gamma(t)} = \xi_t
            \quad
            (\forall t \in J)
    \end{equation}
    が成り立つことをいう。
    $\widetilde{\xi}$を
    \term{$\xi$の拡張}{拡張!曲線に沿う切断の---}[かくちょう]という。
\end{definition}

\begin{example}[拡張可能でない例]
    8の字曲線$\gamma \colon (-\pi, \pi) \to \R^2, \;
    t \mapsto (\sin t, \sin t \cos t)$の
    速度ベクトル$\dot{\gamma}$は拡張可能でない。
    なぜならば、8の字の中央部分で速度ベクトルが2方向に出ているからである。
\end{example}

\begin{definition}[曲線に沿う共変微分]
    上の定義の状況で、
    さらに$\nabla$を$E$の接続とし、
    $\xi$は拡張可能であるとする。
    このとき、$\xi$の拡張$\widetilde{\xi} \in \Gamma(E)$をひとつ選び
    \begin{equation}
        \nabla_{\dot{\gamma}(t)} \xi
            \coloneqq \nabla_{\dot{\gamma}(t)} \widetilde{\xi}
            \quad
            (t \in J)
    \end{equation}
    と定義し、これを
    \term{曲線$\gamma$に沿う共変微分}[covariant derivative along $\gamma$]
    {曲線に沿う共変微分}[きょくせんにそうきょうへんびぶん]という。
    これは$\widetilde{\xi}$の選び方によらず well-defined に定まる (このあと示す)。
\end{definition}

\begin{proposition}
    上の定義の状況で、
    さらに$\widetilde{\xi} \in \Gamma(E)$を$\xi$の拡張、
    $t \in J$、
    $U$を$M$における$\gamma(t)$の開近傍、
    $e_1, \dots, e_r$を$U$上の$E$の局所フレーム、
    $x^1, \dots, x^n$を$U$上の$M$の局所座標とする。
    $\widetilde{\xi}$を局所的に
    \begin{alignat}{1}
        \widetilde{\xi} &= \widetilde{\xi}^\lambda e_\lambda
            \quad
            (\widetilde{\xi}^\lambda \in \smooth(U))
    \end{alignat}
    と表し、$\xi^\lambda \coloneqq \widetilde{\xi}^\lambda \circ \gamma$とおく。
    また$\nabla e_\mu$を局所的に
    \begin{alignat}{1}
        \nabla e_\mu
            &= \omega_\mu^\lambda \otimes e_\lambda
                \quad
                (\omega_\mu^\lambda \in A^1(U)) \\
            &= \Gamma^\lambda_{\mu i} dx^i \otimes e_\lambda
                \quad
                (\Gamma^\lambda_{\mu i} \in \smooth(U))
    \end{alignat}
    と表す。
    このとき
    \begin{equation}
        \nabla_{\dot{\gamma}(t)} \xi
            = \left\{
                \frac{d\xi^\lambda}{dt}(t)
                +
                \xi^\mu (t)
                \Gamma^\lambda_{\mu i} (\gamma(t))
                \frac{d\gamma^i}{dt}(t)
            \right\}
            (e_\lambda)_{\gamma(t)}
    \end{equation}
    が成り立つ。
    したがってとくに$\nabla_{\dot{\gamma}(t)} \xi$の値は
    $\widetilde{\xi}$の選び方によらず well-defined に定まる。
\end{proposition}

\begin{proof}
    まず記法を整理すると
    \begin{equation}
        \nabla_{\dot{\gamma}(t)} \xi
            = \nabla_{\dot{\gamma}(t)} \widetilde{\xi}
            = (\nabla \widetilde{\xi}) (\dot{\gamma}(t))
            = \underbrace{
                (\nabla \widetilde{\xi})_{\gamma(t)}
            }_{\in \, T^*_{\gamma(t)} M \otimes E_{\gamma(t)}}
            (\underbrace{\dot{\gamma}(t)}_{\in \, T_{\gamma(t)} M})
    \end{equation}
    と書けることに注意する。
    そこで$\nabla \widetilde{\xi}$を変形すると
    \begin{alignat}{1}
        \nabla \widetilde{\xi}
            &= d\widetilde{\xi}^\lambda \otimes e_\lambda
                + \widetilde{\xi}^\mu \nabla e_\mu \\
            &= d\widetilde{\xi}^\lambda \otimes e_\lambda
                + \widetilde{\xi}^\mu \Gamma^\lambda_{\mu i} dx^i \otimes e_\lambda \\
            &= \left\{
                d\widetilde{\xi}^\lambda
                + \widetilde{\xi}^\mu \Gamma^\lambda_{\mu i} dx^i
            \right\} \otimes e_\lambda
    \end{alignat}
    となるから、点$\gamma(t)$での値は
    \begin{alignat}{1}
        (\nabla \widetilde{\xi})_{\gamma(t)}
            &= \left\{
                d\widetilde{\xi}^\lambda_{\gamma(t)}
                + \widetilde{\xi}^\mu (\gamma(t))
                \Gamma^\lambda_{\mu i} (\gamma(t))
                dx^i_{\gamma(t)}
            \right\} \otimes (e_\lambda)_{\gamma(t)} \\
            &= \left\{
                d\widetilde{\xi}^\lambda_{\gamma(t)}
                + \xi^\mu (\gamma(t))
                \Gamma^\lambda_{\mu i} (\gamma(t))
                dx^i_{\gamma(t)}
            \right\} \otimes (e_\lambda)_{\gamma(t)}
    \end{alignat}
    である。
    ここで
    \begin{alignat}{1}
        (d\widetilde{\xi}^\lambda)_{\gamma(t)} (\dot{\gamma}(t))
            &= \dd{t}\bigg|_{t = t} \widetilde{\xi}^\lambda \circ \gamma(t)
            = \frac{d\xi^\lambda}{dt}(t) \\
        (dx^i)_\gamma(t) (\dot{\gamma}(t))
            &= \dd{t}\bigg|_{t = t} x^i \circ \gamma(t)
            = \frac{d\gamma^i}{dt}(t)
    \end{alignat}
    だから
    \begin{alignat}{1}
        \nabla_{\dot{\gamma}(t)} \xi
            = (\nabla \widetilde{\xi})_{\gamma(t)}
            = \left\{
                \frac{d\xi^\lambda}{dt}(t)
                +
                \xi^\mu (t)
                \Gamma^\lambda_{\mu i} (\gamma(t))
                \frac{d\gamma^i}{dt}(t)
            \right\}
            (e_\lambda)_{\gamma(t)}
    \end{alignat}
    を得る。
    関数$\xi^\lambda$は
    $\widetilde{\xi}$の選び方によらないから
    well-defined 性もいえた。
\end{proof}

測地線の一般化として、
平行の概念を定義する。

\begin{definition}[平行]
    $M$を多様体、$E \to M$をベクトル束、
    $\nabla$を$E$の接続、
    $J$を$\R$の区間、
    $\gamma \colon J \to M$を{\smooth}曲線、
    $\xi$を曲線$\gamma$に沿う$E$の切断とする。
    $\xi$が
    \begin{equation}
        \nabla_{\dot{\gamma}(t)} \xi = 0
            \quad
            (\forall t \in J)
    \end{equation}
    をみたすとき、$\xi$は
    \term{曲線$\gamma$に沿って平行}[parallel along $\gamma$]{平行}[へいこう]
    であるという。
    上の命題より、これは次の斉次1階常微分方程式系が成り立つことと同値である:
    \begin{equation}
        \frac{d\xi^\lambda}{dt}(t)
            + \xi^\mu (t)
            \Gamma^\lambda_{\mu i} (\gamma(t))
            \frac{d\gamma^i}{dt}(t)
            = 0
            \quad
            (\lambda = 1, \ldots, r)
    \end{equation}
    $\bm{\xi} \coloneqq \up{t}(\xi^1, \dots, \xi^r), \;
    A \coloneqq \left(
        \Gamma^{\lambda}_{\mu i} \frac{d\gamma^i}{dt}
    \right)_{\lambda, \mu}$
    とおけば
    \begin{equation}
        \frac{d\bm{\xi}}{dt} = - A \bm{\xi}
    \end{equation}
    と書ける。
\end{definition}

\begin{remark}
    測地線とは、
    その速度ベクトルが自身に沿って平行な曲線のことである。
\end{remark}

\begin{definition}[平行移動]
    上の命題の状況で
    さらに$J = [a, b], \; a, b \in \R$とするとき、
    初期値問題の解の存在と一意性より
    任意の$\xi_a \in E_{\gamma(a)}$に対し
    $\xi(a) = \xi_a$なる
    解$\xi$が一意に定まる。
    このとき、$\xi$は
    $\xi_a$を曲線$\gamma$に沿って
    \term{平行移動}[parallel displacement]{平行移動}[へいこういどう]
    して得られたという\footnote{
        最適化の分野では、
        指数写像や平行移動の数値計算のために、
        これらの代替となる
        \term{レトラクション}[retraction]{レトラクション}[れとらくしょん]
        や
        \term{ベクトル輸送}[vector transport]{ベクトル輸送}[べくとるゆそう]
        が用いられる。
    }。
\end{definition}

\begin{proposition}
    上の定義の状況で、
    写像
    \begin{equation}
        E_{\gamma(a)} \to E_{\gamma(b)},
        \quad
        \xi_a \mapsto \xi_b \coloneqq \xi(b)
    \end{equation}
    は$\R$-線型同型である。
\end{proposition}

\begin{proof}
    初期値問題の解の存在と一意性より、写像であることはよい。
    全射性は$t = b$での$\xi$の値を指定した初期値問題を考えればよい。
    $\R$-スカラー倍を保つことは次のようにしてわかる:
    $\xi$が$\xi(a) = \xi_a$なる解であったとすると、
    各$c \in \R$に対し
    $\eta(t) \coloneqq c \xi(t)$は$\eta(a) = c \xi_a$をみたすただひとつの解であるから、
    $c \xi_a = \eta(a)$を曲線$\gamma$に沿って平行移動して得られる値は
    $\eta(b) = c \xi(b) = c \xi_b$に他ならない。
    和を保つことも同様にして示せる。
    よって命題の写像は全射$\R$-線型写像である。
    $\dim_\R E_{\gamma(a)} = \dim_\R E_{\gamma(b)}$より
    $\R$-線型同型であることが従う。
\end{proof}

\begin{definition}[ベクトル束の接続のホロノミー群]
    $x_0 \in M$とする。
    $x_0$を基点とする区分的に{\smooth}な任意の閉曲線$c$に対し、
    平行移動により$\R$-ベクトル空間$E_{x_0}$の自己同型写像
    ($\tau_c$とおく) が得られる。
    そこで
    \begin{equation}
        \Psi_{x_0} \coloneqq \{
            \tau_c \in GL(E_{x_0})
            \mid
            \text{$c$は$x_0$を基点とする区分的に{\smooth}な閉曲線}
        \}
    \end{equation}
    とおくと、
    $\Psi_{x_0}$は$GL(E_{x_0})$の部分群となる (このあと示す)。
    $\Psi_{x_0}$を$x_0$を基点とする
    \term{ホロノミー群}[holonomy group]{ホロノミー群}[ほろのみーぐん]という。
\end{definition}

\begin{proof}
    \uline{写像の合成について閉じていること} \quad
    $\tau_c, \tau_{c'} \in \Psi_{x_0}$とすると
    $c \circ c'$は$x_0$を基点とする区分的に{\smooth}な閉曲線であり、
    $\tau_c \circ \tau_{c'} = \tau_{c \circ c'}$が成り立つ。

    \uline{単位元を含むこと} \quad
    定値曲線$x_0$に対し$\tau_{x_0} \in \Psi_{x_0}$が恒等写像となる。

    \uline{逆元を含むこと} \quad
    $\tau_c \in \Psi_{x_0}$とする。
    $c$を逆向きに動く曲線$d$を
    $d(t) \coloneqq c(a + b - t) \; t \in [a, b]$で定め、
    $\xi$を逆向きに動く曲線$\eta$を
    $\eta(t) \coloneqq \xi(a + b - t) \; t \in [a, b]$で定める。
    このとき$d$は$x_0$を基点とする区分的に{\smooth}な閉曲線だから
    $\tau_d \in \Psi_{x_0}$である。
    また、$\eta$は$d$に沿う$E$の切断である。
    さらに$\eta$が$d$に沿って平行であることは、
    $\xi$の拡張を$\widetilde{\xi}$として
    (これは$\eta$の拡張でもある)
    \begin{alignat}{1}
        \nabla_{\dot{d}(t)} \eta
            &= \nabla_{\dot{d}(t)} \widetilde{\xi} \\
            &= \nabla_{- \dot{c}(a + b - t)} \widetilde{\xi} \\
            &= - \nabla_{\dot{c}(a + b - t)} \widetilde{\xi} \\
            &= - \nabla_{\dot{c}(a + b - t)} \xi \\
            &= 0
    \end{alignat}
    よりわかる。
    よって$\xi_b = \eta(a)$を$d$に沿って平行移動すると
    $\eta(b) = \xi(a) = \xi_a$が得られる。
    したがって$\eta_d = \eta_c^{-1}$である。
\end{proof}

%\begin{proposition}
%    $M$を多様体、$E \to M$をベクトル束、
%    $g$を$E$の内積、
%    $\nabla$を$g$を保つ$E$の接続とする。
%    このとき、内積は平行移動で不変である\TODO{どういう意味?}。
%\end{proposition}
%
%\begin{proof}
%    \TODO{}
%\end{proof}

\subsection{主ファイバー束の平行移動とホロノミー}

\begin{definition}[水平な曲線]
    $M$を多様体、
    $G$を Lie 群、
    $p \colon P \to M$を主$G$束、
    $\omega$を$P$の接続形式、
    $J \subset \R$を区間とする。
    {\smooth}曲線$u \colon J \to P$が
    \term{水平}[horizontal]{水平}[すいへい]であるとは、
    $u$の速度ベクトル$\dot{u}$がつねに水平部分空間に含まれること、すなわち
    \begin{equation}
        \omega(\dot{u}(t)) = 0
            \quad (\forall t \in J)
    \end{equation}
    が成り立つことをいう。
\end{definition}

\begin{definition}[平行移動]
    $M$を多様体、
    $G$を Lie 群、
    $p \colon P \to M$を主$G$束、
    $\omega$を$P$の接続形式、
    $J \subset \R$を区間、
    $x \colon J \to M$を$x_0 \in M$を始点とする{\smooth}曲線
    とする。
    このとき各$u_0 \in P_{x_0}$に対し、
    $u_0$を始点とする水平な{\smooth}曲線$u \colon J \to P$であって
    \begin{equation}
        \pi(u(t)) = x(t) \quad (t \in J)
    \end{equation}
    をみたすものが一意に存在する (証明略)。
    このとき、
    $u$は曲線$x$に沿った$u_0$の
    \term{平行移動}[parallel displacement]{平行移動}[へいこういどう]
    であるという。
    \begin{equation}
        \begin{tikzcd}
            & P \ar{d}{p} \\
            J \ar{ru}{u} \ar{r}[swap]{x}
                & M
        \end{tikzcd}
    \end{equation}
\end{definition}

\begin{proposition}
    $u$が水平ならば、任意の$s \in G$に対し
    $u(t) . s$も水平である。
\end{proposition}

\begin{proof}
    水平接分布が$G$の作用で保たれることより明らか。
\end{proof}

\begin{definition}[主ファイバー束の接続のホロノミー群]
    $u_0 \in P$とし、$x_0 = p(u_0)$とおく。
    $x_0$を始点とする$M$内の任意の閉曲線$c$に対し、
    $x$に沿った$u_0$の平行移動を$u$とおくと
    \begin{equation}
        u(b) = u_0 . \tau_c
    \end{equation}
    なる$\tau_c \in G$が一意に定まる。
    このような$\tau_c$全体の集合を$\Psi_{u_0}$とおくと、
    $\Psi_{u_0}$は$G$の部分群となる。
    $\Psi_{u_0}$を$u_0$を始点とする$\omega$の
    \term{ホロノミー群}[holonomy group]{ホロノミー群}[ほろのみーぐん]という。
\end{definition}

\begin{proposition}[ホロノミー群の共役]
    $u_0, u_1 \in P$とし、
    $x_0 = p(u_0), \; x_1 = p(u_1)$とおく。
    $c_0$を$x_0$から$x_1$への区分的に{\smooth}な曲線とし、
    曲線$c_0$に沿った$u_0$の平行移動を$\widetilde{c}_0$とおく。
    すると$\widetilde{c}_0(b) = u_1 . a$なる$a \in G$がただひとつ存在するが、
    このとき$\Psi_{u_1} = a \Psi_{u_0} a^{-1}$が成り立つ。
\end{proposition}

\begin{proof}
    $a \Psi_{u_0} a^{-1} \subset \Psi_{u_1}$および
    $a^{-1} \Psi_{u_1} a \subset \Psi_{u_0}$を示せばよい。
    実際、これらが示されたならば
    $a \Psi_{u_0} a^{-1} \subset \Psi_{u_1}
    = aa^{-1} \Psi_{u_1} aa^{-1} \subset a \Psi_{u_0} a^{-1}$
    より$a \Psi_{u_0} a^{-1} = \Psi_{u_1}$が従う。
    さらに$u_0, u_1$に関する対称性より
    $a \Psi_{u_0} a^{-1} \subset \Psi_{u_1}$を示せば十分。
    そこで$\tau_c \in \Psi_{u_1}$とし、
    $c$に沿う$u_0$の平行移動を$\widetilde{c}$とおき、
    $a \tau_c a^{-1} \in \Psi_{u_0}$を示す。
    そのためには$a \tau_c a^{-1} = \tau_{c_0 \circ c \circ c_0^{-1}}$であること、
    すなわち$c_0 \circ c \circ c_0^{-1}$に沿う
    $u_1$の平行移動の終点が$u_1 . a \tau_c a^{-1}$であることをいえばよい。

    まず$c_0^{-1}$に沿う$u_1$の平行移動は
    $R_{a^{-1}} \circ \widetilde{c}_0^{-1}$であり、
    その終点は$u_0 . a^{-1}$である。

    つぎに$c$に沿う$u_0 . a^{-1}$の平行移動は
    $R_{a^{-1}} \circ \widetilde{c}$であり、
    その終点は$u_0 . \tau_c a^{-1}$である。

    最後に$c_0$に沿う$u_0 . \tau_c a^{-1}$の平行移動は
    $R_{\tau_c a^{-1}} \circ \widetilde{c}_0$であり、
    その終点は$u_1 . a \tau_c a^{-1}$である。
    これが示したいことであった。
\end{proof}






% ============================================================
%
% ============================================================
\chapter{特性類}

特性類について述べる。
特性類はベクトル束の位相不変量である。

% ------------------------------------------------------------
%
% ------------------------------------------------------------
\section{複素ベクトル束}

\begin{definition}[Complex Vector Bundles]
    \TODO{}
\end{definition}

% ------------------------------------------------------------
%
% ------------------------------------------------------------
\section{Euler 類}

\TODO{}

% ------------------------------------------------------------
%
% ------------------------------------------------------------
\section{Chern 類}

\TODO{}




\end{document}

\part{体}
\documentclass[report]{jlreq}
\usepackage{global}
\usepackage{./local}
\subfiletrue
\begin{document}

% ============================================================
%
% ============================================================
\chapter{関数列と関数項級数}

% ------------------------------------------------------------
%
% ------------------------------------------------------------
\section{関数列}

関数列の収束について基礎的な事項を整理する。

\TODO{関数族の場合も含めてネットで定義する}

\begin{definition}[各点収束と一様収束、広義一様収束]
    $I \subset \R$、
    $f, f_n \; (n \in \N)$を$I$上の実数値関数とする。
    \begin{enumerate}
        \item 関数列$\{ f_n \}_{n \in \N}$が
            $f$に
            \term{$I$上各点収束}[converge pointwise on $I$]
                {各点収束}[かくてんしゅうそく]
            するとは、
            $\forall \eps > 0$と
            $\forall x \in I$に対し、
            $\exists N \in \R$が存在し、
            $\forall n \ge N$に対し
            \begin{equation}
                |f_n(x) - f(x)| < \eps
            \end{equation}
            が成り立つことをいう。
        \item 関数列$\{ f_n \}_{n \in \N}$が
            $f$に
            \term{$I$上一様収束}[converge uniformly on $I$]
                {一様収束}[いちようしゅうそく]
            するとは、
            $\forall \eps > 0$に対し、
            $\exists N \in \R$が存在し、
            $\forall x \in I$と
            $\forall n \ge N$に対し
            \begin{equation}
                |f_n(x) - f(x)| < \eps
            \end{equation}
            が成り立つことをいう。
        \item 関数列$\{ f_n \}_{n \in \N}$が
            $I$の任意のコンパクト部分集合上で一様収束するとき、
            関数列$\{ f_n \}_{n \in \N}$は
            $f$に
            \term{$I$上広義一様収束する}[converge compactly on $I$]
                {広義一様収束}[こうぎいちようしゅうそく]
            という。
    \end{enumerate}
\end{definition}

\begin{proposition}[一様 Cauchy 条件]
    $I \subset \R$、
    $f_n,\, f \colon I \to \R$とする。
    このとき、次の条件は同値である。
    \begin{enumerate}
        \item $f_n \to f \;\text{in}\; C(I) \;\text{as}\; n \to \infty$
        \item $\| f_n - f \| \to 0 \;\text{as}\; n \to \infty$
        \item $\forall \eps > 0$に対し、
            $\exists N \in \N$が存在し、
            $\forall n, m \ge N$に対し
            \begin{equation}
                \forall x \in I,\, |f_n(x) - f_m(x)| < \eps
            \end{equation}
            が成り立ち、
            $f_n(x)$の$n \to \infty$での各点収束の極限は$f(x)$である
    \end{enumerate}
\end{proposition}

\begin{proof}
    \underline{(1) \Leftrightarrow (2)}\, 一様収束の定義から明らか。

    \underline{(1) \Rightarrow (3)}\,
    $\forall \eps > 0$をとる。ある$N \in \N$が存在して、
    $\forall n \ge N$に対し
    \begin{equation}
        \forall x \in I,\, |f_n(x) - f(x)| \le \eps / 2
    \end{equation}
    なので、$\forall n, m \ge N$に対し
    \begin{equation}
        \forall x \in I,\, |f_n(x) - f_m(x)| \le |f_n(x) - f(x)| + |f_m(x) - f(x)| \le \eps
    \end{equation}
    である。

    \underline{(3) \Rightarrow (1)}\,
    $x$ごとに$\{ f_n(x) \}_{n \in \N}$は Cauchy 列なので、
    実数の完備性より確かに
    $\lim_{n \to \infty} f_n(x) \eqqcolon f(x) \cdots$ (1) が$\forall x \in I$に対し存在する。
    (3)より、$\forall \eps > 0$に対しある$N \in \N$が存在して、
    \begin{equation}
        \forall n, m \ge N,\, \forall x \in I,\, |f_n(x) - f_m(x)| < \eps / 2
    \end{equation}
    すなわち\footnote{
        混乱の恐れがなければ、以降の議論をすっ飛ばして単に
        「$m \to \infty$とすれば$\forall n \ge N, \forall x \in I, |f_n(x) - f(x)| < \eps$」
        と言ってしまう手もあります。
        次の\cref{1:prop:2}や第4回の\cref{4:lemma:1}でも
        これと似たような内容の論証を少しずつ異なる言い回しで試みているので、ぜひ見比べてみてください。
    }
    \begin{equation}
        \forall n \ge N,\, \forall x \in I,\, \forall m \ge N,\, |f_n(x) - f_m(x)| < \eps / 2
    \end{equation}
    である。よって、$\forall n \ge N,\, \forall x \in I$をとって、(1)から定まる$N'$ s.t.
    \begin{equation}
        \forall m \ge N',\, |f_m(x) - f(x)| < \eps / 2
    \end{equation}
    に対し$m \ge \max\{N, N'\}$をひとつ選べば
    \begin{equation}
        |f_n(x) - f(x)| \le |f_n(x) - f_m(x)| + |f_m(x) - f(x)| \le \eps
    \end{equation}
    である。
\end{proof}

\begin{proposition}
    $I$を任意の区間とする。
    $I$上の連続関数列$\{ f_n \}_{n \in \N}$が$f$に$I$上広義一様収束するならば、
    $f$も$I$上の連続関数である。
    \label{1:prop:2}
\end{proposition}

もちろん、$I$は有界や閉区間でなくてもかまいません。

\begin{proof}
    $I$がコンパクトでない場合は
    点$x \in I$を含む$I$のコンパクト部分集合をとりなおせばよいから、
    以下では$I$がコンパクトの場合のみ示す。

    $x' \in I$を固定し、$f$が$x'$で連続であることを示そう。
    広義一様収束の仮定から、$\forall \eps > 0$に対し$\exists N \in \N$\, s.t.
    \begin{equation}
        n \ge N \Rightarrow \| f_n - f \| < \eps / 3
    \end{equation}
    である。$f_N$は$x'$で連続だから、$x'$のある近傍$U$が存在して
    \begin{equation}
        x \in U \cap I \Rightarrow |f_N(x') - f_N(x)| < \eps / 3
    \end{equation}
    が成り立つ。よって$\forall x \in U$に対し
    \begin{equation}
        \begin{split}
            |f(x) - f(x')|
                &\le |f(x) - f_N(x)| + |f_N(x) - f_N(x')| + |f_N(x') - f(x')| \\
                &\le \| f - f_N \| + \eps / 3 + \| f_N - f \| \\
                &< \eps
        \end{split}
    \end{equation}
    である。したがって$f$は$x'$で連続である。
\end{proof}

\begin{theorem}[項別積分]
    $I \coloneqq [a, b]$上の連続関数列$\{ f_n \}_{n \in \N}$が
    $n \to \infty$のとき$f$に$I$上一様収束するならば
    \begin{equation}
        \lim_{n \to \infty} \int_a^b f_n(x)\, dx = \int_a^b f(x)\, dx
    \end{equation}
\end{theorem}

\begin{proof}
    一様性があるので積分を外から抑えられます。
\end{proof}

\begin{theorem}[項別微分]
    $I$を任意の区間とする。このとき
    \begin{enumerate}
        \item $\{ f_n \}_{n \in \N} \subset C^1(I)$が
            $n \to \infty$で$f$に各点収束
        \item $\{ f_n' \}_{n \in \N} \subset C(I)$が
            $n \to \infty$で$g$に$I$上広義一様収束
    \end{enumerate}
    ならば
    \begin{enumerate}
        \item $\{ f_n \}_{n \in \N} \subset C^1(I)$が
            $n \to \infty$で$f$に$I$上一様収束し$C^1$級
        \item $I$の各点で$g(x) = f'(x)$
    \end{enumerate}
\end{theorem}

\begin{proof}
    $I$が有界閉区間の場合を以下の流れに沿って示した後、
    一般の区間$I$に対しては各点$x$を含む有界閉区間がとれることを用いて示します。
    $x$ごとに微積分学の基本定理を使って一様収束を示します。
    $f$の微分可能性は積分の平均値定理を使って示します。
\end{proof}




% ------------------------------------------------------------
%
% ------------------------------------------------------------
\section{関数項級数}

関数項級数の収束について基礎的な事項を整理します。

\begin{definition}[関数項級数の収束]
    簡単なので省略
\end{definition}

\begin{proposition}[関数項級数の項別積分]
    $I \coloneqq [a, b]$上の連続関数列$\{ f_n \}_{n \in \N}$による関数項級数
    $\sum_{k \in \N} f_k(x)$が$I$上一様収束であるとき
    \begin{enumerate}
        \item $\sum_{k \in \N} f_k(x)$も連続関数
        \item $\sum_{k \in \N} \int_a^b f_k(x) dx = \int_a^b \sum_{k \in N} f_k(x) dx$
    \end{enumerate}
\end{proposition}

\begin{proof}
    関数列の場合と同様なので省略
\end{proof}

\begin{proposition}[関数項級数の項別微分]
    $I$を\textcolor{red}{任意の区間}とし、$\{ f_n \}_{n \in \N} \subset C^1(I)$とする。このとき
    \begin{enumerate}
        \item $\sum_{k \in \N} f_k(x)$が
            $n \to \infty$で各点収束
        \item $\sum_{k \in \N} f_k'(x)$が
            $n \to \infty$で$I$上広義一様収束
    \end{enumerate}
    ならば
    \begin{enumerate}
        \item $\sum_{k \in \N} f_k(x)$が
            $n \to \infty$で$I$上一様収束し$C^1$級
        \item $I$の各点で$\dd{x} \sum_{k \in \N} f_k(x) = \sum_{k \in \N} f_k'(x)$
    \end{enumerate}
\end{proposition}

\begin{proof}
    関数列の場合と同様なので省略
\end{proof}

\begin{theorem}[Weierstrass のMテスト]
    $a<b,\, I=[a,b]$とし、$f_n: I\to\R\,(\forall n \in \N)$とする。
    ある実数列$\{M_n\}_{n\in\N}$が存在して次を満たすと仮定する:
    \begin{enumerate}
        \item 十分大きな$\forall n\in\N$に対し$\| f_n \| \le M_n$
        \item 級数$\sum_{k\in\N} M_k$は収束する
    \end{enumerate}
    このとき、級数$\sum_{k\in\N} f_k$は$I$上一様収束する。
\end{theorem}

この定理は$f_n,\, f$が多変数関数の場合にも拡張できます。

\begin{proof}
    一様 Cauchy 条件に帰着させて示します。
\end{proof}

% ------------------------------------------------------------
%
% ------------------------------------------------------------
\newpage
\section{演習問題}

\begin{problem}[東大数理 2006A]
    ~
    \begin{enumerate}
        \item 正の整数$n$に対し、
            定積分$I_n = \int_0^{\pi / 2} \cos^n \theta \, d\theta$
            の値を求めよ。
        \item $\R$上の関数
            \begin{equation}
                f(x) = \int_0^{\pi / 2} \cos(x \cos \theta) \, d\theta
            \end{equation}
            を$x = 0$を中心として Taylor 展開し、
            その$n + 1$次以上の項を無視して得られる多項式を
            $p_n(x)$とする。$p_n(x)$を求めよ。
        \item $K$を$\R$の有界な部分集合とする。
            $n \to \infty$のとき
            $p_n(x)$は$f(x)$に
            $K$上一様収束することを示せ。
    \end{enumerate}
\end{problem}

\begin{proof}
    \uline{(1)} \quad
    $n \ge 2$のとき、
    $\cos^n \theta = \cos^{n - 2} \theta (1 - \sin^2 \theta)$
    に注意すると
    \begin{alignat}{1}
        I_n
            &= \underbrace{
                \int_0^{\pi / 2} \cos^{n - 2} \theta \, d\theta
            }_{= I_{n - 2}}
                + \int_0^{\pi / 2} \cos^{n - 2} \theta \sin^2 \theta \, d\theta
    \end{alignat}
    である。右辺第2項は部分積分により
    \begin{alignat}{1}
        \mybracket{
            - \frac{1}{n - 1} \cos^{n - 1} \theta \sin \theta
        }_0^{\pi / 2}
            + \frac{1}{n - 1}
            \int_0^{\pi / 2} \cos^{n - 1} \theta \cos \theta \, d\theta
            &= \frac{1}{n - 1} \int_0^{\pi / 2} \cos^n \theta \, d\theta \\
            &= \frac{1}{n - 1} I_n
    \end{alignat}
    と変形される。したがって
    $I_n = I_{n - 2} + \frac{1}{n - 1} I_n$となり、
    整理して
    $I_n = \frac{n - 1}{n} I_{n - 2}$を得る。
    具体的計算により$I_0 = \frac{\pi}{2}, \; I_1 = 1$だから、
    求める答えは
    \begin{equation}
        I_n = \begin{cases}
            \frac{n - 1}{n} \frac{n - 3}{n - 2} \cdots \frac{1}{2} \frac{\pi}{2}
                & (\text{$n$が偶数}) \\[1ex]
            \frac{n - 1}{n} \frac{n - 3}{n - 2} \cdots \frac{2}{3}
                & (\text{$n$が奇数})
        \end{cases}
    \end{equation}
    である。

    \uline{(2)} \quad
    \begin{equation}
        p_n(x) = \sum_{0 \le k \le n/2}
            \frac{(-1)^k}{k!} I_{2k} x^k
    \end{equation}
    \TODO{}

    \uline{(3)} \quad
    Taylor 展開の剰余項を
    $R_n(x) \coloneqq f(x) - p_n(x)$とおく。
    $R_n(x)$が$n \to \infty$で$K$上$0$に一様収束することをいえばよい。
    いま$K$は有界だから、
    ある$R > 0$が存在して、
    すべての$x \in K$に対して$|x| < R$が成り立つ。
    また(1)の結果より、
    すべての$k \in \Z_{\ge 0}$に対し
    $I_{2k} < \frac{\pi}{2}$が成り立つ。
    したがって
    \begin{alignat}{1}
        |R_n(x)|
            &\le \sum_{k > n/2} \frac{1}{k!} I_{2k} |x|^k \\
            &\le \frac{\pi}{2} \sum_{k > n/2} \frac{1}{k!} R^k \\
            &= \frac{\pi}{2} \myparen{
                e^R - \sum_{0 \le k \le n/2} \frac{1}{k!} R^k
            } \\
            &\to 0 \quad (\text{$n \to \infty$})
    \end{alignat}
    を得る。
    すなわち$R_n(x)$は$K$上$0$に一様収束する。
    これが示したいことであった。
\end{proof}



% ============================================================
%
% ============================================================
\chapter{Fourier 級数と Fourier 変換}

% ------------------------------------------------------------
%
% ------------------------------------------------------------
\section{熱方程式に対するフーリエの方法}

偏微分方程式
\begin{equation}
    \deldel[u]{t}(x, t) = \frac{k}{c} \frac{\del^2 u}{\del x^2}(x, t)
\end{equation}
を\textbf{熱方程式}と呼びます。以下、簡単のため$c = k = 1$とします。
これに
\begin{equation}
    \begin{split}
        &\text{境界条件}\quad u(0, t) = 0,\, u(1, t) = 0 \\
        &\text{初期条件}\quad u(x, 0) = a(x) \not\equiv 0 \quad \text{for}\, x \in [0, 1]
            \quad \left(\text{ただし } \int_0^1 a(x)^2 dx < \infty \right)
    \end{split}
\end{equation}
を付け加えた問題を考えてみます。
求解の方針は次の3ステップです。
\begin{enumerate}
    \item 解の形を変数分離形$u(x, t) = \phai(x) \eta(t)$に仮定し、
    \item 境界条件から$\phai_n(x)$と$\eta_n(t)$を順に求め、
    \item $\phai_n(x) \eta_n(t)$の無限個の重ね合わせをとり、初期条件をみたす係数を求める。
\end{enumerate}
(3)の無限和の収束性に一旦目をつぶれば、"解"は
\begin{equation}
    u(x, t) = \sum_{n = 1}^\infty c_n e^{-(n\pi)^2 t} \sin (n\pi x),\quad
    c_n = 2 \int_0^1 a(x) \sin(n\pi x) dx
    \label{eq:1:1}
\end{equation}
と求まります\footnote{
    $c_n$の式の先頭に現れる$2$は$\int_0^1 \sin^2(n\pi x) dx = \frac{1}{2}$に由来します。
}。
そして、実はこの級数はきちんと収束します。


\begin{proposition}
    フーリエの方法で求めた解(\ref{eq:1:1})は$[0, 1] \times (0, \infty)$上広義一様収束する。
\end{proposition}

\begin{proof}
    $[0, 1] \times [\tau, T],\, 0 < \tau < T$を任意にとる。
    Weierstrass の定理を用いて示す。
    充分大きな任意の$n$に対し
    \begin{equation}
        \begin{split}
            |c_n u_n(x, t)|
                &\le |c_n|\, e^{-(n\pi)^2 \tau} \sin (n\pi x) \\
                &\le |c_n|\, e^{-(n\pi)^2 \tau} \\
                &\le \frac{|c_n|}{(n\pi)^2 \tau} (n\pi)^2 \tau\, e^{-(n\pi)^2 \tau} \\[0.5em]
            \therefore\quad |c_n u_n(x, t)|
                &= O\left(\frac{|c_n|}{n^2}\right) \quad (n \to \infty)
        \end{split}
    \end{equation}
    であり、
    \begin{equation}
        \begin{split}
            \sum_{n = 1}^\infty \frac{|c_n|}{n^2}
                &\le \left(\sum_{n = 1}^\infty |c_n|^2\right)^{1/2}
                    \left(\sum_{n = 1}^\infty \frac{1}{n^4} \right)^{1/2} \\
                &= \left(2 \int_0^1 a(x)^2 dx\right)^{1/2}
                    \left(\sum_{n = 1}^\infty \frac{1}{n^4} \right)^{1/2} \\
                &< \infty
        \end{split}
    \end{equation}
    なので、(\ref{eq:1:1})の級数は$[0, 1] \times [\tau, T]$上一様収束、
    したがって$[0, 1] \times (0, \infty)$上広義一様収束する。
\end{proof}




% ------------------------------------------------------------
%
% ------------------------------------------------------------
\section{フーリエ級数展開}

周期$2\pi$の関数$f: \R \to \R$に対し、
同じく周期$2\pi$の関数からなる関数系
$\{ \textcolor{red}{1 \big/\! \sqrt{\mathstrut 2}},\, \cos nx,\, \sin nx \}\, (n = 1, 2, \dots)$
による展開\footnote{
    関数系$\{ 1 \big/\! \sqrt{\mathstrut 2},\, \cos nx,\, \sin nx \}$は
    内積$\langle f, g \rangle = \textcolor{red}{\frac{1}{\pi}} \int_{-\pi}^{\pi} f\, g\, dx$のもとで
    正規直交関数系となっています。
}
\begin{equation}
    \frac{1}{2} a_0 + \sum_{n=1}^\infty (a_n \cos(nx) + b_n \sin(nx))
\end{equation}
を$f$の\textbf{フーリエ級数展開}と呼び、
\begin{equation}
    S_N[f](x) := \frac{1}{2} a_0 + \sum_{n=1}^N (a_n \cos(nx) + b_n \sin(nx))
\end{equation}
を$f$の\textbf{第$N$フーリエ部分和}と呼びます。
フーリエ級数展開が$f$に一致するかどうかはまだわかりませんが、
一致すると仮定すれば、三角関数の直交性から\textbf{フーリエ係数}$a_n, b_n$は
\begin{equation}
    \begin{split}
        a_n &= \frac{1}{\pi} \int_{-\pi}^{\pi} f(x) \cos(nx) dx \\
        b_n &= \frac{1}{\pi} \int_{-\pi}^{\pi} f(x) \sin(nx) dx
    \end{split}
\end{equation}
と表せることがわかります\footnote{
    $f$が偶関数ならば$a_n = \frac{2}{\pi} \int_0^\pi,\, b_n = 0$に、
    奇関数ならば$a_n = 0,\, b_n = \frac{2}{\pi} \int_0^\pi$になります。
}。
そこで、フーリエ級数展開が$f$に一致するという仮定は一旦忘れて、
$a_n, b_n$を上のように定義して議論をスタートします。

ここからは、フーリエ級数展開が収束するかどうか、するとしたらどこに収束するか、ということを考えていきます。
まずはフーリエ級数展開が一様収束するための条件をいくつか確認しておきます。

\begin{theorem}
    $f$が$C^2$級の$2\pi$周期関数であるとき、$S_N[f]$は$\R$上一様収束する。
\end{theorem}

\begin{proof}
    フーリエ係数$a_n$に対して
    \begin{alignat}{1}
        a_n &= \frac{1}{\pi} \int_{-\pi}^{\pi} f(x) \cos(nx) dx \\
            &= - \frac{1}{\pi} \int_{-\pi}^{\pi} f'(x) \sin(nx) dx \quad (\text{\because\, 部分積分}) \\
            &= - \frac{1}{n^2\pi} \int_{-\pi}^{\pi} f''(x) \cos(nx) dx \quad (\text{\because\, 部分積分}) \\
            &= O\left(\frac{1}{n^2}\right) \quad (n \to \infty)
    \end{alignat}
    である。
    よって
    \begin{equation}
        |a_n \cos(nx) + b_n \sin(nx)| \le |a_n| + |b_n| = O\left(\frac{1}{n^2}\right) \quad (n \to \infty)
    \end{equation}
    なので、Weierstrassの定理により、$S_N[f]$は$\R$上一様収束する。
\end{proof}



\begin{theorem}
    $f$が$C^{\textcolor{red}{1}}$級の$2\pi$周期関数であるとき、$S_N[f]$は$\R$上一様収束する。
\end{theorem}

\begin{proof}
    $m > n > 0$なる自然数$m, n$を任意にとると
    \begin{alignat}{1}
        \sum_{k=n}^m |a_k|
            &= \sum_{k=n}^m \left| \frac{1}{\pi} \int_{-\pi}^{\pi} f(x) \cos(kx) dx \right| \quad (\text{\because\, 定義}) \\
            &= \frac{1}{\pi} \sum_{k=n}^m \left| \frac{1}{k} \int_{-\pi}^{\pi} f'(x) \cos(kx) dx \right| \quad (\text{\because\, 部分積分}) \\
            &\le \frac{1}{\pi}
                \left\{ \sum_{k=n}^m \frac{1}{k^2} \right\}^{1/2}
                \left\{ \sum_{k=n}^m \left( \int_{-\pi}^{\pi} f'(x) \cos(kx) dx \right)^2 \right\}^{1/2}
                \quad (\text{\because\, Schwartzの不等式}) \\
            &\le \frac{1}{\pi}
                \left\{ \sum_{k=n}^m \frac{1}{k^2} \right\}^{1/2}
                \cdot \left\{ \frac{1}{\pi} \int_{-\pi}^{\pi} f'(x)^2 dx \right\}^{1/2}
                \quad (\text{\because\, Parsevalの等式}) \\
            &< \infty
    \end{alignat}
    なので、Cauchy の収束条件により級数$\sum |a_n|$は収束する。
    同様にして$\sum |b_n|$も収束する。
    したがって
    \begin{equation}
        |a_n \cos(nx) + b_n \sin(nx)| \le |a_n| + |b_n|
    \end{equation}
    の右辺は収束するから、Weierstrassの定理により、$S_N[f]$は$\R$上一様収束する。
\end{proof}

さて、フーリエ級数展開が一様収束するための条件はいくつか確認できたので、
次は肝心の「どこに収束するか?」を考えていきます。
そのためには各点収束の極限を考えればよいのですが、その前にいくつかの補題を準備しておきます。

\begin{lemma}
    フーリエの部分和は
    \begin{equation}
        S_N[f](x) = \frac{1}{2\pi}
            \int_{-\pi}^{\pi} f(x+y) \frac{\sin\left(N+\frac{1}{2}\right)y}{\sin\frac{1}{2}y} dy
    \end{equation}
    と書ける。
    \label{3:lemma1}
\end{lemma}

\begin{proof}
    三角数列の和の公式
    \begin{equation}
        \cos\alpha + \cos 2\alpha + \cdots + \cos n\alpha
            = \frac{\cos\left(\frac{n+1}{2}\alpha\right) \sin\left(\frac{n}{2}\alpha\right)}{\sin\frac{\alpha}{2}}
    \end{equation}
    と、$f$の周期性を利用した置換を用いて示します。
\end{proof}

    \begin{lemma}
        区間$[-\pi, \pi]$上で区分的に連続な関数$g$に対して次が成り立つ:
        \begin{equation}
            \lim_{n\to\infty} \frac{1}{\pi} \int_{-\pi}^{\pi} g(x) \sin\left(N + \frac{1}{2}\right) x dx = 0
        \end{equation}
        \label{3:lemma2}
    \end{lemma}

\begin{proof}
    リーマン・ルベーグの定理\footnote{
        リーマン・ルベーグの定理の主張は以下のとおりです。証明は参考文献\cite[第3章 例題3.3]{杉浦+89}を参照。
        有界閉区間$I=[a,b]$上で関数$f$が可積分であるとき、
        $\lim_{t\to\infty} \int_a^b f(x) \sin(tx) dx = 0$
        および
        $\lim_{t\to\infty} \int_a^b f(x) \cos(tx) dx = 0$
        が成り立つ。
    }
    より明らか。
\end{proof}

\begin{lemma}
    $f$を$\R$上の区分的$C^1$級関数とし、
    $x \in \R$を任意にとる。このとき
    \begin{gather}
        y \mapsto \dfrac{f(x + y) - f(x - 0)}{\sin(y/2)}
            \;\text{は}\; -\pi \le y \le 0 \text{ で、} \label{eq:lem3:a} \\[+1em]
        y \mapsto \dfrac{f(x + y) - f(x + 0)}{\sin(y/2)}
            \;\text{は}\; 0 \le y \le \pi \text{ で、} \label{eq:lem3:b}
    \end{gather}
    それぞれ区分的に連続である。
    \label{3:lemma3}
\end{lemma}

\begin{proof}
    $x \in \R$を任意に固定する。$x$は$f$の不連続点であってもよい。
    $y \neq 0$のときは(\ref{eq:lem3:a}), (\ref{eq:lem3:b})の分母は$0$でないから、
    $f$の区分的連続性により定理の主張が成り立つ。
    したがって$y \to 0$のときを考えればよく、
    以下$y \to -0$の場合を示す。$y \to +0$の場合も同様にして示せる。
    \begin{equation}
        \alpha(y) \coloneqq \dfrac{f(x + y) - f(x - 0)}{\sin(y/2)}
    \end{equation}
    とおくと
    \begin{equation}
        \begin{split}
            \alpha(y)
                &= \frac{f(x + y) - \lim_{\eps \to +0} f(x - \eps)}{\sin(y/2)} \\
                &= 2\, \frac{f(x + y) - \lim_{\eps \to +0} f(x - \eps)}{y} \frac{y/2}{\sin(y/2)} \\
                &= 2 \lim_{\eps \to +0} \frac{f(x - \eps + y) - f(x - \eps)}{y} \frac{y/2}{\sin(y/2)}
        \end{split}
    \end{equation}
    である。ただし、最後の式変形では$|y|$が充分小さいとき$f$が点$x + y$で連続であることを用いた。
    $\lim_{\eps \to +0}$の部分を$\eps$-$\delta$論法で書き直すと、
    $\forall \eta > 0$に対し$\exists \eps_\eta > 0\;$ s.t. $\; 0 < \forall \eps < \eps_\eta$に対し
    \begin{equation}
        \alpha(y) - \eta
            < 2\, \frac{f(x - \eps + y) - f(x - \eps)}{y} \frac{y/2}{\sin(y/2)}
            < \alpha(y) + \eta
            \label{eq:lem3:2}
    \end{equation}
    である。
    $\eps$を充分小さくとれば$f$は点$x - \eps$で微分可能、とくに左微分可能なので、
    式(\ref{eq:lem3:2})の第~2辺には$y \to -0$の極限が存在するが、
    それはとくに下極限と一致するから、
    式(\ref{eq:lem3:2})の各辺の$y \to -0$の下極限をとって
    \begin{equation}
        \liminf_{y \to -0} \alpha(y) - \eta \le 2 f'(x - \eps) \le \liminf_{y \to -0} \alpha(y) + \eta
    \end{equation}
    を得る。
    ふたたび$\eps$-$\delta$論法による極限の定義を思い出せば
    \begin{equation}
        2 f'(x - 0) = \liminf_{y \to -0} \alpha(y)
    \end{equation}
    を得る。
    同様の議論により
    \begin{equation}
        2 f'(x - 0) = \limsup_{y \to -0} \alpha(y)
    \end{equation}
    も示せる。$f$は区分的$C^1$級なので$f'(x - 0)$が存在し、したがって
    \begin{equation}
        \lim_{y \to -0} \alpha(y)
            = \liminf_{y \to -0} \alpha(y)
            = \limsup_{y \to -0} \alpha(y)
            = 2 f'(x - 0)
            \in \R
    \end{equation}
    である。
\end{proof}

\begin{theorem}
    $f$が区分的に$C^1$級の$2\pi$周期関数であるとき、任意の$x \in [-\pi, \pi]$に対して
    \begin{equation}
        S_N[f](x) \to \frac{f(x+0) - f(x-0)}{2} \quad (N \to \infty)
    \end{equation}
    が成り立つ。
\end{theorem}

この定理は、$f$のフーリエ級数展開が
\begin{itemize}
    \item $x$が連続点のときは$f(x)$に
    \item $x$が不連続点のときはそこでの "跳躍" の中央に
\end{itemize}
各点で収束するということを主張しています。

\begin{proof}
    \cref{3:lemma1}より、
    \begin{equation}
        S_N[f](x) = \frac{1}{2\pi}
            \int_{-\pi}^{\pi} f(y+x)\frac{\sin\left(N+\frac{1}{2}\right)y}{\sin\frac{1}{2}y} dy
    \end{equation}
    である。さらに
    \begin{equation}
        \int_{-\pi}^{0} \frac{\sin\left(N+\frac{1}{2}\right)y}{\sin\frac{1}{2}y} dy
        = \int_{0}^{\pi} \frac{\sin\left(N+\frac{1}{2}\right)y}{\sin\frac{1}{2}y} dy
        = \pi
    \end{equation}
    である。よって
    \begin{alignat}{2}
        &&&S_N[f](x) - \frac{f(x-0) + f(x+0)}{2} \\
        &=&&\frac{1}{2\pi} \int_{-\pi}^0 f(y+x) \frac{\sin\left(N+\frac{1}{2}\right)y}{\sin\frac{1}{2}y} dy
            -\frac{f(x-0)}{2} \nonumber \\
        &&+\,&\frac{1}{2\pi} \int_0^\pi f(y+x) \frac{\sin\left(N+\frac{1}{2}\right)y}{\sin\frac{1}{2}y} dy
            -\frac{f(x+0)}{2} \\
        &=&&\frac{1}{2\pi} \int_{-\pi}^0 f(y+x) \frac{\sin\left(N+\frac{1}{2}\right)y}{\sin\frac{1}{2}y} dy
            -\frac{1}{2\pi} \int_{-\pi}^0 f(x-0) \frac{\sin\left(N+\frac{1}{2}\right)y}{\sin\frac{1}{2}y} dy \nonumber \\
        &&+\,&\frac{1}{2\pi} \int_0^\pi f(y+x) \frac{\sin\left(N+\frac{1}{2}\right)y}{\sin\frac{1}{2}y} dy
            -\frac{1}{2\pi} \int_0^\pi f(x+0) \frac{\sin\left(N+\frac{1}{2}\right)y}{\sin\frac{1}{2}y} dy \\
        &=&&\frac{1}{2\pi}
            \int_{-\pi}^0 \frac{f(y+x) - f(x-0)}{\sin\frac{1}{2}y} \sin\left(N+\frac{1}{2}\right)y dy \nonumber \\
        &&+\,&\frac{1}{2\pi}
            \int_0^\pi \frac{f(y+x) - f(x+0)}{\sin\frac{1}{2}y} \sin\left(N+\frac{1}{2}\right)y dy \label{eq:thm3:1}
    \end{alignat}
    である。ここで
    \begin{equation}
        g(y) = \begin{cases}
            \displaystyle \frac{f(y+x) - f(x-0)}{\sin\frac{1}{2}y} & (-\pi \le y \le 0) \\[+1em]
            \displaystyle \frac{f(y+x) - f(x+0)}{\sin\frac{1}{2}y} & (0 < y \le \pi)
        \end{cases}
    \end{equation}
    とおけば、\cref{3:lemma3}により$g$は区間$[-\pi, \pi]$で区分的に連続である。
    したがって、\cref{3:lemma2}により
    \begin{equation}
        \text{(式(\ref{eq:thm3:1})) }
            = \frac{1}{2\pi} \int_{-\pi}^\pi g(y) \sin\left(N+\frac{1}{2}\right)y dy
            \to 0 \quad (N\to\infty)
    \end{equation}
    である。
    すなわち
    \begin{equation}
        S_N[f](x) \to \frac{f(x+0) - f(x-0)}{2} \quad (N \to \infty)
    \end{equation}
    がいえた。
\end{proof}

\begin{theorem}
    $f$が区分的な周期関数で$\int_{-\pi}^{\pi} f(x)^2 dx < \infty\,(f \in L^2(-\pi, \pi))$ならば
    \begin{equation}
        \int_{-\pi}^{\pi} (f(x) - S_N[f])^2 dx \to 0 \quad (N \to \infty)
    \end{equation}
    が成り立つ。
\end{theorem}

\begin{proof}
    難しいので省略\footnote{
        参考文献\cite[定理8.2.1]{吉田21}を参照
    }
\end{proof}




% ------------------------------------------------------------
%
% ------------------------------------------------------------
\section{複素フーリエ級数展開}
周期$2\pi l,\, l > 0$の複素数値関数$f: \R \to \C$に対し、
\begin{equation}
    \sum_{n = -\infty}^\infty c_n e^{i \frac{n}{l} x}
\end{equation}
を$f$の\textbf{複素フーリエ級数展開}と呼びます。
フーリエ係数$c_n$は、指数関数の直交性から
\begin{equation}
    c_n = \frac{1}{2\pi l} \int_{-l\pi}^{l\pi} f(x) e^{-i\frac{n}{l}x} dx
\end{equation}
と表せることがわかります。











\begin{problem}
    関数列
    \begin{equation}
        f_n(x) = \frac{e^{nx} - e^{-nx}}{e^{nx} + e^{-nx}}
    \end{equation}
    の極限$(n \to \infty)$を求めよ。

    解答:
    \begin{equation}
        f_n(x) \to \left\{\begin{alignedat}{2}
            \,&1 \quad &(x > 0) \\
            &0 \quad &(x = 0) \\
            &-1 \quad &(x < 0)
        \end{alignedat}
        \right.
    \end{equation}
\end{problem}

\begin{problem}
    $I = [0, 1],\, f_n(x) = x^n\, (n \in \N)$とおく。
    関数列$\{f_n\}_n$は各点収束するが一様収束しないことを示せ。
\end{problem}

\begin{problem}
    $I = [0, 2]$、
    \begin{equation}
        f_n(x) \coloneqq \begin{cases}
            n^2 x &(x \in [0, 1/n]) \\
            -n^2 (x - 2/n) &(x \in [1/n, 2/n]) \\
            0 &(\text{otherwise})
        \end{cases}
    \end{equation}
    とおく。
    関数列$\{f_n\}_n$は各点収束するが一様収束しないことを示せ。
    また、項別積分が一致しないことを確かめよ。
\end{problem}

\begin{problem}
    $I = [0, \infty),\, f_n(x) \coloneqq \frac{x}{nx + 1}$
    とおく。
    関数列$\{f_n\}_n$は$I$上一様収束することを示せ。
\end{problem}

\begin{problem}
    $I = [0, 1]$上の関数$f_n(x) \coloneqq \frac{x}{nx + 1}$は
    $n \to \infty$で$I$上一様収束することを示せ。
    また、項別積分が一致することを確かめよ。
\end{problem}

\begin{problem}
    $I = [-R, R]\, (R > 0)$上の関数項級数
    \begin{equation}
        \sum_{k \in \N} \frac{1}{x^2 + k^2}
    \end{equation}
    が一様収束することを示せ。
\end{problem}

\begin{problem}
    $I = [0, 2]$上の関数項級数
    \begin{equation}
        \sum_{k \in \N} \frac{1}{x^2 + k^2}
    \end{equation}
    が一様収束することを示せ。
\end{problem}

\begin{problem}
    \,
    \begin{itemize}
        \item \cite[第III章 例題6.1.1]{杉浦+89}
        \item \cite[第III章 問6.1.2 (2)]{杉浦+89}
        \item \cite[第III章 例題6.2.1]{杉浦+89}
        \item \cite[第III章 問6.2.1 (1),(2)]{杉浦+89}
    \end{itemize}
    を読者の演習問題とする。
\end{problem}

\begin{problem}
    次の初期値・境界値問題の解を求めよ。
    \begin{equation}
        \begin{split}
            &u_t(x, t) - u_{xx}(x, t) = 0 \quad \text{for $(x, t) \in (0, 1) \times (0, \infty)$} \\
            &u(0, t) = u(1, t) = 0 \\
            &u(x, 0) = 2 \sin(3\pi x) + 5 \sin(8\pi x) \quad \text{for $x \in [0, 1]$}
        \end{split}
    \end{equation}

    解答:
    \begin{equation}
        u(x, t) = 2 e^{-(3\pi)^2 t} \sin(3\pi x) + 5 e^{-(8\pi)^2 t} \sin(8\pi x)
    \end{equation}
\end{problem}

\begin{problem}
    次の初期値・境界値問題の解を求めよ。
    \begin{equation}
        \begin{split}
            &u_t(x, t) - u_{xx}(x, t) = 0 \quad \text{for $(x, t) \in (0, L) \times (0, \infty)$} \\
            &u(0, t) = u(L, t) = 0 \\
            &u(x, 0) = a(x) \quad \text{for $x \in [0, L]$}
        \end{split}
    \end{equation}
    ただし$\int_0^L a(x)^2 dx < \infty$とする。

    解答:
    \begin{equation}
        u(x, t) = \sum_{n=1}^\infty c_n \exp(-\lambda_n^2 t)\, \sin \frac{n\pi x}{L},\quad
        \lambda = \frac{n\pi}{L},\quad
        c_n = \frac{2}{L} \int_0^L a(x) \sin\frac{n\pi x}{L} dx
    \end{equation}
\end{problem}

\begin{problem}
    次の初期値・境界値問題の解を求めよ。
    \begin{equation}
        \begin{split}
            &u_t(x, t) - \textcolor{red}{u(x, t)} - u_{xx}(x, t) = 0 \quad \text{for $(x, t) \in (0, 10) \times (0, \infty)$} \\
            &u(0, t) = u(10, t) = 0 \\
            &u(x, 0) = 3 \sin(2\pi x) - 7 \sin(4\pi x) \quad \text{for $x \in [0, 10]$}
        \end{split}
    \end{equation}

    解答:
    \begin{equation}
        u(x, t) = e^t \left( 3 e^{-(2\pi)^2 t} \sin(2\pi x) - 7 e^{-(4\pi)^2 t} \sin(4\pi x) \right)
    \end{equation}
\end{problem}

\begin{problem}
    関数$x \mapsto \pi - |x| \quad(x \in [-\pi, \pi])$を$\R$全体に周期拡張した関数を$f$とおく。
    $f$のフーリエ級数展開を求めよ。

    解答:
    \begin{equation}
        f(x) \sim
            \frac{\pi}{2} + \frac{4}{\pi} \sum_{n=1}^\infty \frac{\cos((2n-1)x)}{(2n-1)^2}
    \end{equation}
\end{problem}

\begin{problem}
    関数
    \begin{equation}
        x \mapsto \begin{cases}
            1 &(0 < x \le \pi) \\
            0 &(x = 0) \\
            -1 &(-\pi < x < 0)
        \end{cases}
    \end{equation}
    を$\R$全体に周期拡張した関数を$f$とおく。
    $f$のフーリエ級数展開を求めよ。

    解答:
    \begin{equation}
        f(x) \sim
            \frac{\pi}{4} \sum_{n=1}^\infty \frac{\sin((2n-1)x)}{2n-1}
    \end{equation}
\end{problem}

\begin{problem}
    $0 \le x \le \pi$に対し$f(x) = -x (x - \pi)$とおく。
    $f$の周期$2\pi$の奇関数拡張、偶関数拡張のフーリエ級数展開を求めよ。

    解答:\\
    奇関数拡張:$b_n = \frac{4}{n^3 \pi} (1 - (-1)^n)$\\[0.5em]
    偶関数拡張:$\frac{1}{2} a_0 = \frac{1}{6} \pi^2,\, a_n = - \frac{2}{n^2} (1 + (-1)^n)$
\end{problem}

\begin{problem}
    \,
    \begin{itemize}
        \item \cite[第III章 例題3.3]{杉浦+89}
        \item \cite[第III章 問11.1 (1)-(4)]{杉浦+89}
    \end{itemize}
    を読者の演習問題とする。
\end{problem}


% ------------------------------------------------------------
%
% ------------------------------------------------------------
\section{パラメータを含む積分}

ここからはパラメータを含む積分の一様収束性など基礎的な事項を整理します。

以下では\mbox{2変数}関数$f(x, s)$が登場しますが、$x$の方がパラメータで、$s$は主変数です。
$x$の変域$\Omega$はコンパクトの場合のみを考えます。
一方で$s$の変域$I$は、まずコンパクトの場合を考えてから非有界区間の場合に拡張しますが、
このときに積分の一様収束性が仮定に加わることになります。

\begin{theorem}[$I$がコンパクトの場合]
    $\R$上の有界閉区間$\Omega \coloneq [\alpha, \beta],\, I \coloneq [a, b]$をとる。
    関数$f(x, s)$が$\Omega \times I$上連続ならば次が成り立つ。
    \begin{enumerate}
        \vspace{1em}
        \item $x$の関数$\displaystyle \int_a^b f(x, s)\, ds$は$\Omega$上連続である。
        \item $\displaystyle \int_\alpha^\beta \left( \int_a^b f(x, s)\, ds \right) dx \
            = \int_a^b \left( \int_\alpha^\beta f(x, s)\, dx \right) ds$
        \vspace{1em}
    \end{enumerate}
    さらに$\dfrac{\partial f}{\partial x}(x, s)$が存在して$\Omega \times I$上連続ならば次も成り立つ。
    \begin{enumerate}
        \setcounter{enumi}{2}
        \vspace{1em}
        \item $\displaystyle \dd{x} \int_a^b f(x, s)\, ds = \int_a^b \deldel[f]{x}(x, s)\, ds$
            と書けて$\Omega$上連続
    \end{enumerate}
    \label{4:thm:1}
\end{theorem}

\begin{proof}
    (1)
    $f$はコンパクト集合$\Omega \times I$上連続なので、
    $\Omega \times I$上一様連続でもある。
    したがって$\eps > 0$を任意にとると、$\exists \delta > 0$\, s.t.
    \begin{equation}
        | (x, s) - (y, t) | < \delta
        \quad \Rightarrow \quad
        |f(x, s) - f(y, t)| < \eps / (b - a)
    \end{equation}
    とくに
    \begin{equation}
        |x - y| < \delta
        \quad \Rightarrow \quad
        |f(x, s) - f(y, s)| < \eps / (b - a) \quad (\forall s \in I)
    \end{equation}
    すなわち
    \begin{equation}
        |x - y| < \delta
        \quad \Rightarrow \quad
        \| f(x,\, \cdot) - f(y,\, \cdot) \| \le \eps / (b - a)
    \end{equation}
    である。そこで$|x - y| < \delta$のとき
    \begin{equation}
        \begin{split}
            \left| \int_a^b f(x, s)\, ds - \int_a^b f(y, s)\, ds\right|
                &\le \int_a^b | f(x, s) - f(y, s) |\, ds \\
                &\le \| f(x,\, \cdot) - f(y,\, \cdot) \|\, (b - a) \\
                &\le \eps
        \end{split}
    \end{equation}
    である。
    したがって、$\int_a^b f(x, s)\, ds$は$\Omega$上(一様)連続である。

    (2) 長いので省略\footnote{\cite[第IV章 \S{7}]{杉浦80}}

    (3)
    $\deldel[f]{x}(x, s)$が連続ならば
    (1) より$\int_a^b \deldel[f]{x}(x, s) ds$も連続である。
    よって
    \begin{equation}
        \begin{split}
            \int_\alpha^x \left\{ \int_a^b \deldel[f]{x}(\xi, s)\, ds \right\} d\xi
                &= \int_a^b \left\{ \int_\alpha^x \deldel[f]{x}(\xi, s)\, d\xi \right\} ds \\
                &= \int_a^b (f(x, s) - f(\alpha, s)) ds \\
                &= \int_a^b f(x, s) ds - \underbrace{\int_a^b f(\alpha, s) ds}_{定数}
        \end{split}
    \end{equation}
    この両辺を$x$で微分して定理の式を得る。
\end{proof}

    \begin{theorem}[$I$が非有界区間の場合]
        $\R$上の有界閉区間$\Omega \coloneq [\alpha, \beta]$と
        非有界区間$I \coloneq [a, \infty),\, a \in \R$をとる。
        関数$f(x, s)$が$\Omega \times I$上連続で、
        \textcolor{red}{広義積分$\displaystyle \int_a^\infty f(x, s)\, ds$
        が$\Omega$上広義一様収束する}
        ならば次が成り立つ。
        \begin{enumerate}
            \vspace{1em}
            \item $x$の関数$\displaystyle \int_a^\infty f(x, s)\, ds$は$\Omega$上連続である。
            \item $\displaystyle \int_\alpha^\beta \left( \int_a^\infty f(x, s)\, ds \right) dx \
                = \int_a^\infty \left( \int_\alpha^\beta f(x, s)\, dx \right) ds$
            \vspace{1em}
        \end{enumerate}
        さらに$\dfrac{\partial f}{\partial x}(x, s)$が存在して$\Omega \times I$上連続で、
        \textcolor{red}{$\displaystyle \int_a^\infty \deldel[f]{x}(x, s)\, ds$
        が$\Omega$上広義一様収束する}
        ならば次も成り立つ。
        \begin{enumerate}
            \setcounter{enumi}{2}
            \vspace{1em}
            \item $\displaystyle \dd{x} \int_a^\infty f(x, s)\, ds = \int_a^\infty \deldel[f]{x}(x, s)\, ds$
                と書けて$\Omega$上連続
        \end{enumerate}
    \end{theorem}

実際は(3)だけを言いたいのであれば$\displaystyle \int_a^\infty f(x, s)\, ds$は各点収束でも問題ないのですが、
ここではこのまま進めます。

\begin{proof}
    (1), (2) \cref{4:thm:1}と、連続関数の広義一様収束極限も連続関数であるという定理を用いれば示せます。

    (3) \cref{4:thm:1}と同様の論法で示せます。
\end{proof}

さて、広義積分の一様収束性を判定する定理として、
第2回で登場した関数項級数に関する Weierstrass のMテストの類似が成り立ちます。
補題をひとつ提示してから定理を示します。

    \begin{lemma}[連続関数全体の空間の完備性]
        $\Omega \subset \R^n$とする。$\Omega$上の連続関数全体の集合に
        $\sup$ノルムから誘導される距離を入れた空間$C(\Omega)$は完備である。
        \label{4:lemma:1}
    \end{lemma}

\begin{proof}
    $C(\Omega)$の完備性を示すには、$C(\Omega)$の任意の Cauchy 列が
    $C(\Omega)$に極限を持つことをいえばよい。
    そこで、$\Omega$上の連続関数列$\{ F_n \}_{n \in \N}$であって
    \begin{equation}
        \lim_{n, m \to \infty} \| F_n - F_m \| = 0
        \label{4:eq:1}
    \end{equation}
    であるものを任意にとる。
    すると、$x \in \Omega$ごとに$\{ F_n \}_{n \in \N}$は Cauchy 列なので、
    $\R^n$の完備性より$\lim_{n \to \infty} F_n(x) \eqqcolon F(x) \cdots$ (1) が
    $\forall x \in \Omega$に対し存在する。
    よって、あとは$F \in C(\Omega)$を示せば定理がいえる。
    そこで$x' \in \Omega$を固定し、$x'$での$F$の連続性を示そう。

    (\ref{4:eq:1})より、$\forall \eps > 0$に対しある$N \in \N$が存在して
    \begin{equation}
        \forall n, m \ge N,\, \forall x \in \Omega,\, |F_n(x) - F_m(x)| < \eps / 3
    \end{equation}
    なので、$m \to \infty$として
    \begin{equation}
        \forall n \ge N,\, \forall x \in \Omega,\, |F_n(x) - F(x)| \le \eps / 3
    \end{equation}
    である。$F_n$の連続性より、$x'$の近傍$U$が存在して
    \begin{equation}
        x \in U \in \Omega \Rightarrow |F_n(x) - F_n(x')| < \eps / 3
    \end{equation}
    なので、$\forall x \in U$に対し
    \begin{equation}
        |F(x) - F(x')|
            \le |F(x) - F_n(x)| + |F_n(x) - F_n(x')| + |F_n(x') - F(x')| < \eps
    \end{equation}
    である。したがって$F$は$x'$で連続である。
\end{proof}

\begin{theorem}[広義積分に関する Weierstrass のMテスト]
    $\R$上の有界閉区間$\Omega \coloneq [\alpha, \beta]$と
    非有界区間$I \coloneq [a, \infty),\, a \in \R$をとる。
    与えられた連続関数$f \colon \Omega \times I \to \R$に対し、
    ある関数$\varphi: I \to \R$が存在して次を満たすと仮定する:
    \begin{enumerate}
        \item 十分大きな$\forall s \in I$に対し$\| f(\,\cdot\, , s) \| \le \varphi(s)$
        \item $\int_a^\infty \varphi(s)\, ds$が広義可積分
    \end{enumerate}
    このとき、広義積分$\int_a^\infty f(x, s)\, ds$は
    $\Omega$上一様収束する。
\end{theorem}

\begin{proof}
    $\int_a^\infty f(x, s)\, ds$に関する一様 Cauchy 条件の成立を、
    $\int_a^\infty \varphi(s)\, ds$に関する Cauchy 条件を用いて示します。
    極限関数の存在は、$\Omega$上の連続関数全体の空間が$\sup$ノルムに関して完備であることを用いて示します。
\end{proof}





% ------------------------------------------------------------
%
% ------------------------------------------------------------
\section{フーリエ変換}

フーリエ級数展開は直交関数系$\{ e^{i\frac{n}{l} x} \}_{n \in \Z}$による展開でしたが、
周期を持つとは限らない関数でも展開できるようにするため、関数系を非可算に拡張することを考えます。
天下り的に定義を述べると、$\R$上の複素数値広義可積分関数$f: \R \to \C$に対し、
\begin{equation}
    \calF [f] (\xi) := \frac{1}{\sqrt{2\pi}} \int_\R f(x)\, e^{-i\xi x} dx
    \label{eq:4:1:1}
\end{equation}
を$f$の\textbf{フーリエ変換}と呼び、
\begin{equation}
    \calF^{-1} [f] (x) := \frac{1}{\sqrt{2\pi}} \int_\R f(\xi)\, e^{i\xi x} d\xi
    \label{eq:4:1:2}
\end{equation}
を$f$の\textbf{逆フーリエ変換}と呼びます\footnote{
    係数の$1/\sqrt{2\pi}$は複素指数関数の周期$2\pi$に由来しており、
    この係数が出てこないような関数系を用いる流儀も存在します。
    なお、フーリエ変換も逆フーリエ変換も$\int_\R$が付いているのでどっちがどっちだかややこしいですが、
    フーリエ変換は "展開係数" であり、
    逆フーリエ変換は "重ね合わせ" にあたります。
}。
ここで、ある良い性質を持った$f$に対しては
\begin{equation}
    f(x) = \frac{1}{\sqrt{2\pi}} \int_\R \calF [f](\xi)\, e^{i\xi x} d\xi
\end{equation}
が成り立つことがわかっています。
すなわち、非可算な関数系$\{ e^{i\xi x} \}_{\xi \in \R}$を用いて、
$\{\calF [f](\xi)\}_{\xi \in \R}$を展開係数とした$f$の展開が得られるということです。

% ------------------------------------------------------------
%
% ------------------------------------------------------------
\section{急減少関数空間}

\begin{definition}
    関数$f: \R \to \C$が\textbf{急減少関数}であるとは、$f$が次をみたすことをいう\footnotemark :
    \begin{enumerate}
        \item $f \in C^{\infty}(\R)$
        \item 任意の$m, n \in \N$に対し
        \begin{equation}
            |f|_{m,n} := \sup_{x \in \R} |x|^m |f^{(n)} (x)| < \infty
        \end{equation}
    \end{enumerate}
    急減少関数全体の集合を$\calS = \calS(\R)$と書く。
\end{definition}

\footnotetext{
    「微分して\quad $x$たちを\quad 掛けたとて\quad その絶対値\quad 限りありけり」 — 詠み人知らず
}

    \begin{theorem}
        $f \in \calS(\R)$に対し$\calF [f],\, \calF^{-1} [f] \in \calS(\R)$
    \end{theorem}

\begin{proof}
    ややこしいので省略\footnote{\cite[第VII章 定理6.7]{杉浦85}}
\end{proof}

    \begin{theorem}[反転公式]
        $f \in \calS(\R)$とする。このとき、任意の$x \in \R$に対し\textbf{反転公式}
        \begin{equation}
            f(x) = \calF^{-1} \calF [f](x)
        \end{equation}
        が成り立つ。
    \end{theorem}

\begin{proof}
    ややこしいので省略\footnote{\cite[第VII章 定理6.7]{杉浦85}}
\end{proof}



% ------------------------------------------------------------
%
% ------------------------------------------------------------
\section{フーリエ変換の性質}

まず次の写像を準備しておきます:
\begin{itemize}
    \item 平行移動$\tau_h: x \mapsto x - h$
    \item 拡大・縮小$d_t: x \mapsto tx \quad (t > 0)$
    \item 反転$f^{\vee}(x) = f(-x)$
\end{itemize}

\begin{proposition}
    $f \in \calS(\R)$とする。このとき次が成り立つ:
    \begin{enumerate}
        \item $f \circ \tau_h,\, f \circ d_t,\, f^{\vee} \in \calS(\R)$
        \item $\calF [f \circ \tau_h](\xi) = e^{ih\xi} \calF [f](\xi)$
        \item $\calF [f \circ d_t](\xi) = \frac{1}{t} \calF [f] \left(\dfrac{\xi}{t}\right)$
        \item $\calF [f^{\vee}](\xi) = (\calF [f])^{\vee} (\xi) = \calF^{-1} [f](\xi)$
    \end{enumerate}
\end{proposition}

\begin{proof}
    (1) 簡単, (2) 自明, (4) 明らか

    (3)
    \vspace{-2em}\begin{equation}
        \begin{split}
            \calF[f \circ d_t](\xi)
                &= \frac{1}{\sqrt{2\pi}} \int_\R f \circ d_t (x) e^{-i\xi x} dx \\
                &= \frac{1}{\sqrt{2\pi}} \int_\R f(tx) e^{-i\xi x} dx \\
                &= \frac{1}{t} \frac{1}{\sqrt{2\pi}} \int_\R f(x) e^{-i\xi x / t} dx \\
                &= \frac{1}{t} \calF[f] (\xi / t)
        \end{split}
    \end{equation}
\end{proof}

\begin{theorem}
    $f, g \in \calS(\R)$とする。このとき次が成り立つ:
    \begin{itemize}
        \item 線形性:
            \begin{equation}
                \calF (f + g) = \calF f + \calF g, \quad \calF (\alpha f) = \alpha \calF f
                \tag{F1}
            \end{equation}
        \item 微分演算:
            \begin{equation}
                \calF \left[\dd{x} f\right] = i\xi\calF f,
                \quad \calF^{-1}\left[\dd{\xi} g\right] = -ix\calF^{-1} g
                \tag{F2}
            \end{equation}
        \item 掛け算:
            \begin{equation}
                \calF [xf] = i \dd{\xi} (\calF f)
                \tag{F3}
            \end{equation}
        \item 畳み込み\footnotemark: $\displaystyle (f * g)(x) := \int_\R f(x - t) g(t) dt$とおくと
            \begin{equation}
                \calF (f * g) = \sqrt{2 \pi} (\calF f) (\calF g)
                \tag{F4}
            \end{equation}
    \end{itemize}
\end{theorem}

\footnotetext{畳み込みは確率変数の和$X + Y$の確率分布を表すときに使ったりします。}

\begin{proof}
    (F1)は明らか。

    (F2)について、
    \begin{equation}
        \begin{split}
            \sqrt{2 \pi} \calF [f'](\xi)
                &= \int_\R f'(x) e^{-i\xi x} dx \\
                &= \left[ f(x) e^{-i\xi x} \right]_{x = -\infty}^\infty
                    + i\xi \int_\R f(x) e^{-i\xi x} dx \\
                &= i\xi \sqrt{2 \pi} \calF [f](\xi)
        \end{split}
    \end{equation}
    である。ただし、$f$が急減少関数であることからとくに$\sup_{x \in \R} |x| |f(x)| < \infty$、したがって
    \begin{equation}
        \lim_{x \to \pm\infty} |f(x) e^{i\xi x}| = \lim_{x \to \pm\infty} |f(x)| = 0
    \end{equation}
    であることを用いた。$\calF^{-1}$についても同様に示せる。

    (F3)について、
    \begin{equation}
        \begin{split}
            \sqrt{2\pi} \calF [xf](\xi)
                &= \int_\R xf(x) e^{-i\xi x} dx \\
                &= \int_\R f(x) \left(-\frac{1}{i}\right) \frac{\partial}{\partial \xi} (e^{-i\xi x}) dx \\
                &= \left(-\frac{1}{i}\right) \dd{\xi} \int_\R f(x) e^{-i\xi x} dx \\
                &= i \sqrt{2\pi} \dd{\xi} (\calF f) (\xi)
        \end{split}
    \end{equation}
    である。

    (F4)について、
    \begin{equation}
        \begin{split}
            \calF [f * g](\xi)
                &= \frac{1}{\sqrt{2\pi}} \int_\R \left( \int_\R f(x - t)\, g(t) dt \right) e^{-i\xi x} dx \\
                &= \frac{1}{\sqrt{2\pi}} \int_\R \int_\R f(x - t)\, e^{-i\xi (x - t)} g(t)\, e^{-i\xi t} dt dx \\
                &= \frac{1}{\sqrt{2\pi}} \int_\R
                    \left( \int_\R f(x - t)\, e^{-i\xi (x - t)} dx \right)
                    g(t)\, e^{-i\xi t} dt \\
                &= \frac{1}{\sqrt{2\pi}} \int_\R
                    \left( \int_\R f(x)\, e^{-i\xi x} dx \right)
                    g(t)\, e^{-i\xi t} dt \\
                &= \frac{1}{\sqrt{2\pi}} \int_\R
                    \sqrt{2\pi} (\calF f)(\xi)\,
                    g(t)\, e^{-i\xi t} dt \\
                &= \sqrt{2\pi} (\calF f)(\xi)
                    \frac{1}{\sqrt{2\pi}} \int_\R g(t)\, e^{-i\xi t} dt \\
                &= \sqrt{2\pi} (\calF f)(\xi) (\calF g)(\xi)
        \end{split}
    \end{equation}
    である。
\end{proof}





\begin{problem}
    $e^{-x^2} \in \calS$を確認せよ。
\end{problem}

\begin{problem}
    $\frac{1}{1+x^2} \not\in \calS$を確認せよ。
\end{problem}

\begin{problem}[ポアソン核]
    $f(x) = e^{-k|x|}\, (k > 0)$のフーリエ変換を求めよ。

    解答:
    \begin{equation}
        \calF[f](\xi) = \sqrt{\frac{2}{\pi}} \frac{k}{\xi^2 + k^2}
    \end{equation}
\end{problem}

\begin{problem}[ガウス核]
    $f(x) = e^{-\alpha x^2}\, (\alpha > 0)$のフーリエ変換を求めよ。

    解答:
    \begin{equation}
        \calF[f](\xi) = \sqrt{\frac{1}{2\alpha}} \exp\left(-\frac{\xi^2}{4\alpha}\right)
    \end{equation}
\end{problem}

\begin{problem}[全空間上の熱方程式]
    熱方程式の初期値問題
    \begin{equation}
        \begin{split}
            &u_t = u_{xx} \quad (-\infty < x < \infty,\, t > 0) \\
            &|u| \to 0 \quad (|x| \to \infty) \\
            &u(x, 0) = a(x)
        \end{split}
    \end{equation}
    の解を求めよ。

    解答;
    \begin{equation}
        u(x, t) = \int_{-\infty}^\infty \frac{1}{\sqrt{4\pi t}} e^{-\frac{(x-z)^2}{4t}} a(z) dz
    \end{equation}
\end{problem}

\begin{problem}
    次の関数のフーリエ変換を求めよ。$\alpha > 0$に対して
    \begin{equation}
        f(x) = \begin{cases}
            1 \quad (|x| \le \alpha) \\
            0 \quad (|x| > \alpha)
        \end{cases}
    \end{equation}

    解答:
    \begin{equation}
        \calF[f](\xi) = \sqrt{\frac{2}{\pi}} \frac{\sin(\alpha \xi)}{\xi}
    \end{equation}
\end{problem}

\begin{problem}
    次の関数のフーリエ変換を求めよ。$L > 0$に対して
    \begin{equation}
        f(x) = \begin{cases}
            x \quad (0 \le x \le L) \\
            0 \quad (\text{otherwise})
        \end{cases}
    \end{equation}

    解答:
    \begin{equation}
        \calF[f](\xi) = \frac{1}{\sqrt{2\pi}} \frac{\exp(-iL\xi)\, (1 + iL\xi) - 1}{\xi^2}
    \end{equation}
\end{problem}

\begin{problem}
    次の関数のフーリエ変換を求めよ。
    \begin{equation}
        f(x) = \frac{1}{\sqrt{|x|}}
    \end{equation}
    ただし次のことは用いてよい。
    \begin{equation}
        \int_0^\infty \sin x^2 dx = \int_0^\infty \cos x^2 dx = \sqrt{\frac{\pi}{8}}
    \end{equation}

    解答:
    \begin{equation}
        \calF[f](\xi) = \sqrt{\frac{1}{|\xi|}}
    \end{equation}
\end{problem}

\begin{problem}
    次の積分を計算せよ。
    \begin{equation}
        F(x) = \int_0^\infty e^{-t^2} \cos(xt) dt
    \end{equation}

    解答: $\frac{\sqrt{\pi}}{2} \exp\frac{-x^2}{4}$
\end{problem}

\begin{problem}
    連続関数$a(x)$は定数$A > 0$に対して$|a(x)| \le A\, (x \in \R)$をみたすものとし、
    $(x, t) \in \R \times [0, \infty)$で定義された関数
    \begin{equation}
        u(x, t) = \int_{-\infty}^\infty
            \frac{1}{2\sqrt{\pi t}} \exp\left\{ - \frac{(x - y)^2}{4t} \right\} a(y)\, dy
    \end{equation}
    を考える。
    このとき、$u_x(x, t)$は$\R \times (0, \infty)$上連続で
    \begin{equation}
        u_x(x, t) = \int_{-\infty}^\infty
            \frac{-1}{2\sqrt{\pi t}}
            \frac{x - y}{2t}
            \exp\left\{ - \frac{(x - y)^2}{4t} \right\} a(y)\, dy
    \end{equation}
    と表せることを示せ。
\end{problem}

\begin{problem}
    \,
    \begin{itemize}
        \item \cite[第VII章{\S}6 問題1)]{杉浦85}
        \item \cite[第III章 問7.1 (1)]{杉浦+89}
        \item \cite[第III章 問7.2 (1)]{杉浦+89}
    \end{itemize}
    を読者の演習問題とする。
\end{problem}




% ============================================================
%
% ============================================================
\chapter{Lagrange の未定乗数法}


% ------------------------------------------------------------
%
% ------------------------------------------------------------
\section{条件付き極値問題}

ここでは$U$を$\R^n$の空でない開集合とし、$f \colon U \to \R,\, \bm{g} \colon U \to \R^m$を$C^1$級とします。
さらに
\begin{equation}
    S_g \coloneqq \{ \bm{x} \in U \mid \bm{g}(\bm{x}) = 0 \}
\end{equation}
とおきます。さて、$S_g$は拘束条件$\bm{g}(\bm{x}) = 0$をみたす零点集合ですが、
$\bm{x}$がこの集合に沿って動くときの$f(\bm{x})$の極値を求めたいというのが
Lagrange の未定乗数法のモチベーションです。

\begin{definition}
    $\bm{x_0} \in S_g$とする。$\exists r > 0$\quad s.t.
    \begin{equation}
        f(\bm{x}) \le f(\bm{x_0}) \quad \text{for $\forall \bm{x} \in B_r(\bm{x_0}) \cap S_g$}
    \end{equation}
    が成り立つとき、$f$は\textbf{$\bm{x_0}$において$S_g$上の極大値をとる}という。
\end{definition}

$\bm{x}$が$S_g$に沿って動くときの$f(\bm{x})$の極値を求めるには、
素朴なアイディアとしては$\bm{x}$が$S_g$に沿って動くときの$f$の方向微分を考えればよさそうです。
しかし、そのような条件をきちんと考慮するのは結構面倒です。
そこで登場するのが、方向微分など持ち出さずとも単なる勾配$\nabla f$を考えればよいことを保証してくれる次の定理です。

\begin{theorem}
    $U, f, \bm{g}$と$\bm{a} \in S_g$に対し
    \begin{enumerate}
        \item $\bm{a}$において$f$は$S_g$上の極値をとる
        \item $\rank D\bm{g}(\bm{a}) = m$
            \quad ただし$D\bm{g}(\bm{a})
                = \left( g_{i x_j}(\bm{a}) \right)_{\substack{1 \le i \le m \\ 1 \le j \le n}}$
    \end{enumerate}
    が成り立つならば、$\exists \bm{\lambda} = (\lambda_1, \dots, \lambda_m) \in \R^m$\quad s.t.
    \begin{equation}
        \nabla f(\bm{a}) = \bm{\lambda} D\bm{g}(\bm{a})
            \quad\text{i.e.}\quad f_{x_j}(\bm{a}) = \sum_{i=1}^m \lambda_i g_{i x_j}(\bm{a})
    \end{equation}
\end{theorem}

\begin{proof}
    $\rank D\bm{g}(\bm{a}) = m$なので$n \ge m$であり、
    $\rank$の性質より行列$D\bm{g}(\bm{a})$の$0$でない小行列式の最大次数が$m$である。
    そこで、議論の一般性を失うことなく、必要ならば$x_j$の番号を取り替えて
    \begin{equation}
        \deldel[(g_1, \dots, g_m)]{(x_{n-m+1}, \dots, x_{n})} \neq 0
    \end{equation}
    とできる。
    要するに行列$D\bm{g}(\bm{a})$の "右側" が正則となるように並び替えるわけである。
    ここで、行列$D\bm{g}(\bm{a})$を "右側" と "左側" に分けたのにあわせて、
    定理で与えられたベクトルも成分を書き分けておく。すなわち
    \begin{equation}
        \bm{x} \coloneqq \begin{pmatrix}
            \bm{y} \\
            \bm{z}
        \end{pmatrix},
        \quad
        \bm{y} \coloneqq \begin{pmatrix}
            x_1 \\
            \vdots \\
            x_{n-m}
        \end{pmatrix},
        \quad
        \bm{z} \coloneqq \begin{pmatrix}
            x_{n-m+1} \\
            \vdots \\
            x_{n}
        \end{pmatrix},
        \quad
        \bm{a} \coloneqq \begin{pmatrix}
            \bm{b} \\
            \bm{c}
        \end{pmatrix}
    \end{equation}
    とおく。
    陰関数定理より、方程式$\bm{g}(\bm{y}, \bm{z}) = 0$は点$\bm{a}$の近傍で
    $\bm{z} = \phai(\bm{y})$と解ける。
    ここで$F \colon V \to \R,$
    \begin{equation}
        F(\bm{y}) \coloneqq f(\bm{y}, \phai(\bm{y}))
    \end{equation}
    とおくと、これは$f(\bm{x})$を点$\bm{x}$が$S_g$に沿って動くようにしたものとみなせる。
    よって$F$は$\bm{y} = \bm{b}$で極値をとる。
    すなわち$\nabla F(\bm{b}) = 0$である。
    一方、
    \begin{equation}
        \begin{alignedat}{2}
            \nabla F(\bm{y})
                &= \nabla_y f(\bm{y}, \phai(\bm{y}))
                    + \nabla_z f(\bm{y}, \phai(\bm{y}))\, D \phai(\bm{y})
                    &&\quad (\because\, \text{連鎖律}) \\
                &= \nabla_y f(\bm{y}, \phai(\bm{y}))
                    - \textcolor{blue}{
                        \nabla_z f(\bm{y}, \phai(\bm{y}))\,
                        D_z \bm{g}(\bm{y}, \phai(\bm{y}))^{-1}
                    }
                    D_y \bm{g}(\bm{y}, \phai(\bm{y}))
                    &&\quad (\because\, \text{陰関数定理})
        \end{alignedat}
    \end{equation}
    であるが、青文字の部分に$\bm{y} = \bm{b}$を代入したものを
    \begin{equation}
        \bm{\lambda} \coloneqq
            \nabla_z f(\bm{b}, \phai(\bm{b}))\,
            D_z \bm{g}(\bm{b}, \phai(\bm{b}))^{-1}
    \end{equation}
    とおけば、$\nabla F(\bm{b}) = 0$より
    \begin{equation}
        \begin{split}
            \nabla_y f(\bm{b}, \phai(\bm{b})) &= \bm{\lambda} D_y \bm{g}(\bm{b}, \phai(\bm{b})) \\
            \nabla_z f(\bm{b}, \phai(\bm{b})) &= \bm{\lambda} D_z \bm{g}(\bm{b}, \phai(\bm{b}))
        \end{split}
        \hspace{3em} \text{i.e.} \hspace{3em}
        \begin{split}
            \nabla_y f(\bm{a}) &= \bm{\lambda} D_y \bm{g}(\bm{a}) \\
            \nabla_z f(\bm{a}) &= \bm{\lambda} D_z \bm{g}(\bm{a})
        \end{split}
    \end{equation}
    すなわち
    \begin{equation}
        \nabla f(\bm{a}) = \bm{\lambda} D \bm{g}(\bm{a})
    \end{equation}
    を得る。
\end{proof}

\begin{corollary}[Lagrange の未定乗数法]
    $U, f, \bm{g}$と$\bm{a} \in S_g$に対し、
    \begin{enumerate}
        \item $\bm{a}$において$f$は$S_g$上の極値をとる
    \end{enumerate}
    ならば、次のいずれか一方のみが成り立つ。
    \begin{enumerate}
        \item $F \colon \R^{n+m} \to \R,\;
            F(\bm{x}, \bm{\lambda}) = f(\bm{x}) - \bm{\lambda} \cdot \bm{g}(\bm{x})$
            に対し$\exists \bm{\lambda_0} \in \R^m$\, s.t.
            \begin{equation}
                \nabla F(\bm{a}, \bm{\lambda_0}) = 0
            \end{equation}
        \item $\rank D\bm{g}(\bm{a}) < m$
    \end{enumerate}
\end{corollary}

\begin{proof}
    簡単なので省略
\end{proof}












\begin{problem}
    方程式
    \begin{equation}
        f(x, y, z) \coloneqq x^2 + (x - y^2 + 1) z - z^3 = 0
    \end{equation}
    を満たす点$(x, y, z)$が点$(0, 0, 1)$の近傍で$z = \phai(x, y)$と書けることを示せ。
    また、この点における$\deldel[z]{x},\, \deldel[z]{y}$の値を求めよ。

    解答:
    \begin{equation}
        \deldel[z]{x} = \frac{1}{2},\quad \deldel[z]{y} = 0
    \end{equation}
\end{problem}

\begin{problem}
    変数$x, y, z, u, v$が
    \begin{equation}
        \begin{cases}
            &xy + uv = 0 \\
            &x^2 + y^2 + z^2 = u^2 + v^2
        \end{cases}
    \end{equation}
    を満たすとする。
    点$(2, 0, 1, 0, \sqrt{5})$の近傍で$(u, v) = \phai(x, y, z)$と書けることを示せ。
    また、点$(x, y, z) = (2, 0, 1)$における$\phai$のヤコビ行列を求めよ。

    解答:
    \begin{equation}
        \frac{1}{\sqrt{5}} \begin{bmatrix}
            0 & -2 & 0 \\
            2 & 0 & 1
        \end{bmatrix}
    \end{equation}
\end{problem}

\begin{problem}
    \cite[第II章 問6.2]{杉浦+89}
    を読者の演習問題とする。
\end{problem}



\begin{problem}
    写像$f: \R^3 \to \R^3,$
    \begin{equation}
        f(x_1, x_2, x_3) = \left( \sum_i x_i, \sum_i x_i^2, \sum_i x_i^3\right)
    \end{equation}
    が点$(a_1, a_2, a_3)$の近傍で一対一対応となるような$a_i$の条件を求めよ。
\end{problem}

\begin{problem}
    変数$x = (x_1, x_2, x_3),\, u = (u_1, u_2, u_3),\, v = (v_1, v_2, v_3)$の間に
    \begin{equation}
        \begin{cases}
            u_1 = x_1 + x_2 + x_3 \\
            u_2 = x_1 x_2 + x_2 x_3 + x_3 x_1 \\
            u_3 = x_1 x_2 x_3
        \end{cases}
        \quad
        \begin{cases}
            v_1 = x_1^2 + x_2^2 \\
            v_2 = x_2^2 + x_3^2 \\
            v_3 = x_3^2 + x_1^2
        \end{cases}
    \end{equation}
    という関係があるとし、$a = (a_1, a_2, a_3)$とする。
    $a_i \neq a_j\, (i \neq j)$のとき、
    $x = a$の近傍で$v$は$u$の関数として表せることを示せ。
    さらに$a_1 a_2 a_3 \neq 0$とし、
    $x = a$に対し定まる$u$を$b$とおく。
    このとき$x = a$の近傍で$u$は$v$の関数として表せることを示し、
    写像$v \mapsto u$の点$b$におけるヤコビ行列式を求めよ。

    解答:
    \begin{equation}
        \frac{(a_1 - a_2)(a_2 - a_3)(a_1 - a_3)}{16 a_1 a_2 a_3}
    \end{equation}
\end{problem}

\begin{problem}
    \cite[第II章 例題6.2]{杉浦+89}、
    \cite[第II章 問6.3]{杉浦+89}
    を読者の演習問題とする。
\end{problem}


\begin{problem}
    $a \in \R^n\, (a \neq0),\, b \in \R$とし、
    \begin{equation}
        g(x) \coloneqq \sum_{i = 1}^n a_i x_i + b
    \end{equation}
    とおく。$S_g \coloneqq \{ x \in \R^n \mid g(x) = 0 \}$のもとで
    \begin{equation}
        f(x) \coloneqq |x|^2
    \end{equation}
    の最小値を求めよ。

    解答:$\frac{b^2}{|a|^2}$
\end{problem}


\begin{problem}
    $S_g \coloneqq \{(x, y, z) \mid g(x, y, z) = x^2 + y^2 + z^2 - 1 = 0\}$のもとで
    \begin{equation}
        f(x, y, z) \coloneqq x^2 + y^2 - z^2 + 4xz + 4yz
    \end{equation}
    の極値を求め、極大・極小を判定せよ。

    解答:極小値$-3$、極大値$3$
\end{problem}

\begin{problem}
    $(x, y) \in \R^2$で定義された関数$f(x, y) = xy (x^2 + y^2 - 1)$の極値を求めよ。

    解答:極小値$f(\pm 1/2, \pm 1/2) = -1/8$、極大値$(\pm 1/2, \mp 1/2) = 1/8$
\end{problem}

\begin{problem}
    $(x, y) \in \R^2$に対し
    \begin{equation}
        \begin{split}
            f(x, y) &= x^4 + y^4 \\
            g(x, y) &= xy - 4
        \end{split}
    \end{equation}
    を考える。$M_g \coloneqq \{ (x, y) \mid g(x, y) = 0 \}$上での$f$の最大値、最小値を求めよ。

    解答:最大値なし、最小値$32$
\end{problem}

\begin{problem}
    \cite[第II章 問題7.1, 7.2, 7.4]{杉浦+89}を読者の演習問題とする。
\end{problem}



% ============================================================
%
% ============================================================
\chapter{最小二乗法}

% ------------------------------------------------------------
%
% ------------------------------------------------------------
\section{最小二乗法}

$n$個の点$(x_i, y_i)\, (i = 1, \dots, n)$が与えられたとします。
$x_i, y_i$の間には、パラメータ$a, b \in \R$によって
\begin{equation}
    y_i = a x_i + b\quad (i = 1, \dots, n)
\end{equation}
の関係があると仮定します。このとき、誤差
\begin{equation}
    e_i \coloneqq y_i - (a x_i + b)\quad (i = 1, \dots, n)
\end{equation}
の2乗和
\begin{equation}
    J(a, b) \coloneqq \sum_{i=1}^n e_i^2 = \sum_{i=1}^n (y_i - a x_i - b)^2
\end{equation}
が最小となるような$(a, b)$を求めることを考えます。
この問題は$J(a, b)$の極小値を求める問題に帰着されるので、以上の状況設定で充分といえば充分なのですが、
多変数(すなわち各点が$(x_{i1}, \dots, x_{im}, y_i)$である場合)への拡張を見据えて
もう少し一般性のある形に書き換えてみます。
すなわち、
\begin{equation}
    \begin{split}
        \bm{x} \coloneqq (x_1, \dots, x_n)^\tra,\quad
        \bm{y} \coloneqq (y_1, \dots, y_n)^\tra \\
        Q \coloneqq \begin{bmatrix}
            x_1 & 1 \\
            \vdots & \vdots \\
            x_n & 1
        \end{bmatrix},\quad
        \bm{a} \coloneqq \begin{bmatrix}
            a \\
            b
        \end{bmatrix},\quad
        \bm{e} \coloneqq \bm{y} - Q \bm{a}
    \end{split}
\end{equation}
とおきます。すると簡単な計算から
\begin{equation}
    J(a, b) = \|\bm{y}\|^2 + \langle Q^\tra Q \bm{a}, \bm{a} \rangle - \langle 2Q^\tra \bm{y}, \bm{a} \rangle
\end{equation}
が成り立つので、$J(a, b)$の最小化問題は
\begin{equation}
    F(\bm{v}) \coloneqq \langle Q^\tra Q \bm{v}, \bm{v} \rangle - \langle 2Q^\tra \bm{y}, \bm{v} \rangle
\end{equation}
の最小化問題に帰着されます。
ここで、$J$が極値をとるための必要条件$\nabla J(a, b) = 0$は、少し計算すると
\begin{equation}
    Q^\tra Q \bm{a} = Q^\tra \bm{y}
    \label{eq:10:1}
\end{equation}
と表せることがわかります。
したがって、$Q^\tra Q$が正則ならば$\bm{a}$が一意に定まってくれて嬉しいのですが、
次の命題によれば、実用上ほとんどの場合$Q^\tra Q$は正則だということがわかります。

    \begin{proposition}
        $Q$に対し次が成り立つ。
        \begin{enumerate}
            \item $Q^\tra Q$は非負定値対称行列である
            \item $x_1 = \dots = x_n$ではないとすると、$Q^\tra Q$は正定値対称行列である。
                したがって正則である。
        \end{enumerate}
    \end{proposition}

\begin{proof}
    (1)は内積を行列の積の形に書き直せばすぐわかります。

    (2)は成分ごとの方程式を考えて矛盾をいえば示せます。
\end{proof}

(\ref{eq:10:1})をみたす$\bm{a}$が$J(a, b)$の最小化問題の一意的な解であることを
明確に述べたのが次の定理です。

\begin{theorem}
    $x_1 = \dots = x_n$でなければ、(\ref{eq:10:1})をみたす$\bm{a}$は
    \begin{equation}
        F(\bm{a}) < F(\bm{v}) \quad (\bm{v} \in \R^2,\, \bm{v} \neq \bm{a})
        \label{eq:10:2}
    \end{equation}
    をみたす。
\end{theorem}

\begin{proof}
    $\bm{v} = \bm{a} + (\bm{v} - \bm{a})$と分解して
    $F(\bm{v})$と$F(\bm{a})$の間の不等式を導けば示せます。
    途中で$\bm{a}$が(\ref{eq:10:1})をみたすという性質を使って式を綺麗にします。
\end{proof}

上の定理は逆も成り立ちます。

\begin{theorem}
    (\ref{eq:10:2})をみたす$\bm{a}$は(\ref{eq:10:1})をみたす。
\end{theorem}

\begin{proof}
    $f(t) \coloneqq F(\bm{a} + t\bm{w})$は$t$に関して下に凸な2次関数ですが、
    $f(t)$が$t = 0$で極値をとることから$Q^\tra Q \bm{a} = Q^\tra \bm{y}$を導くことができます。
\end{proof}



% ============================================================
%
% ============================================================
\chapter{Gamma 関数と Beta 関数}

% ------------------------------------------------------------
%
% ------------------------------------------------------------
\section{Gamma 関数とBeta 関数}

ここでは\textbf{Gamma 関数}および\textbf{Beta 関数}という特殊関数を扱います。
本題に入る前に、まず重積分の変数変換公式を確認しておきます。

\begin{theorem}[変数変換公式]
    $U, V$を$\R^n$の有界部分集合とし、
    写像$\Phi \colon U \to V,$
    \begin{equation}
        \Phi(u) = (X_1(u), \dots, X_n(u))
    \end{equation}
    は全単射かつ$C^1$級であるとする。
    さらに$\forall u \in U$に対し
    \begin{equation}
        \deldel[(X_1, \dots, X_n)]{(u_1, \dots, u_n)}(u) \neq 0
    \end{equation}
    とする。
    このとき、$U$の任意の体積確定部分集合$U_1$と
    $V_1 \coloneqq \Phi(U_1)$に対し
    \begin{equation}
        \int_{U_1} f(x) dx = \int_{V_1} f(\Phi(u))\, |\det J_\Phi(u)|\, du
    \end{equation}
    が成り立つ。
\end{theorem}

\begin{proof}
    長いので省略\footnote{
        参考文献\cite[第VII章 \S{4}]{杉浦85}を参照。
    }
\end{proof}

\begin{example*}[$n$次元極座標変換]
    $\R^n$の極座標変換は
    \begin{equation}
        \begin{cases}
            x_1 &= r \cos \theta_1 \\
            x_2 &= r \sin \theta_1 \cos \theta_2 \\
            x_3 &= r \sin \theta_1 \sin \theta_2 \cos \theta_3 \\
            \vdots \\
            x_{n-1} &= r \sin \theta_1 \cdots \sin \theta_{n-2} \cos \theta_{n-1} \\
            x_{n} &= r \sin \theta_1 \cdots \sin \theta_{n-2} \sin \theta_{n-1}
        \end{cases}
    \end{equation}
    ただし
    \begin{equation}
        \begin{split}
            &0 \le r < \infty \\
            &0 \le \theta_i \le \pi \quad (i = 1, \dots, n - 2) \\
            &0 \le \theta_{n-1} \le 2\pi
        \end{split}
    \end{equation}
    で与えられます。ヤコビアンは
    \begin{equation}
        \det J_\Phi(r, \theta_1, \dots, \theta_{n-1})
            = r^{n-1} \sin^{n-2} \theta_1 \sin^{n-3} \theta_2 \cdots \sin \theta_{n-2}
    \end{equation}
    です。
\end{example*}

\begin{theorem}
    \begin{enumerate}
        \item $x > 0$に対し、広義積分
            \begin{equation}
                \int_0^\infty e^{-t} t^{x - 1} dt
            \end{equation}
            は各点で絶対収束する。
        \item $x, y > 0$に対し、広義積分
            \begin{equation}
                \int_0^1 t^{x - 1} (1 - t)^{y - 1} dt
            \end{equation}
            は各点で絶対収束する。
    \end{enumerate}
    \label{11:thm:1}
\end{theorem}

\begin{definition}
    \cref{11:thm:1}の(1)で定義される関数$\Gamma(x)$を\textbf{Gamma 関数}、
    (2)で定義される関数$B(x)$を\textbf{Beta 関数}という。
\end{definition}

見ての通り、$\Gamma(x)$や$B(x, y)$はパラメータを含む広義積分で定義された関数です。
%実はもっと強く広義一様収束までいえるのですが、ここではとりあえず各点収束を示します。

\begin{proof}[\cref{11:thm:1}の証明.]
    (1)
    $x > 0$を任意にとる。
    積分範囲が非有界な$\int_1^\infty e^{-t} t^{x-1} dt$と
    被積分関数が非有界な$\int_0^1 e^{-t} t^{x-1} dt$とに分けて収束性を考える。
    まず$\int_1^\infty e^{-t} t^{x-1} dt$を考える。任意の正整数$n$に対し
    \begin{equation}
        e^{-t} = O(t^{-n})\quad (t \to \infty)
    \end{equation}
    なので、
    \begin{equation}
        e^{-t} t^{x-1} = O(t^{x-n-1})\quad (t \to \infty)
    \end{equation}
    である。$n > x$をひとつ選べば
    \begin{equation}
        \int_1^\infty t^{x-n-1} dt
            = \left[ -\frac{1}{n-x} t^{-(n-x)} \right]_1^\infty
            = \frac{1}{n - x}
            \in \R
    \end{equation}
    なので、優関数の方法により
    \begin{equation}
        \int_1^\infty e^{-t} t^{x - 1} dt
    \end{equation}
    も絶対収束する。

    つぎに$\int_0^1 e^{-t} t^{x-1} dt$を考える。
    $e^{-t}$は$t = 0$の近傍で有界なので
    \begin{equation}
        e^{-t} t^{x-1} = O(t^{x-1})\quad (t \to +0)
    \end{equation}
    である。$0 < x < 1$のとき
    \begin{equation}
        \int_0^1 t^{x-1} dt
            = \left[ \frac{1}{x} t^{x} \right]_0^1
            = \frac{1}{x}
            \in \R
    \end{equation}
    なので、優関数の方法により
    \begin{equation}
        \int_0^1 e^{-t} t^{x - 1} dt
    \end{equation}
    も絶対収束する。$x \ge 1$のときは$\int_0^1 e^{-t} t^{x-1} dt$は広義でない普通の積分である。

    $x > 0$は任意であったから、$\int_0^\infty e^{-t} t^{x - 1} dt$は$x > 0$の各点で絶対収束する。
    \\

    (2) $x, y < 1$のときを考えれば充分です。(1)と同様に広義積分を分けて収束性を議論すれば示せます。
\end{proof}

\begin{proposition}[Gamma 関数と Beta 関数の基本性質]
    $x, y > 0$に対し次が成り立つ。
    \begin{enumerate}
        \item $\Gamma(1) = 1, \Gamma(x + 1) = x \Gamma(x)$
        \item $\Gamma(x + n) = (x + n - 1)(x + n - 2) \cdots x \Gamma(x)$
            \quad とくに \quad
            $\Gamma(n + 1) = n!$
        \item $B(x, y) = B(y, x)$ \vspace{0.5em}
        \item $B(x, y) = \frac{\Gamma(x)\, \Gamma(y)}{\Gamma(x + y)}$
    \end{enumerate}
\end{proposition}

\begin{proof}
    (1), (2), (3) は簡単です。

    (4)
    変数変換によって
    \begin{equation}
        \begin{split}
            \Gamma(x) &= 2 \int_0^\infty e^{-u^2} u^{2x-1} du \\
            B(x, y) &= 2 \int_0^{\pi/2} \sin^{2x-1} \theta \cos^{2y-1} \theta\, d\theta
        \end{split}
    \end{equation}
    と書けることに注意する。
    集合列$\{J_R\}_{R \in \N},\, \{I_R\}_{\R \in \N}$をそれぞれ
    \begin{equation}
        \begin{split}
            J_R &\coloneqq [0, R] \times [0, R] \\
            I_R &\coloneqq \{ (u, v) \in \R^2 \mid u^2 + v^2 \le R^2,\, u \ge 0,\, v \ge 0 \}
        \end{split}
    \end{equation}
    と定めると、これらは$[0, \infty) \times [0, \infty)$のコンパクト近似列\footnote{
        コンパクト近似列の定義は参考文献\cite[第VII章 \S{1}]{杉浦85}を参照。
    }になっている。
    \begin{alignat}{3}
        \Gamma(x) \Gamma(y)
            &= 4 \int_0^\infty e^{-u^2} u^{2x-1} du
                \int_0^\infty e^{-v^2} v^{2y-1} dv \\
            &= 4 \lim_{R \to \infty}
                \int_0^R e^{-u^2} u^{2x-1} du
                \int_0^R e^{-v^2} v^{2y-1} dv \\
            &= 4 \lim_{R \to \infty}
                \int_0^R \int_0^R e^{-(u^2 + v^2)} u^{2x-1} v^{2y-1} du\, dv \\
            &= 4 \lim_{R \to \infty}
                \iint_{J_R} e^{-(u^2 + v^2)} u^{2x-1} v^{2y-1} du\, dv \\
        \intertext{広義重積分可能ならば近似列を交換できるから}
            &= 4 \lim_{R \to \infty}
                \iint_{I_R} e^{-(u^2 + v^2)} u^{2x-1} v^{2y-1} du\, dv \\
            &= 4 \lim_{R \to \infty}
                \int_0^{\pi/2} \int_0^R
                    e^{-r^2} r^{2(x + y) - 1} \cos^{2x - 1} \theta \sin^{2y - 1} \theta dr\, d\theta \\
            &= \lim_{R \to \infty}
                2 \int_0^R e^{-r^2} r^{2(x + y) - 1} dr
                \cdot 2 \int_0^{\pi/2} \cos^{2x - 1} \theta \sin^{2y - 1} \theta d\theta \\
            &= 2 \int_0^\infty e^{-r^2} r^{2(x + y) - 1} dr
                \cdot 2 \int_0^{\pi/2} \cos^{2x - 1} \theta \sin^{2y - 1} \theta d\theta \\
            &= \Gamma(x + y) B(x, y)
    \end{alignat}
    より定理の式が成り立つ。
\end{proof}


\begin{proposition}
    \begin{enumerate}
        \item $\Gamma(x)$は$x > 0$上で{\smooth}級であり
            \begin{equation}
                \Gamma^{(n)}(x) = \int_0^\infty e^{-t} t^{x-1} (\log t)^n dt
                \label{11:eq:2}
            \end{equation}
            である。
        \item $\log \Gamma(x)$は$x > 0$上の凸関数である。
    \end{enumerate}
\end{proposition}

(2)はすこし唐突な印象もありますが、
実はこのあと出てくる Bohr-Mollerup の定理において重要な役割を果たします。

\begin{proof}
    (1)
    $\forall n \in \N$に対し広義積分
    \begin{equation}
        \int_0^\infty e^{-t} t^{x-1} (\log t)^n dt
        \label{11:eq:1}
    \end{equation}
    が$x > 0$上広義一様収束することさえ示せば
    積分記号下の微分ができるので、
    あとは数学的帰納法により (1) が示せる
    (数学的帰納法の部分は簡単なのでここでは省略する)。
    そこでまず$0 < \forall x_0 < \forall x_1 < \infty$を固定し、
    $I \coloneqq [x_0, x_1]$上での一様収束性を示そう。
    表記の簡略化のために
    \begin{equation}
        f_n(x, t) \coloneqq e^{-t} t^{x-1} (\log t)^n
    \end{equation}
    とおくと、$t \to +0$で
    \begin{equation}
        \begin{cases}
            e^{-t} &= O(1) \\
            t^{x-1} &= O(t^{x_0 - 1}) \\
            \log t &= O(t^{-\alpha}) \qquad (\text{ただし$\alpha$は$0 < \alpha < x_0/n$なる適当な定数})
        \end{cases}
    \end{equation}
    なので
    \begin{equation}
        f_n(x, t) = O(t^{x_0 - n\alpha - 1})
    \end{equation}
    である。$x_0 - n\alpha - 1 > -1$ゆえに
    $\int_0^1 t^{\alpha - 1} dt$は収束するから、
    Weierstrass の定理により広義積分
    \begin{equation}
        \int_0^1 f_n(x, t) dt
    \end{equation}
    は$I$上一様収束する。
    一方、$t \to \infty$で
    \begin{equation}
        \begin{cases}
            e^{-t} &= O(t^{- x_1 - n - 1}) \\
            t^{x-1} &= O(t^{x_1 - 1}) \\
            \log t &= O(t)
        \end{cases}
    \end{equation}
    なので
    \begin{equation}
        f_n(x, t) = O(t^{-2})
    \end{equation}
    である。
    $\int_1^\infty t^{-2} dt$は収束するから、
    Weierstrass の定理により広義積分
    \begin{equation}
        \int_1^\infty f_n(x, t) dt
    \end{equation}
    は$I$上一様収束する。
    以上より、広義積分(\ref{11:eq:1})は$x > 0$上広義一様収束する。

    (2)
    上で示した(\ref{11:eq:1})より、$\forall u \in \R$に対し
    \begin{equation}
        \begin{split}
            \Gamma(x) u^2 + 2 \Gamma'(x) u + \Gamma''(x)
                &= \int_0^\infty e^{-t} t^{x-1} (u + \log t)^2 dt \\
                &\ge 0
        \end{split}
    \end{equation}
    である。よって判別式は
    \begin{equation}
        D/4 = \Gamma'(x) - \Gamma(x) \Gamma''(x) \le 0
    \end{equation}
    をみたす。
    したがって
    \begin{equation}
        \begin{split}
            (\log \Gamma(x))''
                &= \left(\frac{\Gamma'(x)}{\Gamma(x)}\right)' \\
                &= \frac{\Gamma(x) \Gamma''(x) - \Gamma'(x) }{\Gamma(x)^2} \\
                &\ge 0
        \end{split}
    \end{equation}
    すなわち$\Gamma(x)$は$x > 0$上の凸関数である。
\end{proof}




% ------------------------------------------------------------
%
% ------------------------------------------------------------
\section{Bohr-Mollerup の定理}

    \begin{theorem}[Bohr-Mollerup の定理、$\Gamma$関数の特徴付け]
        $f \colon (0, \infty) \to \R$が
        \begin{enumerate}
            \item $f(x + 1) = x f(x)$
            \item $f(x) > 0$かつ$\log f(x)$は凸関数
            \item $f(1) = 1$
        \end{enumerate}
        をみたすとする。このとき、任意の$x > 0$に対し
        \begin{equation}
            f(x) = \Gamma(x) = \lim_{n \to \infty} \frac{n! n^x}{x(x + 1) \cdots (x + n)}
            \quad (\text{ガウスの公式})
            \label{eq:12:1}
        \end{equation}
        \label{11:thm:2}
    \end{theorem}

$f(x) = \Gamma(x)$の証明の際は、
$\Gamma$の積分表式を持ち出すのではなく、
$f$も$\Gamma$も式(\ref{eq:12:1})の極限式で書けるから一致するという論法で示します。

\begin{proof}
    条件(1)から、$\forall n \ge 1$に対し
    \begin{equation}
        f(x + n) = (x + n - 1) \cdots (x + 1) f(x)
        \label{11:eq:5}
    \end{equation}
    である。さらに$x = 1$として条件(3)を用いれば
    \begin{equation}
        f(n + 1) = n!
        \label{11:eq:6}
    \end{equation}
    が成り立つ。また、条件(2)より$g(x) \coloneqq \log f(x)$は凸関数であるから、
    $0 < {}^\forall a < {}^\forall t < {}^\forall b$に対し
    \begin{equation}
        \frac{g(t) - g(a)}{t - a}
            < \frac{g(b) - g(a)}{b - a}
            < \frac{g(b) - g(t)}{b - t}
        \label{11:eq:3}
    \end{equation}
    が成り立つ。

    \underline{Step 1:}
    さて、ひとまず$x \in (0, 1]$を固定して
    ガウスの公式(\ref{eq:12:1})を示していこう。
    ここで、任意の自然数$n \ge 2$に対し
    \begin{equation}
        \begin{split}
            \log (n - 1)
                &= \log f(n) - \log f(n - 1) \\
                &\le \frac{\log f(n + 1) - \log f(n)}{x} \\
                &\le \log f(n + 1) - \log f(n) \\
                &= \log n
        \end{split}
        \label{11:eq:4}
    \end{equation}
    が成り立つ。
    ただし、途中の不等式は$(a, t, b) = (n - 1, n, n + x),\, (n, n + x, n + 1)$に対し
    不等式(\ref{11:eq:3})を適用したものである。
    不等式(\ref{11:eq:4})の各辺に$x$を掛け、指数関数の値をとり、さらに$f(x)$を掛ければ
    \begin{equation}
        (n - 1)^x f(n) \le f(n + x) \le n^x f(n)
    \end{equation}
    を得る。これと式(\ref{11:eq:5})から
    \begin{equation}
        \frac{(n - 1)^x f(n)}{x (x + 1) \cdots (x + n - 1)}
            \le f(x)
            \le \frac{n^x f(n)}{x (x + 1) \cdots (x + n - 1)}
    \end{equation}
    であり、左の不等式だけ$n$を$n + 1$に置きなおして式(\ref{11:eq:6})を用いると
    \begin{equation}
        \underbrace{\frac{n! n^x}{x (x + 1) \cdots (x + n)}}_{\text{$a_n(x)$とおく}}
            \le f(x)
            \le \underbrace{\frac{n! n^x}{x (x + 1) \cdots (x + n)}}_{a_n(x)} \frac{x + n}{n}
    \end{equation}
    を得る。したがって、$n \to \infty$の極限を考えれば
    \begin{equation}
        f(x) = \lim_{n \to \infty} a_n(x)
    \end{equation}
    がいえる。
    $x \in (0, 1]$は任意であったから、
    $x \in (0, 1]$においてガウスの公式(\ref{eq:12:1})の成立がいえた。

    \underline{Step 2:}
    $x > 1$の場合は、
    $x = y + m\, (0 < y \le 1,\, m \in \N)$とおけば、
    $\forall n > m$に対し
    \begin{equation}
        \begin{split}
            \frac{n! n^x}{x \cdots (x + n)}
                &= \frac{n! n^y n^m}{(y + m) \cdots (y + m + n)} \\
                &= \frac{n! n^y}{y \cdots (y + n)}
                    \frac{n^m y \cdots (y + n)}{(y + m) \cdots (y + m + n)} \\
                &= \frac{n! n^y}{y \cdots (y + n)}
                    \frac{n^m y \cdots (y + m - 1) \cancel{(y + m) \cdots (y + n)}}
                        {\cancel{(y + m) \cdots (y + n)} (y + n + 1) \cdots (y + n + m)} \\
                &= \frac{n! n^y}{y \cdots (y + n)}
                    \underbrace{\frac{n^m}{(y + n + 1) \cdots (y + n + m)}}_{\to 1\; (n \to \infty)}
                    y \cdots (y + m - 1)
        \end{split}
    \end{equation}
    が成り立つから、
    \begin{equation}
        \begin{split}
            \lim_{n \to \infty} \frac{n! n^x}{x \cdots (x + n)}
                &= y \cdots (y + m - 1) f(y) \\
                &= f(y + m) \\
                &= f(x)
        \end{split}
    \end{equation}
    である。
    したがって
    $x > 1$においてもガウスの公式(\ref{eq:12:1})の成立がいえた。

    \underline{Step 3:}
    $\Gamma$も定理の仮定を満たすから、
    $\Gamma$も$x > 0$でガウスの公式(\ref{eq:12:1})をみたす。
    したがって$x > 0$で$f(x) = \Gamma(x)$である。
\end{proof}

    \begin{corollary}
        上の定理の条件(3)を除くと
        \begin{equation}
            f(x) = f(1) \Gamma(x) \quad (x > 0)
        \end{equation}
        が成り立つ。
        \label{11:cor:1}
    \end{corollary}

\begin{proof}
    $h(x) \coloneqq f(x) / f(1)$が条件(1),(2),(3)をみたすことから直ちに成り立つ。
\end{proof}

    \begin{proposition}
        $\forall x \in D \coloneqq \R - (- \N)$に対し
        \begin{equation}
            \lim_{n \to \infty} \frac{n! n^x}{x(x + 1) \cdots (x + n)} \in \R
        \end{equation}
        が存在する。
        \label{11:prop:1}
    \end{proposition}

\begin{proof}
    $x \in D$を任意にとる。$x > 0$の場合は成立がわかっているから$x < 0$の場合のみ考えればよい。
    すると、ある$m \in \N$が存在して$y = x + m > 0$なので、
    充分大きな任意の$n$に対し
    \begin{equation}
        \begin{split}
            \frac{n! n^x}{x \cdots (x + n)}
                &= \frac{n! n^y}{n^m (y - m) \cdots (y - m + n)} \\
                &= \frac{1}{(y - m) \cdots (y - 1)}
                    \underbrace{\frac{n! n^y}{y \cdots (y + n)}}_{\to \Gamma(y)\; (n \to \infty)}
                    \underbrace{\frac{(y - m + n + 1) \cdots (y + n)}{n^m}}_{\to 1\; (n \to \infty)}
        \end{split}
    \end{equation}
    が成り立つ。したがって
    \begin{equation}
        \lim_{n \to \infty} \frac{n! n^x}{x(x + 1) \cdots (x + n)} \in \R
    \end{equation}
    が存在する。
\end{proof}

\begin{definition}[$\Gamma$の極限式による定義]
    $\forall x \in D \coloneqq \R - (- \N)$に対し
    \begin{equation}
        \lim_{n \to \infty} \frac{n! n^x}{x(x + 1) \cdots (x + n)} \in \R
    \end{equation}
    と定義する。
\end{definition}

\begin{proposition}
    $\forall x \in D$に対し次が成り立つ。
    \begin{enumerate}
        \item \begin{equation}
            \Gamma(x) \neq 0
        \end{equation}
        \item \begin{equation}
            \mathrm{sign}\, \Gamma(x) = \begin{cases}
                1 \quad &(x > 0) \\
                (-1)^m \quad &(-m < x < -m+1,\, m \in \N)
            \end{cases}
        \end{equation}
        \item \begin{equation}
            \lim_{x \to -n} (x + n) \Gamma(x) = \frac{(-1)^n}{n!} \quad (n \in \N)
        \end{equation}
    \end{enumerate}
\end{proposition}

(3)は複素数に拡張された$\Gamma(x)$の点$-n$における留数を表しています。

\begin{proof}
    (1), (2) は\cref{11:prop:1}から明らか。

    (3)
    \begin{equation}
        \begin{split}
            (x + n) \Gamma(x)
                &= \frac{(x + n) \cdots x \Gamma(x)}{(x + n - 1) \cdots x} \\
                &= \frac{\Gamma(x + n - 1)}{(x + n - 1) \cdots x} \\
                &\to \frac{\Gamma(1)}{(-1)(-2) \cdots (-n)} \quad (x \to -n) \\
                &= \frac{(-1)^n}{n!}
        \end{split}
    \end{equation}
\end{proof}


% ------------------------------------------------------------
%
% ------------------------------------------------------------
\section{Stirling の公式}

ここでは$x \to \infty$における$\Gamma$の挙動を漸近的に評価する方法を考えていきます。
まず出発地点として$n!$の値を$\int_1^n \log x\, dx$で評価してみましょう。
$\int_1^n \log x\, dx$の値を "短冊" で近似することを考えると
\begin{equation}
    \int_1^n \log x\, dx
        = \log 2 + \cdots \log (n - 1) + \frac{1}{2} \log n + \delta_n
\end{equation}
が成り立ちます(ただし$\delta_n$は誤差)。
すると、簡単な計算により
\begin{equation}
    \begin{split}
        \log (n-1)! &= \left(n - \frac{1}{2}\right) \log n - n + 1 - \delta_n \\
        \therefore \quad \Gamma(n) &= n^{n - \frac{1}{2}} e^{-n} e^{1-\delta_n}
    \end{split}
\end{equation}
と表せることがわかります。
そこで、$x > 0$に対しても何らかの関数$\mu(x)$によって
\begin{equation}
    f(x) \coloneqq x^{x - \frac{1}{2}} e^{-x} e^{\mu(x)}
    \label{11:eq:7}
\end{equation}
を$\Gamma(x)$の定数倍に一致させられないだろうか?
というのが Stirling の公式の基本的なアイディアです。
ここで Bohr-Mollerup の定理(\cref{11:thm:2})によれば、$f$が$x > 0$で条件
\begin{enumerate}
    \item $f(x + 1) = x f(x)$
    \item $f(x) > 0$かつ$\log f(x)$は凸関数
\end{enumerate}
をみたしてくれれば目標達成です。以下、このことを確認していきます。

\begin{lemma}
    式(\ref{11:eq:7})で定義された関数$f$が$x > 0$で条件(1)をみたすには、
    $\mu(x)$が
    \begin{equation}
        \mu(x) - \mu(x + 1)
            = \underbrace{
                \left(x + \frac{1}{2}\right) \log \left(1 + \frac{1}{x}\right) - 1
            }_{\text{$g(x)$とおく}}
    \end{equation}
    をみたすことが必要十分である。
    \label{11:lem:1}
\end{lemma}

\begin{proof}
    $x > 0$とする。
    $f$の定義式(\ref{11:eq:7})によれば
    \begin{equation}
        \frac{f(x + 1)}{f(x)} = x \left(1 + \frac{1}{2}\right)^{x + \frac{1}{2}} e^{\mu(x + 1) - \mu(x) - 1}
    \end{equation}
    なので、$f$が条件(1)、すなわち
    \begin{equation}
        \frac{f(x + 1)}{f(x)} = x
    \end{equation}
    をみたすには、$\mu$が
    \begin{equation}
        \left(1 + \frac{1}{2}\right) \log \left(1 + \frac{1}{x}\right) - 1 = \mu(x) - \mu(x + 1)
    \end{equation}
    をみたすことが必要十分である。
\end{proof}

    \begin{lemma}
        級数$\sum_{n=0}^\infty g(x + n)$は$x > 0$上で各点収束し、
        $x \to \infty$で$0$に収束する。
    \end{lemma}

\begin{proof}
    関数$\frac{1}{2} \log \frac{1+y}{1-y}\; (= \artanh y)$は$|y| < 1$で
    \begin{equation}
        \frac{1}{2} \log \frac{1+y}{1-y}
            = y + \frac{y^3}{3} + \frac{y^5}{5} + \cdots
    \end{equation}
    と収束級数に展開できる。
    右辺に$y = \frac{1}{2x + 1}$を代入することで
    \begin{equation}
        \begin{split}
            g(x)
                &= \left(x + \frac{1}{2}\right) \log \left(1 + \frac{1}{x}\right) - 1 \\
                &= (2x + 1) \frac{1}{2} \log \left(1 + \frac{1}{x}\right) - 1 \\
                &= (2x + 1) \sum_{n=0}^\infty \frac{1}{(2n + 1) (2x + 1)^{2n + 1}} - 1 \\
                &= \sum_{n=1}^\infty \frac{1}{(2n + 1) (2x + 1)^{2n}}
        \end{split}
    \end{equation}
    を得る。したがって
    \begin{equation}
        \begin{split}
            0 < g(x) &< \frac{1}{3} \sum_{n=1}^\infty \frac{1}{(2x + 1)^{2n}} \\
                &= \frac{1}{3} \left(\frac{1}{2x + 1}\right)^2
                    \frac{1}{1 - \left(\frac{1}{2x + 1}\right)^2} \\
                &= \frac{1}{12x(x+1)} \\
                &= \frac{1}{12x} - \frac{1}{12(x+1)}
        \end{split}
    \end{equation}
    である。よって$\forall N \in \N$に対し
    \begin{equation}
        \begin{split}
            0 < \sum_{n=0}^N g(x + n)
                &< \sum_{n=0}^N \left\{ \frac{1}{12(x + n)} - \frac{1}{12(x+n+1)} \right\} \\
                &\le \sum_{n=0}^\infty \left\{ \frac{1}{12(x + n)} - \frac{1}{12(x+n+1)} \right\} \\
                &= \frac{1}{12x}
        \end{split}
    \end{equation}
    である。したがって部分和$\sum_{n=0}^N g(x + n)$は上に有界な正数列なので
    $\sum_{n=0}^\infty g(x + n)$は収束し、
    上の不等式から$x \to \infty$で$0$に収束することも示せた。
\end{proof}

\begin{lemma}
    $\mu$を以下のように定めれば\cref{11:lem:1}の条件が達成される。
    \begin{equation}
        \begin{split}
            \mu(x) &= \sum_{k=0}^\infty g(x + k)
        \end{split}
    \end{equation}
    \label{11:lem:2}
\end{lemma}

\begin{proof}
    級数$\sum_{k=0}^\infty g(x + k)$が収束することから
    \begin{equation}
        \begin{split}
            \mu(x) - \mu(x + 1)
                &= \sum_{k=0}^\infty g(x + k) - \sum_{k=0}^\infty g(x + k + 1) \\
                &= \sum_{k=0}^\infty (g(x + k) - g(x + k + 1)) \\
                &= \lim_{N \to \infty} \sum_{k=0}^N (g(x + k) 0 g(x + k + 1)) \\
                &= \lim_{N \to \infty} (g(x) - g(x + N + 1)) \\
                &= g(x)
        \end{split}
    \end{equation}
    である。
\end{proof}

\begin{lemma}
    定義式(\ref{11:eq:7})と\cref{11:lem:2}の$\mu$によって定まる$f$は
    条件(2)をみたす。
    \label{11:lem:3}
\end{lemma}

\begin{proof}
    $f$の定義式から明らかに$f(x) > 0$である。
    また
    \begin{equation}
        \log f(x) = \left(x - \frac{1}{2}\right) \log x - x + \mu(x)
    \end{equation}
    であり、右辺の$\mu(x)$を除く項は凸であるから、
    $\log f$が凸であることを示すには$\mu$が凸であることをいえばよい。
    そのためには$g$が凸であることをいえば充分だが、
    \begin{equation}
        g''(x) = \frac{1}{2x^2 (x+1)^2} > 0
    \end{equation}
    なので$g$も$x > 0$で凸である。
    したがって$f$は条件(2)をみたす。
\end{proof}

    \begin{lemma}[Wallis の公式]
        \begin{equation}
            \sqrt{\pi} = \lim_{n \to \infty} \frac{(2n)!!}{(2n - 1)!! \sqrt{n}}
        \end{equation}
    \end{lemma}

\begin{proof}
    長いので省略\footnote{
        参考文献\cite[第IV章 定理15.6系]{杉浦80}を参照。
    }
\end{proof}

    \begin{theorem}[Stirling の公式]
        $x > 0$に対し
        \begin{equation}
            \Gamma(x) = \sqrt{2\pi} x^{x - \frac{1}{2}} e^{-x} e^{\mu(x)}
        \end{equation}
        が成り立つ。
        ただし$\mu$は\cref{11:lem:2}で定めたものである。
    \end{theorem}

\begin{proof}
    \cref{11:lem:2}と\cref{11:lem:3}より、
    定義式(\ref{11:eq:7})と$\mu$によって定まる関数$f$は
    Bohr-Mollerup の定理(\cref{11:thm:2})の条件 (1), (2) をみたす。
    したがって\cref{11:cor:1}より
    \begin{equation}
        \Gamma(x) = a f(x) \quad (x > 0)
    \end{equation}
    となる定数$a > 0$が存在する。$a$は Wallis の公式によって求めることができ、
    $n! = \Gamma(n+1) = n\Gamma(n) = anf(n)$に注意すれば
    \begin{equation}
        \begin{split}
            \sqrt{\pi}
                &= \lim_{n \to \infty} \frac{(2n)!!}{(2n - 1)!! \sqrt{n}} \\
                &= \lim_{n \to \infty} \frac{2^{2n} (n!)^2}{(2n)! \sqrt{n}} \\
                &= \lim_{n \to \infty} \frac{2^{2n} (an f(n))^2}{a \cdot 2n f(2n) \sqrt{n}} \\
                &= \lim_{n \to \infty} \frac{2^{2n} a^2 n^2 \left(n^{n-1/2} e^{-n} e^{\mu(n)}\right)^2}
                    {2 an (2n)^{2n-1/2} e^{-2n} e^{\mu(2n)} \sqrt{n}} \\
                &= \lim_{n \to \infty} \frac{a}{\sqrt{2}} \frac{e^{2\mu(n)}}{e^{\mu(2n)}} \\
                &= \frac{a}{\sqrt{2}}
        \end{split}
    \end{equation}
    よって$a = \sqrt{2\pi}$である。
    以上より定理の主張が示せた。
\end{proof}



\begin{problem}
    $\Gamma(1/2),\, \Gamma(n + 1/2)$の値を求めよ。

    解答:
    \begin{equation}
        \Gamma(1/2) = \sqrt{\pi},\quad \Gamma(n + 1/2) = \frac{(2n - 1)!!}{2^n} \sqrt{\pi}
    \end{equation}
\end{problem}

\begin{problem}
    $p, q > -1$に対し
    \begin{equation}
        \int_0^{\pi/2} \cos^p \theta \sin^q \theta d\theta
            = \frac{1}{2} B\!\left(\frac{p+1}{2}, \frac{q+1}{2}\right)
    \end{equation}
    を示せ。
\end{problem}

\begin{problem}
    $n$次元球の体積と表面積を$\Gamma$を用いて表わせ。

    解答:
    \begin{equation}
        |B_n(R)| = \frac{\sqrt{\pi} R^n}{\Gamma(n/2 + 1)},\quad
        |\del B_n(R)| = \frac{2 \sqrt{\pi} R^{n-1}}{\Gamma(n/2)}
    \end{equation}
\end{problem}

\begin{problem}
    $0 < m+1 < n$とする。
    \begin{equation}
        I = \int_0^\infty \frac{x^m}{1 + x^n} dx
    \end{equation}
    を Gamma 関数、Beta 関数を用いて表わせ。

    解答:
    \begin{equation}
        I = \frac{1}{n} B\left( \frac{m+1}{n}, 1 - \frac{m+1}{n} \right)
            = \frac{1}{n} \Gamma\left(\frac{m+1}{n}\right) \Gamma\left(1 - \frac{m+1}{n}\right)
    \end{equation}
\end{problem}

\begin{problem}
    \begin{equation}
        \int_0^\infty \frac{t^{y-1}}{1+t^x} dt \quad (x > y > 0)
    \end{equation}
    を$\Gamma$関数で表わせ。

    解答:
    \begin{equation}
        \frac{1}{x} \Gamma\left(1 - \frac{y}{x}\right) \Gamma\left(\frac{y}{x}\right)
    \end{equation}
\end{problem}

\begin{problem}
    \begin{equation}
        \int_0^{\pi/2} \frac{\cos^{2x-1} \theta \sin^{2y - 1} \theta}
            {(a \cos^2\theta + b \sin^2\theta)^{x+y}} d\theta \quad (a, b, x, y > 0)
    \end{equation}
    を$\Gamma$関数で表わせ。

    解答:
    \begin{equation}
        \frac{1}{2 a^x b^y} \frac{\Gamma(x) \Gamma(y)}{\Gamma(x + y)}
    \end{equation}
\end{problem}

\begin{problem}
    \begin{equation}
        \left(\int_0^{\pi/2} \sqrt{\sin \theta} d\theta\right)
        \left(\int_0^{\pi/2} \frac{d\theta}{\sqrt{\sin\theta}}\right)
        = \pi
    \end{equation}
    を示せ。
\end{problem}

\begin{problem}[積分]
    \,
    \begin{itemize}
        \item \cite[第III章 例題8.1]{杉浦+89}
        \item \cite[第III章 問8.1 (1)-(6)]{杉浦+89}
        \item \cite[第III章 問8.2 (1),(2)]{杉浦+89}
        \item \cite[第III章 問8.3 (1)]{杉浦+89}
    \end{itemize}
    を読者の演習問題とする。
\end{problem}

\begin{problem}[Stirling の公式]
    \,
    \begin{itemize}
        \item \cite[第III章 例題8.5]{杉浦+89}
    \end{itemize}
    を読者の演習問題とする。
\end{problem}

\end{document}

% ============================================================
%
% ============================================================
\newpage
\phantomsection
\addcontentsline{toc}{part}{演習問題の解答}
\part*{演習問題の解答}

\includecollection{answers}

% ============================================================
%
% ============================================================
\newpage
\phantomsection
\addcontentsline{toc}{part}{参考文献}
\renewcommand{\bibname}{参考文献}
\markboth{\bibname}{}
\part*{参考文献}

群、環、加群、体についてひととおり知るには
\cite{松坂76}や\cite{Lan02}がよいと思う。
\cite{松坂76}では初等整数論についても触れられている。
\cite{AB95}には群の表現について簡潔にまとまっている。
可換環論の本は\cite{AM69}が有名である。
非可換環論の本は\cite{AF92}がある。
結合的代数については\cite{Pie82}がよいと思う。
\cite{Rot15}にはテンソル代数などの話題も載っている。

{
    \renewcommand{\bibsection}{}
    \bibliographystyle{amsalpha}
    \bibliography{../mybibliography}
}

% ============================================================
%
% ============================================================
\newpage
\phantomsection
\addcontentsline{toc}{part}{記号一覧}
\part*{記号一覧}

{
    \renewcommand{\glossarysection}[2][]{}
    \printglossary[title={記号一覧}]
}

% ============================================================
%
% ============================================================
\newpage
\phantomsection
\addcontentsline{toc}{part}{索引}
\printindex

\end{document}