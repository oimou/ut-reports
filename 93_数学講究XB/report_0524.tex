\documentclass[report, notitlepage]{jlreq}
\usepackage{global}
\usepackage{./sub/local}
\def\assetspath{./}
%\makeindex
\makeglossaries

\title{
    数学講究XB レポート \\[1em]
    シンプレクティック多様体において \\[0.5ex]
    $C^\infty(M, \R)$が
    Poisson 括弧により \\[0.5ex]
    Lie 代数となることの証明
}
\author{05-220542 \\ Keiji Yahata}
\date{}

\lhead{数学講究XB レポート}
\rhead{05-220542 Keiji Yahata}

\begin{document}

\maketitle

% ============================================================
%
% ============================================================
\newpage
\setcounter{section}{1}

本レポートでは、
シンプレクティック多様体$(M, \omega)$において
$C^\infty(M, \R)$が
Poisson 括弧$\{ \cdot, \cdot \}$により
Lie 代数となることを証明する。
まずいくつか用語の定義を整理しておく。

\begin{definition}[シンプレクティック多様体]
    $M$を有限次元\smooth 多様体、
    $\omega$を$M$上の2-形式とする。
    組$(M, \omega)$が
    \termsilent{シンプレクティック多様体}であるとは、
    $\omega^n \neq 0$かつ$d\omega = 0$が成り立つことをいう。
    ただし$\omega^n$は
    $\omega \wedge \cdots \wedge \omega$のことである。
\end{definition}

\begin{propdef}[\smooth 関数の Hamilton ベクトル場]
    $(M, \omega)$をシンプレクティック多様体とする。
    このとき、
    各$f \in C^\infty(M, \R)$に対し、
    あるベクトル場$H_f \in \mathfrak{X}(M)$がただひとつ存在して、
    任意のベクトル場$Y \in \mathfrak{X}(M)$に対して
    \begin{equation}
        \omega(H_f, Y) = df(Y)
    \end{equation}
    をみたす。
    この$H_f$を\termsilent{$f$の Hamilton ベクトル場}という。
\end{propdef}

\begin{proof}
    \TODO{}
\end{proof}

\begin{definition}[Poisson 括弧]
    $(M, \omega)$をシンプレクティック多様体とする。
    各$f, g \in C^\infty(M, \R)$に対し、
    $f, g$の\termsilent{Poisson 括弧}$\{ f, g \} \in C^\infty(M, \R)$を
    \begin{equation}
        \{ f, g \} \coloneqq \omega(H_f, H_g)
    \end{equation}
    と定義する。
\end{definition}

目標の定理を示す。

\begin{theorem}
    $(M, \omega)$をシンプレクティック多様体とする。
    このとき、
    $C^\infty(M, \R)$は Poisson 括弧$\{ \cdot, \cdot \}$を
    括弧積として Lie 代数となる。
\end{theorem}

\begin{proof}
    示すべきことは、Poisson 括弧が次をみたすことである:
    \begin{description}
        \item[($\R$-双線型性)]
            任意の$f, g, h \in C^\infty(M, \R)$と
            $a, b \in \R$に対して、
            $\{ af + bg, h \} = a \{ f, h \} + b \{ g, h \}$
            および
            $\{ h, af + bg \} = a \{ h, f \} + b \{ h, g \}$
            が成り立つ。
        \item[(反対称性)]
            任意の$f, g \in C^\infty(M, \R)$に対して、
            $\{ f, g \} = - \{ g, f \}$が成り立つ。
        \item[(Jacobi 恒等式)]
            任意の$f, g, h \in C^\infty(M, \R)$に対して、
            $\{ \{ f, g \}, h \} + \{ \{ g, h \}, f \} + \{ \{ h, f \}, g \} = 0$
            が成り立つ。
    \end{description}
    \TODO{}
\end{proof}

\end{document}