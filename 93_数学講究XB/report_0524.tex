\documentclass[report, notitlepage]{jlreq}
\usepackage{global}
\usepackage{./sub/local}
\def\assetspath{./}
%\makeindex
\makeglossaries

\title{
    数学講究XB レポート \\[1em]
    シンプレクティック多様体において \\[0.5ex]
    $C^\infty(M, \R)$が
    Poisson 括弧により \\[0.5ex]
    Lie 代数となることの証明
}
\author{05-220542 \\ Keiji Yahata}
\date{}

\lhead{数学講究XB レポート}
\rhead{05-220542 Keiji Yahata}

\begin{document}

\maketitle

% ============================================================
%
% ============================================================
\newpage
\setcounter{section}{1}

本レポートでは、
シンプレクティック多様体$(M, \omega)$において
$C^\infty(M, \R)$が
Poisson 括弧$\{ {-}, {-} \}$により
Lie 代数となることを証明する。
まずいくつか用語の定義を整理しておく。

\begin{definition}[シンプレクティック多様体]
    $M$を有限次元実\smooth 多様体、
    $\omega$を$M$上の2-形式とする。
    組$(M, \omega)$が
    \termsilent{シンプレクティック多様体}であるとは、
    $\omega^n \neq 0$かつ$d\omega = 0$が成り立つことをいう。
    ただし$\omega^n$とは
    $\underbrace{\omega \wedge \cdots \wedge \omega}_{\text{$n$個}}$
    のことである。
\end{definition}

以降、$(M, \omega)$をシンプレクティック多様体とする。

\begin{propdef}[\smooth 関数の Hamilton ベクトル場]
    各$f \in C^\infty(M, \R)$に対し、
    あるベクトル場$H_f \in \mathfrak{X}(M)$がただひとつ存在して、
    任意のベクトル場$Y \in \mathfrak{X}(M)$に対して
    \begin{equation}
        \omega(H_f, Y) = df(Y)
    \end{equation}
    をみたす。
    この$H_f$を\termsilent{$f$の Hamilton ベクトル場}という。
\end{propdef}

\begin{proof}
    $\omega^n \neq 0$ゆえに$\omega$は非退化であるから、
    写像
    $\Gamma(TM) \to \Gamma(T^*M), \; X \mapsto \omega(X, {-})$は
    $\smooth(M)$-加群の同型となる。
    このことから直ちに
    $H_f$の存在と一意性が従う。
\end{proof}

\begin{definition}[Poisson 括弧]
    各$f, g \in C^\infty(M, \R)$に対し、
    $f, g$の\termsilent{Poisson 括弧}$\{ f, g \} \in C^\infty(M, \R)$を
    \begin{equation}
        \{ f, g \} \coloneqq \omega(H_f, H_g)
    \end{equation}
    と定義する。
\end{definition}

本レポートの目標の定理は次である:

\begin{theorem}
    \label[theorem]{thm:main}
    $C^\infty(M, \R)$は Poisson 括弧$\{ {-}, {-} \}$を
    括弧積として Lie 代数となる。
\end{theorem}

この定理の証明には次の補題を用いる:

\begin{lemma}[Poisson 括弧と Lie 括弧の関係]
    \label[lemma]{lemma:poisson-lie}
    任意の$f, g \in \smooth(M, \R)$に対し
    \begin{equation}
        H_{\{f, g\}} = - [H_f, H_g]
    \end{equation}
    が成り立つ。
\end{lemma}

\begin{proof}
    すべての$Y \in \Gamma(TM)$に対し
    $\omega(H_{\{f, g\}}, Y) + \omega([H_f, H_g], Y) = 0$
    が成り立つことを示せばよい。
    まず
    Hamilton ベクトル場および Poisson 括弧の定義より
    \begin{alignat}{1}
        \omega(H_{\{f, g\}}, Y)
            &=
                Y \{ f, g \}
                \\
            &=
                Y(\omega(H_f, H_g))
                \\
            &=
                Y H_g f
    \end{alignat}
    である。
    一方、
    $i_{H_g} \omega = dg$および$d\omega = 0$より
    Cartan の公式
    $L_{H_g} \omega = d(i_{H_g} \omega) + i_{H_g} d\omega$
    \; ($i$は内部積)
    の右辺は$0$となるから、
    \begin{alignat}{1}
        0
            &=
                (L_{H_g} \omega) (H_f, Y)
                \\
            &=
                H_g(\omega(H_f, Y)) - \omega([H_g, H_f], Y) - \omega(H_f, [H_g, Y])
                \\
            &=
                H_g Y f
                - \omega([H_g, H_f], Y)
                - [H_g, Y] f
                \\
            &=
                - \omega([H_g, H_f], Y)
                + Y H_g f
                \\
        \therefore \quad
        \omega([H_f, H_g], Y)
            &=
                - \omega([H_g, H_f], Y)
            =
                - Y H_g f
    \end{alignat}
    を得る。
    したがって
    \begin{equation}
        \omega(H_{\{f, g\}}, Y) + \omega([H_f, H_g], Y)
            = 0
    \end{equation}
    が成り立つ。
    よって$H_{\{f, g\}} = - [H_f, H_g]$が示された。
\end{proof}

\begin{proof}[\cref{thm:main}の証明]
    示すべきことは、Poisson 括弧が次をみたすことである:
    \begin{description}
        \item[($\R$-双線型性)]
            任意の$f, g, h \in C^\infty(M, \R)$と
            $a, b \in \R$に対して、
            $\{ af + bg, h \} = a \{ f, h \} + b \{ g, h \}$
            および
            $\{ h, af + bg \} = a \{ h, f \} + b \{ h, g \}$
            が成り立つ。
        \item[(反対称性)]
            任意の$f, g \in C^\infty(M, \R)$に対して、
            $\{ f, g \} = - \{ g, f \}$が成り立つ。
        \item[(Jacobi 恒等式)]
            任意の$f, g, h \in C^\infty(M, \R)$に対して、
            $\{ \{ f, g \}, h \} + \{ \{ g, h \}, f \} + \{ \{ h, f \}, g \} = 0$
            が成り立つ。
    \end{description}

    \uline{Step 1: $\R$-双線型性} \quad
    $\R$-双線型性を示す。
    まず Hamilton ベクトル場の定義より
    \begin{alignat}{1}
        \omega(H_{af + bg}, {-})
            &=
                d(af + bg)
                \\
            &=
                a df + b dg
                \\
            &=
                a \omega(H_f, {-}) + b \omega(H_g, {-})
                \\
            &=
                \omega(aH_f + bH_g, {-})
                \\
        \therefore \quad
        H_{af + bg}
            &=
                aH_f + bH_g
    \end{alignat}
    が成り立つことに注意すれば、
    第1引数に関する線型性は
    \begin{alignat}{1}
        \{ af + bg, h \}
            &=
                \omega(H_{af + bg}, H_h)
                \\
            &=
                \omega(aH_f + bH_g, H_h)
                \\
            &=
                a\omega(H_f, H_h) + b\omega(H_g, H_h)
                \\
            &=
                a\{ f, h \} + b\{ g, h \}
    \end{alignat}
    より従う。第2引数に関しても同様である。
    よって$\R$-双線型性が示された。

    \uline{Step 2: 反対称性} \quad
    $\omega$の交代性より明らか。

    \uline{Step 3: Jacobi 恒等式} \quad
    任意の$f, g, h \in C^\infty(M, \R)$に対し
    \begin{alignat}{1}
        \{ \{ f, g \}, h \}
            &=
                \omega(H_{\{ f, g \}}, H_h)
                \\
            &=
                - \omega([H_f, H_g], H_h)
                \quad
                (\text{\cref{lemma:poisson-lie}})
                \\
            &=
                \omega(H_h, [H_f, H_g])
                \\
            &=
                [H_f, H_g] h
                \\
            &=
                H_f H_g h - H_g H_f h
                \\
            &=
                H_f \{ h, g \} - H_g \{ h, f \}
                \\
            &=
                \{ \{ h, g \}, f \} - \{ \{ h, f \}, g \}
                \\
            &=
                - \{ \{ g, h \}, f \} - \{ \{ h, f \}, g \}
                \quad
                (\text{$\R$-双線型性および反対称性})
    \end{alignat}
    が成り立つから、
    Jacobi 恒等式が示された。
\end{proof}

\end{document}