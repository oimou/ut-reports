\documentclass[report, notitlepage]{jlreq}
\usepackage{global}
\usepackage{./sub/local}
\def\assetspath{./}
%\makeindex
\makeglossaries

\title{数学講究XB レポート}
\author{Keiji Yahata}
\date{}

\lhead{数学講究XB レポート}
\rhead{Keiji Yahata}

\begin{document}

\maketitle

% ============================================================
%
% ============================================================
\newpage
\section{導入}

本レポートでは、
Ising モデルを情報幾何学的に調べる。

1次元だと計量は必ず平坦になるので、
Levi-Civita 接続から定まる
曲率や捩率を使って相転移点を見つけることはできない。

% ============================================================
%
% ============================================================
\newpage
\phantomsection
\addcontentsline{toc}{part}{参考文献}
\renewcommand{\bibname}{参考文献}
\markboth{\bibname}{}
\part*{参考文献}

{
    \renewcommand{\bibsection}{}
    \bibliographystyle{amsalpha}
    \bibliography{../mybibliography}
}

% ============================================================
%
% ============================================================
\newpage
\phantomsection
\addcontentsline{toc}{part}{記号一覧; Nomenclature}
\printglossary[title={記号一覧; Nomenclature}]

% ============================================================
%
% ============================================================
\newpage
\phantomsection
\addcontentsline{toc}{part}{索引}
\printindex

\end{document}