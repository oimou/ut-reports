\documentclass[report, notitlepage]{jlreq}
\usepackage{global}
\usepackage{./sub/local}
\def\assetspath{./}
%\makeindex
\makeglossaries

\title{数学講究XB レポート \\[1ex] Gromov-Hausdorff 距離が距離であることの証明}
\author{05-220542 \\ Keiji Yahata}
\date{}

\lhead{数学講究XB レポート}
\rhead{05-220542 Keiji Yahata}

\newcommand{\CMet}{\mathrm{CMet}}
\newcommand{\GH}{\mathrm{GH}}

\begin{document}

\maketitle

% ============================================================
%
% ============================================================
\newpage
\setcounter{section}{1}

本レポートでは Gromov-Hausdorff 距離が距離であることの証明を与える。
まず Gromov-Hausdorff 距離の定義を行う。

\begin{propdef}[Gromov-Hausdorff 距離]
    コンパクト距離空間の同型類全体の集合を$\CMet$と書く。
    ただし「同型」とは等長同型$\cong$の意味である。
    このとき、
    $\CMet$上の2変数関数
    $d_\GH \colon \CMet \times \CMet \to [0, +\infty]$であって
    次をみたすものがただひとつ存在する:
    \begin{itemize}
        \item 各$\calX, \calY \in \CMet$に対し、
            代表元$X \in \calX, \; Y \in \calY$をひとつずつ選ぶと
            \begin{equation}
                \label{eq:def}
                \begin{alignedat}{2}
                    d_\GH(\calX, \calY)
                        =
                            \inf \{
                                \eps \in \R_{> 0}
                                \mid
                                \exists d \colon \text{metric on $X \sqcup Y$}
                                \; \text{s.t.} \;
                                &d|_X = d_X, \; d|_Y = d_Y,
                                    \\
                                &\forall x \in X, \;
                                    \exists y \in Y, \;
                                    d(x, y) < \eps, \;
                                    \\
                                &\forall y \in Y, \;
                                    \exists x \in X, \;
                                    d(x, y) < \eps
                            \}
                \end{alignedat}
            \end{equation}
            が成り立つ。
            ただし$d_X, d_Y$はそれぞれ$X, Y$に定まっている距離である。
    \end{itemize}
    $d_\GH$を\termsilent{Gromov-Hausdorff 距離}という。
\end{propdef}

\begin{proof}
    用語および記法の準備として、
    正実数$\eps \in \R_{> 0}$と
    コンパクト距離空間$X, Y$に対し、
    $X \sqcup Y$上の距離$d$に関する条件
    \begin{equation}
        \begin{cases}
            d|_X = d_X,
                \quad
                d|_Y = d_Y
                \\
            \forall x \in X, \;
                \exists y \in Y, \;
                d(x, y) < \eps \;
                \\
            \forall y \in Y, \;
                \exists x \in X, \;
                d(x, y) < \eps
        \end{cases}
    \end{equation}
    を条件$P(\eps, X, Y)$と呼ぶことにし、
    集合$A_{X, Y} \subset \R_{> 0}$を
    \begin{equation}
        \label{eq:def-A}
        A_{X, Y}
            \coloneqq
                \{
                    \eps \in \R_{> 0}
                    \mid
                    \text{$X \sqcup Y$上の距離$d$で条件$P(\eps, X, Y)$をみたすものが存在する}
                \}
    \end{equation}
    と定める。
    すると式\cref{eq:def}の右辺は
    $\inf A_{X, Y}$と表せることに注意しておく。

    $d_\GH$を式\cref{eq:def}で定義するために、
    まず well-defined 性を示す。
    すなわち、
    $\calX, \calY \in \CMet$を任意とし、
    $\inf A_{X, Y}$の値は
    代表元の選び方によらないことを示す。
    そのためには、
    $\calX, \calY$の
    任意の代表元$X, X' \in \calX, \; Y, Y' \in \calY$に対し
    $A_{X, Y} = A_{X', Y'}$が成り立つことをいえば十分である。
    さらに$X, Y$と$X', Y'$の立場を入れ替えても同様の議論が成り立つから、
    $A_{X, Y} \subset A_{X', Y'}$を示せばよい。
    そこで$\eps \in A_{X, Y}$を任意とする。
    すると$A_{X, Y}$の定義より、
    $X \sqcup Y$上の距離$d$であって
    条件$P(\eps, X, Y)$をみたすものが存在する。
    目標である$\eps \in A_{X', Y'}$を示すためには、
    $X' \sqcup Y'$上の距離$d'$であって
    条件$P(\eps, X', Y')$をみたすものを構成すればよい。

    \uline{Step 1: 距離$d'$の構成} \quad
    等長同型写像$f \colon X \cong X', \; g \colon Y \cong Y'$をひとつずつ選び、
    写像$d' \colon (X' \sqcup Y') \times (X' \sqcup Y') \to [0, +\infty)$を
    $d'(s', t') \coloneqq d((f \sqcup g)^{-1}(s'), (f \sqcup g)^{-1}(t'))$で定める。
    すると、
    $d$が$X \sqcup Y$上の距離であることから
    $d'$は正値性、対称性、三角不等式をみたし、
    さらに$f \sqcup g$が$X \sqcup Y \to X' \sqcup Y'$の全単射であることから、
    $d'$は非退化性もみたす。
    したがって$d'$は$X' \sqcup Y'$上の距離となる。

    \uline{Step 2: 距離$d'$が条件$P(\eps, X', Y')$をみたすこと} \quad
    $d'|_{X'} = d_{X'}$であることは、
    各$s', t' \in X'$に対し
    \begin{alignat}{2}
        d'|_{X'}(s', t')
            &=
                d((f \sqcup g)^{-1}(s'), (f \sqcup g)^{-1}(t'))
                &&
                \\
            &=
                d(f^{-1}(s'), f^{-1}(t'))
                \qquad
                &&(\because s', t' \in X')
                \\
            &=
                d_X(f^{-1}(s'), f^{-1}(t'))
                \qquad
                &&(\because d|_X = d_X, \; f^{-1}(s'), f^{-1}(t') \in X)
                \\
            &=
                d_{X'}(s', t')
                \qquad
                &&(\because \text{$f$は等長同型写像})
    \end{alignat}
    となることより従う。
    $d'|_{Y'} = d_{Y'}$についても同様である。
    さらに
    「$\forall x' \in X', \; \exists y' \in Y', \; d'(x', y') < \eps$」
    について、
    各$x' \in X'$に対し、
    $f^{-1}(x') \in X$ゆえに
    ある$y \in Y$が存在して$d(f^{-1}(x'), y) < \eps$となるから、
    $y' \coloneqq g(y) \in Y'$とおけば
    \begin{alignat}{2}
        d'(x', y')
            &=
                d((f \sqcup g)^{-1}(x'), (f \sqcup g)^{-1}(y'))
                &&
                \\
            &=
                d(f^{-1}(x'), g^{-1}(y'))
                \qquad
                &&(\because x' \in X', \; y' \in Y')
                \\
            &=
                d(f^{-1}(x'), y)
                \\
            &<
                \eps
    \end{alignat}
    が成り立つ。
    「$\forall y' \in Y', \; \exists x' \in X', \; d'(x', y') < \eps$」
    についても同様である。
    
    以上で$\eps \in A_{X', Y'}$がいえた。
    したがって$A_{X, Y} \subset A_{X', Y'}$
    ひいては$A_{X, Y} = A_{X', Y'}$が示され、
    $\inf A_{X, Y}$の値は代表元の選び方によらないことが示された。
\end{proof}

Gromov-Hausdorff 距離が距離であることを示す。

\begin{theorem}[Gromov-Hausdorff 距離は距離]
    \label[theorem]{thm:GH_is_distance}
    Gromov-Hausdorff 距離$d_\GH$は$\CMet$上の距離である。
\end{theorem}

この定理の証明はいくつかの補題に分けて行う。
まず三角不等式を示す。

\begin{lemma}
    \label[lemma]{lemma:GH_triangle_inequality}
    $d_\GH$は三角不等式をみたす。
\end{lemma}

\begin{proof}
    $\calX, \calY, \calZ \in \calM$とし、
    $a \coloneqq d_\GH(\calX, \calZ), \;
        b \coloneqq d_\GH(\calX, \calY), \;
        c \coloneqq d_\GH(\calY, \calZ)$
    とおく。
    示すべき不等式は$a \le b + c$である。
    $b = \infty$または$c = \infty$の場合は明らかだから、
    $b, c < \infty$の場合を考える。

    $Y = \emptyset$の場合は
    $b, c < \infty$より
    $X = \emptyset, \; Z = \emptyset$となるから、
    $a = b = c = 0$となり
    証明は終わる。

    以降$Y \neq \emptyset$の場合を考える。
    すると$a \le b + c$を示すためには、
    任意の$s > b, \; t > c$に対し
    $a \le s + t$が成り立つことを示せばよく、
    そのためには
    $\calX, \calY, \calZ$の代表元
    $X, Y, Z$をひとつずつ選んで
    $s + t \in A_{X, Z}$を示せばよい
    ($A_{X, Z}$は\cref{eq:def-A}で定義したもの)。
    そこで、
    $X \sqcup Z$上の距離$d_{X \sqcup Z}$であって
    条件$P(s + t, X, Z)$をみたすものを構成することを考える。

    \uline{Step 1: $d_{X \sqcup Z}$の構成} \quad
    いま$s > b, \; t > c$ゆえに
    $s \in A_{X, Y}, \; t \in A_{Y, Z}$だから、
    $X \sqcup Y$上の距離$d_{X \sqcup Y}$であって
    条件$P(s, X, Y)$をみたすものと、
    $Y \sqcup Z$上の距離$d_{Y \sqcup Z}$であって
    条件$P(t, Y, Z)$をみたすものがそれぞれ存在する。
    これらを用いて
    写像$d_{X \sqcup Z} \colon (X \sqcup Z) \times (X \sqcup Z) \to [0, +\infty)$を
    \begin{alignat}{1}
        d_{X \sqcup Z}(x, x')
            &\coloneqq
                d_{X \sqcup Y}(x, x')
                \qquad
                (x, x' \in X)
                \\
        d_{X \sqcup Z}(z, z')
            &\coloneqq
                d_{Y \sqcup Z}(z, z')
                \qquad
                (z, z' \in Z)
                \\
        d_{X \sqcup Z}(x, z)
            &\coloneqq
                d_{X \sqcup Z}(z, x)
                \coloneqq
                \inf_{y \in Y} \mybrace{
                    d_{X \sqcup Y}(x, y) + d_{Y \sqcup Z}(y, z)
                }
                \qquad
                (x \in X, \; z \in Z)
    \end{alignat}
    と定義する。

    \uline{Step 2: $d_{X \sqcup Z}$が距離であること} \quad
    $d_{X \sqcup Z}$は
    定義から明らかに正値性、非退化性、対称律をみたすから、
    あとは三角不等式の成立を確かめればよい。
    $x, x' \in X, \; z \in Z$として、
    経路$x \leadsto z \leadsto x'$に沿った距離の和は
    \begin{alignat}{1}
        d_{X \sqcup Z}(x, z)
            + d_{X \sqcup Z}(z, x')
            &=
                \inf_{y \in Y} \mybrace{
                    d_{X \sqcup Y}(x, y) + d_{Y \sqcup Z}(y, z)
                }
                +
                \inf_{y' \in Y} \mybrace{
                    d_{Y \sqcup Z}(z, y') + d_{X \sqcup Y}(y', x')
                }
                \\
            &=
                \inf_{y, y' \in Y} \mybrace{
                    d_{X \sqcup Y}(x, y) + d_{Y \sqcup Z}(y, z)
                    +
                    d_{Y \sqcup Z}(z, y') + d_{X \sqcup Y}(y', x')
                }
                \\
            &\ge
                \inf_{y, y' \in Y} \mybrace{
                    d_{X \sqcup Y}(x, y)
                    + d_{Y \sqcup Z}(y, y')
                    + d_{X \sqcup Y}(y', x')
                }
                \\
            &=
                \inf_{y, y' \in Y} \mybrace{
                    d_{X \sqcup Y}(x, y)
                    + d_{X \sqcup Y}(y, y')
                    + d_{X \sqcup Y}(y', x')
                }
                \\
            &\ge
                d_{X \sqcup Y}(x, x')
    \end{alignat}
    より三角不等式をみたす。
    経路$x \leadsto x' \leadsto z$に沿った距離の和は
    \begin{alignat}{1}
        d_{X \sqcup Z}(x, x')
            + d_{X \sqcup Z}(x', z)
            &=
                d_{X \sqcup Y}(x, x')
                +
                \inf_{y \in Y} \mybrace{
                    d_{X \sqcup Y}(x', y) + d_{Y \sqcup Z}(y, z)
                }
                \\
            &=
                \inf_{y \in Y} \mybrace{
                    d_{X \sqcup Y}(x, x')
                    + d_{X \sqcup Y}(x', y)
                    + d_{Y \sqcup Z}(y, z)
                }
                \\
            &\ge
                \inf_{y \in Y} \mybrace{
                    d_{X \sqcup Y}(x, y)
                    + d_{Y \sqcup Z}(y, z)
                }
                \\
            &=
                d_{X \sqcup Z}(x, z)
    \end{alignat}
    より三角不等式をみたす。
    他の経路についても同様にして三角不等式をみたすことがわかる。
    したがって$d_{X \sqcup Z}$は$X \sqcup Z$上の距離である。

    \uline{Step 3: $d_{X \sqcup Z}$が$P(s + t, X, Z)$をみたすこと} \quad
    $d_{X \sqcup Z}|_X = d_X, \;
        d_{X \sqcup Z}|_Z = d_Z$
    は定義より明らかである。
    つぎに
    「$\forall x \in X, \; \exists z \in Z, \; d_{X \sqcup Z}(x, z) < s + t$」を示す。
    そこで$x \in X$を任意とする。
    すると$d_{X \sqcup Y}$が$P(s, X, Y)$をみたすことから、
    ある$y \in Y$が存在して$d_{X \sqcup Y}(x, y) < s$となる。
    一方$d_{Y \sqcup Z}$が$P(t, Y, Z)$をみたすことから、
    ある$z \in Z$が存在して$d_{Y \sqcup Z}(y, z) < t$となる。
    したがって
    \begin{equation}
        d_{X \sqcup Z}(x, z)
            \le
                d_{X \sqcup Y}(x, y) + d_{Y \sqcup Z}(y, z)
            <
                s + t
    \end{equation}
    が成り立つ。
    同様にして
    「$\forall z \in Z, \; \exists x \in X, \; d_{X \sqcup Z}(x, z) < s + t$」も示される。
    したがって$d_{X \sqcup Z}$は$P(s + t, X, Z)$をみたす。

    以上より
    $s + t \in A_{X, Z}$が示された。
    したがって$d_\GH$は三角不等式をみたす。
\end{proof}

まず非退化性の右向きの含意を示す。

\begin{lemma}
    \label[lemma]{lemma:GH_nondegeneracy_1}
    $\calX = \calY \implies d_\GH(\calX, \calY) = 0$が成り立つ。
\end{lemma}

\begin{proof}
    $\calX = \calY$とする。
    $d_\GH(\calX, \calY) = 0$を示すには、
    $\calX, \calY$の代表元$X, Y$をひとつずつ選んで、
    すべての$n \in \Z_{\ge 1}$に対し
    $\tfrac{1}{n} \in A_{X, Y}$が成り立つことをいえば十分である
    ($A_{X, Y}$は\cref{eq:def-A}で定義したもの)。
    そのためには、$X \sqcup Y$上の距離$d_{X \sqcup Y}$であって
    条件$P(\tfrac{1}{n}, X, Y)$をみたすものを構成すればよい。
    そこで、写像
    $d_{X \sqcup Y} \colon (X \sqcup Y) \times (X \sqcup Y) \to [0, +\infty)$を
    次のように定める。
    まず$\calX = \calY$より$X \cong Y$であるから、
    等長同型写像$f \colon X \stackrel{\sim}{\to} Y$をひとつ選ぶことができる。
    これを用いて
    \begin{alignat}{1}
        d_{X \sqcup Y}(x, x')
            &\coloneqq
                d_X(x, x')
                \qquad
                (x, x' \in X)
                \\
        d_{X \sqcup Y}(y, y')
            &\coloneqq
                d_Y(y, y')
                \qquad
                (y, y' \in Y)
                \\
        d_{X \sqcup Y}(x, y)
            &\coloneqq
                d_{X \sqcup Y}(y, x)
            \coloneqq
                d_Y(f(x), y)
                + \frac{1}{2n}
                \qquad
                (x \in X, \; y \in Y)
    \end{alignat}
    と定める。
    すると$d_{X \sqcup Y}$は明らかに
    正値性、非退化性、対称律、三角不等式をみたし、
    $X \sqcup Y$上の距離となる。

    $d_{X \sqcup Y}$が条件$P(\tfrac{1}{n}, X, Y)$をみたすことを確かめる。
    まず$d_{X \sqcup Y}|_X = d_X, \; d_{X \sqcup Y}|_Y = d_Y$は定義より明らかである。
    つぎに
    「$\forall x \in X, \;
        \exists y \in Y, \;
        d_{X \sqcup Y}(x, y) < \tfrac{1}{n}$」を示す。
    各$x \in X$に対し、$y \coloneqq f(x)$とおけば
    \begin{alignat}{1}
        d_{X \sqcup Y}(x, y)
            &=
                d_Y(f(x), y)
                + \frac{1}{2n}
                \\
            &=
                d_X(x, x)
                + \frac{1}{2n}
                \qquad
                (\because \text{$f$は等長})
                \\
            &<
                \frac{1}{n}
    \end{alignat}
    が成り立つ。
    最後に
    「$\forall y \in Y, \;
        \exists x \in X, \;
        d_{X \sqcup Y}(x, y) < \tfrac{1}{n}$」を示す。
    各$y \in Y$に対し、$x \coloneqq f^{-1}(y)$とおけば
    (ここで$f$が逆写像を持つことを用いた)
    \begin{alignat}{1}
        d_{X \sqcup Y}(x, y)
            &=
                d_Y(f(x), y)
                + \frac{1}{2n}
                \\
            &=
                d_Y(y, y)
                + \frac{1}{2n}
                \qquad
                (\because \text{$f$は等長})
                \\
            &<
                \frac{1}{n}
    \end{alignat}
    が成り立つ。
    したがって$d_{X \sqcup Y}$は条件$P(\tfrac{1}{n}, X, Y)$をみたす。

    以上より、
    すべての$n \in \Z_{\ge 1}$に対し
    $\tfrac{1}{n} \in A_{X, Y}$が成り立つことがわかった。
    したがって$d_\GH(\calX, \calY) = \inf A_{X, Y} = 0$である。
\end{proof}

非退化性の左向きの含意を示す。

\begin{lemma}
    \label[lemma]{lemma:GH_nondegeneracy_2}
    $d_\GH(\calX, \calY) = 0 \implies \calX = \calY$が成り立つ。
\end{lemma}

この補題の証明には次の補題を用いる。
イメージとしては、"ほとんど"等長な全射の存在を示すものである。

\begin{lemma}
    \label[lemma]{lemma:GH_approx_map}
    $d_\GH(\calX, \calY) = 0$とする。
    このとき、
    各$n \in \Z_{\ge 1}$に対し、
    ある写像$f_n \colon X \to Y$であって
    次をみたすものが存在する:
    \begin{enumerate}
        \item すべての$x, x' \in X$に対し
            $\myabs{
                d_X(x, x') - d_Y(f_n(x), f_n(x'))
            } < \tfrac{1}{n}$
            が成り立つ。
        \item $Y = \widebar{B}_{\tfrac{1}{n}}(f_n(X))$が成り立つ。
            ただし$\widebar{B}_{\tfrac{1}{n}}(\cdot)$は
            $\tfrac{1}{n}$-閉近傍を表す。
    \end{enumerate}
\end{lemma}

\begin{proof}
    $n \in \Z_{\ge 1}$とし、
    写像$f_n$の構成を行う。
    まず$d_\GH(\calX, \calY) = 0$より、
    $X \sqcup Y$上の距離$d_{X \sqcup Y}$であって
    条件$P(\tfrac{1}{2n}, X, Y)$をみたすものが存在する。
    これを用いて、
    各$x \in X$に対し、
    $d_{X \sqcup Y}(x, y) < \tfrac{1}{2n}$をみたす
    $y \in Y$をひとつ選んで
    $f_n(x) \coloneqq y$と定める。
    すると、各$x, x' \in X$に対して
    \begin{alignat}{1}
        \phantom{=}
            &\myabs{
                d_X(x, x') - d_Y(f_n(x), f_n(x'))
            }
            \\
        &=
            \myabs{
                d_{X \sqcup Y}(x, x') - d_{X \sqcup Y}(f_n(x), f_n(x'))
            }
            \\
        &=
            \myabs{
                d_{X \sqcup Y}(x, x') - d_{X \sqcup Y}(x, f_n(x'))
                +
                d_{X \sqcup Y}(f_n(x), f_n(x')) - d_{X \sqcup Y}(x, f_n(x'))
            }
            \\
        &\le
            \myabs{
                d_{X \sqcup Y}(x, x') - d_{X \sqcup Y}(x, f_n(x'))
            }
            +
            \myabs{
                d_{X \sqcup Y}(f_n(x), f_n(x')) - d_{X \sqcup Y}(x, f_n(x'))
            }
            \\
        &\le
            d_{X \sqcup Y}(x', f_n(x'))
            +
            d_{X \sqcup Y}(f_n(x), x)
            \\
        &<
            \frac{1}{2n} + \frac{1}{2n}
            \\
        &=
            \frac{1}{n}
    \end{alignat}
    が成り立つ。
    これで(1)がいえた。

    また、各$y \in Y$に対し、
    $d_{X \sqcup Y}(x, y) < \tfrac{1}{2n}$をみたす
    $x \in X$をひとつ選ぶことができ、
    \begin{alignat}{1}
        d_Y(y, f_n(x))
            &=
                d_{X \sqcup Y}(y, f_n(x))
                \\
            &\le
                d_{X \sqcup Y}(y, x)
                +
                d_{X \sqcup Y}(x, f_n(x))
                \\
            &\le
                \frac{1}{2n}
                +
                \frac{1}{2n}
                \\
            &=
                \frac{1}{n}
    \end{alignat}
    が成り立つ。
    これで(2)がいえた。

    したがってこの$f_n$が求める写像である。
\end{proof}

\begin{proof}[\cref{lemma:GH_nondegeneracy_2}の証明]
    補題を示すためには、
    $d_\GH(\calX, \calY) = 0$とし、
    $\calX, \calY$の代表元$X, Y$をひとつずつ選んで
    等長同型写像$F \colon X \stackrel{\sim}{\to} Y$を構成すればよい。
    これは次の3ステップで行う:
    \begin{enumerate}[label=Step \arabic*:]
        \setlength{\leftskip}{1.5em}
        \item \cref{lemma:GH_approx_map}を用いて、
            $X$のある稠密部分集合から$Y$への一様連続写像$f$を構成する。
        \item $f$の$X$上への連続拡張$F$が等長写像であることを示す。
        \item $F$が全射、したがって等長同型であることを示す。
    \end{enumerate}

    \uline{Step 1:} \quad
    \cref{lemma:GH_approx_map}が存在を主張する写像列
    $(f_n)_n$をひとつ選ぶ。
    いま$X$はコンパクト距離空間だからとくに可分である。
    すなわち、$X$の稠密かつ可算なある部分集合
    $D \coloneqq \{ x_k \in X \mid k \in \N \}$
    が存在する。
    さらに$Y$はコンパクト距離空間だから、
    とくに点列コンパクトであり、
    したがって
    任意の$k \in \N$に対し、
    $Y$の点列$(f_n(x_k))_{n \in \N}$は収束部分列を持つ。
    したがって対角線論法により、
    写像列$(f_n)_{n \in \N}$のある部分列$(f_{n(j)})_{j \in \N}$が存在して、
    任意の$k \in \N$に対し、
    $(f_{n(j)}(x_k))_{j \in \N}$は
    $Y$内の点に収束する。
    このことを用いて写像
    $f \colon D \to Y$を
    $f(x_k) \coloneqq \lim_{j \to \infty} f_{n(j)}(x_k) \; (k \in \N)$
    と定める。
    このとき、$f$は
    \cref{lemma:GH_approx_map}の(1)より明らかに
    $D$上の等長写像だから、とくに$D$上一様連続である。

    \uline{Step 2:} \quad
    $Y$がコンパクト距離空間ゆえに完備距離空間であることに注意すれば、
    $f$は
    距離空間$X$の稠密部分集合$D$から
    完備距離空間$Y$への一様連続写像だから、
    $X$上への連続拡張$F \colon X \to Y$が一意に存在する。
    $f$は$D$上等長で
    $F$は$X$上連続だから、
    $F$は$X$上等長である。

    \uline{Step 3:} \quad
    $F$が全射であることを示す。
    すなわち$y \in Y$とし、ある$p \in X$であって
    $F(p) = y$となるものの存在を示す。
    いま\cref{lemma:GH_approx_map}の(2)より、
    $X$の点列$(p_n)_{n \in \N}$であって、
    各$n \in \N$に対し
    $d_Y(y, f_n(p_n)) \le \frac{1}{n}$
    をみたすものが存在する。
    この収束部分列を改めて$(p_n)_n$とおき、
    その極限を$p \; (\in X)$とおく。
    示すべきことは$F(p) = y$の成立であり、
    そのためには
    任意の$\eps > 0$に対し
    $d_Y(y, F(p)) < \eps$が成り立つことを示せばよい。

    次の補助図は、
    以下に箇条書きで述べる(i)-(vi)および(1)-(5)の関係性を
    模式的に表したものである
    (ただし、実線は2点間の距離が上から評価されることを表し、
    破線は$X$の点と$Y$の点のつながりを表す):
    \begin{equation}
        \begin{tikzcd}[every arrow/.append style={no head,no tail},row sep=huge]
            f_N(d)
                \ar{rr}{(2)}
                \ar{dd}{\text{(v)}}[swap]{(3)}
                \ar[dashed]{rd}[near start]{\text{(iii)}}
                &&f_N(p_N)
                    \ar{rr}{(1)}[swap]{\text{(ii)}}
                    \ar[dashed]{rd}[near start]{\text{(iii)}}
                &&y
                \\
            &d
                \ar{rr}[swap]{\text{(vi)}}
                \ar[bend left]{rrrr}[swap]{\text{(i)}}
                &&p_N
                    \ar{rr}[swap]{\text{(iv)}}
                &&p
                \\
            F(d)
                \ar{rr}{(4)}
                \ar[dashed]{ru}
                &&F(p_N)
                    \ar{rr}{(5)}
                    \ar[dashed]{ru}
                &&F(p)
                    \ar[dashed]{ru}
        \end{tikzcd}
    \end{equation}
    いま次が成り立っていることに注意する:
    \begin{enumerate}[label=(\roman*)]
        \item $X$における$D$の稠密性より、
            ある$d \in D$が存在して、
            $d_X(p, d) < \frac{\eps}{10}$が成り立つ。
        \item $(p_n)_n$の定め方より、
            ある$N_1 \in \N$が存在して、
            すべての$n \ge N_1$に対し
            $d_Y(y, f_n(p_n)) < \frac{\eps}{5}$が成り立つ。
        \item \cref{lemma:GH_approx_map}の(1)より、
            ある$N_2 \ge N_1$が存在して、
            すべての$n \ge N_2$および
            すべての$x, x' \in X$に対し
            $|d_X(x, x') - d_Y(f_n(x), f_n(x'))| < \frac{\eps}{10}$
            が成り立つ。
        \item $(p_n)_n$が$p$に収束することより、
            ある$N_3 \ge N_2$が存在して、
            すべての$n \ge N_3$に対し
            $d_X(p_n, p) < \frac{\eps}{10}$が成り立つ。
        \item $F(d) = f(d) = \lim_{j \to \infty} f_{n(j)}(d)$であることより、
            ある$J_0 \in \N$が存在して、
            すべての$j \ge J_0$に対し
            $n(j) \ge N_3$かつ
            $d_Y(F(d), f_{n(j)}(d)) < \frac{\eps}{5}$が成り立つ。
        \item (iv)より
            $d_X(p_N, p) < \frac{\eps}{10}$が成り立つことと、
            さらに(i)より
            $d_X(p, d) < \frac{\eps}{10}$が成り立つことより、
            $d_X(d, p_N) < \frac{\eps}{5}$が成り立つ。
    \end{enumerate}
    以上のことを用いて$d_Y(y, F(p)) < \eps$を示す。
    そこで$N \coloneqq n(J_0) \; (\ge N_3 \ge N_2 \ge N_1)$
    とおくと次が成り立つ:
    \begin{enumerate}
        \item (ii)より
            $d_Y(y, f_N(p_N)) < \frac{\eps}{5}$が成り立つ。
        \item (iii)より
            $|d_X(p_N, d) - d_Y(f_N(p_N), f_N(d))| < \frac{\eps}{10}$が成り立つことと、
            さらに(vi)より
            $d_X(p_N, d) < \frac{\eps}{10}$が成り立つことより、
            $d_Y(f_N(p_N), f_N(d)) < \frac{\eps}{5}$が成り立つ。
        \item (v)より
            $d_Y(f_N(d), F(d)) < \frac{\eps}{5}$が成り立つ。
        \item (vi)より$d_X(d, p_N) < \frac{\eps}{5}$が成り立つことと、
            $F$が$X$上の等長写像であることより、
            $d_Y(F(d), F(p_N)) < \frac{\eps}{5}$が成り立つ。
        \item (iv)より$d_X(p_N, p) < \frac{\eps}{10} < \frac{\eps}{5}$が成り立つことと、
            $F$が$X$上の等長写像であることより、
            $d_Y(F(p_N), F(p)) < \frac{\eps}{5}$が成り立つ。
    \end{enumerate}
    したがって、三角不等式より$d_Y(y, F(p)) < \eps$が成り立つ。
    $\eps > 0$は任意だったので、
    $d_Y(y, F(p)) = 0$が成り立ち、
    したがって$y = F(p)$が示された。
    よって$F$は全射、したがって等長同型である。
    これで Step 3 が完了した。

    以上より$X \cong Y$となり、
    したがって$\calX = \calY$となることが示された。
\end{proof}

最後に目標の定理を証明する。

\begin{proof}[\cref{thm:GH_is_distance}の証明.]
    $d_\GH$が$\CMet$上の距離であることを示すには、
    正値性、非退化性、対称律、三角不等式を示せばよい。
    正値性と対称律は$d_\GH$の定義から明らかである。
    また、三角不等式と非退化性は
    \cref{lemma:GH_triangle_inequality,lemma:GH_nondegeneracy_1,lemma:GH_nondegeneracy_2}
    で示した。
    したがって$d_\GH$は$\CMet$上の距離である。
\end{proof}

\end{document}