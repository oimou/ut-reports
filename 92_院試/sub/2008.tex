\documentclass[report]{jlreq}
\usepackage{global}
\usepackage{./local}
\subfiletrue
\def\assetspath{../}
%\makeindex
\chead{2008}
\begin{document}

% ------------------------------------------------------------
%
% ------------------------------------------------------------
\section{A}

\begin{problem}[第1問]
    $3$次の実正方行列全体を$M_3(\R)$と表し、これを自然に$\R$上のベクトル空間とみなす。
    実数$a, b, c$に対して
    \[
        A = \begin{pmatrix}
            1 & 0 & c \\
            0 & 1 & b \\
            c & b & a
        \end{pmatrix}
    \]
    とおく。以下の問に答えよ。
    \begin{enumerate}
        \item $A$と可換な行列全体からなる$M_3(\R)$の部分集合$W$は、
            $M_3(\R)$の部分ベクトル空間をなすことを示せ。
        \item $W$の$\R$上のベクトル空間としての次元を求めよ。
    \end{enumerate}
\end{problem}

\begin{answer}
    \TODO{}
\end{answer}

\begin{problem}[第2問]
    $g(x, y) = y^4 - y^6 - 3(x^2 + x^4)$とおく。以下の問に答えよ。
    \begin{enumerate}
        \item $S = \{(x, y) \in \R^2 \mid g(x, y) = g_x(x, y) = g_y(x, y) = 0\}$を求めよ。
        \item 曲線$C = \{(x, y) \in \R^2 \setminus S \mid g(x, y) = 0, y > 0\}$上で
            $f(x, y) = x^2 + y^2$が極値をとる点をすべて求め、その値が極大であるか極小であるかを判定せよ。
    \end{enumerate}
\end{problem}

\begin{answer}
    \TODO{概形を書くと$(0, 1)$しかなさそう?}
\end{answer}

% ------------------------------------------------------------
%
% ------------------------------------------------------------
\section{B}

\begin{problem}[第8問]
    $M_2(\R)$を2次の実正方行列全体とする。対応
    \[
        \begin{pmatrix}
            x & z \\
            y & w
        \end{pmatrix}
            \mapsto
            (x, y, z, w)
    \]
    により$M_2(\R)$を$\R^4$と同一視し、
    これによって$M_2(\R)$上に座標$x, y, z, w$と
    標準的なリーマン計量$\langle,\rangle$を与える。
    2次の実対称行列全体を$H$とし、
    写像$F\colon M_2(\R) \to H$を$F(A) = \up{t}A A$により定める。
    ただし$\up{t}A$は$A$の転置行列である。
    このとき以下の問に答えよ。
    \begin{enumerate}
        \item 写像$F$の$A \in M_2(\R)$における微分$(dF)_A$を求め、
            $F$の正則点全体の集合を決定せよ。
        \item $A \in M_2(\R)$における$M_2(\R)$の接ベクトル$X_A$を
            \[
                X_A = \frac{d}{dt}(R_t A) \Bigg|_{t=0} \in T_AM_2(\R)
            \]
            ただし$R_t = \begin{pmatrix}
                \cos t & -\sin t \\
                \sin t & \cos t
            \end{pmatrix}$と定める。
            さらに、開部分多様体$P = \{A \in M_2(\R) \mid \det A > 0 \}$上の
            1次微分形式$\theta$を、すべての$A \in M_2(\R)$について
            次の条件(a),(b)が満たされるように定める:
            \begin{enumerate}[label=(\alph*)]
                \item $\theta(X_A) = 1$
                \item $\langle X_A, V \rangle = 0$ならば$\theta(V) = 0$である。
            \end{enumerate}
            ここに、$V$は$A$における$M_2(\R)$の接ベクトルであり、
            $\langle,\rangle$は上で定めたリーマン計量である。
            このとき、微分形式$\theta$を座標$x,y,z,w$を用いて表せ。
        \item 写像$F \colon M_{2}(\R) \to H$を
            $P$へ制限して得られる写像を$\pi \colon P \to B$とする。
            ただし、$B = F(P)$である。
            このとき、(2)で定めた微分形式$\theta$の外微分$d\theta$は、
            像$B$上のある微分形式$\omega$の$\pi$による引き戻し
            $\pi^*\omega$に等しいことを証明せよ。
    \end{enumerate}
\end{problem}

\begin{answer}
    \uline{(1)} \quad
    $(dF)_A$をチャートに関する行列表示で表す。
    $A$の属する$M_2(\R)$のチャートとして
    \begin{equation}
        \varphi \colon M_2(\R) \to \R,
            \quad
            \begin{bmatrix}
                x & z \\
                y & w
            \end{bmatrix}
            \mapsto
            (x, y, z, w)
    \end{equation}
    を考え、
    $F(A)$の属する$H$のチャートとして
    \begin{equation}
        \psi \colon H \to \R,
            \quad
            \begin{bmatrix}
                a & b \\
                b & c
            \end{bmatrix}
            \mapsto
            (a, b, c)
    \end{equation}
    を考える。
    これらのチャートに関する$F$の座標表示
    $\what{F} = \psi \circ F \circ \varphi^{-1}$は
    $\what{F} \colon \R^4 \to \R^3, \;
        (x, y, z, w) \mapsto (x^2 + y^2, xz + yw, z^2 + w^2)$
    である。
    したがって$(dF)_A$は、
    チャート$\varphi, \psi$に関する行列表示が
    $\what{F}$の Jacobi 行列
    \begin{equation}
        \begin{bmatrix}
            2x & 2y & 0 & 0 \\
            z & w & x & y \\
            0 & 0 & 2z & 2w
        \end{bmatrix}
    \end{equation}
    となるような$\R$-線型写像$T_AM_2(\R) \to T_{F(A)}H$である。

    次に$F$の正則点全体の集合を決定する。
    $A = \begin{bmatrix}
        x & z \\
        y & w
    \end{bmatrix} \in M_2(\R)$に関し、
    点$A$が$F$の正則点であるための必要十分条件は、
    $(J\what{F})_{(x, y, z, w)}$がフルランクとなることである。
    ここで、
    $(J\what{F})_{(x, y, z, w)}$がフルランクでないための必要十分条件、
    すなわち
    $(J\what{F})_{(x, y, z, w)}$の3個の行ベクトルが
    $\R$上1次従属となるための必要十分条件は
    \begin{equation}
        \exists \; s, t \in \R
            \quad \text{s.t.} \quad
            (x, y) = s(z, w), \; (z, w) = t(x, y)
    \end{equation}
    である。
    これは$\det A = 0$と同値である。
    したがって、
    $F$の正則点全体の集合は
    $\{ A \in M_2(\R) \mid \det A \neq 0 \}$である。

    \uline{(2)} \quad
    まず各$A = \begin{bmatrix}
        x & z \\
        y & w
    \end{bmatrix} \in M_2(\R)$に対し
    \begin{equation}
        X_A
            = \frac{d}{dt}(R_tA) \Bigg|_{t=0}
            = \frac{d}{dt}R_t \Bigg|_{t=0} A
            = \begin{bmatrix}
                0 & -1 \\
                1 & 0
            \end{bmatrix} A
            = \begin{bmatrix}
                -y & -w \\
                x & z
            \end{bmatrix}
    \end{equation}
    である。
    $\theta$の成分表示を求める。
    そこで計量$\langle,\rangle$に関する
    $T_AM_2(\R)$の直交分解
    $T_AM_2(\R) = \R X_A \oplus (\R X_A)^{\perp}$
    を考え、
    $\R X_A$の基底$X_A$と
    $(\R X_A)^{\perp}$の基底$b_i \; (i = 1, \dots, 3)$をあわせた
    $T_AM_2(\R)$の基底$X_A, b_i \; (i = 1, \dots, 3)$をひとつ選ぶ。
    すると条件(a), (b)より
    \begin{alignat}{2}
        \langle X_A, \cdot \rangle &\colon
            X_A \mapsto \| X_A \|^2,
            \quad
            &&b_i \mapsto 0 \quad (i = 1, \dots, 3) \\
        \theta_A &\colon
            X_A \mapsto 1,
            \quad
            &&b_i \mapsto 0 \quad (i = 1, \dots, 3)
    \end{alignat}
    が成り立つから
    $\theta_A = \langle X_A, \cdot \rangle / \| X_A \|^2$
    である。
    したがって、$\theta$を座標$x, y, z, w$を用いて表すと
    \begin{equation}
        \theta
            = \frac{1}{x^2 + y^2 + z^2 + w^2}
                \myparen{
                    -y \, dx + x \, dy - w \, dz + z \, dw
                }
    \end{equation}
    となる。

    \uline{(3)} \quad
    \TODO{よくわからない}
\end{answer}

\end{document}