\documentclass[report]{jlreq}
\usepackage{global}
\usepackage{./local}
\subfiletrue
\def\assetspath{../}
%\makeindex
\chead{2016}
\begin{document}

% ------------------------------------------------------------
%
% ------------------------------------------------------------
\begin{proof}[2016 A1.]
    \uline{(1)} \quad
    $A_1$の階数は
    $A_1$の$\C$上1次独立な行ベクトルの最大個数に等しい。
    したがって、
    $a = 0$または$a = 1$のとき$\rank A_1 = 3$であり、
    $a \neq 0, a \neq 1$のとき$\rank A_1 = 4$である。

    \uline{(2)} \quad
    答えは$a = 1$で尽くされることを以下で示す。

    まず$a = 0, 1$が必要であることを示す。
    そこで$a \neq 0, \, a \neq 1$を仮定すると、
    (1)より$\dim V_1 = 5 - 4 = 1$であり、
    $\rank A_2 = 4$より$\dim V_2 = 5 - 4 = 1$であり、
    さらに$\dim V_3 = 2$である。
    したがって
    $\dim (V_1 + V_2 + V_3) \le 1 + 1 + 2 = 4 < 5$となり、
    $V_1 \oplus V_2 \oplus V_3 = \C^5$とはなりえない。
    よって$a \neq 0, \, a \neq 1$が必要である。

    $a = 0$の場合を考える。
    各$v \in \C^5$に対し$v$の第2成分を$(v)_2$と書くことにすれば、
    任意の$v_i \in V_i \; (i = 1, 2, 3)$に対し
    $(v_1)_2 = (A_1 v_1)_2 = 0, \;
        (v_2)_2 = (A_2 v_2)_2 = 0, \;
        (v_3)_2 = 0$
    が成り立つから、
    $e_2 \not\in V_1 + V_2 + V_3$である。
    よって$V_1 \oplus V_2 \oplus V_3 = \C^5$とはなりえない。

    $a = 1$の場合を考える。
    明らかに
    $V_3 = \C \up{t}(1, 1, 0, 0, 0)$である。
    また、$\up{t}(x_1, \dots, x_5) \in \C^5$が
    $V_1$に属する条件は
    \begin{alignat}{1}
        \begin{pmatrix}
            x_1 \\ x_2 \\ x_3 \\ x_4 \\ x_5
        \end{pmatrix}
            \in V_1
            &\iff
                \begin{pmatrix}
                    x_1 + x_4 \\
                    x_2 \\
                    x_3 \\
                    0
                \end{pmatrix}
                = 0
            \iff
                \exists \, s, t \in \C
                \quad \text{s.t.} \quad
                \left(
                    \;
                    \begin{aligned}
                        x_1 &= s \\
                        x_2 &= 0 \\
                        x_3 &= 0 \\
                        x_4 &= -s \\
                        x_5 &= t
                    \end{aligned}
                \right.
    \end{alignat}
    となるから
    $V_1 = \C \up{t}(1, 0, 0, -1, 0) \oplus \C \up{t}(0, 0, 0, 0, 1)$である。
    同様にして
    $V_2 = \C \up{t}(1, 0, 1, -1, 0) \oplus \C \up{t}(0, 1, 0, 0, 0)$
    を得る。
    以上の$V_1, V_2, V_3$の基底たちを並べた行列
    $\begin{pmatrix}
        1 & 1 & 0 & 1 & 0 \\
        1 & 0 & 0 & -1 & 1 \\
        0 & 0 & 0 & 1 & 0 \\
        0 & -1 & 0 & -1 & 0 \\
        0 & 0 & 1 & 0 & 0
    \end{pmatrix}$
    は行と列の基本変形により単位行列となるから階数は$5$で、
    したがって正則である。
    よってこの行列の列ベクトルは$\C^5$の基底であり、
    $V_1 \oplus V_2 \oplus V_3 = \C^5$が示された。
\end{proof}

\begin{proof}[2016 A2.]
    \uline{(1)} \quad
    $I_n \coloneqq \int_0^{\pi / 2} (\sin x)^{2n + 1} \, dx \; (n \in \Z_{\ge 0})$とおくと、
    $n = 0$に対しては
    \begin{equation}
        I_0 = \int_0^{\pi / 2} \sin x \, dx
            = \mybrace{ - \cos x }_0^{\pi / 2}
            = 1
    \end{equation}
    であり、
    各$n \in \Z_{\ge 1}$に対しては
    部分積分より
    \begin{alignat}{1}
        I_n
            &=
                \int_0^{\pi / 2} (\sin x)^{2n + 1} \, dx
                \\
            &=
                \mybrace{
                    - \cos x (\sin x)^{2n}
                }_0^{\pi / 2}
                +
                \int_0^{\pi / 2} 2n \cos^2 x (\sin x)^{2n - 1} \, dx
                \\
            &=
                2n \int_0^{\pi / 2} (1 - \sin^2 x) (\sin x)^{2n - 1} \, dx
                \\
            &=
                2n I_{n - 1} - 2n I_n
                \\
        \therefore \quad
        I_n
            &=
                \frac{2n}{2n + 1} I_{n - 1}
    \end{alignat}
    となる。
    したがって
    $I_n
        =
            \frac{2n}{2n + 1} I_{n - 1}
        =
            \cdots
        =
            \frac{2n}{2n + 1} \cdot \frac{2(n - 1)}{2(n - 1) + 1} \cdots \frac{2}{3}$
    である。

    \uline{(2)} \quad
    $\sum_{n = 0}^\infty \frac{1}{(2n + 1)!}$が優級数となることを示せばよい。
    まず、この級数が収束することは
    $\frac{1}{(2n + 1)!} \le \frac{1}{n!} \; (n \in \Z_{\ge 0})$と
    $\sum_{n = 0}^\infty \frac{1}{n!} = e$より従う。
    また、各$n \ge 0$に対し
    \begin{equation}
        \myabs{
            \frac{(-1)^n}{(2n + 1)!}
            (\sin x)^{2n + 1}
        }
            \le
                \frac{1}{(2n + 1)!}
                \quad
                (\forall x \in [0, \pi / 2])
    \end{equation}
    が成り立つ。
    したがって、
    $\sum_{n = 0}^\infty \frac{1}{(2n + 1)!}$を優級数として
    $\sum_{n = 0}^\infty \frac{(-1)^n}{(2n + 1)!} (\sin x)^{2n + 1}$は
    $[0, \pi / 2]$上一様収束する。

    \uline{(3)} \quad
    求める答えは$0.89$であることを示す。
    そのためには
    \begin{equation}
        \myabs{
            \int_0^{\pi / 2}
                \sin (\sin x) \, dx
            - 0.895
        }
            < 0.005
    \end{equation}
    の成立を示せば十分である。
    まず所与の積分を式変形すると
    \begin{alignat}{1}
        \int_0^{\pi / 2} \sin (\sin x) \, dx
            &=
                \int_0^{\pi / 2}
                    \sum_{n = 0}^\infty
                        \frac{(-1)^n}{(2n + 1)!} (\sin x)^{2n + 1}
                    \, dx
                \\
            &=
                \sum_{n = 0}^\infty
                    \frac{(-1)^n}{(2n + 1)!}
                    \int_0^{\pi / 2} (\sin x)^{2n + 1}
                    \, dx
                \qquad
                (\text{一様収束性より項別積分が可能})
                \\
            &=
                \sum_{n = 0}^\infty
                    \frac{(-1)^n}{(2n + 1)!}
                    I_n
    \end{alignat}
    となる。
    よって
    \begin{alignat}{1}
        \myabs{
            \int_0^{\pi / 2}
                \sin (\sin x) \, dx
            -
            0.895
        }
            &=
                \myabs{
                    \sum_{n = 0}^\infty
                        \frac{(-1)^n}{(2n + 1)!}
                        I_n
                    -
                    0.895
                }
                \\
            &\le
                \myabs{
                    \sum_{n = 3}^\infty
                        \frac{(-1)^n}{(2n + 1)!}
                        I_n
                }
                +
                \myabs{
                    \myparen{
                        I_0 - \frac{1}{3!} I_1 + \frac{1}{5!} I_2
                    }
                    -
                    0.895
                }
                \locallabel{eq:1}
    \end{alignat}
    の右辺を評価すればよい。

    まず\localcref{eq:1}の第1項を評価すると
    \begin{alignat}{1}
        \myabs{
            \sum_{n = 3}^\infty
                \frac{(-1)^n}{(2n + 1)!}
                I_n
        }
            &\le
                \sum_{n = 3}^\infty
                    \frac{1}{(2n + 1)!}
                \qquad
                (|I_n| \le 1)
                \\
            &\le
                \sum_{n = 3}^\infty
                    \frac{1}{10^n}
                \qquad
                (\text{*})
                \\
            &=
                \frac{1}{10^3} \frac{1}{9}
                \\
            &\le
                \frac{1}{10^3}
                \frac{1}{8}
                \\
            &=
                0.00125
    \end{alignat}
    を得る。
    ただし(*)の式変形は、
    帰納法よりすべての$n \ge 3$に対し
    $(2n + 1)! \ge 10^n$が成り立つことを用いた。

    次に\localcref{eq:1}の第2項を評価すると
    \begin{alignat}{1}
        \myabs{
            \myparen{
                I_0 - \frac{1}{3!} I_1 + \frac{1}{5!} I_2
            }
            -
            0.895
        }
            &=
                \myabs{
                    1 - \frac{1}{3!} \frac{2}{3} + \frac{1}{5!} \frac{4}{5} \frac{2}{3}
                    -
                    0.895
                }
                \\
            &=
                \myabs{
                    \frac{105}{1000}
                    - \frac{1}{3!} \frac{2}{3} + \frac{1}{5!} \frac{4}{5} \frac{2}{3}
                }
                \\
            &=
                \frac{3}{2^3 \cdot 3^2 \cdot 5^2}
                \\
            &=
                \frac{30}{2^4 \cdot 3^2 \cdot 5^3}
                \\
            &<
                \frac{36}{2^4 \cdot 3^2 \cdot 5^3}
                \\
            &=
                \frac{4}{2^4 \cdot 5^3}
                \\
            &=
                0.002
    \end{alignat}
    を得る。

    以上の評価を用いて\localcref{eq:1}から
    \begin{equation}
        \myabs{
            \int_0^{\pi / 2}
                \sin (\sin x)
                \, dx
            -
            0.895
        }
            < 0.00125 + 0.002
            < 0.005
    \end{equation}
    が得られた。
\end{proof}

\begin{proof}[2016 A4.]
    \uline{(1)} \quad
    以下$T \in \R$は固定し、
    $S_T \coloneqq \{ x \in X \mid g(x) \le T \}$とおいておく。
    $X = \R^n$だから、
    Heine-Borel の定理より、
    $S_T$がコンパクトであることを示すには
    $S_T$が$X$で有界かつ閉であることを示せばよい。

    \uline{Step 1: $S_T$が閉であること} \quad
    $S_T$は
    $\R$の閉集合$(\infty, T]$の$g$による逆像であり、
    また$g$はふたつの連続写像$d_Y({-}, y_0) \colon Y \to \R$
    および$f \colon X \to Y$の合成ゆえに連続だから、
    $S_T$は$X$で閉である。

    \uline{Step 2: $S_T$が有界であること} \quad
    任意の$x \in S_T$に対し
    $d_X(x, x_0) \in T + d_Y(f(x_0), y_0)$が成り立つことを示せば十分。
    \begin{alignat}{1}
        d_X(x, x_0)
            &\le
                d_Y(f(x), f(x_0))
                \qquad
                (\text{問題の仮定})
                \\
            &\le
                d_Y(f(x), y_0)
                    + d_Y(f(x_0), y_0)
                \\
            &\le
                g(x)
                    + d_Y(f(x_0), y_0)
                \\
            &\le
                T
                    + d_Y(f(x_0), y_0)
                \qquad
                (g(x) \in S_T)
                \\
    \end{alignat}
    より、$S_T$は有界である。

    以上より$S_T$は$X$で有界かつ閉だから、コンパクトである。

    \uline{(2)} \quad
    $T_0 \coloneqq g(x_0)$とおくと、
    集合$S_{T_0}$は(1)よりコンパクトであり、
    $x_0$を含むから非空である。
    よって連続関数$g$の$S_{T_0}$上への制限は、
    非空コンパクト空間上の連続関数だから最小値をもつ。
    すなわち、
    ある$x_1 \in S_{T_0}$が存在して、
    すべての$x \in S_{T_0}$に対し
    $g(x) \ge g(x_1)$が成り立つ。
    このとき$g(x_1)$は$X$上の$g$の最小値でもある。
    実際、
    $x \in X$に関し
    $x \in S_{T_0}$ならば$g(x) \ge g(x_1)$であるし、
    $x \not\in S_{T_0}$ならば
    $g(x) > T_0 \ge g(x_1)$である。
    よって$g$は$X$上で最小値をもつことが示された。
\end{proof}

\begin{proof}[2016 A6.]
    求める解のひとつは
    \begin{equation}
        \begin{cases}
            x(t) = (t^2 + 2t) e^{2t} \\
            y(t) = (-2t^2 - 2t + 1) e^{2t}
        \end{cases}
    \end{equation}
    であり、初期値問題の解の一意性より、これが求める解のすべてである。
    以下に求め方を記す。

    所与の方程式から$y$を消去すると
    \begin{equation}
        \locallabel{eq:1}
        x'' - 4x' + 4x = 2e^{2t}
    \end{equation}
    を得る。
    この斉次解は$x = c e^{2t}, \; (\text{$c$は任意定数})$である。
    定数変化法により
    $c$を$t$の関数とみると
    \begin{equation}
        x'' - 4x' + 4x = c'' e^{2t}
    \end{equation}
    となるから、
    \localcref{eq:1}と比較して
    $c'' = 2$、
    したがって
    $c = t^2 + c_0 t + c_1 \; (\text{$c_0, c_1$は任意定数})$を得る。
    よって$x = (t^2 + c_0 t + c_1) e^{2t}$であるが、
    初期条件$x(0) = 0$より
    $c_1 = 0$だから
    $x = (t^2 + c_0 t) e^{2t}$である。
    所与の方程式の第1式より
    \begin{alignat}{1}
        y
            &=
                x' - 4x - e^{2t}
                \\
    \intertext{$x = (t^2 + c_0 t) e^{2t}$を代入して整理すると}
            &=
                (-2t^2 + 2(-c_0 + 1)t + c_0 - 1) e^{2t}
    \end{alignat}
    を得る。
    初期条件$y(0) = 1$より
    $c_0 = 2$を得る。
    以上より
    $x(t) = (t^2 + 2t) e^{2t}, \;
        y(t) = (-2t^2 - 2t + 1) e^{2t}$
    が得られた。
\end{proof}

\end{document}