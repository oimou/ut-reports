\documentclass[report]{jlreq}
\usepackage{global}
\usepackage{./local}
\subfiletrue
\def\assetspath{../}
%\makeindex
\chead{2017}
\begin{document}

% ------------------------------------------------------------
%
% ------------------------------------------------------------
\begin{proof}[2017 A6.]
    \uline{(1)} \quad
    $\dim \Ker A \neq \dim \Ker B \implies [A] \neq [B]$
    より
    $X = \{ [O] \} \sqcup \{ [I] \} \sqcup \{ [A] : \dim \Ker A = 1 \}$だから、
    $\{ [A] : \dim \Ker A = 1 \}$が
    $\R P^1$に同相であることを示せばよい。
    \begin{description}
        \item[claim] 集合として次のような一致が成り立つ:
            \begin{equation}
                \{ [A] \in X : \dim \Ker A = 1 \}
                    =
                        \mybrace{
                            \mybracket{
                                P_\theta
                            }
                            \in X
                            \, \Bigg| \,
                            P_\theta =
                                \begin{pmatrix}
                                    \cos^2 \theta & \cos \theta \sin \theta \\
                                    \cos \theta \sin \theta & \sin^2 \theta
                                \end{pmatrix}, \;
                            \theta \in \R
                        }
            \end{equation}
    \end{description}
    \begin{innerproof}
        「$\supset$」は明らかなので、「$\subset$」を示す。
        各$\alpha \in X$に対し、
        $\alpha = [A]$となる$A \in M_2(\R), \; \dim \Ker A = 1$
        をひとつ選ぶ。
        このとき
        $\dim (\Ker f_A)^\perp = 2 - \dim \Ker f_A = 1$
        ゆえに
        $(\Ker f_A)^\perp$は
        $\R^2$内の原点を通る直線だから、
        ある$\theta \in \R$であって
        $(\Ker f_A)^\perp = \R e^{i\theta}$
        なるものが存在する。
        このとき
        $P_\theta$は$(\Ker f_A)^\perp$への直交射影である。
        したがって$\Ker A = \Ker P_\theta$だから
        $\alpha = [A] = [P_\theta]$となる。
        これで「$\subset$」が示されて、claim の証明が完了した。
    \end{innerproof}

    あとは
    $X' \coloneqq \{ [P_\theta] : \theta \in \R \}$
    とおいて
    $X' \approx \R P^1$を示せばよい。
    まず写像$\R \to M_2(\R), \; \theta \mapsto P_\theta$は連続かつ
    $2\pi\Z$の作用で不変だから、
    連続写像$S^1 \to M_2(\R), \; e^{i\theta} \mapsto P_\theta$が誘導される。
    これより連続全射
    $S^1 \to X', \; e^{i\theta} \mapsto [P_\theta]$
    が得られる。
    これは
    $\{ \pm 1 \}$の作用で不変
    (\because \,
        $P_{-\theta}$は$\R \up{t}(\cos(-\theta), \sin(-\theta))
            = \R \up{t}e^{i\theta}$
        への直交射影ゆえに$\Ker P_{-\theta} = \Ker P_\theta$)
    だから、
    連続全射$\R P^1 \to X', \; [e^{i\theta}]_{\text{proj}} \mapsto [P_\theta]$が誘導される。
    ここで、この写像は単射である。
    \begin{innerproof}
        $\theta, \theta' \in \R$に関し、
        $[P_\theta] = [P_{\theta'}]$ならば
        $\Ker P_\theta = \Ker P_{\theta'}$ゆえに
        $(\Ker P_\theta)^\perp = (\Ker P_{\theta'})^\perp$であり、
        $e^{i\theta} \in (\Ker P_\theta)^\perp, \;
            e^{i\theta'} \in (\Ker P_{\theta'})^\perp$
        であることとあわせて
        $e^{i\theta}, e^{i\theta'}$は同じ直線上にあることがわかる。
        したがって$[e^{i\theta}]_{\text{proj}} = [e^{i\theta'}]_{\text{proj}}$である。
    \end{innerproof}
    コンパクト空間から Hausdorff 空間への連続全単射は同相であるから、
    $\R P^1 \approx X'$が示された。

    \uline{(2)} \quad
    $[O] \in U \opensubset X$とすると、
    $O \in \pi^{-1}(U) \opensubset M_2(\R)$である。
    ここで、
    任意の$A \in M_2(\R)$に対し、
    ある$c > 0$であって
    $cA \in \pi^{-1}(U)$となるものが存在することに注意すれば、
    $[A] = [cA] \in U$
    である。
    したがって$U = X$である。

    \uline{(3)} \quad
    次のことが成り立つ:
    \begin{itemize}
        \item $\{ [O] \}$は$X$の開集合でなく、閉集合である。
            $\{ [O] \}$を含む開集合は$X$全体のみである。
        \item $\{ [I] \}$は$X$の開集合であり、閉集合でない。
        \item $X'$は$X$の開集合でない。
            $X'$を含む開集合として$X \setminus \{ [O] \}$が存在する。
    \end{itemize}
    したがって
    $X$の自己同相写像は$[O], [I]$を固定する。
    よって、
    (1)の同相$\R P^1 \to X'$を$F$とおけば、
    各$\varphi \in \mathrm{Homeo}(X)$に対し
    $F^{-1} \circ (\varphi|_{X'}) \circ F \in \mathrm{Homeo}(\R P^1)$
    を割り当てる対応が
    全単射$\mathrm{Homeo}(X) \to \mathrm{Homeo}(\R P^1)$を与える。
\end{proof}

\end{document}