\documentclass[report]{jlreq}
\usepackage{global}
\usepackage{./local}
\subfiletrue
\def\assetspath{../}
%\makeindex
\chead{2021}
\begin{document}

% ------------------------------------------------------------
%
% ------------------------------------------------------------
\section{A}

\begin{proof}[2021 A1.]
    $V$の基底$1, x, x^2, x^3$の双対基底を
    $\omega_0, \omega_1, \omega_2, \omega_3 \in V^*$とおく。
    各$a \in \R$に対し、
    $\Ker g_a, \Ker h_a$を双対基底を使って書き直しておくと
    \begin{alignat}{1}
        \Ker g_a
            &=
                \myparen{
                    \R (a^3 \omega_3 + a^2 \omega_2 + a \omega_1 + \omega_0)
                }^\perp,
                \\
        \Ker h_a
            &=
                \myparen{
                    \R (3a^2 \omega_3 + 2a \omega_2 + \omega_1)
                }^\perp
    \end{alignat}
    となる。

    \uline{(1)} \quad
    問題文の和空間は、各項について
    \begin{alignat}{1}
        \Ker g_0 \cap \Ker g_1
            &=
                (\R \omega_0)^\perp
                \cap
                (\R (\omega_3 + \omega_2 + \omega_1 + \omega_0))^\perp
                \\
            &=
                \mybrace{
                    \R \omega_0
                    \oplus
                    \R (\omega_3 + \omega_2 + \omega_1 + \omega_0)
                }^\perp
                \\
            &=
                \mybrace{
                    \R \omega_0
                    \oplus
                    \R (\omega_3 + \omega_2 + \omega_1)
                }^\perp
                \\
        \Ker g_1 \cap \Ker g_2
            &=
                (\R (\omega_3 + \omega_2 + \omega_1 + \omega_0))^\perp
                \cap
                (\R (8 \omega_3 + 4 \omega_2 + 2 \omega_1))^\perp
                \\
            &=
                \mybrace{
                    \R (\omega_3 + \omega_2 + \omega_1 + \omega_0)
                    \oplus
                    \R (8 \omega_3 + 4 \omega_2 + 2 \omega_1 + \omega_0)
                }^\perp
                \\
            &=
                \mybrace{
                    \R (\omega_3 + \omega_2 + \omega_1 + \omega_0)
                    \oplus
                    \R (7 \omega_3 + 3 \omega_2 + \omega_1)
                }^\perp
    \end{alignat}
    と表せるから、
    \begin{alignat}{1}
        &\phantom{=} (\Ker g_0 \cap \Ker g_1) + (\Ker g_1 \cap \Ker g_2)
            \\
        &=
            \mybrace{
                (
                    \R \omega_0
                    \oplus
                    \R (\omega_3 + \omega_2 + \omega_1)
                )
                \cap
                (
                    \R (\omega_3 + \omega_2 + \omega_1 + \omega_0)
                    \oplus
                    \R (7 \omega_3 + 3 \omega_2 + \omega_1)
                )
            }^\perp
    \end{alignat}
    となる。
    直接計算より波括弧$\mybrace{ \cdots }$の中身は
    $\R (\omega_3 + \omega_2 + \omega_1 + \omega_0)$
    に一致するから、
    \begin{equation}
        (\Ker g_0 \cap \Ker g_1) + (\Ker g_1 \cap \Ker g_2)
            = \mybrace{
                \R (\omega_3 + \omega_2 + \omega_1 + \omega_0)
            }^\perp
    \end{equation}
    である。
    したがって、
    この空間の次元は
    $\dim V - \dim \mybrace{\R (\omega_3 + \omega_2 + \omega_1 + \omega_0)} = 4 - 1 = 3$
    であり、基底のひとつは
    $1 - x, \; x - x^2, \; x^2 - x^3$である。

    \uline{(2)} \quad
    問題文の和空間の第1項と第2項の共通部分が
    $0$とならない実数$a$を求めればよい。
    そこで共通部分を式変形すると
    \begin{alignat}{1}
        &\phantom{=}
            (\Ker g_0 \cap \Ker h_a) \cap (\Ker g_1 \cap \Ker h_0)
            \\
        &=
            (\R \omega_0)^\perp
            \cap
            (\R (3a^2 \omega_3 + 2a \omega_2 + \omega_1))^\perp
            \cap
            (\R (\omega_3 + \omega_2 + \omega_1 + \omega_0))^\perp
            \cap
            (\R \omega_1)^\perp
            \\
    \intertext{第2項以外をまとめて}
        &=
            (\R (3a^2 \omega_3 + 2a \omega_2 + \omega_1))^\perp
            \cap
            \mybrace{
                (\R \omega_0)
                \oplus
                (\R \omega_1)
                \oplus
                (\R (\omega_3 + \omega_2 + \omega_1 + \omega_0))
            }^\perp
            \\
        &=
            (\R (3a^2 \omega_3 + 2a \omega_2 + \omega_1))^\perp
            \cap
            \mybrace{
                (\R \omega_0)
                \oplus
                (\R \omega_1)
                \oplus
                (\R (\omega_3 + \omega_2))
            }^\perp
            \\
        &=
            (\R (3a^2 \omega_3 + 2a \omega_2 + \omega_1))^\perp
            \cap
            \R (x^2 - x^3)
    \end{alignat}
    となる。
    この共通部分が$0$とならないための必要十分条件は
    $(3a^2 \omega_3 + 2a \omega_2 + \omega_1)(x^2 - x^3) = 0$
    すなわち
    $-3a^2 + 2a = 0$
    が成り立つことである。
    したがって、$a = 0, \frac{2}{3}$が求める答えである。
\end{proof}

\begin{proof}[2021 A3.]
    \uline{(1)} \quad
    $f$は$\Z_{\ge 1}$上の数え上げ測度に関する積分とみなす。
    $f, g$それぞれについて、
    $\frac{\sin nx}{n^2}, \; \frac{\sin tx}{t^2}$の
    $x$に関する連続性と、
    優関数$\frac{1}{n^2}, \; \frac{1}{t^2}$の存在より、
    優収束定理を用いて
    絶対収束性と連続性が従う。

    \uline{(2)} \quad
    $C = 2$が求める定数のひとつであることを示す。
    そのために$x > 0$とし、
    $h_x \colon [1, \infty) \to \R, \;
        y \mapsto \frac{\sin yx}{y^2}$
    と定めると、
    示すべきことは、
    すべての$n \in \Z_{\ge 1}$と
    $n \le t \le n + 1$なる$t \in \R$に対し
    $|h_x(t) - h_x(n)| \le \frac{2x}{n^2}$が成り立つことである。
    $n = t$の場合は明らかだから、$n < t$の場合を考える。
    そこで区間$[n, t]$上で$h_x$に平均値定理を用いると、
    ある$t' \in (n, t)$が存在して
    $\frac{h_x(t) - h_x(n)}{t - n} = h_x'(t')$
    が成り立つ。
    よって
    \begin{alignat}{1}
        |h_x(t) - h_x(n)|
            &=
                |t - n| \myabs{
                    - \frac{2x \cos t'x}{t'^3}
                }
                \\
            &\le
                \frac{2 (t - n) x}{t^3}
                \\
            &\le
                \frac{2x}{n^2}
    \end{alignat}
    となり、目的の不等式が得られた。

    \uline{(3)} \quad
    (1)で示した$g$の連続性より$g(x) \to g(0) = 0 \; \text{as} \; x \to +0$である。
    そこで L'H\^opital の定理を用いるため、
    $g$の$\R_{> 0}$上での微分可能性を確かめる。
    まず部分積分と変数変換により
    $g(x) = \sin x + x \int_x^\infty \frac{\cos t}{t} \, dt \; (x \in \R)$
    である。
    この右辺の積分について、
    $\frac{\cos t}{t}$の
    $\R_{> 0}$上の原始関数は
    $\log t + \alpha(t) \;
        (\text{$\alpha$は$\R$上の解析関数})$
    の形であるから、
    そのひとつを選んで$G(t)$とおけば
    $\int_x^\infty \frac{\cos t}{t} \, dt
        = \lim_{t \to \infty} G(t) - G(x)
        \;
        (x > 0)$
    が成り立つ。
    このとき
    $\lim_{t \to \infty} G(t)$は$x$によらない実定数だから、
    これを$c \coloneqq \lim_{t \to \infty} G(t) \in \R$とおく。
    すると
    $g(x) = \sin x + x(c - G(x)) \; (x > 0)$
    が成り立つから、$g$は$\R_{> 0}$上微分可能である。
    さらに$g$の具体的な形より
    \begin{equation}
        \frac{
            g'(x)
        }{
            \myparen{
                x \log \frac{1}{x}
            }'
        }
            =
            \frac{
                \cos x + c - G(x) - xG'(x)
            }{
                - \log x + x^2
            }
            =
            \frac{
                \cos x + c - \log x - \alpha(x) - 1 - x\alpha'(x)
            }
            {
                - \log x + x^2
            }
            \to 1
            \quad
            \text{as}
            \quad
            x \to +0
    \end{equation}
    となるから、
    L'H\^opital の定理より
    $\lim_{x \to +0} \frac{g(x)}{x \log \frac{1}{x}} = 1$を得る。

    \uline{(4)} \quad
    (2)より、すべての$x > 0, \; t \ge 1$に対し
    $\myabs{
        \frac{\sin tx}{t^2}
        -
        \frac{\sin \lfloor t \rfloor x}{\lfloor t \rfloor^2}
    }
        \le
            \frac{Cx}{\lfloor t \rfloor^2}$
    が成り立つ ($\lfloor \cdot \rfloor$は床関数) から、
    すべての$x > 0$に対し
    \begin{alignat}{1}
        |f(x) - g(x)|
            &=
                \myabs{
                    \int_1^\infty
                        \myparen{
                            \frac{\sin tx}{t^2}
                            -
                            \frac{\sin \lfloor t \rfloor x}{\lfloor t \rfloor^2}
                        }
                        \, dt
                }
                \\
            &\le
                \int_1^\infty
                    \myabs{
                        \frac{\sin tx}{t^2}
                        -
                        \frac{\sin \lfloor t \rfloor x}{\lfloor t \rfloor^2}
                    }
                    \, dt
                \\
            &\le
                \int_1^\infty
                    \frac{Cx}{\lfloor t \rfloor^2}
                    \, dt
                \\
            &=
                Cx \sum_{n = 1}^\infty
                    \frac{1}{n^2}
    \end{alignat}
    したがって
    $\beta \coloneqq \sum_{n = 1}^\infty \frac{1}{n^2} \in \R$
    とおいて
    \begin{equation}
        \myabs{
            \frac{f(x)}{x \log \frac{1}{x}}
            -
            \frac{g(x)}{x \log \frac{1}{x}}
        }
            \le
                \myabs{
                    \frac{C \beta x}{x \log \frac{1}{x}}
                }
            =
                C \beta \frac{1}{\myabs{
                    \log \frac{1}{x}
                }}
            \to
                0
                \quad
                \text{as}
                \quad
                x \to +0
    \end{equation}
    が成り立つ。
    よって
    $\lim_{x \to +0}
        \frac{f(x)}{x \log \frac{1}{x}}
        =
        \lim_{x \to +0}
        \frac{g(x)}{x \log \frac{1}{x}}
        =
        1$
    を得る。
\end{proof}

\end{document}