\documentclass[report]{jlreq}
\usepackage{global}
\usepackage{./local}
\subfiletrue
\def\assetspath{../}
%\makeindex
\chead{2020}
\begin{document}

% ------------------------------------------------------------
%
% ------------------------------------------------------------
\begin{proof}[2020 A6.]
    気持ち: 各$m$に対し$\myparen{ a_{m, n} }_{n = 1}^\infty$は
    $\Z_{\ge 1}$上の確率密度関数になっている。

    \uline{(1)} \quad
    級数$\sum_{n = 1}^\infty a_{m, n}$が
    $m$に関し一様収束したと仮定すると、
    ある$N \in \Z_{\ge 1}$が存在して、
    すべての$m \in \Z_{\ge 1}$に対し
    $\sum_{n = N + 1}^\infty a_{m, n} < \frac{1}{2}$が成り立つ。
    さらに条件(c)より、
    ある$M \in \Z_{\ge 1}$が存在して
    $\sum_{n = 1}^N a_{M, n} < \frac{1}{2}$が成り立つ。
    以上より
    $\sum_{n = 1}^\infty a_{M, n} < \frac{1}{2} + \frac{1}{2} = 1$
    が成り立つが、これは条件(b)に矛盾する。

    \uline{(2)} \quad
    \uline{Step 1:} \quad
    まず各$m \in \Z_{\ge 1}$に対し
    級数$\sum_{n = 1}^\infty a_{m, n} s_n$の収束を示す。
    そのためには
    実数列$\myparen{
        \sum_{n = 1}^k a_{m, n} s_n
    }_{k = 1}^\infty$が
    $\R$の Cauchy 列であることを示せばよい。
    そこで$\eps > 0$を任意とすると、
    級数$\sum_{n = 1}^\infty a_{m, n}$が収束すること (条件(b))
    と実数列$(s_n)_n$が$s$に収束すること (小問の仮定) より、
    ある$N \in \Z_{\ge 1}$が存在して、
    すべての$k, k' \ge N, \; k < k'$に対し
    $\sum_{n = k + 1}^{k'} a_{m, n} < \eps$
    かつ
    $|s_k - s| < \eps$
    が成り立つ。
    したがって、
    すべての$k, k' \ge N, \; k < k'$に対し
    \begin{alignat}{2}
        \myabs{
            \sum_{n = k + 1}^{k'} a_{m, n} s_n
        }
            &\le
                \sum_{n = k + 1}^{k'} a_{m, n} |s_n|
                \qquad
                &&(\because \text{条件(a)})
                \\
            &\le
                \sum_{n = k + 1}^{k'}
                    a_{m, n}
                    (|s_n - s| + |s|)
                \qquad
                &&
                \\
            &\le
                \sum_{n = k + 1}^{k'}
                    a_{m, n}
                    (\eps + |s|)
                \qquad
                &&
                \\
            &\le
                \eps (\eps + |s|)
    \end{alignat}
    が成り立つ。
    よって実数列
    $\myparen{
        \sum_{n = 1}^k a_{m, n} s_n
    }_k$は$\R$の Cauchy 列である ($\eps$をとり直す議論は省略)。

    \uline{Step 2:} \quad
    つぎに
    $\lim_{m \to \infty} \sum_{n = 1}^\infty a_{m, n} s_n = s$
    となることを示す。
    そこで$\eps > 0$を任意とする。
    いま実数列$(s_n)_n$が$s$に収束すること (小問の仮定) より、
    ある$N \in \Z_{\ge 1}$が存在して、
    すべての$k \ge N$に対し
    $|s_k - s| < \eps$
    が成り立つ。
    そこで実定数$c$を
    $c \coloneqq \sup_{1 \le n \le N} |s_n - s| + 1$
    とおいておく。
    さらに条件(c)より、
    ある$M \in \Z_{\ge 1}$が存在して
    すべての$m \ge M$に対し
    $\sum_{n = 1}^N a_{m, n} < \frac{\eps}{c}$
    が成り立つ。
    したがって、各$m \ge M$に対し
    \begin{alignat}{2}
        \myabs{
            \sum_{n = 1}^\infty a_{m, n} s_n - s
        }
            &=
                \myabs{
                    \sum_{n = 1}^\infty a_{m, n} (s_n - s)
                }
                \qquad
                &&(\because \text{条件(b)})
                \\
            &\le
                \sum_{n = 1}^N a_{m, n} |s_n - s|
                + \sum_{n = N + 1}^\infty a_{m, n} |s_n - s|
                \qquad
                &&(\because \text{条件(a)})
                \\
            &\le
                \frac{\eps}{c}
                \sup_{1 \le n \le N} |s_n - s|
                +
                \eps
                \qquad
                &&(\because \text{条件(b)})
                \\
            &\le
                2\eps
    \end{alignat}
    が成り立つ。
    よって
    $\lim_{m \to \infty} \sum_{n = 1}^\infty a_{m, n} s_n = s$
    が示された。
\end{proof}

\newpage
\begin{proof}[2020 A7.]
    まず集合族$\calL$は$X$の被覆であって2元の交叉で閉じているから、
    $X$の位相の開基である。

    \uline{(1)} \quad
    $X$における$\{ (0, 0) \}$の閉包$\Cl_X \{ (0, 0) \}$は
    $F \coloneqq (-\infty, 0] \times (-\infty, 0]$であることを示す。
    まず$F$は$F = X \setminus \myparen{
        \bigcup_{n \ge 1} O(-n, 0)
        \cup
        \bigcup_{n \ge 1} O(0, -n)
    }$
    というように開集合の補集合の形に表せるから、
    $X$の閉集合である。
    そこで背理法のために$F = \Cl_X \{ (0, 0) \}$でなかったとすると、
    閉包の最小性より
    $F \supsetneq \Cl_X \{ (0, 0) \}$となるから、
    ある点$(x, y) \in F \setminus \Cl_X \{ (0, 0) \}$が存在する。
    このとき
    点$(x, y)$は$\Cl_X \{ (0, 0) \}$の外点だから、
    $(x, y)$のある近傍$U$であって
    $\Cl_X \{ (0, 0) \}$と交わらないものが存在する。
    一方、
    $(x, y) \in F$すなわち
    $x \le 0, \; y \le 0$であるから、
    $\calL$が開基であることより、
    ある実数$a < x \le 0$と$b < y \le 0$であって
    $O(a, b) \subset U$となるものが存在する。
    したがって$U$は
    点$(0, 0)$において$\Cl_X \{ (0, 0) \}$と交わることになり、
    矛盾が従う。

    \uline{(2)} \quad
    $K$のいかなる開被覆も
    点$(0, 0)$を含むある開集合$U$を含むが、
    $\calL$が開基であることより、
    ある実数$a < 0$と$b < 0$であって
    $O(a, b) \subset U$となるものが存在する。
    したがって$K \subset O(a, b) \subset U$が成り立つから、
    $\{ U \}$は$K$の有限部分被覆となる。
    したがって$K$は$X$のコンパクト部分集合である。

    \uline{(3)} \quad
    $F$を$X$の非空な閉集合として、
    $F$がコンパクトでないことを示す。
    $(x_0, y_0) \in F$をひとつ選ぶと、
    $F$が閉であることより
    $\Cl_X \{ (x_0, y_0) \} \subset F$だから、
    (1)と同様の議論により
    $(-\infty, x_0] \times (-\infty, y_0]
        = \Cl_X \{ (x_0, y_0) \}
        \subset F$
    が成り立つ。
    すると、
    $F$の開被覆
    $\{ O(-n, -n) \mid n \ge 1 \}$
    は有限個の元で
    $(-\infty, x_0] \times (-\infty, y_0]$
    を覆うことはできず、
    したがって
    $F$を覆うこともできないから、
    有限部分被覆を持たない。
    よって$F$はコンパクトでない。
\end{proof}

\end{document}